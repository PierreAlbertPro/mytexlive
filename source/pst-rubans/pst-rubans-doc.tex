%% $Id: pst-rubans-doc.tex 511 2011-06-06 15:39:40Z herbert $
\documentclass[11pt,english,BCOR10mm,DIV12,bibliography=totoc,parskip=false,
   smallheadings, headexclude,footexclude,oneside]{pst-doc}
\usepackage[latin9]{inputenc}
\usepackage{pst-rubans}
\let\pstRubansFV\fileversion
\renewcommand\bgImage{%
%\begin{pspicture}(-10,0)(0,0.2)
%\pshelices[lightsrc=30 5 17,incolor=yellow!50,R=0.3,h=12,spires=10,dZ=0.2,RotX=90,viewpoint=100 45 0,resolution=180,Decran=100]%
%\end{pspicture}
\psset{viewpoint=20 20 70 rtp2xyz,Decran=20,lightsrc=viewpoint,resolution=360}
\psset{unit=0.6}
\begin{pspicture}(-5,-5)(5,5)
\psSpiralRing[incolor=yellow!50,
              r1=4,r0=1,hue=0 1]%
\end{pspicture}
}
\lstset{pos=l,wide=false,language=PSTricks,
    morekeywords={psSpiralRing},basicstyle=\footnotesize\ttfamily}
%
\begin{document}

\title{\texttt{pst-rubans}}
\subtitle{Plotting 3D bands -- an extension of \nxLPack{pst-solides3d}; v.\pstRubansFV}
\author{Manuel Luque\\Herbert Vo\ss}
\docauthor{}
\date{\today}
\maketitle

\tableofcontents

\clearpage


\section{introduction}
\LPack{pst-rubans} est une extension de \LPack{pst-solides3d} qui permet de tracer des rubans sur
certains solides de r�volution : cylindre, tore, sph�re, parabolo�de et c�ne. La largeur du ruban, 
le nombre de spires, la couleur de la face externe ainsi que celle de la face interne peuvent 
�tre param�tr�es. Dans le cas des h�lices circulaire et conique, on peut aussi choisir le nombre de rubans.

\section{Ruban enroul� autour d'un cylindre}
On peut param�trer le rayon \LKeyword{R=2}, la hauteur \LKeyword{h=6}, l'�paisseur(ou largeur) du 
ruban \LKeyword{dZ=0.5}, le nombre de spires \LKeyword{spires=10} de chaque h�lice, le nombre de 
rubans \LKeyword{rubans=1}, le nombre d'�l�ments par ruban \LKeyword{resolution=36}, la couleur 
de l'int�rieur et de l'ext�rieur du ruban.

La base et le couvercle ne font pas partie de la commande.
\subsection{Nombre de rubans \nxLkeyword{rubans=1}}

\begin{BDef}
\Lcs{pshelices}\OptArgs\OptArg*{\Largr{$x_0,y_0,z_0$}}
\end{BDef}



{\psset{lightsrc=viewpoint,viewpoint=50 45 30 rtp2xyz,Decran=50,dZ=0.2,resolution=72,spires=2,unit=0.5}
\begin{LTXexample}[pos=l,width=4cm]
\begin{pspicture}(-2,-1)(2,7)
\psSolid[object=cylindre,r=1.5,h=0.2,ngrid=1 36](0,0,-0.2)
\pshelices[incolor=yellow!50,R=1.5,h=6,hue=0 1,grid](0,0,0)
\psSolid[object=cylindre,r=1.5,h=0.2,ngrid=1 36](0,0,6)
\end{pspicture}
\end{LTXexample}

\subsection{Nombre de rubans \nxLkeyword{rubans=2}, nombre de spires \nxLkeyword{spires=2}}
\begin{LTXexample}[pos=l,width=4cm]
\begin{pspicture}(-2,-1)(2,7)
\psSolid[object=cylindre,r=1.5,h=0.2,ngrid=1 36](0,0,-0.2)
\pshelices[incolor=yellow!50,R=1.5,h=6,hue=0 1,rubans=2,grid](0,0,0)
\psSolid[object=cylindre,r=1.5,h=0.2,ngrid=1 36](0,0,6)
\end{pspicture}
\end{LTXexample}

\clearpage
\subsection{Nombre de rubans \nxLkeyword{rubans=3}, nombre de spires \nxLkeyword{spires=2}}
\begin{LTXexample}[pos=l,width=4cm]
\begin{pspicture}(-2,-1)(2,7)
\psSolid[object=cylindre,r=1.5,h=0.2,ngrid=1 36](0,0,-0.2)
\pshelices[incolor=yellow!50,R=1.5,h=6,hue=0 1,rubans=3,grid](0,0,0)
\psSolid[object=cylindre,r=1.5,h=0.2,ngrid=1 36](0,0,6)
\end{pspicture}
\end{LTXexample}

\subsection{Nombre de rubans \nxLkeyword{rubans=4}, nombre de spires \nxLkeyword{spires=2}}
\begin{LTXexample}[pos=l,width=4cm]
\begin{pspicture}(-2,-1)(2,7)
\psSolid[object=cylindre,r=1.5,h=0.2,ngrid=1 36](0,0,-0.2)%
\pshelices[incolor=yellow!50,R=1.5,h=6,hue=0 1,rubans=4,grid](0,0,0)
\psSolid[object=cylindre,r=1.5,h=0.2,ngrid=1 36](0,0,6)%
\end{pspicture}
\end{LTXexample}
}

\subsection{Nombre de rubans \nxLkeyword{rubans=4}, on fait varier la hauteur \texttt{[h=2,3,4,5]}}

\begin{LTXexample}[pos=t,wide]
\psset{lightsrc=30 5 17,viewpoint=50 45 10,Decran=50,resolution=90,spires=2}
\multido{\nH=2+1}{4}{%
  \begin{pspicture}(-1.5,-1)(1.5,4)
   \psSolid[object=cylindre,r=1.5,h=0.2,ngrid=1 36,grid](0,0,-0.2)%
   \pshelices[incolor=yellow!50,R=1.5,h=\nH,linewidth=0.5\pslinewidth,dZ=0.15,hue=0 1,rubans=4,grid](0,0,0)
   \psSolid[object=cylindre,r=1.5,h=0.2,ngrid=1 36,grid](0,0,\nH\space dz add)%
 \end{pspicture}}
\end{LTXexample}

\clearpage
\subsection{Rubans dispos�s sur un plan}

\begin{LTXexample}[width=8cm,preset=\centering]
\psset{lightsrc=30 5 17,Decran=50,resolution=90,spires=2}
\begin{pspicture}(-4,-1)(4,4)
\psset{R=1,h=2,dZ=0.2,linewidth=0.5\pslinewidth,viewpoint=50 60 10}
\psSolid[object=grille,base=-5 5 -5 5]%
\psSolid[object=cylindre,r=1,h=0.2,ngrid=1 36](-3,-3,0)
\psSolid[object=cylindre,r=1,h=0.2,ngrid=1 36](-3,3,0)
\psSolid[object=cylindre,r=1,h=0.2,ngrid=1 36](3,3,0)
\psSolid[object=cylindre,r=1,h=0.2,ngrid=1 36](3,-3,0)
\pshelices[incolor=red!50,rubans=2](-3,-3,0.2)
\pshelices[incolor=yellow!50,rubans=2](-3,3,0.2)
\pshelices[incolor=green!50,rubans=2](3,3,0.2)
\pshelices[incolor=gray!50,rubans=2](3,-3,0.2)
\end{pspicture}
\end{LTXexample}

\section{Ruban enroul� autour d'un tore}

\begin{BDef}
\Lcs{psSpiralRing}\OptArgs\OptArg*{\Largr{$x_0,y_0,z_0$}}
\end{BDef}

%\begin{lstlisting}
%\psSpiralRing[incolor=yellow!50,r1=4,r0=1,hue,lightsrc=30 5 17](x,y,z)
%\end{lstlisting}

On peut param�trer le rayon moyen \LKeyword{r1=2}, le rayon int�rieur \LKeyword{r0=1}, 
l'�paisseur du ruban \LKeyword{dPHI=2}, en degr�s suivant la latitude, le nombre de 
spires \LKeyword{spires=10}
du ruban, le nombre de brins du ruban \LKeyword{resolution=36}, la couleur de 
l'int�rieur et de l'ext�rieur du ruban.
{
\psset{viewpoint=20 20 70 rtp2xyz,Decran=20,lightsrc=viewpoint,resolution=360}
\begin{LTXexample}[pos=l,width=7.5cm]
\psset{unit=0.75}
\begin{pspicture}(-5,-5)(5,5)
\psSpiralRing[incolor=yellow!50,
              r1=4,r0=1,hue=0 1]%
\end{pspicture}
\end{LTXexample}

\begin{LTXexample}[pos=l,width=7.5cm]
\psset{unit=0.75}
\begin{pspicture}(-5,-5)(5,5)
\psframe*(-5,-5)(5,5)
\psSpiralRing[incolor=yellow!20,r1=4,r0=1,
              fillcolor=orange,grid,
              dPHI=10,spires=5]%
\end{pspicture}
\end{LTXexample}


\begin{LTXexample}[pos=l,width=7.5cm]
\psset{unit=0.75}
\begin{pspicture}(-5,-5)(5,5)
\psSpiralRing[r1=4,r0=1,fillcolor=blue,
              incolor=gray!50,grid,
              dPHI=2,spires=20,
              resolution=720]%
\end{pspicture}
\end{LTXexample}

\begin{LTXexample}[pos=l,width=7.5cm]
\psset{unit=0.75}
\begin{pspicture}(-5,-5)(5,5)
\psSolid[object=tore,r0=1,r1=4,action=draw*,ngrid=18 36]%
\defFunction[algebraic]{solenoidetorique}(t){(4+cos(10*t))*cos(t)}{(4+cos(10*t))*sin(t)}{sin(10*t)}
\psSolid[object=courbe,linecolor=blue,r=0,
         resolution=720,range=0 2 pi mul,
         function=solenoidetorique]%
\end{pspicture}
\end{LTXexample}
}
\section{Ruban enroul� autour d'un parabolo�de}

\begin{BDef}
\Lcs{psSpiralParaboloid}\OptArgs\OptArg*{\Largr{$x_0,y_0,z_0$}}
\end{BDef}

On peut param�trer la hauteur du parabolo�de \LKeyword{h=6}, l'�paisseur du ruban 
\LKeyword{dz=.25}, le nombre de spires \LKeyword{spires=10}
du ruban, le nombre de brins du ruban \LKeyword{resolution=36}, le param�tre de la 
parabole \LKeyword{p=2}, la couleur de l'int�rieur et de l'ext�rieur du ruban.

\begin{LTXexample}[wide,preset=\centering]
\psset{lightsrc=40 25 17}
\psset{unit=0.75,viewpoint=50 30 20,Decran=50}
\begin{pspicture}(-7,-3)(7,8)
\psSpiralParaboloid[incolor=yellow!50,h=6,hue=0 1,resolution=360,spires=5,grid,dZ=0.5]
\gridIIID[QZ=3,Zmin=0,Zmax=6](-5,5)(-5,5)
\end{pspicture}
\end{LTXexample}


{\psset{lightsrc=40 25 17,viewpoint=50 30 20,Decran=50}
\begin{LTXexample}[pos=l,width=9cm,wide]
\psset{unit=0.60}
\begin{pspicture}(-5,-4)(5,9)
\defFunction[algebraic]{spiraleparabolique}(t)
   {2*(sqrt(t))*cos(10*t)}{2*(sqrt(t))*sin(10*t)}{t}
\psSolid[object=courbe,linecolor=blue,r=0,
         resolution=720,range=0 6,
         function=spiraleparabolique]%
\gridIIID[QZ=3,Zmin=0,Zmax=6](-5,5)(-5,5)
\end{pspicture}
\end{LTXexample}

\begin{LTXexample}[pos=l,width=9cm,wide]
\psset{unit=0.60}
\begin{pspicture}(-7,-4)(7,9)
\psSpiralParaboloid[incolor=yellow!50,
                    h=6,hue=0 1,
                    resolution=360,
                    dZ=0.25]
\gridIIID[QZ=3,Zmin=0,Zmax=6](-5,5)(-5,5)
\end{pspicture}
\end{LTXexample}
}
\section{Ruban enroul� autour d'une sph�re}

\begin{BDef}
\Lcs{psSphericalSpiral}\OptArgs\OptArg*{\Largr{$x_0,y_0,z_0$}}
\end{BDef}

%\begin{lstlisting}
%\psSphericalSpiral[incolor=yellow!50,R=4,hue=0 1,dPHI=2,spires=10](x,y,z)
%\end{lstlisting}

On peut param�trer le rayon \LKeyword{R=2}, l'�paisseur du ruban 
\LKeyword{dPHI=2}, en degr�s suivant la latitude, le nombre de spires \LKeyword{spires=10}
de chaque h�lice, le nombre de brins du ruban \LKeyword{resolution=36}, 
la couleur de l'int�rieur et de l'ext�rieur du ruban.

\psset{viewpoint=50 20 10,Decran=50,resolution=180}
\begin{LTXexample}[width=9cm]
\psset{unit=0.75}
\begin{pspicture}(-5,-5)(5,5)
\defFunction[algebraic]{helicespherique}(t){4*cos(10*t)*cos(t)}{4*sin(10*t)*cos(t)}{4*sin(t)}
\psSolid[object=courbe,linecolor=blue,r=0,
        resolution=720,range=pi 2 div neg pi 2 div,
        function=helicespherique]%
\psSolid[object=sphere,r=4,action=draw,ngrid=18 18]%
\gridIIID[Zmin=-4,Zmax=4](-4,4)(-4,4)
\end{pspicture}
\end{LTXexample}

\begin{LTXexample}[width=9cm]
\psset{unit=0.75}
\begin{pspicture}(-5,-5)(5,5)
\psset{viewpoint=50 20 10,Decran=50,resolution=180}
\psSphericalSpiral[incolor=yellow!50,R=4,hue=0 1,lightsrc=30 5 17]%
\gridIIID[Zmin=-4,Zmax=4](-4,4)(-4,4)
\end{pspicture}
\end{LTXexample}

\begin{LTXexample}[width=9cm]
\psset{unit=0.75}
\begin{pspicture}(-5,-5)(5,5)
\psframe*(-5,-5)(5,5)
\psset{viewpoint=50 20 30,Decran=50,resolution=360}
\psSphericalSpiral[incolor=yellow!20,R=4,fillcolor=orange,lightsrc=40 15 25,grid,dPHI=10](0,0,0)
\end{pspicture}
\end{LTXexample}

\begin{LTXexample}[width=9cm]
\begin{pspicture}(-4,-4)(4,4)
\psframe*(-4,-4)(4,4)
\psset{unit=0.6}
\psset{viewpoint=50 20 10,Decran=50,resolution=360}
\psSphericalSpiral[incolor=yellow!50,R=4,fillcolor=green!50,lightsrc=32 38 -10,grid]
\white%
\gridIIID[Zmin=-4,Zmax=4,linecolor=white](-4,4)(-4,4)
\end{pspicture}
\end{LTXexample}

\section{Rubans enroul�s autour d'un c�ne}

\begin{BDef}
\Lcs{psSpiralCone}\OptArgs\OptArg*{\Largr{$x_0,y_0,z_0$}}
\end{BDef}



On peut param�trer le rayon \LKeyword{R=2}, la hauteur \LKeyword{h=6}: 
il s'agit d'un double c�ne vertical, l'�paisseur(ou largeur) du ruban 
\LKeyword{dZ=0.5}, le nombre de spires \LKeyword{spires=10} de chaque h�lice, 
le nombre de rubans \LKeyword{rubans=1}, le nombre d'�l�ments par ruban 
\LKeyword{resolution=36}, la couleur de l'int�rieur et de l'ext�rieur du ruban.

\psset{unit=0.75}
\begin{LTXexample}[pos=t]
\begin{pspicture}(-7,-7)(7,9)
\psset{viewpoint=50 20 10 rtp2xyz,Decran=50,lightsrc=viewpoint,spires=10}
\psSolid[object=cone,
         h=5,r=4,mode=4,
         action=draw](0,0,-5)
\psSolid[object=cone,RotX=180,
         h=5,r=4,mode=4,
         action=draw](0,0,5)
\psSpiralCone[incolor=yellow!50,h=10,hue=0 1,resolution=360,dZ=0.25,grid]%
\gridIIID[Zmin=-5,Zmax=5,planmarks=true](-4,4)(-4,4)
\end{pspicture}
\end{LTXexample}

\psset{unit=0.75}
\begin{LTXexample}[pos=t]
\begin{pspicture}(-7,-8)(7,9)
\psframe*(-7,-8)(7,9)
\psset{viewpoint=50 20 30 rtp2xyz,Decran=50,lightsrc=viewpoint,spires=10,rubans=4}
\psSpiralCone[incolor=yellow!50,h=10,R=5,fillcolor=green!50,resolution=720,dZ=0.1,grid]%
\end{pspicture}
\end{LTXexample}



\clearpage
\section{List of all optional arguments for \texttt{pst-ruban}}

\xkvview{family=pst-rubans,columns={key,type,default}}


\bgroup
\nocite{*}
\raggedright
\bibliographystyle{plain}
\bibliography{\jobname}
\egroup

\printindex


\end{document}
