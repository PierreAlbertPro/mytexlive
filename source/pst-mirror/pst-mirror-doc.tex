%% $Id: pst-mirror-doc.tex 377 2010-08-30 12:15:40Z herbert $
\documentclass[11pt,frenchb,BCOR10mm,DIV12,bibliography=totoc,parskip=false,smallheadings
    headexclude,footexclude,oneside]{pst-doc}
\usepackage[utf8]{inputenc}
\usepackage{pst-mirror}
\let\myFV\fileversion
\usepackage{pst-grad}
\usepackage{tabularx}
\usepackage{longtable}

\def\UrlFont{\small\ttfamily}
\makeatletter
\def\verbatim@font{\small\normalfont\ttfamily}
\makeatother
\let\myPart\part
\renewcommand{\part}[1]{\clearpage\myPart{#1}}
\newcommand\verbI[1]{\small\texttt{#1}}

\psset{subgriddiv=0,gridlabels=7pt,gridcolor=black!15}
\hypersetup{pdfauthor={Herbert Voss},pdftitle={Sphere mirror}}
%
\newcounter{boua}
\newcommand\itemBoua{\addtocounter{boua}{1}\strut\indent\textit{\theboua}\textsuperscript{o} --- }
\makeatletter
% macro d'Etienne RIGA sur fr.comp.text.tex
\newcount\r@pport  \newdimen\r@ppord
\newcount\kslant   \newdimen\kslantd

\def\ARC#1{\setbox0\hbox{$\m@th\displaystyle#1$}\kslant=\ht0
   \divide\kslant by1000\multiply\kslant by\fontdimen1\textfont1
   \divide\kslant by10000\kslantd=\kslant\fontdimen6\textfont1
      \divide\kslantd by7750\kern\kslantd
   \r@ppord=\wd0\multiply\r@ppord by100\divide\r@ppord by\ht0
   \multiply\r@ppord by300\advance\r@ppord by\ht0
   \pspicture(0,0)
   \psparabola[linewidth=.3pt](0,1.05\ht0)(.5\wd0,1.21\r@ppord)
   \endpspicture\kern-\kslantd\box0}
\makeatother

\def\oeil{\psarc[linewidth=2pt](0,2.5){2.5}{215}{270}%
  \psarc[linewidth=2pt](0,-2.5){2.5}{90}{140}%
  \psarc(-2.5,0){1}{-30}{30}\psarc(0,0){1.75}{160}{200}
  \psclip{%
    \pscircle[linestyle=none](0,0){1.75}}
    \pscircle[fillstyle=solid,fillcolor=lightgray](-2.5,0){0.9}
  \endpsclip}%

\def\decor{%
% plan de front
  \psset{Sphere=false}
  {\psset{normale=0 0, Xorigine=40, Yorigine=0, Zorigine=0}
  \pstSphereFrame[fillstyle=solid,fillcolor=OrangePale](-70,-50)(70,10)
  \pstSphereGrid[linecolor=blue,linewidth=0.2mm,grille=10,Ymin=-50,Ymax=10,Xmax=70,Xmin=-70](40,0,0)}%
  % plan horizontal
  {\psset{normale=0 90, Xorigine=40, Yorigine=0, Zorigine=-10}
  \pstSphereFrame[fillstyle=solid,fillcolor=cyan,opacity=0.5](-70,-30)(70,0)
  \pstSphereGrid[linecolor=blue,linewidth=0.2mm,grille=10,Ymin=-30,Ymax=0,Xmax=70,Xmin=-70](40,0,-10)}%
  % plan vertical gauche
  {\psset{normale=90 0, Xorigine=40, Yorigine=70, Zorigine=-10}
  \pstSphereFrame[fillstyle=solid,fillcolor=yellow](0,0)(-30,-60)
  \pstSphereGrid[linecolor=blue,linewidth=0.05mm,grille=10,Ymin=-60,Ymax=0,Xmin=-30,Xmax=0](40,70,-10)}
  % plan vertical droit
  {\psset{normale=90 0,%
                Xorigine=40,%
                Yorigine=-70,%
                Zorigine=-10}
\pstSphereFrame[fillstyle=solid,fillcolor=yellow](0,0)(-30,-60)
\pstSphereGrid[linecolor=blue,linewidth=0.05mm,grille=10,Ymin=-60,Ymax=0,Xmin=-30,Xmax=0](40,-70,-10)}}

\def\bgImage{\psscalebox{0.7}{\begin{pspicture}(-7,-7)(7,7)
  \newpsstyle{GradWhiteYellow}{fillstyle=gradient,gradbegin=yellow,gradend=yellow!20,
     linecolor=yellow!50,GradientCircle=true,gradmidpoint=0,GradientPos={(1,1)}}%
  \pscircle[style=GradWhiteYellow]{7.07}
  \pstSphereGrid[linecolor=red,grille=10,Ymin=-50,Ymax=50,Xmax=80,Xmin=-80,normale=0 0](20,0,0)
  \pstSphereGrid[linecolor=blue,grille=10,Ymin=-40,Ymax=-20,Xmax=80,Xmin=-80,normale=0 90](40,0,-10)%
  \pstTextSphere[fillstyle=solid,fillcolor=red,normale=0 0,fontscale=40,PSfont=Time-Roman,yO=0](20,0,10){pst-mirror}
  \pstTextSphere[fillstyle=solid,fillcolor=black,normale=0 0,fontscale=20,PSfont=Helvetica,yO=0](20,0,35){PSTricks}
  \pstTextSphere[fillstyle=solid,fillcolor=blue,normale=0 90,fontscale=10,PSfont=Helvetica,yO=2.5](10,0,-10){A Spherical Mirror}%
\end{pspicture}}}

\lstset{basicstyle=\ttfamily\footnotesize}

\begin{document}
\author{Manuel Luque\\Herbert Vo\ss}
\docauthor{}
\date{\today}
\title{Images on a spherical mirror}
\subtitle{A PSTricks package for drawing 3D objects on a sphere,  v\myFV}
\maketitle%

\tableofcontents

\clearpage
\begin{abstract}

\vfill\noindent
Thanks for feedback and contributions to:\\
Bruno Guegan; Jean-Paul Vignault
\end{abstract}


\section{Objectifs }
On voit souvent dans les villes, lorsque la visibilité à la sortie
d'un garage ou à un carrefour est insuffisante, ou bien
dans certains magasins pour des motifs de surveillance, des
miroirs bombés donnant de l'environnement un panorama trés large.
On peut prendre aussi une boule argentée ou dorée que l'on suspend
aux branches des sapins de Noél, pour voir le décor se réfléchir dans la Sphere
avec une distorsion bien particuliére.

L'objectif est de définir des éléments de décor en 3D, \textit{des
briques}, pour reconstruire ce qu'un observateur placé é une
certaine distance d'une boule miroir, é la méme hauteur que son
centre verrait.


Les \textit{briques} é notre disposition sont :
\begin{compactitem}
  \item Un parallélépipéde rectangle donné par ses trois
  dimensions \verb+A,B,C+ : il peut donc se transformer en cube et
  méme en dé.
  \item Un quadrillage plan défini par son origine, la normale au
  plan et ses limites.
  \item Une ligne.
  \item Un rectangle défini comme le quadrillage.
  \item Un polygone.
  \item Un cercle défini par la normale é son plan, son origine et
  son rayon, un arc de cercle défini comme le cercle avec ses deux angles limites.
  \item Un point de l'espace.
  \item Un tétraédre donné par les coordonnées du centre de sa base et le rayon du
  cercle inscrivant chaque face, que l'on peut faire
  tourner.
  \item Une pyramide é base carrée donnée par le demi-cété de sa
  base et sa hauteur que l'on peut faire tourner et placer oé on
  veut.
  \item Une sphére donnée par les coordonnées du centre \verb+CX=...+,\verb+CY=...+,\verb+CZ=...+
   et le rayon, que l'on peut faire
  tourner avec les paramétres
  \verb+RotX=...+,\verb+RotY=...+,\verb+RotZ=...+.
  \item Un cylindre vertical défini par son rayon, sa hauteur, que l'on peut faire
  tourner avec les paramétres
  \verb+RotX=...+,\verb+RotY=...+,\verb+RotZ=...+ et dont le centre
  de la base se placera avec les paramétres \verb+CX=...+,\verb+CY=...+,\verb+CZ=...+
  \item Un céne et tronc de céne définis par le rayon de la base,
  la hauteur et la fraction de hauteur qu'on souhaite retenir (pour le tronc ce céne).
  On peut les placer ou l'on veut et les faire tourner comme les autres éléments.
  \item Image d'un texte écrit sur un plan quelconque.
\end{compactitem}

\clearpage
\section{Présentation du probléme théorique}

\begin{center}
\begin{pspicture}(-5,-6)(11,6)
\pnode(0,0){O}
\pnode(! /xA 8 def /yA 5 def  xA yA){A}
\pnode(! /Theta 10 def /xI1 5 Theta cos mul def /yI1 5 Theta sin mul def
        xI1 yI1){I1}
\psline[linecolor=red](I1)(A)
\pnode(!/Alpha1 yA yI1 sub xA xI1 sub atan def
        /iAngle Alpha1 Theta sub def
        /Alpha1' Theta iAngle sub def
        /xP1' xI1 6 Alpha1' cos mul add def
        /yP1' yI1 6 Alpha1' sin mul add def
        xP1' yP1'){P1'}
\psline[linecolor=red](I1)(P1') \pcline[linecolor=blue,nodesepB=-4](O)(I1)
\pnode(!/Theta 20 def /xI2 5 Theta cos mul def /yI2 5 Theta sin mul def xI2 yI2){I2}
\pnode(! 
        /Alpha2 yA yI2 sub xA xI2 sub atan def
        /iAngle Alpha2 Theta sub def
        /Alpha2' Theta iAngle sub def
        /xP2' xI2 6 Alpha2' cos mul add def
        /yP2' yI2 6 Alpha2' sin mul add def
        xP2' yP2'){P2'}
\psline[linecolor=red](I2)(P2')
\pcline[linecolor=blue,nodesepB=-4](O)(I2)
\pnode(!
    /tanAlpha1' Alpha1' dup sin exch cos div def
    /tanAlpha2' Alpha2' dup sin exch cos div def
    /xA' yI2 yI1 sub xI2 tanAlpha2' mul sub xI1 tanAlpha1' mul add
         tanAlpha1' tanAlpha2' sub div def
    /yA' xA' tanAlpha1' mul yI1 add xI1 tanAlpha1' mul sub def
    xA' yA'){A'}
\psdot(A')
\pswedge[fillstyle=solid,fillcolor=black](A'){0.5}{-35}{-5}
\uput{0.5}[-15](A'){$\varepsilon'$}
\psline[linestyle=dashed](A')(I1)
\psline[linestyle=dashed](A')(I2)
\pscircle{5}
\psdot(A)
\psline[linestyle=dotted](O)(A)
\uput[0](A){A}
\uput[45](A'){A'}
\uput[180](O){O}
\uput[80](I2){$\mathrm{I_1}$}
\uput[-135](I1){$\mathrm{I_2}$}
\psline[linestyle=dotted](O)(A')
\pswedge[fillstyle=solid,fillcolor=black](A){2}{224.9}{233.335}
\uput{2.1}[230](A){$\varepsilon$}
\psline[linecolor=red](I2)(A)
\psline[linecolor=red](I1)(A)
\pswedge[fillstyle=solid,fillcolor=black](O){2}{10}{20}
\uput{2.1}[15](O){$\omega$}
\psarc(O){2}{10}{20}
\psarc[doubleline=true](I1){0.75}{10}{53.335}
\psarc[doubleline=true](I1){0.95}{-33.335}{10}
\uput{1}[-12](I1){$i+\mathrm{d}i$}
\psarc(I2){0.75}{20}{44.9}
\psarc(I2){0.95}{-4.9}{20}
\uput{0.8}[31](I2){$i$}
\uput{1}[7.5](I2){$i$}
% \psaxes(0,0)(-5,-6)(10,6)
\end{pspicture}
\end{center}

Il s'agit d'étudier la réflexion d'un mince pinceau conique émis
par~A se réfléchissant sur le miroir sphérique et qui pénétrera dans l'\oe{}il de
l'observateur. C'est un probléme trés complexe qui a été abordé
dans sa généralité par Henri Bouasse dans son livre : \textit{Optique supérieure}, paru en 1917 aux éditions Delagrave.

Le point lumineux A, quel qu'il soit, est dans le plan du grand
cercle de la sphére défini par $\mathrm{OI_1A}$. Je donne le début
de cette étude théorique, qui est une adaptation des pages~108 et suivantes de son livre,
sans la développer jusqu'é son terme, car
comme l'indique Henri Bouasse lui-méme \guillemotleft \textit{ Le
probléme est généralement inextricable, sauf dans le cas oé la
symétrie du systéme indique quels rayons il faut regrouper aprés
réflexion} \guillemotright. Et bien que la fin de cette remarque
s'applique au miroir sphérique, encore faudrait-il que les calculs
soient facilement programmables et relativement rapides et j'avoue
avoir été découragé par leur complexité. Heureusement, nous
verrons que dans certaines conditions, de judicieuses
approximations permettent d'obtenir une solution réaliste.

Afin d'apprécier la difficulté du probléme, on considére d'abord les limites du pinceau 
dans ce plan, c'est é dire sur l'arc  \ARC{\mathrm{\hspace{0.5em}I_1I_2}}.

Les angles sont exagérément grands
pour des raisons de clarté.

Dans le triangle $\mathrm{OI_1A}$,
$$i=\mathrm{\widehat{I_1OA}+\widehat{OAI_1}}$$
Dans le triangle $\mathrm{OI_2A}$,
$$\begin{array}{rcl}
i+\mathrm{d}i&=&\mathrm{\widehat{I_2OA}+\widehat{OAI_2}}\\
&=&\mathrm{\widehat{I_2OI_1}+\widehat{I_1OA}+\widehat{OAI_1}+\widehat{I_1AI_2}}\\
&=&\mathrm{\omega+\widehat{I_1OA}+\widehat{OAI_1}+\varepsilon}
\end{array}
$$
On en déduit :
\begin{equation}
\mathrm{d}i=\omega+\varepsilon
\label{di_one}
\end{equation}
Dans le triangle $\mathrm{OI_1A'}$
$$i=\mathrm{\pi-(\widehat{I_1OA'}+\widehat{OA'I_1})}$$
Dans le triangle $\mathrm{OI_2A'}$
$$\begin{array}{rcl}
i+\mathrm{d}i&=&\pi-(\mathrm{\widehat{I_2OA'}+\widehat{OA'I_2})}\\
&=&\pi-\left(\mathrm{(\widehat{I_2OI_1}+\widehat{I_1OA'})+(\widehat{OAI'_1}-\widehat{I_1A'I_2})}\right)\\
&=&\pi-\left(\mathrm{(\omega+\widehat{I_1OA'})+(\widehat{OA'I_1}-\varepsilon')}\right)
\end{array}
$$
On en déduit :
\begin{equation}
\mathrm{d}i=-\omega+\varepsilon'
\label{di_two}
\end{equation}
En rapprochant (\ref{di_one}) et (\ref{di_two}) :
\begin{equation}
\omega+\varepsilon=-\omega+\varepsilon'\Longleftrightarrow2\omega=\varepsilon'-\varepsilon
\label{omega}
\end{equation}
En se rappelant qu'en réalité $\varepsilon$, $\varepsilon'$ et
$\omega$ sont de petits angles, on établit les relations
suivantes, en posant :
$$t=\mathrm{AI_1}\textrm{ et } t'=\mathrm{A'I_1}$$
$$\mathrm{d}s=\ARC{\mathrm{\hspace{0.5em}I_1I_2}}=R\omega=\frac{t\varepsilon}{\cos i}=\frac{t'\varepsilon'}{\cos i}$$
$$\varepsilon=\frac{\mathrm{R}\omega\cos i}{t}\qquad\varepsilon'=\frac{\mathrm{R}\omega\cos
i}{t'}$$
Compte-tenu de la relation (\ref{omega}), on en déduit la formule
fondamentale :
$$\frac{1}{t'}-\frac{1}{t}=\frac{2}{\mathrm{R}\cos i}$$
S'il n'y avait que le faisceau de rayons balayant depuis A l'arc
\ARC{\mathrm{I_1I_2}}, l'image virtuelle que verrait
l'\oe{}il serait $\mathrm{A'}$, qui est dans ce plan la trace de la
\textit{focale tangentielle}.

Il faut maintenant envisager toutes les autres génératrices du
céne. On en prend deux de particuliéres, celles qui sont sur les
bouts de l'arc de cercle de la sphére perpendiculaire au grand
cercle précédent. Nous ne pouvons plus nous placer maintenant dans
un plan pour faire le schéma. Ces deux génératrices permettent de
déterminer la \textit{focale radiale} $\mathrm{A''}$. Les deux focales $\mathrm{A'}$ et
$\mathrm{A''}$ sont alignées avec le rayon réfléchi correspondant
au rayon moyen du pinceau incident.


En conclusion, é la page 134, Henri Bouasse écrit :
  \guillemotleft \textit{ l'\oe{}il n'utilise qu'un pinceau
  extrémement étroit ; il s'accommode sur l'une des focales ou sur le cercle de
  diffusion.} \guillemotright. Nous avons un peu avancé, mais quel(s) point(s)
  calculer?

Les milieux des focales tangentielles, radiales et du cercle de
diffusion se trouvent alignés avec l'\oe{}il et le point
d'incidence du rayon moyen du faisceau (celui qui aprés réflexion
passe par l'\oe{}il). L'image obtenue d'un objet quelconque n'est
 pas plane. On pourrait contourner la difficulté en imaginant un plan
 fictif perpendiculaire à l'axe \oe{}il-centre de la boule et
 passant par une valeur moyenne des cercles de diffusion, sur
 lequel on calculerait les intersections des rayons. Il faut bien
 reconnaétre que tout cela est passablement compliqué.

Henri Bouasse propose, dans certaines conditions, une méthode approximative qui donne de bons
résultats et qui a l'avantage d'étre simple. C'est
l'intégralité de cette solution qu'il donne dans les
pages 133, 134 et 135 de son livre, qui est développée ci-aprés.

\section{Vision dans une boule de jardin}
\begin{center}
\begin{pspicture}(-5,-2)(8,14)
\pscircle{2}
\psline[linewidth=0.8mm](8,0.2)(8,14) \psline[linewidth=0.8mm](8,-0.2)(8,-2)
\psline(-5,0)(8.2,0)\psline(4;30)\psarc(0,0){1}{0}{30}\uput[15](1;15){$i$}
\pnode(!2 30 cos mul 2 30 sin mul){I}
\psline[linestyle=dashed](!0  2 30 sin mul)(I) \psline(!0 2 30 sin mul)(0,0)
\uput[180](!0 1 30 sin mul){$\rho'$} \psline(I)(!8 2 30 sin mul)
\rput(I){\psarc(0,0){1}{0}{30} \uput[15](1;15){$i$}
\psline(! /IH 8 2 30 cos mul sub 2 mul def IH 60 cos mul IH 60 sin mul)}
\uput[270](0,0){C} \uput[45](8,0){T} \uput[180](8,-2){E} \uput[180](-2,0){B}
\uput[270](5,0){D} \uput[180](8,6){$\rho$}
\end{pspicture}
\end{center}

\itemBoua Pour effectuer les expériences suivantes, on se sert
d'une boule de jardin achetée dans un bazar, ou d'un petit ballon
de verre de 5~cm de diamétre environ dont on argente la surface
extérieure. On peut argenter extérieurement une bille d'ivoire.

Sur une trés grande feuille de carton blanc E, on trace un
quadrillage en traits noirs épais. On regarde l'image de ce
quadrillage par un petit trou percé au centre du carton.

La distance $\mathrm{D=TB}$ sera prise d'une trentaine de
centimétres.

Le quadrillage apparaét déformé \textit{en barillet}, mais
\textit{parfaitement net}.

\itemBoua
Pour avoir une idée de la loi de déformation supposons l'\oe{}il
assez loin pour que les rayons réfléchis qui y parviennent, soient
quasi-paralléles. Appelons $\rho'$ la distance d'un rayon é la
droite TB.

Soit R le rayon de la boule. l'angle d'incidence est :
$$i=\arcsin(\rho':\mathrm{R})$$
Le point oé le rayon conjugué coupe l'écran E, est é une distance
$\rho$ de TB donnée par la formule~:
$$\rho=\mathrm{D}\tan2i$$
D'oé la loi approchée :
$$\arctan\frac{\rho}{\mathrm{D}}=2\arcsin\frac{\rho'}{\mathrm{R}}$$
$\rho$ devient infini pour :
$$\arcsin(\rho':\mathrm{R})=45^\mathrm{o}\qquad \rho'=0.7\mathrm{R}$$
En conséquence, si grand que soit le carton, son image occupera
moins des sept dixiémes du diamétre.

\section{Exemples}
\begin{pspicture}(-8,-8)(8,8)
\pstSphereFrame[normale=0 90,Xorigine=50,Yorigine=0,Zorigine=-20,fillstyle=solid,
  fillcolor=OrangePale](-70,-50)(70,50)
\pstSphereGrid[normale=0 90,linecolor=blue,linewidth=0.2mm,grille=10,
  Ymin=-50,Ymax=50,Xmax=70,Xmin=-70](50,0,-20)
{\psset{fillstyle=solid,fillstyle=solid,fillcolor=GrisClair,A=5,B=A,C=A}
\multido{\iX=95+-20}{4}{\multido{\iZ=55+-10}{6}{\psset{CX=\iX,CZ=\iZ,CY=45}\pstCube}
                        \multido{\iZ=-15+10}{2}{\psset{CX=\iX,CZ=\iZ,CY=45}\pstCube}
                        \multido{\iZ=55+-10}{6}{\psset{CX=\iX,CZ=\iZ,CY=-45}\pstCube}
                        \multido{\iZ=-15+10}{2}{\psset{CX=\iX,CZ=\iZ,CY=-45}\pstCube}
                        \multido{\iY=35+-10}{4}{\psset{CX=\iX,CZ=55,CY=\iY}\pstCube}
                        \multido{\iY=-35+10}{4}{\psset{CX=\iX,CZ=55,CY=\iY}\pstCube}}}
\multido{\iZ=-15+10,\iRotZ=20+45}{2}{%
  \psset{fillstyle=solid,fillstyle=gradient,fillcolor=GrisClair,A=5,B=A,C=A,
         CX=5,CY=-65,CZ=\iZ,RotZ=\iRotZ}\pstCube}
\multido{\iZ=-15+10}{2}{%
  \psset{fillstyle=solid,fillstyle=solid,fillcolor=GrisClair,A=5,B=A,C=A,CX=15,CY=45,CZ=\iZ}\pstCube
  \psset{CY=-45}\pstCube}
\multido{\iZ=15+-10,\iRotZ=30+30}{2}{%
  \psset{fillstyle=solid,fillstyle=solid,fillcolor=GrisClair,A=5,B=A,C=A,
    CX=15,CY=45,CZ=\iZ,RotZ=\iRotZ}\pstCube}
\multido{\iZ=25+-10,\iRotZ=30+20}{3}{%
  \psset{fillstyle=solid,fillstyle=solid,fillcolor=GrisClair,A=5,B=A,C=A,
    CX=15,CY=-45,CZ=\iZ,RotZ=\iRotZ}\pstCube}
\multido{\iX=95+-20}{5}{\multido{\iY=-25+10}{6}{%
  \psset{fillstyle=gradient,gradbegin=white,gradend=red,gradmidpoint=0.2,linewidth=0.05mm,
    CY=\iY,CZ=-20,CX=\iX}
  \ifnum\iX=15
    \ifnum\iY=-15 {\psset{RotY=90,RotZ=-45,CZ=-17.5}\pstSphereCylinder{2.5}{-10}}
    \else \pstSphereCylinder{2.5}{10}\fi
  \else\pstSphereCylinder{2.5}{10}\fi}}
\multido{\iY=25+-10,\iRotY=0+60}{6}{%
  \psset{fillstyle=solid,linewidth=0.05mm,CX=55,CY=\iY,CZ=-10,A=2.5,Hpyramide=5}\pstPyramide}
\multido{\iY=25+-10,\iRotY=0+60}{6}{%
  \psset{fillstyle=gradient,gradbegin=white,gradend=red,gradmidpoint=0.2,
    linewidth=0.05mm,CX=35,CY=\iY,CZ=-10}\pstSphereCone{2.5}{2.5}}
\multido{\iY=25+-10,\iRotY=0+60}{6}{%
\psset{fillstyle=gradient,gradbegin=white,gradend=blue,gradmidpoint=0.2,linecolor=cyan,linewidth=0.1mm}
\ifnum\iY=-15 {\psset{CX=7.5,CY=-5,CZ=-17.5,RotY=\iRotY} \pstMirrorSphere{2.5}}
\else {\psset{CX=15,CY=\iY,CZ=-7.5,RotY=\iRotY}\pstMirrorSphere{2.5}}\fi}
\psset{CX=2.5,CZ=-10,CY=15,fracHcone=0.5,linecolor=red,fillstyle=gradient,gradbegin=yellow,
  gradend=red,gradmidpoint=0,linewidth=0.05mm}
\pstSphereCone{2.5}{5}
{\psset{RotX=0,RotZ=20,RotY=30,CX=2.5,CZ=-10,CY=2.5,fillstyle=solid,linecolor=black,A=2,Hpyramide=3}
\pstPyramide}
{\psset{linecolor=red,CZ=5,A=2.5,B=A,C=A,RotZ=-50,RotX=20,RotY=30,CY=0,CX=20,%
  fillstyle=solid,fillcolor=OrangePale}
\pstDie}
\end{pspicture}

\begin{lstlisting}
\begin{pspicture}(-8,-8)(8,8)
\pstSphereFrame[normale=0 90,Xorigine=50,Yorigine=0,Zorigine=-20,fillstyle=solid,
  fillcolor=OrangePale](-70,-50)(70,50)
\pstSphereGrid[normale=0 90,linecolor=blue,linewidth=0.2mm,grille=10,
  Ymin=-50,Ymax=50,Xmax=70,Xmin=-70](50,0,-20)
{\psset{fillstyle=solid,fillstyle=solid,fillcolor=GrisClair,A=5,B=A,C=A}
\multido{\iX=95+-20}{4}{\multido{\iZ=55+-10}{6}{\psset{CX=\iX,CZ=\iZ,CY=45}\pstCube}
                        \multido{\iZ=-15+10}{2}{\psset{CX=\iX,CZ=\iZ,CY=45}\pstCube}
                        \multido{\iZ=55+-10}{6}{\psset{CX=\iX,CZ=\iZ,CY=-45}\pstCube}
                        \multido{\iZ=-15+10}{2}{\psset{CX=\iX,CZ=\iZ,CY=-45}\pstCube}
                        \multido{\iY=35+-10}{4}{\psset{CX=\iX,CZ=55,CY=\iY}\pstCube}
                        \multido{\iY=-35+10}{4}{\psset{CX=\iX,CZ=55,CY=\iY}\pstCube}}}
\multido{\iZ=-15+10,\iRotZ=20+45}{2}{%
  \psset{fillstyle=solid,fillstyle=gradient,fillcolor=GrisClair,A=5,B=A,C=A,
         CX=5,CY=-65,CZ=\iZ,RotZ=\iRotZ}\pstCube}
\multido{\iZ=-15+10}{2}{%
  \psset{fillstyle=solid,fillstyle=solid,fillcolor=GrisClair,A=5,B=A,C=A,CX=15,CY=45,CZ=\iZ}\pstCube
  \psset{CY=-45}\pstCube}
\multido{\iZ=15+-10,\iRotZ=30+30}{2}{%
  \psset{fillstyle=solid,fillstyle=solid,fillcolor=GrisClair,A=5,B=A,C=A,
    CX=15,CY=45,CZ=\iZ,RotZ=\iRotZ}\pstCube}
\multido{\iZ=25+-10,\iRotZ=30+20}{3}{%
  \psset{fillstyle=solid,fillstyle=solid,fillcolor=GrisClair,A=5,B=A,C=A,
    CX=15,CY=-45,CZ=\iZ,RotZ=\iRotZ}\pstCube}
\multido{\iX=95+-20}{5}{\multido{\iY=-25+10}{6}{%
  \psset{fillstyle=gradient,gradbegin=white,gradend=red,gradmidpoint=0.2,linewidth=0.05mm,
    CY=\iY,CZ=-20,CX=\iX}
  \ifnum\iX=15
    \ifnum\iY=-15 {\psset{RotY=90,RotZ=-45,CZ=-17.5}\pstSphereCylinder{2.5}{-10}}
    \else \pstSphereCylinder{2.5}{10}\fi
  \else\pstSphereCylinder{2.5}{10}\fi}}
\multido{\iY=25+-10,\iRotY=0+60}{6}{%
  \psset{fillstyle=solid,linewidth=0.05mm,CX=55,CY=\iY,CZ=-10,A=2.5,Hpyramide=5}\pstPyramide}
\multido{\iY=25+-10,\iRotY=0+60}{6}{%
  \psset{fillstyle=gradient,gradbegin=white,gradend=red,gradmidpoint=0.2,
    linewidth=0.05mm,CX=35,CY=\iY,CZ=-10}\pstSphereCone{2.5}{2.5}}
\multido{\iY=25+-10,\iRotY=0+60}{6}{%
\psset{fillstyle=gradient,gradbegin=white,gradend=blue,gradmidpoint=0.2,linecolor=cyan,linewidth=0.1mm}
\ifnum\iY=-15 {\psset{CX=7.5,CY=-5,CZ=-17.5,RotY=\iRotY} \pstMirrorSphere{2.5}}
\else {\psset{CX=15,CY=\iY,CZ=-7.5,RotY=\iRotY}\pstMirrorSphere{2.5}}\fi}
\psset{CX=2.5,CZ=-10,CY=15,fracHcone=0.5,linecolor=red,fillstyle=gradient,gradbegin=yellow,
  gradend=red,gradmidpoint=0,linewidth=0.05mm}
\pstSphereCone{2.5}{5}
{\psset{RotX=0,RotZ=20,RotY=30,CX=2.5,CZ=-10,CY=2.5,fillstyle=solid,linecolor=black,A=2,Hpyramide=3}
\pstPyramide}
{\psset{linecolor=red,CZ=5,A=2.5,B=A,C=A,RotZ=-50,RotX=20,RotY=30,CY=0,CX=20,%
  fillstyle=solid,fillcolor=OrangePale}
\pstDie}
\end{pspicture}
\end{lstlisting}




\begin{pspicture}(-8,-8)(8,8)
%\psset{scale=2}
\pstSphereFrame[Xorigine=20,Yorigine=0,Zorigine=0,fillstyle=solid,fillcolor=OrangePale](-70,-50)(70,10)
\pstSphereGrid[linecolor=blue,linewidth=0.2mm,grille=5,Ymin=-50,Ymax=50,Xmax=70,Xmin=-70](20,0,0)%
%
\pstSphereFrame[normale=90 0,Xorigine=10,Yorigine=70,Zorigine=0,fillstyle=solid,fillcolor=yellow](-10,-50)(10,10)
\multido{\nX=17.5+-10.0}{2}{%
  \multido{\nY=-67.5+10.0}{14}{%
    \pstSphereFrame[normale=0 90,Xorigine=\nX,Yorigine=\nY,Zorigine=-10,fillstyle=solid,
      fillcolor=yellow](-2.5,-2.5)(2.5,2.5)}}
\multido{\nX=12.5+-10.0}{2}{%
  \multido{\nY=-62.5+10.0}{14}{%
    \pstSphereFrame[normale=0 90,Xorigine=\nX,Yorigine=\nY,Zorigine=-10,fillstyle=solid,
      fillcolor=yellow](-2.5,-2.5)(2.5,2.5)}}
\pstSphereGrid[normale=90 0,linecolor=blue,linewidth=0.2mm,grille=5,
  Xmin=-10,Xmax=10,Ymax=10,Ymin=-50](10,70,0)%
\pstSphereGrid[normale=0 90,linecolor=blue,linewidth=0.2mm,grille=5,
  Xmin=-70,Xmax=70,Ymax=10,Ymin=-10](10,0,-10)%
{\psset{linecolor=black,CZ=52,A=5,B=70,C=2,CY=0,CX=15,fillstyle=solid}\pstCube}
{\psset{linecolor=red,CX=15,A=5,B=A,C=A,CY=-10,CZ=-5,RotZ=0,fillstyle=solid,fillcolor=blue}\pstDie}%
\multido{\nZ=47.5+-5.0}{10}{%
  \psset{linecolor=red,CY=22.5,A=2.5,B=A,C=A,CZ=\nZ,CX=17.5,fillstyle=solid}\pstCube}
\multido{\nZ=-7.5+5.0}{2}{%
  \psset{linecolor=red,CY=22.5,A=2.5,B=A,C=A,CZ=\nZ,CX=17.5,fillstyle=solid}\pstCube}
{\psset{fillstyle=gradient,gradbegin=white,gradend=blue,gradmidpoint=0.2,linecolor=cyan,linewidth=0.1mm,%
  CX=15,RotX=45,RotY=60}
\pstMirrorSphere{5}}
{\psset{fillstyle=gradient,gradbegin=white,gradend=red,gradmidpoint=0.2,linewidth=0.1mm,CX=10,CY=20,CZ=-10}
\pstSphereCylinder{5}{7.5}}
{\psset{fillstyle=gradient,gradbegin=white,gradend=red,gradmidpoint=0.2,linewidth=0.1mm,%
    CX=15,CY=10,CZ=-7.5,RotY=90,RotX=-45}\pstSphereCylinder{2.5}{-10}}
{\psset{linecolor=red,CX=10,A=2.5,B=A,C=A,CY=-40,CZ=-7.5,RotZ=60,fillstyle=solid,fillcolor=yellow}\pstDie}%
{\psset{CZ=-10,CY=-2.5,CX=2.5,RotZ=-20,linecolor=black,fillstyle=gradient,Rtetraedre=2.5}\pstTetraedre}
\multido{\nY=-67.5+10.0}{9}{%
  {\psset{linecolor=blue,CZ=17.5,A=1,B=2.5,C=2.5,CY=\nY,CX=19,fillstyle=solid,linewidth=0.1mm}\pstCube}}
\multido{\nY=-62.5+10.0}{8}{\psset{RotX=0,RotZ=0,RotY=-90,CX=20,CZ=12.5,CY=\nY,
  fillstyle=solid,linecolor=black,A=2.5,Hpyramide=3}\pstPyramide}
\end{pspicture}

\begin{lstlisting}
\begin{pspicture}(-8,-8)(8,8)
%\psset{scale=2}
\pstSphereFrame[Xorigine=20,Yorigine=0,Zorigine=0,fillstyle=solid,fillcolor=OrangePale](-70,-50)(70,10)
\pstSphereGrid[linecolor=blue,linewidth=0.2mm,grille=5,Ymin=-50,Ymax=50,Xmax=70,Xmin=-70](20,0,0)%
%
\pstSphereFrame[normale=90 0,Xorigine=10,Yorigine=70,Zorigine=0,fillstyle=solid,fillcolor=yellow](-10,-50)(10,10)
\multido{\nX=17.5+-10.0}{2}{%
  \multido{\nY=-67.5+10.0}{14}{%
    \pstSphereFrame[normale=0 90,Xorigine=\nX,Yorigine=\nY,Zorigine=-10,fillstyle=solid,
      fillcolor=yellow](-2.5,-2.5)(2.5,2.5)}}
\multido{\nX=12.5+-10.0}{2}{%
  \multido{\nY=-62.5+10.0}{14}{%
    \pstSphereFrame[normale=0 90,Xorigine=\nX,Yorigine=\nY,Zorigine=-10,fillstyle=solid,
      fillcolor=yellow](-2.5,-2.5)(2.5,2.5)}}
\pstSphereGrid[normale=90 0,linecolor=blue,linewidth=0.2mm,grille=5,
  Xmin=-10,Xmax=10,Ymax=10,Ymin=-50](10,70,0)%
\pstSphereGrid[normale=0 90,linecolor=blue,linewidth=0.2mm,grille=5,
  Xmin=-70,Xmax=70,Ymax=10,Ymin=-10](10,0,-10)%
{\psset{linecolor=black,CZ=52,A=5,B=70,C=2,CY=0,CX=15,fillstyle=solid}\pstCube}
{\psset{linecolor=red,CX=15,A=5,B=A,C=A,CY=-10,CZ=-5,RotZ=0,fillstyle=solid,fillcolor=blue}\pstDie}%
\multido{\nZ=47.5+-5.0}{10}{%
  \psset{linecolor=red,CY=22.5,A=2.5,B=A,C=A,CZ=\nZ,CX=17.5,fillstyle=solid}\pstCube}
\multido{\nZ=-7.5+5.0}{2}{%
  \psset{linecolor=red,CY=22.5,A=2.5,B=A,C=A,CZ=\nZ,CX=17.5,fillstyle=solid}\pstCube}
{\psset{fillstyle=gradient,gradbegin=white,gradend=blue,gradmidpoint=0.2,linecolor=cyan,linewidth=0.1mm,%
  CX=15,RotX=45,RotY=60}
\pstMirrorSphere{5}}
{\psset{fillstyle=gradient,gradbegin=white,gradend=red,gradmidpoint=0.2,linewidth=0.1mm,CX=10,CY=20,CZ=-10}
\pstSphereCylinder{5}{7.5}}
{\psset{fillstyle=gradient,gradbegin=white,gradend=red,gradmidpoint=0.2,linewidth=0.1mm,%
    CX=15,CY=10,CZ=-7.5,RotY=90,RotX=-45}\pstSphereCylinder{2.5}{-10}}
{\psset{linecolor=red,CX=10,A=2.5,B=A,C=A,CY=-40,CZ=-7.5,RotZ=60,fillstyle=solid,fillcolor=yellow}\pstDie}%
{\psset{CZ=-10,CY=-2.5,CX=2.5,RotZ=-20,linecolor=black,fillstyle=gradient,Rtetraedre=2.5}\pstTetraedre}
\multido{\nY=-67.5+10.0}{9}{%
  {\psset{linecolor=blue,CZ=17.5,A=1,B=2.5,C=2.5,CY=\nY,CX=19,fillstyle=solid,linewidth=0.1mm}\pstCube}}
\multido{\nY=-62.5+10.0}{8}{\psset{RotX=0,RotZ=0,RotY=-90,CX=20,CZ=12.5,CY=\nY,
  fillstyle=solid,linecolor=black,A=2.5,Hpyramide=3}\pstPyramide}
\end{pspicture}
\end{lstlisting}


\clearpage
\section{Utilisation des briques}
\subsection{Le repére}
l'origine est placée contre la boule, sur l'horizontale joignant le
centre~C de la boule é l'\oe{}il de l'observateur. O$z$ est
vertical, O$xy$ forme le plan horizontal. Tous les objets seront
donc placés au-delé de O avec $x>0$.
\begin{center}
\begin{pspicture}(-2,-2)(10,5)
\pscircle{2}
\rput(2,0){%
\psline{->}(5,0)
\psline{->}(0,5)
\psline{->}(4;30)
\uput[-90](5,0){$x$}
\uput[180](0,5){$z$}
\uput[30](4;30){$y$}}
\rput(10,0){\oeil}
\uput[180](2,0){O}
\psdot(0,0)
\end{pspicture}
\end{center}

\subsection{Placer un quadrillage}
Le plan du quadrillage est défini par une origine
(\textsf{Xorigine,Yorigine,Zorigine}), le vecteur unitaire normal au plan donné par
sa longitude et sa latitude $(\theta,\phi)$ lesquelles sont paramétrées, avec des valeurs en degrés avec :
\textsf{%
\begin{compactitem}
  \item normale =$\theta$ $\phi$
\end{compactitem}}
Un vecteur unitaire quelconque a pour coordonnées, en fonction de $\theta$ et $\phi$ :
$$
\vec{u}=\left(
\begin{array}{c}
\cos\theta\cos\phi\\
\sin\theta\cos\phi\\
\sin\phi
\end{array}
\right)
$$
$\vec{i},\vec{j},\vec{k}$ sont les vecteurs unitaires dirigés
respectivement suivant O$x$, O$y$ et O$z$.

Le vecteur unitaire normal é un plan de front, c'est-é-dire paralléle é O$yz$,
est le vecteur $\vec{i}$
ou $-\vec{i}$, $\vec{u}(\theta=0\textrm{ ou } \pi,\phi=0)$.

Un plan horizontal est paralléle é O$xy$, un vecteur normal est
soit $\vec{k}$, soit $-\vec{k}$, il est défini par
$\vec{u}(\theta=\textrm{valeur indifférente} (0),\phi=\pi/2\textrm{ ou }-\pi/2)$.

Un plan vertical paralléle é O$xz$ a pour normale $\vec{j}$ ou
$-\vec{j}$, donc $\theta=\pi/2\textrm{ ou }-\pi/2$ et $\phi=0$.

\textbf{Une remarque importante} : le vecteur unitaire choisi normal au plan $\vec{u}$, détermine
l'orientation des vecteurs $\vec{I}$ et $\vec{J}$ du plan, avec
les régles habituelles d'orientation : le triédre $(\vec{I},\vec{J},\vec{u})$ est direct,
 \textbf{ mais le vecteur $\vec{I}$ a étépris dans le plan de référence O$xy$}.

%$
Pour que le quadrillage soit parfaitement déterminé, on fixe les
valeurs de deux sommets opposé avec :
\textsf{Ymin=-50,Ymax=10,Xmax=70,Xmin=-70}. Ainsi que le pas de la
grille avec \textsf{grille=10}, par exemple.
\begin{center}
\begin{pspicture}(-5,-4)(5,5)
  \decor
  % les graduations
  \multido{\iY=-70+10}{15}{%
    \pstSpherePoint(40,\iY,0){A}
    \uput[90](A){\footnotesize\red\iY}}
  \multido{\iX=10+10}{4}{%
    \pstSpherePoint(\iX,0,-10){A}
    \uput[-45](A){\footnotesize\blue\iX}}
  \multido{\iZ=-10+10}{7}{%
    \pstSpherePoint(40,0,\iZ){A}
    \uput[135](A){\footnotesize\green\iZ}}
\end{pspicture}
\end{center}


\begin{lstlisting}
\begin{pspicture}(-5,-4)(5,5)
  \decor% les graduations, see preamble
  \multido{\iY=-70+10}{15}{\pstSpherePoint(40,\iY,0){A}\uput[90](A){\footnotesize\red\iY}}
  \multido{\iX=10+10}{4}{\pstSpherePoint(\iX,0,-10){A} \uput[-45](A){\footnotesize\blue\iX}}
  \multido{\iZ=-10+10}{7}{\pstSpherePoint(40,0,\iZ){A} \uput[135](A){\footnotesize\green\iZ}}
\end{pspicture}
\end{lstlisting}


En bleu graduations de l'axe O$x$, en rouge celles de O$x$ et en
vert celles de O$z$.

Le quadrillage du plan de front est défini par :

\begin{lstlisting}
\pstSphereGrid[normale=0 0,% default
  linecolor=blue,linewidth=0.2mm,grille=10,Ymin=-50,Ymax=10,Xmax=70,Xmin=-70](40,0,0)%
\end{lstlisting}
Le quadrillage du plan horizontal par :

\begin{lstlisting}
\pstSphereGrid[normale=0 90,linecolor=blue,linewidth=0.2mm,grille=10,
    Ymin=-30,Ymax=0,Xmax=70,Xmin=-70](40,0,-10)%
\end{lstlisting}

Les quadrillages des plans latéraux par :

\begin{lstlisting}
\pstSphereGrid[normale=90 0,linecolor=blue,linewidth=0.05mm,grille=10,
    Ymin=-60,Ymax=0,Xmin=-30,Xmax=0](40,70,-10)
\pstSphereGrid[normale=90 0,linecolor=blue,linewidth=0.05mm,grille=10,%
    Ymin=-60,Ymax=0,Xmin=-30,Xmax=0](40,-70,-10)
\end{lstlisting}

On peut se rendre compte de ce qui se produit en changeant l'orientation
du vecteur normal. J'inverse l'orientation de la normale pour les quatre plans.

On obtient la méme chose que précédemment en faisant :

\begin{lstlisting}
\begin{pspicture}(-5,-4)(5,4)
% plan de front
\psset{Sphere=false}
{\psset{normale=180 0,Xorigine=40,Yorigine=0,Zorigine=0}
\pstSphereFrame[fillstyle=solid,fillcolor=OrangePale](-70,-50)(70,10)
\pstSphereGrid[linecolor=blue,linewidth=0.2mm,grille=10,
                  Ymin=-50,Ymax=10,Xmax=70,Xmin=-70](40,0,0)}
% plan horizontal
{\psset{normale=0 -90,Xorigine=40,Yorigine=0,Zorigine=-10}
\pstSphereFrame[fillstyle=solid,fillcolor=cyan](70,30)(-70,0)
\pstSphereGrid[linecolor=blue,linewidth=0.2mm,grille=10,
  Ymin=0,Ymax=30,Xmax=70,Xmin=-70](40,0,-10)}%
% plan vertical de gauche
{\psset{normale=-90 0,Xorigine=40,Yorigine=70,Zorigine=-10}
\pstSphereFrame[fillstyle=solid,fillcolor=yellow](0,0)(30,-60)
\pstSphereGrid[linecolor=blue,linewidth=0.05mm,grille=10,
  Ymin=-60,Ymax=0,Xmin=0,Xmax=30](40,70,-10)}
% plan vertical a droite
{\psset{normale=-90 0,Xorigine=40,Yorigine=-70,Zorigine=-10}
\pstSphereFrame[fillstyle=solid,fillcolor=yellow](0,0)(30,-60)
\pstSphereGrid[linecolor=blue,linewidth=0.05mm,grille=10,Ymin=-60,
  Ymax=0,Xmin=0,Xmax=30](40,-70,-10)}
\end{pspicture}
\end{lstlisting}


\subsection{Dessiner un rectangle}
Dans l'exemple précédent, j'ai dessiné les rectangles
correspondants aux quadrillages. La commande est identique é celle
du quadrillage avec les mémes paramétres pour l'origine et le
vecteur normal au plan. On donne les coordonnées de deux sommets
opposés comme dans la commande \verb+\psframe+ de PSTricks, dans le
plan.

\subsection{Tracer une ligne, marquer un point}
Un segment de droite est déterminé par les coordonnées de ses
extrémités. Un point par ses coordonnées. On peut tracer une ligne brisée en plaéant les points é la suite, comme dans la commande de PStricks \verb+\psline+.
\begin{lstlisting}
\pstSpherePoint(40,0,40){A}
\pstSpherePoint(10,-40,-10){B}
\pstSpherePoint(10,40,-10){C}
\psdots[dotsize=1mm,linecolor=green](A)(B)(C)
\pstSphereLine[linecolor=red](10,40,-10)(40,0,40)(10,-40,-10)
\end{lstlisting}
Vous remarquerez que les coordonnées de $y$ sont inversées. Je
rappelle que le plan couleur \textsf{cyan} est é la cote $z=-10$.
\begin{center}
\begin{pspicture}(-5,-4)(5,5)
\decor
\pstSphereLine[linecolor=red,linewidth=1mm](10,40,-10)(40,0,40)(10,-40,-10)
\pstSpherePoint(40,0,40){A}
\pstSpherePoint(10,-40,-10){B}
\pstSpherePoint(10,40,-10){C}
\psdots[dotsize=2mm,linecolor=green](A)(B)(C)
\end{pspicture}
\end{center}

\subsection{Dessiner un polygone}
Commande identique é celle de PStricks.
\begin{lstlisting}
\psSpherePolygon[options](x0,y0,z0)(x1,y1,z1)...(xn,yn,zn)
\end{lstlisting}

\clearpage
\subsection{Parallélépipéde}
Le parallélépipéde est défini par ses trois dimensions
\textsf{A,B,C}. On peut le déplacer en fixant les coordonnées de
son centre avec : \textsf{CX,CY,CZ} et le faire tourner avec autour
des axes avec : \textsf{RotX,RotY,RotZ} dont les valeurs sont par défauts, positionnées é 0.
\begin{lstlisting}
\psset{A=2.5,B=5,C=A,fillstyle=solid,fillcolor=GrisClair,linecolor=red}
\psset{CX=10,CZ=-5,CY=0}
\pstCube
\end{lstlisting}
\begin{center}
\begin{pspicture}(-5,-4)(5,5)
\decor
\psset{A=2.5,B=5,C=A,fillstyle=solid,fillcolor=GrisClair,linecolor=red}
\psset{CX=15,CZ=-5,CY=0}
\pstCube
\end{pspicture}
\end{center}

On le fait tourner autour de O$z$ avec \Lkeyword{RotZ}=30 etc.

\begin{center}
\begin{pspicture}(-5,-4)(5,5)
\decor
\psset{A=2.5,B=5,C=A,fillstyle=solid,fillcolor=GrisClair,linecolor=red}
\psset{CX=15,CZ=-5,CY=0,RotZ=30}
\pstCube
\end{pspicture}
\end{center}

\begin{lstlisting}
\psset{CX=15,CZ=-5,CY=0,RotZ=30}
\end{lstlisting}

\clearpage
\subsection{Le dé}
Le dé posséde une commande propre \Lcs{pstDie}, qui en interne utilise celle
du \Lcs{pstCube} en affichant les points. Voici quelques dés.
\begin{center}
\begin{pspicture}(-5,-4)(5,5)
\decor
\psset{A=2.5,B=A,C=A,fillstyle=solid,fillcolor=GrisClair,linecolor=red}
\psset{CX=15,CZ=-5,CY=10,RotZ=60,RotX=90}
\pstDie
\psset{A=5,B=A,C=A,fillstyle=solid,fillcolor=OrangeTresPale,linecolor=red}
\psset{CX=25,CZ=5,CY=-5,RotZ=45,RotY=20,RotX=-60}
\pstDie
\end{pspicture}
\end{center}

\begin{lstlisting}
\psset{A=2.5,B=A,C=A,fillstyle=solid,fillcolor=GrisClair,linecolor=red}
\psset{CX=15,CZ=-5,CY=10,RotZ=60,RotX=90}
\pstDie
\psset{A=5,B=A,C=A,fillstyle=solid,fillcolor=OrangeTresPale,linecolor=red}
\psset{CX=25,CZ=5,CY=-5,RotZ=45,RotY=20,RotX=-60}
\pstDie
\end{lstlisting}

\subsection{Tracer un cercle ou un arc de cercle}
Le cercle et l'arc de cercle seront définis par le plan auquel ils
appartiennent. Comme précédemment le plan sera défini par une
origine et le vecteur unitaire normal é ce plan. Ces données étant
posées, il n'y plus que le rayon du cercle é fixer et les deux
angles limites s'il s'agit de l'arc de cercle.

\begin{BDef}
\Lcs{pstSphereCircle}\OptArgs\Largb{radius}\\
\Lcs{pstSphereArc}\OptArgs\Largb{radius}\Largb{angle\_min}\Largb{angle\_max}
\end{BDef}

\verb+\pstSphereArc+ : respectez l'ordre des angles limites sinon l'interpréteur
\textsf{PostScript} signalera une erreur.

\begin{center}
\begin{pspicture}(-5,-4)(5,5)
  \decor
  \psset{Xorigine=40}%
  \multido{\iYorigine=-65+10}{14}{\psset{Yorigine=\iYorigine,Zorigine=15}
    \pstSphereCircle[linecolor=red]{5}}
  \multido{\iYorigine=-65+10}{14}{\psset{Yorigine=\iYorigine,Zorigine=50}%
    \pstSphereArc[linecolor=blue,linewidth=0.5mm,fillstyle=solid,fillcolor=OrangePale]{5}{180}{360}}
\end{pspicture}
\end{center}

Des cercles et des arcs de cercle sur le plan de front :
\begin{lstlisting}
\psset{Xorigine=40}
\multido{\iYorigine=-65+10}{14}{%
  \psset{Yorigine=\iYorigine,Zorigine=15}\pstSphereCircle[linecolor=red]{5}}
\multido{\iYorigine=-65+10}{14}{\psset{Yorigine=\iYorigine,Zorigine=50}
  \pstSphereArc[linecolor=blue,linewidth=0.5mm,fillstyle=solid,fillcolor=OrangePale]{5}{180}{360}}
\end{lstlisting}

\clearpage
\subsection{La sphére}
Elle est définie par son rayon, son centre se place avec les
paramétres \Lkeyword{CX}, \Lkeyword{CY}, \Lkeyword{CZ}, on la fait tourner autour de son
centre avec \Lkeyword{RotX}, \Lkeyword{RotY}, \Lkeyword{RotZ}.

\begin{center}
\begin{pspicture}(-5,-4)(5,5)
\decor% defined in the preamble
\multido{\iCY=-65+10,\iRotX=0+20}{7}{%
\psset{CY=\iCY,CZ=-5,CX=15,RotX=\iRotX}
\psset{fillstyle=gradient,gradbegin=white,gradend=red,%
       gradmidpoint=0.2,linecolor=magenta,linewidth=0.05mm}
\ifnum\multidocount=7
\psset{CZ=0}
\fi
\ifnum\multidocount=3
\psset{CZ=20}
\fi
\pstMirrorSphere{5}}
\multido{\iCY=65+-10,\iRotX=0+15}{7}{%
\psset{CY=\iCY,CZ=-5,CX=15,RotX=\iRotX}
\psset{fillstyle=gradient,gradbegin=white,gradend=red,%
       gradmidpoint=0.2,linecolor=magenta,linewidth=0.05mm}
\ifnum\multidocount=6
\psset{CZ=10}
\fi
\ifnum\multidocount=3
\psset{CZ=20,CY=5}
\fi
\pstMirrorSphere{5}}
\end{pspicture}
\end{center}
\begin{lstlisting}
\multido{\iCY=-65+10,\iRotX=0+20}{7}{%
  \psset{CY=\iCY,CZ=-5,CX=15,RotX=\iRotX,fillstyle=gradient,gradbegin=white,gradend=red,%
       gradmidpoint=0.2,linecolor=magenta,linewidth=0.05mm}\pstMirrorSphere{5}}
\end{lstlisting}

\clearpage
\subsection{Dessiner un cylindre}
Le cylindre sera donné par le rayon de sa base et sa hauteur. Le
centre de la base sera positionné avec \Lkeyword{CX}, \Lkeyword{CY}, \Lkeyword{CZ}, on le fait tourner autour
autour des axes avec \Lkeyword{RotX}, \Lkeyword{RotY}, \Lkeyword{RotZ}.

\begin{BDef}
\Lcs{pstSphereCylinder}\OptArgs\Largb{radius}\Largb{hauteur}
\end{BDef}

\begin{center}
\begin{pspicture}(-5,-4)(5,5)
  \decor
%  \pstTextSphere[fillstyle=solid,fillcolor=red,fontscale=10,PSfont=Helvetica,normale=0 90](15,0,-10){l'aventure mathématique}
  \psset{fillstyle=gradient,gradbegin=white,gradend=red,gradmidpoint=0.2,linewidth=0.1mm}
  { \psset{CX=20,CY=20,CZ=-10} \pstSphereCylinder{5}{7.5} }
  { \psset{CX=15,CY=10,CZ=-7.5,RotY=90,RotX=-45} \pstSphereCylinder{2.5}{-10}}
%  \pstTextSphere[fillstyle=solid,fillcolor=red,fontscale=25,PSfont=Bookman-demibold,
%    normale=0 0](40,0,10){Tangente}
\end{pspicture}
\end{center}

\begin{lstlisting}
\psset{fillstyle=gradient,gradbegin=white,gradend=red,%
       gradmidpoint=0.2,linewidth=0.1mm}
{\psset{CX=20,CY=20,CZ=-10}
\pstSphereCylinder{5}{7.5}}
{\psset{CX=15,CY=10,CZ=-7.5,RotY=90,RotX=-45}
\pstSphereCylinder{2.5}{-10}}
\end{lstlisting}

\clearpage
\subsection{Dessiner un céne}
Le céne se définit comme le cylindre, rayon de la base, hauteur et
en option la fraction de hauteur de céne si l'on souhaite avoir un
tronc de céne.

\begin{BDef}
\Lcs{pstSphereCone}\OptArgs\Largb{radius}\Largb{hauteur}
\end{BDef}


\Lkeyword{fracHcone} est obligatoirement un nombre compris entre 0 et 1.

\begin{center}
\begin{pspicture}(-5,-4)(5,5)
\decor
\psset{linecolor=red,fillstyle=gradient,gradbegin=yellow,gradend=red,gradmidpoint=0,linewidth=0.05mm}
{\psset{CX=15,CZ=-10,CY=5,fracHcone=0.5}\pstSphereCone{5}{10}}
{\psset{CX=40,CZ=5,CY=-25,RotY=-90}     \pstSphereCone{5}{10}}
{\psset{CX=25,CZ=5,CY=-70,RotX=-90}     \pstSphereCone{5}{30}}
\end{pspicture}
\end{center}

\begin{lstlisting}
{\psset{CX=15,CZ=-10,CY=5,fracHcone=0.5}\pstSphereCone{5}{10}}
{\psset{CX=40,CZ=5,CY=-25,RotY=-90}     \pstSphereCone{5}{10}}
{\psset{CX=25,CZ=5,CY=-70,RotX=-90}     \pstSphereCone{5}{30}}
\end{lstlisting}

\clearpage
\subsection{Tracer une pyramide}
La pyramide é base carrée est donnée par le demi-cété (\Lkeyword{A})de sa
  base et sa hauteur (\Lkeyword{Hpyramide}), que l'on peut faire tourner et placer oé on
  veut, comme les objets précédents.

\begin{center}
\begin{pspicture}(-5,-4)(5,5)
\decor
\psset{fillstyle=solid,linecolor=black,A=5}
\multido{\nY=-65+20.0}{7}{%
\psset{RotX=0,RotZ=0,RotY=-90,CX=40,CZ=5,CY=\nY,Hpyramide=5}
\pstPyramide}
\multido{\nY=65+-20.0,\iRotZ=0+20}{6}{%
\psset{RotX=0,RotZ=\iRotZ,RotY=0,CX=20,CZ=-10,CY=\nY,Hpyramide=8}
\pstPyramide}
\end{pspicture}
\end{center}

\begin{lstlisting}
\psset{fillstyle=solid,linecolor=black,A=5}
\multido{\nY=-65+20.0}{4}{%
  \psset{RotX=0,RotZ=0,RotY=-90,CX=40,CZ=5,CY=\nY,Hpyramide=5}\pstPyramide}
\multido{\nY=65+-20.0,\iRotZ=0+20}{6}{%
  \psset{RotX=0,RotZ=\iRotZ,RotY=0,CX=20,CZ=-10,CY=\nY,Hpyramide=8}\pstPyramide}
\end{lstlisting}

\section{Image d'un texte}

C'est la commande \Lcs{pstTextSphere}\OptArgs\Largs{x,y,z}\Largb{texte} qui permet d'afficher le texte 
souhaité avec les paramétres suivants :
\begin{compactitem}
  \item Le plan qui contient le texte est défini par :
  \begin{compactitem}
    \item sa normale \Lkeyword{normale}=$\theta$ $\phi$ ;
    \item et son origine \texttt{(x,y,z)}.
  \end{compactitem}
  \item La taille des caractéres est fixée par \Lkeyword{fontscale}=1 : 1 cm par défaut.
  \item Le type de fonte par \Lkeyset{PSfont=Times-Roman} par défaut.
  \item Le booléen \Lkeyword{isolatin} qui est positionné é \texttt{true} par défaut, permet l'écriture 
    des caractéres accentués. Si l'on souhaite utiliser la police \texttt{Symbol}, permettant d'écrire 
    les caractéres grecs, il faut positionner ce booléen é \texttt{false}.
\end{compactitem}
\begin{center}
\begin{pspicture}(-8,-10)(8,10)
%\psset{Rayon=8}
\newpsstyle{GradWhiteYellow}{fillstyle=gradient,%
    gradbegin=yellow,gradend=yellow!20,linecolor=yellow!50,GradientCircle=true,gradmidpoint=0,GradientPos={(1,1)}}%
\pscircle[style=GradWhiteYellow]{7.07}
\pstSphereGrid[linecolor=red,grille=10,Ymin=-50,Ymax=50,Xmax=80,Xmin=-80,normale=0 0](20,0,0)
\pstSphereGrid[linecolor=blue,grille=10,Ymin=-40,Ymax=-20,Xmax=80,Xmin=-80,normale=0 90](40,0,-10)%
\pstTextSphere[fillstyle=solid,fillcolor=red,normale=0 0,fontscale=30,PSfont=Bookman-demibold,yO=0](20,0,10){Tangente}
\pstTextSphere[fillstyle=solid,fillcolor=blue,normale=0 90,fontscale=10,
  PSfont=Helvetica,yO=2.5](10,0,-10){l'aventure math\'ematique}%
\end{pspicture}
\end{center}

\begin{lstlisting}
\begin{pspicture}(-8,-10)(8,10)
%\psset{Rayon=8}
\newpsstyle{GradWhiteYellow}{fillstyle=gradient,gradbegin=yellow,gradend=yellow!20,
  linecolor=yellow!50,GradientCircle=true,gradmidpoint=0,GradientPos={(1,1)}}%
\pscircle[style=GradWhiteYellow]{7.07}
\pstSphereGrid[linecolor=red,grille=10,Ymin=-50,Ymax=50,Xmax=80,Xmin=-80,normale=0 0](20,0,0)
\pstSphereGrid[linecolor=blue,grille=10,Ymin=-40,Ymax=-20,Xmax=80,Xmin=-80,normale=0 90](40,0,-10)%
\pstTextSphere[fillstyle=solid,fillcolor=red,normale=0 0,fontscale=30,
  PSfont=Bookman-demibold,yO=0](20,0,10){Tangente}
\pstTextSphere[fillstyle=solid,fillcolor=blue,normale=0 90,fontscale=10,
  PSfont=Helvetica,yO=2.5](10,0,-10){l'aventure math\'ematique}%
\end{pspicture}
\end{lstlisting}

\section{Conclusion}
Nous pouvons maintenant nous atteler é la téche de construire des
décors en 3D vus dans une boule qui soient intéressants et
artistiques, un prolongement vers une animation est toujours possible.
J'ai pensé inclure dans ce package, é l'aide d'une option,
la 3D conventionnelle : il suffit de changer les formules
relatives é la boule et d'y rajouter les paramétres relatifs au
point de vue, mais finalement cela aurait trop alourdi cette
étude -- vous pouvez regardez la commande \Lcs{pnodeTroisD} et les
nombreuses applications développées dans ma page personnelle.

\textbf{Une remarque importante :} si le calcul des faces cachées
est pris en compte pour chaque objet pris individuellement, il en
est pas de méme pour l'ensemble du panorama. Il convient donc de
placer en premier les objets les plus éloignés de boule, puis ceux
qui sont de plus en plus proche de la boule.


\section{List of all optional arguments for \texttt{pst-mirror}}

\xkvview{family=pst-mirror,columns={key,type,default}}


\bgroup
\nocite{*}
\raggedright
\bibliographystyle{plain}
\bibliography{\jobname}
\egroup

\printindex


\end{document}
