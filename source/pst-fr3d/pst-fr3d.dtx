% \iffalse meta-comment, etc.
%%
%% Package `pst-fr3d.dtx'
%%
%% Denis Girou (CNRS/IDRIS - France) <Denis.Girou@idris.fr>
%%
%% February 25, 2002
%%
%% This program can be redistributed and/or modified under the terms
%% of the LaTeX Project Public License Distributed from CTAN archives
%% in directory macros/latex/base/lppl.txt.
%%
%% DESCRIPTION:
%%   `pst-fr3d' is a PSTricks package to draw three dimensional grids
%%   with various customizations.
%%
% \fi
%
% \DoNotIndex{\\,\^,\@}
% \DoNotIndex{\@ifnextchar,\@nameuse,\@nil,\@undefined}
% \DoNotIndex{\advance}
% \DoNotIndex{\begin}
% \DoNotIndex{\catcode,\CodelineIndex,\csname}
% \DoNotIndex{\def,\define@key,\definecolor,\divide,\DocInput}
% \DoNotIndex{\documentclass,\dp}
% \DoNotIndex{\edef,\else,\EnableCrossrefs,\end,\endcsname,\endinput}
% \DoNotIndex{\endpspicture}
% \DoNotIndex{\fi,\filedate,\fileversion}
% \DoNotIndex{\GetFileInfo}
% \DoNotIndex{\hbadness,\hbox,\hfuzz,\ht}
% \DoNotIndex{\ifdim,\ifx,\input}
% \DoNotIndex{\let}
% \DoNotIndex{\message}
% \DoNotIndex{\newhsbcolor,\newif}
% \DoNotIndex{\OnlyDescription}
% \DoNotIndex{\p@,\ProvidesPackage,\psframebox,\psframesep}
% \DoNotIndex{\psline,\pslinewidth,\pspolygon,\psset,\PstAtCode,\pst@boxg}
% \DoNotIndex{\pst@dima,\pst@dimb,\pst@dimc,\pst@dimd,\pst@dimh}
% \DoNotIndex{\pst@dimtonum,\pst@expandafter,\pst@getnumiii,\pst@hbox}
% \DoNotIndex{\pst@tempa,\pst@tempb,\pst@tempc,\pst@tempg,\pst@temph}
% \DoNotIndex{\pst@tempi,\PSTricksLoaded}
% \DoNotIndex{\RecordChanges,\relax}
% \DoNotIndex{\setbox,\setkeys,\space}
% \DoNotIndex{\the,\tw@}
% \DoNotIndex{\usepackage}
% \DoNotIndex{\wd}
% \DoNotIndex{\z@}
%
% \setcounter{IndexColumns}{2}
%
% \newcommand{\PstFrameBoxThreeDPackage}{`\textsf{pst-fr3d}'}
% \newcommand{\PstFrameBoxThreeDMacro}{\cs{PstFrameBoxThreeD}}
% \newcommand{\ColorPackage}{`\textsf{color}'}
% \newcommand{\PstColPackage}{`\textsf{pstricks}'}
%
% ^^A From ltugboat.cls
%
% ^^A Typeset the name of an environment
% \providecommand\env[1]{\textsf{#1}}
% \providecommand\clsname[1]{\textsf{#1}}
% \providecommand\pkgname[1]{\textsf{#1}}
% \providecommand\optname[1]{\textsf{#1}}
% \providecommand\progname[1]{\textsf{#1}}
%
% ^^A A list of options for a package/class
% \newenvironment{optlist}{\begin{description}%
%   \renewcommand\makelabel[1]{%
%     \descriptionlabel{\mdseries\optname{##1}}}%
%   \itemsep0.25\itemsep}%
%  {\end{description}}
%
% ^^A Utility macros
%
%
% ^^A Example macros - adapted from the `fvrb-ex' package
% ^^A ---------------------------------------------------
%
% ^^A Take care that we use here the four /?_W characters as escape
% ^^A characters, so we can't use these characters in the examples!
%
% \makeatletter
%
% ^^A To highlight some verbatim sequences (comments, macro names, etc.)
% \def\HLEmphasize#1{\textit{#1}}
% \newcommand{\BS}{\texttt{\symbol{`\\}}}
% \def\HLMacro#1{\BS{}def\HLMacro@i#1\@nil}
% \def\HLMacro@i#1def#2\@nil{\HLReverse{#2}}
% \def\HLReverse#1{{\setlength{\fboxsep}{1pt}\HLReverse@i{#1}}}
% \def\HLReverse@i#1{\colorbox{black}{\textcolor{white}{\textbf{#1}}}}
%
% \def\Example{\FV@Environment{}{Example}}
% \def\endExample{%
% \end{VerbatimOut}
% \Below@Example{\input{\jobname.tmp}}
% \endgroup}
%
% \def\CenterExample{\FV@Environment{}{Example}}
% \def\endCenterExample{%
% \end{VerbatimOut}
% \begin{center}
%   \Below@Example{\input{\jobname.tmp}}
% \end{center}
% \endgroup}
%
% \def\SideBySideExample{\FV@Environment{}{Example}}
% \def\endSideBySideExample{%
% \end{VerbatimOut}
% \SideBySide@Example{\input{\jobname.tmp}}
% \endgroup}
%
% \def\FVB@Example{%
% \begingroup
% \FV@UseKeyValues
% \parindent=0pt
% \multiply\topsep by 2
% \VerbatimEnvironment
% \begin{VerbatimOut}[gobble=4,codes={\catcode`\W=12}]{\jobname.tmp}}
%
% \def\Below@Example#1{%
% \VerbatimInput[gobble=0,commentchar=W,commandchars=/?_,frame=single,
%                numbers=left,numbersep=3pt]{\jobname.tmp}
% \catcode`\%=14\relax
% \catcode`\W=9\relax
% ^^A We suppress the effect of the highlighting macros
% \catcode`/=0\relax
% \catcode`?=1\relax
% \catcode`_=2\relax
% \def\HLEmphasize##1{##1}%
% \def\HLMacro##1{##1}%
% \def\HLReverse##1{##1}%
% #1
% \par}
%
% \def\SideBySide@Example#1{%
% \vskip 1mm
% \@tempdimb=\FV@XRightMargin
% \advance\@tempdimb -5mm
% \begin{minipage}[c]{\@tempdimb}
%   \fvset{xrightmargin=0pt}
%   \catcode`\%=14\relax
%   \catcode`\W=9\relax
%   ^^A We suppress the effect of the highlighting macros
%   \catcode`/=0\relax
%   \catcode`?=1\relax
%   \catcode`_=2\relax
%   \def\HLEmphasize##1{##1}%
%   \def\HLMacro##1{##1}%
%   \def\HLReverse##1{##1}%
%   #1
% \end{minipage}%
% \@tempdimb=\textwidth
% \advance\@tempdimb -\FV@XRightMargin
% \advance\@tempdimb 5mm
% \begin{minipage}[c]{\@tempdimb}
%   \VerbatimInput[gobble=0,commentchar=W,commandchars=/?_,
%                  frame=single,numbers=left,numbersep=3pt,
%                  xleftmargin=5mm,xrightmargin=0pt]{\jobname.tmp}
% \end{minipage}
% \vskip 1mm}
%
% \makeatother
%
% ^^A End of example macros from `fvrb-ex'
%
% ^^A For the possible index and changes log
% \setlength{\columnseprule}{0.6pt}
%
% ^^A Beginning of the documentation itself
%
% \title{The \PstFrameBoxThreeDPackage{} package\\
%        A PSTricks package\\
%        for three dimensional framed boxes}
% \author{Denis \textsc{Girou}\thanks{CNRS/IDRIS ---
%         Centre National de la Recherche Scientifique /
%         Institut du D\'eveloppement et des Ressources en Informatique
%         Scientifique --- Orsay --- France ---
%         \mbox{\texttt{<Denis.Girou@idris.fr>}}.
%   Version 1.1 prepared by Herbert Vo\ss for using \texttt{pst-xkey}}}
% \date{Version 1.1\\October 10, 2005\\
%       {\small Documentation revised October 10, 2005}}
%
% \maketitle
%
% \begin{abstract}
%     This package allow to draw three dimensional framed boxes using the
%   macro \PstFrameBoxThreeDMacro. This is specially useful to draw 3d-like
%   buttons.
% \end{abstract}
%
% \tableofcontents
%
% \section{Introduction}
%
%   \PstFrameBoxThreeDPackage{} offer a unique main macro with few parameters to
% interact on it. This is just a small extension of the \cs{psframebox} macro.
%
%   All the relevant PSTricks parameters can be use with it, but the
% \texttt{linestyle}, \texttt{linecolor}, \texttt{doubleline} and
% \texttt{doublecolor} ones must not be used, as they would be overridden.
% Nevertheless, some usual parameters are not relevant for this kind of boxes
% and would produce ``strange'' results (for instance \texttt{border},
% \texttt{linearc} and \texttt{shadow}).
%
%   As usual, \LaTeX{} users are supposed to have loaded the \PstColPackage{}
% package before this one, to ensure the compatibility with the standard
% \ColorPackage{} \LaTeX{} package.
%
%   The syntax of the macro is the same than the one for the other macros for
% framed boxes, so simply:
%
% \vspace{1mm}
% \noindent%
% \fbox{\PstFrameBoxThreeDMacro\texttt{[optional\_parameters]\{\emph{stuff}\}}}
%
% \vspace{1mm}
%   We can put all the valid material inside such 3d boxes, including other
% framed boxes.
%
%   Note also that, obviously, this documentation is very color oriented, and
% is done to be read on a screen or a colored paper.
%
% \section{Usage}
%
% \subsection{Default behavior}
%
%   The default behavior is the following:
%
% \begin{CenterExample}
%   \PstFrameBoxThreeD{Button}\hfill
%   \PstFrameBoxThreeD{\shortstack{Yes!\\Yes!\\Yes!}}\hfill
%   \PstFrameBoxThreeD{\Huge$\frac{1}{\sqrt{3}}$}%
% \end{CenterExample}
%
%   Of course, we can use all the relevant generic PSTricks parameters for the
% framed boxes (but not those relative to \emph{double} lines), specially
% those applying to boxes:
%
% \begin{CenterExample}
%   \PstFrameBoxThreeD[doublesep=0.1]{Button}\hfill
%   \PstFrameBoxThreeD[doublesep=0.2]{Button}\hfill
%   \PstFrameBoxThreeD[doublesep=0.5]{Button}%
% \end{CenterExample}
%
%   Specially look at the behavior of the \texttt{framesep},
% \texttt{doublesep} and \texttt{linewidth} parameters:
%
% \begin{CenterExample}
%   \PstFrameBoxThreeD[framesep=0.3]{Button}\hfill
%   \PstFrameBoxThreeD[doublesep=0.3]{Button}\hfill
%   \PstFrameBoxThreeD[linewidth=0.3]{Button}\hfill
%   \PstFrameBoxThreeD[doublesep=0.3,linewidth=0.3]{Button}\hfill
%   \PstFrameBoxThreeD[framearc=0.5]{Button}%
% \end{CenterExample}
%
% \subsection{Parameters}
%
%   To change the way the boxes are drawn, we can also use \textbf{four}
% specific parameters:
%
% \begin{optlist}
%   \item [FrameBoxThreeDColorHSB (real real real)]: to define the colors
% used, in the HSB (\emph{hue}, \emph{saturation}, \emph{brightness}) model.
% This is a sequence of three real values, separated by spaces. This will
% define the color of the interior of the frame. And for the borders we will
% use it too, changing it brightness in light and dark.
%
%         Note also that if the \emph{brightness} value is negative, no color
% will fill the interior of the box (this is useful if the general background
% is not white).
%   (\emph{Default:~0~0~0.5} --- it must be three numbers between 0~and~1, but
% as explained the last one can be negative, which has a special meaning).
% \end{optlist}
%
% \begin{CenterExample}
%   \psset{doublesep=0.1}
%   \PstFrameBoxThreeD[/HLEmphasize?FrameBoxThreeDColorHSB_=/HLReverse?0.1 0.3 0.5_]{%
%     Button}\hfill
%   \PstFrameBoxThreeD[/HLEmphasize?FrameBoxThreeDColorHSB_=/HLReverse?0.6 0.3 0.5_]{%
%     Button}\hfill
%   \PstFrameBoxThreeD[/HLEmphasize?FrameBoxThreeDColorHSB_=/HLReverse?0.4 0.1 0.5_]{%
%     \Huge$\frac{1}{\sqrt{3}}$}\hfill
%   \PstFrameBoxThreeD[/HLEmphasize?FrameBoxThreeDColorHSB_=/HLReverse?0.9 0.5 0.5_]{%
%     \psovalbox[fillstyle=solid,fillcolor=yellow]{Button}}\hfill
%   \PstFrameBoxThreeD[/HLEmphasize?FrameBoxThreeDColorHSB_=/HLReverse?0.4 0.7 0.2_]{%
%     \PstFrameBoxThreeD[/HLEmphasize?FrameBoxThreeDColorHSB_=/HLReverse?0.4 0.7 0.5_]{%
%       \PstFrameBoxThreeD[/HLEmphasize?FrameBoxThreeDColorHSB_=/HLReverse?0.4 0.7 0.8_]{%
%         Button}}}%
% \end{CenterExample}
%
% \begin{CenterExample}
%   \psset{doublesep=0.2}
%   \PstFrameBoxThreeD[/HLEmphasize?FrameBoxThreeDColorHSB_=/HLReverse?0 0 0.3_]{Button}\hfill
%   \PstFrameBoxThreeD[/HLEmphasize?FrameBoxThreeDColorHSB_=/HLReverse?0 0 0.7_]{Button}\hfill
%   \PstFrameBoxThreeD[/HLEmphasize?FrameBoxThreeDColorHSB_=/HLReverse?0 0 -1_]{Button}\hfill
%   \definecolor{MyGray}{gray}{0.85}
%   \psframe*[linecolor=MyGray](-0.3,-0.65)(2.1,0.9)
%   \PstFrameBoxThreeD[/HLEmphasize?FrameBoxThreeDColorHSB_=/HLReverse?0 0 -1_]{Button}%
% \end{CenterExample}
%
% \begin{SideBySideExample}[xrightmargin=3cm]
%   \PstFrameBoxThreeD%
%     [linewidth=0.2,
%      /HLEmphasize?FrameBoxThreeDColorHSB_=/HLReverse?0.3 0.2 0.5_]{%
%     \Huge\textcolor{white}{\textbf{21}}}
% \end{SideBySideExample}
%
% \begin{optlist}
%   \item [FrameBoxThreeDOn (boolean)]: to simulate an ON / OFF button, by
% exchange of the dark and light colors of the borders (this will give the
% same result that to use a negative value for the
% \texttt{FrameBoxThreeDBrightnessDistance} parameter)
%   (\emph{Default:~true}).
% \end{optlist}
%
% \begin{CenterExample}
%   \PstFrameBoxThreeD[/HLEmphasize?FrameBoxThreeDColorHSB_=/HLReverse?0.1 0.9 0.5_]{%
%     Button}\hfill
%   \PstFrameBoxThreeD[/HLEmphasize?FrameBoxThreeDColorHSB_=/HLReverse?0.1 0.9 0.5_,
%                      /HLEmphasize?FrameBoxThreeDOn_=/HLReverse?false_]{Button}\hfill
%   \PstFrameBoxThreeD[/HLEmphasize?FrameBoxThreeDColorHSB_=/HLReverse?0.1 0.9 0.5_,
%                      /HLEmphasize?FrameBoxThreeDOn_=/HLReverse?false_,linewidth=0.2]{Button}%
% \end{CenterExample}
%
% \begin{CenterExample}
%   \PstFrameBoxThreeD[linewidth=0.5,/HLEmphasize?FrameBoxThreeDOn_=/HLReverse?false_,
%                      /HLEmphasize?FrameBoxThreeDColorHSB_=/HLReverse?0.8 0.5 0.5_]{%
%     \Large\textcolor{yellow}{\textbf{This is your last chance...}}}
% \end{CenterExample}
%
% \begin{optlist}
%   \item [FrameBoxThreeDOpposite (boolean)]: to use the same color for the
%   left and right borders, and the same other for the bottom and top ones, or
%   rather for left and top and right and bottom.
%   (\emph{Default:~false}).
% \end{optlist}
%
% \begin{CenterExample}
%   \psset{linewidth=0.2}
%   \newcommand{\Button}{\textcolor{white}{\textbf{Button}}}
%   \PstFrameBoxThreeD[/HLEmphasize?FrameBoxThreeDColorHSB_=/HLReverse?0.7 0.8 0.8_]{%
%     \Button}\hfill
%   \PstFrameBoxThreeD[/HLEmphasize?FrameBoxThreeDColorHSB_=/HLReverse?0.7 0.8 0.8_,
%                      /HLEmphasize?FrameBoxThreeDOn_=/HLReverse?false_]{\Button}\hfill
%   \PstFrameBoxThreeD[/HLEmphasize?FrameBoxThreeDColorHSB_=/HLReverse?0.7 0.8 0.8_,
%                      /HLEmphasize?FrameBoxThreeDOpposite_=/HLReverse?true_]{\Button}\hfill
%   \PstFrameBoxThreeD[/HLEmphasize?FrameBoxThreeDColorHSB_=/HLReverse?0.7 0.8 0.8_,
%                      /HLEmphasize?FrameBoxThreeDOn_=/HLReverse?false_,
%                      /HLEmphasize?FrameBoxThreeDOpposite_=/HLReverse?true_]{\Button}%
% \end{CenterExample}
%
% \begin{optlist}
%   \item [FrameBoxThreeDBrightnessDistance (real)]: the distance brightness
%   of the color used to fill the interior of the box. If this is a negative
%   number, we will got the same effect that the one of setting
%   \texttt{FrameBoxThreeDOn} to false. And if it value is 0, the borders will
%   have obviously the same color that the interior.
%   (\emph{Default:~0.15} --- it must be a number between -0.5~and~0.5, but the
%   \emph{brightness} plus and minus it must stay between 0~and~1).
% \end{optlist}
%
% \begin{CenterExample}
%   \psset{doublesep=0.2}
%   \PstFrameBoxThreeD[/HLEmphasize?FrameBoxThreeDColorHSB_=/HLReverse?0.6 0.3 0.5_]{%
%     Button}\hfill % /HLEmphasize?Default = 0.15_
%   \PstFrameBoxThreeD[/HLEmphasize?FrameBoxThreeDColorHSB_=/HLReverse?0.6 0.3 0.5_,
%                      /HLEmphasize?FrameBoxThreeDBrightnessDistance_=/HLReverse?0.3_]{%
%     Button}\hfill
%   \PstFrameBoxThreeD[/HLEmphasize?FrameBoxThreeDColorHSB_=/HLReverse?0.6 0.3 0.5_,
%                      /HLEmphasize?FrameBoxThreeDBrightnessDistance_=/HLReverse?0.5_]{%
%     Button}\hfill
%   \PstFrameBoxThreeD[/HLEmphasize?FrameBoxThreeDColorHSB_=/HLReverse?0.6 0.3 0.7_,
%                      /HLEmphasize?FrameBoxThreeDBrightnessDistance_=/HLReverse?0.3_]{%
%     Button}\hfill
%   \PstFrameBoxThreeD[/HLEmphasize?FrameBoxThreeDColorHSB_=/HLReverse?0.6 0.3 0.5_,
%                      /HLEmphasize?FrameBoxThreeDBrightnessDistance_=/HLReverse?-0.2_]{%
%     Button}%
% \end{CenterExample}
%
% \section{Examples}
%
%   We give here more advanced examples.
%
% \begin{CenterExample}
%   \psset{doublesep=0.12}
%   \multido{\nButtonA=0.1+0.4}{3}{%
%     \multido{\nButtonB=0+0.05}{11}{%
%       \PstFrameBoxThreeD%
%         [FrameBoxThreeDColorHSB=\nButtonA\space 0.5 0.5,
%          FrameBoxThreeDBrightnessDistance=\nButtonB]{OK}}\\}
% \end{CenterExample}
%
% \begin{CenterExample}
%   \psset{doublesep=0.05}
%   \multido{\nButtonA=0+0.1}{11}{%
%     \multido{\nButtonB=0+0.05}{21}{%
%       \PstFrameBoxThreeD[FrameBoxThreeDColorHSB=%
%                          \nButtonB\space \nButtonA\space 0.8]{\$}}\\}
% \end{CenterExample}
%
% \begin{CenterExample}
%   \psset{gradmidpoint=1,gradangle=90}
%   \psdblframebox[framesep=0.3,fillstyle=gradient,
%                  gradbegin=yellow,gradend=LemonChiffon]{%
%     \shortstack[l]{%
%       \makebox[8cm]{\LARGE\textbf{Introduction}\dotfill}\\
%       {\Large by Catherine \textsc{Parker}}\hfill
%          \raisebox{5.5mm}{%
%            \PstFrameBoxThreeD[linewidth=0.15,
%                               FrameBoxThreeDColorHSB=0.3 0.2 0.5]{%
%              \LARGE\textcolor{white}{3}}}}}
% \end{CenterExample}
%
% \begin{CenterExample}
%   \psshadowbox[framesep=0.3,fillstyle=gradient,
%                gradbegin=Orange,gradend=yellow]{%
%     \setlength{\tabcolsep}{0mm}%
%     \begin{tabular}{ll}
%       \makebox[10cm]{%
%           \LARGE\textbf{Description of the problem\dotfill}} &
%         \raisebox{-5mm}{%
%           \PstFrameBoxThreeD[linewidth=0.15,
%                              FrameBoxThreeDColorHSB=0.9 0.5 0.5]{%
%             \LARGE\textcolor{white}{\textbf{25}}}}\\[-3mm]
%       \textcolor{red}{\large by Helen \textsc{Mitchell}
%       and Ann \textsc{Cartridge}}
%     \end{tabular}}
% \end{CenterExample}
%
% \begin{CenterExample}
%   \newcommand{\MyNode}[2]{%
%   \TR{\PstFrameBoxThreeD[linewidth=0.1,FrameBoxThreeDOn=#2,
%                          FrameBoxThreeDColorHSB=0.2 0.4 0.5,
%                          FrameBoxThreeDBrightnessDistance=0.2]{#1}}}
%   \renewcommand{\psedge}[2]{\ncangle{#2}{#1}}
%
%   \psset{angleB=-90,angleA=90,levelsep=2,treesep=3}
%   \pstree{\MyNode{Richard}{true}}
%          {\pstree{\MyNode{Paul}{true}}
%                  {\pstree{\MyNode{Jack}{true}}
%                          {\MyNode{Michel}{false}}
%                           \MyNode{Adam}{false}}
%                   \MyNode{Herbert}{false}}
% \end{CenterExample}
%
% \begin{CenterExample}
%   \newcommand{\EntryA}[1]{%
%   \PstFrameBoxThreeD[FrameBoxThreeDColorHSB=0.4 0.6 0.5]{%
%     \makebox[2.2cm]{#1}}}
%   \newcommand{\EntryB}[1]{%
%   \PstFrameBoxThreeD[FrameBoxThreeDColorHSB=0.16 1 0.8]{%
%     \makebox[0.5cm]{#1}}}
%   \newcommand{\EntryC}[1]{%
%   \PstFrameBoxThreeD[FrameBoxThreeDColorHSB=0.7 0.2 0.8]{%
%     \makebox[1.4cm]{#1}}}
%   \newcommand{\EntryD}[1]{%
%   \PstFrameBoxThreeD[FrameBoxThreeDColorHSB=0.9 0.5 0.5]{%
%     \makebox[1.4cm]{#1}}}
%
%   \begin{table}[htbp]
%     \centering
%     \large\bfseries
%     \setlength{\tabcolsep}{5mm}%
%     \setlength{\belowcaptionskip}{3mm}%
%     \caption{Results of the year}
%     \PstFrameBoxThreeD[framesep=0.4,doublesep=0.5,
%                        FrameBoxThreeDColorHSB=0.3 0.2 0.4]{%
%       \psset{framesep=0.3,doublesep=0.2}%
%       \begin{tabular}{lrr}
%         \\
%         \EntryA{1st Quarter}&\EntryB{723}&                  \\[1.5cm]
%         \EntryA{2nd Quarter}&\EntryB{819}&\EntryC{+13 \%}   \\[1.5cm]
%         \EntryA{3rd Quarter}&\EntryB{687}&\EntryD{--16.1 \%}\\[1.5cm]
%         \EntryA{4th Quarter}&\EntryB{894}&\EntryC{+30.1 \%} \\[1cm]
%       \end{tabular}}
%   \end{table}
% \end{CenterExample}
%
%
% \StopEventually{}
%
% ^^A .................... End of the documentation part ....................
%
% \section{Driver file}
%
%   The next bit of code contains the documentation driver file for \TeX{},
% i.e., the file that will produce the documentation you are currently
% reading. It will be extracted from this file by the \texttt{docstrip}
% program.
%
%    \begin{macrocode}
%<*driver>
\documentclass{ltxdoc}
\GetFileInfo{pst-fr3d.dtx}
\usepackage{fancyvrb}
\usepackage{multido}
\usepackage{pstricks}
\usepackage{pst-tree}
\usepackage{pst-grad}
\usepackage{pst-fr3d}
\EnableCrossrefs
\CodelineIndex
\RecordChanges
\OnlyDescription                % Comment it for implementation details
\hbadness=7000                  % Over and under full box warnings
\hfuzz=3pt
\definecolor{LemonChiffon}{rgb}{1.,0.98,0.8}
\definecolor{Orange}      {rgb}{1.,0.65,0.}
\definecolor{Pink}        {rgb}{1.,0.75,0.8}
\begin{document}
  \DocInput{pst-fr3d.dtx}
\end{document}
%</driver>
%    \end{macrocode}
%
% \section{\PstFrameBoxThreeDPackage{} \LaTeX{} wrapper}
%
%    \begin{macrocode}
%<*latex-wrapper>
\RequirePackage{pstricks}
\ProvidesPackage{pst-fr3d}[2005/10/10 package wrapper for 
  pst-fr3d.tex (hv)]
%%
%% This is file `pst-fr3d.tex',
%% generated with the docstrip utility.
%%
%% The original source files were:
%%
%% pst-fr3d.dtx  (with options: `pst-fr3d')
%% 
%% IMPORTANT NOTICE:
%% 
%% For the copyright see the source file.
%% 
%% Any modified versions of this file must be renamed
%% with new filenames distinct from pst-fr3d.tex.
%% 
%% For distribution of the original source see the terms
%% for copying and modification in the file pst-fr3d.dtx.
%% 
%% This generated file may be distributed as long as the
%% original source files, as listed above, are part of the
%% same distribution. (The sources need not necessarily be
%% in the same archive or directory.)
%%
%% Package `pst-fr3d.dtx'
%%
%% Denis Girou (CNRS/IDRIS - France) <Denis.Girou@idris.fr>
%%
%% February 25, 2002
%%
%% This program can be redistributed and/or modified under the terms
%% of the LaTeX Project Public License Distributed from CTAN archives
%% in directory macros/latex/base/lppl.txt.
%%
%% DESCRIPTION:
%%   `pst-fr3d' is a PSTricks package to draw three dimensional grids
%%   with various customizations.
%%
\csname PstFrameBoxThreeDLoaded\endcsname
\let\PstFrameBoxThreeDLoaded\endinput
\ifx\PSTricksLoaded\endinput\else\input pstricks.tex\fi
\ifx\PSTXKeyLoaded\endinput\else\input pst-xkey \fi
\edef\PstAtCode{\the\catcode`\@}
\catcode`\@=11\relax
\pst@addfams{pst-fr3d}
\def\fileversion{1.1}
\def\filedate{2005/10/10}
\message{`Pst-FrameBox3d' v\fileversion, \filedate\space (Denis Girou)}
\newif\ifPstFrameBoxThreeD@On
\define@key[psset]{pst-fr3d}{FrameBoxThreeDOn}[true]{\@nameuse{PstFrameBoxThreeD@On#1}}
\newif\ifPstFrameBoxThreeD@Opposite
\define@key[psset]{pst-fr3d}{FrameBoxThreeDOpposite}[true]{%
\@nameuse{PstFrameBoxThreeD@Opposite#1}}
\define@key[psset]{pst-fr3d}{FrameBoxThreeDColorHSB}{%
\PstFrameBoxThreeD@ColorHSB@i#1\@nil}
\def\PstFrameBoxThreeD@ColorHSB@i#1\@nil{%
\pst@expandafter\pst@getnumiii{#1} {} {} {} {}\@nil
\edef\PstFrameBoxThreeD@Hue{\pst@tempg}%
\edef\PstFrameBoxThreeD@Saturation{\pst@temph}%
\edef\PstFrameBoxThreeD@Brightness{\pst@tempi}}
\define@key[psset]{pst-fr3d}{FrameBoxThreeDBrightnessDistance}{%
\edef\PstFrameBoxThreeD@BrightnessDistance{#1}}
\psset{%
  FrameBoxThreeDOn=true,FrameBoxThreeDOpposite=false,
  FrameBoxThreeDColorHSB=0 0 0.5,FrameBoxThreeDBrightnessDistance=0.15}%
\def\PstFrameBoxThreeD{%
\@ifnextchar[{\PstFrameBoxThreeD@i}{\PstFrameBoxThreeD@i[]}}
\def\PstFrameBoxThreeD@i[#1]#2{{%
\psset{dimen=middle}%
\psset{#1}%
\ifdim\PstFrameBoxThreeD@Brightness\p@<\z@%
  \def\PstFrameBoxThreeD@FillStyle{none}%
  \def\pst@tempa{0.5}%
\else
  \def\PstFrameBoxThreeD@FillStyle{solid}%
  \def\pst@tempa{\PstFrameBoxThreeD@Brightness}%
\fi
\pst@dimh=\pst@tempa\p@
\advance\pst@dimh\PstFrameBoxThreeD@BrightnessDistance\p@
\pst@dimtonum{\pst@dimh}{\pst@tempb}%
\advance\pst@dimh-\PstFrameBoxThreeD@BrightnessDistance\p@
\advance\pst@dimh-\PstFrameBoxThreeD@BrightnessDistance\p@
\pst@dimtonum{\pst@dimh}{\pst@tempc}%
\ifx\definecolor\@undefined
  \newhsbcolor{PstFrameBoxThreeD@ColorC}{%
    \PstFrameBoxThreeD@Hue \PstFrameBoxThreeD@Saturation \pst@tempa}%
  \ifPstFrameBoxThreeD@On
    \newhsbcolor{PstFrameBoxThreeD@ColorA}{%
      \PstFrameBoxThreeD@Hue \PstFrameBoxThreeD@Saturation \pst@tempc}%
    \newhsbcolor{PstFrameBoxThreeD@ColorB}{%
      \PstFrameBoxThreeD@Hue \PstFrameBoxThreeD@Saturation \pst@tempb}%
  \else
    \newhsbcolor{PstFrameBoxThreeD@ColorA}{%
      \PstFrameBoxThreeD@Hue \PstFrameBoxThreeD@Saturation \pst@tempb}%
    \newhsbcolor{PstFrameBoxThreeD@ColorB}{%
      \PstFrameBoxThreeD@Hue \PstFrameBoxThreeD@Saturation \pst@tempc}%
  \fi
\else
  \definecolor{PstFrameBoxThreeD@ColorC}{hsb}{%
    \PstFrameBoxThreeD@Hue,\PstFrameBoxThreeD@Saturation,\pst@tempa}%
  \ifPstFrameBoxThreeD@On
    \definecolor{PstFrameBoxThreeD@ColorA}{hsb}{%
      \PstFrameBoxThreeD@Hue,\PstFrameBoxThreeD@Saturation,\pst@tempc}%
    \definecolor{PstFrameBoxThreeD@ColorB}{hsb}{%
      \PstFrameBoxThreeD@Hue,\PstFrameBoxThreeD@Saturation,\pst@tempb}%
  \else
    \definecolor{PstFrameBoxThreeD@ColorA}{hsb}{%
      \PstFrameBoxThreeD@Hue,\PstFrameBoxThreeD@Saturation,\pst@tempb}%
    \definecolor{PstFrameBoxThreeD@ColorB}{hsb}{%
      \PstFrameBoxThreeD@Hue,\PstFrameBoxThreeD@Saturation,\pst@tempc}%
  \fi
\fi
\psset{linestyle=none}%
\psframebox[fillstyle=\PstFrameBoxThreeD@FillStyle,
            fillcolor=PstFrameBoxThreeD@ColorC,doubleline=true]{#2}%
\setbox\pst@boxg=\hbox{\psframebox{#2}}%
\setbox\pst@hbox=\hbox{\psframebox[doubleline=true]{#2}}%
\pst@dima=\wd\pst@hbox
\advance\pst@dima-\wd\pst@boxg
\divide\pst@dima\tw@
\pst@dimb=\pst@dima
\advance\pst@dima\wd\pst@boxg
\psset{fillstyle=solid}%
\pspolygon[fillcolor=PstFrameBoxThreeD@ColorA]
          (-\wd\pst@hbox,-\dp\pst@hbox)(-\pst@dima,-\dp\pst@boxg)
          (-\pst@dima,\ht\pst@boxg)(-\wd\pst@hbox,\ht\pst@hbox)
\pspolygon[fillcolor=PstFrameBoxThreeD@ColorB]
          (-\wd\pst@hbox,\ht\pst@hbox)(-\pst@dima,\ht\pst@boxg)
          (-\pst@dimb,\ht\pst@boxg)(0,\ht\pst@hbox)
\ifPstFrameBoxThreeD@Opposite
  \def\pst@tempa{PstFrameBoxThreeD@ColorA}%
  \def\pst@tempb{PstFrameBoxThreeD@ColorB}%
\else
  \def\pst@tempa{PstFrameBoxThreeD@ColorB}%
  \def\pst@tempb{PstFrameBoxThreeD@ColorA}%
\fi
\pspolygon[fillcolor=\pst@tempa]
          (0,-\dp\pst@hbox)(-\pst@dimb,-\dp\pst@boxg)
          (-\pst@dimb,\ht\pst@boxg)(0,\ht\pst@hbox)
\pspolygon[fillcolor=\pst@tempb]
          (-\wd\pst@hbox,-\dp\pst@hbox)(-\pst@dima,-\dp\pst@boxg)
          (-\pst@dimb,-\dp\pst@boxg)(0,-\dp\pst@hbox)
\ifPstFrameBoxThreeD@Opposite
\else
  \psset{linestyle=solid,linewidth=0.2pt,arrows=cc-cc}%
  \psline[linecolor=PstFrameBoxThreeD@ColorA]
         (-\pst@dimb,\ht\pst@boxg)(0,\ht\pst@hbox)
  \psline[linecolor=PstFrameBoxThreeD@ColorB]
         (-\wd\pst@hbox,-\dp\pst@hbox)(-\pst@dima,-\dp\pst@boxg)
\fi}}
\catcode`\@=\PstAtCode\relax
\endinput
%%
%% End of file `pst-fr3d.tex'.

\ProvidesFile{pst-fr3d.tex}
  [\filedate\space v\fileversion\space `PST-fr3d' (dg)]
%</latex-wrapper>
%    \end{macrocode}
%
% \section{\PstFrameBoxThreeDPackage{} code}
%
%    \begin{macrocode}
%<*pst-fr3d>
%    \end{macrocode}
%
% \subsection{Preambule}
%
%   Who we are.
%
%    \begin{macrocode}
\csname PstFrameBoxThreeDLoaded\endcsname
\let\PstFrameBoxThreeDLoaded\endinput
%    \end{macrocode}
%
%   Require the PSTricks package.
%
%    \begin{macrocode}
\ifx\PSTricksLoaded\endinput\else\input pstricks.tex\fi
%    \end{macrocode}
%
%   David \textsc{Carlisle} interface to the `\textsf{keyval}' package.
%
%    \begin{macrocode}
\ifx\PSTXKeyLoaded\endinput\else\input pst-xkey \fi
%    \end{macrocode}
%
%   Catcodes changes.
%
%    \begin{macrocode}
\edef\PstAtCode{\the\catcode`\@}
\catcode`\@=11\relax
\pst@addfams{pst-fr3d}
\def\fileversion{1.1}
\def\filedate{2005/10/10}
\message{`Pst-FrameBox3d' v\fileversion, \filedate\space (Denis Girou)}
%    \end{macrocode}
%
% \subsection{Definition of the parameters}
%
%   \texttt{FrameBoxThreeDOn} to draw the 3d framed box with an ON / OFF
% effect.
%
%    \begin{macrocode}
\newif\ifPstFrameBoxThreeD@On
\define@key[psset]{pst-fr3d}{FrameBoxThreeDOn}[true]{\@nameuse{PstFrameBoxThreeD@On#1}}
%    \end{macrocode}
%
%   \texttt{FrameBoxThreeDOpposite} to use the same colors for the opposite
% sides or the contiguous ones.
%
%    \begin{macrocode}
\newif\ifPstFrameBoxThreeD@Opposite
\define@key[psset]{pst-fr3d}{FrameBoxThreeDOpposite}[true]{%
\@nameuse{PstFrameBoxThreeD@Opposite#1}}
%    \end{macrocode}
%
%   \texttt{FrameBoxThreeDColorHSB} to define the colors used, in the HSB
% (\emph{hue}, \emph{saturation}, \emph{brightness}) model. This is a sequence
% of three real values, separated by spaces. This will define the color of the
% interior of the frame. And for the borders we will use it too, changing it
% brightness in light and dark.
%
%    \begin{macrocode}
\define@key[psset]{pst-fr3d}{FrameBoxThreeDColorHSB}{%
\PstFrameBoxThreeD@ColorHSB@i#1\@nil}
\def\PstFrameBoxThreeD@ColorHSB@i#1\@nil{%
\pst@expandafter\pst@getnumiii{#1} {} {} {} {}\@nil
\edef\PstFrameBoxThreeD@Hue{\pst@tempg}%
\edef\PstFrameBoxThreeD@Saturation{\pst@temph}%
\edef\PstFrameBoxThreeD@Brightness{\pst@tempi}}
%    \end{macrocode}
%
%   \texttt{FrameBoxThreeDBrightnessDistance} to define the brightness
% distance between the brightness used for the material of the box
% (\texttt{BrightnessC}) and the two brightnesss used for the borders
% (\texttt{BrightnessA} and \texttt{BrightnessB}): \texttt{BrightnessA} =
% \texttt{BrightnessC} + \texttt{FrameBoxThreeDBrightnessDistance} and
% \texttt{BrightnessB} = \texttt{BrightnessC} --
% \texttt{FrameBoxThreeDBrightnessDistance}. It can be a negative number. And
% if the brightness is negative, we will use a value of 0.5 to compute the two
% other colors.
%
%    \begin{macrocode}
\define@key[psset]{pst-fr3d}{FrameBoxThreeDBrightnessDistance}{%
\edef\PstFrameBoxThreeD@BrightnessDistance{#1}}
%    \end{macrocode}
%
%   Next, we set the default values for all these new parameters.
%
%    \begin{macrocode}
\psset{%
  FrameBoxThreeDOn=true,FrameBoxThreeDOpposite=false,
  FrameBoxThreeDColorHSB=0 0 0.5,FrameBoxThreeDBrightnessDistance=0.15}%
%    \end{macrocode}
%
% \subsection{Main macro}
%
%   The general \cs{PstFrameBoxThreeD} macro to draw three dimensional boxed
% frames.
%
% \begin{macro}{\PstFrameBoxThreeD}
%    \begin{macrocode}
\def\PstFrameBoxThreeD{%
\@ifnextchar[{\PstFrameBoxThreeD@i}{\PstFrameBoxThreeD@i[]}}
%    \end{macrocode}
% \end{macro}
%
% \begin{macro}{\PstFrameBoxThreeD@i}
%    \begin{macrocode}
\def\PstFrameBoxThreeD@i[#1]#2{{%
%    \end{macrocode}
%
%   We force ``dimmen=middle''.
%
%    \begin{macrocode}
\psset{dimen=middle}%
%    \end{macrocode}
%
%   Then we set the local changes of parameters.
%
%    \begin{macrocode}
\psset{#1}%
%    \end{macrocode}
%
%
%   If the brightness is negative, the interior of the frame will not be
% filled with color, and a value of 0.5 will be used to compute the two colors
% of the borders.
%
%    \begin{macrocode}
\ifdim\PstFrameBoxThreeD@Brightness\p@<\z@%
  \def\PstFrameBoxThreeD@FillStyle{none}%
  \def\pst@tempa{0.5}%
\else
  \def\PstFrameBoxThreeD@FillStyle{solid}%
  \def\pst@tempa{\PstFrameBoxThreeD@Brightness}%
\fi
%    \end{macrocode}
%
%   Computation of the colors of the borders.
%
%    \begin{macrocode}
\pst@dimh=\pst@tempa\p@
\advance\pst@dimh\PstFrameBoxThreeD@BrightnessDistance\p@
\pst@dimtonum{\pst@dimh}{\pst@tempb}%
\advance\pst@dimh-\PstFrameBoxThreeD@BrightnessDistance\p@
\advance\pst@dimh-\PstFrameBoxThreeD@BrightnessDistance\p@
\pst@dimtonum{\pst@dimh}{\pst@tempc}%
%    \end{macrocode}
%
%   If the \ColorPackage{} \LaTeX{} package was not loaded (using the
% \PstColPackage{} compatibility package which is hightly recommanded for
% \LaTeX{} users), we will use the PSTricks color model.
%
%    \begin{macrocode}
\ifx\definecolor\@undefined
  \newhsbcolor{PstFrameBoxThreeD@ColorC}{%
    \PstFrameBoxThreeD@Hue \PstFrameBoxThreeD@Saturation \pst@tempa}%
  \ifPstFrameBoxThreeD@On
    \newhsbcolor{PstFrameBoxThreeD@ColorA}{%
      \PstFrameBoxThreeD@Hue \PstFrameBoxThreeD@Saturation \pst@tempc}%
    \newhsbcolor{PstFrameBoxThreeD@ColorB}{%
      \PstFrameBoxThreeD@Hue \PstFrameBoxThreeD@Saturation \pst@tempb}%
  \else
    \newhsbcolor{PstFrameBoxThreeD@ColorA}{%
      \PstFrameBoxThreeD@Hue \PstFrameBoxThreeD@Saturation \pst@tempb}%
    \newhsbcolor{PstFrameBoxThreeD@ColorB}{%
      \PstFrameBoxThreeD@Hue \PstFrameBoxThreeD@Saturation \pst@tempc}%
  \fi
%    \end{macrocode}
%
%   If the \ColorPackage{} \LaTeX{} package was loaded, we will use it as
% color model.
%
%    \begin{macrocode}
\else
  \definecolor{PstFrameBoxThreeD@ColorC}{hsb}{%
    \PstFrameBoxThreeD@Hue,\PstFrameBoxThreeD@Saturation,\pst@tempa}%
  \ifPstFrameBoxThreeD@On
    \definecolor{PstFrameBoxThreeD@ColorA}{hsb}{%
      \PstFrameBoxThreeD@Hue,\PstFrameBoxThreeD@Saturation,\pst@tempc}%
    \definecolor{PstFrameBoxThreeD@ColorB}{hsb}{%
      \PstFrameBoxThreeD@Hue,\PstFrameBoxThreeD@Saturation,\pst@tempb}%
  \else
    \definecolor{PstFrameBoxThreeD@ColorA}{hsb}{%
      \PstFrameBoxThreeD@Hue,\PstFrameBoxThreeD@Saturation,\pst@tempb}%
    \definecolor{PstFrameBoxThreeD@ColorB}{hsb}{%
      \PstFrameBoxThreeD@Hue,\PstFrameBoxThreeD@Saturation,\pst@tempc}%
  \fi
\fi
%    \end{macrocode}
%
%   Now, we can \emph{really} draw the box and the borders.
%
%   No lines here.
%
%    \begin{macrocode}
\psset{linestyle=none}%
%    \end{macrocode}
%
%   We draw the material in a standard \cs{psframebox} macro, defining
% \texttt{doubleline} to \texttt{true}, to compute correctly the bounding box
% of the box with it borders.
%
%    \begin{macrocode}
\psframebox[fillstyle=\PstFrameBoxThreeD@FillStyle,
            fillcolor=PstFrameBoxThreeD@ColorC,doubleline=true]{#2}%
%    \end{macrocode}
%
%   We save the material to frame in an internal box, without the size of it
% borders.
%
%    \begin{macrocode}
\setbox\pst@boxg=\hbox{\psframebox{#2}}%
%    \end{macrocode}
%
%   Then we save in another internal box the same thing, this time with the
% size of it borders.
%
%    \begin{macrocode}
\setbox\pst@hbox=\hbox{\psframebox[doubleline=true]{#2}}%
%    \end{macrocode}
%
%   We compute the four internal coordinates for the borders.
%
%    \begin{macrocode}
\pst@dima=\wd\pst@hbox
\advance\pst@dima-\wd\pst@boxg
\divide\pst@dima\tw@
%    \end{macrocode}
%
%    \begin{macrocode}
\pst@dimb=\pst@dima
%    \end{macrocode}
%
%    \begin{macrocode}
\advance\pst@dima\wd\pst@boxg
%    \end{macrocode}
%
%   Now, we can draw the borders, of course in a \texttt{solid} mode.
%
%    \begin{macrocode}
\psset{fillstyle=solid}%
%    \end{macrocode}
%
%   We draw first the left then the top borders.
%
%    \begin{macrocode}
\pspolygon[fillcolor=PstFrameBoxThreeD@ColorA]
          (-\wd\pst@hbox,-\dp\pst@hbox)(-\pst@dima,-\dp\pst@boxg)
          (-\pst@dima,\ht\pst@boxg)(-\wd\pst@hbox,\ht\pst@hbox)
\pspolygon[fillcolor=PstFrameBoxThreeD@ColorB]
          (-\wd\pst@hbox,\ht\pst@hbox)(-\pst@dima,\ht\pst@boxg)
          (-\pst@dimb,\ht\pst@boxg)(0,\ht\pst@hbox)
%    \end{macrocode}
%
%   We must compute the colors of the two other borders.
%
%    \begin{macrocode}
\ifPstFrameBoxThreeD@Opposite
  \def\pst@tempa{PstFrameBoxThreeD@ColorA}%
  \def\pst@tempb{PstFrameBoxThreeD@ColorB}%
\else
  \def\pst@tempa{PstFrameBoxThreeD@ColorB}%
  \def\pst@tempb{PstFrameBoxThreeD@ColorA}%
\fi
%    \end{macrocode}
%
%   Now, we draw the right then the bottom borders.
%
%    \begin{macrocode}
\pspolygon[fillcolor=\pst@tempa]
          (0,-\dp\pst@hbox)(-\pst@dimb,-\dp\pst@boxg)
          (-\pst@dimb,\ht\pst@boxg)(0,\ht\pst@hbox)
\pspolygon[fillcolor=\pst@tempb]
          (-\wd\pst@hbox,-\dp\pst@hbox)(-\pst@dima,-\dp\pst@boxg)
          (-\pst@dimb,-\dp\pst@boxg)(0,-\dp\pst@hbox)
%    \end{macrocode}
%
%   And, to finish, a dark oblique line in the middle of the light border and
% a light oblique line in the middle of the dark border, if the identical
% colors are contiguous.
%
%    \begin{macrocode}
\ifPstFrameBoxThreeD@Opposite
\else
  \psset{linestyle=solid,linewidth=0.2pt,arrows=cc-cc}%
  \psline[linecolor=PstFrameBoxThreeD@ColorA]
         (-\pst@dimb,\ht\pst@boxg)(0,\ht\pst@hbox)
  \psline[linecolor=PstFrameBoxThreeD@ColorB]
         (-\wd\pst@hbox,-\dp\pst@hbox)(-\pst@dima,-\dp\pst@boxg)
\fi}}
%    \end{macrocode}
% \end{macro}
%
% \subsection{Closing}
%
%   Catcodes restoration.
%
%    \begin{macrocode}
\catcode`\@=\PstAtCode\relax
%    \end{macrocode}
%
%    \begin{macrocode}
%</pst-fr3d>
%    \end{macrocode}
%
% \Finale
% \PrintIndex
% \PrintChanges
%
\endinput
%%
%% End of file `pst-fr3d.dtx'
