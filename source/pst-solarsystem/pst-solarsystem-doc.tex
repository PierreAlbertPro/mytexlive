%% $Id: pst-solarsystem-doc.tex 620 2012-01-01 14:09:57Z herbert $
\documentclass[11pt,english,BCOR10mm,DIV12,bibliography=totoc,parskip=false,
   smallheadings, headexclude,footexclude,oneside]{pst-doc}
\usepackage[latin1]{inputenc}
\usepackage{pst-solarsystem}
\let\pstFV\fileversion
\usepackage{pstricks-add}
\renewcommand\bgImage{\resizebox{0.6\linewidth}{!}{\SolarSystem[Day=1,Month=1,Year=2012,Hour=12,viewpoint=1 -1 2]}}

\lstset{language=PSTricks,
    morekeywords={psGammaDist,psChiIIDist,psTDist,psFDist,psBetaDist,psPlotImpl},basicstyle=\footnotesize\ttfamily}
%
\begin{document}

\title{\texttt{pst-solarsystem}}
\subtitle{Position of the visible planets, projected on the plane of the ecliptic; v.\pstFV}
\author{Manuel Luque \\Herbert Vo\ss}
\docauthor{}
\date{\today}
\maketitle

\tableofcontents

\vspace{2cm}
For the method of calculation, I was guided by:
\begin{itemize}
\item that given by \textit{Jean Meeus} astronomical calculations in the book for use by
published by the Amateur Astronomical Society of France.
\item and that of Guy Serane in \textit{Astronomy \& PC published} by Wiley \& Sons.
\end{itemize}

As we can not represent all the planets in the real proportions, only Mercury, Venus, Earth and Mars are the proportions of the orbits and
their relative sizes observed. Saturn and Jupiter are in the right direction, but
obviously not at the right distance.

The orbits are shown in solid lines for the portion above the ecliptic
and dashed for the portion located below.

We can compare the view obtained with the following representation:

\url{http://users.skynet.be/fa274406/rubriques/live/orbites/orbites.htm}

The use of the command is very simple, just specify the date of observation
with the following parameters, for example:
\begin{verbatim}
\SolarSystem[Day=31,Month=06,Year=2001,Hour=23,Minute=59,Second=59]
\end{verbatim}
\xLkeyword{Day}\xLkeyword{Month}\xLkeyword{Year}\xLkeyword{Hour}\xLkeyword{Minute}\xLkeyword{Second}

By default, if no parameter is specified, \Lcs{SolarSystem} gives the configuration
day 0 hours to compile.

The \Lkeyword{values} is enabled by default. It displays the values of
longitude, latitude, and the distance in astronomical units.

The accuracy of the calculations is about 0.1 to 0.3 degrees (comparing to ephemeris
the Bureau des Longitudes), which is more than enough for a performance
graph.

\url{http://www.imcce.fr/fr/ephemerides/formulaire/form_ephepos.php}
\begin{center}
\SolarSystem

\SolarSystem[Day=30,Month=06,Year=2001,Hour=23,Minute=59,Second=59,viewpoint=1 -1 2,values=false]
\end{center}


\begin{verbatim}
\SolarSystem[Day=30,Month=06,Year=2001,
             Hour=23,Minute=59,Second=59,
             viewpoint=1 -1 2,values=false]
\end{verbatim}


\clearpage
\section{List of all optional arguments for \texttt{pst-solarsystem}}

\xkvview{family=pst-solarsystem,columns={key,type,default}}




\bgroup
\raggedright
\nocite{*}
\bibliographystyle{plain}
\bibliography{pst-solarsystem-doc}
\egroup

\printindex



\end{document} 
