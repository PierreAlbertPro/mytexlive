%% $Id: pst-magneticfield-docEN.tex 347 2010-06-12 06:33:02Z herbert $
\documentclass[11pt,english,BCOR10mm,DIV12,bibliography=totoc,parskip=false,smallheadings
    headexclude,footexclude,oneside]{pst-doc}
\usepackage[latin1]{inputenc}
\usepackage{pst-magneticfield}
\let\pstMFfv\fileversion
\lstset{pos=t,language=PSTricks,
    morekeywords={psmagneticfield,psmagneticfieldThreeD},basicstyle=\footnotesize\ttfamily}

%\newenvironment{postscript}{}{} % uncomment, when running with latex

\newcommand\Cadre[1]{\psframebox[fillstyle=solid,fillcolor=black,linestyle=none,framesep=0]{#1}}
\def\bgImage{}
%
\begin{document}

\title{\texttt{pst-magneticfield}}
\subtitle{Magnetic field lines of a solenoid; v.\pstMFfv}
\author{Juergen Gilg\\ Manuel Luque\\Herbert Vo\ss}
\date{\today}
\maketitle


\clearpage% 
\begin{abstract}
The package \LPack{pst-magneticfield} aims to trace the shape of field lines
of a solenoid. The physical parameters are the radius of the solenoid, the number of
turns and its length, the default values are given below:

\begin{enumerate}
  \item the number of turns: \LKeyset{N=6} ;
  \item the radius : \LKeyset{R=2} ;
  \item the length : \LKeyset{L=4}.
\end{enumerate}

The field lines were calculated with the Runge-Kutta 2 algorithm, which, after several tries, 
seemed to be the best compromise between speed and accuracy of calculations for the path.
The calculation of elliptic integrals for the evaluation of the magnetic field
was achieved by polynomial approximations from the "Handbook of Mathematical
Functions With Formulas, Graph, And Mathematical Tables" by Milton Abramowitz and
Irene.\,A. Stegun (\url{http://www.math.sfu.ca/~cbm/aands/}).~\cite{abramowitz}
\end{abstract}

\clearpage
\tableofcontents


\clearpage

\section{Introduction}

The route options, with the default values are as follows:
\begin{enumerate}
  \item The maximum number of points on each line of the entire coil: \LKeyset{pointsB=500};
  \item the maximum number of points on lines around turns selected: \LKeyset{pointsS=1000};
  \item the number of lines of the entire coil: \LKeyset{nL=8};
  \item differential steps for the lines of the entire coil: \LKeyset{PasB=0.02};
  \item differential steps for the lines around turns selected: \LKeyset{PasS=0.00275};
  \item the choice of individual coils to improve the rendering of its
      layout: \LKeyset{numSpires=\{\}}, we place following the sign "=" the numbers of turns \textsf{1 2 3 etc.} 
      starting from the top spire. By default, all the turns are targeted.
  \item The number of field lines around the turns selected: \LKeyset{nS=1}.
  \item We may decide not to represent the solenoid with the option \LKeyset{drawSelf=false}
      is useful for 3D representation.
  \item The route options for the turns (color, thickness, arrows) are:
  \begin{enumerate}
        \item The color and thickness of the coils: \Lkeyset{styleSpire=styleSpire};
        \item the current direction signs: \Lkeyset{styleCourant=sensCourant}.
  \end{enumerate}
\begin{verbatim}
\newpsstyle{styleSpire}{linecap=1,linecolor=red,linewidth=2\pslinewidth}
\newpsstyle{sensCourant}{linecolor=red,linewidth=2\pslinewidth,arrowinset=0.1}
\end{verbatim}
      
 \item The color and thickness of the field lines can be adjusted with the 
      usual \LPack{pstricks} parameters: \Lkeyword{linecolor} and  \Lkeyword{linewidth}
\end{enumerate}

A command \Lcs{psmagneticfieldThreeD} allows 3D visualization of the solenoid and
field lines. 

\begin{BDef}
\Lcs{psmagneticfield}\OptArgs\OptArg*{\coord1}\OptArg*{\coord2}\\
\Lcs{psmagneticfieldThreeD}\OptArgs\OptArg*{\coord1}\OptArg*{\coord2}
\end{BDef}

Missing coordinates are substituted to \verb+(-6,-5)(6,5)+!
\clearpage
\section{Influence of physical parameters on the map magnetic field}

\subsection{The length of the solenoid}

\begin{center}
\begin{postscript}
\psset{unit=0.5cm}
\begin{pspicture*}[showgrid](-7,-8)(7,8)
\psmagneticfield[linecolor={[HTML]{006633}},N=3,R=2,nS=1](-7,-8)(7,8)
\psframe*[linecolor={[HTML]{99FF66}}](-7,-8)(7,-7)
\rput(0,-7.5){[\Cadre{\textcolor{white}{L=4}},N=3,R=2,nS=1]}
\end{pspicture*}
\begin{pspicture*}[showgrid](-7,-8)(7,8)
\psmagneticfield[linecolor={[HTML]{006633}},L=8,N=3,R=2,nS=1,PasB=0.0025,pointsB=5500](-7,-8)(7,8)
\psframe*[linecolor={[HTML]{99FF66}}](-7,-8)(7,-7)
\rput(0,-7.5){[\Cadre{\textcolor{white}{L=8}},N=3,R=2,nS=1]}
\end{pspicture*}
\end{postscript}
\end{center}

\begin{lstlisting}
\psset{unit=0.5cm}
\begin{pspicture*}[showgrid](-7,-8)(7,8)
\psmagneticfield[linecolor={[HTML]{006633}},N=3,R=2,nS=1](-7,-8)(7,8)
\psframe*[linecolor={[HTML]{99FF66}}](-7,-8)(7,-7)
\rput(0,-7.5){[\Cadre{\textcolor{white}{L=4}},N=3,R=2,nS=1]}
\end{pspicture*}
\begin{pspicture*}[showgrid](-7,-8)(7,8)
\psmagneticfield[linecolor={[HTML]{006633}},L=8,N=3,R=2,nS=1,PasB=0.0025,pointsB=5500](-7,-8)(7,8)
\psframe*[linecolor={[HTML]{99FF66}}](-7,-8)(7,-7)
\rput(0,-7.5){[\Cadre{\textcolor{white}{L=8}},N=3,R=2,nS=1]}
\end{pspicture*}
\end{lstlisting}



\textbf{Note:} To refine the layout of the second solenoid, we had to increase the
points and lower the pitch of the route: 
\begin{postscript}
\Cadre{\textcolor{white}{pointsB=5500,PasB=0.0025}}
\end{postscript}, which
takes more time for the calculations.



\clearpage

\subsection{The number of turns}
\begin{center}
\begin{postscript}
\psset{unit=0.5}
\begin{pspicture*}[showgrid](-7,-8)(7,8)
\psmagneticfield[linecolor={[HTML]{006633}},N=1,R=2,nS=0](-7,-8)(7,8)
\psframe*[linecolor={[HTML]{99FF66}}](-7,-8)(7,-7)
\rput(0,-7.5){[\Cadre{\textcolor{white}{N=1}},R=2,nS=0]}
\end{pspicture*}
\begin{pspicture*}[showgrid](-7,-8)(7,8)
\psmagneticfield[linecolor={[HTML]{006633}},N=2,R=2,L=2,PasS=0.003,nS=2](-7,-8)(7,8)
\psframe*[linecolor={[HTML]{99FF66}}](-7,7)(7,8)
\rput(0,7.5){\Cadre{\textcolor{white}{Bobines de Helmholtz}}}
\psframe*[linecolor={[HTML]{99FF66}}](-7,-8)(7,-7)
\rput(0,-7.5){[\Cadre{\textcolor{white}{N=2}},R=2,L=2,PasS=0.003,nS=2]}
\end{pspicture*}
\end{postscript}
\end{center}

\begin{lstlisting}
\psset{unit=0.5}
\begin{pspicture*}[showgrid](-7,-8)(7,8)
\psmagneticfield[linecolor={[HTML]{006633}},N=1,R=2,nS=0](-7,-8)(7,8)
\psframe*[linecolor={[HTML]{99FF66}}](-7,-8)(7,-7)
\rput(0,-7.5){[\Cadre{\textcolor{white}{N=1}},R=2,nS=0]}
\end{pspicture*}
\begin{pspicture*}[showgrid](-7,-8)(7,8)
\psmagneticfield[linecolor={[HTML]{006633}},N=2,R=2,L=2,PasS=0.003,nS=2](-7,-8)(7,8)
\psframe*[linecolor={[HTML]{99FF66}}](-7,7)(7,8)
\rput(0,7.5){\Cadre{\textcolor{white}{Bobines de Helmholtz}}}
\psframe*[linecolor={[HTML]{99FF66}}](-7,-8)(7,-7)
\rput(0,-7.5){[\Cadre{\textcolor{white}{N=2}},R=2,L=2,PasS=0.003,nS=2]}
\end{pspicture*}
\end{lstlisting}


\begin{center}
\begin{postscript}
\psset{unit=0.5}
\begin{pspicture*}[showgrid](-7,-8)(7,8)
\psmagneticfield[linecolor={[HTML]{006633}},N=4,R=2,numSpires=2 3](-7,-8)(7,8)
\psframe*[linecolor={[HTML]{99FF66}}](-7,-8)(7,-7)
\rput(0,-7.5){[\Cadre{\textcolor{white}{N=4}},R=2,L=4]}
\end{pspicture*}
\begin{pspicture*}[showgrid](-7,-8)(7,8)
\psmagneticfield[linecolor={[HTML]{006633}},N=5,R=2,L=5,PasS=0.004,numSpires=2 3 4](-7,-8)(7,8)
\psframe*[linecolor={[HTML]{99FF66}}](-7,-8)(7,-7)
\rput(0,-7.5){[\Cadre{\textcolor{white}{N=5}},R=2,L=5]}
\end{pspicture*}
\end{postscript}
\end{center}

\begin{lstlisting}
\psset{unit=0.5}
\begin{pspicture*}[showgrid](-7,-8)(7,8)
\psmagneticfield[linecolor={[HTML]{006633}},N=4,R=2,numSpires=2 3](-7,-8)(7,8)
\psframe*[linecolor={[HTML]{99FF66}}](-7,-8)(7,-7)
\rput(0,-7.5){[\Cadre{\textcolor{white}{N=4}},R=2,L=4]}
\end{pspicture*}
\begin{pspicture*}[showgrid](-7,-8)(7,8)
\psmagneticfield[linecolor={[HTML]{006633}},N=5,R=2,L=5,PasS=0.004,numSpires=2 3 4](-7,-8)(7,8)
\psframe*[linecolor={[HTML]{99FF66}}](-7,-8)(7,-7)
\rput(0,-7.5){[\Cadre{\textcolor{white}{N=5}},R=2,L=5]}
\end{pspicture*}
\end{lstlisting}


\clearpage
\section{The three route options}
\subsection{The number of field lines}

Due to the symmetry of the problem the number of field lines given
(\Lkeyword{nL}) option is half the number actually represented with an added line
identic to the the axis of revolution. We must also add the lines around the turns \Lkeyword{nS},
these turns can be selected individually \Lkeyword{numSpires}.



\begin{center}
\begin{postscript}
\psset{unit=0.5}
\begin{pspicture*}[showgrid](-7,-8)(7,8)
\psmagneticfield[linecolor={[HTML]{000099}},N=1,R=2](-7,-8)(7,8)
\psframe*[linecolor={[HTML]{3399FF}}](-7,-8)(7,-7)
\rput(0,-7.5){[\Cadre{\textcolor{white}{nL=8}},N=1,R=2]}
\end{pspicture*}
\begin{pspicture*}[showgrid](-7,-8)(7,8)
\psmagneticfield[linecolor={[HTML]{000099}},N=1,R=2,nL=12](-7,-8)(7,8)
\psframe*[linecolor={[HTML]{3399FF}}](-7,-8)(7,-7)
\rput(0,-7.5){[\Cadre{\textcolor{white}{nL=12}},N=1,R=2]}
\end{pspicture*}
\end{postscript}
\end{center}

\begin{lstlisting}
\psset{unit=0.5}
\begin{pspicture*}[showgrid](-7,-8)(7,8)
\psmagneticfield[linecolor={[HTML]{000099}},N=1,R=2](-7,-8)(7,8)
\psframe*[linecolor={[HTML]{3399FF}}](-7,-8)(7,-7)
\rput(0,-7.5){[\Cadre{\textcolor{white}{nL=8}},N=1,R=2]}
\end{pspicture*}
\begin{pspicture*}[showgrid](-7,-8)(7,8)
\psmagneticfield[linecolor={[HTML]{000099}},N=1,R=2,nL=12](-7,-8)(7,8)
\psframe*[linecolor={[HTML]{3399FF}}](-7,-8)(7,-7)
\rput(0,-7.5){[\Cadre{\textcolor{white}{nL=12}},N=1,R=2]}
\end{pspicture*}
\end{lstlisting}

\clearpage
\subsection{The number of points for the path}
  The plot of field lines is achieved by a numerical method (RK2) and
follows the step of the route and the number of selected points affect the accuracy of the route,
as in the two examples below:


\begin{center}
\begin{postscript}
\psset{unit=0.5}
\begin{pspicture*}[showgrid](-7,-8)(7,8)
\psmagneticfield[linecolor={[HTML]{660066}},N=2,R=2,L=2,PasB=0.1,nS=0,nL=7,pointsB=100](-7,-8)(7,8)
\psframe*[linecolor={[HTML]{996666}}](-7,7)(7,8)
\rput(0,7.5){\Cadre{\textcolor{white}{Bobines de Helmholtz}}}
\psframe*[linecolor={[HTML]{996666}}](-7,-8)(7,-7)
\rput(0,-7.5){[\Cadre{\textcolor{white}{PasB=0.1,nL=4,pointsB=100}}]}
\end{pspicture*}
\begin{pspicture*}[showgrid](-7,-8)(7,8)
\psmagneticfield[linecolor={[HTML]{660066}},N=2,R=2,L=2,PasB=0.4,nS=0,nL=7,pointsB=100](-7,-8)(7,8)
\psframe*[linecolor={[HTML]{996666}}](-7,7)(7,8)
\rput(0,7.5){\Cadre{\textcolor{white}{Bobines de Helmholtz}}}
\psframe*[linecolor={[HTML]{996666}}](-7,-8)(7,-7)
\rput(0,-7.5){[\Cadre{\textcolor{white}{PasS=0.4,pointsB=100}}]}
\end{pspicture*}
\end{postscript}
\end{center}

\begin{lstlisting}
\psset{unit=0.5}
\begin{pspicture*}[showgrid](-7,-8)(7,8)
\psmagneticfield[linecolor={[HTML]{660066}},N=2,R=2,L=2,PasB=0.1,nS=0,nL=7,pointsB=100](-7,-8)(7,8)
\psframe*[linecolor={[HTML]{996666}}](-7,7)(7,8)
\rput(0,7.5){\Cadre{\textcolor{white}{Bobines de Helmholtz}}}
\psframe*[linecolor={[HTML]{996666}}](-7,-8)(7,-7)
\rput(0,-7.5){[\Cadre{\textcolor{white}{PasB=0.1,nL=4,pointsB=100}}]}
\end{pspicture*}
\begin{pspicture*}[showgrid](-7,-8)(7,8)
\psmagneticfield[linecolor={[HTML]{660066}},N=2,R=2,L=2,PasB=0.4,nS=0,nL=7,pointsB=100](-7,-8)(7,8)
\psframe*[linecolor={[HTML]{996666}}](-7,7)(7,8)
\rput(0,7.5){\Cadre{\textcolor{white}{Bobines de Helmholtz}}}
\psframe*[linecolor={[HTML]{996666}}](-7,-8)(7,-7)
\rput(0,-7.5){[\Cadre{\textcolor{white}{PasS=0.4,pointsB=100}}]}
\end{pspicture*}
\end{lstlisting}


If the defaults do not suit it must be found by testing the
values that give a correct path.



\clearpage

\section{The parameter \nxLkeyword{numSpires}}
\begin{center}
\begin{postscript}
\psset{unit=0.5}
\begin{pspicture*}[showgrid](-8,-10)(8,10)
\psset{linecolor=blue}
\psmagneticfield[R=2,L=12,N=8,pointsS=500,nL=14,nS=1,numSpires=1 3 6 8,PasB=0.075](-8,-10)(8,10)
\psframe*[linecolor={[HTML]{99FF66}}](-8,-10)(8,-9)
\rput(0,-9.5){[\Cadre{\textcolor{white}{numSpires=1 3 6 8}},R=2,L=14]}
\multido{\i=0+1}{8}{\rput[l](!6 6 12 7 div \i\space mul sub){\the\multidocount}}
\end{pspicture*}\quad
\begin{pspicture*}[showgrid](0,-10)(16,10)
\psset{linecolor=blue}
\psmagneticfield[R=2,L=12,N=8,pointsS=500,nL=14,numSpires=,nS=1,PasB=0.075](0,-10)(16,10)
\psframe*[linecolor={[HTML]{99FF66}}](0,-10)(16,-9)
\rput(8,-9.5){[\Cadre{\textcolor{white}{numSpires=all}},R=2,L=14]}
\multido{\i=0+1}{8}{\rput[l](!6 6 12 7 div \i\space mul sub){\the\multidocount}}
\end{pspicture*}
\end{postscript}
\end{center}


\begin{lstlisting}
\psset{unit=0.5}
\begin{pspicture*}[showgrid](-8,-10)(8,10)
\psset{linecolor=blue}
\psmagneticfield[R=2,L=12,N=8,pointsS=500,nL=14,nS=1,numSpires=1 3 6 8,PasB=0.075](-8,-10)(8,10)
\psframe*[linecolor={[HTML]{99FF66}}](-8,-10)(8,-9)
\rput(0,-9.5){[\Cadre{\textcolor{white}{numSpires=1 3 6 8}},R=2,L=14]}
\multido{\i=0+1}{8}{\rput[l](!6 6 12 7 div \i\space mul sub){\the\multidocount}}
\end{pspicture*}\quad
\begin{pspicture*}[showgrid](0,-10)(16,10)
\psset{linecolor=blue}
\psmagneticfield[R=2,L=12,N=8,pointsS=500,nL=14,numSpires=,nS=1,PasB=0.075](0,-10)(16,10)
\psframe*[linecolor={[HTML]{99FF66}}](0,-10)(16,-9)
\rput(8,-9.5){[\Cadre{\textcolor{white}{numSpires=all}},R=2,L=14]}
\multido{\i=0+1}{8}{\rput[l](!6 6 12 7 div \i\space mul sub){\the\multidocount}}
\end{pspicture*}
\end{lstlisting}

\clearpage
\section{The parameter \nxLkeyword{AntiHelmholtz}}
\begin{center}
\begin{postscript}
\psset{unit=0.75,AntiHelmholtz,N=2,
  R=2,pointsB=500,pointsS=1000,PasB=0.02,PasS=0.00275,nS=10,
  nL=2,drawSelf=true,styleSpire=styleSpire,styleCourant=sensCourant}
\newpsstyle{grille}{subgriddiv=0,gridcolor=blue!50,griddots=10}
\newpsstyle{cadre}{linecolor=yellow!50}
\begin{pspicture*}[showgrid](-7,-6)(7,6)
\psframe*[linecolor={[HTML]{996666}}](-7,6)(7,6)
\psmagneticfield[linecolor={[HTML]{660066}}]
\end{pspicture*}
\end{postscript}
\end{center}

\begin{lstlisting}
\psset{unit=0.75,AntiHelmholtz,N=2,
  R=2,pointsB=500,pointsS=1000,PasB=0.02,PasS=0.00275,nS=10,
  nL=2,drawSelf=true,styleSpire=styleSpire,styleCourant=sensCourant}
\newpsstyle{grille}{subgriddiv=0,gridcolor=blue!50,griddots=10}
\newpsstyle{cadre}{linecolor=yellow!50}
\begin{pspicture*}[showgrid](-7,-6)(7,6)
\psframe*[linecolor={[HTML]{996666}}](-7,6)(7,6)
\psmagneticfield[linecolor={[HTML]{660066}}]
\end{pspicture*}
\end{lstlisting}


\clearpage
\section{3D views}
3D views are possible with the macros

\begin{BDef}
\Lcs{psmagneticfield}\OptArgs\coord1\coord2\\
\Lcs{psmagneticfieldThreeD}\OptArgs\coord1\coord2
\end{BDef}

in which options are settings \Lcs{psmagneticfield} and \verb+(x1,y1)(x2,y2)+
coordinates of bottom left corner and upper right framework
is encapsulated as the field map for \Lcs{psframe}. We can use the option
\Lkeyword{viewpoint} of the \LPack{pst-3d} package to change the view.
  The options framework are by default, the following:
\begin{verbatim}
\newpsstyle{grille}{subgriddiv=0,gridcolor=lightgray,griddots=10}
\newpsstyle{cadre}{linecolor=green!20}
\end{verbatim}

In the following example we can see the handling of these two psstyles.

\begin{center}
\begin{postscript}
\psset{unit=0.7cm}
\newpsstyle{grille}{subgriddiv=0,gridcolor=blue!50,griddots=10}
\newpsstyle{cadre}{linecolor=yellow!50}
\begin{pspicture}(-7,-6)(7,6)
\psmagneticfieldThreeD[N=8,R=2,L=8,pointsB=1200,linecolor=blue,pointsS=2000](-7,-6)(7,6)
\end{pspicture}
\end{postscript}
\end{center}

\begin{lstlisting}
\psset{unit=0.7cm}
\newpsstyle{grille}{subgriddiv=0,gridcolor=blue!50,griddots=10}
\newpsstyle{cadre}{linecolor=yellow!50}
\begin{pspicture}(-7,-6)(7,6)
\psmagneticfieldThreeD[N=8,R=2,L=8,pointsB=1200,linecolor=blue,pointsS=2000](-7,-6)(7,6)
\end{pspicture}
\end{lstlisting}


\begin{center}
\begin{postscript}
\psset{unit=0.7cm}
\begin{pspicture}(-7,-6)(7,6)
\psmagneticfieldThreeD[N=2,R=2,L=2,linecolor=blue](-7,-6)(7,6)
\ThreeDput{\rput(0,-7){\textbf{Bobines de HELMHOLTZ}}}
\end{pspicture}
\end{postscript}
\end{center}

\begin{lstlisting}
\psset{unit=0.7cm}
\begin{pspicture}(-7,-6)(7,6)
\psmagneticfieldThreeD[N=2,R=2,L=2,linecolor=blue](-7,-6)(7,6)
\ThreeDput{\rput(0,-7){\textbf{Bobines de HELMHOLTZ}}}
\end{pspicture}
\end{lstlisting}

\begin{center}
\begin{postscript}
\psset{unit=0.75cm,AntiHelmholtz,N=2,
  R=2,pointsB=500,pointsS=1000,PasB=0.02,PasS=0.00275,nS=10,
  nL=2,drawSelf,styleSpire=styleSpire,styleCourant=sensCourant}
\newpsstyle{grille}{subgriddiv=0,gridcolor=blue!50,griddots=10}
\newpsstyle{cadre}{linecolor=yellow!50}
\begin{pspicture}(-7,-6)(7,6)
\psmagneticfieldThreeD[linecolor={[HTML]{660066}}](-7,-6)(7,6)
\end{pspicture}
\end{postscript}
\end{center}

\begin{lstlisting}
\psset{unit=0.75cm,AntiHelmholtz,N=2,
  R=2,pointsB=500,pointsS=1000,PasB=0.02,PasS=0.00275,nS=10,
  nL=2,drawSelf,styleSpire=styleSpire,styleCourant=sensCourant}
\newpsstyle{grille}{subgriddiv=0,gridcolor=blue!50,griddots=10}
\newpsstyle{cadre}{linecolor=yellow!50}
\begin{pspicture}(-7,-6)(7,6)
\psmagneticfieldThreeD[linecolor={[HTML]{660066}}](-7,-6)(7,6)
\end{pspicture}
\end{lstlisting}

\section{Density plots}
The optional argument \Lkeyword{StreamDensityPlot} allows to plot the
magnetic field as a colored stream density. A gray colored output is possioble
with setting the keyword \Lkeyword{setgray}. 

\begin{center}
\begin{postscript}
\begin{pspicture}(-6,-4)(6,4)
\psmagneticfield[N=3,R=2,L=2,StreamDensityPlot](-6,-4)(6,4)
\end{pspicture}
\end{postscript}
\end{center}

\begin{lstlisting}
\begin{pspicture}(-6,-4)(6,4)
\psmagneticfield[N=3,R=2,L=2,StreamDensityPlot](-6,-4)(6,4)
\end{pspicture}
\end{lstlisting}

\begin{center}
\begin{postscript}
\psset{unit=0.75}
\begin{pspicture}(-6,-5)(6,5)
\psmagneticfield[N=2,R=2,L=1,StreamDensityPlot,setgray](-6,-5)(6,5)
\end{pspicture}
\end{postscript}
\end{center}

\begin{lstlisting}
\psset{unit=0.75}
\begin{pspicture}(-6,-5)(6,5)
\psmagneticfield[N=2,R=2,L=1,StreamDensityPlot,setgray](-6,-5)(6,5)
\end{pspicture}
\end{lstlisting}


\begin{center}
\begin{postscript}
\psset{unit=0.75,AntiHelmholtz,
  R=2,pointsB=500,pointsS=2000,PasB=0.02,PasS=0.00275,nS=10,
  nL=2,drawSelf=true,styleSpire=styleSpire,styleCourant=sensCourant}
\begin{pspicture*}(-7,-6)(7,6)
\psmagneticfield[linecolor={[HTML]{660066}},StreamDensityPlot](-7,-6)(7,6)
\end{pspicture*}
\end{postscript}
\end{center}


\begin{lstlisting}
\psset{unit=0.75,AntiHelmholtz,
  R=2,pointsB=500,pointsS=2000,PasB=0.02,PasS=0.00275,nS=10,
  nL=2,drawSelf=true,styleSpire=styleSpire,styleCourant=sensCourant}
\begin{pspicture*}(-7,-6)(7,6)
\psmagneticfield[linecolor={[HTML]{660066}},StreamDensityPlot](-7,-6)(7,6)
\end{pspicture*}
\end{lstlisting}


\clearpage
\section{Stream density}

\begin{BDef}
\Lcs{psmagneticfield}\OptArgs\coord1\coord2\\
\Lcs{psmagneticfieldThreeD}\OptArgs\coord1\coord2
\end{BDef}


\begin{center}
\psset{unit=0.75cm}
\begin{pspicture*}(-7,-6)(7,6)
\psmagneticfieldDensity[N=4,R=2,L=3]
\end{pspicture*}
\end{center}

\begin{lstlisting}
\psset{unit=0.75cm}
\begin{pspicture*}(-7,-6)(7,6)
\psmagneticfieldDensity[N=4,R=2,L=3]
\end{pspicture*}
\end{lstlisting}

\begin{center}
\psset{unit=0.75cm}
\begin{pspicture*}[showgrid](-4,-3)(4,3)
\psmagneticfieldDensity[N=2,R=2,L=2]
\end{pspicture*}
\end{center}

\begin{lstlisting}
\psset{unit=0.75cm}
\begin{pspicture*}[showgrid](-4,-3)(4,3)
\psmagneticfieldDensity[N=2,R=2,L=2]
\end{pspicture*}
\end{lstlisting}


\clearpage
\section{List of all optional arguments for \texttt{pst-magneticfield}}

\xkvview{family=pst-magneticfield,columns={key,type,default}}

\nocite{*}
\bgroup
\raggedright
\bibliographystyle{plain}
\bibliography{pst-magneticfield-doc}
\egroup


\printindex




\end{document}
