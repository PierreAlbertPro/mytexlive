\section {Le code jps}

Nous appelons \textsl {code jps\/} tout code postscript utilisant la
biblioth�que d�velopp�e pour le logiciel \textsl {jps2ps}.

Le fichier \verb+solides.pro+ du package \verb+solides3d+
est essentiellement constitu� d'�l�ments en provenance de cette 
biblioth�que, qui contient environ $4\, 500$~fonctions et
proc�dures. 

Son utilisation permet de disposer de commandes adapt�es au dessin
math�matique, sans qu'il soit besoin de tout reconstruire � partir des
primitives \verb+moveto+, \verb+lineto+, \verb+curveto+, etc...

Par exemple, on peut d�finir une fonction $F$ telle que $F(t) =
(3\cos^3 t, 3\sin^3 t)$, et demander le trac� de la courbe avec la
code jps \texttt{0 360 {F} CourbeR2}.

Si on veut seulement le chemin de cette courbe, on utilise le code
\verb+0 360 {F} CourbeR2_+, et si on veut le d�pot sur la pile des
points de la courbe, on utilise \texttt{0 360 {F} CourbeR2+}.

Dans chacun des $3$~exemples ci-dessus, le nombre de points est
d�termin� par la variable globale \textsl {resolution}.

Autrement dit, avec la fonction $F$ pr�cit�e et une r�solution fix�e �
36, le code jps 
\begin{verbatim}
   0 350 {F} CourbeR2+
\end{verbatim}
est  �quivalent au code postscript  
\begin{verbatim}
   0 10 350 {
      /angle exch def
      3 angle cos 3 exp mul
      3 angle sin 3 exp mul
   } for
\end{verbatim}

Nous n'avons pas encore d�velopp� la documentation sur la partie
sp�cifique de cette biblioth�que embarqu�e dans le fichier
\verb+solides.pro+. Pour le moment, nous renvoyons le lecteur
int�ress� au \textsl {Guide de l'utilisateur de jps2ps\/} disponible
sur le site \url {melusine.eu.org/syracuse/bbgraf}.