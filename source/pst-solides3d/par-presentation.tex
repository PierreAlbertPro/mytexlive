\section {Pr�sentation}

Le package \texttt{pst-solides3d} permet d'obtenir, avec PSTricks, des
vues en 3d de solides pr�d�finis ou construits par
l'utilisateur. On trouvera la plupart des solides usuels que l'on peut
repr�senter avec ou sans les ar�tes cache�s et dont la couleur peut
varier avec l'�clairage.

Ce package permet �galement de projeter des textes ou des dessins
simples en 2d sur un plan quelconque ou sur une face d'un solide d�j�
construit.

Du point de vue utilisateur, la plupart des fonctionnalit�s sont
accessibles par trois macros \TeX~: \verb+\psSolid+, qui sert �
manipuler les objets � $3$ dimensions, \verb+\psSurface+, cousine de
la premi�re et d�di�e � la repr�sentation de surfaces d�finies par une
�quation du type $f (x, y) = z$, et \verb+\psProjection+ qui permet de
projeter un dessin en $2$~dimensions sur un plan quelconque de la
sc�ne $3$d repr�sent�e.

Dans l'utilisation, deux langages cohabitent~: d'une part PSTricks et
ses macros o� l'utilisateur retrouvera la syntaxe usuelle, d'autre
part Postscript que l'on voit appara�tre dans les argument optionnels
des pr�c�dentes.

Le parti pris a �t� de limiter strictement le champ d'action de
PSTricks, pour le cantonner au r�le d'interface entre \TeX {} et
Postscript. Plus pr�cis�ment, le r�le de PSTricks a
strictement �t� circonscrit � celui de la transmission des param�tres
vers Postscript, ce dernier s'occupant de la totalit� des calculs
n�cessaires puis de l'affichage.

Pour l'ensemble de ces proc�dures de calculs et d'affichages, nous
utilisons une librairie Postscript d�velopp�e pour
une autre application (le logiciel \textsl {jps2ps}).
Le code postscript utilisant cette librairie est appel� \textsl{code
jps}. 

Le but de ce pr�sent document est de d�crire la syntaxe PSTricks pour
chacune des op�rations offertes par le package.
