\section{Tracer une ligne bris�e}

Cette commande est adapt�e de la macro \verb+\pstThreeDLine+ du package \texttt{pst-3dplot} de H.Voss\footnote{\url{http://tug.ctan.org/tex-archive/graphics/pstricks/contrib/pst-3dplot}.}

On l'utilise : \psframebox[fillstyle=solid,fillcolor=yellow,linecolor=yellow]{\texttt{$\backslash$psLineIIID[options](x0,y0,z0)(x1,y1,z1)\ldots(xn,yn,zn)}},
avec les options suivantes possibles :
\begin{itemize}
  \item \texttt{linecolor=couleur} ;
  \item \texttt{doubleline=true} ;
  \item \texttt{linearc=valeur}.
\end{itemize}
On ne peut pas fl�cher les extr�mit�s d'une ligne.

\psset{SphericalCoor,viewpoint=50 20 30,Decran=50}
\begin{LTXexample}[width=6.5cm]
\begin{pspicture}(-3,-4)(4,4)
\psframe(-3,-4)(4,4)
\psSolid[object=cube,a=4,action=draw*,fillcolor=magenta!20]%
\psLineIIID[linecolor=blue,linewidth=0.1,linearc=0.5,doubleline=true](-2,-2,-2)(2,2,2)(2,2,-2)(2,-2,0)
\psPoint(2,-2,0){A}\psPoint(-2,-2,-2){B}
\psPoint(2,2,2){C}\psPoint(2,2,-2){D}
\psdot[dotsize=0.2](A)\psdot[dotsize=0.2](B)
\psdot[dotsize=0.2](C)\psdot[dotsize=0.2](D)
\psLineIIID[linecolor=green](-2,-2,-2)(2,2,2)(2,2,-2)(2,-2,0)
\psPolygonIIID[linecolor=red,fillstyle=vlines,linearc=0.5,linewidth=0.1](-2,-2,2)(-2,2,2)(2,2,2)(2,-2,2)
\axesIIID(2,2,2)(4,4,4)
\end{pspicture}
\end{LTXexample}

%% \section{Tracer une ligne bris�e}
%% 
%% Cette commande est adapt�e de la macro \verb+\pstThreeDLine+ du package \texttt{pst-3dplot} de H.Voss\footnote{\url{http://tug.ctan.org/tex-archive/graphics/pstricks/contrib/pst-3dplot}.}
%% 
%% On l'utilise : \psframebox[fillstyle=solid,fillcolor=yellow,linecolor=yellow]{\texttt{$\backslash$psLineIIID[options](x0,y0,z0)(x1,y1,z1)\ldots(xn,yn,zn)}},
%% avec les options suivantes possibles :
%% \begin{itemize}
%%   \item \texttt{linecolor=couleur} ;
%%   \item \texttt{doubleline=true} ;
%%   \item \texttt{linearc=valeur}.
%% \end{itemize}
%% On ne peut pas fl�cher les extr�mit�s d'une ligne.
%% 
%% \psset{SphericalCoor,viewpoint=50 20 30,Decran=50}
%% \begin{LTXexample}[width=6.5cm]
%% \begin{pspicture}(-3,-4)(4,4)
%% \psframe(-3,-4)(4,4)
%% \psSolid[object=cube,a=4,action=draw*,fillcolor=magenta!20]%
%% \psLineIIID[linecolor=blue,linewidth=0.1,linearc=0.5,doubleline=true](-2,-2,-2)(2,2,2)(2,2,-2)(2,-2,0)
%% \psPoint(2,-2,0){A}\psPoint(-2,-2,-2){B}
%% \psPoint(2,2,2){C}\psPoint(2,2,-2){D}
%% \psdot[dotsize=0.2](A)\psdot[dotsize=0.2](B)
%% \psdot[dotsize=0.2](C)\psdot[dotsize=0.2](D)
%% \psLineIIID[linecolor=green](-2,-2,-2)(2,2,2)(2,2,-2)(2,-2,0)
%% \psPolygonIIID[linecolor=red,fillstyle=vlines,linearc=0.5,linewidth=0.1](-2,-2,2)(-2,2,2)(2,2,2)(2,-2,2)
%% \axesIIID(2,2,2)(4,4,4)
%% \end{pspicture}
%% \end{LTXexample}
%% 
%% \section{Tracer un polygone}
%% On utilise : \psframebox[fillstyle=solid,fillcolor=yellow,linecolor=yellow]{\texttt{$\backslash$psPolygonIIID[options](x0,y0,z0)(x1,y1,z1)\ldots(xn,yn,zn)}},
%% avec les options suivantes possibles :
%% \begin{itemize}
%%   \item \texttt{linecolor=couleur} ;
%%   \item \texttt{doubleline=true} ;
%%   \item \texttt{linearc=valeur} ;
%%   \item \texttt{fillstyle=solid} ;
%%   \item \texttt{fillstyle=vlines} ou \texttt{fillstyle=hlines} ou \texttt{fillstyle=crosshatch}.
%% \end{itemize}
%% \newpage
%% 
%% \section{Transformer un point et le m�moriser}
%% Soit un point initial $A(x,y,z)$. On fait subir � ce point des rotations autour des axes $Ox$, $Oy$ et $Oz$ d'angles respectifs :
%% \texttt{[RotX=valeurX,RotX=valeurY,RotX=valeurZ]}, dans cet ordre, puis on op�re une translation de vecteur $(v_x,v_y,v_z)$. Le probl�me a �t� de r�cup�rer les
%% coordonn�es du point final $A'(x',y',z')$.
%% 
%% La commande \psframebox[fillstyle=solid,fillcolor=yellow,linecolor=yellow]{\texttt{$\backslash$psTransformPoint[RotX=valeurX,RotX=valeurY,RotX=valeurZ](x y z)(vx,vy,vz)\{A'\}}}
%%  permet de stocker dans le n\oe{}ud $A'$, les coordonn�es du point transform�.
%% 
%% Dans l'exemple suivant $A(2,2,2)$ est l'un des sommets du cube initial, dont le centre est plac� � l'origine du rep�re.
%% {\red
%% \begin{verbatim}
%% \psSolid[object=cube,a=4,action=draw*,linecolor=red]%
%% \end{verbatim}
%% }
%% Ce cube subit diff�rentes transformations :
%% {\red
%% \begin{verbatim}
%% \psSolid[object=cube,a=4,action=draw*,RotX=-30,RotY=60,RotZ=-60](7.5,11.25,10)%
%% \end{verbatim}
%% }
%% Pour obtenir l'image de $A$, on applique la commande suivante :
%% {\red
%% \begin{verbatim}
%% \psTransformPoint[RotX=-30,RotY=60,RotZ=-60](2 2 2)(7.5,11.25,10){A'}
%% \end{verbatim}
%% }
%% Ce qui permet, par exemple, de nommer ces points et de dessiner le vecteur $\overrightarrow{AA'}$.
%% \begin{center}
%% \begin{pspicture}(-2,-4)(6,6)
%% \psframe(-2,-4)(6,6)
%% \psset{unit=0.5}
%% %\psSolid[object=cube,a=4,action=draw,linecolor=red,fontsize=40,numfaces=all]%
%% \psSolid[object=cube,a=4,action=draw*,linecolor=red]%
%% \psPoint(2,2,2){A}\psdot(A)
%% %\psSolid[object=cube,a=4,action=draw,RotX=-30,RotY=60,RotZ=-60,fontsize=40,numfaces=all](7.5,11.25,10)%
%% \psSolid[object=cube,a=4,action=draw*,RotX=-30,RotY=60,RotZ=-60](7.5,11.25,10)%
%% \psTransformPoint[RotX=-30,RotY=60,RotZ=-60](2 2 2)(7.5,11.25,10){A'}
%% \psdot(A')\psline[linecolor=blue,arrowsize=0.3]{{o-v}}(A)(A')
%% \uput[u](A'){$A'$}\uput[u](A){$A$}
%% \psset{solidmemory}
%% \psSolid[object=cube,a=4,
%%    name=A1,
%%    action=none](0,0,0)
%% \psset{fontsize=100,
%%    phi=90,
%%    no=0,
%%    solidname=A1}
%% \psProjection[object=texte,linecolor=red,text=A]%
%% \psset{fontsize=100,
%%    phi=180,
%%    no=1,
%%    solidname=A1}
%% \psProjection[object=texte,linecolor=red,text=B]%
%% \psset{fontsize=100,
%%    phi=90,
%%    no=4,
%%    solidname=A1}
%% \psProjection[object=texte,linecolor=red,text=E]%
%% %
%% \psset{solidmemory}
%% \psSolid[object=cube,a=4,RotX=-30,RotY=60,RotZ=-60,
%%    name=A2,
%%    action=none](7.5,11.25,10)
%% \psset{fontsize=100,
%%    phi=20,
%%    no=0,
%%    solidname=A2}
%% \psProjection[object=texte,text=A]%
%% \psset{fontsize=100,
%%    phi=160,
%%    no=2,
%%    solidname=A2}
%% \psProjection[object=texte,text=B]%
%% \psset{fontsize=100,
%%    phi=160,
%%    no=1,
%%    solidname=A2}
%% \psProjection[object=texte,text=C]%
%% \axesIIID(2,2,2)(10,10,8)
%% \end{pspicture}
%% \end{center}