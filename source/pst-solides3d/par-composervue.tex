\section {Composer une vue d'un ensemble d'objets avec \texttt{codejps}}

Le calcul des parties cach�es est effectu� par les routines du code
\texttt{PostScript} du fichier \texttt{solides.pro}, mais les lignes
de code sont ``encapsul�es'' dans un environnement \texttt{pspicture}
gr�ce � la commande \verb+\codejps{code ps}+. Cette partie est
d�velopp�e dans le document
\texttt{documentation-fusion-jps.tex(pdf)}. En voici un exemple : 

\begin{minipage}{6cm}
\setlength{\columnseprule}{1pt}
\begin{pspicture}(-3,-4)(3,4)
\psset{lightsrc=100 -50 -10,lightintensity=3,viewpoint=200 20 10,Decran=20,SphericalCoor}
\psframe(-3,-3)(3,3)
{\psset{linewidth=0.5\pslinewidth}
\codejps{
/Cl {9.02  [18 16] newsphere
 {-90 0 0 rotateOpoint3d} solidtransform
 dup (Green) outputcolors} def
/Cl1 { Cl {10.25 10.25 10.25 translatepoint3d} solidtransform } def
/Cl2 { Cl {10.25 -10.25 10.25 translatepoint3d} solidtransform } def
/Cl3 { Cl {-10.25 -10.25 10.25 translatepoint3d} solidtransform } def
/Cl4 { Cl {-10.25 10.25 10.25 translatepoint3d} solidtransform } def
/Cl5 { Cl {10.25 10.25 -10.25 translatepoint3d} solidtransform } def
/Cl6 { Cl {10.25 -10.25 -10.25 translatepoint3d} solidtransform } def
/Cl7 { Cl {-10.25 -10.25 -10.25 translatepoint3d} solidtransform } def
/Cl8 { Cl {-10.25 10.25 -10.25 translatepoint3d} solidtransform } def
/Cs {8.38  [18 16] newsphere
 dup (White) outputcolors} def
%
/Cl12{ Cl1 Cl2 solidfuz} def
/Cl123{ Cl12 Cl3 solidfuz} def
/Cl1234{ Cl123 Cl4 solidfuz} def
/Cl12345{ Cl1234 Cl5 solidfuz} def
/Cl123456{ Cl12345 Cl6 solidfuz} def
/Cl1234567{ Cl123456 Cl7 solidfuz} def
/Cl12345678{ Cl1234567 Cl8 solidfuz} def
/C_Cs { Cl12345678 Cs solidfuz} def
C_Cs  drawsolid**}}%
%\psSolid[object=cube,a=20.5,action=draw,linestyle=dashed]%
\psPoint(0,0,0){P}
\psPoint(10.25,10.25,10.25){Cl1}
\psPoint(10.25,-10.25,10.25){Cl2}
\psPoint(-10.25,-10.25,10.25){Cl3}
\psPoint(-10.25,10.25,10.25){Cl4}
\psPoint(10.25,10.25,-10.25){Cl5}
\psPoint(10.25,-10.25,-10.25){Cl6}
\psPoint(-10.25,-10.25,-10.25){Cl7}
\psPoint(-10.25,10.25,-10.25){Cl8}
\pspolygon[linestyle=dashed](Cl1)(Cl2)(Cl3)(Cl4)
\pspolygon[linestyle=dashed](Cl5)(Cl6)(Cl7)(Cl8)
\psline[linestyle=dashed](Cl2)(Cl6)
\psline[linestyle=dashed](Cl3)(Cl7)
\psline[linestyle=dashed](Cl1)(Cl5)
\psline[linestyle=dashed](Cl4)(Cl8)
\pcline[offset=0.5]{<->}(Cl2)(Cl1)
\aput{:U}{a}
\pcline[offset=0.5]{<->}(Cl6)(Cl2)
\aput{:U}{a}
\end{pspicture}
\end{minipage}
\hfill
\begin{minipage}{14cm}
\footnotesize
\begin{verbatim}
\codejps{
/Cl {9.02  [12 8] newsphere
 {-90 0 0 rotateOpoint3d} solidtransform
 dup (Green) outputcolors} def
/Cl1 { Cl {10.25 10.25 10.25 translatepoint3d} solidtransform } def
/Cl2 { Cl {10.25 -10.25 10.25 translatepoint3d} solidtransform } def
/Cl3 { Cl {-10.25 -10.25 10.25 translatepoint3d} solidtransform } def
/Cl4 { Cl {-10.25 10.25 10.25 translatepoint3d} solidtransform } def
/Cl5 { Cl {10.25 10.25 -10.25 translatepoint3d} solidtransform } def
/Cl6 { Cl {10.25 -10.25 -10.25 translatepoint3d} solidtransform } def
/Cl7 { Cl {-10.25 -10.25 -10.25 translatepoint3d} solidtransform } def
/Cl8 { Cl {-10.25 10.25 -10.25 translatepoint3d} solidtransform } def
/Cs {8.38  [12 8] newsphere
 dup (White) outputcolors} def
/Cl12{ Cl1 Cl2 solidfuz} def
/Cl123{ Cl12 Cl3 solidfuz} def
/Cl1234{ Cl123 Cl4 solidfuz} def
/Cl12345{ Cl1234 Cl5 solidfuz} def
/Cl123456{ Cl12345 Cl6 solidfuz} def
/Cl1234567{ Cl123456 Cl7 solidfuz} def
/Cl12345678{ Cl1234567 Cl8 solidfuz} def
/C_Cs { Cl12345678 Cs solidfuz} def
C_Cs  drawsolid**}
\end{verbatim}
\end{minipage}

On d�finit l'ion chlorure $\mathrm{Cl^-}$ :
\begin{verbatim}
/Cl {9.02  [12 8] newsphere
 {-90 0 0 rotateOpoint3d} solidtransform
 dup (Green) outputcolors} def
\end{verbatim}
que l'on recopie aux sommets du cube :
\begin{verbatim}
/Cl1 { Cl {10.25 10.25 10.25 translatepoint3d} solidtransform } def
/Cl2 { Cl {10.25 -10.25 10.25 translatepoint3d} solidtransform } def
/Cl3 { Cl {-10.25 -10.25 10.25 translatepoint3d} solidtransform } def
/Cl4 { Cl {-10.25 10.25 10.25 translatepoint3d} solidtransform } def
/Cl5 { Cl {10.25 10.25 -10.25 translatepoint3d} solidtransform } def
/Cl6 { Cl {10.25 -10.25 -10.25 translatepoint3d} solidtransform } def
/Cl7 { Cl {-10.25 -10.25 -10.25 translatepoint3d} solidtransform } def
/Cl8 { Cl {-10.25 10.25 -10.25 translatepoint3d} solidtransform } def
\end{verbatim}
Puis l'ion c�sium $\mathrm{Cs^+}$, plac� au centre :
\begin{verbatim}
/Cs {8.38  [12 8] newsphere
 dup (White) outputcolors} def
\end{verbatim}
Ensuite on fusionne deux par deux les diff�rentes sph�res.
