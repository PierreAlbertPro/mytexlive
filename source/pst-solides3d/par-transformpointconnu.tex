\section {Transformer un point et le m�moriser}

Soit un point initial $A(x,y,z)$. On fait subir � ce point des
rotations autour des axes $Ox$, $Oy$ et $Oz$ d'angles respectifs :
\texttt{[RotX=valeurX,RotX=valeurY,RotX=valeurZ]}, dans cet ordre,
puis on op�re une translation de vecteur $(v_x,v_y,v_z)$. Le probl�me
a �t� de r�cup�rer les coordonn�es du point final $A'(x',y',z')$.

La commande
 \psframebox[fillstyle=solid,fillcolor=yellow,linecolor=yellow]{\texttt{$\backslash$psTransformPoint[RotX=valeurX,RotX=valeurY,RotX=valeurZ](x
 y z)(vx,vy,vz)\{A'\}}} 
 permet de stocker dans le n\oe{}ud $A'$, les coordonn�es du point
 transform�. 

Dans l'exemple suivant $A(2,2,2)$ est l'un des sommets du cube
initial, dont le centre est plac� � l'origine du rep�re. 
{\red
\begin{verbatim}
\psSolid[object=cube,a=4,action=draw*,linecolor=red]%
\end{verbatim}
}
Ce cube subit diff�rentes transformations :
{\red
\begin{verbatim}
\psSolid[object=cube,a=4,action=draw*,RotX=-30,RotY=60,RotZ=-60](7.5,11.25,10)%
\end{verbatim}
}
Pour obtenir l'image de $A$, on applique la commande suivante :
{\red
\begin{verbatim}
\psTransformPoint[RotX=-30,RotY=60,RotZ=-60](2 2 2)(7.5,11.25,10){A'}
\end{verbatim}
}
Ce qui permet, par exemple, de nommer ces points et de dessiner le vecteur $\overrightarrow{AA'}$.
\begin{center}
\begin{pspicture}(-2,-4)(6,6)
\psframe(-2,-4)(6,6)
\psset{unit=0.5}
%\psSolid[object=cube,a=4,action=draw,linecolor=red,fontsize=40,numfaces=all]%
\psSolid[object=cube,a=4,action=draw*,linecolor=red]%
\psPoint(2,2,2){A}\psdot(A)
%\psSolid[object=cube,a=4,action=draw,RotX=-30,RotY=60,RotZ=-60,fontsize=40,numfaces=all](7.5,11.25,10)%
\psSolid[object=cube,a=4,action=draw*,RotX=-30,RotY=60,RotZ=-60](7.5,11.25,10)%
\psTransformPoint[RotX=-30,RotY=60,RotZ=-60](2 2 2)(7.5,11.25,10){A'}
\psdot(A')\psline[linecolor=blue,arrowsize=0.3]{{o-v}}(A)(A')
\uput[u](A'){$A'$}\uput[u](A){$A$}
\psset{solidmemory}
\psSolid[object=cube,a=4,
   name=A1,
   action=none](0,0,0)
\psset{fontsize=100,
%   phi=90,
   no=0,
   solidname=A1}
\psProjection[object=texte,linecolor=red,text=A]%
\psset{fontsize=100,
%   phi=180,
   no=1,
   solidname=A1}
\psProjection[object=texte,linecolor=red,text=B]%
\psset{fontsize=100,
%   phi=90,
   no=4,
   solidname=A1}
\psProjection[object=texte,linecolor=red,text=E]%
%
\psset{solidmemory}
\psSolid[object=cube,a=4,RotX=-30,RotY=60,RotZ=-60,
   name=A2,
   action=none](7.5,11.25,10)
\psset{fontsize=100,
%   phi=25,
   no=0,
   solidname=A2}
\psProjection[object=texte,text=A]%
\psset{fontsize=100,
%   phi=160,
   no=2,
   solidname=A2}
\psProjection[object=texte,text=C]%
\psset{fontsize=100,
%   phi=255,
   no=1,
   solidname=A2}
\psProjection[object=texte,text=B]%
\axesIIID(2,2,2)(10,10,8)
\end{pspicture}
\end{center}

