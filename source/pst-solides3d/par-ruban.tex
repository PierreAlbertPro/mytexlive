\section {Solide ruban}

Le ruban est un paravent pos� sur le sol horizontal. La base du paravent est d�finie sur le plan $Oxy$ par
les coordonn�es des sommets plac�s dans le sens trigonom�trique par le param�tre \texttt{base} :
\begin{verbatim}
\psSolid[object=ruban,h=3,base=x1 y1 x2 y2 x3 y3 ...xn yn,ngrid=n](0,0,0)%
\end{verbatim}
\subsection{Un simple paravent}
\begin{center}
\psset{lightsrc=10 0 10,SphericalCoor,viewpoint=50 -20 30,Decran=50,unit=0.75}
\begin{pspicture}(-5,-4)(6,7)
\psframe(-5,-4)(6,7)
\psSolid[object=grille,base=-4 6 -4 4,action=draw,linecolor=gray](0,0,0)
\psSolid[object=ruban,h=3,fillcolor=red!50,
      base=0 0 2 2 4 0 6 2,
      num=0 1 2 3,
      show=0 1 2 3,
      ngrid=3
      ](0,0,0)
\axesIIID(0,2,0)(6,6,6)
\end{pspicture}
\end{center}
\begin{verbatim}
\begin{pspicture}(-5,-4)(6,7)
\psframe(-5,-4)(6,7)
\psSolid[object=grille,base=-4 6 -4 4,action=draw](0,0,0)
\psSolid[object=ruban,h=3,fillcolor=red!50,
      base=0 0 2 2 4 0 6 2,
      num=0 1 2 3,
      show=0 1 2 3,
      ngrid=3
      ](0,0,0)
\axesIIID(0,2,0)(6,6,6)
\end{pspicture}
\end{verbatim}

\subsection{Un paravent sinuso�dal}

\begin{center}
\psset{unit=0.6}
\psset{lightsrc=10 30 10,SphericalCoor,viewpoint=50 50 20,Decran=50}
\begin{pspicture}(-10,-5)(12,7)
\psframe(-10,-5)(12,7)
\defFunction{F}(t){2 t 4 mul cos mul}{t 20 div}{}
\psSolid[object=grille,base=-6 6 -10 10,action=draw,linecolor=gray](0,0,0)
\psSolid[object=ruban,h=2,fillcolor=red!50,
      resolution=72,
      base=-200 200 {F} CourbeR2+,  %% -200 5 200 {/Angle ED 2 Angle 4 mul cos mul Angle 20 div } for,
      ngrid=4](0,0,0)
\axesIIID(5,10,0)(7,11,6)
\end{pspicture}
\end{center}

\begin{verbatim}
\psset{unit=0.6}
\psset{lightsrc=10 30 10,SphericalCoor,viewpoint=50 50 20,Decran=50}
\begin{pspicture}(-10,-5)(12,7)
\psframe(-10,-5)(12,7)
\defFunction{F}(t){2 t 4 mul cos mul}{t 20 div}{}
\psSolid[object=grille,base=-6 6 -10 10,action=draw,linecolor=gray](0,0,0)
\psSolid[object=ruban,h=2,fillcolor=red!50,
      resolution=72,
      base=-200 200 {F} CourbeR2+, 
      ngrid=4](0,0,0)
\axesIIID(5,10,0)(7,11,6)
\end{pspicture}
\end{verbatim}

\subsection{Une surface ondul�e}

C'est le m�me objet que pr�c�demment en lui faisant subir une rotation de $90^{\mathrm{o}}$ autour de $Oy$.
\begin{center}
\psset{unit=0.6}
\psset{lightsrc=10 30 10,SphericalCoor,viewpoint=50 50 20,Decran=30}
\begin{pspicture}(-14,-7)(8,7)
\psframe(-14,-7)(8,7)
\defFunction{F}(t){t 4 mul cos}{t 20 div}{}
\psSolid[object=grille,base=0 16 -10 10,action=draw,linecolor=gray](0,0,0)
\psSolid[object=ruban,h=16,fillcolor=red!50,RotY=90,incolor=green!20,
      resolution=72,
      base=-200 200 {F} CourbeR2+, 
      ngrid=16](0,0,1)
\axesIIID(16,10,0)(20,12,6)
\end{pspicture}
\end{center}
\begin{verbatim}
\psset{unit=0.6}
\psset{lightsrc=10 30 10,SphericalCoor,viewpoint=50 50 20,Decran=30}
\begin{pspicture}(-14,-7)(8,7)
\defFunction{F}(t){t 4 mul cos}{t 20 div}{}
\psSolid[object=grille,base=0 16 -10 10,action=draw,linecolor=gray](0,0,0)
\psSolid[object=ruban,h=16,fillcolor=red!50,RotY=90,incolor=green!20,
      resolution=72,
      base=-200 200 {F} CourbeR2+, 
      ngrid=16](0,0,1)
\psframe(-14,-7)(8,7)
\axesIIID(16,10,0)(20,12,6)
\end{pspicture}
\end{verbatim}
On peut ensuite l'imaginer comme toit en t�le ondul�e d'un abri quelconque.

\subsection{Un paravent �toil� : version 1}
Le contour du paravent est d�fini dans une boucle :
\begin{verbatim}
       base=0 72 360 {/Angle ED 5 Angle cos mul 5 Angle sin mul
            3 Angle 36 add cos mul 3 Angle 36 add sin mul} for
\end{verbatim}
la surface bleut�e du fond est d�finie � l'aide d'un polygone dont les
sommets sont calcul�s par la commande \verb+\psPoint(x,y,z){P}+ 
\begin{verbatim}
\multido{\iA=0+72,\iB=36+72,\i=0+1}{6}{%
    \psPoint(\iA\space cos 5 mul,\iA\space sin 5 mul,0){P\i}
    \psPoint(\iB\space cos 3 mul,\iB\space sin 3 mul,0){p\i}
    }%
\pspolygon[fillstyle=solid,fillcolor=blue!50](P0)(p0)(P1)(p1)(P2)(p2)
                                             (P3)(p3)(P4)(p4)(P5)(p5)
\end{verbatim}
\begin{center}
\psset{unit=0.75}
\psset{lightsrc=10 0 10,SphericalCoor,viewpoint=50 20 30,Decran=50}
\begin{pspicture}(-9,-5)(9,7)
\psframe(-9,-5)(9,7)
\multido{\iA=0+72,\iB=36+72,\i=0+1}{6}{%
    \psPoint(\iA\space cos 5 mul,\iA\space sin 5 mul,0){P\i}
    \psPoint(\iB\space cos 3 mul,\iB\space sin 3 mul,0){p\i}
    }%
\pspolygon[fillstyle=solid,fillcolor=blue!50](P0)(p0)(P1)(p1)(P2)(p2)(P3)(p3)(P4)(p4)(P5)(p5)
\defFunction{F}(t){t cos 5 mul}{t sin 5 mul}{}
\defFunction{G}(t){t 36 add cos 3 mul}{t 36 add sin 3 mul}{}
\psSolid[object=grille,base=-6 6 -6 6,action=draw,linecolor=gray](0,0,0)
\psSolid[object=ruban,h=1,fillcolor=red!50,
      base=0 72 360 {/Angle exch def Angle F Angle G} for,
      num=0 1 2 3,
      show=0 1 2 3,
      ngrid=2](0,0,0)
\axesIIID(5,5,0)(6,6,6)
\end{pspicture}
\end{center}
\begin{verbatim}
\begin{pspicture}(-9,-5)(9,7)
\psframe(-9,-5)(9,7)
\multido{\iA=0+72,\iB=36+72,\i=0+1}{6}{%
    \psPoint(\iA\space cos 5 mul,\iA\space sin 5 mul,0){P\i}
    \psPoint(\iB\space cos 3 mul,\iB\space sin 3 mul,0){p\i}
    }%
\pspolygon[fillstyle=solid,fillcolor=blue!50](P0)(p0)(P1)(p1)(P2)(p2)(P3)(p3)(P4)(p4)(P5)(p5)
\defFunction{F}(t){t cos 5 mul}{t sin 5 mul}{}
\defFunction{G}(t){t 36 add cos 3 mul}{t 36 add sin 3 mul}{}
\psSolid[object=grille,base=-6 6 -6 6,action=draw,linecolor=gray](0,0,0)
\psSolid[object=ruban,h=1,fillcolor=red!50,
      base=0 72 360 {/Angle exch def Angle F Angle G} for,
      num=0 1 2 3,
      show=0 1 2 3,
      ngrid=2](0,0,0)
\axesIIID(5,5,0)(6,6,6)
\end{pspicture}
\end{verbatim}

\subsection{Un paravent �toil� : version 2}

Le fond du r�cipient est d�fini par l'objet \texttt{face} avec l'option \texttt{biface}~:
\begin{center}
\psset{unit=0.75}
\psset{lightsrc=10 0 10,SphericalCoor,viewpoint=50 -20 20,Decran=50}
\begin{pspicture}(-9,-5)(9,7)
\psframe(-9,-5)(9,7)
\defFunction{F}(t){t cos 5 mul}{t sin 5 mul}{}
\defFunction{G}(t){t 36 add cos 3 mul}{t 36 add sin 3 mul}{}
\psSolid[object=face,fillcolor=blue!50,
      biface,
      base=0 72 360 {/Angle exch def Angle F Angle G} for,
      ](0,0,0)
\psSolid[object=grille,base=-6 6 -6 6,action=draw,linecolor=gray](0,0,0)
\psSolid[object=ruban,h=1,fillcolor=red!50,
      base=0 72 360 {/Angle exch def Angle F Angle G} for,
      ngrid=2](0,0,0)
\axesIIID(5,5,0)(6,6,6)
\end{pspicture}
\end{center}
\begin{verbatim}
\psset{lightsrc=10 0 10,SphericalCoor,viewpoint=50 -20 20,Decran=50}
\begin{pspicture}(-9,-5)(9,7)
\psframe(-9,-5)(9,7)
\defFunction{F}(t){t cos 5 mul}{t sin 5 mul}{}
\defFunction{G}(t){t 36 add cos 3 mul}{t 36 add sin 3 mul}{}
\psSolid[object=face,fillcolor=blue!50,
      biface,
      base=0 72 360 {/Angle exch def Angle F Angle G} for,
      ](0,0,0)
\psSolid[object=grille,base=-6 6 -6 6,action=draw,linecolor=gray](0,0,0)
\psSolid[object=ruban,h=1,fillcolor=red!50,
      base=0 72 360 {/Angle exch def Angle F Angle G} for,
      ngrid=2](0,0,0)
\axesIIID(5,5,0)(6,6,6)
\end{pspicture}
\end{verbatim}

