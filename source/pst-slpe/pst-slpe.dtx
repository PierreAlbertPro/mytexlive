%\iffalse -*-mode:Latex;tex-command:"latex *;dvips pst-slpe -o"-*- \fi
%\iffalse
%
% Copyright 1998-2008 Martin Giese, mgiese@risc.uni-linz.ac.at
%                     Herbert Voss (using xkeyval,\psBall,fading)
% 
%% This program can be redistributed and/or modified under the terms
%% of the LaTeX Project Public License Distributed from CTAN archives
%% in directory macros/latex/base/lppl.txt.
%%
%\fi
% \changes{v1.31}{2011/10/25}{ fix bug in psball (hv)}
% \changes{v1.3}{2008/09/20}{ fading option (hv)}
% \changes{v1.2}{2008/06/19}{ \textbackslash psBall (hv)}
% \changes{v1.1}{2006/06/19}{%
%	using the extended pst-xkey instead of the old pst-key package; 
%	creating a dtx file (hv)}
% \changes{v1.0}{2005/03/05}{More compatible to the other PStricks 
% packages. (RN)}
%
%
% \DoNotIndex{\!,\",\#,\$,\%,\&,\',\(,\+,\*,\,,\-,\.,\/,\:,\;,\<,\=,\>,\?}
% \DoNotIndex{\@,\@B,\@K,\@cTq,\@f,\@fPl,\@ifnextchar,\@nameuse,\@oVk}
% \DoNotIndex{\[,\\,\],\^,\_,\ }
% \DoNotIndex{\^,\\^,\\\^,$\^$,$\\^$,$\\^$}
% \DoNotIndex{\0,\2,\4,\5,\6,\7,\8,}
% \DoNotIndex{\A,\a}
% \DoNotIndex{\B,\b,\Bc,\begin,\Bq,\Bqc}
% \DoNotIndex{\C,\c,\catcode,\cJA,\CodelineIndex,\csname}
% \DoNotIndex{\D,\def,\define@key,\Df,\divide,\DocInput,\documentclass,\pst@addfams}
% \DoNotIndex{\eCN,\edef,\else,\eHd,\eMcj,\EnableCrossrefs,\end,\endcsname}
% \DoNotIndex{\endCenterExample,\endExample,\endinput,\endpsclip}
% \DoNotIndex{\PrintIndex,\PrintChanges,\ProvidesFile}
% \DoNotIndex{\endpspicture,\endSideBySideExample,\Example}
% \DoNotIndex{\F,\f,\FdUrr,\fi,\filedate,\fileversion,\FV@Environment}
% \DoNotIndex{\FV@UseKeyValues,\FV@XRightMargin,\FVB@Example,\fvset}
% \DoNotIndex{\G,\g,\GetFileInfo,\gr,\GradientLoaded,\gsFKrbK@o,\gsj,\gsOX}
% \DoNotIndex{\hbadness,\hfuzz,\HLEmphasize,\HLMacro,\HLMacro@i}
% \DoNotIndex{\HLReverse,\HLReverse@i,\hqcu,\HqY}
% \DoNotIndex{\I,\i,\ifx,\input,\Ir,\IU}
% \DoNotIndex{\j,\jl,\JT,\JVodH}
% \DoNotIndex{\K,\k,\kfSlL}
% \DoNotIndex{\L,\let}
% \DoNotIndex{\message,\mHNa,\mIU}
% \DoNotIndex{\N,\nB,\newcmykcolor,\newdimen,\newif,\nW}
% \DoNotIndex{\O,\oCDJDo,\ocQhVI,\OnlyDescription,\oRKJ}
% \DoNotIndex{\P,\p,\ProvidesPackage,\psframe,\pslinewidth,\psset}
% \DoNotIndex{\PstAtCode,\PSTricksLoaded}
% \DoNotIndex{\q,\Qr,\qssRXq,\qu,\qXjFQp,\qYL}
% \DoNotIndex{\R,\r,\RecordChanges,\relax,\RlaYI,\rN,\Rp,\rp,\RPDXNn,\rput}
% \DoNotIndex{\S,\scalebox,\SgY,\SideBySide@Example,\SideBySideExample}
% \DoNotIndex{\SgY,\sk,\Sp,\space,\sZb}
% \DoNotIndex{\T,\the,\tw@}
% \DoNotIndex{\u,\UiSWGEf@,\uJi,\usepackage,\uVQdMM,\UYj}
% \DoNotIndex{\VerbatimEnvironment,\VerbatimInput,\VrC@}
% \DoNotIndex{\WhZ,\WjKCYb,\WNs}
% \DoNotIndex{\XkN,\XW}
% \DoNotIndex{\Z,\ZCM,\Ze}
% \DoNotIndex{\addtocounter,\advance,\alph,\arabic,\AtBeginDocument,\AtEndDocument}
% \DoNotIndex{\AtEndOfPackage,\begingroup,\bfseries,\bgroup,\box,\csname}
% \DoNotIndex{\else,\endcsname,\endgroup,\endinput,\expandafter,\fi}
% \DoNotIndex{\TeX,\z@,\p@,\@one,\xdef,\thr@@,\string,\sixt@@n,\reset,\or,\multiply,\repeat,\RequirePackage}
% \DoNotIndex{\@cclvi,\@ne,\@ehpa,\@nil,\copy,\dp,\global,\hbox,\hss,\ht,\ifodd,\ifdim,\ifcase,\kern}
% \DoNotIndex{\chardef,\loop,\leavevmode,\ifnum,\lower}
% \setcounter{IndexColumns}{2}
%
%\iffalse
%<*!prolog>
\def\pstslpefileversion{1.31}
\def\pstslpefiledate{2011/10/25}
%</!prolog>
%\fi
%
% \title{\textsf{pst-slpe} package \\ version \pstslpefileversion}
% \author{Martin Giese\footnote{\texttt{giese@ira.uka.de}} and 
%         Herbert Vo\ss\footnote{\texttt{hvoss@tug.org}}}
% \date{\pstslpefiledate}
% \maketitle
%
%\section{Introduction}
%As of the 97 release, PSTricks contains the |pst-grad| 
%package, which provides a gradient fill style for arbitrary shapes.
%Although it often produces nice results, it has a number of
%deficiencies:
%\begin{enumerate}
%\item It is not possible to go from a colour $A$ to $B$ to $C$,
%etc. The most evident application of such a multi-colour gradient are
%of course rainbow effects.  But they can also be useful in informative 
%contexts, eg to identify modes of operation in a scale of values
%(normal/danger/overload).
%\item Colours are interpolated linearly in the RGB space.  This is
%often OK, but when you want to go from red $(1,0,0)$ to green
%$(0,1,0)$, it looks much better to get there via yellow $(1,1,0)$ than 
%via brown $(0.5,0.5,0)$.  The point is, that to get from one saturated
%colour to another, the colours on the way should also be saturated to
%produce an optically pleasing result.
%\item |pst-grad| is limited to {\em linear} gradients, ie~there
%is a (possibly rotated) rectilinear coordinate system, such that the
%colour at every point depends only on the $x$ coordinate of the
%point.  In particular, there is no way to get circular patterns.
%\end{enumerate}
%|pst-slpe| solves {\em all} of the mentioned
%problems in {\em one} package.  
%
%Problems 1.~is addressed by permitting the user to specify an
%arbitrary number of colours, along with the points at which these are
%to be reached.  A special form of each of the fill styles is provided,
%which just needs two colours as parameters, and goes from one to the
%other.  This makes the fill styles easier to use in that simple case.
%
%Problem 2.~is solved by interpolating in the hue-saturation-value
%colour space.  Conversion between RGB and HSV is done behind the
%scenes.  The user specifies colours in RGB.
%
%Finally, |pst-slpe| provides {\em concentric} and {\em radial}
%gradients.  What these mean is best explained with a polar coordinate
%system:  In a concentric pattern, the colour of a point depends on the
%radius coordinate, while in a radial pattern, it depends on the angle
%coordinate.  
%
%As a special bonus, the PostScript part of |pst-slpe| is somewhat
%optimized for speed.  In |ghostscript|, rendering is about 30\% faster 
%than with |pst-grad|.
%\medskip
%
%For most of these problems, solutions have been posted in the
%appropriate \TeX\ newsgroup over the years.  |pst-slpe| has however
%been developed independently from these proposals.  It is based on 
%the original PSTricks 0.93 |gradient| code, most of which has been 
%changed or replaced.  The
%author is indebted to Denis Girou, whose encouragement triggered the
%process of making this a shipable package instead of a private
%experiment.
%\medskip
%
%The new fill styles and the
%graphics parameters provided to use them are described in
%section 2 of this document.  Section 3, if present, documents the 
%implementation consisting of a generic \TeX\ file and a PostScript
%header for the |dvi|-to-PostScript converter.  You can get section 3
%by calling \LaTeX\ as follows on most relevant systems:
%\begin{verbatim}
%latex '\AtBeginDocument{\AlsoImplementation}%\iffalse -*-mode:Latex;tex-command:"latex *;dvips pst-slpe -o"-*- \fi
%\iffalse
%
% Copyright 1998-2008 Martin Giese, mgiese@risc.uni-linz.ac.at
%                     Herbert Voss (using xkeyval,\psBall,fading)
% 
%% This program can be redistributed and/or modified under the terms
%% of the LaTeX Project Public License Distributed from CTAN archives
%% in directory macros/latex/base/lppl.txt.
%%
%\fi
% \changes{v1.31}{2011/10/25}{ fix bug in psball (hv)}
% \changes{v1.3}{2008/09/20}{ fading option (hv)}
% \changes{v1.2}{2008/06/19}{ \textbackslash psBall (hv)}
% \changes{v1.1}{2006/06/19}{%
%	using the extended pst-xkey instead of the old pst-key package; 
%	creating a dtx file (hv)}
% \changes{v1.0}{2005/03/05}{More compatible to the other PStricks 
% packages. (RN)}
%
%
% \DoNotIndex{\!,\",\#,\$,\%,\&,\',\(,\+,\*,\,,\-,\.,\/,\:,\;,\<,\=,\>,\?}
% \DoNotIndex{\@,\@B,\@K,\@cTq,\@f,\@fPl,\@ifnextchar,\@nameuse,\@oVk}
% \DoNotIndex{\[,\\,\],\^,\_,\ }
% \DoNotIndex{\^,\\^,\\\^,$\^$,$\\^$,$\\^$}
% \DoNotIndex{\0,\2,\4,\5,\6,\7,\8,}
% \DoNotIndex{\A,\a}
% \DoNotIndex{\B,\b,\Bc,\begin,\Bq,\Bqc}
% \DoNotIndex{\C,\c,\catcode,\cJA,\CodelineIndex,\csname}
% \DoNotIndex{\D,\def,\define@key,\Df,\divide,\DocInput,\documentclass,\pst@addfams}
% \DoNotIndex{\eCN,\edef,\else,\eHd,\eMcj,\EnableCrossrefs,\end,\endcsname}
% \DoNotIndex{\endCenterExample,\endExample,\endinput,\endpsclip}
% \DoNotIndex{\PrintIndex,\PrintChanges,\ProvidesFile}
% \DoNotIndex{\endpspicture,\endSideBySideExample,\Example}
% \DoNotIndex{\F,\f,\FdUrr,\fi,\filedate,\fileversion,\FV@Environment}
% \DoNotIndex{\FV@UseKeyValues,\FV@XRightMargin,\FVB@Example,\fvset}
% \DoNotIndex{\G,\g,\GetFileInfo,\gr,\GradientLoaded,\gsFKrbK@o,\gsj,\gsOX}
% \DoNotIndex{\hbadness,\hfuzz,\HLEmphasize,\HLMacro,\HLMacro@i}
% \DoNotIndex{\HLReverse,\HLReverse@i,\hqcu,\HqY}
% \DoNotIndex{\I,\i,\ifx,\input,\Ir,\IU}
% \DoNotIndex{\j,\jl,\JT,\JVodH}
% \DoNotIndex{\K,\k,\kfSlL}
% \DoNotIndex{\L,\let}
% \DoNotIndex{\message,\mHNa,\mIU}
% \DoNotIndex{\N,\nB,\newcmykcolor,\newdimen,\newif,\nW}
% \DoNotIndex{\O,\oCDJDo,\ocQhVI,\OnlyDescription,\oRKJ}
% \DoNotIndex{\P,\p,\ProvidesPackage,\psframe,\pslinewidth,\psset}
% \DoNotIndex{\PstAtCode,\PSTricksLoaded}
% \DoNotIndex{\q,\Qr,\qssRXq,\qu,\qXjFQp,\qYL}
% \DoNotIndex{\R,\r,\RecordChanges,\relax,\RlaYI,\rN,\Rp,\rp,\RPDXNn,\rput}
% \DoNotIndex{\S,\scalebox,\SgY,\SideBySide@Example,\SideBySideExample}
% \DoNotIndex{\SgY,\sk,\Sp,\space,\sZb}
% \DoNotIndex{\T,\the,\tw@}
% \DoNotIndex{\u,\UiSWGEf@,\uJi,\usepackage,\uVQdMM,\UYj}
% \DoNotIndex{\VerbatimEnvironment,\VerbatimInput,\VrC@}
% \DoNotIndex{\WhZ,\WjKCYb,\WNs}
% \DoNotIndex{\XkN,\XW}
% \DoNotIndex{\Z,\ZCM,\Ze}
% \DoNotIndex{\addtocounter,\advance,\alph,\arabic,\AtBeginDocument,\AtEndDocument}
% \DoNotIndex{\AtEndOfPackage,\begingroup,\bfseries,\bgroup,\box,\csname}
% \DoNotIndex{\else,\endcsname,\endgroup,\endinput,\expandafter,\fi}
% \DoNotIndex{\TeX,\z@,\p@,\@one,\xdef,\thr@@,\string,\sixt@@n,\reset,\or,\multiply,\repeat,\RequirePackage}
% \DoNotIndex{\@cclvi,\@ne,\@ehpa,\@nil,\copy,\dp,\global,\hbox,\hss,\ht,\ifodd,\ifdim,\ifcase,\kern}
% \DoNotIndex{\chardef,\loop,\leavevmode,\ifnum,\lower}
% \setcounter{IndexColumns}{2}
%
%\iffalse
%<*!prolog>
\def\pstslpefileversion{1.31}
\def\pstslpefiledate{2011/10/25}
%</!prolog>
%\fi
%
% \title{\textsf{pst-slpe} package \\ version \pstslpefileversion}
% \author{Martin Giese\footnote{\texttt{giese@ira.uka.de}} and 
%         Herbert Vo\ss\footnote{\texttt{hvoss@tug.org}}}
% \date{\pstslpefiledate}
% \maketitle
%
%\section{Introduction}
%As of the 97 release, PSTricks contains the |pst-grad| 
%package, which provides a gradient fill style for arbitrary shapes.
%Although it often produces nice results, it has a number of
%deficiencies:
%\begin{enumerate}
%\item It is not possible to go from a colour $A$ to $B$ to $C$,
%etc. The most evident application of such a multi-colour gradient are
%of course rainbow effects.  But they can also be useful in informative 
%contexts, eg to identify modes of operation in a scale of values
%(normal/danger/overload).
%\item Colours are interpolated linearly in the RGB space.  This is
%often OK, but when you want to go from red $(1,0,0)$ to green
%$(0,1,0)$, it looks much better to get there via yellow $(1,1,0)$ than 
%via brown $(0.5,0.5,0)$.  The point is, that to get from one saturated
%colour to another, the colours on the way should also be saturated to
%produce an optically pleasing result.
%\item |pst-grad| is limited to {\em linear} gradients, ie~there
%is a (possibly rotated) rectilinear coordinate system, such that the
%colour at every point depends only on the $x$ coordinate of the
%point.  In particular, there is no way to get circular patterns.
%\end{enumerate}
%|pst-slpe| solves {\em all} of the mentioned
%problems in {\em one} package.  
%
%Problems 1.~is addressed by permitting the user to specify an
%arbitrary number of colours, along with the points at which these are
%to be reached.  A special form of each of the fill styles is provided,
%which just needs two colours as parameters, and goes from one to the
%other.  This makes the fill styles easier to use in that simple case.
%
%Problem 2.~is solved by interpolating in the hue-saturation-value
%colour space.  Conversion between RGB and HSV is done behind the
%scenes.  The user specifies colours in RGB.
%
%Finally, |pst-slpe| provides {\em concentric} and {\em radial}
%gradients.  What these mean is best explained with a polar coordinate
%system:  In a concentric pattern, the colour of a point depends on the
%radius coordinate, while in a radial pattern, it depends on the angle
%coordinate.  
%
%As a special bonus, the PostScript part of |pst-slpe| is somewhat
%optimized for speed.  In |ghostscript|, rendering is about 30\% faster 
%than with |pst-grad|.
%\medskip
%
%For most of these problems, solutions have been posted in the
%appropriate \TeX\ newsgroup over the years.  |pst-slpe| has however
%been developed independently from these proposals.  It is based on 
%the original PSTricks 0.93 |gradient| code, most of which has been 
%changed or replaced.  The
%author is indebted to Denis Girou, whose encouragement triggered the
%process of making this a shipable package instead of a private
%experiment.
%\medskip
%
%The new fill styles and the
%graphics parameters provided to use them are described in
%section 2 of this document.  Section 3, if present, documents the 
%implementation consisting of a generic \TeX\ file and a PostScript
%header for the |dvi|-to-PostScript converter.  You can get section 3
%by calling \LaTeX\ as follows on most relevant systems:
%\begin{verbatim}
%latex '\AtBeginDocument{\AlsoImplementation}%\iffalse -*-mode:Latex;tex-command:"latex *;dvips pst-slpe -o"-*- \fi
%\iffalse
%
% Copyright 1998-2008 Martin Giese, mgiese@risc.uni-linz.ac.at
%                     Herbert Voss (using xkeyval,\psBall,fading)
% 
%% This program can be redistributed and/or modified under the terms
%% of the LaTeX Project Public License Distributed from CTAN archives
%% in directory macros/latex/base/lppl.txt.
%%
%\fi
% \changes{v1.31}{2011/10/25}{ fix bug in psball (hv)}
% \changes{v1.3}{2008/09/20}{ fading option (hv)}
% \changes{v1.2}{2008/06/19}{ \textbackslash psBall (hv)}
% \changes{v1.1}{2006/06/19}{%
%	using the extended pst-xkey instead of the old pst-key package; 
%	creating a dtx file (hv)}
% \changes{v1.0}{2005/03/05}{More compatible to the other PStricks 
% packages. (RN)}
%
%
% \DoNotIndex{\!,\",\#,\$,\%,\&,\',\(,\+,\*,\,,\-,\.,\/,\:,\;,\<,\=,\>,\?}
% \DoNotIndex{\@,\@B,\@K,\@cTq,\@f,\@fPl,\@ifnextchar,\@nameuse,\@oVk}
% \DoNotIndex{\[,\\,\],\^,\_,\ }
% \DoNotIndex{\^,\\^,\\\^,$\^$,$\\^$,$\\^$}
% \DoNotIndex{\0,\2,\4,\5,\6,\7,\8,}
% \DoNotIndex{\A,\a}
% \DoNotIndex{\B,\b,\Bc,\begin,\Bq,\Bqc}
% \DoNotIndex{\C,\c,\catcode,\cJA,\CodelineIndex,\csname}
% \DoNotIndex{\D,\def,\define@key,\Df,\divide,\DocInput,\documentclass,\pst@addfams}
% \DoNotIndex{\eCN,\edef,\else,\eHd,\eMcj,\EnableCrossrefs,\end,\endcsname}
% \DoNotIndex{\endCenterExample,\endExample,\endinput,\endpsclip}
% \DoNotIndex{\PrintIndex,\PrintChanges,\ProvidesFile}
% \DoNotIndex{\endpspicture,\endSideBySideExample,\Example}
% \DoNotIndex{\F,\f,\FdUrr,\fi,\filedate,\fileversion,\FV@Environment}
% \DoNotIndex{\FV@UseKeyValues,\FV@XRightMargin,\FVB@Example,\fvset}
% \DoNotIndex{\G,\g,\GetFileInfo,\gr,\GradientLoaded,\gsFKrbK@o,\gsj,\gsOX}
% \DoNotIndex{\hbadness,\hfuzz,\HLEmphasize,\HLMacro,\HLMacro@i}
% \DoNotIndex{\HLReverse,\HLReverse@i,\hqcu,\HqY}
% \DoNotIndex{\I,\i,\ifx,\input,\Ir,\IU}
% \DoNotIndex{\j,\jl,\JT,\JVodH}
% \DoNotIndex{\K,\k,\kfSlL}
% \DoNotIndex{\L,\let}
% \DoNotIndex{\message,\mHNa,\mIU}
% \DoNotIndex{\N,\nB,\newcmykcolor,\newdimen,\newif,\nW}
% \DoNotIndex{\O,\oCDJDo,\ocQhVI,\OnlyDescription,\oRKJ}
% \DoNotIndex{\P,\p,\ProvidesPackage,\psframe,\pslinewidth,\psset}
% \DoNotIndex{\PstAtCode,\PSTricksLoaded}
% \DoNotIndex{\q,\Qr,\qssRXq,\qu,\qXjFQp,\qYL}
% \DoNotIndex{\R,\r,\RecordChanges,\relax,\RlaYI,\rN,\Rp,\rp,\RPDXNn,\rput}
% \DoNotIndex{\S,\scalebox,\SgY,\SideBySide@Example,\SideBySideExample}
% \DoNotIndex{\SgY,\sk,\Sp,\space,\sZb}
% \DoNotIndex{\T,\the,\tw@}
% \DoNotIndex{\u,\UiSWGEf@,\uJi,\usepackage,\uVQdMM,\UYj}
% \DoNotIndex{\VerbatimEnvironment,\VerbatimInput,\VrC@}
% \DoNotIndex{\WhZ,\WjKCYb,\WNs}
% \DoNotIndex{\XkN,\XW}
% \DoNotIndex{\Z,\ZCM,\Ze}
% \DoNotIndex{\addtocounter,\advance,\alph,\arabic,\AtBeginDocument,\AtEndDocument}
% \DoNotIndex{\AtEndOfPackage,\begingroup,\bfseries,\bgroup,\box,\csname}
% \DoNotIndex{\else,\endcsname,\endgroup,\endinput,\expandafter,\fi}
% \DoNotIndex{\TeX,\z@,\p@,\@one,\xdef,\thr@@,\string,\sixt@@n,\reset,\or,\multiply,\repeat,\RequirePackage}
% \DoNotIndex{\@cclvi,\@ne,\@ehpa,\@nil,\copy,\dp,\global,\hbox,\hss,\ht,\ifodd,\ifdim,\ifcase,\kern}
% \DoNotIndex{\chardef,\loop,\leavevmode,\ifnum,\lower}
% \setcounter{IndexColumns}{2}
%
%\iffalse
%<*!prolog>
\def\pstslpefileversion{1.31}
\def\pstslpefiledate{2011/10/25}
%</!prolog>
%\fi
%
% \title{\textsf{pst-slpe} package \\ version \pstslpefileversion}
% \author{Martin Giese\footnote{\texttt{giese@ira.uka.de}} and 
%         Herbert Vo\ss\footnote{\texttt{hvoss@tug.org}}}
% \date{\pstslpefiledate}
% \maketitle
%
%\section{Introduction}
%As of the 97 release, PSTricks contains the |pst-grad| 
%package, which provides a gradient fill style for arbitrary shapes.
%Although it often produces nice results, it has a number of
%deficiencies:
%\begin{enumerate}
%\item It is not possible to go from a colour $A$ to $B$ to $C$,
%etc. The most evident application of such a multi-colour gradient are
%of course rainbow effects.  But they can also be useful in informative 
%contexts, eg to identify modes of operation in a scale of values
%(normal/danger/overload).
%\item Colours are interpolated linearly in the RGB space.  This is
%often OK, but when you want to go from red $(1,0,0)$ to green
%$(0,1,0)$, it looks much better to get there via yellow $(1,1,0)$ than 
%via brown $(0.5,0.5,0)$.  The point is, that to get from one saturated
%colour to another, the colours on the way should also be saturated to
%produce an optically pleasing result.
%\item |pst-grad| is limited to {\em linear} gradients, ie~there
%is a (possibly rotated) rectilinear coordinate system, such that the
%colour at every point depends only on the $x$ coordinate of the
%point.  In particular, there is no way to get circular patterns.
%\end{enumerate}
%|pst-slpe| solves {\em all} of the mentioned
%problems in {\em one} package.  
%
%Problems 1.~is addressed by permitting the user to specify an
%arbitrary number of colours, along with the points at which these are
%to be reached.  A special form of each of the fill styles is provided,
%which just needs two colours as parameters, and goes from one to the
%other.  This makes the fill styles easier to use in that simple case.
%
%Problem 2.~is solved by interpolating in the hue-saturation-value
%colour space.  Conversion between RGB and HSV is done behind the
%scenes.  The user specifies colours in RGB.
%
%Finally, |pst-slpe| provides {\em concentric} and {\em radial}
%gradients.  What these mean is best explained with a polar coordinate
%system:  In a concentric pattern, the colour of a point depends on the
%radius coordinate, while in a radial pattern, it depends on the angle
%coordinate.  
%
%As a special bonus, the PostScript part of |pst-slpe| is somewhat
%optimized for speed.  In |ghostscript|, rendering is about 30\% faster 
%than with |pst-grad|.
%\medskip
%
%For most of these problems, solutions have been posted in the
%appropriate \TeX\ newsgroup over the years.  |pst-slpe| has however
%been developed independently from these proposals.  It is based on 
%the original PSTricks 0.93 |gradient| code, most of which has been 
%changed or replaced.  The
%author is indebted to Denis Girou, whose encouragement triggered the
%process of making this a shipable package instead of a private
%experiment.
%\medskip
%
%The new fill styles and the
%graphics parameters provided to use them are described in
%section 2 of this document.  Section 3, if present, documents the 
%implementation consisting of a generic \TeX\ file and a PostScript
%header for the |dvi|-to-PostScript converter.  You can get section 3
%by calling \LaTeX\ as follows on most relevant systems:
%\begin{verbatim}
%latex '\AtBeginDocument{\AlsoImplementation}%\iffalse -*-mode:Latex;tex-command:"latex *;dvips pst-slpe -o"-*- \fi
%\iffalse
%
% Copyright 1998-2008 Martin Giese, mgiese@risc.uni-linz.ac.at
%                     Herbert Voss (using xkeyval,\psBall,fading)
% 
%% This program can be redistributed and/or modified under the terms
%% of the LaTeX Project Public License Distributed from CTAN archives
%% in directory macros/latex/base/lppl.txt.
%%
%\fi
% \changes{v1.31}{2011/10/25}{ fix bug in psball (hv)}
% \changes{v1.3}{2008/09/20}{ fading option (hv)}
% \changes{v1.2}{2008/06/19}{ \textbackslash psBall (hv)}
% \changes{v1.1}{2006/06/19}{%
%	using the extended pst-xkey instead of the old pst-key package; 
%	creating a dtx file (hv)}
% \changes{v1.0}{2005/03/05}{More compatible to the other PStricks 
% packages. (RN)}
%
%
% \DoNotIndex{\!,\",\#,\$,\%,\&,\',\(,\+,\*,\,,\-,\.,\/,\:,\;,\<,\=,\>,\?}
% \DoNotIndex{\@,\@B,\@K,\@cTq,\@f,\@fPl,\@ifnextchar,\@nameuse,\@oVk}
% \DoNotIndex{\[,\\,\],\^,\_,\ }
% \DoNotIndex{\^,\\^,\\\^,$\^$,$\\^$,$\\^$}
% \DoNotIndex{\0,\2,\4,\5,\6,\7,\8,}
% \DoNotIndex{\A,\a}
% \DoNotIndex{\B,\b,\Bc,\begin,\Bq,\Bqc}
% \DoNotIndex{\C,\c,\catcode,\cJA,\CodelineIndex,\csname}
% \DoNotIndex{\D,\def,\define@key,\Df,\divide,\DocInput,\documentclass,\pst@addfams}
% \DoNotIndex{\eCN,\edef,\else,\eHd,\eMcj,\EnableCrossrefs,\end,\endcsname}
% \DoNotIndex{\endCenterExample,\endExample,\endinput,\endpsclip}
% \DoNotIndex{\PrintIndex,\PrintChanges,\ProvidesFile}
% \DoNotIndex{\endpspicture,\endSideBySideExample,\Example}
% \DoNotIndex{\F,\f,\FdUrr,\fi,\filedate,\fileversion,\FV@Environment}
% \DoNotIndex{\FV@UseKeyValues,\FV@XRightMargin,\FVB@Example,\fvset}
% \DoNotIndex{\G,\g,\GetFileInfo,\gr,\GradientLoaded,\gsFKrbK@o,\gsj,\gsOX}
% \DoNotIndex{\hbadness,\hfuzz,\HLEmphasize,\HLMacro,\HLMacro@i}
% \DoNotIndex{\HLReverse,\HLReverse@i,\hqcu,\HqY}
% \DoNotIndex{\I,\i,\ifx,\input,\Ir,\IU}
% \DoNotIndex{\j,\jl,\JT,\JVodH}
% \DoNotIndex{\K,\k,\kfSlL}
% \DoNotIndex{\L,\let}
% \DoNotIndex{\message,\mHNa,\mIU}
% \DoNotIndex{\N,\nB,\newcmykcolor,\newdimen,\newif,\nW}
% \DoNotIndex{\O,\oCDJDo,\ocQhVI,\OnlyDescription,\oRKJ}
% \DoNotIndex{\P,\p,\ProvidesPackage,\psframe,\pslinewidth,\psset}
% \DoNotIndex{\PstAtCode,\PSTricksLoaded}
% \DoNotIndex{\q,\Qr,\qssRXq,\qu,\qXjFQp,\qYL}
% \DoNotIndex{\R,\r,\RecordChanges,\relax,\RlaYI,\rN,\Rp,\rp,\RPDXNn,\rput}
% \DoNotIndex{\S,\scalebox,\SgY,\SideBySide@Example,\SideBySideExample}
% \DoNotIndex{\SgY,\sk,\Sp,\space,\sZb}
% \DoNotIndex{\T,\the,\tw@}
% \DoNotIndex{\u,\UiSWGEf@,\uJi,\usepackage,\uVQdMM,\UYj}
% \DoNotIndex{\VerbatimEnvironment,\VerbatimInput,\VrC@}
% \DoNotIndex{\WhZ,\WjKCYb,\WNs}
% \DoNotIndex{\XkN,\XW}
% \DoNotIndex{\Z,\ZCM,\Ze}
% \DoNotIndex{\addtocounter,\advance,\alph,\arabic,\AtBeginDocument,\AtEndDocument}
% \DoNotIndex{\AtEndOfPackage,\begingroup,\bfseries,\bgroup,\box,\csname}
% \DoNotIndex{\else,\endcsname,\endgroup,\endinput,\expandafter,\fi}
% \DoNotIndex{\TeX,\z@,\p@,\@one,\xdef,\thr@@,\string,\sixt@@n,\reset,\or,\multiply,\repeat,\RequirePackage}
% \DoNotIndex{\@cclvi,\@ne,\@ehpa,\@nil,\copy,\dp,\global,\hbox,\hss,\ht,\ifodd,\ifdim,\ifcase,\kern}
% \DoNotIndex{\chardef,\loop,\leavevmode,\ifnum,\lower}
% \setcounter{IndexColumns}{2}
%
%\iffalse
%<*!prolog>
\def\pstslpefileversion{1.31}
\def\pstslpefiledate{2011/10/25}
%</!prolog>
%\fi
%
% \title{\textsf{pst-slpe} package \\ version \pstslpefileversion}
% \author{Martin Giese\footnote{\texttt{giese@ira.uka.de}} and 
%         Herbert Vo\ss\footnote{\texttt{hvoss@tug.org}}}
% \date{\pstslpefiledate}
% \maketitle
%
%\section{Introduction}
%As of the 97 release, PSTricks contains the |pst-grad| 
%package, which provides a gradient fill style for arbitrary shapes.
%Although it often produces nice results, it has a number of
%deficiencies:
%\begin{enumerate}
%\item It is not possible to go from a colour $A$ to $B$ to $C$,
%etc. The most evident application of such a multi-colour gradient are
%of course rainbow effects.  But they can also be useful in informative 
%contexts, eg to identify modes of operation in a scale of values
%(normal/danger/overload).
%\item Colours are interpolated linearly in the RGB space.  This is
%often OK, but when you want to go from red $(1,0,0)$ to green
%$(0,1,0)$, it looks much better to get there via yellow $(1,1,0)$ than 
%via brown $(0.5,0.5,0)$.  The point is, that to get from one saturated
%colour to another, the colours on the way should also be saturated to
%produce an optically pleasing result.
%\item |pst-grad| is limited to {\em linear} gradients, ie~there
%is a (possibly rotated) rectilinear coordinate system, such that the
%colour at every point depends only on the $x$ coordinate of the
%point.  In particular, there is no way to get circular patterns.
%\end{enumerate}
%|pst-slpe| solves {\em all} of the mentioned
%problems in {\em one} package.  
%
%Problems 1.~is addressed by permitting the user to specify an
%arbitrary number of colours, along with the points at which these are
%to be reached.  A special form of each of the fill styles is provided,
%which just needs two colours as parameters, and goes from one to the
%other.  This makes the fill styles easier to use in that simple case.
%
%Problem 2.~is solved by interpolating in the hue-saturation-value
%colour space.  Conversion between RGB and HSV is done behind the
%scenes.  The user specifies colours in RGB.
%
%Finally, |pst-slpe| provides {\em concentric} and {\em radial}
%gradients.  What these mean is best explained with a polar coordinate
%system:  In a concentric pattern, the colour of a point depends on the
%radius coordinate, while in a radial pattern, it depends on the angle
%coordinate.  
%
%As a special bonus, the PostScript part of |pst-slpe| is somewhat
%optimized for speed.  In |ghostscript|, rendering is about 30\% faster 
%than with |pst-grad|.
%\medskip
%
%For most of these problems, solutions have been posted in the
%appropriate \TeX\ newsgroup over the years.  |pst-slpe| has however
%been developed independently from these proposals.  It is based on 
%the original PSTricks 0.93 |gradient| code, most of which has been 
%changed or replaced.  The
%author is indebted to Denis Girou, whose encouragement triggered the
%process of making this a shipable package instead of a private
%experiment.
%\medskip
%
%The new fill styles and the
%graphics parameters provided to use them are described in
%section 2 of this document.  Section 3, if present, documents the 
%implementation consisting of a generic \TeX\ file and a PostScript
%header for the |dvi|-to-PostScript converter.  You can get section 3
%by calling \LaTeX\ as follows on most relevant systems:
%\begin{verbatim}
%latex '\AtBeginDocument{\AlsoImplementation}\input{pst-slpe.dtx}'
%\end{verbatim}
%\section{Package Usage}
% To use |pst-slpe|, you have to say
% \begin{verbatim}
%   \usepackage{pst-slpe}
% \end{verbatim}
% in the document prologue for \LaTeX, and 
% \begin{verbatim}
%   \input pst-slpe.tex
% \end{verbatim}
% in ``plain'' \TeX.
%
% \section{New macro and fill styles}
% \DescribeMacro{\psBall}
%    It takes the (optional) coordinates of the ball center, the color
%    and the radius as parameter and uses |\pscircle| for painting
%    the bullet. 
%
%    \vspace{1cm}
%    \psBall{black}{2ex}
%    \psBall(1,0){blue}{3ex}
%    \psBall(2.5,0){red}{4ex}
%    \psBall(4,0){green!50!blue!60}{5ex}
%
%    \vspace{1cm}
% \begin{verbatim}
%    \psBall{black}{2ex}
%    \psBall(1,0){blue}{3ex}
%    \psBall(2.5,0){red}{4ex}
%    \psBall(4,0){green!50!blue!60}{5ex}
% \end{verbatim}
%
%    The predinied options can be overwritten in the usual way:
%
%    \vspace{1cm}
%    \psBall{black}{2ex}
%    \psBall[sloperadius=10pt](1,0){blue}{3ex}
%    \psBall(2.5,0){red}{4ex}
%    \psBall[slopebegin=red](4,0){green!50!blue!60}{5ex}
%
%    \vspace{1cm}
% \begin{verbatim}
%    \psBall{black}{2ex}
%    \psBall[sloperadius=10pt](1,0){blue}{3ex}
%    \psBall(2.5,0){red}{4ex}
%    \psBall[slopebegin=red](4,0){green!50!blue!60}{5ex}
% \end{verbatim}
%
% \DescribeMacro{slope}
% \DescribeMacro{slopes}
% \DescribeMacro{ccslope}
% \DescribeMacro{ccslopes}
% \DescribeMacro{radslope}
% \DescribeMacro{radslopes}
% |pst-slpe| provides six new fill styles called |slope|, |slopes|,
% |ccslope|, |ccslopes|, |radslope| and |radslopes|.  These obviously
% come in pairs: The $\ldots$|slope|-styles are simplified versions of
% the general $\ldots$|slopes|-styles.\footnote{By the way, I use slope
% as a synonym for gradient.  It sounds less pretentious and avoids
% name clashes.}  The |cc|$\ldots$ styles paint concentric patterns,
% and the |rad|$\ldots$ styles do radial ones.  
%
% Here is a little
% overview of what they look like:
% \newcommand{\st}{$\vcenter to30pt{}$}
% \begin{quote}\LARGE
%  \begin{tabular}{cc}
%   \psframebox[fillstyle=slope]{\st|slope|} &\qquad
%   \psframebox[fillstyle=slopes]{\st|slopes|} \\[2ex]
%   \psframebox[fillstyle=ccslope]{\st|ccslope|} &\qquad
%   \psframebox[fillstyle=ccslopes]{\st|ccslopes|} \\[2ex]
%   \psframebox[fillstyle=radslope]{\st|radslope|} &\qquad
%   \psframebox[fillstyle=radslopes]{\st|radslopes|} \\[2ex]
%  \end{tabular}
% \end{quote}
% These examples were produced by saying simply
% \begin{verbatim}
%   \psframebox[fillstyle=slope]{...}
% \end{verbatim}
% etc.~without setting any further graphics parameters.  The package
% provides a number of parameters that can be used to control
% the way these patterns
% are painted.
% \medskip
%
% \DescribeMacro{slopebegin}
% \DescribeMacro{slopeend}
% The graphics parameters |slopebegin| and |slopeend| set the colours
% between which the three $\ldots$|slope| styles should interpolate.
% Eg,
% \begin{verbatim}
%   \psframebox[fillstyle=slope,slopebegin=red,slopeend=green]{...}
% \end{verbatim}
% produces:
% \begin{quote}\Large
%  \psframebox[fillstyle=slope,slopebegin=red,slopeend=green]{\st slopes!}
% \end{quote}
% The same settings of |slopebegin| and |slopeend| for the |ccslope|
% and |radslope| fillstyles produce
% \begin{quote}\Large
%  \psframebox[fillstyle=ccslope,slopebegin=red,slopeend=green]{\st slopes!}
%   \quad{\normalsize resp.}\quad
%  \psframebox[fillstyle=radslope,slopebegin=red,slopeend=green]{\st slopes!}
% \end{quote}
% The default settings go from a greenish yellow to pure blue.
% \medskip
%
% \DescribeMacro{slopecolors}
% If you want to interpolate between more than two colours, you have
% to use the $\ldots$|slopes| styles, which are controlled by the 
% |slopecolors| parameter instead of |slopebegin| and |slopeend|.  The 
% idea is to specify the colour to use at certain points `on the
% way'.  To fill a shape with |slopes|, imagine a linear scale 
% from its left edge to its right edge.  The left edge must lie at
% coordinate 0.  Pick an arbitrary value for the right edge, say 23.
% Now you want to get light yellow at the left edge, a pastel green at $17/23$ 
% of the way and dark cyan at the right edge, like this:
% \begin{quote}\psset{unit=0.45cm}
%  \begin{pspicture}(-1,0)(24,6)
%   \pscustom[fillstyle=slopes,
% slopecolors=0 1 1 .9  17 .5 1 .5  23 0 0.5 0.5  3]{
%    \psccurve(0,2.5)(12,3.5)(20,4)(23,2)(17,2.5)}
%   \psaxes(0,5)(-0.01,5)(23.01,5)
%   \psline(0,5)(0,1)
%   \psline(17,5)(17,1)
%   \psline(23,5)(23,1)
%  \end{pspicture}
% \end{quote}
% The RGB values for the three colours are $(1,1,0.9)$, $(0.5,1,0.5)$
% and $(0,0.5,0.5)$.  The value for the |slopecolors| parameter is a list
% of `colour infos' followed by the number of `colour infos'. 
% Each `colour info' consists
% of the coordinate value where a colour is to be specified, followed by
% the RGB values of that colour.  All these values are separated by
% white space.  The correct setting for the example is thus:
% \begin{verbatim}
%   slopecolors=0 1 1 .9   17 .5 1 .5   23 0 .5 .5   3
% \end{verbatim}
% For |ccslopes|, specify the colours from the center outward.  
% For |radslopes| (with no rotation specified), 0 represents the ray
% going `eastward'.  Specify the colours anti-clockwise.  If you want a
% smooth gradient at the beginning and starting ray of |radslopes|, you
% should pick the first and last colours identical.
%
% Please note, that the |slopecolors| parameter is not subject to any
% parsing on the \TeX\ side.  If you forget a number or specify the wrong
% number of segments, the PostScript interpreter will probably crash.
%
% The default value for |slopecolors| specifies a rainbow.
%
% \medskip
%
% \DescribeMacro{slopesteps}
% The parameter |slopesteps| controls the number of distinct colour steps
% rendered.  Higher values for this parameter result in better quality
% but proportionally slower rendering.  Eg, setting
% |slopesteps| to 5 with the |slope| fill style results in
% \begin{quote}\Large
%  \psframebox[fillstyle=slope,slopesteps=5]{\st slopes!}
% \end{quote}
%
% The default value is 100, which
% suffices for most purposes.  Remember that the number of distinct colours
% reproducible by a given device is limited.  Pushing |slopesteps| to
% high will result only in loss of performance at no gain in quality.
% \medskip
%
% \DescribeMacro{slopeangle}
% The |slope(s)| and |radslope(s)| patterns may be rotated.  As usual,
% the angles are given anti-clockwise.  Eg, an angle of 30 degrees
% gives
% \begin{quote}\Large\psset{slopeangle=30}
%  \psframebox[fillstyle=slope]{\st slopes!}
%   \quad{\normalsize and}\quad
%  \psframebox[fillstyle=radslope]{\st slopes!}
% \end{quote}
% with the |slope| and |radslope| fillstyles.
% \medskip
%
% \DescribeMacro{slopecenter}
% For the |cc|$\ldots$ and |rad|$\ldots$ styles, it is possible to
% set the center of the pattern.  The |slopecenter| parameter is set to
% the coordinates of that center relative to the bounding box of the
% current path.  The following effect:
% \begin{quote}\psset{unit=0.45cm}
%  \begin{pspicture}(-1,-1)(24,5)
%   \pscustom[fillstyle=radslope,slopecenter=0.2 0.4]{
%    \pspolygon(0,2.5)(12,2.5)(20,4)(23,2)(17,2.5)(3,0)}
%   \psaxes[axesstyle=frame,Dx=0.1,dx=2.2999,Dy=0.2,dy=0.7999](0,0)(23,4) 
%   \psline(4.6,0)(4.6,4)
%   \psline(0,1.6)(23,1.6)
%  \end{pspicture}
% \end{quote}
%  was achieved with 
% \begin{verbatim}
%    fillstyle=radslope,slopecenter=0.2 0.4
% \end{verbatim}
% The default value for |slopecenter| is |0.5 0.5|, which is the
% center for symmetrical shapes.  Note that this parameter is not
% parsed by \TeX, so setting it to anything else than two numbers 
% between 0 and 1 might crash the PostScript interpreter.
% \medskip
%
% \DescribeMacro{sloperadius}
% Normally, the |cc|$\ldots$ and |rad|$\ldots$ styles distribute the
% given colours so that the center is painted in the first colour given,
% and the points of the shape furthest from the center are painted in 
% the last colour.  In other words the maximum radius to which the
% |slopecolors| parameter refers is the maximum distance from the
% center (defined by |slopecenter|) to any point on the periphery
% of the shape.  This radius can be explicitly set with |sloperadius|.
% Eg, setting |sloperadius=0.5cm| gives
% \begin{quote}\Large\psset{sloperadius=0.5cm}
%  \psframebox[fillstyle=ccslope]{\st slopes!}
% \end{quote}
% Any point further from the center than the given |sloperadius| is
% painted with the last colour in |slopeclours|, resp.~|slopeend|.
%
% The default value for |sloperadius| is 0, which invokes the default
% behaviour of automatically calculating the radius.
%
% \DescribeMacro{fading}
% \DescribeMacro{startfading}
% \DescribeMacro{endfading}
% The optional boolean keyword |fading| allows a transparency effect of
% the filled area, starting with the opacity value |startfading| and
% ending with the value of |endfading|. Both values must be of the
% intervall [0\ldots1], with 0 for total opacity and 1 for no
% opacity. The values are preset by 0 and 1.
%
% Here is a little
% overview of what they look like:
% \begin{quote}\LARGE\color{white}
%  \begin{tabular}{cc}
%   \psframe*(-0.3,-0.25)(3,20pt)\psframebox[fading,fillstyle=slope]{\st|  slope  |} &\qquad
%   \psframe*(-0.3,-0.25)(3,20pt)\psframebox[fading,fillstyle=slopes]{\st| slopes |} \\[2ex]
%   \psframe*(-0.3,-0.25)(3,20pt)\psframebox[fading,fillstyle=ccslope]{\st|ccslope |} &\qquad
%   \psframe*(-0.3,-0.25)(3,20pt)\psframebox[fading,fillstyle=ccslopes]{\st|ccslopes|} \\[2ex]
%   \psframe*(-0.3,-0.25)(3,20pt)\psframebox[fading,fillstyle=radslope]{\st|radslope|} &\qquad
%   \psframe*(-0.3,-0.25)(3,20pt)\psframebox[fading,fillstyle=radslopes]{\st|radslopes|} \\[2ex]
%  \end{tabular}
% \end{quote}
% \color{black}
%
% These examples were produced by saying simply
% \begin{verbatim}
%   \psframebox[fading,fillstyle=...]{...}
% \end{verbatim}
%
% \begin{quote}\LARGE\color{white}
% \psset{fading,startfading=0.3,endfading=0.8}
%  \begin{tabular}{cc}
%   \psframe*(-0.3,-0.25)(3,20pt)\psframebox[fillstyle=slope]{\st|  slope  |} &\qquad
%   \psframe*(-0.3,-0.25)(3,20pt)\psframebox[fillstyle=slopes]{\st| slopes |} \\[2ex]
%   \psframe*(-0.3,-0.25)(3,20pt)\psframebox[fillstyle=ccslope]{\st|ccslope |} &\qquad
%   \psframe*(-0.3,-0.25)(3,20pt)\psframebox[fillstyle=ccslopes]{\st|ccslopes|} \\[2ex]
%   \psframe*(-0.3,-0.25)(3,20pt)\psframebox[fillstyle=radslope]{\st|radslope|} &\qquad
%   \psframe*(-0.3,-0.25)(3,20pt)\psframebox[fillstyle=radslopes]{\st|radslopes|} \\[2ex]
%  \end{tabular}
% \end{quote}
% \color{black}
%
% These examples were produced by saying simply
% \begin{verbatim}
%   \psframebox[fading,startfading=0.3,endfading=0.8,fillstyle=...]{...}
% \end{verbatim}
%
% \StopEventually{}
%
%\section{The Code}
% \subsection{Producing the documentation}
%
%    A short driver is provided that can be extracted if necessary by
%    the \textsc{docstrip} program provided with \LaTeXe.
%    \begin{macrocode}
%<*driver>
\NeedsTeXFormat{LaTeX2e}
\documentclass{ltxdoc}
\usepackage{pst-slpe}
\usepackage{pst-plot}
\DisableCrossrefs
\MakeShortVerb{\|}
\newcommand\Lopt[1]{\textsf{#1}}
\newcommand\file[1]{\texttt{#1}}
\AtEndDocument{
\PrintChanges
\PrintIndex
}
%\OnlyDescription
\begin{document}
\DocInput{pst-slpe.dtx}
\end{document}
%</driver>
%    \end{macrocode}
%
% \subsection{The \file{pst-slpe.sty} file}
%    The \file{pst-slpe.sty} file is very simple.  It just loads 
%    the generic \file{pst-slpe.tex} file. 
%    \begin{macrocode}
%<*stylefile>
\RequirePackage{pstricks}
\ProvidesPackage{pst-slpe}[2005/03/05 package wrapper for `pst-slpe.tex']
\input{pst-slpe.tex}
\ProvidesFile{pst-slpe.tex}
  [\pstslpefiledate\space v\pstslpefileversion\space 
    `pst-slpe' (mg,hv)]
\IfFileExists{pst-slpe.pro}{%
  \ProvidesFile{pst-slpe.pro}
    [2008/06/19 v. 0.01,  PostScript prologue file (hv)]
    \@addtofilelist{pst-slpe.pro}}{}%
%</stylefile>
%    \end{macrocode}
%
% \subsection{The \file{pst-slpe.tex} file}
%    \file{pst-slpe.tex} contains the \TeX-side of things.  We begin
%    by identifying ourselves and setting things up, the same as in 
%    other PSTricks packages.
%    \begin{macrocode}
%<*texfile>
\message{ v\pstslpefileversion, \pstslpefiledate}
\csname PstSlopeLoaded\endcsname
\let\PstSlopeLoaded\endinput
\ifx\PSTricksLoaded\endinput\else
  \def\next{\input pstricks.tex }\expandafter\next
\fi
\ifx\PSTXKeyLoaded\endinput\else\input pst-xkey \fi % --> hv
\edef\TheAtCode{\the\catcode`\@}
\catcode`\@=11
\pst@addfams{pst-slpe}                              % --> hv
\pstheader{pst-slpe.pro}
%    \end{macrocode}
%    \begin{macro}{slopebegin}
%    \begin{macro}{slopeend}
%    \begin{macro}{slopesteps}
%    \begin{macro}{slopeangle}
%
% \subsubsection{New graphics parameters}
%    We now define the various new parameters needed by the |slope|
%    fill styles and install default values.  First come the colours, 
%    ie~graphics parameters |slopebegin| and |slopeend|, followed
%    by the number of steps, |slopesteps|, and the rotation angle, 
%    |slopeangle|.
%    \begin{macrocode}
\newrgbcolor{slopebegin}{0.9 1 0}
\define@key[psset]{pst-slpe}{slopebegin}{\pst@getcolor{#1}\psslopebegin}% --> hv
\psset[pst-slpe]{slopebegin=slopebegin}                                 % --> hv

\newrgbcolor{slopeend}{0 0 1}
\define@key[psset]{pst-slpe}{slopeend}{\pst@getcolor{#1}\psslopeend}% --> hv
\psset[pst-slpe]{slopeend=slopeend}% --> hv

\define@key[psset]{pst-slpe}{slopesteps}{\pst@getint{#1}\psslopesteps}% --> hv
\psset[pst-slpe]{slopesteps=100}% --> hv

\define@key[psset]{pst-slpe}{slopeangle}{\pst@getangle{#1}\psx@slopeangle}% --> hv
\psset[pst-slpe]{slopeangle=0}% --> hv
%    \end{macrocode}
%    \end{macro}
%    \end{macro}
%    \end{macro}
%    \end{macro}
%    \begin{macro}{slopecolors}
%    The value for |slopecolors| is not parsed.  It is directly copied 
%    to the PostScript output.  This is certainly not the way it
%    should be, but it's simple.  The default value is a rainbow from
%    red to magenta.
%    \begin{macrocode}
\define@key[psset]{pst-slpe}{slopecolors}{\def\psx@slopecolors{#1}}% --> hv
\psset[pst-slpe]{slopecolors={% --> hv
0.0 1 0 0		
0.4 0 1 0
0.8 0 0 1
1.0 1 0 1
4}}
%    \end{macrocode}
%    \end{macro}
%    \begin{macro}{slopecenter}
%    The argument to |slopecenter| isn't parsed either.  But there's
%    probably not much that can go wrong with two decimal numbers.
%    \begin{macrocode}
\define@key[psset]{pst-slpe}{slopecenter}{\def\psx@slopecenter{#1}}% --> hv
\psset[pst-slpe]{slopecenter={0.5 0.5}}% --> hv
%    \end{macrocode}
%    \end{macro}
%    \begin{macro}{sloperadius}
%    The default value for |sloperadius| is 0, which makes the
%    PostScript procedure |PatchRadius| determine a value for the radius.
%    \begin{macrocode}
\define@key[psset]{pst-slpe}{sloperadius}{\pst@getlength{#1}\psx@sloperadius}% --> hv
\psset[pst-slpe]{sloperadius=0}% --> hv
%    \end{macrocode}
%    \end{macro}
%    \begin{macro}{fading}
%    The default value for |fading| is false, which is no transparency effect at all.
%    With |fading=true| the package takes the values |startfading| and |endfading|
%    into account for the opacity effect of the filled area.
%    \end{macro}
%    \begin{macrocode}
\define@boolkey[psset]{pst-slpe}[PST@]{fading}[true]{}% --> hv
\psset[pst-slpe]{fading=false}% --> hv
%    \end{macrocode}
%    \begin{macro}{startfading}
%    The relativ number for the starting value (0\ldots1), preset by 0.
%    \end{macro}
%    \begin{macrocode}
\define@key[psset]{pst-slpe}{startfading}{\pst@checknum{#1}\psk@startfading }% --> hv
%    \end{macrocode}
%    \begin{macro}{endfading}
%    The relativ number for the end value (0\ldots1), preset by 1.
%    \end{macro}
%    \begin{macrocode}
\define@key[psset]{pst-slpe}{endfading}{\pst@checknum{#1}\psk@endfading }% --> hv
\psset[pst-slpe]{startfading=0,endfading=1}% --> hv
%    \end{macrocode}
%
% \subsubsection{Fill style macros}
%
%    Now come the fill style definitions that use these parameters.
%    There is one macro for each fill style named |\psfs@|$style$.  
%    PSTricks calls this macro whenever the current path needs to
%    be filled in that style.  The current path should not be 
%    clobbered by the PostScript code output by the macro.
%
%    \begin{macro}{slopes}
%    For the slopes fill style we produce PostScript code that
%    first puts the |slopecolors| parameter onto the stack.  Note that
%    the number of colours listed, which comes last in |slopecolors| is
%    now on the top of the stack.  Next come the |slopesteps| and
%    |slopeangle| parameters.  We switch to the dictionary established
%    by the \file{pst-slop.pro} Prolog and call |SlopesFill|, which
%    does the artwork and takes care to leave the path alone.
%    \begin{macrocode}
\def\psfs@slopes{%
 \addto@pscode{
  \psx@slopecolors\space
  \psslopesteps
  \psx@slopeangle 
  \ifPST@fading \psk@startfading \psk@endfading true \else false \fi
  tx@PstSlopeDict begin SlopesFill end}}
%    \end{macrocode}
%    \end{macro}
%
%    \begin{macro}{slope}
%    The |slope| style uses parameters |slopebegin| and |slopeend|
%    instead of |slopecolors|.  So the produced PostScript uses these
%    parameters to build a stack in |slopecolors| format.  The 
%    |\pst@usecolor| generates PostScript to set the current colour.
%    We can query the RGB values with |currentrgbcolor|.  
%    A |gsave|/|grestore| pair is used to avoid changing the
%    PostScript graphics state.  Once the stack is set up,
%    |SlopesFill| is called as before.
%    \begin{macrocode}
\def\psfs@slope{%
 \addto@pscode{%
  gsave
    0 \pst@usecolor\psslopebegin currentrgbcolor
    1 \pst@usecolor\psslopeend currentrgbcolor
    2
  grestore
  \psslopesteps \psx@slopeangle 
  \ifPST@fading \psk@startfading \psk@endfading true \else false \fi
  tx@PstSlopeDict begin SlopesFill end}}
%    \end{macrocode}
%    \end{macro}
%
%    \begin{macro}{ccslopes}
%    \begin{macro}{ccslope}
%    \begin{macro}{radslopes}
%    The code for the other fill styles is about the same, except for a few
%    parameters more or less and different PostScript procedures called
%    to do the work.
%    \begin{macrocode}
\def\psfs@ccslopes{%
 \addto@pscode{%
  \psx@slopecolors\space  
  \psslopesteps \psx@slopecenter\space \psx@sloperadius\space
  \ifPST@fading \psk@startfading \psk@endfading true \else false \fi
  tx@PstSlopeDict begin CcSlopesFill end}}
\def\psfs@ccslope{%
 \addto@pscode{%
  gsave 0 \pst@usecolor\psslopebegin currentrgbcolor
    1 \pst@usecolor\psslopeend currentrgbcolor
    2 grestore
  \psslopesteps \psx@slopecenter\space \psx@sloperadius\space
  \ifPST@fading \psk@startfading \psk@endfading true \else false \fi
  tx@PstSlopeDict begin CcSlopesFill end}}
\def\psfs@radslopes{%
 \addto@pscode{%
  \psx@slopecolors\space  
  \psslopesteps\psx@slopecenter\space\psx@sloperadius\space\psx@slopeangle
  \ifPST@fading \psk@startfading \psk@endfading true \else false \fi
  tx@PstSlopeDict begin RadSlopesFill end}}
%    \end{macrocode}
%    \end{macro}
%    \end{macro}
%    \end{macro}
%    \begin{macro}{radslope}
%    |radslope| is slightly different:  Just going from one colour to
%    another in 360 degrees is usually not what is wanted.  |radslope| just
%    does something pretty with the colours provided.
%    \begin{macrocode}
\def\psfs@radslope{%
 \addto@pscode{%
  gsave 0 \pst@usecolor\psslopebegin currentrgbcolor
    1 \pst@usecolor\psslopeend currentrgbcolor
    2 \pst@usecolor\psslopebegin currentrgbcolor
    3 \pst@usecolor\psslopeend currentrgbcolor
    4 \pst@usecolor\psslopebegin currentrgbcolor
    5 grestore
  \psslopesteps\psx@slopecenter\space\psx@sloperadius\space\psx@slopeangle
  \ifPST@fading \psk@startfading \psk@endfading true \else false \fi
  tx@PstSlopeDict begin RadSlopesFill end}}
%    \end{macrocode}
%    \end{macro}
%
%    \begin{macro}{\psBall}
%    \begin{macrocode}
\def\psBall{\pst@object{psBall}}
\def\psBall@i{\@ifnextchar(\psBall@ii{\psBall@ii(0,0)}}
\def\psBall@ii(#1,#2)#3#4{{%
  \pst@killglue
  \pssetlength\pst@dima{#4}%%%%%  20111025 hv
  \pst@dimb=\pst@dima%%%%%%%%%%%  20111025 hv
  \advance\pst@dima by 0.075\pst@dimb%
  \addbefore@par{sloperadius=\the\pst@dima,fillstyle=ccslope,
   slopebegin=white,slopeend=#3,slopecenter=0.4 0.6,linestyle=none}%
  \use@par%
  \pscircle(#1,#2){#4}%
  }\ignorespaces%
}
%    \end{macrocode}
%    \end{macro}
%
%
%    \begin{macrocode}
\catcode`\@=\TheAtCode\relax
%</texfile>
%    \end{macrocode}
%
% \subsection{The \file{pst-slpe.pro} file}
%    The file \file{pst-slpe.pro} contains PostScript definitions
%    to be included in the PostScript output by the 
%    |dvi|-to-PostScript converter, eg |dvips|.
%    First thing is to define a
%    dictionary to keep definitions local.
%    \begin{macrocode}
%<*prolog>
/tx@PstSlopeDict 60 dict def tx@PstSlopeDict begin
%    \end{macrocode}
%
%    \begin{macro}{Opacity++}
%    This macro increments the Opacity index
%    \begin{macrocode}
/Opacity 1 def % preset, no transparency
/Opacity++ { Opacity dOpacity add /Opacity ED } def
%    \end{macrocode}
%    \end{macro}
%    \begin{macro}{max}
%    $x1 \quad x2 \quad \mathtt{max}\quad max$\\
%    |max| is a utility function that calculates the maximum
%    of two numbers.
%    \begin{macrocode}
/max {2 copy lt {exch} if pop} bind def
%    \end{macrocode}
%    \end{macro}
%
%    \begin{macro}{Iterate}
%    $p_1\quad r_1\quad g_1\quad b_1\quad\ldots\quad
%    p_n\quad r_n\quad g_n\quad b_n\quad n\quad \mathtt{Iterate}\quad -$\\
%    This is the actual iteration, which goes through the colour
%    information and plots the segments.  
%    It uses the value of |NumSteps| which is set by the wrapper
%    procedures.  |DrawStep| is called all of |NumSteps| times, so 
%    it had better be fast.
%
%    First, the number of colour infos is read from the
%    top of the stack and decremented, to get the number of segments. 
%    \begin{macrocode}
/Iterate {
  1 sub /NumSegs ED
%    \end{macrocode}
%    Now we get the first colour.  This is really the {\em last}
%    colour given in the |slopecolors| argument.  We have to work
%    {\em down} the stack, so we shall be careful to plot the segments
%    in reverse order.  The |dup mul| stuff squares the RGB
%    components.  This does a kind-of-gamma correction, without
%    which primary colours tend to take up too much space in the
%    slope.  This is nothing deep, it just looks better in my opinion.
%    The following lines convert RGB to HSB and store the resulting
%    components, as well as the |Pt| coordinate in four variables.
%    \begin{macrocode}
  dup mul 3 1 roll dup mul 3 1 roll dup mul 3 1 roll
  setrgbcolor currenthsbcolor 
  /ThisB ED
  /ThisS ED
  /ThisH ED
  /ThisPt ED
%    \end{macrocode}
%    To avoid gaps, we fill the whole  path in that first colour.
%    \begin{macrocode}
  Opacity .setopacityalpha 
  gsave 
  fill 
  grestore
%    \end{macrocode}
%    The body of the following outer loop is executed
%    once for each segment.
%    It expects a current colour and |Pt| coordinate in the |This*|
%    variables and pops the next colour and point from the stack. It
%    then draws the single steps of that segment.
%    \begin{macrocode}
  NumSegs {
    dup mul 3 1 roll dup mul 3 1 roll dup mul 3 1 roll
    setrgbcolor currenthsbcolor
    /NextB ED
    /NextS ED
    /NextH ED
    /NextPt ED
%    \end{macrocode}
%    |NumSteps| always contains the remaining number of steps available.
%    These are evenly distributed between |Pt| coordinates |ThisPt| 
%    to 0, so for the current segment we may use 
%    $|NumSteps|*(|ThisPt|-|NextPt|)/|ThisPt|$ steps.
%    \begin{macrocode}
    ThisPt NextPt sub ThisPt div NumSteps mul cvi /SegSteps exch def
    /NumSteps NumSteps SegSteps sub def
%    \end{macrocode}
%    |SegSteps| may be zero.  In that case there is nothing to do for
%    this segment.
%    \begin{macrocode}
    SegSteps 0 eq not {
%    \end{macrocode}
%    If one of the colours is gray, ie~0 saturation, its hue is
%    useless.  In this case, instead of starting of with a random hue,
%    we take the hue of the other endpoint.  (If both have saturation
%    0, we have a pure gray scale and no harm is done)
%    \begin{macrocode}
      ThisS 0 eq {/ThisH NextH def} if
      NextS 0 eq {/NextH ThisH def} if
%    \end{macrocode}
%    To interpolate between two colours of different hue, we want to
%    go the shorter way around the colour circle.  The following code
%    assures that this happens if we go linearly from |This*| to 
%    |Next*| by conditionally adding 1.0 to one of the hue values.
%    The new hue values can lie between 0.0 and 2.0, so we will later
%    have to subtract 1.0 from values greater than one.
%    \begin{macrocode}
      ThisH NextH sub 0.5 gt
        {/NextH NextH 1.0 add def} 
        { NextH ThisH sub 0.5 ge {/ThisH ThisH 1.0 add def} if }
      ifelse
%    \end{macrocode}
%    We define three variables to hold the current colour coordinates
%    and calculate the corresponding increments per step.
%    \begin{macrocode}
      /B ThisB def
      /S ThisS def
      /H ThisH def
      /BInc NextB ThisB sub SegSteps div def
      /SInc NextS ThisS sub SegSteps div def
      /HInc NextH ThisH sub SegSteps div def
%    \end{macrocode}
%    The body of the following inner loop sets the current colour,
%    according to |H|, |S| and |B| and
%    undoes the kind-of-gamma correction by converting to RGB colour.
%    It then calls |DrawStep|, which draws one step and maybe updates
%    the current point or user space, or variables of its own.  Finally,
%    it increments the three colour variables.
%    \begin{macrocode}
      SegSteps {
        H dup 1. gt {1. sub} if S B sethsbcolor 
        currentrgbcolor 
        sqrt 3 1 roll sqrt 3 1 roll sqrt 3 1 roll
        setrgbcolor
        DrawStep
        /H H HInc add def
        /S S SInc add def
        /B B BInc add def
      } bind repeat
%    \end{macrocode}
%    The outer loop ends by moving on to the |Next| colour and point.
%
%    \begin{macrocode}
      /ThisH NextH def
      /ThisS NextS def
      /ThisB NextB def
      /ThisPt NextPt def
    } if
  } bind repeat
} def
%    \end{macrocode}
%    \end{macro}
%
%    \begin{macro}{PatchRadius}
%    $-\quad\mathtt{PatchRadius}\quad-$\\
%    This macro inspects the value of the variable |Radius|.  If it is
%    0, it is set to the maximum distance of any point in the 
%    current path from the origin of user space.  This has the effect
%    that the current path will be totally filled.  To find the maximum
%    distance, we flatten the path and call |UpdRR| for each endpoint
%    of the generated polygon.  The current maximum square distance is
%    gathered in |RR|.  
%    \begin{macrocode}
/PatchRadius {
  Radius 0 eq { 
    /UpdRR { dup mul exch dup mul add RR max /RR ED } bind def
    gsave
    flattenpath
    /RR 0 def
    {UpdRR} {UpdRR} {} {} pathforall
    grestore
    /Radius RR sqrt def
  } if
} def
%    \end{macrocode}
%    \end{macro}
%
%    \begin{macro}{SlopesFill}
%    $p_1\quad r_1\quad g_1\quad b_1\quad\ldots\quad
%    p_n\quad r_n\quad g_n\quad b_n\quad n\quad s\quad\alpha\quad    
%    \mathtt{SlopesFill}\quad -$\\
%    Fill the current path with a slope described by $p_1,\ldots,b_n,n$.
%    Use a total of $s$ single steps.  Rotate the slope by $\alpha$ 
%    degrees, 0 meaning $r_1,g_1,b_1$ left to $r_n,g_n,b_n$ right.
%
%    After saving the current path, we do the rotation and get the
%    number of steps, which is later needed by |Iterate|.  Remember,
%    that iterate calls |DrawStep| in the reverse order, ie~from
%    right to left.  We work around this by adding 180 degrees to 
%    the rotation.    Filling
%    works by clipping to the path and painting an appropriate sequence
%    of rectangles.  |DrawStep| is set up for |Iterate| to draw a 
%    rectangle of width |XInc| high enough to cover the whole
%    clippath (we use the Level 2 operator |rectfill| for speed) and
%    translate the user system by |XInc|.
%    \begin{macrocode}
/SlopesFill {
  /Fading ED		% do we have fading?
  Fading {
    /FadingEnd ED % the last opacity value
    dup /FadingStart ED % the first opacity value
    /Opacity ED % the opacity start value
  } if
  gsave
  180 add rotate
  /NumSteps ED
  Fading { /dOpacity FadingEnd FadingStart sub NumSteps div def } if
  clip
  pathbbox
  /h ED /w ED
  2 copy translate
  h sub neg /h ED
  w sub neg /w ED
  /XInc w NumSteps div def
  /DrawStep {
    Fading {			% do we have a fading?
      Opacity .setopacityalpha  % set opacity value
      Opacity++			% increase opacity
    } if
    0 0 XInc h rectfill
    XInc 0 translate
  } bind def
  Iterate
  grestore
} def
%    \end{macrocode}
%    \end{macro}
%
%    \begin{macro}{CcSlopesFill} $p_1\quad r_1\quad g_1\quad
%    b_1\quad\ldots\quad p_n\quad r_n\quad g_n\quad b_n\quad n\quad
%    c_x\quad c_y \quad r\quad \mathtt{CcSlopesFill}\quad -$\\ Fills
%    the current path with a concentric pattern,
%    ie~in a polar coordinate system, the colour depends on the
%    radius and not on the angle.
%    Centered around a point with coordinates $(c_x,c_y)$ relative to
%    the bounding box of the path, ie~for a rectangle, $(0,0)$ will
%    center the pattern around the lower left corner of the rectangle,
%    $(0.5,0.5)$ around its center.  The largest circle has a radius of
%    $r$.  If $r=0$, $r$ is taken to be the maximum distance of any
%    point on the current path from the center defined by $(c_x,c_y)$.
%    The colours are given from the center outwards,
%    ie~$(r_1,g_1,b_1)$ describe the colour at the center.
%
%    The code is similar to that of |SlopesFill|.  The main differences
%    are the call to |PatchRadius|, which catches the case that $r=0$
%    and the different definition for |DrawStep|, Which now fills a
%    circle of radius |Rad| and decreases that Variable.  Of course,
%    drawing starts on the outside, so we work down the stack and circles
%    drawn later partially cover those drawn first.  Painting
%    non-overlapping, `donut-shapes' would be slower. 
%    \begin{macrocode}
/CcSlopesFill {
  /Fading ED		% do we have fading?
  Fading {
    /FadingEnd ED % the last opacity value
    dup /FadingStart ED % the first opacity value
    /Opacity ED % the opacity start value
  } if
  gsave
  /Radius ED
  /CenterY ED
  /CenterX ED
  /NumSteps ED
  Fading { /dOpacity FadingEnd FadingStart sub NumSteps div def } if
  clip
  pathbbox
  /h ED /w ED
  2 copy translate
  h sub neg /h ED
  w sub neg /w ED
  w CenterX mul h CenterY mul translate
  PatchRadius
  /RadPerStep Radius NumSteps div neg def
  /Rad Radius def
  /DrawStep {
    Fading {			% do we have a fading?
      Opacity .setopacityalpha  % set opacity value
      Opacity++			% increase opacity
    } if
    0 0 Rad 0 360 arc
    closepath fill
    /Rad Rad RadPerStep add def
  } bind def
  Iterate
  grestore
} def
%    \end{macrocode}
%    \end{macro}
%
%    \begin{macro}{RadSlopesFill}
%    $p_1\quad r_1\quad g_1\quad b_1\quad\ldots
%    \quad p_n\quad r_n\quad g_n\quad b_n\quad n\quad
%    c_x\quad c_y \quad r\quad\alpha\quad \mathtt{CcSlopesFill}\quad -$\\
%    This fills the current path with a radial pattern, ie~in a
%    polar coordinate system the colour depends on the angle and not on
%    the radius.  All this is very similar to |CcSlopesFill|.  There
%    is an extra parameter $\alpha$, which rotates the pattern.
%
%    The only new thing in the code is the |DrawStep| procedure.
%    This does {\em not} draw a circular arc, but a triangle, which is
%    considerably faster.  One of the short sides of the triangle is
%    determined by |Radius|, the other one by |dY|, which is calculated
%    as $|dY|:=|Radius|\times\tan(|AngleIncrement|)$.
%    \begin{macrocode}
/RadSlopesFill {
  /Fading ED		% do we have fading?
  Fading {
    /FadingEnd ED % the last opacity value
    dup /FadingStart ED % the first opacity value
    /Opacity ED % the opacity start value
  } if
  gsave
  rotate
  /Radius ED
  /CenterY ED
  /CenterX ED
  /NumSteps ED
  Fading { /dOpacity FadingEnd FadingStart sub NumSteps div def } if
  clip
  pathbbox
  /h ED /w ED
  2 copy translate
  h sub neg /h ED
  w sub neg /w ED
  w CenterX mul h CenterY mul translate
  PatchRadius
  /AngleIncrement 360 NumSteps div neg def
  /dY AngleIncrement sin AngleIncrement cos div Radius mul def
  /DrawStep {
    Fading {			% do we have a fading?
      Opacity .setopacityalpha  % set opacity value
      Opacity++			% increase opacity
    } if
    0 0 moveto
    Radius 0 rlineto
    0 dY rlineto
    closepath fill
    AngleIncrement rotate
  } bind def
  Iterate
  grestore
} def
%    \end{macrocode}
%    \end{macro}
%
% Last, but not least, we have to close the private dictionary.
%    \begin{macrocode}
end
%</prolog>
%    \end{macrocode}
% \Finale
%
'
%\end{verbatim}
%\section{Package Usage}
% To use |pst-slpe|, you have to say
% \begin{verbatim}
%   \usepackage{pst-slpe}
% \end{verbatim}
% in the document prologue for \LaTeX, and 
% \begin{verbatim}
%   \input pst-slpe.tex
% \end{verbatim}
% in ``plain'' \TeX.
%
% \section{New macro and fill styles}
% \DescribeMacro{\psBall}
%    It takes the (optional) coordinates of the ball center, the color
%    and the radius as parameter and uses |\pscircle| for painting
%    the bullet. 
%
%    \vspace{1cm}
%    \psBall{black}{2ex}
%    \psBall(1,0){blue}{3ex}
%    \psBall(2.5,0){red}{4ex}
%    \psBall(4,0){green!50!blue!60}{5ex}
%
%    \vspace{1cm}
% \begin{verbatim}
%    \psBall{black}{2ex}
%    \psBall(1,0){blue}{3ex}
%    \psBall(2.5,0){red}{4ex}
%    \psBall(4,0){green!50!blue!60}{5ex}
% \end{verbatim}
%
%    The predinied options can be overwritten in the usual way:
%
%    \vspace{1cm}
%    \psBall{black}{2ex}
%    \psBall[sloperadius=10pt](1,0){blue}{3ex}
%    \psBall(2.5,0){red}{4ex}
%    \psBall[slopebegin=red](4,0){green!50!blue!60}{5ex}
%
%    \vspace{1cm}
% \begin{verbatim}
%    \psBall{black}{2ex}
%    \psBall[sloperadius=10pt](1,0){blue}{3ex}
%    \psBall(2.5,0){red}{4ex}
%    \psBall[slopebegin=red](4,0){green!50!blue!60}{5ex}
% \end{verbatim}
%
% \DescribeMacro{slope}
% \DescribeMacro{slopes}
% \DescribeMacro{ccslope}
% \DescribeMacro{ccslopes}
% \DescribeMacro{radslope}
% \DescribeMacro{radslopes}
% |pst-slpe| provides six new fill styles called |slope|, |slopes|,
% |ccslope|, |ccslopes|, |radslope| and |radslopes|.  These obviously
% come in pairs: The $\ldots$|slope|-styles are simplified versions of
% the general $\ldots$|slopes|-styles.\footnote{By the way, I use slope
% as a synonym for gradient.  It sounds less pretentious and avoids
% name clashes.}  The |cc|$\ldots$ styles paint concentric patterns,
% and the |rad|$\ldots$ styles do radial ones.  
%
% Here is a little
% overview of what they look like:
% \newcommand{\st}{$\vcenter to30pt{}$}
% \begin{quote}\LARGE
%  \begin{tabular}{cc}
%   \psframebox[fillstyle=slope]{\st|slope|} &\qquad
%   \psframebox[fillstyle=slopes]{\st|slopes|} \\[2ex]
%   \psframebox[fillstyle=ccslope]{\st|ccslope|} &\qquad
%   \psframebox[fillstyle=ccslopes]{\st|ccslopes|} \\[2ex]
%   \psframebox[fillstyle=radslope]{\st|radslope|} &\qquad
%   \psframebox[fillstyle=radslopes]{\st|radslopes|} \\[2ex]
%  \end{tabular}
% \end{quote}
% These examples were produced by saying simply
% \begin{verbatim}
%   \psframebox[fillstyle=slope]{...}
% \end{verbatim}
% etc.~without setting any further graphics parameters.  The package
% provides a number of parameters that can be used to control
% the way these patterns
% are painted.
% \medskip
%
% \DescribeMacro{slopebegin}
% \DescribeMacro{slopeend}
% The graphics parameters |slopebegin| and |slopeend| set the colours
% between which the three $\ldots$|slope| styles should interpolate.
% Eg,
% \begin{verbatim}
%   \psframebox[fillstyle=slope,slopebegin=red,slopeend=green]{...}
% \end{verbatim}
% produces:
% \begin{quote}\Large
%  \psframebox[fillstyle=slope,slopebegin=red,slopeend=green]{\st slopes!}
% \end{quote}
% The same settings of |slopebegin| and |slopeend| for the |ccslope|
% and |radslope| fillstyles produce
% \begin{quote}\Large
%  \psframebox[fillstyle=ccslope,slopebegin=red,slopeend=green]{\st slopes!}
%   \quad{\normalsize resp.}\quad
%  \psframebox[fillstyle=radslope,slopebegin=red,slopeend=green]{\st slopes!}
% \end{quote}
% The default settings go from a greenish yellow to pure blue.
% \medskip
%
% \DescribeMacro{slopecolors}
% If you want to interpolate between more than two colours, you have
% to use the $\ldots$|slopes| styles, which are controlled by the 
% |slopecolors| parameter instead of |slopebegin| and |slopeend|.  The 
% idea is to specify the colour to use at certain points `on the
% way'.  To fill a shape with |slopes|, imagine a linear scale 
% from its left edge to its right edge.  The left edge must lie at
% coordinate 0.  Pick an arbitrary value for the right edge, say 23.
% Now you want to get light yellow at the left edge, a pastel green at $17/23$ 
% of the way and dark cyan at the right edge, like this:
% \begin{quote}\psset{unit=0.45cm}
%  \begin{pspicture}(-1,0)(24,6)
%   \pscustom[fillstyle=slopes,
% slopecolors=0 1 1 .9  17 .5 1 .5  23 0 0.5 0.5  3]{
%    \psccurve(0,2.5)(12,3.5)(20,4)(23,2)(17,2.5)}
%   \psaxes(0,5)(-0.01,5)(23.01,5)
%   \psline(0,5)(0,1)
%   \psline(17,5)(17,1)
%   \psline(23,5)(23,1)
%  \end{pspicture}
% \end{quote}
% The RGB values for the three colours are $(1,1,0.9)$, $(0.5,1,0.5)$
% and $(0,0.5,0.5)$.  The value for the |slopecolors| parameter is a list
% of `colour infos' followed by the number of `colour infos'. 
% Each `colour info' consists
% of the coordinate value where a colour is to be specified, followed by
% the RGB values of that colour.  All these values are separated by
% white space.  The correct setting for the example is thus:
% \begin{verbatim}
%   slopecolors=0 1 1 .9   17 .5 1 .5   23 0 .5 .5   3
% \end{verbatim}
% For |ccslopes|, specify the colours from the center outward.  
% For |radslopes| (with no rotation specified), 0 represents the ray
% going `eastward'.  Specify the colours anti-clockwise.  If you want a
% smooth gradient at the beginning and starting ray of |radslopes|, you
% should pick the first and last colours identical.
%
% Please note, that the |slopecolors| parameter is not subject to any
% parsing on the \TeX\ side.  If you forget a number or specify the wrong
% number of segments, the PostScript interpreter will probably crash.
%
% The default value for |slopecolors| specifies a rainbow.
%
% \medskip
%
% \DescribeMacro{slopesteps}
% The parameter |slopesteps| controls the number of distinct colour steps
% rendered.  Higher values for this parameter result in better quality
% but proportionally slower rendering.  Eg, setting
% |slopesteps| to 5 with the |slope| fill style results in
% \begin{quote}\Large
%  \psframebox[fillstyle=slope,slopesteps=5]{\st slopes!}
% \end{quote}
%
% The default value is 100, which
% suffices for most purposes.  Remember that the number of distinct colours
% reproducible by a given device is limited.  Pushing |slopesteps| to
% high will result only in loss of performance at no gain in quality.
% \medskip
%
% \DescribeMacro{slopeangle}
% The |slope(s)| and |radslope(s)| patterns may be rotated.  As usual,
% the angles are given anti-clockwise.  Eg, an angle of 30 degrees
% gives
% \begin{quote}\Large\psset{slopeangle=30}
%  \psframebox[fillstyle=slope]{\st slopes!}
%   \quad{\normalsize and}\quad
%  \psframebox[fillstyle=radslope]{\st slopes!}
% \end{quote}
% with the |slope| and |radslope| fillstyles.
% \medskip
%
% \DescribeMacro{slopecenter}
% For the |cc|$\ldots$ and |rad|$\ldots$ styles, it is possible to
% set the center of the pattern.  The |slopecenter| parameter is set to
% the coordinates of that center relative to the bounding box of the
% current path.  The following effect:
% \begin{quote}\psset{unit=0.45cm}
%  \begin{pspicture}(-1,-1)(24,5)
%   \pscustom[fillstyle=radslope,slopecenter=0.2 0.4]{
%    \pspolygon(0,2.5)(12,2.5)(20,4)(23,2)(17,2.5)(3,0)}
%   \psaxes[axesstyle=frame,Dx=0.1,dx=2.2999,Dy=0.2,dy=0.7999](0,0)(23,4) 
%   \psline(4.6,0)(4.6,4)
%   \psline(0,1.6)(23,1.6)
%  \end{pspicture}
% \end{quote}
%  was achieved with 
% \begin{verbatim}
%    fillstyle=radslope,slopecenter=0.2 0.4
% \end{verbatim}
% The default value for |slopecenter| is |0.5 0.5|, which is the
% center for symmetrical shapes.  Note that this parameter is not
% parsed by \TeX, so setting it to anything else than two numbers 
% between 0 and 1 might crash the PostScript interpreter.
% \medskip
%
% \DescribeMacro{sloperadius}
% Normally, the |cc|$\ldots$ and |rad|$\ldots$ styles distribute the
% given colours so that the center is painted in the first colour given,
% and the points of the shape furthest from the center are painted in 
% the last colour.  In other words the maximum radius to which the
% |slopecolors| parameter refers is the maximum distance from the
% center (defined by |slopecenter|) to any point on the periphery
% of the shape.  This radius can be explicitly set with |sloperadius|.
% Eg, setting |sloperadius=0.5cm| gives
% \begin{quote}\Large\psset{sloperadius=0.5cm}
%  \psframebox[fillstyle=ccslope]{\st slopes!}
% \end{quote}
% Any point further from the center than the given |sloperadius| is
% painted with the last colour in |slopeclours|, resp.~|slopeend|.
%
% The default value for |sloperadius| is 0, which invokes the default
% behaviour of automatically calculating the radius.
%
% \DescribeMacro{fading}
% \DescribeMacro{startfading}
% \DescribeMacro{endfading}
% The optional boolean keyword |fading| allows a transparency effect of
% the filled area, starting with the opacity value |startfading| and
% ending with the value of |endfading|. Both values must be of the
% intervall [0\ldots1], with 0 for total opacity and 1 for no
% opacity. The values are preset by 0 and 1.
%
% Here is a little
% overview of what they look like:
% \begin{quote}\LARGE\color{white}
%  \begin{tabular}{cc}
%   \psframe*(-0.3,-0.25)(3,20pt)\psframebox[fading,fillstyle=slope]{\st|  slope  |} &\qquad
%   \psframe*(-0.3,-0.25)(3,20pt)\psframebox[fading,fillstyle=slopes]{\st| slopes |} \\[2ex]
%   \psframe*(-0.3,-0.25)(3,20pt)\psframebox[fading,fillstyle=ccslope]{\st|ccslope |} &\qquad
%   \psframe*(-0.3,-0.25)(3,20pt)\psframebox[fading,fillstyle=ccslopes]{\st|ccslopes|} \\[2ex]
%   \psframe*(-0.3,-0.25)(3,20pt)\psframebox[fading,fillstyle=radslope]{\st|radslope|} &\qquad
%   \psframe*(-0.3,-0.25)(3,20pt)\psframebox[fading,fillstyle=radslopes]{\st|radslopes|} \\[2ex]
%  \end{tabular}
% \end{quote}
% \color{black}
%
% These examples were produced by saying simply
% \begin{verbatim}
%   \psframebox[fading,fillstyle=...]{...}
% \end{verbatim}
%
% \begin{quote}\LARGE\color{white}
% \psset{fading,startfading=0.3,endfading=0.8}
%  \begin{tabular}{cc}
%   \psframe*(-0.3,-0.25)(3,20pt)\psframebox[fillstyle=slope]{\st|  slope  |} &\qquad
%   \psframe*(-0.3,-0.25)(3,20pt)\psframebox[fillstyle=slopes]{\st| slopes |} \\[2ex]
%   \psframe*(-0.3,-0.25)(3,20pt)\psframebox[fillstyle=ccslope]{\st|ccslope |} &\qquad
%   \psframe*(-0.3,-0.25)(3,20pt)\psframebox[fillstyle=ccslopes]{\st|ccslopes|} \\[2ex]
%   \psframe*(-0.3,-0.25)(3,20pt)\psframebox[fillstyle=radslope]{\st|radslope|} &\qquad
%   \psframe*(-0.3,-0.25)(3,20pt)\psframebox[fillstyle=radslopes]{\st|radslopes|} \\[2ex]
%  \end{tabular}
% \end{quote}
% \color{black}
%
% These examples were produced by saying simply
% \begin{verbatim}
%   \psframebox[fading,startfading=0.3,endfading=0.8,fillstyle=...]{...}
% \end{verbatim}
%
% \StopEventually{}
%
%\section{The Code}
% \subsection{Producing the documentation}
%
%    A short driver is provided that can be extracted if necessary by
%    the \textsc{docstrip} program provided with \LaTeXe.
%    \begin{macrocode}
%<*driver>
\NeedsTeXFormat{LaTeX2e}
\documentclass{ltxdoc}
\usepackage{pst-slpe}
\usepackage{pst-plot}
\DisableCrossrefs
\MakeShortVerb{\|}
\newcommand\Lopt[1]{\textsf{#1}}
\newcommand\file[1]{\texttt{#1}}
\AtEndDocument{
\PrintChanges
\PrintIndex
}
%\OnlyDescription
\begin{document}
\DocInput{pst-slpe.dtx}
\end{document}
%</driver>
%    \end{macrocode}
%
% \subsection{The \file{pst-slpe.sty} file}
%    The \file{pst-slpe.sty} file is very simple.  It just loads 
%    the generic \file{pst-slpe.tex} file. 
%    \begin{macrocode}
%<*stylefile>
\RequirePackage{pstricks}
\ProvidesPackage{pst-slpe}[2005/03/05 package wrapper for `pst-slpe.tex']
%%
%% This is file `pst-slpe.tex',
%% generated with the docstrip utility.
%%
%% The original source files were:
%%
%% pst-slpe.dtx  (with options: `texfile')
%% 
%% IMPORTANT NOTICE:
%% 
%% For the copyright see the source file.
%% 
%% Any modified versions of this file must be renamed
%% with new filenames distinct from pst-slpe.tex.
%% 
%% For distribution of the original source see the terms
%% for copying and modification in the file pst-slpe.dtx.
%% 
%% This generated file may be distributed as long as the
%% original source files, as listed above, are part of the
%% same distribution. (The sources need not necessarily be
%% in the same archive or directory.)
%% This program can be redistributed and/or modified under the terms
%% of the LaTeX Project Public License Distributed from CTAN archives
%% in directory macros/latex/base/lppl.txt.
%%
\def\pstslpefileversion{1.31}
\def\pstslpefiledate{2011/10/25}
\message{ v\pstslpefileversion, \pstslpefiledate}
\csname PstSlopeLoaded\endcsname
\let\PstSlopeLoaded\endinput
\ifx\PSTricksLoaded\endinput\else
  \def\next{\input pstricks.tex }\expandafter\next
\fi
\ifx\PSTXKeyLoaded\endinput\else\input pst-xkey \fi % --> hv
\edef\TheAtCode{\the\catcode`\@}
\catcode`\@=11
\pst@addfams{pst-slpe}                              % --> hv
\pstheader{pst-slpe.pro}
\newrgbcolor{slopebegin}{0.9 1 0}
\define@key[psset]{pst-slpe}{slopebegin}{\pst@getcolor{#1}\psslopebegin}% --> hv
\psset[pst-slpe]{slopebegin=slopebegin}                                 % --> hv

\newrgbcolor{slopeend}{0 0 1}
\define@key[psset]{pst-slpe}{slopeend}{\pst@getcolor{#1}\psslopeend}% --> hv
\psset[pst-slpe]{slopeend=slopeend}% --> hv

\define@key[psset]{pst-slpe}{slopesteps}{\pst@getint{#1}\psslopesteps}% --> hv
\psset[pst-slpe]{slopesteps=100}% --> hv

\define@key[psset]{pst-slpe}{slopeangle}{\pst@getangle{#1}\psx@slopeangle}% --> hv
\psset[pst-slpe]{slopeangle=0}% --> hv
\define@key[psset]{pst-slpe}{slopecolors}{\def\psx@slopecolors{#1}}% --> hv
\psset[pst-slpe]{slopecolors={% --> hv
0.0 1 0 0
0.4 0 1 0
0.8 0 0 1
1.0 1 0 1
4}}
\define@key[psset]{pst-slpe}{slopecenter}{\def\psx@slopecenter{#1}}% --> hv
\psset[pst-slpe]{slopecenter={0.5 0.5}}% --> hv
\define@key[psset]{pst-slpe}{sloperadius}{\pst@getlength{#1}\psx@sloperadius}% --> hv
\psset[pst-slpe]{sloperadius=0}% --> hv
\define@boolkey[psset]{pst-slpe}[PST@]{fading}[true]{}% --> hv
\psset[pst-slpe]{fading=false}% --> hv
\define@key[psset]{pst-slpe}{startfading}{\pst@checknum{#1}\psk@startfading }% --> hv
\define@key[psset]{pst-slpe}{endfading}{\pst@checknum{#1}\psk@endfading }% --> hv
\psset[pst-slpe]{startfading=0,endfading=1}% --> hv
\def\psfs@slopes{%
 \addto@pscode{
  \psx@slopecolors\space
  \psslopesteps
  \psx@slopeangle
  \ifPST@fading \psk@startfading \psk@endfading true \else false \fi
  tx@PstSlopeDict begin SlopesFill end}}
\def\psfs@slope{%
 \addto@pscode{%
  gsave
    0 \pst@usecolor\psslopebegin currentrgbcolor
    1 \pst@usecolor\psslopeend currentrgbcolor
    2
  grestore
  \psslopesteps \psx@slopeangle
  \ifPST@fading \psk@startfading \psk@endfading true \else false \fi
  tx@PstSlopeDict begin SlopesFill end}}
\def\psfs@ccslopes{%
 \addto@pscode{%
  \psx@slopecolors\space
  \psslopesteps \psx@slopecenter\space \psx@sloperadius\space
  \ifPST@fading \psk@startfading \psk@endfading true \else false \fi
  tx@PstSlopeDict begin CcSlopesFill end}}
\def\psfs@ccslope{%
 \addto@pscode{%
  gsave 0 \pst@usecolor\psslopebegin currentrgbcolor
    1 \pst@usecolor\psslopeend currentrgbcolor
    2 grestore
  \psslopesteps \psx@slopecenter\space \psx@sloperadius\space
  \ifPST@fading \psk@startfading \psk@endfading true \else false \fi
  tx@PstSlopeDict begin CcSlopesFill end}}
\def\psfs@radslopes{%
 \addto@pscode{%
  \psx@slopecolors\space
  \psslopesteps\psx@slopecenter\space\psx@sloperadius\space\psx@slopeangle
  \ifPST@fading \psk@startfading \psk@endfading true \else false \fi
  tx@PstSlopeDict begin RadSlopesFill end}}
\def\psfs@radslope{%
 \addto@pscode{%
  gsave 0 \pst@usecolor\psslopebegin currentrgbcolor
    1 \pst@usecolor\psslopeend currentrgbcolor
    2 \pst@usecolor\psslopebegin currentrgbcolor
    3 \pst@usecolor\psslopeend currentrgbcolor
    4 \pst@usecolor\psslopebegin currentrgbcolor
    5 grestore
  \psslopesteps\psx@slopecenter\space\psx@sloperadius\space\psx@slopeangle
  \ifPST@fading \psk@startfading \psk@endfading true \else false \fi
  tx@PstSlopeDict begin RadSlopesFill end}}
\def\psBall{\pst@object{psBall}}
\def\psBall@i{\@ifnextchar(\psBall@ii{\psBall@ii(0,0)}}
\def\psBall@ii(#1,#2)#3#4{{%
  \pst@killglue
  \pssetlength\pst@dima{#4}%%%%%  20111025 hv
  \pst@dimb=\pst@dima%%%%%%%%%%%  20111025 hv
  \advance\pst@dima by 0.075\pst@dimb%
  \addbefore@par{sloperadius=\the\pst@dima,fillstyle=ccslope,
   slopebegin=white,slopeend=#3,slopecenter=0.4 0.6,linestyle=none}%
  \use@par%
  \pscircle(#1,#2){#4}%
  }\ignorespaces%
}
\catcode`\@=\TheAtCode\relax
\endinput
%%
%% End of file `pst-slpe.tex'.

\ProvidesFile{pst-slpe.tex}
  [\pstslpefiledate\space v\pstslpefileversion\space 
    `pst-slpe' (mg,hv)]
\IfFileExists{pst-slpe.pro}{%
  \ProvidesFile{pst-slpe.pro}
    [2008/06/19 v. 0.01,  PostScript prologue file (hv)]
    \@addtofilelist{pst-slpe.pro}}{}%
%</stylefile>
%    \end{macrocode}
%
% \subsection{The \file{pst-slpe.tex} file}
%    \file{pst-slpe.tex} contains the \TeX-side of things.  We begin
%    by identifying ourselves and setting things up, the same as in 
%    other PSTricks packages.
%    \begin{macrocode}
%<*texfile>
\message{ v\pstslpefileversion, \pstslpefiledate}
\csname PstSlopeLoaded\endcsname
\let\PstSlopeLoaded\endinput
\ifx\PSTricksLoaded\endinput\else
  \def\next{\input pstricks.tex }\expandafter\next
\fi
\ifx\PSTXKeyLoaded\endinput\else\input pst-xkey \fi % --> hv
\edef\TheAtCode{\the\catcode`\@}
\catcode`\@=11
\pst@addfams{pst-slpe}                              % --> hv
\pstheader{pst-slpe.pro}
%    \end{macrocode}
%    \begin{macro}{slopebegin}
%    \begin{macro}{slopeend}
%    \begin{macro}{slopesteps}
%    \begin{macro}{slopeangle}
%
% \subsubsection{New graphics parameters}
%    We now define the various new parameters needed by the |slope|
%    fill styles and install default values.  First come the colours, 
%    ie~graphics parameters |slopebegin| and |slopeend|, followed
%    by the number of steps, |slopesteps|, and the rotation angle, 
%    |slopeangle|.
%    \begin{macrocode}
\newrgbcolor{slopebegin}{0.9 1 0}
\define@key[psset]{pst-slpe}{slopebegin}{\pst@getcolor{#1}\psslopebegin}% --> hv
\psset[pst-slpe]{slopebegin=slopebegin}                                 % --> hv

\newrgbcolor{slopeend}{0 0 1}
\define@key[psset]{pst-slpe}{slopeend}{\pst@getcolor{#1}\psslopeend}% --> hv
\psset[pst-slpe]{slopeend=slopeend}% --> hv

\define@key[psset]{pst-slpe}{slopesteps}{\pst@getint{#1}\psslopesteps}% --> hv
\psset[pst-slpe]{slopesteps=100}% --> hv

\define@key[psset]{pst-slpe}{slopeangle}{\pst@getangle{#1}\psx@slopeangle}% --> hv
\psset[pst-slpe]{slopeangle=0}% --> hv
%    \end{macrocode}
%    \end{macro}
%    \end{macro}
%    \end{macro}
%    \end{macro}
%    \begin{macro}{slopecolors}
%    The value for |slopecolors| is not parsed.  It is directly copied 
%    to the PostScript output.  This is certainly not the way it
%    should be, but it's simple.  The default value is a rainbow from
%    red to magenta.
%    \begin{macrocode}
\define@key[psset]{pst-slpe}{slopecolors}{\def\psx@slopecolors{#1}}% --> hv
\psset[pst-slpe]{slopecolors={% --> hv
0.0 1 0 0		
0.4 0 1 0
0.8 0 0 1
1.0 1 0 1
4}}
%    \end{macrocode}
%    \end{macro}
%    \begin{macro}{slopecenter}
%    The argument to |slopecenter| isn't parsed either.  But there's
%    probably not much that can go wrong with two decimal numbers.
%    \begin{macrocode}
\define@key[psset]{pst-slpe}{slopecenter}{\def\psx@slopecenter{#1}}% --> hv
\psset[pst-slpe]{slopecenter={0.5 0.5}}% --> hv
%    \end{macrocode}
%    \end{macro}
%    \begin{macro}{sloperadius}
%    The default value for |sloperadius| is 0, which makes the
%    PostScript procedure |PatchRadius| determine a value for the radius.
%    \begin{macrocode}
\define@key[psset]{pst-slpe}{sloperadius}{\pst@getlength{#1}\psx@sloperadius}% --> hv
\psset[pst-slpe]{sloperadius=0}% --> hv
%    \end{macrocode}
%    \end{macro}
%    \begin{macro}{fading}
%    The default value for |fading| is false, which is no transparency effect at all.
%    With |fading=true| the package takes the values |startfading| and |endfading|
%    into account for the opacity effect of the filled area.
%    \end{macro}
%    \begin{macrocode}
\define@boolkey[psset]{pst-slpe}[PST@]{fading}[true]{}% --> hv
\psset[pst-slpe]{fading=false}% --> hv
%    \end{macrocode}
%    \begin{macro}{startfading}
%    The relativ number for the starting value (0\ldots1), preset by 0.
%    \end{macro}
%    \begin{macrocode}
\define@key[psset]{pst-slpe}{startfading}{\pst@checknum{#1}\psk@startfading }% --> hv
%    \end{macrocode}
%    \begin{macro}{endfading}
%    The relativ number for the end value (0\ldots1), preset by 1.
%    \end{macro}
%    \begin{macrocode}
\define@key[psset]{pst-slpe}{endfading}{\pst@checknum{#1}\psk@endfading }% --> hv
\psset[pst-slpe]{startfading=0,endfading=1}% --> hv
%    \end{macrocode}
%
% \subsubsection{Fill style macros}
%
%    Now come the fill style definitions that use these parameters.
%    There is one macro for each fill style named |\psfs@|$style$.  
%    PSTricks calls this macro whenever the current path needs to
%    be filled in that style.  The current path should not be 
%    clobbered by the PostScript code output by the macro.
%
%    \begin{macro}{slopes}
%    For the slopes fill style we produce PostScript code that
%    first puts the |slopecolors| parameter onto the stack.  Note that
%    the number of colours listed, which comes last in |slopecolors| is
%    now on the top of the stack.  Next come the |slopesteps| and
%    |slopeangle| parameters.  We switch to the dictionary established
%    by the \file{pst-slop.pro} Prolog and call |SlopesFill|, which
%    does the artwork and takes care to leave the path alone.
%    \begin{macrocode}
\def\psfs@slopes{%
 \addto@pscode{
  \psx@slopecolors\space
  \psslopesteps
  \psx@slopeangle 
  \ifPST@fading \psk@startfading \psk@endfading true \else false \fi
  tx@PstSlopeDict begin SlopesFill end}}
%    \end{macrocode}
%    \end{macro}
%
%    \begin{macro}{slope}
%    The |slope| style uses parameters |slopebegin| and |slopeend|
%    instead of |slopecolors|.  So the produced PostScript uses these
%    parameters to build a stack in |slopecolors| format.  The 
%    |\pst@usecolor| generates PostScript to set the current colour.
%    We can query the RGB values with |currentrgbcolor|.  
%    A |gsave|/|grestore| pair is used to avoid changing the
%    PostScript graphics state.  Once the stack is set up,
%    |SlopesFill| is called as before.
%    \begin{macrocode}
\def\psfs@slope{%
 \addto@pscode{%
  gsave
    0 \pst@usecolor\psslopebegin currentrgbcolor
    1 \pst@usecolor\psslopeend currentrgbcolor
    2
  grestore
  \psslopesteps \psx@slopeangle 
  \ifPST@fading \psk@startfading \psk@endfading true \else false \fi
  tx@PstSlopeDict begin SlopesFill end}}
%    \end{macrocode}
%    \end{macro}
%
%    \begin{macro}{ccslopes}
%    \begin{macro}{ccslope}
%    \begin{macro}{radslopes}
%    The code for the other fill styles is about the same, except for a few
%    parameters more or less and different PostScript procedures called
%    to do the work.
%    \begin{macrocode}
\def\psfs@ccslopes{%
 \addto@pscode{%
  \psx@slopecolors\space  
  \psslopesteps \psx@slopecenter\space \psx@sloperadius\space
  \ifPST@fading \psk@startfading \psk@endfading true \else false \fi
  tx@PstSlopeDict begin CcSlopesFill end}}
\def\psfs@ccslope{%
 \addto@pscode{%
  gsave 0 \pst@usecolor\psslopebegin currentrgbcolor
    1 \pst@usecolor\psslopeend currentrgbcolor
    2 grestore
  \psslopesteps \psx@slopecenter\space \psx@sloperadius\space
  \ifPST@fading \psk@startfading \psk@endfading true \else false \fi
  tx@PstSlopeDict begin CcSlopesFill end}}
\def\psfs@radslopes{%
 \addto@pscode{%
  \psx@slopecolors\space  
  \psslopesteps\psx@slopecenter\space\psx@sloperadius\space\psx@slopeangle
  \ifPST@fading \psk@startfading \psk@endfading true \else false \fi
  tx@PstSlopeDict begin RadSlopesFill end}}
%    \end{macrocode}
%    \end{macro}
%    \end{macro}
%    \end{macro}
%    \begin{macro}{radslope}
%    |radslope| is slightly different:  Just going from one colour to
%    another in 360 degrees is usually not what is wanted.  |radslope| just
%    does something pretty with the colours provided.
%    \begin{macrocode}
\def\psfs@radslope{%
 \addto@pscode{%
  gsave 0 \pst@usecolor\psslopebegin currentrgbcolor
    1 \pst@usecolor\psslopeend currentrgbcolor
    2 \pst@usecolor\psslopebegin currentrgbcolor
    3 \pst@usecolor\psslopeend currentrgbcolor
    4 \pst@usecolor\psslopebegin currentrgbcolor
    5 grestore
  \psslopesteps\psx@slopecenter\space\psx@sloperadius\space\psx@slopeangle
  \ifPST@fading \psk@startfading \psk@endfading true \else false \fi
  tx@PstSlopeDict begin RadSlopesFill end}}
%    \end{macrocode}
%    \end{macro}
%
%    \begin{macro}{\psBall}
%    \begin{macrocode}
\def\psBall{\pst@object{psBall}}
\def\psBall@i{\@ifnextchar(\psBall@ii{\psBall@ii(0,0)}}
\def\psBall@ii(#1,#2)#3#4{{%
  \pst@killglue
  \pssetlength\pst@dima{#4}%%%%%  20111025 hv
  \pst@dimb=\pst@dima%%%%%%%%%%%  20111025 hv
  \advance\pst@dima by 0.075\pst@dimb%
  \addbefore@par{sloperadius=\the\pst@dima,fillstyle=ccslope,
   slopebegin=white,slopeend=#3,slopecenter=0.4 0.6,linestyle=none}%
  \use@par%
  \pscircle(#1,#2){#4}%
  }\ignorespaces%
}
%    \end{macrocode}
%    \end{macro}
%
%
%    \begin{macrocode}
\catcode`\@=\TheAtCode\relax
%</texfile>
%    \end{macrocode}
%
% \subsection{The \file{pst-slpe.pro} file}
%    The file \file{pst-slpe.pro} contains PostScript definitions
%    to be included in the PostScript output by the 
%    |dvi|-to-PostScript converter, eg |dvips|.
%    First thing is to define a
%    dictionary to keep definitions local.
%    \begin{macrocode}
%<*prolog>
/tx@PstSlopeDict 60 dict def tx@PstSlopeDict begin
%    \end{macrocode}
%
%    \begin{macro}{Opacity++}
%    This macro increments the Opacity index
%    \begin{macrocode}
/Opacity 1 def % preset, no transparency
/Opacity++ { Opacity dOpacity add /Opacity ED } def
%    \end{macrocode}
%    \end{macro}
%    \begin{macro}{max}
%    $x1 \quad x2 \quad \mathtt{max}\quad max$\\
%    |max| is a utility function that calculates the maximum
%    of two numbers.
%    \begin{macrocode}
/max {2 copy lt {exch} if pop} bind def
%    \end{macrocode}
%    \end{macro}
%
%    \begin{macro}{Iterate}
%    $p_1\quad r_1\quad g_1\quad b_1\quad\ldots\quad
%    p_n\quad r_n\quad g_n\quad b_n\quad n\quad \mathtt{Iterate}\quad -$\\
%    This is the actual iteration, which goes through the colour
%    information and plots the segments.  
%    It uses the value of |NumSteps| which is set by the wrapper
%    procedures.  |DrawStep| is called all of |NumSteps| times, so 
%    it had better be fast.
%
%    First, the number of colour infos is read from the
%    top of the stack and decremented, to get the number of segments. 
%    \begin{macrocode}
/Iterate {
  1 sub /NumSegs ED
%    \end{macrocode}
%    Now we get the first colour.  This is really the {\em last}
%    colour given in the |slopecolors| argument.  We have to work
%    {\em down} the stack, so we shall be careful to plot the segments
%    in reverse order.  The |dup mul| stuff squares the RGB
%    components.  This does a kind-of-gamma correction, without
%    which primary colours tend to take up too much space in the
%    slope.  This is nothing deep, it just looks better in my opinion.
%    The following lines convert RGB to HSB and store the resulting
%    components, as well as the |Pt| coordinate in four variables.
%    \begin{macrocode}
  dup mul 3 1 roll dup mul 3 1 roll dup mul 3 1 roll
  setrgbcolor currenthsbcolor 
  /ThisB ED
  /ThisS ED
  /ThisH ED
  /ThisPt ED
%    \end{macrocode}
%    To avoid gaps, we fill the whole  path in that first colour.
%    \begin{macrocode}
  Opacity .setopacityalpha 
  gsave 
  fill 
  grestore
%    \end{macrocode}
%    The body of the following outer loop is executed
%    once for each segment.
%    It expects a current colour and |Pt| coordinate in the |This*|
%    variables and pops the next colour and point from the stack. It
%    then draws the single steps of that segment.
%    \begin{macrocode}
  NumSegs {
    dup mul 3 1 roll dup mul 3 1 roll dup mul 3 1 roll
    setrgbcolor currenthsbcolor
    /NextB ED
    /NextS ED
    /NextH ED
    /NextPt ED
%    \end{macrocode}
%    |NumSteps| always contains the remaining number of steps available.
%    These are evenly distributed between |Pt| coordinates |ThisPt| 
%    to 0, so for the current segment we may use 
%    $|NumSteps|*(|ThisPt|-|NextPt|)/|ThisPt|$ steps.
%    \begin{macrocode}
    ThisPt NextPt sub ThisPt div NumSteps mul cvi /SegSteps exch def
    /NumSteps NumSteps SegSteps sub def
%    \end{macrocode}
%    |SegSteps| may be zero.  In that case there is nothing to do for
%    this segment.
%    \begin{macrocode}
    SegSteps 0 eq not {
%    \end{macrocode}
%    If one of the colours is gray, ie~0 saturation, its hue is
%    useless.  In this case, instead of starting of with a random hue,
%    we take the hue of the other endpoint.  (If both have saturation
%    0, we have a pure gray scale and no harm is done)
%    \begin{macrocode}
      ThisS 0 eq {/ThisH NextH def} if
      NextS 0 eq {/NextH ThisH def} if
%    \end{macrocode}
%    To interpolate between two colours of different hue, we want to
%    go the shorter way around the colour circle.  The following code
%    assures that this happens if we go linearly from |This*| to 
%    |Next*| by conditionally adding 1.0 to one of the hue values.
%    The new hue values can lie between 0.0 and 2.0, so we will later
%    have to subtract 1.0 from values greater than one.
%    \begin{macrocode}
      ThisH NextH sub 0.5 gt
        {/NextH NextH 1.0 add def} 
        { NextH ThisH sub 0.5 ge {/ThisH ThisH 1.0 add def} if }
      ifelse
%    \end{macrocode}
%    We define three variables to hold the current colour coordinates
%    and calculate the corresponding increments per step.
%    \begin{macrocode}
      /B ThisB def
      /S ThisS def
      /H ThisH def
      /BInc NextB ThisB sub SegSteps div def
      /SInc NextS ThisS sub SegSteps div def
      /HInc NextH ThisH sub SegSteps div def
%    \end{macrocode}
%    The body of the following inner loop sets the current colour,
%    according to |H|, |S| and |B| and
%    undoes the kind-of-gamma correction by converting to RGB colour.
%    It then calls |DrawStep|, which draws one step and maybe updates
%    the current point or user space, or variables of its own.  Finally,
%    it increments the three colour variables.
%    \begin{macrocode}
      SegSteps {
        H dup 1. gt {1. sub} if S B sethsbcolor 
        currentrgbcolor 
        sqrt 3 1 roll sqrt 3 1 roll sqrt 3 1 roll
        setrgbcolor
        DrawStep
        /H H HInc add def
        /S S SInc add def
        /B B BInc add def
      } bind repeat
%    \end{macrocode}
%    The outer loop ends by moving on to the |Next| colour and point.
%
%    \begin{macrocode}
      /ThisH NextH def
      /ThisS NextS def
      /ThisB NextB def
      /ThisPt NextPt def
    } if
  } bind repeat
} def
%    \end{macrocode}
%    \end{macro}
%
%    \begin{macro}{PatchRadius}
%    $-\quad\mathtt{PatchRadius}\quad-$\\
%    This macro inspects the value of the variable |Radius|.  If it is
%    0, it is set to the maximum distance of any point in the 
%    current path from the origin of user space.  This has the effect
%    that the current path will be totally filled.  To find the maximum
%    distance, we flatten the path and call |UpdRR| for each endpoint
%    of the generated polygon.  The current maximum square distance is
%    gathered in |RR|.  
%    \begin{macrocode}
/PatchRadius {
  Radius 0 eq { 
    /UpdRR { dup mul exch dup mul add RR max /RR ED } bind def
    gsave
    flattenpath
    /RR 0 def
    {UpdRR} {UpdRR} {} {} pathforall
    grestore
    /Radius RR sqrt def
  } if
} def
%    \end{macrocode}
%    \end{macro}
%
%    \begin{macro}{SlopesFill}
%    $p_1\quad r_1\quad g_1\quad b_1\quad\ldots\quad
%    p_n\quad r_n\quad g_n\quad b_n\quad n\quad s\quad\alpha\quad    
%    \mathtt{SlopesFill}\quad -$\\
%    Fill the current path with a slope described by $p_1,\ldots,b_n,n$.
%    Use a total of $s$ single steps.  Rotate the slope by $\alpha$ 
%    degrees, 0 meaning $r_1,g_1,b_1$ left to $r_n,g_n,b_n$ right.
%
%    After saving the current path, we do the rotation and get the
%    number of steps, which is later needed by |Iterate|.  Remember,
%    that iterate calls |DrawStep| in the reverse order, ie~from
%    right to left.  We work around this by adding 180 degrees to 
%    the rotation.    Filling
%    works by clipping to the path and painting an appropriate sequence
%    of rectangles.  |DrawStep| is set up for |Iterate| to draw a 
%    rectangle of width |XInc| high enough to cover the whole
%    clippath (we use the Level 2 operator |rectfill| for speed) and
%    translate the user system by |XInc|.
%    \begin{macrocode}
/SlopesFill {
  /Fading ED		% do we have fading?
  Fading {
    /FadingEnd ED % the last opacity value
    dup /FadingStart ED % the first opacity value
    /Opacity ED % the opacity start value
  } if
  gsave
  180 add rotate
  /NumSteps ED
  Fading { /dOpacity FadingEnd FadingStart sub NumSteps div def } if
  clip
  pathbbox
  /h ED /w ED
  2 copy translate
  h sub neg /h ED
  w sub neg /w ED
  /XInc w NumSteps div def
  /DrawStep {
    Fading {			% do we have a fading?
      Opacity .setopacityalpha  % set opacity value
      Opacity++			% increase opacity
    } if
    0 0 XInc h rectfill
    XInc 0 translate
  } bind def
  Iterate
  grestore
} def
%    \end{macrocode}
%    \end{macro}
%
%    \begin{macro}{CcSlopesFill} $p_1\quad r_1\quad g_1\quad
%    b_1\quad\ldots\quad p_n\quad r_n\quad g_n\quad b_n\quad n\quad
%    c_x\quad c_y \quad r\quad \mathtt{CcSlopesFill}\quad -$\\ Fills
%    the current path with a concentric pattern,
%    ie~in a polar coordinate system, the colour depends on the
%    radius and not on the angle.
%    Centered around a point with coordinates $(c_x,c_y)$ relative to
%    the bounding box of the path, ie~for a rectangle, $(0,0)$ will
%    center the pattern around the lower left corner of the rectangle,
%    $(0.5,0.5)$ around its center.  The largest circle has a radius of
%    $r$.  If $r=0$, $r$ is taken to be the maximum distance of any
%    point on the current path from the center defined by $(c_x,c_y)$.
%    The colours are given from the center outwards,
%    ie~$(r_1,g_1,b_1)$ describe the colour at the center.
%
%    The code is similar to that of |SlopesFill|.  The main differences
%    are the call to |PatchRadius|, which catches the case that $r=0$
%    and the different definition for |DrawStep|, Which now fills a
%    circle of radius |Rad| and decreases that Variable.  Of course,
%    drawing starts on the outside, so we work down the stack and circles
%    drawn later partially cover those drawn first.  Painting
%    non-overlapping, `donut-shapes' would be slower. 
%    \begin{macrocode}
/CcSlopesFill {
  /Fading ED		% do we have fading?
  Fading {
    /FadingEnd ED % the last opacity value
    dup /FadingStart ED % the first opacity value
    /Opacity ED % the opacity start value
  } if
  gsave
  /Radius ED
  /CenterY ED
  /CenterX ED
  /NumSteps ED
  Fading { /dOpacity FadingEnd FadingStart sub NumSteps div def } if
  clip
  pathbbox
  /h ED /w ED
  2 copy translate
  h sub neg /h ED
  w sub neg /w ED
  w CenterX mul h CenterY mul translate
  PatchRadius
  /RadPerStep Radius NumSteps div neg def
  /Rad Radius def
  /DrawStep {
    Fading {			% do we have a fading?
      Opacity .setopacityalpha  % set opacity value
      Opacity++			% increase opacity
    } if
    0 0 Rad 0 360 arc
    closepath fill
    /Rad Rad RadPerStep add def
  } bind def
  Iterate
  grestore
} def
%    \end{macrocode}
%    \end{macro}
%
%    \begin{macro}{RadSlopesFill}
%    $p_1\quad r_1\quad g_1\quad b_1\quad\ldots
%    \quad p_n\quad r_n\quad g_n\quad b_n\quad n\quad
%    c_x\quad c_y \quad r\quad\alpha\quad \mathtt{CcSlopesFill}\quad -$\\
%    This fills the current path with a radial pattern, ie~in a
%    polar coordinate system the colour depends on the angle and not on
%    the radius.  All this is very similar to |CcSlopesFill|.  There
%    is an extra parameter $\alpha$, which rotates the pattern.
%
%    The only new thing in the code is the |DrawStep| procedure.
%    This does {\em not} draw a circular arc, but a triangle, which is
%    considerably faster.  One of the short sides of the triangle is
%    determined by |Radius|, the other one by |dY|, which is calculated
%    as $|dY|:=|Radius|\times\tan(|AngleIncrement|)$.
%    \begin{macrocode}
/RadSlopesFill {
  /Fading ED		% do we have fading?
  Fading {
    /FadingEnd ED % the last opacity value
    dup /FadingStart ED % the first opacity value
    /Opacity ED % the opacity start value
  } if
  gsave
  rotate
  /Radius ED
  /CenterY ED
  /CenterX ED
  /NumSteps ED
  Fading { /dOpacity FadingEnd FadingStart sub NumSteps div def } if
  clip
  pathbbox
  /h ED /w ED
  2 copy translate
  h sub neg /h ED
  w sub neg /w ED
  w CenterX mul h CenterY mul translate
  PatchRadius
  /AngleIncrement 360 NumSteps div neg def
  /dY AngleIncrement sin AngleIncrement cos div Radius mul def
  /DrawStep {
    Fading {			% do we have a fading?
      Opacity .setopacityalpha  % set opacity value
      Opacity++			% increase opacity
    } if
    0 0 moveto
    Radius 0 rlineto
    0 dY rlineto
    closepath fill
    AngleIncrement rotate
  } bind def
  Iterate
  grestore
} def
%    \end{macrocode}
%    \end{macro}
%
% Last, but not least, we have to close the private dictionary.
%    \begin{macrocode}
end
%</prolog>
%    \end{macrocode}
% \Finale
%
'
%\end{verbatim}
%\section{Package Usage}
% To use |pst-slpe|, you have to say
% \begin{verbatim}
%   \usepackage{pst-slpe}
% \end{verbatim}
% in the document prologue for \LaTeX, and 
% \begin{verbatim}
%   \input pst-slpe.tex
% \end{verbatim}
% in ``plain'' \TeX.
%
% \section{New macro and fill styles}
% \DescribeMacro{\psBall}
%    It takes the (optional) coordinates of the ball center, the color
%    and the radius as parameter and uses |\pscircle| for painting
%    the bullet. 
%
%    \vspace{1cm}
%    \psBall{black}{2ex}
%    \psBall(1,0){blue}{3ex}
%    \psBall(2.5,0){red}{4ex}
%    \psBall(4,0){green!50!blue!60}{5ex}
%
%    \vspace{1cm}
% \begin{verbatim}
%    \psBall{black}{2ex}
%    \psBall(1,0){blue}{3ex}
%    \psBall(2.5,0){red}{4ex}
%    \psBall(4,0){green!50!blue!60}{5ex}
% \end{verbatim}
%
%    The predinied options can be overwritten in the usual way:
%
%    \vspace{1cm}
%    \psBall{black}{2ex}
%    \psBall[sloperadius=10pt](1,0){blue}{3ex}
%    \psBall(2.5,0){red}{4ex}
%    \psBall[slopebegin=red](4,0){green!50!blue!60}{5ex}
%
%    \vspace{1cm}
% \begin{verbatim}
%    \psBall{black}{2ex}
%    \psBall[sloperadius=10pt](1,0){blue}{3ex}
%    \psBall(2.5,0){red}{4ex}
%    \psBall[slopebegin=red](4,0){green!50!blue!60}{5ex}
% \end{verbatim}
%
% \DescribeMacro{slope}
% \DescribeMacro{slopes}
% \DescribeMacro{ccslope}
% \DescribeMacro{ccslopes}
% \DescribeMacro{radslope}
% \DescribeMacro{radslopes}
% |pst-slpe| provides six new fill styles called |slope|, |slopes|,
% |ccslope|, |ccslopes|, |radslope| and |radslopes|.  These obviously
% come in pairs: The $\ldots$|slope|-styles are simplified versions of
% the general $\ldots$|slopes|-styles.\footnote{By the way, I use slope
% as a synonym for gradient.  It sounds less pretentious and avoids
% name clashes.}  The |cc|$\ldots$ styles paint concentric patterns,
% and the |rad|$\ldots$ styles do radial ones.  
%
% Here is a little
% overview of what they look like:
% \newcommand{\st}{$\vcenter to30pt{}$}
% \begin{quote}\LARGE
%  \begin{tabular}{cc}
%   \psframebox[fillstyle=slope]{\st|slope|} &\qquad
%   \psframebox[fillstyle=slopes]{\st|slopes|} \\[2ex]
%   \psframebox[fillstyle=ccslope]{\st|ccslope|} &\qquad
%   \psframebox[fillstyle=ccslopes]{\st|ccslopes|} \\[2ex]
%   \psframebox[fillstyle=radslope]{\st|radslope|} &\qquad
%   \psframebox[fillstyle=radslopes]{\st|radslopes|} \\[2ex]
%  \end{tabular}
% \end{quote}
% These examples were produced by saying simply
% \begin{verbatim}
%   \psframebox[fillstyle=slope]{...}
% \end{verbatim}
% etc.~without setting any further graphics parameters.  The package
% provides a number of parameters that can be used to control
% the way these patterns
% are painted.
% \medskip
%
% \DescribeMacro{slopebegin}
% \DescribeMacro{slopeend}
% The graphics parameters |slopebegin| and |slopeend| set the colours
% between which the three $\ldots$|slope| styles should interpolate.
% Eg,
% \begin{verbatim}
%   \psframebox[fillstyle=slope,slopebegin=red,slopeend=green]{...}
% \end{verbatim}
% produces:
% \begin{quote}\Large
%  \psframebox[fillstyle=slope,slopebegin=red,slopeend=green]{\st slopes!}
% \end{quote}
% The same settings of |slopebegin| and |slopeend| for the |ccslope|
% and |radslope| fillstyles produce
% \begin{quote}\Large
%  \psframebox[fillstyle=ccslope,slopebegin=red,slopeend=green]{\st slopes!}
%   \quad{\normalsize resp.}\quad
%  \psframebox[fillstyle=radslope,slopebegin=red,slopeend=green]{\st slopes!}
% \end{quote}
% The default settings go from a greenish yellow to pure blue.
% \medskip
%
% \DescribeMacro{slopecolors}
% If you want to interpolate between more than two colours, you have
% to use the $\ldots$|slopes| styles, which are controlled by the 
% |slopecolors| parameter instead of |slopebegin| and |slopeend|.  The 
% idea is to specify the colour to use at certain points `on the
% way'.  To fill a shape with |slopes|, imagine a linear scale 
% from its left edge to its right edge.  The left edge must lie at
% coordinate 0.  Pick an arbitrary value for the right edge, say 23.
% Now you want to get light yellow at the left edge, a pastel green at $17/23$ 
% of the way and dark cyan at the right edge, like this:
% \begin{quote}\psset{unit=0.45cm}
%  \begin{pspicture}(-1,0)(24,6)
%   \pscustom[fillstyle=slopes,
% slopecolors=0 1 1 .9  17 .5 1 .5  23 0 0.5 0.5  3]{
%    \psccurve(0,2.5)(12,3.5)(20,4)(23,2)(17,2.5)}
%   \psaxes(0,5)(-0.01,5)(23.01,5)
%   \psline(0,5)(0,1)
%   \psline(17,5)(17,1)
%   \psline(23,5)(23,1)
%  \end{pspicture}
% \end{quote}
% The RGB values for the three colours are $(1,1,0.9)$, $(0.5,1,0.5)$
% and $(0,0.5,0.5)$.  The value for the |slopecolors| parameter is a list
% of `colour infos' followed by the number of `colour infos'. 
% Each `colour info' consists
% of the coordinate value where a colour is to be specified, followed by
% the RGB values of that colour.  All these values are separated by
% white space.  The correct setting for the example is thus:
% \begin{verbatim}
%   slopecolors=0 1 1 .9   17 .5 1 .5   23 0 .5 .5   3
% \end{verbatim}
% For |ccslopes|, specify the colours from the center outward.  
% For |radslopes| (with no rotation specified), 0 represents the ray
% going `eastward'.  Specify the colours anti-clockwise.  If you want a
% smooth gradient at the beginning and starting ray of |radslopes|, you
% should pick the first and last colours identical.
%
% Please note, that the |slopecolors| parameter is not subject to any
% parsing on the \TeX\ side.  If you forget a number or specify the wrong
% number of segments, the PostScript interpreter will probably crash.
%
% The default value for |slopecolors| specifies a rainbow.
%
% \medskip
%
% \DescribeMacro{slopesteps}
% The parameter |slopesteps| controls the number of distinct colour steps
% rendered.  Higher values for this parameter result in better quality
% but proportionally slower rendering.  Eg, setting
% |slopesteps| to 5 with the |slope| fill style results in
% \begin{quote}\Large
%  \psframebox[fillstyle=slope,slopesteps=5]{\st slopes!}
% \end{quote}
%
% The default value is 100, which
% suffices for most purposes.  Remember that the number of distinct colours
% reproducible by a given device is limited.  Pushing |slopesteps| to
% high will result only in loss of performance at no gain in quality.
% \medskip
%
% \DescribeMacro{slopeangle}
% The |slope(s)| and |radslope(s)| patterns may be rotated.  As usual,
% the angles are given anti-clockwise.  Eg, an angle of 30 degrees
% gives
% \begin{quote}\Large\psset{slopeangle=30}
%  \psframebox[fillstyle=slope]{\st slopes!}
%   \quad{\normalsize and}\quad
%  \psframebox[fillstyle=radslope]{\st slopes!}
% \end{quote}
% with the |slope| and |radslope| fillstyles.
% \medskip
%
% \DescribeMacro{slopecenter}
% For the |cc|$\ldots$ and |rad|$\ldots$ styles, it is possible to
% set the center of the pattern.  The |slopecenter| parameter is set to
% the coordinates of that center relative to the bounding box of the
% current path.  The following effect:
% \begin{quote}\psset{unit=0.45cm}
%  \begin{pspicture}(-1,-1)(24,5)
%   \pscustom[fillstyle=radslope,slopecenter=0.2 0.4]{
%    \pspolygon(0,2.5)(12,2.5)(20,4)(23,2)(17,2.5)(3,0)}
%   \psaxes[axesstyle=frame,Dx=0.1,dx=2.2999,Dy=0.2,dy=0.7999](0,0)(23,4) 
%   \psline(4.6,0)(4.6,4)
%   \psline(0,1.6)(23,1.6)
%  \end{pspicture}
% \end{quote}
%  was achieved with 
% \begin{verbatim}
%    fillstyle=radslope,slopecenter=0.2 0.4
% \end{verbatim}
% The default value for |slopecenter| is |0.5 0.5|, which is the
% center for symmetrical shapes.  Note that this parameter is not
% parsed by \TeX, so setting it to anything else than two numbers 
% between 0 and 1 might crash the PostScript interpreter.
% \medskip
%
% \DescribeMacro{sloperadius}
% Normally, the |cc|$\ldots$ and |rad|$\ldots$ styles distribute the
% given colours so that the center is painted in the first colour given,
% and the points of the shape furthest from the center are painted in 
% the last colour.  In other words the maximum radius to which the
% |slopecolors| parameter refers is the maximum distance from the
% center (defined by |slopecenter|) to any point on the periphery
% of the shape.  This radius can be explicitly set with |sloperadius|.
% Eg, setting |sloperadius=0.5cm| gives
% \begin{quote}\Large\psset{sloperadius=0.5cm}
%  \psframebox[fillstyle=ccslope]{\st slopes!}
% \end{quote}
% Any point further from the center than the given |sloperadius| is
% painted with the last colour in |slopeclours|, resp.~|slopeend|.
%
% The default value for |sloperadius| is 0, which invokes the default
% behaviour of automatically calculating the radius.
%
% \DescribeMacro{fading}
% \DescribeMacro{startfading}
% \DescribeMacro{endfading}
% The optional boolean keyword |fading| allows a transparency effect of
% the filled area, starting with the opacity value |startfading| and
% ending with the value of |endfading|. Both values must be of the
% intervall [0\ldots1], with 0 for total opacity and 1 for no
% opacity. The values are preset by 0 and 1.
%
% Here is a little
% overview of what they look like:
% \begin{quote}\LARGE\color{white}
%  \begin{tabular}{cc}
%   \psframe*(-0.3,-0.25)(3,20pt)\psframebox[fading,fillstyle=slope]{\st|  slope  |} &\qquad
%   \psframe*(-0.3,-0.25)(3,20pt)\psframebox[fading,fillstyle=slopes]{\st| slopes |} \\[2ex]
%   \psframe*(-0.3,-0.25)(3,20pt)\psframebox[fading,fillstyle=ccslope]{\st|ccslope |} &\qquad
%   \psframe*(-0.3,-0.25)(3,20pt)\psframebox[fading,fillstyle=ccslopes]{\st|ccslopes|} \\[2ex]
%   \psframe*(-0.3,-0.25)(3,20pt)\psframebox[fading,fillstyle=radslope]{\st|radslope|} &\qquad
%   \psframe*(-0.3,-0.25)(3,20pt)\psframebox[fading,fillstyle=radslopes]{\st|radslopes|} \\[2ex]
%  \end{tabular}
% \end{quote}
% \color{black}
%
% These examples were produced by saying simply
% \begin{verbatim}
%   \psframebox[fading,fillstyle=...]{...}
% \end{verbatim}
%
% \begin{quote}\LARGE\color{white}
% \psset{fading,startfading=0.3,endfading=0.8}
%  \begin{tabular}{cc}
%   \psframe*(-0.3,-0.25)(3,20pt)\psframebox[fillstyle=slope]{\st|  slope  |} &\qquad
%   \psframe*(-0.3,-0.25)(3,20pt)\psframebox[fillstyle=slopes]{\st| slopes |} \\[2ex]
%   \psframe*(-0.3,-0.25)(3,20pt)\psframebox[fillstyle=ccslope]{\st|ccslope |} &\qquad
%   \psframe*(-0.3,-0.25)(3,20pt)\psframebox[fillstyle=ccslopes]{\st|ccslopes|} \\[2ex]
%   \psframe*(-0.3,-0.25)(3,20pt)\psframebox[fillstyle=radslope]{\st|radslope|} &\qquad
%   \psframe*(-0.3,-0.25)(3,20pt)\psframebox[fillstyle=radslopes]{\st|radslopes|} \\[2ex]
%  \end{tabular}
% \end{quote}
% \color{black}
%
% These examples were produced by saying simply
% \begin{verbatim}
%   \psframebox[fading,startfading=0.3,endfading=0.8,fillstyle=...]{...}
% \end{verbatim}
%
% \StopEventually{}
%
%\section{The Code}
% \subsection{Producing the documentation}
%
%    A short driver is provided that can be extracted if necessary by
%    the \textsc{docstrip} program provided with \LaTeXe.
%    \begin{macrocode}
%<*driver>
\NeedsTeXFormat{LaTeX2e}
\documentclass{ltxdoc}
\usepackage{pst-slpe}
\usepackage{pst-plot}
\DisableCrossrefs
\MakeShortVerb{\|}
\newcommand\Lopt[1]{\textsf{#1}}
\newcommand\file[1]{\texttt{#1}}
\AtEndDocument{
\PrintChanges
\PrintIndex
}
%\OnlyDescription
\begin{document}
\DocInput{pst-slpe.dtx}
\end{document}
%</driver>
%    \end{macrocode}
%
% \subsection{The \file{pst-slpe.sty} file}
%    The \file{pst-slpe.sty} file is very simple.  It just loads 
%    the generic \file{pst-slpe.tex} file. 
%    \begin{macrocode}
%<*stylefile>
\RequirePackage{pstricks}
\ProvidesPackage{pst-slpe}[2005/03/05 package wrapper for `pst-slpe.tex']
%%
%% This is file `pst-slpe.tex',
%% generated with the docstrip utility.
%%
%% The original source files were:
%%
%% pst-slpe.dtx  (with options: `texfile')
%% 
%% IMPORTANT NOTICE:
%% 
%% For the copyright see the source file.
%% 
%% Any modified versions of this file must be renamed
%% with new filenames distinct from pst-slpe.tex.
%% 
%% For distribution of the original source see the terms
%% for copying and modification in the file pst-slpe.dtx.
%% 
%% This generated file may be distributed as long as the
%% original source files, as listed above, are part of the
%% same distribution. (The sources need not necessarily be
%% in the same archive or directory.)
%% This program can be redistributed and/or modified under the terms
%% of the LaTeX Project Public License Distributed from CTAN archives
%% in directory macros/latex/base/lppl.txt.
%%
\def\pstslpefileversion{1.31}
\def\pstslpefiledate{2011/10/25}
\message{ v\pstslpefileversion, \pstslpefiledate}
\csname PstSlopeLoaded\endcsname
\let\PstSlopeLoaded\endinput
\ifx\PSTricksLoaded\endinput\else
  \def\next{\input pstricks.tex }\expandafter\next
\fi
\ifx\PSTXKeyLoaded\endinput\else\input pst-xkey \fi % --> hv
\edef\TheAtCode{\the\catcode`\@}
\catcode`\@=11
\pst@addfams{pst-slpe}                              % --> hv
\pstheader{pst-slpe.pro}
\newrgbcolor{slopebegin}{0.9 1 0}
\define@key[psset]{pst-slpe}{slopebegin}{\pst@getcolor{#1}\psslopebegin}% --> hv
\psset[pst-slpe]{slopebegin=slopebegin}                                 % --> hv

\newrgbcolor{slopeend}{0 0 1}
\define@key[psset]{pst-slpe}{slopeend}{\pst@getcolor{#1}\psslopeend}% --> hv
\psset[pst-slpe]{slopeend=slopeend}% --> hv

\define@key[psset]{pst-slpe}{slopesteps}{\pst@getint{#1}\psslopesteps}% --> hv
\psset[pst-slpe]{slopesteps=100}% --> hv

\define@key[psset]{pst-slpe}{slopeangle}{\pst@getangle{#1}\psx@slopeangle}% --> hv
\psset[pst-slpe]{slopeangle=0}% --> hv
\define@key[psset]{pst-slpe}{slopecolors}{\def\psx@slopecolors{#1}}% --> hv
\psset[pst-slpe]{slopecolors={% --> hv
0.0 1 0 0
0.4 0 1 0
0.8 0 0 1
1.0 1 0 1
4}}
\define@key[psset]{pst-slpe}{slopecenter}{\def\psx@slopecenter{#1}}% --> hv
\psset[pst-slpe]{slopecenter={0.5 0.5}}% --> hv
\define@key[psset]{pst-slpe}{sloperadius}{\pst@getlength{#1}\psx@sloperadius}% --> hv
\psset[pst-slpe]{sloperadius=0}% --> hv
\define@boolkey[psset]{pst-slpe}[PST@]{fading}[true]{}% --> hv
\psset[pst-slpe]{fading=false}% --> hv
\define@key[psset]{pst-slpe}{startfading}{\pst@checknum{#1}\psk@startfading }% --> hv
\define@key[psset]{pst-slpe}{endfading}{\pst@checknum{#1}\psk@endfading }% --> hv
\psset[pst-slpe]{startfading=0,endfading=1}% --> hv
\def\psfs@slopes{%
 \addto@pscode{
  \psx@slopecolors\space
  \psslopesteps
  \psx@slopeangle
  \ifPST@fading \psk@startfading \psk@endfading true \else false \fi
  tx@PstSlopeDict begin SlopesFill end}}
\def\psfs@slope{%
 \addto@pscode{%
  gsave
    0 \pst@usecolor\psslopebegin currentrgbcolor
    1 \pst@usecolor\psslopeend currentrgbcolor
    2
  grestore
  \psslopesteps \psx@slopeangle
  \ifPST@fading \psk@startfading \psk@endfading true \else false \fi
  tx@PstSlopeDict begin SlopesFill end}}
\def\psfs@ccslopes{%
 \addto@pscode{%
  \psx@slopecolors\space
  \psslopesteps \psx@slopecenter\space \psx@sloperadius\space
  \ifPST@fading \psk@startfading \psk@endfading true \else false \fi
  tx@PstSlopeDict begin CcSlopesFill end}}
\def\psfs@ccslope{%
 \addto@pscode{%
  gsave 0 \pst@usecolor\psslopebegin currentrgbcolor
    1 \pst@usecolor\psslopeend currentrgbcolor
    2 grestore
  \psslopesteps \psx@slopecenter\space \psx@sloperadius\space
  \ifPST@fading \psk@startfading \psk@endfading true \else false \fi
  tx@PstSlopeDict begin CcSlopesFill end}}
\def\psfs@radslopes{%
 \addto@pscode{%
  \psx@slopecolors\space
  \psslopesteps\psx@slopecenter\space\psx@sloperadius\space\psx@slopeangle
  \ifPST@fading \psk@startfading \psk@endfading true \else false \fi
  tx@PstSlopeDict begin RadSlopesFill end}}
\def\psfs@radslope{%
 \addto@pscode{%
  gsave 0 \pst@usecolor\psslopebegin currentrgbcolor
    1 \pst@usecolor\psslopeend currentrgbcolor
    2 \pst@usecolor\psslopebegin currentrgbcolor
    3 \pst@usecolor\psslopeend currentrgbcolor
    4 \pst@usecolor\psslopebegin currentrgbcolor
    5 grestore
  \psslopesteps\psx@slopecenter\space\psx@sloperadius\space\psx@slopeangle
  \ifPST@fading \psk@startfading \psk@endfading true \else false \fi
  tx@PstSlopeDict begin RadSlopesFill end}}
\def\psBall{\pst@object{psBall}}
\def\psBall@i{\@ifnextchar(\psBall@ii{\psBall@ii(0,0)}}
\def\psBall@ii(#1,#2)#3#4{{%
  \pst@killglue
  \pssetlength\pst@dima{#4}%%%%%  20111025 hv
  \pst@dimb=\pst@dima%%%%%%%%%%%  20111025 hv
  \advance\pst@dima by 0.075\pst@dimb%
  \addbefore@par{sloperadius=\the\pst@dima,fillstyle=ccslope,
   slopebegin=white,slopeend=#3,slopecenter=0.4 0.6,linestyle=none}%
  \use@par%
  \pscircle(#1,#2){#4}%
  }\ignorespaces%
}
\catcode`\@=\TheAtCode\relax
\endinput
%%
%% End of file `pst-slpe.tex'.

\ProvidesFile{pst-slpe.tex}
  [\pstslpefiledate\space v\pstslpefileversion\space 
    `pst-slpe' (mg,hv)]
\IfFileExists{pst-slpe.pro}{%
  \ProvidesFile{pst-slpe.pro}
    [2008/06/19 v. 0.01,  PostScript prologue file (hv)]
    \@addtofilelist{pst-slpe.pro}}{}%
%</stylefile>
%    \end{macrocode}
%
% \subsection{The \file{pst-slpe.tex} file}
%    \file{pst-slpe.tex} contains the \TeX-side of things.  We begin
%    by identifying ourselves and setting things up, the same as in 
%    other PSTricks packages.
%    \begin{macrocode}
%<*texfile>
\message{ v\pstslpefileversion, \pstslpefiledate}
\csname PstSlopeLoaded\endcsname
\let\PstSlopeLoaded\endinput
\ifx\PSTricksLoaded\endinput\else
  \def\next{\input pstricks.tex }\expandafter\next
\fi
\ifx\PSTXKeyLoaded\endinput\else\input pst-xkey \fi % --> hv
\edef\TheAtCode{\the\catcode`\@}
\catcode`\@=11
\pst@addfams{pst-slpe}                              % --> hv
\pstheader{pst-slpe.pro}
%    \end{macrocode}
%    \begin{macro}{slopebegin}
%    \begin{macro}{slopeend}
%    \begin{macro}{slopesteps}
%    \begin{macro}{slopeangle}
%
% \subsubsection{New graphics parameters}
%    We now define the various new parameters needed by the |slope|
%    fill styles and install default values.  First come the colours, 
%    ie~graphics parameters |slopebegin| and |slopeend|, followed
%    by the number of steps, |slopesteps|, and the rotation angle, 
%    |slopeangle|.
%    \begin{macrocode}
\newrgbcolor{slopebegin}{0.9 1 0}
\define@key[psset]{pst-slpe}{slopebegin}{\pst@getcolor{#1}\psslopebegin}% --> hv
\psset[pst-slpe]{slopebegin=slopebegin}                                 % --> hv

\newrgbcolor{slopeend}{0 0 1}
\define@key[psset]{pst-slpe}{slopeend}{\pst@getcolor{#1}\psslopeend}% --> hv
\psset[pst-slpe]{slopeend=slopeend}% --> hv

\define@key[psset]{pst-slpe}{slopesteps}{\pst@getint{#1}\psslopesteps}% --> hv
\psset[pst-slpe]{slopesteps=100}% --> hv

\define@key[psset]{pst-slpe}{slopeangle}{\pst@getangle{#1}\psx@slopeangle}% --> hv
\psset[pst-slpe]{slopeangle=0}% --> hv
%    \end{macrocode}
%    \end{macro}
%    \end{macro}
%    \end{macro}
%    \end{macro}
%    \begin{macro}{slopecolors}
%    The value for |slopecolors| is not parsed.  It is directly copied 
%    to the PostScript output.  This is certainly not the way it
%    should be, but it's simple.  The default value is a rainbow from
%    red to magenta.
%    \begin{macrocode}
\define@key[psset]{pst-slpe}{slopecolors}{\def\psx@slopecolors{#1}}% --> hv
\psset[pst-slpe]{slopecolors={% --> hv
0.0 1 0 0		
0.4 0 1 0
0.8 0 0 1
1.0 1 0 1
4}}
%    \end{macrocode}
%    \end{macro}
%    \begin{macro}{slopecenter}
%    The argument to |slopecenter| isn't parsed either.  But there's
%    probably not much that can go wrong with two decimal numbers.
%    \begin{macrocode}
\define@key[psset]{pst-slpe}{slopecenter}{\def\psx@slopecenter{#1}}% --> hv
\psset[pst-slpe]{slopecenter={0.5 0.5}}% --> hv
%    \end{macrocode}
%    \end{macro}
%    \begin{macro}{sloperadius}
%    The default value for |sloperadius| is 0, which makes the
%    PostScript procedure |PatchRadius| determine a value for the radius.
%    \begin{macrocode}
\define@key[psset]{pst-slpe}{sloperadius}{\pst@getlength{#1}\psx@sloperadius}% --> hv
\psset[pst-slpe]{sloperadius=0}% --> hv
%    \end{macrocode}
%    \end{macro}
%    \begin{macro}{fading}
%    The default value for |fading| is false, which is no transparency effect at all.
%    With |fading=true| the package takes the values |startfading| and |endfading|
%    into account for the opacity effect of the filled area.
%    \end{macro}
%    \begin{macrocode}
\define@boolkey[psset]{pst-slpe}[PST@]{fading}[true]{}% --> hv
\psset[pst-slpe]{fading=false}% --> hv
%    \end{macrocode}
%    \begin{macro}{startfading}
%    The relativ number for the starting value (0\ldots1), preset by 0.
%    \end{macro}
%    \begin{macrocode}
\define@key[psset]{pst-slpe}{startfading}{\pst@checknum{#1}\psk@startfading }% --> hv
%    \end{macrocode}
%    \begin{macro}{endfading}
%    The relativ number for the end value (0\ldots1), preset by 1.
%    \end{macro}
%    \begin{macrocode}
\define@key[psset]{pst-slpe}{endfading}{\pst@checknum{#1}\psk@endfading }% --> hv
\psset[pst-slpe]{startfading=0,endfading=1}% --> hv
%    \end{macrocode}
%
% \subsubsection{Fill style macros}
%
%    Now come the fill style definitions that use these parameters.
%    There is one macro for each fill style named |\psfs@|$style$.  
%    PSTricks calls this macro whenever the current path needs to
%    be filled in that style.  The current path should not be 
%    clobbered by the PostScript code output by the macro.
%
%    \begin{macro}{slopes}
%    For the slopes fill style we produce PostScript code that
%    first puts the |slopecolors| parameter onto the stack.  Note that
%    the number of colours listed, which comes last in |slopecolors| is
%    now on the top of the stack.  Next come the |slopesteps| and
%    |slopeangle| parameters.  We switch to the dictionary established
%    by the \file{pst-slop.pro} Prolog and call |SlopesFill|, which
%    does the artwork and takes care to leave the path alone.
%    \begin{macrocode}
\def\psfs@slopes{%
 \addto@pscode{
  \psx@slopecolors\space
  \psslopesteps
  \psx@slopeangle 
  \ifPST@fading \psk@startfading \psk@endfading true \else false \fi
  tx@PstSlopeDict begin SlopesFill end}}
%    \end{macrocode}
%    \end{macro}
%
%    \begin{macro}{slope}
%    The |slope| style uses parameters |slopebegin| and |slopeend|
%    instead of |slopecolors|.  So the produced PostScript uses these
%    parameters to build a stack in |slopecolors| format.  The 
%    |\pst@usecolor| generates PostScript to set the current colour.
%    We can query the RGB values with |currentrgbcolor|.  
%    A |gsave|/|grestore| pair is used to avoid changing the
%    PostScript graphics state.  Once the stack is set up,
%    |SlopesFill| is called as before.
%    \begin{macrocode}
\def\psfs@slope{%
 \addto@pscode{%
  gsave
    0 \pst@usecolor\psslopebegin currentrgbcolor
    1 \pst@usecolor\psslopeend currentrgbcolor
    2
  grestore
  \psslopesteps \psx@slopeangle 
  \ifPST@fading \psk@startfading \psk@endfading true \else false \fi
  tx@PstSlopeDict begin SlopesFill end}}
%    \end{macrocode}
%    \end{macro}
%
%    \begin{macro}{ccslopes}
%    \begin{macro}{ccslope}
%    \begin{macro}{radslopes}
%    The code for the other fill styles is about the same, except for a few
%    parameters more or less and different PostScript procedures called
%    to do the work.
%    \begin{macrocode}
\def\psfs@ccslopes{%
 \addto@pscode{%
  \psx@slopecolors\space  
  \psslopesteps \psx@slopecenter\space \psx@sloperadius\space
  \ifPST@fading \psk@startfading \psk@endfading true \else false \fi
  tx@PstSlopeDict begin CcSlopesFill end}}
\def\psfs@ccslope{%
 \addto@pscode{%
  gsave 0 \pst@usecolor\psslopebegin currentrgbcolor
    1 \pst@usecolor\psslopeend currentrgbcolor
    2 grestore
  \psslopesteps \psx@slopecenter\space \psx@sloperadius\space
  \ifPST@fading \psk@startfading \psk@endfading true \else false \fi
  tx@PstSlopeDict begin CcSlopesFill end}}
\def\psfs@radslopes{%
 \addto@pscode{%
  \psx@slopecolors\space  
  \psslopesteps\psx@slopecenter\space\psx@sloperadius\space\psx@slopeangle
  \ifPST@fading \psk@startfading \psk@endfading true \else false \fi
  tx@PstSlopeDict begin RadSlopesFill end}}
%    \end{macrocode}
%    \end{macro}
%    \end{macro}
%    \end{macro}
%    \begin{macro}{radslope}
%    |radslope| is slightly different:  Just going from one colour to
%    another in 360 degrees is usually not what is wanted.  |radslope| just
%    does something pretty with the colours provided.
%    \begin{macrocode}
\def\psfs@radslope{%
 \addto@pscode{%
  gsave 0 \pst@usecolor\psslopebegin currentrgbcolor
    1 \pst@usecolor\psslopeend currentrgbcolor
    2 \pst@usecolor\psslopebegin currentrgbcolor
    3 \pst@usecolor\psslopeend currentrgbcolor
    4 \pst@usecolor\psslopebegin currentrgbcolor
    5 grestore
  \psslopesteps\psx@slopecenter\space\psx@sloperadius\space\psx@slopeangle
  \ifPST@fading \psk@startfading \psk@endfading true \else false \fi
  tx@PstSlopeDict begin RadSlopesFill end}}
%    \end{macrocode}
%    \end{macro}
%
%    \begin{macro}{\psBall}
%    \begin{macrocode}
\def\psBall{\pst@object{psBall}}
\def\psBall@i{\@ifnextchar(\psBall@ii{\psBall@ii(0,0)}}
\def\psBall@ii(#1,#2)#3#4{{%
  \pst@killglue
  \pssetlength\pst@dima{#4}%%%%%  20111025 hv
  \pst@dimb=\pst@dima%%%%%%%%%%%  20111025 hv
  \advance\pst@dima by 0.075\pst@dimb%
  \addbefore@par{sloperadius=\the\pst@dima,fillstyle=ccslope,
   slopebegin=white,slopeend=#3,slopecenter=0.4 0.6,linestyle=none}%
  \use@par%
  \pscircle(#1,#2){#4}%
  }\ignorespaces%
}
%    \end{macrocode}
%    \end{macro}
%
%
%    \begin{macrocode}
\catcode`\@=\TheAtCode\relax
%</texfile>
%    \end{macrocode}
%
% \subsection{The \file{pst-slpe.pro} file}
%    The file \file{pst-slpe.pro} contains PostScript definitions
%    to be included in the PostScript output by the 
%    |dvi|-to-PostScript converter, eg |dvips|.
%    First thing is to define a
%    dictionary to keep definitions local.
%    \begin{macrocode}
%<*prolog>
/tx@PstSlopeDict 60 dict def tx@PstSlopeDict begin
%    \end{macrocode}
%
%    \begin{macro}{Opacity++}
%    This macro increments the Opacity index
%    \begin{macrocode}
/Opacity 1 def % preset, no transparency
/Opacity++ { Opacity dOpacity add /Opacity ED } def
%    \end{macrocode}
%    \end{macro}
%    \begin{macro}{max}
%    $x1 \quad x2 \quad \mathtt{max}\quad max$\\
%    |max| is a utility function that calculates the maximum
%    of two numbers.
%    \begin{macrocode}
/max {2 copy lt {exch} if pop} bind def
%    \end{macrocode}
%    \end{macro}
%
%    \begin{macro}{Iterate}
%    $p_1\quad r_1\quad g_1\quad b_1\quad\ldots\quad
%    p_n\quad r_n\quad g_n\quad b_n\quad n\quad \mathtt{Iterate}\quad -$\\
%    This is the actual iteration, which goes through the colour
%    information and plots the segments.  
%    It uses the value of |NumSteps| which is set by the wrapper
%    procedures.  |DrawStep| is called all of |NumSteps| times, so 
%    it had better be fast.
%
%    First, the number of colour infos is read from the
%    top of the stack and decremented, to get the number of segments. 
%    \begin{macrocode}
/Iterate {
  1 sub /NumSegs ED
%    \end{macrocode}
%    Now we get the first colour.  This is really the {\em last}
%    colour given in the |slopecolors| argument.  We have to work
%    {\em down} the stack, so we shall be careful to plot the segments
%    in reverse order.  The |dup mul| stuff squares the RGB
%    components.  This does a kind-of-gamma correction, without
%    which primary colours tend to take up too much space in the
%    slope.  This is nothing deep, it just looks better in my opinion.
%    The following lines convert RGB to HSB and store the resulting
%    components, as well as the |Pt| coordinate in four variables.
%    \begin{macrocode}
  dup mul 3 1 roll dup mul 3 1 roll dup mul 3 1 roll
  setrgbcolor currenthsbcolor 
  /ThisB ED
  /ThisS ED
  /ThisH ED
  /ThisPt ED
%    \end{macrocode}
%    To avoid gaps, we fill the whole  path in that first colour.
%    \begin{macrocode}
  Opacity .setopacityalpha 
  gsave 
  fill 
  grestore
%    \end{macrocode}
%    The body of the following outer loop is executed
%    once for each segment.
%    It expects a current colour and |Pt| coordinate in the |This*|
%    variables and pops the next colour and point from the stack. It
%    then draws the single steps of that segment.
%    \begin{macrocode}
  NumSegs {
    dup mul 3 1 roll dup mul 3 1 roll dup mul 3 1 roll
    setrgbcolor currenthsbcolor
    /NextB ED
    /NextS ED
    /NextH ED
    /NextPt ED
%    \end{macrocode}
%    |NumSteps| always contains the remaining number of steps available.
%    These are evenly distributed between |Pt| coordinates |ThisPt| 
%    to 0, so for the current segment we may use 
%    $|NumSteps|*(|ThisPt|-|NextPt|)/|ThisPt|$ steps.
%    \begin{macrocode}
    ThisPt NextPt sub ThisPt div NumSteps mul cvi /SegSteps exch def
    /NumSteps NumSteps SegSteps sub def
%    \end{macrocode}
%    |SegSteps| may be zero.  In that case there is nothing to do for
%    this segment.
%    \begin{macrocode}
    SegSteps 0 eq not {
%    \end{macrocode}
%    If one of the colours is gray, ie~0 saturation, its hue is
%    useless.  In this case, instead of starting of with a random hue,
%    we take the hue of the other endpoint.  (If both have saturation
%    0, we have a pure gray scale and no harm is done)
%    \begin{macrocode}
      ThisS 0 eq {/ThisH NextH def} if
      NextS 0 eq {/NextH ThisH def} if
%    \end{macrocode}
%    To interpolate between two colours of different hue, we want to
%    go the shorter way around the colour circle.  The following code
%    assures that this happens if we go linearly from |This*| to 
%    |Next*| by conditionally adding 1.0 to one of the hue values.
%    The new hue values can lie between 0.0 and 2.0, so we will later
%    have to subtract 1.0 from values greater than one.
%    \begin{macrocode}
      ThisH NextH sub 0.5 gt
        {/NextH NextH 1.0 add def} 
        { NextH ThisH sub 0.5 ge {/ThisH ThisH 1.0 add def} if }
      ifelse
%    \end{macrocode}
%    We define three variables to hold the current colour coordinates
%    and calculate the corresponding increments per step.
%    \begin{macrocode}
      /B ThisB def
      /S ThisS def
      /H ThisH def
      /BInc NextB ThisB sub SegSteps div def
      /SInc NextS ThisS sub SegSteps div def
      /HInc NextH ThisH sub SegSteps div def
%    \end{macrocode}
%    The body of the following inner loop sets the current colour,
%    according to |H|, |S| and |B| and
%    undoes the kind-of-gamma correction by converting to RGB colour.
%    It then calls |DrawStep|, which draws one step and maybe updates
%    the current point or user space, or variables of its own.  Finally,
%    it increments the three colour variables.
%    \begin{macrocode}
      SegSteps {
        H dup 1. gt {1. sub} if S B sethsbcolor 
        currentrgbcolor 
        sqrt 3 1 roll sqrt 3 1 roll sqrt 3 1 roll
        setrgbcolor
        DrawStep
        /H H HInc add def
        /S S SInc add def
        /B B BInc add def
      } bind repeat
%    \end{macrocode}
%    The outer loop ends by moving on to the |Next| colour and point.
%
%    \begin{macrocode}
      /ThisH NextH def
      /ThisS NextS def
      /ThisB NextB def
      /ThisPt NextPt def
    } if
  } bind repeat
} def
%    \end{macrocode}
%    \end{macro}
%
%    \begin{macro}{PatchRadius}
%    $-\quad\mathtt{PatchRadius}\quad-$\\
%    This macro inspects the value of the variable |Radius|.  If it is
%    0, it is set to the maximum distance of any point in the 
%    current path from the origin of user space.  This has the effect
%    that the current path will be totally filled.  To find the maximum
%    distance, we flatten the path and call |UpdRR| for each endpoint
%    of the generated polygon.  The current maximum square distance is
%    gathered in |RR|.  
%    \begin{macrocode}
/PatchRadius {
  Radius 0 eq { 
    /UpdRR { dup mul exch dup mul add RR max /RR ED } bind def
    gsave
    flattenpath
    /RR 0 def
    {UpdRR} {UpdRR} {} {} pathforall
    grestore
    /Radius RR sqrt def
  } if
} def
%    \end{macrocode}
%    \end{macro}
%
%    \begin{macro}{SlopesFill}
%    $p_1\quad r_1\quad g_1\quad b_1\quad\ldots\quad
%    p_n\quad r_n\quad g_n\quad b_n\quad n\quad s\quad\alpha\quad    
%    \mathtt{SlopesFill}\quad -$\\
%    Fill the current path with a slope described by $p_1,\ldots,b_n,n$.
%    Use a total of $s$ single steps.  Rotate the slope by $\alpha$ 
%    degrees, 0 meaning $r_1,g_1,b_1$ left to $r_n,g_n,b_n$ right.
%
%    After saving the current path, we do the rotation and get the
%    number of steps, which is later needed by |Iterate|.  Remember,
%    that iterate calls |DrawStep| in the reverse order, ie~from
%    right to left.  We work around this by adding 180 degrees to 
%    the rotation.    Filling
%    works by clipping to the path and painting an appropriate sequence
%    of rectangles.  |DrawStep| is set up for |Iterate| to draw a 
%    rectangle of width |XInc| high enough to cover the whole
%    clippath (we use the Level 2 operator |rectfill| for speed) and
%    translate the user system by |XInc|.
%    \begin{macrocode}
/SlopesFill {
  /Fading ED		% do we have fading?
  Fading {
    /FadingEnd ED % the last opacity value
    dup /FadingStart ED % the first opacity value
    /Opacity ED % the opacity start value
  } if
  gsave
  180 add rotate
  /NumSteps ED
  Fading { /dOpacity FadingEnd FadingStart sub NumSteps div def } if
  clip
  pathbbox
  /h ED /w ED
  2 copy translate
  h sub neg /h ED
  w sub neg /w ED
  /XInc w NumSteps div def
  /DrawStep {
    Fading {			% do we have a fading?
      Opacity .setopacityalpha  % set opacity value
      Opacity++			% increase opacity
    } if
    0 0 XInc h rectfill
    XInc 0 translate
  } bind def
  Iterate
  grestore
} def
%    \end{macrocode}
%    \end{macro}
%
%    \begin{macro}{CcSlopesFill} $p_1\quad r_1\quad g_1\quad
%    b_1\quad\ldots\quad p_n\quad r_n\quad g_n\quad b_n\quad n\quad
%    c_x\quad c_y \quad r\quad \mathtt{CcSlopesFill}\quad -$\\ Fills
%    the current path with a concentric pattern,
%    ie~in a polar coordinate system, the colour depends on the
%    radius and not on the angle.
%    Centered around a point with coordinates $(c_x,c_y)$ relative to
%    the bounding box of the path, ie~for a rectangle, $(0,0)$ will
%    center the pattern around the lower left corner of the rectangle,
%    $(0.5,0.5)$ around its center.  The largest circle has a radius of
%    $r$.  If $r=0$, $r$ is taken to be the maximum distance of any
%    point on the current path from the center defined by $(c_x,c_y)$.
%    The colours are given from the center outwards,
%    ie~$(r_1,g_1,b_1)$ describe the colour at the center.
%
%    The code is similar to that of |SlopesFill|.  The main differences
%    are the call to |PatchRadius|, which catches the case that $r=0$
%    and the different definition for |DrawStep|, Which now fills a
%    circle of radius |Rad| and decreases that Variable.  Of course,
%    drawing starts on the outside, so we work down the stack and circles
%    drawn later partially cover those drawn first.  Painting
%    non-overlapping, `donut-shapes' would be slower. 
%    \begin{macrocode}
/CcSlopesFill {
  /Fading ED		% do we have fading?
  Fading {
    /FadingEnd ED % the last opacity value
    dup /FadingStart ED % the first opacity value
    /Opacity ED % the opacity start value
  } if
  gsave
  /Radius ED
  /CenterY ED
  /CenterX ED
  /NumSteps ED
  Fading { /dOpacity FadingEnd FadingStart sub NumSteps div def } if
  clip
  pathbbox
  /h ED /w ED
  2 copy translate
  h sub neg /h ED
  w sub neg /w ED
  w CenterX mul h CenterY mul translate
  PatchRadius
  /RadPerStep Radius NumSteps div neg def
  /Rad Radius def
  /DrawStep {
    Fading {			% do we have a fading?
      Opacity .setopacityalpha  % set opacity value
      Opacity++			% increase opacity
    } if
    0 0 Rad 0 360 arc
    closepath fill
    /Rad Rad RadPerStep add def
  } bind def
  Iterate
  grestore
} def
%    \end{macrocode}
%    \end{macro}
%
%    \begin{macro}{RadSlopesFill}
%    $p_1\quad r_1\quad g_1\quad b_1\quad\ldots
%    \quad p_n\quad r_n\quad g_n\quad b_n\quad n\quad
%    c_x\quad c_y \quad r\quad\alpha\quad \mathtt{CcSlopesFill}\quad -$\\
%    This fills the current path with a radial pattern, ie~in a
%    polar coordinate system the colour depends on the angle and not on
%    the radius.  All this is very similar to |CcSlopesFill|.  There
%    is an extra parameter $\alpha$, which rotates the pattern.
%
%    The only new thing in the code is the |DrawStep| procedure.
%    This does {\em not} draw a circular arc, but a triangle, which is
%    considerably faster.  One of the short sides of the triangle is
%    determined by |Radius|, the other one by |dY|, which is calculated
%    as $|dY|:=|Radius|\times\tan(|AngleIncrement|)$.
%    \begin{macrocode}
/RadSlopesFill {
  /Fading ED		% do we have fading?
  Fading {
    /FadingEnd ED % the last opacity value
    dup /FadingStart ED % the first opacity value
    /Opacity ED % the opacity start value
  } if
  gsave
  rotate
  /Radius ED
  /CenterY ED
  /CenterX ED
  /NumSteps ED
  Fading { /dOpacity FadingEnd FadingStart sub NumSteps div def } if
  clip
  pathbbox
  /h ED /w ED
  2 copy translate
  h sub neg /h ED
  w sub neg /w ED
  w CenterX mul h CenterY mul translate
  PatchRadius
  /AngleIncrement 360 NumSteps div neg def
  /dY AngleIncrement sin AngleIncrement cos div Radius mul def
  /DrawStep {
    Fading {			% do we have a fading?
      Opacity .setopacityalpha  % set opacity value
      Opacity++			% increase opacity
    } if
    0 0 moveto
    Radius 0 rlineto
    0 dY rlineto
    closepath fill
    AngleIncrement rotate
  } bind def
  Iterate
  grestore
} def
%    \end{macrocode}
%    \end{macro}
%
% Last, but not least, we have to close the private dictionary.
%    \begin{macrocode}
end
%</prolog>
%    \end{macrocode}
% \Finale
%
'
%\end{verbatim}
%\section{Package Usage}
% To use |pst-slpe|, you have to say
% \begin{verbatim}
%   \usepackage{pst-slpe}
% \end{verbatim}
% in the document prologue for \LaTeX, and 
% \begin{verbatim}
%   \input pst-slpe.tex
% \end{verbatim}
% in ``plain'' \TeX.
%
% \section{New macro and fill styles}
% \DescribeMacro{\psBall}
%    It takes the (optional) coordinates of the ball center, the color
%    and the radius as parameter and uses |\pscircle| for painting
%    the bullet. 
%
%    \vspace{1cm}
%    \psBall{black}{2ex}
%    \psBall(1,0){blue}{3ex}
%    \psBall(2.5,0){red}{4ex}
%    \psBall(4,0){green!50!blue!60}{5ex}
%
%    \vspace{1cm}
% \begin{verbatim}
%    \psBall{black}{2ex}
%    \psBall(1,0){blue}{3ex}
%    \psBall(2.5,0){red}{4ex}
%    \psBall(4,0){green!50!blue!60}{5ex}
% \end{verbatim}
%
%    The predinied options can be overwritten in the usual way:
%
%    \vspace{1cm}
%    \psBall{black}{2ex}
%    \psBall[sloperadius=10pt](1,0){blue}{3ex}
%    \psBall(2.5,0){red}{4ex}
%    \psBall[slopebegin=red](4,0){green!50!blue!60}{5ex}
%
%    \vspace{1cm}
% \begin{verbatim}
%    \psBall{black}{2ex}
%    \psBall[sloperadius=10pt](1,0){blue}{3ex}
%    \psBall(2.5,0){red}{4ex}
%    \psBall[slopebegin=red](4,0){green!50!blue!60}{5ex}
% \end{verbatim}
%
% \DescribeMacro{slope}
% \DescribeMacro{slopes}
% \DescribeMacro{ccslope}
% \DescribeMacro{ccslopes}
% \DescribeMacro{radslope}
% \DescribeMacro{radslopes}
% |pst-slpe| provides six new fill styles called |slope|, |slopes|,
% |ccslope|, |ccslopes|, |radslope| and |radslopes|.  These obviously
% come in pairs: The $\ldots$|slope|-styles are simplified versions of
% the general $\ldots$|slopes|-styles.\footnote{By the way, I use slope
% as a synonym for gradient.  It sounds less pretentious and avoids
% name clashes.}  The |cc|$\ldots$ styles paint concentric patterns,
% and the |rad|$\ldots$ styles do radial ones.  
%
% Here is a little
% overview of what they look like:
% \newcommand{\st}{$\vcenter to30pt{}$}
% \begin{quote}\LARGE
%  \begin{tabular}{cc}
%   \psframebox[fillstyle=slope]{\st|slope|} &\qquad
%   \psframebox[fillstyle=slopes]{\st|slopes|} \\[2ex]
%   \psframebox[fillstyle=ccslope]{\st|ccslope|} &\qquad
%   \psframebox[fillstyle=ccslopes]{\st|ccslopes|} \\[2ex]
%   \psframebox[fillstyle=radslope]{\st|radslope|} &\qquad
%   \psframebox[fillstyle=radslopes]{\st|radslopes|} \\[2ex]
%  \end{tabular}
% \end{quote}
% These examples were produced by saying simply
% \begin{verbatim}
%   \psframebox[fillstyle=slope]{...}
% \end{verbatim}
% etc.~without setting any further graphics parameters.  The package
% provides a number of parameters that can be used to control
% the way these patterns
% are painted.
% \medskip
%
% \DescribeMacro{slopebegin}
% \DescribeMacro{slopeend}
% The graphics parameters |slopebegin| and |slopeend| set the colours
% between which the three $\ldots$|slope| styles should interpolate.
% Eg,
% \begin{verbatim}
%   \psframebox[fillstyle=slope,slopebegin=red,slopeend=green]{...}
% \end{verbatim}
% produces:
% \begin{quote}\Large
%  \psframebox[fillstyle=slope,slopebegin=red,slopeend=green]{\st slopes!}
% \end{quote}
% The same settings of |slopebegin| and |slopeend| for the |ccslope|
% and |radslope| fillstyles produce
% \begin{quote}\Large
%  \psframebox[fillstyle=ccslope,slopebegin=red,slopeend=green]{\st slopes!}
%   \quad{\normalsize resp.}\quad
%  \psframebox[fillstyle=radslope,slopebegin=red,slopeend=green]{\st slopes!}
% \end{quote}
% The default settings go from a greenish yellow to pure blue.
% \medskip
%
% \DescribeMacro{slopecolors}
% If you want to interpolate between more than two colours, you have
% to use the $\ldots$|slopes| styles, which are controlled by the 
% |slopecolors| parameter instead of |slopebegin| and |slopeend|.  The 
% idea is to specify the colour to use at certain points `on the
% way'.  To fill a shape with |slopes|, imagine a linear scale 
% from its left edge to its right edge.  The left edge must lie at
% coordinate 0.  Pick an arbitrary value for the right edge, say 23.
% Now you want to get light yellow at the left edge, a pastel green at $17/23$ 
% of the way and dark cyan at the right edge, like this:
% \begin{quote}\psset{unit=0.45cm}
%  \begin{pspicture}(-1,0)(24,6)
%   \pscustom[fillstyle=slopes,
% slopecolors=0 1 1 .9  17 .5 1 .5  23 0 0.5 0.5  3]{
%    \psccurve(0,2.5)(12,3.5)(20,4)(23,2)(17,2.5)}
%   \psaxes(0,5)(-0.01,5)(23.01,5)
%   \psline(0,5)(0,1)
%   \psline(17,5)(17,1)
%   \psline(23,5)(23,1)
%  \end{pspicture}
% \end{quote}
% The RGB values for the three colours are $(1,1,0.9)$, $(0.5,1,0.5)$
% and $(0,0.5,0.5)$.  The value for the |slopecolors| parameter is a list
% of `colour infos' followed by the number of `colour infos'. 
% Each `colour info' consists
% of the coordinate value where a colour is to be specified, followed by
% the RGB values of that colour.  All these values are separated by
% white space.  The correct setting for the example is thus:
% \begin{verbatim}
%   slopecolors=0 1 1 .9   17 .5 1 .5   23 0 .5 .5   3
% \end{verbatim}
% For |ccslopes|, specify the colours from the center outward.  
% For |radslopes| (with no rotation specified), 0 represents the ray
% going `eastward'.  Specify the colours anti-clockwise.  If you want a
% smooth gradient at the beginning and starting ray of |radslopes|, you
% should pick the first and last colours identical.
%
% Please note, that the |slopecolors| parameter is not subject to any
% parsing on the \TeX\ side.  If you forget a number or specify the wrong
% number of segments, the PostScript interpreter will probably crash.
%
% The default value for |slopecolors| specifies a rainbow.
%
% \medskip
%
% \DescribeMacro{slopesteps}
% The parameter |slopesteps| controls the number of distinct colour steps
% rendered.  Higher values for this parameter result in better quality
% but proportionally slower rendering.  Eg, setting
% |slopesteps| to 5 with the |slope| fill style results in
% \begin{quote}\Large
%  \psframebox[fillstyle=slope,slopesteps=5]{\st slopes!}
% \end{quote}
%
% The default value is 100, which
% suffices for most purposes.  Remember that the number of distinct colours
% reproducible by a given device is limited.  Pushing |slopesteps| to
% high will result only in loss of performance at no gain in quality.
% \medskip
%
% \DescribeMacro{slopeangle}
% The |slope(s)| and |radslope(s)| patterns may be rotated.  As usual,
% the angles are given anti-clockwise.  Eg, an angle of 30 degrees
% gives
% \begin{quote}\Large\psset{slopeangle=30}
%  \psframebox[fillstyle=slope]{\st slopes!}
%   \quad{\normalsize and}\quad
%  \psframebox[fillstyle=radslope]{\st slopes!}
% \end{quote}
% with the |slope| and |radslope| fillstyles.
% \medskip
%
% \DescribeMacro{slopecenter}
% For the |cc|$\ldots$ and |rad|$\ldots$ styles, it is possible to
% set the center of the pattern.  The |slopecenter| parameter is set to
% the coordinates of that center relative to the bounding box of the
% current path.  The following effect:
% \begin{quote}\psset{unit=0.45cm}
%  \begin{pspicture}(-1,-1)(24,5)
%   \pscustom[fillstyle=radslope,slopecenter=0.2 0.4]{
%    \pspolygon(0,2.5)(12,2.5)(20,4)(23,2)(17,2.5)(3,0)}
%   \psaxes[axesstyle=frame,Dx=0.1,dx=2.2999,Dy=0.2,dy=0.7999](0,0)(23,4) 
%   \psline(4.6,0)(4.6,4)
%   \psline(0,1.6)(23,1.6)
%  \end{pspicture}
% \end{quote}
%  was achieved with 
% \begin{verbatim}
%    fillstyle=radslope,slopecenter=0.2 0.4
% \end{verbatim}
% The default value for |slopecenter| is |0.5 0.5|, which is the
% center for symmetrical shapes.  Note that this parameter is not
% parsed by \TeX, so setting it to anything else than two numbers 
% between 0 and 1 might crash the PostScript interpreter.
% \medskip
%
% \DescribeMacro{sloperadius}
% Normally, the |cc|$\ldots$ and |rad|$\ldots$ styles distribute the
% given colours so that the center is painted in the first colour given,
% and the points of the shape furthest from the center are painted in 
% the last colour.  In other words the maximum radius to which the
% |slopecolors| parameter refers is the maximum distance from the
% center (defined by |slopecenter|) to any point on the periphery
% of the shape.  This radius can be explicitly set with |sloperadius|.
% Eg, setting |sloperadius=0.5cm| gives
% \begin{quote}\Large\psset{sloperadius=0.5cm}
%  \psframebox[fillstyle=ccslope]{\st slopes!}
% \end{quote}
% Any point further from the center than the given |sloperadius| is
% painted with the last colour in |slopeclours|, resp.~|slopeend|.
%
% The default value for |sloperadius| is 0, which invokes the default
% behaviour of automatically calculating the radius.
%
% \DescribeMacro{fading}
% \DescribeMacro{startfading}
% \DescribeMacro{endfading}
% The optional boolean keyword |fading| allows a transparency effect of
% the filled area, starting with the opacity value |startfading| and
% ending with the value of |endfading|. Both values must be of the
% intervall [0\ldots1], with 0 for total opacity and 1 for no
% opacity. The values are preset by 0 and 1.
%
% Here is a little
% overview of what they look like:
% \begin{quote}\LARGE\color{white}
%  \begin{tabular}{cc}
%   \psframe*(-0.3,-0.25)(3,20pt)\psframebox[fading,fillstyle=slope]{\st|  slope  |} &\qquad
%   \psframe*(-0.3,-0.25)(3,20pt)\psframebox[fading,fillstyle=slopes]{\st| slopes |} \\[2ex]
%   \psframe*(-0.3,-0.25)(3,20pt)\psframebox[fading,fillstyle=ccslope]{\st|ccslope |} &\qquad
%   \psframe*(-0.3,-0.25)(3,20pt)\psframebox[fading,fillstyle=ccslopes]{\st|ccslopes|} \\[2ex]
%   \psframe*(-0.3,-0.25)(3,20pt)\psframebox[fading,fillstyle=radslope]{\st|radslope|} &\qquad
%   \psframe*(-0.3,-0.25)(3,20pt)\psframebox[fading,fillstyle=radslopes]{\st|radslopes|} \\[2ex]
%  \end{tabular}
% \end{quote}
% \color{black}
%
% These examples were produced by saying simply
% \begin{verbatim}
%   \psframebox[fading,fillstyle=...]{...}
% \end{verbatim}
%
% \begin{quote}\LARGE\color{white}
% \psset{fading,startfading=0.3,endfading=0.8}
%  \begin{tabular}{cc}
%   \psframe*(-0.3,-0.25)(3,20pt)\psframebox[fillstyle=slope]{\st|  slope  |} &\qquad
%   \psframe*(-0.3,-0.25)(3,20pt)\psframebox[fillstyle=slopes]{\st| slopes |} \\[2ex]
%   \psframe*(-0.3,-0.25)(3,20pt)\psframebox[fillstyle=ccslope]{\st|ccslope |} &\qquad
%   \psframe*(-0.3,-0.25)(3,20pt)\psframebox[fillstyle=ccslopes]{\st|ccslopes|} \\[2ex]
%   \psframe*(-0.3,-0.25)(3,20pt)\psframebox[fillstyle=radslope]{\st|radslope|} &\qquad
%   \psframe*(-0.3,-0.25)(3,20pt)\psframebox[fillstyle=radslopes]{\st|radslopes|} \\[2ex]
%  \end{tabular}
% \end{quote}
% \color{black}
%
% These examples were produced by saying simply
% \begin{verbatim}
%   \psframebox[fading,startfading=0.3,endfading=0.8,fillstyle=...]{...}
% \end{verbatim}
%
% \StopEventually{}
%
%\section{The Code}
% \subsection{Producing the documentation}
%
%    A short driver is provided that can be extracted if necessary by
%    the \textsc{docstrip} program provided with \LaTeXe.
%    \begin{macrocode}
%<*driver>
\NeedsTeXFormat{LaTeX2e}
\documentclass{ltxdoc}
\usepackage{pst-slpe}
\usepackage{pst-plot}
\DisableCrossrefs
\MakeShortVerb{\|}
\newcommand\Lopt[1]{\textsf{#1}}
\newcommand\file[1]{\texttt{#1}}
\AtEndDocument{
\PrintChanges
\PrintIndex
}
%\OnlyDescription
\begin{document}
\DocInput{pst-slpe.dtx}
\end{document}
%</driver>
%    \end{macrocode}
%
% \subsection{The \file{pst-slpe.sty} file}
%    The \file{pst-slpe.sty} file is very simple.  It just loads 
%    the generic \file{pst-slpe.tex} file. 
%    \begin{macrocode}
%<*stylefile>
\RequirePackage{pstricks}
\ProvidesPackage{pst-slpe}[2005/03/05 package wrapper for `pst-slpe.tex']
%%
%% This is file `pst-slpe.tex',
%% generated with the docstrip utility.
%%
%% The original source files were:
%%
%% pst-slpe.dtx  (with options: `texfile')
%% 
%% IMPORTANT NOTICE:
%% 
%% For the copyright see the source file.
%% 
%% Any modified versions of this file must be renamed
%% with new filenames distinct from pst-slpe.tex.
%% 
%% For distribution of the original source see the terms
%% for copying and modification in the file pst-slpe.dtx.
%% 
%% This generated file may be distributed as long as the
%% original source files, as listed above, are part of the
%% same distribution. (The sources need not necessarily be
%% in the same archive or directory.)
%% This program can be redistributed and/or modified under the terms
%% of the LaTeX Project Public License Distributed from CTAN archives
%% in directory macros/latex/base/lppl.txt.
%%
\def\pstslpefileversion{1.31}
\def\pstslpefiledate{2011/10/25}
\message{ v\pstslpefileversion, \pstslpefiledate}
\csname PstSlopeLoaded\endcsname
\let\PstSlopeLoaded\endinput
\ifx\PSTricksLoaded\endinput\else
  \def\next{\input pstricks.tex }\expandafter\next
\fi
\ifx\PSTXKeyLoaded\endinput\else\input pst-xkey \fi % --> hv
\edef\TheAtCode{\the\catcode`\@}
\catcode`\@=11
\pst@addfams{pst-slpe}                              % --> hv
\pstheader{pst-slpe.pro}
\newrgbcolor{slopebegin}{0.9 1 0}
\define@key[psset]{pst-slpe}{slopebegin}{\pst@getcolor{#1}\psslopebegin}% --> hv
\psset[pst-slpe]{slopebegin=slopebegin}                                 % --> hv

\newrgbcolor{slopeend}{0 0 1}
\define@key[psset]{pst-slpe}{slopeend}{\pst@getcolor{#1}\psslopeend}% --> hv
\psset[pst-slpe]{slopeend=slopeend}% --> hv

\define@key[psset]{pst-slpe}{slopesteps}{\pst@getint{#1}\psslopesteps}% --> hv
\psset[pst-slpe]{slopesteps=100}% --> hv

\define@key[psset]{pst-slpe}{slopeangle}{\pst@getangle{#1}\psx@slopeangle}% --> hv
\psset[pst-slpe]{slopeangle=0}% --> hv
\define@key[psset]{pst-slpe}{slopecolors}{\def\psx@slopecolors{#1}}% --> hv
\psset[pst-slpe]{slopecolors={% --> hv
0.0 1 0 0
0.4 0 1 0
0.8 0 0 1
1.0 1 0 1
4}}
\define@key[psset]{pst-slpe}{slopecenter}{\def\psx@slopecenter{#1}}% --> hv
\psset[pst-slpe]{slopecenter={0.5 0.5}}% --> hv
\define@key[psset]{pst-slpe}{sloperadius}{\pst@getlength{#1}\psx@sloperadius}% --> hv
\psset[pst-slpe]{sloperadius=0}% --> hv
\define@boolkey[psset]{pst-slpe}[PST@]{fading}[true]{}% --> hv
\psset[pst-slpe]{fading=false}% --> hv
\define@key[psset]{pst-slpe}{startfading}{\pst@checknum{#1}\psk@startfading }% --> hv
\define@key[psset]{pst-slpe}{endfading}{\pst@checknum{#1}\psk@endfading }% --> hv
\psset[pst-slpe]{startfading=0,endfading=1}% --> hv
\def\psfs@slopes{%
 \addto@pscode{
  \psx@slopecolors\space
  \psslopesteps
  \psx@slopeangle
  \ifPST@fading \psk@startfading \psk@endfading true \else false \fi
  tx@PstSlopeDict begin SlopesFill end}}
\def\psfs@slope{%
 \addto@pscode{%
  gsave
    0 \pst@usecolor\psslopebegin currentrgbcolor
    1 \pst@usecolor\psslopeend currentrgbcolor
    2
  grestore
  \psslopesteps \psx@slopeangle
  \ifPST@fading \psk@startfading \psk@endfading true \else false \fi
  tx@PstSlopeDict begin SlopesFill end}}
\def\psfs@ccslopes{%
 \addto@pscode{%
  \psx@slopecolors\space
  \psslopesteps \psx@slopecenter\space \psx@sloperadius\space
  \ifPST@fading \psk@startfading \psk@endfading true \else false \fi
  tx@PstSlopeDict begin CcSlopesFill end}}
\def\psfs@ccslope{%
 \addto@pscode{%
  gsave 0 \pst@usecolor\psslopebegin currentrgbcolor
    1 \pst@usecolor\psslopeend currentrgbcolor
    2 grestore
  \psslopesteps \psx@slopecenter\space \psx@sloperadius\space
  \ifPST@fading \psk@startfading \psk@endfading true \else false \fi
  tx@PstSlopeDict begin CcSlopesFill end}}
\def\psfs@radslopes{%
 \addto@pscode{%
  \psx@slopecolors\space
  \psslopesteps\psx@slopecenter\space\psx@sloperadius\space\psx@slopeangle
  \ifPST@fading \psk@startfading \psk@endfading true \else false \fi
  tx@PstSlopeDict begin RadSlopesFill end}}
\def\psfs@radslope{%
 \addto@pscode{%
  gsave 0 \pst@usecolor\psslopebegin currentrgbcolor
    1 \pst@usecolor\psslopeend currentrgbcolor
    2 \pst@usecolor\psslopebegin currentrgbcolor
    3 \pst@usecolor\psslopeend currentrgbcolor
    4 \pst@usecolor\psslopebegin currentrgbcolor
    5 grestore
  \psslopesteps\psx@slopecenter\space\psx@sloperadius\space\psx@slopeangle
  \ifPST@fading \psk@startfading \psk@endfading true \else false \fi
  tx@PstSlopeDict begin RadSlopesFill end}}
\def\psBall{\pst@object{psBall}}
\def\psBall@i{\@ifnextchar(\psBall@ii{\psBall@ii(0,0)}}
\def\psBall@ii(#1,#2)#3#4{{%
  \pst@killglue
  \pssetlength\pst@dima{#4}%%%%%  20111025 hv
  \pst@dimb=\pst@dima%%%%%%%%%%%  20111025 hv
  \advance\pst@dima by 0.075\pst@dimb%
  \addbefore@par{sloperadius=\the\pst@dima,fillstyle=ccslope,
   slopebegin=white,slopeend=#3,slopecenter=0.4 0.6,linestyle=none}%
  \use@par%
  \pscircle(#1,#2){#4}%
  }\ignorespaces%
}
\catcode`\@=\TheAtCode\relax
\endinput
%%
%% End of file `pst-slpe.tex'.

\ProvidesFile{pst-slpe.tex}
  [\pstslpefiledate\space v\pstslpefileversion\space 
    `pst-slpe' (mg,hv)]
\IfFileExists{pst-slpe.pro}{%
  \ProvidesFile{pst-slpe.pro}
    [2008/06/19 v. 0.01,  PostScript prologue file (hv)]
    \@addtofilelist{pst-slpe.pro}}{}%
%</stylefile>
%    \end{macrocode}
%
% \subsection{The \file{pst-slpe.tex} file}
%    \file{pst-slpe.tex} contains the \TeX-side of things.  We begin
%    by identifying ourselves and setting things up, the same as in 
%    other PSTricks packages.
%    \begin{macrocode}
%<*texfile>
\message{ v\pstslpefileversion, \pstslpefiledate}
\csname PstSlopeLoaded\endcsname
\let\PstSlopeLoaded\endinput
\ifx\PSTricksLoaded\endinput\else
  \def\next{\input pstricks.tex }\expandafter\next
\fi
\ifx\PSTXKeyLoaded\endinput\else\input pst-xkey \fi % --> hv
\edef\TheAtCode{\the\catcode`\@}
\catcode`\@=11
\pst@addfams{pst-slpe}                              % --> hv
\pstheader{pst-slpe.pro}
%    \end{macrocode}
%    \begin{macro}{slopebegin}
%    \begin{macro}{slopeend}
%    \begin{macro}{slopesteps}
%    \begin{macro}{slopeangle}
%
% \subsubsection{New graphics parameters}
%    We now define the various new parameters needed by the |slope|
%    fill styles and install default values.  First come the colours, 
%    ie~graphics parameters |slopebegin| and |slopeend|, followed
%    by the number of steps, |slopesteps|, and the rotation angle, 
%    |slopeangle|.
%    \begin{macrocode}
\newrgbcolor{slopebegin}{0.9 1 0}
\define@key[psset]{pst-slpe}{slopebegin}{\pst@getcolor{#1}\psslopebegin}% --> hv
\psset[pst-slpe]{slopebegin=slopebegin}                                 % --> hv

\newrgbcolor{slopeend}{0 0 1}
\define@key[psset]{pst-slpe}{slopeend}{\pst@getcolor{#1}\psslopeend}% --> hv
\psset[pst-slpe]{slopeend=slopeend}% --> hv

\define@key[psset]{pst-slpe}{slopesteps}{\pst@getint{#1}\psslopesteps}% --> hv
\psset[pst-slpe]{slopesteps=100}% --> hv

\define@key[psset]{pst-slpe}{slopeangle}{\pst@getangle{#1}\psx@slopeangle}% --> hv
\psset[pst-slpe]{slopeangle=0}% --> hv
%    \end{macrocode}
%    \end{macro}
%    \end{macro}
%    \end{macro}
%    \end{macro}
%    \begin{macro}{slopecolors}
%    The value for |slopecolors| is not parsed.  It is directly copied 
%    to the PostScript output.  This is certainly not the way it
%    should be, but it's simple.  The default value is a rainbow from
%    red to magenta.
%    \begin{macrocode}
\define@key[psset]{pst-slpe}{slopecolors}{\def\psx@slopecolors{#1}}% --> hv
\psset[pst-slpe]{slopecolors={% --> hv
0.0 1 0 0		
0.4 0 1 0
0.8 0 0 1
1.0 1 0 1
4}}
%    \end{macrocode}
%    \end{macro}
%    \begin{macro}{slopecenter}
%    The argument to |slopecenter| isn't parsed either.  But there's
%    probably not much that can go wrong with two decimal numbers.
%    \begin{macrocode}
\define@key[psset]{pst-slpe}{slopecenter}{\def\psx@slopecenter{#1}}% --> hv
\psset[pst-slpe]{slopecenter={0.5 0.5}}% --> hv
%    \end{macrocode}
%    \end{macro}
%    \begin{macro}{sloperadius}
%    The default value for |sloperadius| is 0, which makes the
%    PostScript procedure |PatchRadius| determine a value for the radius.
%    \begin{macrocode}
\define@key[psset]{pst-slpe}{sloperadius}{\pst@getlength{#1}\psx@sloperadius}% --> hv
\psset[pst-slpe]{sloperadius=0}% --> hv
%    \end{macrocode}
%    \end{macro}
%    \begin{macro}{fading}
%    The default value for |fading| is false, which is no transparency effect at all.
%    With |fading=true| the package takes the values |startfading| and |endfading|
%    into account for the opacity effect of the filled area.
%    \end{macro}
%    \begin{macrocode}
\define@boolkey[psset]{pst-slpe}[PST@]{fading}[true]{}% --> hv
\psset[pst-slpe]{fading=false}% --> hv
%    \end{macrocode}
%    \begin{macro}{startfading}
%    The relativ number for the starting value (0\ldots1), preset by 0.
%    \end{macro}
%    \begin{macrocode}
\define@key[psset]{pst-slpe}{startfading}{\pst@checknum{#1}\psk@startfading }% --> hv
%    \end{macrocode}
%    \begin{macro}{endfading}
%    The relativ number for the end value (0\ldots1), preset by 1.
%    \end{macro}
%    \begin{macrocode}
\define@key[psset]{pst-slpe}{endfading}{\pst@checknum{#1}\psk@endfading }% --> hv
\psset[pst-slpe]{startfading=0,endfading=1}% --> hv
%    \end{macrocode}
%
% \subsubsection{Fill style macros}
%
%    Now come the fill style definitions that use these parameters.
%    There is one macro for each fill style named |\psfs@|$style$.  
%    PSTricks calls this macro whenever the current path needs to
%    be filled in that style.  The current path should not be 
%    clobbered by the PostScript code output by the macro.
%
%    \begin{macro}{slopes}
%    For the slopes fill style we produce PostScript code that
%    first puts the |slopecolors| parameter onto the stack.  Note that
%    the number of colours listed, which comes last in |slopecolors| is
%    now on the top of the stack.  Next come the |slopesteps| and
%    |slopeangle| parameters.  We switch to the dictionary established
%    by the \file{pst-slop.pro} Prolog and call |SlopesFill|, which
%    does the artwork and takes care to leave the path alone.
%    \begin{macrocode}
\def\psfs@slopes{%
 \addto@pscode{
  \psx@slopecolors\space
  \psslopesteps
  \psx@slopeangle 
  \ifPST@fading \psk@startfading \psk@endfading true \else false \fi
  tx@PstSlopeDict begin SlopesFill end}}
%    \end{macrocode}
%    \end{macro}
%
%    \begin{macro}{slope}
%    The |slope| style uses parameters |slopebegin| and |slopeend|
%    instead of |slopecolors|.  So the produced PostScript uses these
%    parameters to build a stack in |slopecolors| format.  The 
%    |\pst@usecolor| generates PostScript to set the current colour.
%    We can query the RGB values with |currentrgbcolor|.  
%    A |gsave|/|grestore| pair is used to avoid changing the
%    PostScript graphics state.  Once the stack is set up,
%    |SlopesFill| is called as before.
%    \begin{macrocode}
\def\psfs@slope{%
 \addto@pscode{%
  gsave
    0 \pst@usecolor\psslopebegin currentrgbcolor
    1 \pst@usecolor\psslopeend currentrgbcolor
    2
  grestore
  \psslopesteps \psx@slopeangle 
  \ifPST@fading \psk@startfading \psk@endfading true \else false \fi
  tx@PstSlopeDict begin SlopesFill end}}
%    \end{macrocode}
%    \end{macro}
%
%    \begin{macro}{ccslopes}
%    \begin{macro}{ccslope}
%    \begin{macro}{radslopes}
%    The code for the other fill styles is about the same, except for a few
%    parameters more or less and different PostScript procedures called
%    to do the work.
%    \begin{macrocode}
\def\psfs@ccslopes{%
 \addto@pscode{%
  \psx@slopecolors\space  
  \psslopesteps \psx@slopecenter\space \psx@sloperadius\space
  \ifPST@fading \psk@startfading \psk@endfading true \else false \fi
  tx@PstSlopeDict begin CcSlopesFill end}}
\def\psfs@ccslope{%
 \addto@pscode{%
  gsave 0 \pst@usecolor\psslopebegin currentrgbcolor
    1 \pst@usecolor\psslopeend currentrgbcolor
    2 grestore
  \psslopesteps \psx@slopecenter\space \psx@sloperadius\space
  \ifPST@fading \psk@startfading \psk@endfading true \else false \fi
  tx@PstSlopeDict begin CcSlopesFill end}}
\def\psfs@radslopes{%
 \addto@pscode{%
  \psx@slopecolors\space  
  \psslopesteps\psx@slopecenter\space\psx@sloperadius\space\psx@slopeangle
  \ifPST@fading \psk@startfading \psk@endfading true \else false \fi
  tx@PstSlopeDict begin RadSlopesFill end}}
%    \end{macrocode}
%    \end{macro}
%    \end{macro}
%    \end{macro}
%    \begin{macro}{radslope}
%    |radslope| is slightly different:  Just going from one colour to
%    another in 360 degrees is usually not what is wanted.  |radslope| just
%    does something pretty with the colours provided.
%    \begin{macrocode}
\def\psfs@radslope{%
 \addto@pscode{%
  gsave 0 \pst@usecolor\psslopebegin currentrgbcolor
    1 \pst@usecolor\psslopeend currentrgbcolor
    2 \pst@usecolor\psslopebegin currentrgbcolor
    3 \pst@usecolor\psslopeend currentrgbcolor
    4 \pst@usecolor\psslopebegin currentrgbcolor
    5 grestore
  \psslopesteps\psx@slopecenter\space\psx@sloperadius\space\psx@slopeangle
  \ifPST@fading \psk@startfading \psk@endfading true \else false \fi
  tx@PstSlopeDict begin RadSlopesFill end}}
%    \end{macrocode}
%    \end{macro}
%
%    \begin{macro}{\psBall}
%    \begin{macrocode}
\def\psBall{\pst@object{psBall}}
\def\psBall@i{\@ifnextchar(\psBall@ii{\psBall@ii(0,0)}}
\def\psBall@ii(#1,#2)#3#4{{%
  \pst@killglue
  \pssetlength\pst@dima{#4}%%%%%  20111025 hv
  \pst@dimb=\pst@dima%%%%%%%%%%%  20111025 hv
  \advance\pst@dima by 0.075\pst@dimb%
  \addbefore@par{sloperadius=\the\pst@dima,fillstyle=ccslope,
   slopebegin=white,slopeend=#3,slopecenter=0.4 0.6,linestyle=none}%
  \use@par%
  \pscircle(#1,#2){#4}%
  }\ignorespaces%
}
%    \end{macrocode}
%    \end{macro}
%
%
%    \begin{macrocode}
\catcode`\@=\TheAtCode\relax
%</texfile>
%    \end{macrocode}
%
% \subsection{The \file{pst-slpe.pro} file}
%    The file \file{pst-slpe.pro} contains PostScript definitions
%    to be included in the PostScript output by the 
%    |dvi|-to-PostScript converter, eg |dvips|.
%    First thing is to define a
%    dictionary to keep definitions local.
%    \begin{macrocode}
%<*prolog>
/tx@PstSlopeDict 60 dict def tx@PstSlopeDict begin
%    \end{macrocode}
%
%    \begin{macro}{Opacity++}
%    This macro increments the Opacity index
%    \begin{macrocode}
/Opacity 1 def % preset, no transparency
/Opacity++ { Opacity dOpacity add /Opacity ED } def
%    \end{macrocode}
%    \end{macro}
%    \begin{macro}{max}
%    $x1 \quad x2 \quad \mathtt{max}\quad max$\\
%    |max| is a utility function that calculates the maximum
%    of two numbers.
%    \begin{macrocode}
/max {2 copy lt {exch} if pop} bind def
%    \end{macrocode}
%    \end{macro}
%
%    \begin{macro}{Iterate}
%    $p_1\quad r_1\quad g_1\quad b_1\quad\ldots\quad
%    p_n\quad r_n\quad g_n\quad b_n\quad n\quad \mathtt{Iterate}\quad -$\\
%    This is the actual iteration, which goes through the colour
%    information and plots the segments.  
%    It uses the value of |NumSteps| which is set by the wrapper
%    procedures.  |DrawStep| is called all of |NumSteps| times, so 
%    it had better be fast.
%
%    First, the number of colour infos is read from the
%    top of the stack and decremented, to get the number of segments. 
%    \begin{macrocode}
/Iterate {
  1 sub /NumSegs ED
%    \end{macrocode}
%    Now we get the first colour.  This is really the {\em last}
%    colour given in the |slopecolors| argument.  We have to work
%    {\em down} the stack, so we shall be careful to plot the segments
%    in reverse order.  The |dup mul| stuff squares the RGB
%    components.  This does a kind-of-gamma correction, without
%    which primary colours tend to take up too much space in the
%    slope.  This is nothing deep, it just looks better in my opinion.
%    The following lines convert RGB to HSB and store the resulting
%    components, as well as the |Pt| coordinate in four variables.
%    \begin{macrocode}
  dup mul 3 1 roll dup mul 3 1 roll dup mul 3 1 roll
  setrgbcolor currenthsbcolor 
  /ThisB ED
  /ThisS ED
  /ThisH ED
  /ThisPt ED
%    \end{macrocode}
%    To avoid gaps, we fill the whole  path in that first colour.
%    \begin{macrocode}
  Opacity .setopacityalpha 
  gsave 
  fill 
  grestore
%    \end{macrocode}
%    The body of the following outer loop is executed
%    once for each segment.
%    It expects a current colour and |Pt| coordinate in the |This*|
%    variables and pops the next colour and point from the stack. It
%    then draws the single steps of that segment.
%    \begin{macrocode}
  NumSegs {
    dup mul 3 1 roll dup mul 3 1 roll dup mul 3 1 roll
    setrgbcolor currenthsbcolor
    /NextB ED
    /NextS ED
    /NextH ED
    /NextPt ED
%    \end{macrocode}
%    |NumSteps| always contains the remaining number of steps available.
%    These are evenly distributed between |Pt| coordinates |ThisPt| 
%    to 0, so for the current segment we may use 
%    $|NumSteps|*(|ThisPt|-|NextPt|)/|ThisPt|$ steps.
%    \begin{macrocode}
    ThisPt NextPt sub ThisPt div NumSteps mul cvi /SegSteps exch def
    /NumSteps NumSteps SegSteps sub def
%    \end{macrocode}
%    |SegSteps| may be zero.  In that case there is nothing to do for
%    this segment.
%    \begin{macrocode}
    SegSteps 0 eq not {
%    \end{macrocode}
%    If one of the colours is gray, ie~0 saturation, its hue is
%    useless.  In this case, instead of starting of with a random hue,
%    we take the hue of the other endpoint.  (If both have saturation
%    0, we have a pure gray scale and no harm is done)
%    \begin{macrocode}
      ThisS 0 eq {/ThisH NextH def} if
      NextS 0 eq {/NextH ThisH def} if
%    \end{macrocode}
%    To interpolate between two colours of different hue, we want to
%    go the shorter way around the colour circle.  The following code
%    assures that this happens if we go linearly from |This*| to 
%    |Next*| by conditionally adding 1.0 to one of the hue values.
%    The new hue values can lie between 0.0 and 2.0, so we will later
%    have to subtract 1.0 from values greater than one.
%    \begin{macrocode}
      ThisH NextH sub 0.5 gt
        {/NextH NextH 1.0 add def} 
        { NextH ThisH sub 0.5 ge {/ThisH ThisH 1.0 add def} if }
      ifelse
%    \end{macrocode}
%    We define three variables to hold the current colour coordinates
%    and calculate the corresponding increments per step.
%    \begin{macrocode}
      /B ThisB def
      /S ThisS def
      /H ThisH def
      /BInc NextB ThisB sub SegSteps div def
      /SInc NextS ThisS sub SegSteps div def
      /HInc NextH ThisH sub SegSteps div def
%    \end{macrocode}
%    The body of the following inner loop sets the current colour,
%    according to |H|, |S| and |B| and
%    undoes the kind-of-gamma correction by converting to RGB colour.
%    It then calls |DrawStep|, which draws one step and maybe updates
%    the current point or user space, or variables of its own.  Finally,
%    it increments the three colour variables.
%    \begin{macrocode}
      SegSteps {
        H dup 1. gt {1. sub} if S B sethsbcolor 
        currentrgbcolor 
        sqrt 3 1 roll sqrt 3 1 roll sqrt 3 1 roll
        setrgbcolor
        DrawStep
        /H H HInc add def
        /S S SInc add def
        /B B BInc add def
      } bind repeat
%    \end{macrocode}
%    The outer loop ends by moving on to the |Next| colour and point.
%
%    \begin{macrocode}
      /ThisH NextH def
      /ThisS NextS def
      /ThisB NextB def
      /ThisPt NextPt def
    } if
  } bind repeat
} def
%    \end{macrocode}
%    \end{macro}
%
%    \begin{macro}{PatchRadius}
%    $-\quad\mathtt{PatchRadius}\quad-$\\
%    This macro inspects the value of the variable |Radius|.  If it is
%    0, it is set to the maximum distance of any point in the 
%    current path from the origin of user space.  This has the effect
%    that the current path will be totally filled.  To find the maximum
%    distance, we flatten the path and call |UpdRR| for each endpoint
%    of the generated polygon.  The current maximum square distance is
%    gathered in |RR|.  
%    \begin{macrocode}
/PatchRadius {
  Radius 0 eq { 
    /UpdRR { dup mul exch dup mul add RR max /RR ED } bind def
    gsave
    flattenpath
    /RR 0 def
    {UpdRR} {UpdRR} {} {} pathforall
    grestore
    /Radius RR sqrt def
  } if
} def
%    \end{macrocode}
%    \end{macro}
%
%    \begin{macro}{SlopesFill}
%    $p_1\quad r_1\quad g_1\quad b_1\quad\ldots\quad
%    p_n\quad r_n\quad g_n\quad b_n\quad n\quad s\quad\alpha\quad    
%    \mathtt{SlopesFill}\quad -$\\
%    Fill the current path with a slope described by $p_1,\ldots,b_n,n$.
%    Use a total of $s$ single steps.  Rotate the slope by $\alpha$ 
%    degrees, 0 meaning $r_1,g_1,b_1$ left to $r_n,g_n,b_n$ right.
%
%    After saving the current path, we do the rotation and get the
%    number of steps, which is later needed by |Iterate|.  Remember,
%    that iterate calls |DrawStep| in the reverse order, ie~from
%    right to left.  We work around this by adding 180 degrees to 
%    the rotation.    Filling
%    works by clipping to the path and painting an appropriate sequence
%    of rectangles.  |DrawStep| is set up for |Iterate| to draw a 
%    rectangle of width |XInc| high enough to cover the whole
%    clippath (we use the Level 2 operator |rectfill| for speed) and
%    translate the user system by |XInc|.
%    \begin{macrocode}
/SlopesFill {
  /Fading ED		% do we have fading?
  Fading {
    /FadingEnd ED % the last opacity value
    dup /FadingStart ED % the first opacity value
    /Opacity ED % the opacity start value
  } if
  gsave
  180 add rotate
  /NumSteps ED
  Fading { /dOpacity FadingEnd FadingStart sub NumSteps div def } if
  clip
  pathbbox
  /h ED /w ED
  2 copy translate
  h sub neg /h ED
  w sub neg /w ED
  /XInc w NumSteps div def
  /DrawStep {
    Fading {			% do we have a fading?
      Opacity .setopacityalpha  % set opacity value
      Opacity++			% increase opacity
    } if
    0 0 XInc h rectfill
    XInc 0 translate
  } bind def
  Iterate
  grestore
} def
%    \end{macrocode}
%    \end{macro}
%
%    \begin{macro}{CcSlopesFill} $p_1\quad r_1\quad g_1\quad
%    b_1\quad\ldots\quad p_n\quad r_n\quad g_n\quad b_n\quad n\quad
%    c_x\quad c_y \quad r\quad \mathtt{CcSlopesFill}\quad -$\\ Fills
%    the current path with a concentric pattern,
%    ie~in a polar coordinate system, the colour depends on the
%    radius and not on the angle.
%    Centered around a point with coordinates $(c_x,c_y)$ relative to
%    the bounding box of the path, ie~for a rectangle, $(0,0)$ will
%    center the pattern around the lower left corner of the rectangle,
%    $(0.5,0.5)$ around its center.  The largest circle has a radius of
%    $r$.  If $r=0$, $r$ is taken to be the maximum distance of any
%    point on the current path from the center defined by $(c_x,c_y)$.
%    The colours are given from the center outwards,
%    ie~$(r_1,g_1,b_1)$ describe the colour at the center.
%
%    The code is similar to that of |SlopesFill|.  The main differences
%    are the call to |PatchRadius|, which catches the case that $r=0$
%    and the different definition for |DrawStep|, Which now fills a
%    circle of radius |Rad| and decreases that Variable.  Of course,
%    drawing starts on the outside, so we work down the stack and circles
%    drawn later partially cover those drawn first.  Painting
%    non-overlapping, `donut-shapes' would be slower. 
%    \begin{macrocode}
/CcSlopesFill {
  /Fading ED		% do we have fading?
  Fading {
    /FadingEnd ED % the last opacity value
    dup /FadingStart ED % the first opacity value
    /Opacity ED % the opacity start value
  } if
  gsave
  /Radius ED
  /CenterY ED
  /CenterX ED
  /NumSteps ED
  Fading { /dOpacity FadingEnd FadingStart sub NumSteps div def } if
  clip
  pathbbox
  /h ED /w ED
  2 copy translate
  h sub neg /h ED
  w sub neg /w ED
  w CenterX mul h CenterY mul translate
  PatchRadius
  /RadPerStep Radius NumSteps div neg def
  /Rad Radius def
  /DrawStep {
    Fading {			% do we have a fading?
      Opacity .setopacityalpha  % set opacity value
      Opacity++			% increase opacity
    } if
    0 0 Rad 0 360 arc
    closepath fill
    /Rad Rad RadPerStep add def
  } bind def
  Iterate
  grestore
} def
%    \end{macrocode}
%    \end{macro}
%
%    \begin{macro}{RadSlopesFill}
%    $p_1\quad r_1\quad g_1\quad b_1\quad\ldots
%    \quad p_n\quad r_n\quad g_n\quad b_n\quad n\quad
%    c_x\quad c_y \quad r\quad\alpha\quad \mathtt{CcSlopesFill}\quad -$\\
%    This fills the current path with a radial pattern, ie~in a
%    polar coordinate system the colour depends on the angle and not on
%    the radius.  All this is very similar to |CcSlopesFill|.  There
%    is an extra parameter $\alpha$, which rotates the pattern.
%
%    The only new thing in the code is the |DrawStep| procedure.
%    This does {\em not} draw a circular arc, but a triangle, which is
%    considerably faster.  One of the short sides of the triangle is
%    determined by |Radius|, the other one by |dY|, which is calculated
%    as $|dY|:=|Radius|\times\tan(|AngleIncrement|)$.
%    \begin{macrocode}
/RadSlopesFill {
  /Fading ED		% do we have fading?
  Fading {
    /FadingEnd ED % the last opacity value
    dup /FadingStart ED % the first opacity value
    /Opacity ED % the opacity start value
  } if
  gsave
  rotate
  /Radius ED
  /CenterY ED
  /CenterX ED
  /NumSteps ED
  Fading { /dOpacity FadingEnd FadingStart sub NumSteps div def } if
  clip
  pathbbox
  /h ED /w ED
  2 copy translate
  h sub neg /h ED
  w sub neg /w ED
  w CenterX mul h CenterY mul translate
  PatchRadius
  /AngleIncrement 360 NumSteps div neg def
  /dY AngleIncrement sin AngleIncrement cos div Radius mul def
  /DrawStep {
    Fading {			% do we have a fading?
      Opacity .setopacityalpha  % set opacity value
      Opacity++			% increase opacity
    } if
    0 0 moveto
    Radius 0 rlineto
    0 dY rlineto
    closepath fill
    AngleIncrement rotate
  } bind def
  Iterate
  grestore
} def
%    \end{macrocode}
%    \end{macro}
%
% Last, but not least, we have to close the private dictionary.
%    \begin{macrocode}
end
%</prolog>
%    \end{macrocode}
% \Finale
%
