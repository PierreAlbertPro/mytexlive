%% Filename: pst-pad-doc.tex
%% 
%% Author: Patrick Drechsler
%% 
%% Created: Fri Feb  8 02:04:17 2008
%% 
%% Version: $Id: pst-pad-doc.tex 328 2008-03-21 15:22:10Z patrick $
%% 
\listfiles
\documentclass{scrartcl}
\usepackage[english]{babel}
\input{pst-pad-doc-header}
\newcommand{\summary}{A PSTricks package for drawing attachment models}

\begin{document}
\EnableQuotes

\title{\texttt{pst-pad}\\\summary\\\normalsize ver. \verPstPad} \author{Patrick
  Drechsler\thanks{\url{<Patrick.Drechsler@pstricks.de>}}} \date{\today}
\maketitle

\begin{abstract}
\PadPackage{} is a PSTricks package to draw simple attachment systems.
\end{abstract}

\setlength{\columnseprule}{0.6pt}
\begin{multicols}{2}
{\parskip 0pt \tableofcontents}
\end{multicols}

\section{Introduction}
\label{sec:introduction}

The package \PadPackage{} is a collection of graphical elements based on
PStricks that can be used to facilitate display of attachment systems such as
two differently shaped surfaces with or without a fluid wedged in
between. These macros ease the display of wet adhesion models and common
friction systems such as boundary lubrication, elastohydrodynamic lubrication
and hydrodynamic lubrication. The name \PadPackage{} was chosen because I used
this package to display models of different animal attachment pads in my
research.

\section{Parameters}
\label{sec:parameters}

There are specific parameters defined to easily change the behaviour of the pst-pad
objects you are drawing.

\let\HLTTc\textsf
%\begin{multicols}{2}
\begin{description}
\item[\HLTTc{fluid} (boolean):] (\emph{default:~true})
\item[\HLTTc{WallThickness} (dimension):] (\emph{default:~1})
\item[\HLTTc{WallString} (string):] (\emph{default:~})
\item[\HLTTc{WallAngle} (PSTricks color):] (\emph{default:~0})
\item[\HLTTc{WallLineWidth} (dimension):] (\emph{default:~1pt})
\item[\HLTTc{WallLineColor} (PSTricks color):] (\emph{default:~black})
\item[\HLTTc{SphereStartAngle} (dimension):] (\emph{default:~180})
\item[\HLTTc{SphereEndAngle} (boolean):] (\emph{default:~360})
\item[\HLTTc{SphereRadius} (dimension):] (\emph{default:~1})
\item[\HLTTc{ShpereString} (string):] (\emph{default:~})
\item[\HLTTc{SphereOffset} (dimension):] (\emph{default:~0})
\item[\HLTTc{SphereFillColor} (PSTricks color):] (\emph{default:~white})
\item[\HLTTc{SphereLineWidth} (dimension):] (\emph{default:~1pt})
\item[\HLTTc{SphereLineColor} (PSTricks color):] (\emph{default:~black})
\item[\HLTTc{FSphereHeight} (dimension):] (\emph{default:~2})
\item[\HLTTc{FSphereFillColor} (PSTricks color):] (\emph{default:~lightgray})
\item[\HLTTc{FSphereString} (string):] (\emph{default:~})
\item[\HLTTc{FSphereLineWidth} (dimension):] (\emph{default:~1pt})
\item[\HLTTc{FSphereLineColor} (PSTricks color):] (\emph{default:~black})
\item[\HLTTc{FluidMaxRadius} (dimension):] (\emph{default:~1})
\item[\HLTTc{FluidMinRadius} (dimension):] (\emph{default:~.5})
\item[\HLTTc{FluidHeight} (dimension):] (\emph{default:~.5})
\item[\HLTTc{FluidFillColor} (PSTricks color):] (\emph{default:~yellow})
\item[\HLTTc{FluidString} (string):] (\emph{default:~})
\item[\HLTTc{FluidShearOffsetX} (dimension):] (\emph{default:~0})
\item[\HLTTc{FluidShearOffsetY} (dimension):] (\emph{default:~0})
\item[\HLTTc{FluidLineWidth} (dimension):] (\emph{default:~1pt})
\item[\HLTTc{FluidLineColor} (PSTricks color):] (\emph{default:~black})
\item[\HLTTc{StringA} (string):] (\emph{default:~})
\item[\HLTTc{StringB} (string):] (\emph{default:~})
\item[\HLTTc{TotalAngle} (angle):] (\emph{default:~0})
\end{description}

\clearpage

\section{High-level Macros ("Objects")}
\label{sec:high-level-macros}

There are three high-level macros: \macro{PstWallToWall},
\macro{PstSphereToWall} and \macro{PstPad}.

\subsection{\macro{PstWallToWall}}
\label{sec:macropstwalltowall}

\begin{LTXexample}[width=5cm]
\begin{pspicture}(4,4)\psgrid
  \PstWallToWall(2,2)
\end{pspicture}
\end{LTXexample}

\subsection{\macro{PstSphereToWall}}
\label{sec:macropstspheretowall}

\begin{LTXexample}[width=5cm]
\begin{pspicture}(4,4)\psgrid
  \PstSphereToWall[fluid=false](2,2)
\end{pspicture}
\end{LTXexample}


\subsection{\macro{PstPad}}
\label{sec:macropstpad}

\begin{LTXexample}[width=5cm]
\begin{pspicture}(4,4)\psgrid
  \PstPad(2,2)
\end{pspicture}
\end{LTXexample}

\clearpage

\subsection{Basic Usage}
\label{sec:basic-usage}

The following figure shows the usage of some basic parameters with the
\macro{PstPad} macro. Green and brown colored values in parentheses give
default and example values, respectively.

\begin{pspicture}(-5,-4)(7.5,4)%\psgrid
  \PstPad[%
  FluidMaxRadius=4,%
  FluidMinRadius=3,%
  FluidHeight=1,%
  FSphereHeight=2,%
  FSphereFillColor=green!20!white,%
  FluidFillColor=blue!10!white,%
  FluidLineWidth=1.5pt,%
  FSphereLineWidth=0.1pt,%
  WallLineWidth=3pt,%
  WallLineColor=blue,%
  FSphereLineColor=magenta,%
  FluidLineColor=green!50!black%
  ](0,0)
  %%
  \sffamily\footnotesize%
  \psset{linecolor=gray, linewidth=0.2pt}
  %% "Crosshair" for orientation:
  \psline[linestyle=dashed](-5,0)(5,0)
  \psline[linestyle=dashed](0,-3)(0,1.5)
  %% Node definitions:
  \pnode(0,0){center}
  \pnode(-4,-2.25){maxradiusLeft}
  \pnode(0,-2.25){maxradiusRight}
  \pnode(-3,0){minradiusLeft}
  \pnode(3,0){minradiusRight}
  \pnode([nodesep=4.5]center){fluidheightCenter}
  \pnode([nodesep=4.5,offset=-1]center){fluidheightBottom}
  \pnode([offset=-1]fluidheightBottom){wallThicknessBottom}
  \pnode([nodesep=5.2,offset=1]center){fsphereheightBottom}
  \pnode([offset=1]fsphereheightBottom){fsphereheightTop}
  %%
  \psset{arrows=|<->|,linewidth=0.8pt,linecolor=red,arrowscale=1.5}
  %% Center:
  \rput[l]([nodesep=.2,offset=.2]center){\textbf{(0,0)}}
  %% Label FluidMinRadius:
  \ncline{minradiusLeft}{center}
  \nbput{\ParCol{FluidMinRadius}{0.5}{3}}
  %% Label FluidMaxRadius:
  \ncline{maxradiusLeft}{maxradiusRight}
  \nbput{\ParCol{FluidMaxRadius}{1}{4}}
  %% Label FluidHeight:
  \ncline{fluidheightCenter}{fluidheightBottom}
  \naput{\ParCol{FluidHeight}{.5}{1}}
  %% Label WallThickness:
  \ncline{fluidheightBottom}{wallThicknessBottom}
  \naput{\ParCol{WallThickness}{1}{1}}
  %% Label FSpereHeight:
  \ncline{fsphereheightBottom}{fsphereheightTop}
  \nbput{\ParCol{FSphereHeight/2}{2}{2}}
  %% Label FSphereFillColor:
  \psline{->}(-.1,2.2)(-.3,1.6)
  \rput(-.1,2.6){\ParCol{FSphereFillColor}{"lightgray"}{"green!20!white"}}
  %% Label FluidFillColor:
  \psline{->}(4,2.2)(2.5,.6)
  \rput(4,2.6){\ParCol{FluidFillColor}{"yellow"}{"blue!10!white"}}
  %% Label FluidLineWidth:
  \psline{->}([offset=.5,nodesep=3]minradiusRight)(minradiusRight)
  \rput[l]([offset=-.1,nodesep=3.2]minradiusRight){\ParCol{FluidLineWidth}{1pt}{1.5pt}}
  \rput[l]([offset=.8,nodesep=3.2]minradiusRight){\ParCol{FluidLineColor}{"black"}{"green!50!black"}}
  %% Label FSphereLineWidth:
  \psline{->}(-4.3,2.2)(-3.5,1)
  \rput(-4.3,2.6){\ParCol{FSphereLineWidth}{1pt}{0.1pt}}
  \rput(-4.3,3.4){\ParCol{FSphereLineColor}{"black"}{"magenta"}}
  %% Label WallLineWidth:
  \psline{->}(1.5,-2.25)(1.5,-1)
  \rput[t](1.5,-2.5){\ParCol{WallLineWidth}{1pt}{3pt}}
  \rput[t](1.5,-3.4){\ParCol{WallLineColor}{"black"}{"blue"}}
\end{pspicture}

\subsection{Liquid shearing}
\label{sec:liquid-shearing}

Sheared liquids sandwiched in between two surfaces can also be displayed.  The
following figures shows the usage of the
parameters \parameter{FluidShearOffsetX}
and \parameter{FluidShearOffsetY}. Again, green and brown colored values in
parentheses give default and example values, respectively.

\subsubsection[\parameter{FluidShearOffsetX}]{Parameter \parameter{FluidShearOffsetX}}
\label{sec:param-offsetx}

\parameter{FluidShearOffsetX}\ describes the x offset from the current value
of \parameter{FluidMaxRadius}.

\begin{center}
  \begin{pspicture}[showgrid=false](-5,-2)(4.5,2)%
    \PstPad[%
    FluidHeight=1,%
    FluidShearOffsetX=0.6%
    ](0,0)
    %% Labeling:
    \sffamily\footnotesize%
    \psset{linecolor=black, linewidth=0.2pt}
    \psline[linestyle=dotted](-2.5,0)(1.5,0)
    \psline[linestyle=dotted](0,-1)(0,1.5)
    
    %% Node definitions:
    \pnode(0,0){center}
    \pnode(-1,-2.25){maxradiusLeft}
    \pnode(0,-2.25){maxradiusRight}
    
    \pnode(-1,0){maxradiusLeftZero}
    \pnode(0,0){maxradiusRightZero}
    
    \pnode([nodesep=-0.6]maxradiusLeft){Xoffset}
    \pnode([nodesep=-0.6]maxradiusLeftZero){XoffsetZero}
    
    \psline[linestyle=dotted]([offset=-1]maxradiusLeftZero)([offset=.5]maxradiusLeftZero)
    \psline[linestyle=dotted]([offset=-1]XoffsetZero)([offset=.5]XoffsetZero)
    %% 
    %% 
    \psset{arrowscale=2,arrows=|<->|}
    %% Label FluidMaxRadiusLeft:
    \ncline{maxradiusLeft}{maxradiusRight}
    \nbput[npos=0.5]{\ParCol{FluidMaxRadius}{1}{1}}
    %% Label FluidShearOffsetX:
    \ncline[linecolor=red]{maxradiusLeftZero}{XoffsetZero}
    \nbput[npos=1.6]{\ParCol{FluidShearOffsetX}{0}{.6}}
  \end{pspicture}
\end{center}

\subsubsection[\parameter{FluidShearOffsetY}]{Parameter \parameter{FluidShearOffsetY}}
\label{sec:param-offsety}

\parameter{FluidShearOffsetY}\ describes the y offset from zero to determine the
curvature of the meniscus. 

\begin{center}
  \begin{pspicture}[showgrid=false](-4.5,-2)(4.5,2)%
    \PstPad[%
    FluidHeight=1,%
    FluidShearOffsetY=0.4%
    ](0,0)
    %% Labeling:
    \sffamily\footnotesize%
    \psset{linecolor=black, linewidth=0.2pt}
    \psline[linestyle=dotted](-2.5,0)(1.5,0)
    \psline[linestyle=dotted](0,-1)(0,1.5)
    %% Node definitions:
    \pnode(0,0){center}
    \pnode([nodesep=1.5]center){fluidheightCenterRight}
    \pnode([nodesep=-1.5]center){fluidheightCenterLeft}
    \pnode([nodesep=-1.5,offset=1]center){fluidheightTop}
    \pnode([offset=-0.4]fluidheightCenterRight){Yoffset}
    %% 
    \psline[linestyle=dotted](0,-.4)(1.7,-.4)
    %% 
    \psset{arrowscale=2,arrows=|<->|}
    %% Label FluidHeight:
    \ncline{fluidheightCenterLeft}{fluidheightTop}
    \naput{\ParCol{FluidHeight}{.5}{1}}
    %% Label FluidShearOffsetY:
    \ncline[linecolor=red]{fluidheightCenterRight}{Yoffset}
    \naput{\ParCol{FluidShearOffsetY}{0}{.4}}
  \end{pspicture}
\end{center}


\subsection{Tilting parameters}
\label{sec:tilting-parameters}

The following figure shows the usage of the \parameter{TotalAngle}
parameter. The default angle is 0 degrees. Again, green and brown colored
values in parentheses give default and example values, respectively.

\begin{center}
\begin{pspicture}(-2.5,-2)(4,2.5)%
  %%
  \PstPad[TotalAngle=30](0,0)
  %% Labeling:
  \sffamily\footnotesize%
  \psset{linecolor=black, linewidth=0.2pt}
  \psline[linestyle=dotted](-2.5,0)(2.5,0)
  \psline[linestyle=dotted](0,-1)(0,1.5)
  \rput{30}{%
    \psline[linestyle=dashed](-2.5,0)(2.5,0)
    \psline[linestyle=dashed](0,-1)(0,1.5)
  }
  \pnode(0,0){center}
  \psset{linecolor=red}
  \psarc[arrowscale=2]{<->}(center){2.25}{0}{30}
  \rput[l]([nodesep=2.5,angle=15]center){%
    \ParCol{TotalAngle}{0}{30}}%
\end{pspicture}
\end{center}

\section{Example Usage}
\label{sec:example-usage}

\begin{LTXexample}[width=5cm]
\begin{pspicture}(4,4)\psgrid
  \PstWallToWall[%
  FluidFillColor=red,%
  TotalAngle=10%
  ](2,2)
\end{pspicture}
\end{LTXexample}


\begin{LTXexample}[width=5cm]
\begin{pspicture}(4,4)\psgrid
  \tiny\sffamily%
  \PstPad[%
  FluidFillColor=blue!30!white,%
  FluidLineColor=blue!70!black,%
  FSphereFillColor=yellow!40!white,%
  FSphereLineColor=red!70!black,%
  FluidString={fluid A},%
  FluidShearOffsetX=0.1,%
  FluidShearOffsetY=0.2,%
  WallThickness=0.4,%
  FSphereHeight=1,%
  StringA={sliding insect pad}%
  ](2,2)
\end{pspicture}
\end{LTXexample}


\begin{LTXexample}[width=5cm]
\begin{pspicture}(4,4)\psgrid
  \PstSphereToWall[%
  fluid=false,%
  SphereRadius=1.5,%
  SphereStartAngle=190,%
  SphereEndAngle=350,%
  SphereOffset=0.1,%
  StringA={string a},%
  StringB={string b}%
  ](2,2)
\end{pspicture}
\end{LTXexample}

\section{Low-level Macros}
\label{sec:low-level-macros}

There are four low-level macros: \macro{PstWall}, \macro{PstFluid},
\macro{PstSphere} and \macro{PstFlattenedSphere}.

\subsection{\macro{PstWall}}
\label{sec:macropstwall}

\begin{LTXexample}[width=5cm]
  \begin{pspicture}[showgrid=true](4,2)
    \PstWall(2,1.5)
  \end{pspicture}
\end{LTXexample}

\subsection{\macro{PstFluid}}
\label{sec:macropstfluid}

\begin{LTXexample}[width=5cm]
  \begin{pspicture}[showgrid=true](4,2)
    \rput(2,1){\PstFluid[unit=1pt](0,0)}
  \end{pspicture}
\end{LTXexample}


\clearpage

\subsection{\macro{PstSphere}}
\label{sec:macropstsphere}

\begin{LTXexample}[width=5cm]
  \begin{pspicture}[showgrid=true](4,2)
    \PstSphere(2,.5)
  \end{pspicture}
\end{LTXexample}

\subsection{\macro{PstFlattenedSphere}}
\label{sec:macr}

\begin{LTXexample}[width=5cm]
  \begin{pspicture}[showgrid=true](4,2)
    \PstFlattenedSphere(2,.5)
  \end{pspicture}
\end{LTXexample}


\section{TODO}
\label{sec:todo}

\begin{itemize}
\item Enhance calculations and number conversions.
\item Implement different degrees of surface roughness.
\item Implement fluid film contact angle parameter.
\end{itemize}


\section{Acknowledgements}
\label{sec:acknowledgements}

I would like to thank Herbert Voß and Alan Ristow for their help on getting me
started with PSTricks macros and low-level calculations in PostScript and \TeX.

\end{document}

%%% Local Variables: 
%%% TeX-command-default: "Make"
%%% mode: latex
%%% TeX-master: t
%%% End: 
