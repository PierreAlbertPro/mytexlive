\documentclass[12pt]{article}
\usepackage[T1]{fontenc}
\usepackage[latin1]{inputenc}
%\usepackage{geometry}
\usepackage{url,calc}
\usepackage{pstricks}
\usepackage{pst-plot}
\SpecialCoor
%\usepackage[ps2pdf,colorlinks,linktocpage]{hyperref}
\usepackage[colorlinks,linktocpage]{hyperref}
\def\UrlFont{\small\ttfamily}
\makeatletter
\def\verbatim@font{\small\normalfont\ttfamily}
\makeatother
%\usepackage{color}
\definecolor{hellgelb}{rgb}{1,1,0.8}
\definecolor{colKeys}{rgb}{0,0,1}
\definecolor{colIdentifier}{rgb}{0,0,0}
\definecolor{colComments}{rgb}{1,0,0}
\definecolor{colString}{rgb}{0,0.5,0}
%
\usepackage{listings}
\lstset{%
    language=[LaTeX]TEX,%
    float=hbp,%
    basicstyle=\ttfamily\small, %
    identifierstyle=\color{colIdentifier}, %
    keywordstyle=\color{colKeys}, %
    stringstyle=\color{colString}, %
    commentstyle=\color{colComments}, %
    columns=flexible, %
    tabsize=4, %
    frame=single, %
    extendedchars=true, %
    showspaces=false, %
    showstringspaces=false, %
    numbers=left, %
    numberstyle=\tiny, %
    breaklines=true, %
%    backgroundcolor=\color{hellgelb}, %
    breakautoindent=true, %
    captionpos=b,%
	xleftmargin=0pt%
}

%\parindent=0pt
\newcommand\verbI[1]{{\small\texttt{#1}}}
\newcommand\CMD[1]{{\texttt{\textbackslash#1}}}
%
\psset{subgriddiv=0,griddots=5,gridlabels=7pt}
%
\DeclareRobustCommand\cs[1]{\texttt{\char`\\#1}}
\def\PS{PostScript}
%
\begin{document}
\title{\texttt{pstricks -- patch 15}\\new macros and bugfixes for \texttt{pstricks}}
\author{Herbert Vo�\thanks{%
\url{Herbert.Voss@perce.de}} \and Rolf Niepraschk\thanks{\url{Rolf.Niepraschk@ptb.de}}}
\date{2004/05/12}

\maketitle

\begin{abstract}
It is long time ago since \verb+pstricks.tex patch 14+ came out. The new version
\verb+patch 15+ fixes some bugs and provides three new elliptic macros, which were
already present in the old beta version of \verb+PSTricks+.

There is also a new \verb+pstricks.sty+, which makes the \verb+\pstcol+ package
obsolete. It uses the new color package \verb+xcolor+, which provides a much more
powerful color management than \verb+color.sty+ does. The \verb+pstricks.sty+ is
a real \LaTeX{} package, it makes no sense for \TeX{} users. Nevertheless, using
of \verb+pstcol+ or package \verb+color+ is still possible.

Timothy Van Zandt was the one, who creates \verb+PSTricks+, but Denis Girou was
the one who makes it run over many years. Needless to say, how important his work is
for \verb+PSTricks+. Since more than nine month we are unable to get in touch with
Denis, which is the reason why this update of \verb+PSTricks+ comes without any
comments from Denis.
\end{abstract}

\clearpage
\tableofcontents

\clearpage

\section{New fill style \texttt{eofill}}
\PS\ has a special fillstyle, called \verb|eofill|, which is now available 
with the option
\verb|fillstyle=eofill|. The following two images show the difference, 
the first one is filled with
\verb|fillstyle=solid| and the second one with the new option \verb|fillstyle=eofill|.

\begin{minipage}{5.5cm}
\begin{pspicture}(5,2.5)
\pspolygon[unit=0.5cm,%
     fillstyle=solid,%
     fillcolor=yellow](7,3)(0,0)(1,3)
     (5,5)(10,3)(2,2)(7,5)(7,3)
\end{pspicture}
\end{minipage}\hfill%
\begin{minipage}{\linewidth-6cm}
\begin{lstlisting}
\begin{pspicture}(5,2.5)
\pspolygon[unit=0.5cm,%
     fillstyle=solid,%
     fillcolor=yellow](7,3)(0,0)(1,3)
     (5,5)(10,3)(2,2)(7,5)(7,3)
\end{pspicture}
\end{lstlisting}
\end{minipage}


\begin{minipage}{5.5cm}
\begin{pspicture}(5,2.5)
\pspolygon[unit=0.5cm,%
    fillstyle=eofill,%
    fillcolor=yellow](7,3)(0,0)(1,3)%
    (5,5)(10,3)(2,2)(7,5)(7,3)
\end{pspicture}
\end{minipage}\hfill%
\begin{minipage}{\linewidth-6cm}
\begin{lstlisting}
\begin{pspicture}(5,2.5)
\pspolygon[unit=0.5cm,%
    fillstyle=eofill,%
    fillcolor=yellow](7,3)(0,0)(1,3)%
    (5,5)(10,3)(2,2)(7,5)(7,3)
\end{pspicture}
\end{lstlisting}
\end{minipage}

\section{Dashed lines}
By default a dash line can be set with the option \verb+dash=<black> <white>+, 
e.g. \verb+dash=10pt 5pt+. This definition makes it impossible to define a 
dashed/dotted line. \verb+pstricks-add+ redefines this option for a use with 
four parameters \verb+dash=<black> <white> <black> <white>+,
where the last two can be omit. The following examples show different values for these
parameters:


\medskip
\begin{center}
\begin{minipage}{10.5cm}%
\psset{linestyle=dashed,dashadjust=false}
\noindent
\psline[dash=1 1](0,0)(10,0)\\
\psline[linewidth=1mm,dash=2 0.5](0,0)(10,0)\\
\psline[dash=1 0.2 0.05 0.2](0,0)(10,0)\\
\psline[dash=0.05 0.2 1 0.2](0,0)(10,0)\\
\psline[linewidth=1mm,dash=2 1 1 2](0,0)(10,0)\\

\psset{dashadjust=true}
\noindent
\psline[dash=1 1](0,0)(10,0)\\
\psline[linewidth=1mm,dash=2 0](0,0)(10,0)\\
\psline[dash=1 0.2 0.05 0.2](0,0)(10,0)\\
\psline[dash=0.05 0.2 1 0.2](0,0)(10,0)\\
\psline[linewidth=1mm,dash=2 1 1 2](0,0)(10,0)
\end{minipage}
\end{center}


\medskip
\begin{lstlisting}
{\psset{linestyle=dashed,dashadjust=false}
\psline[dash=1 1](0,0)(10,0)\\
\psline[linewidth=1mm,dash=2 0.5](0,0)(10,0)\\
\psline[dash=1 0.2 0.05 0.2](0,0)(10,0)\\
\psline[dash=0.05 0.2 1 0.2](0,0)(10,0)\\
\psline[linewidth=1mm,dash=2 1 1 2](0,0)(10,0)\\

\psset{dashadjust=true}
\psline[dash=1 1](0,0)(10,0)\\
\psline[linewidth=1mm,dash=2 0](0,0)(10,0)\\
\psline[dash=1 0.2 0.05 0.2](0,0)(10,0)\\
\psline[dash=0.05 0.2 1 0.2](0,0)(10,0)\\
\psline[linewidth=1mm,dash=2 1 1 2](0,0)(10,0)}
\end{lstlisting}


As seen in the above code, it is no problem to use dashed lines in the usual way 
with two parameters.


\section{Ellipses}
\verbI{pstricks - patch 14} has only the following macro for drawing an ellipse:

{\small\begin{verbatim}
\psellipse[<option>](x,y)(a,b)
\psellipse*[<option>](x,y)(a,b)
\end{verbatim}}

\noindent whith \verbI{(x,y)} as the center and \verbI{(a,b)} as the two radians 
(figure \ref{pstricks}).

\begin{figure}[htb]
\begin{minipage}{0.39\columnwidth}
\begin{pspicture}(0,-0.25)(5,3.25)
  \psgrid
  \psellipse[%
    linewidth=0.2,%
    linecolor=red](2.5,1.5)(2,1)
  \psellipse*[%
  	linecolor=blue](2.5,1.5)(0.5,1.5)
\end{pspicture}%
\end{minipage}%
\hfill%
\begin{minipage}{0.59\columnwidth}
\begin{lstlisting}
\begin{pspicture}(0,-0.25)(5,3.25)
  \psgrid
  \psellipse[%
    linewidth=0.2,%
    linecolor=red](2.5,1.5)(2,1)
  \psellipse*[%
    linecolor=blue](2.5,1.5)(0.5,1.5)
\end{pspicture}%
\end{lstlisting}
\end{minipage}%
\caption{The \verbI{pstricks} macro \CMD{psellipse}}\label{pstricks}
\end{figure}

\subsection{Ellipse based on \texttt{pst-plot}}

With the \CMD{parametricplot} macro from \verbI{pst-plot} we can define a new macro for 
drawing ellipses:

\begin{lstlisting}
% #1 options
% #2 a
% #3 b
% #4 start angle
% #5 end angle
\newcommand{\pstEllipse}[5][]{%
  \psset{#1}
  \parametricplot{#4}{#5}{#2\space t cos mul #3\space t sin mul}}
\end{lstlisting}

\noindent which has the syntax

{\small\begin{verbatim}
\pstEllipse[<options>]{a}{b}{start angle}{end angle}
\end{verbatim}}

This macro is not part of of \verb|pstricks.tex|, it is only defined for some demonstration.

\newcommand*{\pstEllipse}[5][]{%
  \psset{#1}
  \parametricplot{#4}{#5}{#2\space t cos mul #3\space t sin mul}%
}

\begin{figure}[htb]
\begin{minipage}{0.39\columnwidth}
\begin{pspicture}(-2.25,-1.75)(2.25,1.75)
	\psgrid
	\pstEllipse[%
		linewidth=0.2,%
		linecolor=red]{2}{1}{0}{360}
	\pstEllipse[%
		linewidth=0.1,%
		arrows=|->,%
		arrowsize=0.5,%
		linecolor=blue]{1}{2}{-30}{250}
\end{pspicture}%
\end{minipage}%
\hfill%
\begin{minipage}{0.59\columnwidth}
\begin{lstlisting}
\begin{pspicture}(-2.25,-1.75)(2.25,1.75)
  \psgrid
  \pstEllipse[%
    linewidth=0.2,%
    linecolor=red]{2}{1}{0}{360}
  \pstEllipse[%
    linewidth=0.1,arrows=|->,%
    arrowsize=0.5,%
    linecolor=blue]{1}{2}{-30}{250}
\end{pspicture}%
\end{lstlisting}
\end{minipage}%
\caption{The macro \CMD{pstEllipse} which uses the \CMD{parametricplot} macro 
from \verbI{pst-plot}}\label{parametric}
\end{figure}

As seen in figure \ref{parametric} it is no problem to draw arcs of an ellipse. 
The center of these ellipses are by default $(0,0)$, with the \CMD{rput} macro 
it is also not a problem to put the ellipse anywhere in the coordinate system with
any angle of rotating.


\subsubsection{Wedge of an ellipse}
To define a macro for a wedge of an ellipse (figure \ref{parametric-wedge}) is also easy with
the \verb+\pscustom+ macro. which uses the following code:


%
\newcommand*{\pstEllipseWedge}[5][]{%
  \psset{#1}%
  \pscustom{%
    \parametricplot{#4}{#5}{#2\space t cos mul #3\space t sin mul}%
      \psline(! #2\space #5\space cos mul #3\space #5\space sin mul)%
        (0,0)%
	(! #2\space #4\space cos mul #3\space #4\space sin mul)%
  }%
}
%

\begin{lstlisting}
% #1 options
% #2 a
% #3 b
% #4 start angle
% #5 end angle
\newcommand{\pstEllipseWedge}[5][]{%
  \psset{#1}
  \pscustom{%
    \parametricplot{#4}{#5}{#2\space t cos mul #3\space t sin mul}%
    \psline(! #2\space #5\space cos mul #3\space #5\space sin mul)%
	(0,0)%
	(! #2\space #4\space cos mul #3\space #4\space sin mul)%
  }%
}
\end{lstlisting}

This macro is also not part of of \verb|pstricks.tex|, it is only defined for some 
demonstration.

\begin{figure}[htb]
\begin{minipage}{0.39\columnwidth}
\begin{pspicture}(-2.25,-1.75)(2.25,1.75)
  \psgrid
  \pstEllipseWedge[%
    linewidth=0.05,linecolor=red,%
    fillstyle=hlines,%
    fillcolor=red]{2}{1}{0}{130}
  \pstEllipseWedge[%
    linewidth=0.1,%
    linecolor=blue,%
    fillstyle=vlines,%
    fillcolor=blue]{1}{2}{-30}{190}
\end{pspicture}%
\end{minipage}%
\hfill%
\begin{minipage}{0.59\columnwidth}
\begin{lstlisting}
\begin{pspicture}(-2.25,-1.75)(2.25,1.75)
  \psgrid
  \pstEllipseWedge[%
    linewidth=0.05,linecolor=red,%
    fillstyle=hlines,%
    fillcolor=red]{2}{1}{0}{130}
  \pstEllipseWedge[%
    linewidth=0.1,linecolor=blue,%
    fillstyle=vlines,%
    fillcolor=blue]{1}{2}{-30}{190}
\end{pspicture}%
\end{lstlisting}
\end{minipage}%
\caption{The macro \CMD{pstEllipseWedge} which uses the \CMD{parametricplot} 
macro from \verbI{pst-plot}}\label{parametric-wedge}
\end{figure}

\subsection{New ellipse macros}

All macros defined in this package are original from Timothy Van Zandt and Denis Girou and 
modified by several other authors. The available macros are

{\small\begin{verbatim}
\psellipticarc[<options>]
    {<arrows>}(<center>)(a,b){start angle}{end angle}
\psellipticarcn[<options>]
    {<arrows>}(<center>)(a,b){start angle}{end angle}
\psellipticwedge[<options>]
    {<arrows>}(<center>)(a,b){start angle}{end angle}
\end{verbatim}}

\subsubsection{Arc of an ellipse}

Figure \ref{pst-ellipse-arc} shows different examples for this macro.

\begin{figure}[htb]
\begin{minipage}{0.39\columnwidth}
\begin{pspicture}(-2.25,-2.25)(2.25,2.25)
  \psgrid
  \psellipticarc[linewidth=0.1]{|->}(0,0)(1.5,1){0}{180}
  \psellipticarc[linecolor=red](0,0)(0.5,1.5){30}{320}
\end{pspicture}
\end{minipage}%
\hfill%
\begin{minipage}{0.59\columnwidth}
\begin{lstlisting}
\begin{pspicture}(-2.25,-2.25)(2.25,2.25)
  \psellipticarc[linewidth=0.1]{|->}(0,0)(1.5,1){0}{180}
  \psellipticarc[linecolor=red](0,0)(0.5,1.5){30}{320}
\end{pspicture}
\end{lstlisting}
\end{minipage}%
\caption{The macro \CMD{psellipticarc}}\label{pst-ellipse-arc}
\end{figure}

\subsection{Arc of an ellipse with anti clockwise direction}

Figure \ref{pst-ellipse-arcN} shows different examples for this macro which is the same than the one figure \ref{{pst-ellipse-arc}} only drawn anti clockwise.

\begin{figure}[htb]
\begin{minipage}{0.39\columnwidth}
\begin{pspicture}(-2.25,-2.25)(2.25,2.25)
	\psgrid
	\psellipticarcn[linewidth=0.1]{|->}(0,0)(1.5,1){0}{180}
	\psellipticarcn[linecolor=red](0,0)(0.5,1.5){30}{320}
\end{pspicture}
\end{minipage}%
\hfill%
\begin{minipage}{0.59\columnwidth}
\begin{lstlisting}
\begin{pspicture}(-2.25,-2.25)(2.25,2.25)
  \psellipticarcn[linewidth=0.1]{|->}(0,0)(1.5,1){0}{180}
  \psellipticarcn[linecolor=red](0,0)(0.5,1.5){30}{320}
\end{pspicture}
\end{lstlisting}
\end{minipage}%
\caption{The macro \CMD{psellipticarcn}}\label{pst-ellipse-arcN}
\end{figure}


\subsubsection{Wedge of an ellipse}

Figure \ref{pst-ellipse-wedge} shows different examples for this macro.

\begin{figure}[htb]
\begin{minipage}{0.39\columnwidth}
\begin{pspicture}(-2.25,-2.25)(2.25,2.25)
	\psgrid
	\psellipticwedge[%
		fillstyle=vlines,%
		linewidth=0.1](0,0)(1.5,1){0}{200}
	\psellipticwedge[%
		fillstyle=hlines,%
		linecolor=red](0,0)(0.5,1.5){30}{220}
\end{pspicture}
\end{minipage}%
\hfill%
\begin{minipage}{0.59\columnwidth}
\begin{lstlisting}
\begin{pspicture}(-2.25,-2.25)(2.25,2.25)
  \psgrid
  \psellipticwedge[%
    fillstyle=vlines,%
    linewidth=0.1](0,0)(1.5,1){0}{200}
  \psellipticwedge[%
    fillstyle=hlines,%
    linecolor=red](0,0)(0.5,1.5){30}{220}
\end{pspicture}
\end{lstlisting}
\end{minipage}%
\caption{The macro \CMD{psellipticwedge}}\label{pst-ellipse-wedge}
\end{figure}

\section{\texttt{pstricks.sty}}
In the past there were some problems with \verb+pstricks.tex+ and the package \verb+color.sty+. 
\verb+pstcol.sty+ tried to get rid of them but not with success in any case. The
new package \verb+pstricks.sty+ loads first \verb+pstricks.tex+, does some modifcation
to \verb+pstricks+, loads \verb+xcolor.sty+ and some more modifications to the the code to
get \verb+pstricks+ and colors work in a right way. It also renames the \verb+\scalebox+
macro to \verb+\psscalebox+ to prevend clashes with the one from the package \verb+graphicx.sty+ which has the
same name but another syntax. If you want to use the macro from \verb+graphicx+, then load this
package as the last one.


\end{document}