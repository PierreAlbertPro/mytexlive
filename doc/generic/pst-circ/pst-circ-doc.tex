%% $Id: pst-circ-doc.tex 5 2008-03-21 17:43:39Z herbert $
\listfiles
\documentclass{article}
\usepackage[a4paper]{geometry}
\usepackage[T1]{fontenc}
\usepackage{arev}%  use this if you do not have the palatino math font
%\usepackage{mathpazo}
\usepackage{fancyhdr}
\usepackage{url,array,longtable}
%
\usepackage[dvipsnames]{pstricks}
\usepackage{pst-circ}
\let\verPstCirc\fileversion
\usepackage{multicol}
\usepackage{showexpl,lscape}
\lstset{language=PSTricks,
    morekeywords={psGammaDist,psChiIIDist,psTDist,psFDist,psBetaDist,psPlotImpl}}
\lstdefinestyle{syntax}{backgroundcolor=\color{blue!20},numbers=none,xleftmargin=0pt,xrightmargin=0pt,
    frame=single}
%
%
\makeatletter
\def\@UrlFont{\small\ttfamily}
\renewenvironment{description}
  {\list{}{\labelwidth\z@ \itemindent-\leftmargin
    \itemsep0pt \parsep0pt
    \let\makelabel\descriptionlabel}}
  {\endlist}

\DeclareRobustCommand\cs[1]{\texttt{\char`\\#1}}
\def\PS{PostScript}
\newcommand{\CircPackage}{\textsf{`pst-circ'}}

\lhead{\CircPackage}\rhead{A PSTricks package for drawing electric circuits}
\pagestyle{fancy}

\psset{subgriddiv=0,griddots=10,gridlabels=7pt}%
%\showgrid
\usepackage[colorlinks,linktocpage]{hyperref}

\begin{document}

\title{\texttt{pst-circ}\\[10pt] 
    {\Large A PSTricks package for drawing electric circuits}\\\normalsize ver. \verPstCirc}
\author{Christophe Jorssen\thanks{\url{<CJ@PSTricks.de>}} \and  
Herbert Vo\ss\thanks{\url{<voss@PSTricks.de>}} \and Fran\c{c}ois Boone%
\thanks{\url{francois.boone@usherbrooke.ca} (microwave symbols)}}
\date{\today}
\maketitle

\begin{abstract}
\CircPackage{} is a PSTricks package to draw easily electric circuits. Most
dipoles, tripoles and quadrupoles used in classical electrotechnical
circuits are provided as graphical units which can readily be
interconnectedd to produce circuit diagrams of a reasonable level of complexity.
\end{abstract}

\clearpage

\setlength{\columnseprule}{0.6pt}
\begin{multicols}{2}
{\parskip 0pt \tableofcontents}
\end{multicols}

\clearpage
\section{Introduction}

The package \CircPackage{} is a collection of graphical elements based
on PStricks that can be used to facilitate display of electronic
circuit elements. For example, an equivalent circuit of a voltage
source, its source impedance, and a connected load can easily be
constructed along with arrows indicating current flow and potential
differences. The emphasis is upon the circuit elements and the
details of the exact placement are hidden as much as possible so the
author can focus on the circuitry without the distraction of sorting
out the underlying vector graphics.

\section{The basic system}

\subsection{Parameters}

There are specific paramaters defined to change easily the behaviour of the pst-circ
objects you are drawing.

\begin{longtable}{@{}>{\ttfamily}l l l@{}}
\textrm{\emph{name}} & \emph{type} & \emph{default}\\\hline
\endhead
intensity 	& boolean &  \emph{false} \\
intensitylabel  & string &  \emph{ } \\
intensitylabeloffset & dimension &  \emph{ 0.5} \\
intensitycolor & PSTricks color &  \emph{ black} \\
intensitylabelcolor & PSTricks color &  \emph{ black} \\
intensitywidth & dimension &  \emph{ \texttt{\cs{pslinewidth}}} \\
tension & boolean &  \emph{ false} \\
tensionlabel & string &  \emph{ } \\
tensionoffset & dimension &  \emph{ 1} \\
tensionlabeloffset & dimension &  \emph{ 1.2} \\
tensioncolor & PSTricks color &  \emph{ black} \\
tensionlabelcolor & PSTricks color &  \emph{ black} \\
tensionwidth & dimension &  \emph{ \texttt{\cs{pslinewidth}}} \\
labeloffset & dimension &  \emph{ 0.7} \\
labelangle & PSTricks label angle &  \emph{ 0} \\
labelInside & integer &  \emph{ 0} \\
dipoleconvention & & \emph{ receptor} \\
directconvetion & boolean &  \emph{ true} \\
dipolestyle & string &  \emph{ normal} \\
variable & boolean &  \emph{ false} \\
parallel & boolean &  \emph{ false} \\
parallelarm & dimension &  \emph{ 1.5} \\
parallelsep & real &  \emph{ 0} \\
parallelnode & boolean &  \emph{ false} \\
intersect & boolean &  \emph{ false} \\
intersectA & node & \\
intersectB & node & \\
OAinvert & boolean &  \emph{ true} \\
OAperfect & boolean &  \emph{ true} \\
OAiplus & boolean &  \emph{ false} \\
OAiminus & boolean &  \emph{ false} \\
OAiout & boolean &  \emph{ false} \\
OAipluslabel & string &  \emph{ } \\
OAiminuslabel & string &  \emph{ } \\
OAioutlabel & string &  \emph{ } \\
transistorcircle & boolean &  \emph{ true} \\
transistorinvert & boolean &  \emph{ false} \\
transistoribase & boolean &  \emph{ false} \\
transistoricollector & boolean &  \emph{ false} \\
transistoriemitter & boolean &  \emph{ false} \\
transistoribaselabel & string &  \emph{ } \\
transistoricollectorlabel & string &  \emph{ } \\
transistoriemitterlabel & string &  \emph{ } \\
TRot & angle &  \emph{ 0} \\
edge & macro &  \emph{ \texttt{\textbackslash ncangles}} \\
transistortype & string &  \emph{ PNP} \\
FETchanneltype & string &  \emph{ N} \\
primarylabel & string &  \emph{ } \\
secondarylabel & string &  \emph{ } \\
transformeriprimary & boolean &  \emph{ false} \\
transformerisecondary & boolean &  \emph{ false} \\
transformeriprimarylabel & string &  \emph{ } \\
transformerisecondarylabel & string &  \emph{ } \\
tripolestyle & string &  \emph{ normal}
\end{longtable}

\subsection{Macros}

\subsubsection{Dipole macros}

\begin{LTXexample}[width=3.5cm]
\begin{pspicture}(3,2)\psgrid
  \pnode(0,1){A}
  \pnode(3,1){B}
  \resistor(A)(B){$R$}
\end{pspicture}
\end{LTXexample}

\begin{LTXexample}[width=3.5cm]
\begin{pspicture}(3,2)\psgrid
  \pnode(0,1){A}
  \pnode(3,1){B}
  \capacitor(A)(B){$C$}
\end{pspicture}
\end{LTXexample}

\begin{LTXexample}[width=3.5cm]
\begin{pspicture}(3,2)\psgrid
  \pnode(0,1){A}
  \pnode(3,1){B}
  \battery(A)(B){$E$}
\end{pspicture}
\end{LTXexample}

\begin{LTXexample}[width=3.5cm]
\begin{pspicture}(3,2)\psgrid
  \pnode(0,1){A}
  \pnode(3,1){B}
  \coil(A)(B){$L$}
\end{pspicture}
\end{LTXexample}

\begin{LTXexample}[width=3.5cm]
\begin{pspicture}(3,2)\psgrid
  \pnode(0,1){A}
  \pnode(3,1){B}
  \Ucc(A)(B){$E$}
\end{pspicture}
\end{LTXexample}

\begin{LTXexample}[width=3.5cm]
\begin{pspicture}(3,2)\psgrid
  \pnode(0,1){A}
  \pnode(3,1){B}
  \Icc(A)(B){$\eta$}
\end{pspicture}
\end{LTXexample}

\begin{LTXexample}[width=3.5cm]
\begin{pspicture}(3,2)\psgrid
  \pnode(0,1){A}
  \pnode(3,1){B}
  \switch(A)(B){$K$}
\end{pspicture}
\end{LTXexample}

\begin{LTXexample}[width=3.5cm]
\begin{pspicture}(3,2)\psgrid
  \pnode(0,1){A}
  \pnode(3,1){B}
  \diode(A)(B){$D$}
\end{pspicture}
\end{LTXexample}

\begin{LTXexample}[width=3.5cm]
\begin{pspicture}(3,2)\psgrid
  \pnode(0,1){A}
  \pnode(3,1){B}
  \Zener(A)(B){$D$}
\end{pspicture}
\end{LTXexample}

\begin{LTXexample}[width=3.5cm]
\begin{pspicture}(3,2)\psgrid
  \pnode(0,1){A}
  \pnode(3,1){B}
  \lamp(A)(B){$\mathcal L$}
\end{pspicture}
\end{LTXexample}

\begin{LTXexample}[width=3.5cm]
\begin{pspicture}(3,2)\psgrid
  \pnode(0,1){A}
  \pnode(3,1){B}
  \circledipole(A)(B){$\mathcal G$}
\end{pspicture}
\end{LTXexample}

\begin{LTXexample}[width=3.5cm]
\begin{pspicture}[showgrid=true](3,2)
  \pnode(0,1){A}
  \pnode(3,1){B}
  \circledipole[labeloffset=0](A)(B){\Large\textbf{A}}
\end{pspicture}
\end{LTXexample}

\begin{LTXexample}[width=3.5cm]
\begin{pspicture}(3,2)\psgrid
  \pnode(0,1){A}
  \pnode(3,1){B}
  \LED(A)(B){$\mathcal D$}
\end{pspicture}
\end{LTXexample}

\bigskip
\subsubsection{Tripole macros}

Obviously, tripoles are not node connections. So \CircPackage{} tries its best to adjust the
position of the tripole regarding the three nodes. Internally, the connections are done by the
\cs{ncangle} pst-node macro. However, the auto-positionning and the auto-connections are not always
well chosen\footnote{This is something we are working on. I think that auto-positionning and
auto-connections should be done at PostScript level and not at PSTricks level. If someone has any
ideas, please mail us.}, so don't try to use tripole macros in strange situations!



\begin{LTXexample}[width=5.5cm]
\begin{pspicture}(5,3)\psgrid
  \pnode(0,0){A}
  \pnode(0,3){B}
  \pnode(5,1.5){C}
  \OA(B)(A)(C)
\end{pspicture}
\end{LTXexample}

\begin{LTXexample}[width=5.5cm]
\begin{pspicture}(5,3)\psgrid
  \pnode(0,0){A}
  \pnode(0,3){B}
  \pnode(5,1.5){C}
  \OA[OApower=true](B)(A)(C)
\end{pspicture}
\end{LTXexample}

\begin{LTXexample}[width=5.5cm]
\begin{pspicture}(5,3)\psgrid
  \pnode(0,1.5){A}
  \pnode(5,0){B}
  \pnode(5,3){C}
  \transistor[basesep=2cm,arrows=o-o](A)(B)(C)
\end{pspicture}
\end{LTXexample}

\begin{LTXexample}[width=5.5cm]
\begin{pspicture}(5,3)\psgrid
  \pnode(0,1.5){A}\psset{linewidth=1pt}
  \transistor[basesep=2cm,arrows=o-o](A){Emitter}{Collector}
  \psline{o-}(5,3)(3,3)(3,3|Collector)(Collector)
  \psline{o-}(5,0)(3,0)(3,3|Emitter)(Emitter)
  \psline{o-}(A)([nodesep=2]A)
\end{pspicture}
\end{LTXexample}

\begin{LTXexample}[width=5.5cm]
\begin{pspicture}(5,2)\psgrid
  \pnode(0,2){A}
  \pnode(5,2){B}
  \pnode(0,0){C}
  \Tswitch(A)(B)(C){$K$}
\end{pspicture}
\end{LTXexample}

\begin{LTXexample}[width=3.5cm]
\begin{pspicture}(3,3)\psgrid
  \pnode(0,1){A}
  \pnode(3,1){B}
  \pnode(3,2.25){C}
  \potentiometer[labeloffset=0pt](A)(B)(C){$P$}
\end{pspicture}
\end{LTXexample}

\bigskip
\subsubsection{Quadrupole macros}

\begin{LTXexample}[width=5.5cm]
\begin{pspicture}(5,5)\psgrid
  \pnode(0,5){A}
  \pnode(0,0){B}
  \pnode(5,5){C}
  \pnode(5,0){D}
  \transformer(A)(B)(C)(D){$\mathcal T$}
\end{pspicture}
\end{LTXexample}

\begin{LTXexample}[width=5.5cm]
\begin{pspicture}(5,3)\psgrid
  \pnode(0,2.5){A}
  \pnode(0,0.5){B}
  \pnode(4,2.5){C}
  \pnode(4,0.5){D}
  \optoCoupler(A)(B)(C)(D){$OC$}
\end{pspicture}
\end{LTXexample}

\clearpage
\subsubsection{Multidipole}


\cs{multidipole} is a macro that allows multiple dipoles to be drawn between two specified nodes.
\cs{multidipole} takes as many arguments as you want. \textbf{Note the \rnode{Dot}{dot} that is
after the last dipole.}

\bigskip
\begin{minipage}{7cm}
\begin{pspicture}[showgrid=true](7,7)
  \pnode(0,0){A}
  \pnode(7,7){B}
  \multidipole(A)(B)\resistor{$R$}%
    \capacitor[linecolor=red]{$C$}%
    \diode{$D$}{}\rnode{Dot2}{}.
\end{pspicture}
\end{minipage}\hfill
\begin{minipage}{6cm}
\verb+\begin{pspicture}[showgrid=true](7,7)+\\
\verb+  \pnode(0,0){A}+\\
\verb+  \pnode(7,7){B}+\\
\verb+  \multidipole(A)(B)\resistor{$R$}%+\\
\verb+    \capacitor[linecolor=red]{$C$}%+\\
\verb+    \diode{$D$}{}+\rnode{Dot2}{}.\\
\verb+\end{pspicture}+
\end{minipage}

\bigskip
\ncangles[linestyle=dashed,linecolor=gray,nodesep=3pt,armA=.5cm,angleA=-90,armB=4cm,angleB=0]{->}{Dot}{Dot2} 
Important: for the time being, \cs{multidipole} takes optional arguments but does not 
restore original values. We recommand not using it.


\bigskip
\subsubsection{Wire}

\begin{LTXexample}[width=3.5cm]
\begin{pspicture}(3,2)\psgrid
  \pnode(0,1){A}
  \pnode(3,1){B}
  \wire(A)(B)
\end{pspicture}
\end{LTXexample}

\bigskip
\subsubsection{Potential}

\begin{LTXexample}[width=3.5cm]
\begin{pspicture}(3,2)\psgrid
  \pnode(0,1){A}
  \pnode(3,1){B}
  \tension(A)(B){$u$}
\end{pspicture}
\end{LTXexample}

\bigskip
\subsubsection{ground}

\begin{LTXexample}[width=3.5cm]
\begin{pspicture}(3,2)\psgrid
  \pnode(0.5,1){A}
  \pnode(1,1){B}
  \pnode(2.5,1){C}
  \ground(A)
  \ground{135}(B)
  \ground[linecolor=blue]{180}(C)
\end{pspicture}
\end{LTXexample}

\bigskip
\subsection{Parameters}

\subsubsection{Label parameters}

\begin{LTXexample}[width=3.5cm]
\begin{pspicture}(3,1)\psgrid
  \pnode(0,.5){A}
  \pnode(3,.5){B}
  \resistor[labeloffset=0](A)(B){$R$}
\end{pspicture}
\end{LTXexample}

\begin{LTXexample}[width=3.5cm]
\begin{pspicture}(3,2)\psgrid
  \pnode(0,0){A}
  \pnode(3,2){B}
  \resistor[labelangle=:U](A)(B){$R$}
\end{pspicture}
\end{LTXexample}

\begin{LTXexample}[width=3.5cm]
\begin{pspicture}(3,2)\psgrid
  \pnode(0,0){A}
  \pnode(3,2){B}
  \resistor[labelangle=0](A)(B){$R$}
\end{pspicture}
\end{LTXexample}

\begin{LTXexample}[width=5.5cm]
\begin{pspicture}(5,5)\psgrid
  \pnode(0,5){A}
  \pnode(0,0){B}
  \pnode(5,5){C}
  \pnode(5,0){D}
  \transformer[primarylabel=$n_1$,
    secondarylabel=$n_2$](A)(B)(C)(D){$\mathcal T$}
\end{pspicture}
\end{LTXexample}

\begin{LTXexample}[width=3.5cm]
\begin{pspicture}(3,4.5)\psgrid
  \pnode(0,.5){A}
  \pnode(3,.5){B}
  \Ucc[labelInside=1](A)(B){$V$}
  \pnode(0,2){A}
  \pnode(3,2){B}
  \Ucc[labelInside=2](A)(B){$V$}
  \pnode(0,3.5){A}
  \pnode(3,3.5){B}
  \Ucc[labelInside=3](A)(B){$V$}
\end{pspicture}
\end{LTXexample}

\bigskip
\subsubsection{Current intensity and electrical potential parameters}

If the \texttt{intensity} parameter is set to \texttt{true}, an arrow is drawn on the wire
connecting one of the nodes to the dipole. If the \texttt{tension} parameter is set to \texttt{true},
an arrow is drawn parallel to the dipole.

The way those arrows are drawn is set by \texttt{dipoleconvention} and \texttt{directconvention}
parameters. \texttt{dipoleconvention} can take two values~: \texttt{generator} or \texttt{receptor}.
\texttt{directconvention} is a boolean.


\begin{LTXexample}[width=3.5cm]
\begin{pspicture}(3,2)\psgrid
  \pnode(0,.5){A}
  \pnode(3,.5){B}
  \resistor[intensity,tension](A)(B){}
\end{pspicture}
\end{LTXexample}

\begin{LTXexample}[width=3.5cm]
\begin{pspicture}(3,2)\psgrid
  \pnode(0,.5){A}
  \pnode(3,.5){B}
  \resistor[intensity,tension,
    dipoleconvention=generator](A)(B){}
\end{pspicture}
\end{LTXexample}

\begin{LTXexample}[width=3.5cm]
\begin{pspicture}(3,2)\psgrid
  \pnode(0,.5){A}
  \pnode(3,.5){B}
  \resistor[intensity,tension,
    directconvention=false](A)(B){}
\end{pspicture}
\end{LTXexample}

\begin{LTXexample}[width=3.5cm]
\begin{pspicture}(3,2)\psgrid
  \pnode(0,.5){A}
  \pnode(3,.5){B}
  \resistor[intensity,tension,
    dipoleconvention=generator,directconvention=false](A)(B){}
\end{pspicture}
\end{LTXexample}

If \texttt{intensitylabel} is set to an non empty argument, then \texttt{intensity} is automatically
set to true. If \texttt{tensionlabel} is set to an non empty argument, then \texttt{tension} is
automatically set to true.

\begin{LTXexample}[width=3.5cm]
\begin{pspicture}(3,2)\psgrid
  \pnode(0,.5){A}
  \pnode(3,.5){B}
  \resistor[intensitylabel=$i$,tensionlabel=$u$](A)(B){}
\end{pspicture}
\end{LTXexample}

\begin{LTXexample}[width=3.5cm]
\begin{pspicture}(3,2)\psgrid
  \pnode(0,1.5){A}
  \pnode(3,1.5){B}
  \resistor[intensitylabel=$i$,intensitylabeloffset=-0.5,
    tensionlabel=$u$,tensionlabeloffset=-1.2,
    tensionoffset=-1](A)(B){}
\end{pspicture}
\end{LTXexample}

\begin{LTXexample}[width=3.5cm]
\begin{pspicture}(3,2)\psgrid
  \pnode(0,.5){A}
  \pnode(3,.5){B}
  \resistor[intensitylabel=$i$,intensitywidth=3\pslinewidth,
    intensitycolor=red,intensitylabelcolor=yellow,
    tensionlabel=$u$,tensionwidth=2\pslinewidth,
    tensioncolor=green,tensionlabelcolor=blue](A)(B){}
\end{pspicture}
\end{LTXexample}

Some specific intensity parameters are available for tripoles and quadrupoles.

\begin{LTXexample}[width=5.5cm]
\begin{pspicture}(5,3)\psgrid
  \pnode(0,0){A}
  \pnode(0,3){B}
  \pnode(5,1.5){C}
  \OA[OAipluslabel=$i_+$,
    OAiminuslabel=$i_-$,
    OAioutlabel=$i_o$](B)(A)(C)
\end{pspicture}
\end{LTXexample}

\begin{LTXexample}[width=5.5cm]
\begin{pspicture}(5,3)\psgrid
  \pnode(0,1.5){A}
  \pnode(5,0){B}
  \pnode(5,3){C}
  \transistor[basesep=2cm,transistoribaselabel=$i_B$,
    transistoricollectorlabel=$i_C$,
    transistoriemitterlabel=$i_E$](A)(B)(C)
\end{pspicture}
\end{LTXexample}

\begin{LTXexample}[width=5.5cm]
\begin{pspicture}(5,5)\psgrid
  \pnode(0,5){A}
  \pnode(0,0){B}
  \pnode(5,5){C}
  \pnode(5,0){D}
  \transformer[transformeriprimarylabel=$i_1$,
    transformerisecondarylabel=$i_2$]%
    (A)(B)(C)(D){$\mathcal T$}
\end{pspicture}
\end{LTXexample}


\subsubsection{Parallel parameters}

If the \texttt{parallel} parameter is set to \texttt{true}, the dipole is drawn parallel to the line
connecting the nodes.

\begin{LTXexample}[width=3.5cm]
\begin{pspicture}(3,3)\psgrid
  \pnode(0,.5){A}
  \pnode(3,.5){B}
  \resistor(A)(B){}
  \resistor[parallel](A)(B){}
\end{pspicture}
\end{LTXexample}

\begin{LTXexample}[width=3.5cm]
\begin{pspicture}(3,3)\psgrid
  \pnode(0,.5){A}
  \pnode(3,.5){B}
  \resistor(A)(B){}
  \resistor[parallel,parallelsep=.5](A)(B){}
\end{pspicture}
\end{LTXexample}

\begin{LTXexample}[width=3.5cm]
\begin{pspicture}(3,3)\psgrid
  \pnode(0,.5){A}
  \pnode(3,.5){B}
  \resistor(A)(B){}
  \resistor[parallel,parallelsep=.3,
    parallelarm=2](A)(B){}
\end{pspicture}
\end{LTXexample}

\begin{LTXexample}[width=3.5cm]
\begin{pspicture}(3,3)\psgrid
  \pnode(0,.5){A}
  \pnode(3,.5){B}
  \resistor(A)(B){}
  \resistor[parallel,parallelsep=.3,
    parallelarm=2,parallelnode](A)(B){}
\end{pspicture}
\end{LTXexample}

\begin{LTXexample}[width=8.5cm]
\begin{pspicture}(8,8)\psgrid
  \pnode(0,0){A}
  \pnode(8,8){B}
  \multidipole(A)(B)\resistor{$R$}%
    \capacitor[linecolor=red]{$C$}%
    \coil[parallel,parallelsep=.1]{$L$}%
    \diode{$D$}.
\end{pspicture}
\end{LTXexample}

Note: When used with \cs{multidipole}, the  parallel \texttt{parameter}
must not be set for the first dipole.



\subsubsection{Wire intersections}

\begin{LTXexample}[width=3.5cm]
\begin{pspicture}(3,3)\psgrid
  \pnode(0,0){A}
  \pnode(3,3){B}
  \pnode(0,3){C}
  \pnode(3,0){D}
  \wire(A)(B)
  \wire[intersect,intersectA=A,intersectB=B](C)(D)
\end{pspicture}
\end{LTXexample}

Wire intersect parameters work also with \cs{multidipole}.

\begin{LTXexample}[width=6.5cm]
\begin{pspicture}(7,7)\psgrid
  \pnode(0,0){A}
  \pnode(6,6){B}
  \pnode(0,6){C}
  \pnode(6,0){D}
  \wire(A)(B)
  \multidipole(C)(D)\resistor{$R$}%
    \wire[intersect,intersectA=A,intersectB=B]%
    \capacitor{$C$}.
\end{pspicture}
\end{LTXexample}


\bigskip
\subsubsection{Dipole style parameters}

\begin{LTXexample}[width=3.5cm]
\begin{pspicture}[showgrid=true](3,2)
  \pnode(0,1){A}
  \pnode(3,1){B}
  \resistor[dipolestyle=zigzag](A)(B){$R$}
\end{pspicture}
\end{LTXexample}

\begin{LTXexample}[width=3.5cm]
\begin{pspicture}[showgrid=true](3,2)
  \pnode(0,1){A}
  \pnode(3,1){B}
  \resistor[dipolestyle=varistor](A)(B){U}
\end{pspicture}
\end{LTXexample}

\begin{LTXexample}[width=3.5cm]
\begin{pspicture}[showgrid=true](3,2)
  \pnode(0,1){A}
  \pnode(3,1){B}
  \capacitor[dipolestyle=chemical](A)(B){$C$}
\end{pspicture}
\end{LTXexample}

\begin{LTXexample}[width=3.5cm]
\begin{pspicture}(3,2)\psgrid
  \pnode(0,1){A}
  \pnode(3,1){B}
  \capacitor[dipolestyle=elektor](A)(B){$C$}
\end{pspicture}
\end{LTXexample}

\begin{LTXexample}[width=3.5cm]
\begin{pspicture}(3,2)\psgrid
  \pnode(0,1){A}
  \pnode(3,1){B}
  \capacitor[dipolestyle=elektorchemical](A)(B){$C$}
\end{pspicture}
\end{LTXexample}

\begin{LTXexample}[width=3.5cm]
\begin{pspicture}(3,2)\psgrid
  \pnode(0,1){A}
  \pnode(3,1){B}
  \capacitor[dipolestyle=crystal](A)(B){$Q$}
\end{pspicture}
\end{LTXexample}

\begin{LTXexample}[width=3.5cm]
\begin{pspicture}(3,2)\psgrid
  \pnode(0,1){A}
  \pnode(3,1){B}
  \coil[dipolestyle=rectangle](A)(B){$L$}
\end{pspicture}
\end{LTXexample}

\begin{LTXexample}[width=3.5cm]
\begin{pspicture}(3,2)\psgrid
  \pnode(0,1){A}
  \pnode(3,1){B}
  \coil[dipolestyle=curved](A)(B){$L$}
\end{pspicture}
\end{LTXexample}

\begin{LTXexample}[width=3.5cm]
\begin{pspicture}(3,2)\psgrid
  \pnode(0,1){A}
  \pnode(3,1){B}
  \coil[dipolestyle=elektor](A)(B){$L$}
\end{pspicture}
\end{LTXexample}

\begin{LTXexample}[width=3.5cm]
\begin{pspicture}(3,2)\psgrid
  \pnode(0,1){A}
  \pnode(3,1){B}
  \coil[dipolestyle=elektorcurved](A)(B){$L$}
\end{pspicture}
\end{LTXexample}

\begin{LTXexample}[width=3.5cm]
\begin{pspicture}(3,2)\psgrid
  \pnode(0,1){A}
  \pnode(3,1){B}
  \diode[dipolestyle=thyristor](A)(B){$T$}
\end{pspicture}
\end{LTXexample}

\begin{LTXexample}[width=3.5cm]
\begin{pspicture}(3,2)\psgrid
  \pnode(0,1){A}
  \pnode(3,1){B}
  \diode[dipolestyle=GTO](A)(B){$T$}
\end{pspicture}
\end{LTXexample}

\begin{LTXexample}[width=3.5cm]
\begin{pspicture}(3,2)\psgrid
  \pnode(0,1){A}
  \pnode(3,1){B}
  \diode[dipolestyle=triac](A)(B){$T$}
\end{pspicture}
\end{LTXexample}

\begin{LTXexample}[width=3.5cm]
\begin{pspicture}(3,2)\psgrid
  \pnode(0,1){A}
  \pnode(3,1){B}
  \resistor[variable](A)(B){$R$}
\end{pspicture}
\end{LTXexample}

\begin{LTXexample}[width=3.5cm]
\begin{pspicture}(3,2)\psgrid
  \pnode(0,1){A}
  \pnode(3,1){B}
  \capacitor[variable](A)(B){$C$}
\end{pspicture}
\end{LTXexample}

\begin{LTXexample}[width=3.5cm]
\begin{pspicture}(3,2)\psgrid
  \pnode(0,1){A}
  \pnode(3,1){B}
  \coil[variable](A)(B){$L$}
\end{pspicture}
\end{LTXexample}

\begin{LTXexample}[width=3.5cm]
\begin{pspicture}(3,2)\psgrid
  \pnode(0,1){A}
  \pnode(3,1){B}
  \battery[variable](A)(B){$U$}
\end{pspicture}
\end{LTXexample}

\begin{LTXexample}[width=3.5cm]
\begin{pspicture}(3,2)\psgrid
  \pnode(0,1){A}
  \pnode(3,1){B}
  \coil[dipolestyle=elektor,variable](A)(B){$L$}
\end{pspicture}
\end{LTXexample}

In the following example the parameter \verb|dipolestyle| is used for a tripole and quadrupole, because
the coils are drawn as rectangles and the resistor as a zigzag.

\begin{LTXexample}[width=3.5cm]
\begin{pspicture}(3,3)\psgrid
  \pnode(0,0){A}
  \pnode(3,3){B}
  \pnode(3,1.5){C}
  \potentiometer[dipolestyle=zigzag,%
  	labelangle=:U](A)(B)(C){$P$}
\end{pspicture}
\end{LTXexample}

\begin{LTXexample}[width=4.5cm]
\begin{pspicture}(4,4)\psgrid
  \pnode(0,4){A}
  \pnode(0,0){B}
  \pnode(4,4){C}
  \pnode(4,0){D}
  \transformer[dipolestyle=rectangle](A)(B)(C)(D){$\mathcal T$}
\end{pspicture}
\end{LTXexample}


\subsubsection{Tripole style parameters}

\begin{LTXexample}[width=5.5cm]
\begin{pspicture}(5,3)
  \pnode(0,2){A}
  \pnode(5,2){B}
  \pnode(0,0){C}
  \Tswitch[tripolestyle=left](A)(B)(C){$K$}
\end{pspicture}
\end{LTXexample}

\begin{LTXexample}[width=5.5cm]
\begin{pspicture}(5,3)
  \pnode(0,2){A}
  \pnode(5,2){B}
  \pnode(0,0){C}
  \Tswitch[tripolestyle=right](A)(B)(C){$K$}
\end{pspicture}
\end{LTXexample}

\begin{LTXexample}[width=5.5cm]
\begin{pspicture}(5,3)
  \pnode(0,3){A}
  \pnode(0,0){B}
  \pnode(5,1.5){C}
  \OA[tripolestyle=french](A)(B)(C)
\end{pspicture}
\end{LTXexample}

\subsubsection{Potentiometer tripole}

\begin{pspicture}(3,3)
  \psgrid
  \pnode(0,1){A}
  \pnode(3,1){B}
  \pnode(3,2){C}
  \potentiometer[labeloffset=0pt](A)(B)(C){P}
\end{pspicture}
\hfill
\begin{pspicture}(3,3)
  \psgrid
  \pnode(0,2.5){A}
  \pnode(3,2.5){B}
  \pnode(0,1){C}
  \potentiometer[labeloffset=0pt](A)(B)(C){P}
\end{pspicture}
\hfill
\begin{pspicture}(3,3)
  \psgrid
  \pnode(0,0){A}
  \pnode(3,2){B}
  \pnode(2.5,3){C}
  \potentiometer[labeloffset=0pt,labelangle=:U](A)(B)(C){P}
\end{pspicture}

\vspace{1cm}
\noindent
\begin{pspicture}(3,3)
  \psgrid
  \pnode(1,0){A}
  \pnode(1,3){B}
  \pnode(2.5,0){C}
  \potentiometer[labeloffset=0pt](A)(B)(C){P}
\end{pspicture}
\hfill
\begin{pspicture}(3,3)
  \psgrid
  \pnode(0,3){A}
  \pnode(3,0){B}
  \pnode(2,0){C}
  \potentiometer[labeloffset=0pt,labelangle=:U](A)(B)(C){P}
\end{pspicture}
\hfill
\begin{pspicture}(3,3)
  \psgrid
  \pnode(0,2){A}
  \pnode(3,2){B}
  \pnode(1.5,0){C}
  \potentiometer[labeloffset=0pt](A)(B)(C){P}
\end{pspicture}


\vspace{1cm}
\noindent
\begin{pspicture}(3,3)
  \psgrid
  \pnode(1,0){A}
  \pnode(1,3){B}
  \pnode(2.5,0){C}
  \potentiometer[dipolestyle=zigzag](A)(B)(C){P}
\end{pspicture}
\hfill
\begin{pspicture}(3,3)
  \psgrid
  \pnode(0,3){A}
  \pnode(3,0){B}
  \pnode(2,0){C}
  \potentiometer[dipolestyle=zigzag,labelangle=:U](A)(B)(C){P}
\end{pspicture}
\hfill
\begin{pspicture}(3,3)
  \psgrid
  \pnode(0,2){A}
  \pnode(3,2){B}
  \pnode(1.5,0){C}
  \potentiometer[dipolestyle=zigzag](A)(B)(C){P}
\end{pspicture}

\subsubsection{Other Parameters}

\begin{LTXexample}[width=5.5cm]
\begin{pspicture}(5,3)
  \pnode(0,0){A}
  \pnode(0,3){B}
  \pnode(5,1.5){C}
  \OA[OAinvert=false](B)(A)(C)
\end{pspicture}
\end{LTXexample}

\begin{LTXexample}[width=5.5cm]
\begin{pspicture}(5,3)
  \pnode(0,0){A}
  \pnode(0,3){B}
  \pnode(5,1.5){C}
  \OA[OAperfect=false](B)(A)(C)
\end{pspicture}
\end{LTXexample}

\begin{LTXexample}[width=5.5cm]
\begin{pspicture}(5,3)
  \pnode(0,1.5){A}
  \pnode(5,0){B}
  \pnode(5,3){C}
  \transistor[basesep=2cm,%
    transistorinvert,transistorcircle=false](A)(B)(C)
\end{pspicture}
\end{LTXexample}

\begin{LTXexample}[width=5.5cm]
\begin{pspicture}(5,3)
  \pnode(0,1.5){A}\psset{linewidth=1pt}
  \transistor[basesep=2cm,arrows=o-o,
    transistortype=FET](A){Emitter}{Collector}
  \psline{o-}(5,3)(3,3)(3,3|Collector)(Collector)
  \psline{o-}(5,0)(3,0)(3,3|Emitter)(Emitter)
  \psline{o-}(A)([nodesep=2]A)
\end{pspicture}
\end{LTXexample}

\begin{LTXexample}[width=5.5cm]
\begin{pspicture}(5,3)
  \pnode(0,1.5){A}\psset{linewidth=1pt}
  \transistor[basesep=2cm,arrows=o-o,
    transistortype=FET,
    FETchanneltype=P](A){Emitter}{Collector}
  \psline{o-}(5,3)(3,3)(3,3|Collector)(Collector)
  \psline{o-}(5,0)(3,0)(3,3|Emitter)(Emitter)
  \psline{o-}(A)([nodesep=2]A)
\end{pspicture}
\end{LTXexample}

\clearpage
\subsection{Special objects}

\subsubsection{\texttt{\textbackslash dashpot}}


\begin{LTXexample}[pos=t]
\newcommand*\pswall[3]{% ll ur lr
  \psframe[linecolor=white,fillstyle=hlines,hatchcolor=black](#1)(#2)% (ll)(ur)
  \psline[linecolor=black](#1)(#3)}
\begin{pspicture}[showgrid=true](0.5,1)(8,10)
  \rput(3,9.5){\sffamily \textbf{Viscoelasticity}}
  % Kelvin-Voigt model (spring and dashpot parallel): ===========
  \rput[c](1.75,8.85){\sffamily Kelvin-Voigt}
  \pswall{1,8}{2.5,8.5}{2.5,8}% top
  \psline(1.75,8)(1.75,7)% top vertical line
  % node definitions:
  \pnode(1,7){ul1}\pnode(2.5,7){ur1}  \pnode(1,3){ll1}\pnode(2.5,3){lr1}%
  \psline(ul1)(ur1)% top line 
  \psline(ll1)(lr1)% bottom line
  \resistor[dipolestyle=zigzag,linewidth=0.5pt](ul1)(ll1){}% spring
  \dashpot[linewidth=0.5pt](ur1)(lr1){}% dashpot
  \psline[arrowscale=3]{->}(1.75,3)(1.75,2)% force
  % Maxwell model (spring and dashpot serial): ==================
  \rput[c](4.5,8.85){\sffamily Maxwell}
  \pswall{4,8}{5,8.5}{5,8}% top
  \pnode(4.5,8){t}\pnode(4.5,4){b}% node definitions
  \resistor[dipolestyle=zigzag,linewidth=0.5pt,labeloffset=1.8](t)(b)% spring
  {\sffamily\small\begin{tabular}{c}\textbf{elasticity}\\(Hookean solid)\end{tabular}}% end spring
  \dashpot[linewidth=0.5pt,labeloffset=1.8](4.5,5)(4.5,3)% dashpot
  {\sffamily\small\begin{tabular}{c}\textbf{viscosity}\\(Newtonian fluid)\end{tabular}
  }% end dashpot
  \psline[arrowscale=3]{->}(4.5,3)(4.5,2)% force
\end{pspicture}
\end{LTXexample}



\subsection{Examples}

\begin{LTXexample}[width=8cm]
  \begin{pspicture}(-1.5,-1)(6,5)
%  \psgrid[subgriddiv=1,griddots=10]
  % Node definitions
  \pnode(0,0){A}
  \pnode(0,3){B}
  \pnode(4.5,3){C}
  \pnode(4.5,0){D}
  % Dipole node connection
  \Ucc[tension,dipoleconvention=generator](A)(B){$E$}
  \multidipole(B)(C)%
    \switch[intensitylabel=$i$]{$K$}%
    \resistor[labeloffset=0,tensionlabel=$u_R$]{$R$}.
  \capacitor[tensionlabel={$u_C$},
    tensionlabeloffset=-1.2,tensionoffset=-1,
    directconvention=false](D)(C){$C$}
  % Wire to complete circuit
  \wire(A)(D)
  % Ground
  \ground(D)
  \end{pspicture}
\end{LTXexample}

\begin{LTXexample}[width=8cm]
  \begin{pspicture}(-0.5,0)(7,8)
%  \psgrid[subgriddiv=1,griddots=10]
  % Node definitions
  \pnode(0.5,1){A}
  \pnode(3.5,1){B}
  \pnode(6.5,1){C}
  \pnode(0.5,4){D}
  \pnode(3.5,4){Minus}
  \pnode(3.5,3){Plus}
  \pnode(6.5,5){S}
  \pnode(3.5,5){E}
  % Dipole node connections
  \resistor(D)(Minus){$R_2$}
  \capacitor(E)(S){$C$}
  \resistor[parallel,parallelarm=2](E)(S){$R_1$}
  \OA[intensity](Minus)(Plus)(S)
  % Wires
  \wire(Minus)(E)
  \wire(Plus)(B)
  % Tensions
  \tension(A)(D){$u_E$}
  \makeatletter % (special tricks see below)
  \tension(C)(S@@){$u_S$}
  \tension[linecolor=blue](Plus@@)(Minus@@){$\epsilon$}
  \makeatother
  % Grounds
  \ground(A)
  \ground(B)
  \ground(C)
  \end{pspicture}
\end{LTXexample}

\begin{LTXexample}[width=8.5cm]
  \begin{pspicture}(-1,0)(7,8)
%  \psgrid[subgriddiv=1,griddots=10]
  % Node definitions
  \pnode(1,1){A}
  \pnode(1,7){B}
  \pnode(3,1){C}
  \pnode(3,7){D}
  % Dipole node connections
  \Ucc[tensionlabel=$E$](A)(B){}
  \resistor(B)(D){$R$}
  \coil(D)(C){$L$}
  \capacitor[parallel,parallelarm=2.5](D)(C){$C$}
  % Wire
  \wire(A)(C)
  \end{pspicture}
\end{LTXexample}

\begin{LTXexample}[width=8.5cm]
  \begin{pspicture}(-0.25,-0.25)(6,6)
%  \psgrid[subgriddiv=1,griddots=10]
  % Node definitions
  \pnode(0,3){A}
  \pnode(3,3){B}
  \pnode(6,3){C}
  % Dipole node connections
  \coil[intensitylabel=$i$](A)(B){$L$}
  \coil[intensitylabel=$i'$,intensitycolor=green,%
    parallel,parallelarm=2](B)(C){$L'$}
  \capacitor[parallel,parallelarm=-2](B)(C){$C$}
  \end{pspicture}
\end{LTXexample}

\begin{LTXexample}[width=8.5cm]
  \begin{pspicture}(6,6)
%  \psgrid[subgriddiv=1,griddots=10]
  % Node definitions
  \pnode(0,0){A}\pnode(6,0){B}
  \pnode(0.3,4){Cprime}\pnode(5.7,4){Dprime}
  \pnode(2.5,4){Gprime}\pnode(2.5,0){Hprime}
  \pnode(0,4){C}\pnode(6,4){D}
  \pnode(0.3,6){E}\pnode(5.7,6){F}
  \pnode(4,6){G}\pnode(4,0){H}
  \multidipole(G)(H)%
    \wire[intersect,
      intersectA=C,intersectB=D]
    \resistor{$R'_3$}.
  \resistor(E)(G){$R'_1$}
  \resistor(G)(F){$R'_2$}
  \multidipole(C)(D)\resistor{$R_1$}%
    \wire\resistor{$R_2$}.
  \wire(A)(B)\wire(Cprime)(E)
  \wire(Dprime)(F)
  \resistor(Hprime)(Gprime){$R_3$}
  \end{pspicture}
\end{LTXexample}



\begin{LTXexample}[pos=t]
  \begin{pspicture}(0,-0.25)(9,11)
  % Node definitions
  \pnode(0,0){A}\pnode(9,0){B}\pnode(0,6){C}\pnode(9,6){D}\pnode(4.5,1){E}\pnode(4.5,10.5){F}
  %
  \switch(A)(C){$K$}
  \multidipole(A)(B)\resistor{$R$}\battery[intensitylabel=$i$]{$V$}.
  \wire(B)(D)
  \multidipole(C)(D)\diode{$D$}\wire.
  \resistor[tensionlabel=$U_1$](C)(F){$R_1$} \resistor(C)(E){$R_4$}
  \capacitor[parallel,parallelarm=1.2,parallelsep=1.5](C)(E){$C_2$}
  \coil(E)(D){$L$}
  \capacitor[parallel,parallelarm=1.2,parallelsep=1.5](E)(D){$C_3$}
  \capacitor[tensionlabel=$U_2$](F)(D){$C_1$}
  \multidipole(E)(F)\wire\wire[intersect,intersectA=C,intersectB=D]%
    \circledipole[labeloffset=-0.7]{$E$}%
    \resistor[parallel,parallelsep=.6,parallelarm=.8]{$R$}.
  \end{pspicture}
\end{LTXexample}

\begin{LTXexample}[pos=t]
\begin{pspicture}(0,-0.2)(13,8)
  \psset{intensitycolor=red,intensitylabelcolor=red,tensioncolor=green,
    tensionlabelcolor=green, intensitywidth=3pt}
  \circledipole[tension,tensionlabel=$U_0$,
    tensionoffset=0.75,labeloffset=0](0,0)(0,6){\LARGE\textbf{=}}
  \wire[intensity,intensitylabel=$i_0$](0,6)(2.5,6)
  \diode[dipolestyle=thyristor](2.5,6)(4.5,6){$T_1$}
  \wire[intensity,intensitylabel=$i_1$](4.5,6)(6.5,6)
  \multidipole(6.5,7.5)(2.5,7.5)%
        \coil[dipolestyle=rectangle,labeloffset=-0.75]{$L_5$}%
        \diode[labeloffset=-0.75]{$D_5$}.
  \wire[intensity,intensitylabel=$i_5$](6.5,6)(6.5,7.5)
  \wire(2.5,7.5)(2.5,3)
  \wire[intensity,intensitylabel=$i_c$](2.5,4.5)(2.5,6)
  \qdisk(2.5,6){2pt}\qdisk(6.5,6){2pt}
  \diode[dipolestyle=thyristor](2.5,4.5)(4.5,4.5){$T_2$}
  \wire[intensity,intensitylabel=$i_2$](4.5,4.5)(6.5,4.5)
  \capacitor[tension,tensionlabel=$u_c$,tensionoffset=-0.75,
    tensionlabeloffset=-1](6.5,4.5)(6.5,6){$C_k$}
  \qdisk(2.5,4.5){2pt}\qdisk(6.5,4.5){2pt}
  \wire[intensity,intensitylabel=$i_3$](6.5,4.5)(6.5,3)
  \multidipole(6.5,3)(2.5,3)%
    \coil[dipolestyle=rectangle,labeloffset=-0.75]{$L_3$}%
    \diode[labeloffset=-0.75]{$D_3$}.
  \wire(6.5,6)(9,6)\qdisk(9,6){2pt}
  \diode(9,0)(9,6){$D_4$}
  \wire[intensity,intensitylabel=$i_4$](9,3.25)(9,6)
  \wire[intensity,intensitylabel=$i_a$](9,6)(11,6)
  \multidipole(11,6)(11,0)%
    \resistor{$R_L$}
    \coil[dipolestyle=rectangle]{$L_L$}
    \circledipole[labeloffset=0,tension,tensionoffset=0.7,tensionlabel=$U_B$]{\LARGE\textbf{=}}.
  \wire(0,0)(11,0)\qdisk(9,0){2pt}
  \pnode(12.5,5.5){A}\pnode(12.5,0.5){B}
  \tension(A)(B){$u_a$}
\end{pspicture}
\end{LTXexample}


\makeatletter
%
\def\REG{\@ifnextchar[{\pst@REG}{\pst@REG[]}}
%
\def\pst@REG[#1](#2)(#3)(#4)#5{{%
  \psset{dimen=middle,arm=0}%
  \psset{#1}
  \pst@getcoor{#2}\pst@tempa
  \pst@getcoor{#3}\pst@tempb
  \pst@getcoor{#4}\pst@tempc
  \pnode(!%
    \pst@tempa /Y1 exch \pst@number\psyunit div def
    /X1 exch \pst@number\psxunit div def
    \pst@tempb /Y2 exch \pst@number\psyunit div def
    /X2 exch \pst@number\psxunit div def
    \pst@tempc /Y3 exch \pst@number\psyunit div def
    /X3 exch \pst@number\psxunit div def
    /XC X1 X2 add 2 div def
    /YC Y1 2 mul Y3 add 3 div def
    /Xin XC 1 sub def
    /Yin YC 0.5 add def
    /Xout XC 1 add def
    /Yout Yin def
    /Xref XC def
    /Yref YC 1 sub def
    XC YC){C@}
  \pnode(! Xin Yin){in@}
  \pnode(! Xout Yout){out@}
  \pnode(! Xref Yref){ref@}
  \rput(C@){\pst@draw@REG}
  \ncangle{#2}{in@}
  \ncangle{#3}{out@}
  \ncangle{#4}{ref@}
  \rput(C@){#5}
  }\ignorespaces}
%
\def\pst@draw@REG{%
  \begingroup
  \psset{linewidth=1.5\pslinewidth}%
  \psframe(-1,-0.5)(1,0.75)
  \psline(-1.5,0.5)(-1,0.5)
  \psline(1.5,0.5)(1,0.5)
  \psline(0,-0.5)(0,-1)
  \endgroup
  }
%
\makeatother

The following example was written by Manuel Luque.

\begin{LTXexample}[pos=t]
  \begin{pspicture}(0,-0.5)(14,4)
%  \psgrid[subgriddiv=1,griddots=10]
  \pnode(0,0){B}\pnode(0,3){A}
  \pnode(2.5,3.5){C}\pnode(2.5,-0.5){D}\pnode(5,3){E}\pnode(6.5,1.5){F}
  \pnode(5,0){G}\pnode(3.5,1.5){H}     \pnode(8,2.5){I}\pnode(8,1){J}
  \pnode(10,2.5){K}\pnode(10,1){L}     \pnode(14,2.5){M}\pnode(12,1){N}
  \pnode(3,1){H'}\pnode(14,2.5){O}     \pnode(14,1){P}\pnode(13.5,1){Q}
  \transformer[transformeriprimarylabel=$i_1$,transformerisecondarylabel=$i_2$,
    primarylabel=$n_1$,secondarylabel=$n_2$](A)(B)(C)(D){$T_1$}
  {\psset{fillstyle=solid,fillcolor=black}
  \diode(H)(E){}\diode(H)(G){} \diode(E)(F){}\diode(G)(F){}}
  \capacitor[dipolestyle=chemical](I)(J){}  \capacitor(K)(L){}
  \REG(K)(M)(N)%
    {\shortstack{\textsf{%
    \textbf{\large LM7805}}\\\textbf{+5V}}}
  \ncangle{I}{F}\psline(I)(K)  \ncangle{E}{C}\ncangle{G}{D}
  \ncangle[arm=0]{P}{Q}        \ncangle[arm=0]{H}{H'}
  \ground(H')\ground(J)\ground(L)\ground(N)
  \ground(Q)\qdisk(I){1.5pt}\qdisk(K){1.5pt}\qdisk(E){1.5pt}
  \qdisk(G){1.5pt}\qdisk(H){1.5pt}\qdisk(F){1.5pt}
  \pscircle[fillstyle=solid](A){0.075} \pscircle[fillstyle=solid](B){0.075}
  \pscircle[fillstyle=solid](P){0.075} \pscircle[fillstyle=solid](O){0.075}
  \end{pspicture}
\end{LTXexample}

\clearpage
The following example was written by Lionel Cordesses.



\begin{LTXexample}[pos=t]
  \begin{pspicture}(11,3)
  \psset{dipolestyle=elektor}
  \pnode(1,2){Vin}\pnode(0.5,2){S}\pnode(0.5,0){Sm}
  \pnode(2.5,2){A}\pnode(4.5,2){B}\pnode(6.5,2){C}
  \pnode(8,2){Cd}\pnode(8.5,2){D}\pnode(9.5,2){E}
  \pnode(2.5,0){Am}\pnode(4.5,0){Bm}\pnode(6.5,0){Cm}
  \pnode(8.5,0){Dm}\pnode(9.5,0){Em}
  \Ucc[labeloffset=0.9](Sm)(S){$V_{in}$}\resistor(Vin)(A){$R$}
  \capacitor(A)(Am){$C_1$} \capacitor(B)(Bm){$C_3$}
  \capacitor[labeloffset=-0.7](D)(Dm){$C_n$}\resistor(E)(Em){$R$}
  \coil(A)(B){$L_2$}\coil(B)(C){$L_4$}
  \wire(Am)(Bm)\wire(Bm)(Cm)\wire(Cm)(Dm)\wire(Dm)(Em)\wire(D)(E)
  \wire(Cd)(D)\psline[linestyle=dashed](C)(Cd)
  \wire(S)(Vin)\wire(Sm)(Am)
  \pscircle*(D){2\pslinewidth} \pscircle*(Dm){2\pslinewidth}
  \pscircle*(A){2\pslinewidth} \pscircle*(Am){2\pslinewidth}
  \pscircle*(B){2\pslinewidth} \pscircle*(Bm){2\pslinewidth}
  \end{pspicture}
\end{LTXexample}

\clearpage
The following example was written by Christian Hoffmann.


\begin{LTXexample}[pos=t]
  \SpecialCoor
  \begin{pspicture}(0,-1)(7,6.5)%\psgrid
  \pnode(0,6){plus}
  \pnode(3,3){basis}
  \pnode([nodesep=-2] basis){schalter}
  \pnode(0,0){masse}
  \wire[arrows=o-*](plus)(basis|plus)
  \uput[l](plus){$U_0$}
  \resistor[labeloffset=.8](basis|plus)(basis){$R_1$}
  \transistor[basesep=2cm](basis){emitter}{kollektor}
  \wire[arrows=-*](schalter)(basis)
%  \wire(basis)([nodesep=2] basis)
  \wire(TBaseNode)(basis)
  \switch(schalter|masse)(schalter){S}
  \lamp(kollektor|plus)(kollektor){L}
  \resistor(kollektor|plus)(basis|plus){$R_2$}
  \wire(emitter)(emitter|masse)
  \wire(emitter|masse)(basis|masse)
  \capacitor(basis)(basis|masse){$C_1$}
  \wire[arrows=*-](basis|masse)(schalter|masse)
  \wire[arrows=*-o](schalter|masse)(masse)
  \end{pspicture}
\end{LTXexample}




\clearpage
\section{Microwave symbols}
Since for microwave signal, the direction in which the signal spreads is very important, 
There are  dipoleinput or tripoleinput or quadripoleinput and arrowinput parameters. 
The value of theses parameters are left or right for the first one and true or false for second one. 
%
\begin{verbatim}
%%%%%
\ifPst@inputarrow
   \ifx\psk@Dinput\pst@Dinput@right
       \pcline[arrows=-C](#2)(dipole@1)
       \pcline[arrows=->,arrowinset=0](#3)(dipole@2)
    \else
       \pcline[arrows=->,arrowinset=0](#2)(dipole@1)
       \pcline[arrows=C-](dipole@2)(#3)
   \fi
\else
   \pcline[arrows=-C](#2)(dipole@1)
   \pcline[arrows=C-](dipole@2)(#3)
\fi
\pcline[fillstyle=none,linestyle=none](#2)(#3)
%%%%%
\end{verbatim}
The last line is to correct somme problems when I use colors (see example2)

To add color in components (Monopole, tripole and Quadripole) I add a new 
argument since I don't know how to do this by another way. However, 
I think it is not the optimal solution. For dipole, to put commands for 
color in the first optionnal argument is ok.

Finally, something doesn't work with multidipole: this following example works:
\begin{LTXexample}[width=3.5cm,rframe={}]
\begin{pspicture}(4,2)\psgrid
  \pnode(0.5,1){A}
  \pnode(3.5,1){B}
  \multidipole(A)(B)\filter{BPF}%
    \resistor{$R$}.
\end{pspicture}
\end{LTXexample}

However, this following one doesn't work:
\begin{lstlisting}
\begin{LTXexample}[width=3.5cm,rframe={}]
\begin{pspicture}(4,2)\psgrid
  \pnode(0.5,1){A}
  \pnode(3.5,1){B}
  \multidipole(A)(B)\amplifier{LNA}%
    \resistor{$R$}.
\end{pspicture}
\end{LTXexample}
\end{lstlisting}


\subsection{New monopole components}
\subsubsection{New ground}
\begin{description}
  \item[groundstyle:]  ads | old | triangle
\end{description}

\begin{LTXexample}[width=3.5cm,rframe={}]
\begin{pspicture}(3,2)\psgrid
  \pnode(0.5,1){A}
  \pnode(1,1){B}
  \pnode(2.5,1){C}
  \newground(A)
  \newground[groundstyle=old]{135}(B)
  \newground[linecolor=blue,groundstyle=triangle]{180}(C)
\end{pspicture}
\end{LTXexample}


\subsubsection{Antenna}
\begin{description}
  \item[antennastyle:]  two | three | triangle
\end{description}

\begin{LTXexample}[width=3.5cm,rframe={}]
\begin{pspicture}(3,2)\psgrid
  \pnode(1,0.5){A}
  \antenna[antennastyle=three](A)
\end{pspicture}
\end{LTXexample}

\begin{LTXexample}[width=3.5cm,rframe={}]
\begin{pspicture}(3,2)\psgrid
  \pnode(1,0.5){A}
  \antenna(A)
\end{pspicture}
\end{LTXexample}

\begin{LTXexample}[width=3.5cm,rframe={}]
\begin{pspicture}(3,2)\psgrid
  \pnode(1,0.5){A}
  \antenna[antennastyle=triangle](A)
\end{pspicture}
\end{LTXexample}


\subsection{New monopole macro-components}
\subsubsection{Oscillator}
\begin{description}
  \item[output:]  top | right | bottom | left
  \item[inputarrow:] false | true
  \item[LOstyle:] -- | crystal
\end{description}

\begin{LTXexample}[width=3.5cm,rframe={}]
\begin{pspicture}(3,2)\psgrid
  \pnode(1,1){A}
  \oscillator[output=left,inputarrow=false](A)%
    {$f_{LO}$}{}
\end{pspicture}
\end{LTXexample}

\begin{LTXexample}[width=3.5cm,rframe={}]
\begin{pspicture}(3,2)\psgrid
  \pnode(1,1){A}
  \oscillator[output=top,inputarrow=true,LOstyle=crystal](A)%
    {f$_{\textrm{LO}}$}{}
\end{pspicture}
\end{LTXexample}

\begin{LTXexample}[width=3.5cm,rframe={}]
\begin{pspicture}(3,2)\psgrid
  \pnode(1,1){A}
  \oscillator[output=right,inputarrow=false](A)%
    {$f_{LO}$}{fillstyle=solid,fillcolor=blue}
\end{pspicture}
\end{LTXexample}

\begin{LTXexample}[width=3.5cm,rframe={}]
\begin{pspicture}(3,2)\psgrid
  \pnode(1,1){A}
  \oscillator[output=bottom,inputarrow=false](A)%
    {$f_{LO}$}{}
\end{pspicture}
\end{LTXexample}

\subsection{New dipole macro-components}
\subsubsection{Filters}
\begin{description}
  \item[dipolestyle:]  bandpass | lowpass | highpass
  \item[inputarrow:] false | true
  \item[dipoleinput:] left | right
\end{description}

\begin{LTXexample}[width=3.5cm,rframe={}]
\begin{pspicture}(3,2)\psgrid
  \pnode(0,1){A}
  \pnode(3,1){B}
  \filter(A)(B){BPF}
\end{pspicture}
\end{LTXexample}

\begin{LTXexample}[width=3.5cm,rframe={}]
\begin{pspicture}(3,2)\psgrid
  \pnode(0,1){A}
  \pnode(3,1){B}
  \filter[dipolestyle=lowpass,fillstyle=solid,%
    fillcolor=red](A)(B){LPF}
\end{pspicture}
\end{LTXexample}

\begin{LTXexample}[width=3.5cm,rframe={}]
\begin{pspicture}(3,2)\psgrid
  \pnode(0,1){A}
  \pnode(3,1){B}
  \filter[dipolestyle=highpass,dipoleinput=right,%
    inputarrow=true](A)(B){HPF}
\end{pspicture}
\end{LTXexample}

\begin{LTXexample}[width=3.5cm,rframe={}]
\begin{pspicture}(3,2)\psgrid
  \pnode(0,1){A}
  \pnode(3,1){B}
  \filter[dipolestyle=highpass,inputarrow=true](A)(B){BPF}
\end{pspicture}
\end{LTXexample}

\subsubsection{Isolator}
\begin{description}
  \item[inputarrow:] false | true
  \item[dipoleinput:] left | right
\end{description}

\begin{LTXexample}[width=3.5cm,rframe={}]
\begin{pspicture}(3,2)\psgrid
  \pnode(0,1){A}
  \pnode(3,1){B}
  \isolator[inputarrow=true](A)(B){}
\end{pspicture}
\end{LTXexample}

\begin{LTXexample}[width=3.5cm,rframe={}]
\begin{pspicture}(3,2)\psgrid
  \pnode(0,1){A}
  \pnode(3,1){B}
  \isolator[dipoleinput=right,inputarrow=true,fillstyle=solid,%
  fillcolor=yellow](A)(B){Isolator}
\end{pspicture}
\end{LTXexample}

\begin{LTXexample}[width=3.5cm,rframe={}]
\begin{pspicture}(3,2)\psgrid
  \pnode(0,1){A}
  \pnode(3,1){B}
  \isolator[dipoleinput=left](A)(B){}
\end{pspicture}
\end{LTXexample}

\subsubsection{Frequency multiplier/divider}
\begin{description}
  \item[dipolestyle:] multiplier | divider
  \item[value:] N | $n\in N$
  \item[programmable:] false | true
  \item[inputarrow:] false | true
  \item[dipoleinput:] left | right
\end{description}

\begin{LTXexample}[width=3.5cm,rframe={}]
\begin{pspicture}(3,2)\psgrid
  \pnode(0,1){A}
  \pnode(3,1){B}
  \freqmult[dipolestyle=divider,inputarrow=true](A)(B){}
\end{pspicture}
\end{LTXexample}

\begin{LTXexample}[width=3.5cm,rframe={}]
\begin{pspicture}(3,2)\psgrid
  \pnode(0,1){A}
  \pnode(3,1){B}
  \freqmult[dipolestyle=multiplier,value=10](A)(B){}
\end{pspicture}
\end{LTXexample}

\begin{LTXexample}[width=3.5cm,rframe={}]
\begin{pspicture}(3,3)\psgrid
  \pnode(0,1.5){A}
  \pnode(3,1.5){B}
  \freqmult[dipolestyle=multiplier,programmable=true,%
            labeloffset=-1,%
            dipoleinput=right,%
            inputarrow=true,
            fillstyle=solid,
            fillcolor=green](A)(B){10<N<35}
\end{pspicture}
\end{LTXexample}

\subsubsection{Phase shifter}
\begin{description}
  \item[inputarrow:] false | true
  \item[dipoleinput:] left | right
\end{description}

\begin{LTXexample}[width=3.5cm,rframe={}]
\begin{pspicture}(3,2)\psgrid
  \pnode(0,1){A1}
  \pnode(3,1){A2}
  \phaseshifter(A1)(A2){}
\end{pspicture}
\end{LTXexample}

\begin{LTXexample}[width=3.5cm,rframe={}]
\begin{pspicture}(3,2)\psgrid
  \pnode(0,1){B1}
  \pnode(3,1){B2}
  \phaseshifter[inputarrow=true,%
    dipoleinput=right,fillstyle=solid,fillcolor=red]%
        (B1)(B2){90\ensuremath{^\circ}}
\end{pspicture}
\end{LTXexample}

\subsubsection{VCO}
\begin{description}
  \item[inputarrow:] false | true
  \item[dipoleinput:] left | right
\end{description}

\begin{LTXexample}[width=3.5cm,rframe={}]
\begin{pspicture}(3,2)\psgrid
  \pnode(0,1){A1}
  \pnode(3,1){A2}
  \vco[fillstyle=solid,fillcolor=yellow](A1)(A2){}
\end{pspicture}
\end{LTXexample}

\begin{LTXexample}[width=3.5cm,rframe={}]
\begin{pspicture}(3,2)\psgrid
  \pnode(0,1){B1}
  \pnode(3,1){B2}
  \vco[dipoleinput=right,inputarrow=true](B1)(B2){VCO}
\end{pspicture}
\end{LTXexample}

\subsubsection{Amplifier}
\begin{description}
  \item[inputarrow:] false | true
  \item[dipoleinput:] left | right
\end{description}

\begin{LTXexample}[width=3.5cm,rframe={}]
\begin{pspicture}(3,2)\psgrid
  \pnode(0,1){A}
  \pnode(3,1){B}
  \amplifier[inputarrow=true](A)(B){}
\end{pspicture}
\end{LTXexample}

\begin{LTXexample}[width=3.5cm,rframe={}]
\begin{pspicture}(3,2)\psgrid
  \pnode(0,1){A}
  \pnode(3,1){B}
  \amplifier[dipoleinput=right,inputarrow=true](A)(B){PA}
\end{pspicture}
\end{LTXexample}

\begin{LTXexample}[width=3.5cm,rframe={}]
\begin{pspicture}(3,2)\psgrid
  \pnode(0,1){A}
  \pnode(3,1){B}
  \amplifier[dipoleinput=left](A)(B){LNA}
\end{pspicture}
\end{LTXexample}

\subsubsection{Detector}
\begin{description}
  \item[inputarrow:] false | true
  \item[dipoleinput:] left | right
\end{description}

\begin{LTXexample}[width=3.5cm,rframe={}]
\begin{pspicture}(3,2)\psgrid
  \pnode(0,1){A}
  \pnode(3,1){B}
  \detector[inputarrow=true](A)(B){}
\end{pspicture}
\end{LTXexample}

\begin{LTXexample}[width=3.5cm,rframe={}]
\begin{pspicture}(3,2)\psgrid
  \pnode(0,1){A}
  \pnode(3,1){B}
  \detector[dipoleinput=right,inputarrow=true](A)(B){}
\end{pspicture}
\end{LTXexample}

\begin{LTXexample}[width=3.5cm,rframe={}]
\begin{pspicture}(3,2)\psgrid
  \pnode(0,1){A}
  \pnode(3,1){B}
  \detector[dipoleinput=left](A)(B){}
\end{pspicture}
\end{LTXexample}

\subsection{New tripole macro-components}
\subsubsection{Mixer}
\begin{description}
  \item[tripolestyle:] bottom | top
  \item[tripoleconfig:] left | right
  \item[inputarrow:] false | true
\end{description}

\begin{LTXexample}[width=3.5cm,rframe={}]
\begin{pspicture}(3,2)\psgrid
  \pnode(0.5,1){A}
  \pnode(2.5,1){B}
  \pnode(1.5,2){C}
  \mixer[tripolestyle=top,inputarrow=true](A)(B)(C)%
    {Mixer}{}
\end{pspicture}
\end{LTXexample}

\begin{LTXexample}[width=3.5cm,rframe={}]
\begin{pspicture}(3,2)\psgrid
  \pnode(0.5,1){A}
  \pnode(2.5,1){B}
  \pnode(1.5,0){C}
  \mixer[inputarrow=true,tripoleinput=right](A)(B)(C)%
    {Mixer}{fillstyle=solid,fillcolor=yellow}
\end{pspicture}
\end{LTXexample}

\subsubsection{Circulator}
\begin{description}
  \item[tripolestyle:] circulator | isolator
  \item[inputarrow:] false | true
  \item[tripoleinput:] left | right
\end{description}

\begin{LTXexample}[width=3.5cm,rframe={}]
\begin{pspicture}(3,2)\psgrid
  \pnode(0.5,1){A}
  \pnode(2.5,1){B}
  \pnode(1.5,0){C}
  \circulator{0}(A)(B)(C){Circulator}{}
\end{pspicture}
\end{LTXexample}

\begin{LTXexample}[width=3.5cm,rframe={}]
\begin{pspicture}(3,3)\psgrid
  \pnode(1.5,0.5){A}
  \pnode(1.5,2.5){B}
  \pnode(0.5,1.5){C}
  \circulator[tripolestyle=isolator,inputarrow=true]{90}%
    (A)(B)(C){Isolator}{}
\end{pspicture}
\end{LTXexample}

\begin{LTXexample}[width=3.5cm,rframe={}]
\begin{pspicture}(3,2)\psgrid
  \pnode(0.5,1){A}
  \pnode(2.5,1){B}
  \pnode(1.5,0){C}
  \circulator[tripoleconfig=right,tripolestyle=isolator,%
    inputarrow=true,tripoleinput=right]{0}%
    (B)(A)(C){Isolator}{}
\end{pspicture}
\end{LTXexample}

\begin{LTXexample}[width=3.5cm,rframe={}]
\begin{pspicture}(3,2)\psgrid
  \pnode(0.5,1){A}
  \pnode(2.5,1){B}
  \pnode(1.5,2){C}
  \circulator[tripoleconfig=right,inputarrow=true]{180}%
    (A)(B)(C){Isolator}{fillstyle=solid,fillcolor=red}
\end{pspicture}
\end{LTXexample}

\subsubsection{Agc}
\begin{description}
  \item[inputarrow:] false | true
  \item[tripoleinput:] left | right
\end{description}

\begin{LTXexample}[width=3.5cm,rframe={}]
\begin{pspicture}(3,2)\psgrid
  \pnode(0.5,1){A}
  \pnode(2.5,1){B}
  \pnode(1.5,0){C}
  \agc(A)(B)(C){AGC}{fillstyle=solid,fillcolor=yellow}
\end{pspicture}
\end{LTXexample}

\begin{LTXexample}[width=3.5cm,rframe={}]
\begin{pspicture}(3,2)\psgrid
  \pnode(0.5,1){A}
  \pnode(2.5,1){B}
  \pnode(1.5,0){C}
  \agc[tripoleinput=right,inputarrow=true](A)(B)(C)%
    {AGC}{fillstyle=solid,fillcolor=blue}
\end{pspicture}
\end{LTXexample}

\subsection{New quadripole macro-components}
\subsubsection{Coupler}
\begin{description}
  \item[couplerstyle:] hybrid | directional
  \item[inputarrow:] false | true
  \item[quadripoleinput:] left | right
\end{description}

\begin{LTXexample}[width=3.5cm,rframe={}]
\begin{pspicture}(3,2)\psgrid
  \pnode(0,1.4){A}
  \pnode(0,0.6){B}
  \pnode(3,1.4){C}
  \pnode(3,0.6){D}
  \coupler[couplerstyle=hybrid,inputarrow=true](A)(B)(C)(D)%
    {Hyb. $180$\ensuremath{^\circ}}%
    {fillstyle=solid,fillcolor=yellow}
\end{pspicture}
\end{LTXexample}

\begin{LTXexample}[width=3.5cm,rframe={}]
\begin{pspicture}(3,2)\psgrid
  \pnode(0,1.4){A}
  \pnode(0,0.6){B}
  \pnode(3,1.4){C}
  \pnode(3,0.6){D}
  \coupler[couplerstyle=directional](A)(B)(C)(D){10~dB}{}
\end{pspicture}
\end{LTXexample}

\begin{LTXexample}[width=3.5cm,rframe={}]
\begin{pspicture}(3,2)\psgrid
  \pnode(0,1.4){A}
  \pnode(0,0.6){B}
  \pnode(3,1.4){C}
  \pnode(3,0.6){D}
  \coupler[couplerstyle=hybrid,inputarrow=true,%
    quadripoleinput=right](A)(B)(C)(D)%
    {Hyb. $180$\ensuremath{^\circ}}{}
\end{pspicture}
\end{LTXexample}

\begin{LTXexample}[width=3.5cm,rframe={}]
\begin{pspicture}(3,2)\psgrid
  \pnode(0,1.4){A}
  \pnode(0,0.6){B}
  \pnode(3,1.4){C}
  \pnode(3,0.6){D}
  \coupler[couplerstyle=directional,quadripoleinput=right,%
  inputarrow=true](A)(B)(C)(D){10~dB}{}
\end{pspicture}
\end{LTXexample}


\subsection{Examples}
\subsubsection{Radiometer block diagram example}
From Chang, K., RF and Microwave Wireless Systems, Wiley InterScience, page 319, ISBN 0-471-35199-7

\noindent
\resizebox{\linewidth}{!}{%
\begin{pspicture}(1,2)(19,9)
  \pnode(2,8){A}
  \antenna{90}(A)
  \rput(4,8){\rnode{B}{\psframebox{\begin{tabular}{c}Ferrite\\Switch\end{tabular}}}}
  \ncline{A}{B}
  %%% Branche Calibration
  \pnode(4,6){C}
  \pnode(4,4){D}
  \pnode(5,5){E}
  \circulator[tripolestyle=isolator,tripoleconfig=right]{90}(C)(D)(E){Isolator}{}
  \ncline{B}{C}
  \pnode(3,3){F}
  \pnode(5,3){G}
  \resistor[unit=0.5,dipolestyle=zigzag,variable=true](F)(G){}
  \pnode(4,3){H}
  \ncline{D}{H}
  \rput[t](4,2.75){%
    \begin{tabular}{c}
        Hot and Cold\\
        loads for calibration
        \end{tabular}}
  %%% Branche r�ception
  \pnode(6,8){R1}
  \pnode(8,8){R2}
  \pnode(7,7){X1}
  \circulator[tripolestyle=isolator,tripoleconfig=right]{180}(R1)(R2)(X1){Isolator}{}
  \ncline{B}{R1}
  \pnode(10,8){R3}
  \pnode(9,7){X2}
  \mixer[inputarrow=true](R2)(R3)(X2){Mixer}{}
  \pnode(9,6){X3}
  \oscillator[output=top](X3){LO}{}
  \pnode(12,8){R4}
  \ncline{R3}{R4}
  \naput{0.5~GHZ}
  \pnode(14,8){R5}
  \filter(R4)(R5){BPF}%
  \pnode(16,8){R6}
  \amplifier[inputarrow=true](R5)(R6){IF~Amp}
  \pnode(18,8){R7}
  \detector[inputarrow=true](R6)(R7){Detector}
  \pnode(18,4){R8}
  \amplifier[inputarrow=true,labeloffset=-1](R7)(R8){Amp}
  \pscircle[fillstyle=solid,fillcolor=white](18,4){0.1}
  \rput[t](18,3.9){%
    \begin{tabular}{c}
        Output\\
        for processing
    \end{tabular}}
\end{pspicture}}

\clearpage
\begin{landscape}
\subsubsection{Ku-band Transceiver}
\resizebox{\linewidth}{!}{%
\psset{unit=1cm}
\begin{pspicture}(0,-3.5)(29.25,11)
  \rput[r](1.9,8){70/140MHz}
  \pnode(2,8){N1}
  \pnode(4,8){N2}
  \amplifier[fillstyle=solid,fillcolor=Thistle,inputarrow=true](N1)(N2){IF~Amp}

  \pnode(6,8){N3}
  \pnode(5,7){D1}
  \mixer(N2)(N3)(D1){}{}

  \pnode(5,5){D2}
  \amplifier[fillstyle=solid,fillcolor=CornflowerBlue,labeloffset=-1.5](D2)(D1){L-Band Buffers}
  \pnode(3,5){D3}
  \amplifier[fillstyle=solid,fillcolor=CornflowerBlue](D3)(D2){}
  \pnode(2,5){D4}
  \oscillator[output=right](D4){VCO}{fillstyle=solid,fillcolor=Orange}
  \psframe(1.25,3)(2.75,5.75)
  \rput[b](2,3.1){\large\textbf{L-band}}
  \pnode(5,3){D5}
  \amplifier[fillstyle=solid,fillcolor=CornflowerBlue](D2)(D5){}

  \pnode(2,2){R1}
  \pnode(4,2){R2}
  \amplifier[fillstyle=solid,fillcolor=Thistle](R2)(R1){IF~Amp}
  \rput[r](1.9,2){70/140MHz}
  \pnode(6,2){R3}
  \filter(R3)(R2){}
  \pnode(8,2){R4}
  \pnode(7,3){D6}
  \mixer[tripolestyle=top](R3)(R4)(D6){}{}
  \ncline{D5}{D6}

  \pnode(8,8){N4}
  \filter(N3)(N4){}
  \pnode(10,8){N5}
  \amplifier[fillstyle=solid,fillcolor=NavyBlue](N4)(N5){L-band Amp}
  \pnode(12,8){N6}
  \resistor[unit=0.5,dipolestyle=zigzag,variable=true,labeloffset=-0.8](N5)(N6){RF Atten}
  \pnode(11,8){U0}

  \pnode(14,8){N7}
  \amplifier[fillstyle=solid,fillcolor=NavyBlue](N6)(N7){}
  \pnode(16,8){N8}
  \pnode(15,7){D7}
  \mixer(N7)(N8)(D7){Mixer}{fillstyle=solid,fillcolor=BurntOrange}
  \pnode(18,8){N9}
  \filter(N8)(N9){}
  \pnode(20,8){N10}
  \amplifier[fillstyle=solid,fillcolor=NavyBlue,labeloffset=-0.8](N9)(N10){L-band Amp}
  \pnode(22,8){N11}
  \pnode(21,7){X1}
  \mixer(N10)(N11)(X1){Mixer}{fillstyle=solid,fillcolor=BurntOrange}
  \pnode(24,8){N12}
  \filter(N11)(N12){}
  \pnode(26,8){N13}
  \amplifier[fillstyle=solid,fillcolor=RubineRed,labeloffset=-0.8](N12)(N13){Ku-band Amp}

  \pnode(18,10){U2}
  \pnode(20,10){U3}
  \detector[fillstyle=solid,fillcolor=NavyBlue,dipoleinput=right](U2)(U3){Det L-Band}
  \pnode(11,10){U1}
  \ncline{U2}{U1}
  \ncline{U1}{U0}
  \pnode(24,10){U4}
  \pnode(26,10){U5}
  \detector[fillstyle=solid,fillcolor=RubineRed,dipoleinput=right](U4)(U5){Det Ku-Band}
  \ncline{U4}{U3}
  \ncline{U5}{N13}

  \pnode(15,5){D8}
  \amplifier[fillstyle=solid,fillcolor=CornflowerBlue,labeloffset=-1.5](D8)(D7){L-Band Buffers}
  \pnode(13,5){D9}
  \amplifier[fillstyle=solid,fillcolor=CornflowerBlue](D9)(D8){}
  \pnode(13,4){D10}
  \ncline{D9}{D10}
  \pnode(11,4){D11}
  \vco[fillstyle=solid,fillcolor=Orange](D11)(D10){VCO}
  \rput(10,4){\rnode{D12}{\psframebox{\textbf{PLL}}}}
  \ncline{D11}{D12}
  \pnode(10,6){D13}
  \ncline{D12}{D13}
  \pnode(11,6){D14}
  \ncline{D13}{D14}
  \pnode(13,6){D15}
  \freqmult[fillstyle=solid,fillcolor=Goldenrod,dipolestyle=divider](D14)(D15){Prescaler}
  \ncline{D15}{D9}
  \psframe(9.5,3.25)(13.1,7)
  \rput[tl](9.7,6.8){\large\textbf{L-Band}}

  \pnode(10,2){R5}
  \amplifier[fillstyle=solid,fillcolor=NavyBlue](R5)(R4){L-band Amp}
  \pnode(12,2){R6}
  \resistor[unit=0.5,dipolestyle=zigzag,variable=true,labeloffset=-0.8](R5)(R6){RF Atten}
  \pnode(14,2){R7}
  \amplifier[fillstyle=solid,fillcolor=NavyBlue](R7)(R6){L-band Amp}
  \pnode(16,2){R8}
  \filter(R8)(R7){}
  \pnode(18,2){R9}
  \pnode(17,3){D17}
  \mixer[tripolestyle=top](R8)(R9)(D17){Mixer}{fillstyle=solid,fillcolor=BurntOrange}

  \pnode(15,3){D16}
  \amplifier[fillstyle=solid,fillcolor=CornflowerBlue](D8)(D16){}
  \ncline{D16}{D17}

  \pnode(20,2){R10}
  \amplifier[fillstyle=solid,fillcolor=NavyBlue](R10)(R9){L-band Amp}
  \pnode(22,2){R11}
  \amplifier[fillstyle=solid,fillcolor=NavyBlue](R11)(R10){L-band Amp}
  \pnode(24,2){R12}
  \filter(R12)(R11){}
  \pnode(26,2){R13}
  \pnode(25,1){R15}
  \mixer(R12)(R13)(R15){Mixer}{fillstyle=solid,fillcolor=BurntOrange}
  \pnode(28,2){R14}
  \amplifier[fillstyle=solid,fillcolor=Purple](R14)(R13){Ku-band LNA}

  \pnode(25,-1){R16}
  \amplifier[fillstyle=solid,fillcolor=OliveGreen,labeloffset=-1.6](R16)(R15){Ku-band Buffers}
  \pnode(24,-1){R17}
  \ncline{R16}{R17}
  \pnode(24,-2){R18}
  \ncline{R17}{R18}
  \pnode(22,-2){R19}
  \vco[fillstyle=solid,fillcolor=Red](R18)(R19){Ku-band}
  \rput(21,-2){\rnode{R20}{\psframebox{\textbf{PLL}}}}
  \ncline{R19}{R20}
  \pnode(21,0){R21}
  \ncline{R20}{R21}
  \pnode(22,0){R22}
  \ncline{R21}{R22}
  \pnode(24,0){R23}
  \freqmult[fillstyle=solid,fillcolor=Goldenrod,dipolestyle=divider](R22)(R23){Prescaler}
  \ncline{R23}{R17}
  \psframe(18,-3)(28.5,3)
  \rput[br](28,-2.75){\large\textbf{LNB}}
  \rput[bl](18,3.1){%
    \begin{tabular}{l}
    \textbf{950-1540 MHz}\\
    \textbf{900-1700 MHz}
    \end{tabular}}
  \cnode(29,2){.1}{S2}
  \ncline{R14}{S2}

  \pnode(21,5.5){X2}
  \ncline{X1}{X2}
  \pnode(24,5.5){X3}
  \amplifier[fillstyle=solid,fillcolor=OliveGreen](X3)(X2){Ku-band Buffers}
  \pnode(24,6.25){X4}
  \ncline{X3}{X4}
  \pnode(26,6.25){X5}
  \freqmult[fillstyle=solid,fillcolor=Goldenrod,dipolestyle=divider,labeloffset=-0.7](X4)(X5){Prescaler}
  \pnode(27,6.25){X6}
  \ncline{X5}{X6}
  \rput(27,4.75){\rnode{X7}{\psframebox{\textbf{PLL}}}}
  \ncline{X6}{X7}
  \pnode(26,4.75){X8}
  \ncline{X7}{X8}
  \pnode(24,4.75){X9}
  \vco[fillstyle=solid,fillcolor=Red](X8)(X9){Ku-band}
  \ncline{X9}{X3}
  \psframe(23.75,3.25)(28.5,7)
  \rput[br](28,3.5){\large\textbf{Ku-band}}

  \pnode(28.5,8){N14}
  \amplifier[fillstyle=solid,fillcolor=RubineRed](N13)(N14){}
  \cnode(29,8){.1}{S1}
  \ncline{N14}{S1}
  \psframe(26.25,7.25)(28.5,10)
  \rput[t](27.375,9.75){\large \textbf{SSPA}}
 
  \rput[lt](2,0){\large%
    \begin{tabular}{l}
        \textbf{Tx/GHz: 13.75-14.00, 14.00-14.50}\\
        \textbf{Rx/GHz: 10.95-11.70, 11.20-11.70, 11.70-12.20, 12.25-12.75}
    \end{tabular}}
\end{pspicture}}

\end{landscape}



\clearpage
\section{Flip Flops -- logical elements}

The syntax for all logical base circuits is
\begin{lstlisting}[style=syntax]
logic[<options>](<originX,originY>){Label}
\end{lstlisting}

\noindent where the options and the origin are optional. If they are missing,
then the default options, described in the next section and the default
origin $(0,0)$ is used. The origin specifies the lower left corner
of the logical circuit.

\begin{lstlisting}[style=syntax]
logic{Demo}
logic[logicType=and]{Demo}
logic(0,0){Demo}
logic[logicType=and](0,0){Demo}
\end{lstlisting}

The above four ,,different`` calls of the \verb|logic| macro give the
same output, because they are equivalent. 

\subsection{The Options}

\begin{longtable}{@{}>{\ttfamily}l l l@{}}
\textrm{\emph{name}} & \emph{type} & \emph{default}\\\hline
\endhead
logicShowNode & boolean & \emph{ false} \\
logicShowDot & boolean & \emph{ false} \\
logicNodestyle & command & \emph{ \textbackslash footnotesize} \\
logicSymbolstyle & command & \emph{ \textbackslash large} \\
logicSymbolpos & value & \emph{ 0.5} \\
logicLabelstyle & command & \emph{ \textbackslash small} \\
logicType & string & \emph{ and} \\
logicChangeLR & boolean & \emph{ false} \\
logicWidth & length & \emph{ 1.5} \\
logicHeight & length & \emph{ 2.5} \\
logicWireLength & length & \emph{ 0.5} \\
logicNInput & number & \emph{ 2} \\
logicJInput & number & \emph{ 2} \\
logicKInput & number &\emph{ 2}
\end{longtable}

\subsection{Basic Logical Circuits}
At least the basic objects require a unique label name, otherwise it is
not sure, that all nodes will work well. The label may contain any
alphanumerical character and most of all symbols. But it is save
using only combinations of letters and digits. For example:
\begin{verbatim}
And0
a0
a123
12
NOT123a
\end{verbatim}

\verb|A_1| is not a good choice, the underscore may causes some
problems.

\subsubsection{And}

\psset{subgriddiv=0,griddots=5,gridlabels=7pt}
\begin{LTXexample}[width=4.5cm](3,3)
  \begin{pspicture}(-1,0)(3,3)
  \psgrid
  \logic{AND1}
  \end{pspicture}
\end{LTXexample}

\begin{LTXexample}[width=4.5cm](3,3)
  \begin{pspicture}(-0.5,0)(3,3)
  \logic[logicChangeLR=true]{AND2}
  \end{pspicture}
\end{LTXexample}

\begin{LTXexample}[width=4.5cm](4,6)
  \begin{pspicture}(-0.5,0)(4,5)
  \psgrid
  \logic[logicShowNode=true,%
     logicWidth=2,%
     logicHeight=4,%
     logicNInput=6,%
     logicChangeLR=true](1,1){AND3}
  \end{pspicture}
\end{LTXexample}

\subsubsection{NotAnd}
\begin{LTXexample}[width=4.5cm](3,3)
  \begin{pspicture}(-0.5,0)(3,3)
  \logic[logicType=nand",%
     logicShowNode=true]{NAND1}
  \end{pspicture}
\end{LTXexample}


\begin{LTXexample}[width=4.5cm](3,3)
  \begin{pspicture}(-0.5,0)(3,3)
  \logic[logicType=nand,%
     logicChangeLR=true]{NAND2}
  \end{pspicture}
\end{LTXexample}

\begin{LTXexample}[width=4.5cm](4,6)
  \begin{pspicture}(4,5)
  \psgrid
  \logic[logicType=nand,%
     logicShowNode=true,%
     logicWidth=2,%
     logicHeight=4,%
     logicNInput=6,%
     logicChangeLR=true](1,1){NAND3}
  \end{pspicture}
\end{LTXexample}

\subsubsection{Or}
\begin{LTXexample}[width=4.5cm](3,3)
  \begin{pspicture}(-0.5,0)(3,3)
  \logic[logicType=or",%
     logicShowNode=true]{OR1}
  \end{pspicture}
\end{LTXexample}


\begin{LTXexample}[width=4.5cm](3,3)
  \begin{pspicture}(-0.5,0)(3,3)
  \logic[logicType=or,%
     logicChangeLR=true]{OR2}
  \end{pspicture}
\end{LTXexample}


\begin{LTXexample}[width=4.5cm](4,6)
  \begin{pspicture}(4,5)
  \psgrid
  \logic[logicType=or,%
     logicShowNode=true,%
     logicWidth=2,%
     logicHeight=4,%
     logicNInput=6,%
     logicChangeLR=true](1,1){OR3}
  \end{pspicture}
\end{LTXexample}

\clearpage
\subsubsection{Not Or}

\begin{LTXexample}[width=4.5cm](3,3)
  \begin{pspicture}(-0.5,0)(3,3)
  \logic[logicType=nor",%
     logicShowNode=true]{NOR1}
  \end{pspicture}
\end{LTXexample}


\begin{LTXexample}[width=4.5cm](3,3)
  \begin{pspicture}(-0.5,0)(3,3)
  \logic[logicType=nor,%
     logicChangeLR=true]{NOR2}
  \end{pspicture}
\end{LTXexample}

\begin{LTXexample}[width=4.5cm](4,6)
  \begin{pspicture}(4,5)
  \psgrid
  \logic[logicType=nor,%
     logicShowNode=true,%
     logicWidth=2,%
     logicHeight=4,%
     logicNInput=6,%
     logicChangeLR=true](1,1){NOR3}
  \end{pspicture}
\end{LTXexample}


\subsubsection{Not}

\begin{LTXexample}[width=4.5cm](3,3)
  \begin{pspicture}(-0.5,0)(3,3)
  \logic[logicType=not",%
     logicShowNode=true]{NOT1}
  \end{pspicture}
\end{LTXexample}


\begin{LTXexample}[width=4.5cm](3,3)
  \begin{pspicture}(-0.5,0)(3,3)
  \logic[logicType=not,%
     logicChangeLR=true]{NOT2}
  \end{pspicture}
\end{LTXexample}

\begin{LTXexample}[width=4.5cm](4,6)
  \begin{pspicture}(4,5)
  \psgrid
  \logic[logicType=not,%
     logicShowNode=true,%
     logicWidth=2,%
     logicHeight=4,%
     logicChangeLR=true](1,1){NOT3}
  \end{pspicture}
\end{LTXexample}

\subsubsection{Exclusive OR}

\begin{LTXexample}[width=4.5cm](3,3)
  \begin{pspicture}(-0.5,0)(3,3)
  \logic[logicType=exor",%
     logicShowNode=true]{ExOR1}
  \end{pspicture}
\end{LTXexample}


\begin{LTXexample}[width=4.5cm](3,3)
  \begin{pspicture}(-0.5,0)(3,3)
  \logic[logicType=exor,%
     logicChangeLR=true]{ExOR2}
  \end{pspicture}
\end{LTXexample}

\begin{LTXexample}[width=4.5cm](4,6)
  \begin{pspicture}(4,5)
  \psgrid
  \logic[logicType=exor,%
     logicShowNode=true,%
     logicNInput=6,%
     logicWidth=2,%
     logicHeight=4,%
     logicChangeLR=true](1,1){ExOR3}
  \end{pspicture}
\end{LTXexample}


\clearpage
\subsubsection{Exclusive NOR}

\begin{LTXexample}[width=4.5cm](3,3)
  \begin{pspicture}(-0.5,0)(3,3)
  \logic[logicType=exnor",%
     logicShowNode=true]{ExNOR1}
  \end{pspicture}
\end{LTXexample}


\begin{LTXexample}[width=4.5cm](3,3)
  \begin{pspicture}(-0.5,0)(3,3)
  \logic[logicType=exnor,%
     logicChangeLR=true]{ExNOR2}
  \end{pspicture}
\end{LTXexample}

\begin{LTXexample}[width=4.5cm](4,6)
  \begin{pspicture}(4,5)
  \psgrid
  \logic[logicType=exnor,%
     logicShowNode=true,%
     logicNInput=6,%
     logicWidth=2,%
     logicHeight=4,%
     logicChangeLR=true](1,1){ExNOR3}
  \end{pspicture}
\end{LTXexample}


\subsection{RS Flip Flop}

\begin{LTXexample}[width=4.5cm](3,4.5)
  \begin{pspicture}(-1,-1)(3,3)
  \logic[logicShowNode=true,%
     logicType=RS"]{RS1}
  \end{pspicture}
\end{LTXexample}


\begin{LTXexample}[width=4.5cm](3,4.5)
  \begin{pspicture}(-1,-1)(3,3)
  \logic[logicShowNode=true,%
     logicType=RS,%
     logicChangeLR=true]{RS2}
  \end{pspicture}
\end{LTXexample}


\subsection{D Flip Flop}

\begin{LTXexample}[width=4.5cm](3,4.5)
  \begin{pspicture}(-1,-1)(3,3)
  \logic[logicShowNode=true,%
     logicType=D"]{D1}
  \end{pspicture}
\end{LTXexample}

\begin{LTXexample}[width=4.5cm](3,4.5)
  \begin{pspicture}(-1,-1)(3,3)
  \logic[logicShowNode=true,%
     logicType=D,%
     logicChangeLR=true]{D2}
  \end{pspicture}
\end{LTXexample}


\subsection{JK Flip Flop}
\begin{LTXexample}[width=4.5cm](3,4.5)
  \begin{pspicture}(-1,-1)(3,3)
  \logic[logicShowNode=true,%
     logicType=JK",%
     logicKInput=2,%
     logicJInput=2]{JK1}
  \end{pspicture}
\end{LTXexample}

\begin{LTXexample}[width=4.5cm](3,4.5)
  \begin{pspicture}(-1,-1)(3,3)
  \logic[logicShowNode=true,%
     logicType=JK,%
     logicKInput=2, logicJInput=4,%
     logicChangeLR=true]{JK2}
  \end{pspicture}
\end{LTXexample}

\subsection{Other Options}

\begin{LTXexample}[width=3.5cm](3,3)
  \begin{pspicture}(-0.5,0)(3,2.5)
  \logic[logicShowDot=true]{A0}
  \end{pspicture}
\end{LTXexample}

\begin{LTXexample}[width=4.5cm](4,3)
  \begin{pspicture}(-1,0)(3,2.5)
  \logic[logicWireLength=1,%
     logicShowDot=true]{A1}
  \end{pspicture}
\end{LTXexample}

\bigskip
The unit of \verb|logicWireLength| is the same than the actual one for pstricks, set by
the \verb|unit| option.

\subsection{The Node Names}
Every logic circuit is defined with its name, which should be a unique one.
If we have the following NAND circuit, then \verb|pst-circ| defines the nodes
\begin{verbatim}
NAND11, NAND12, NAND13, NAND14, NAND1Q
\end{verbatim}

\noindent If there exists an inverted output, like for alle Flip Flops,
then the negated one gets the appendix \verb|neg| to the node name. For 
example:
\begin{verbatim}
NAND1Q, NAND1Qneg
\end{verbatim}

\begin{LTXexample}[width=3cm](3,3.5)
  \begin{pspicture}(-0.5,0)(2.5,3)
  \logic[logicShowNode=true,%
      logicLabelstyle=\footnotesize,%
      logicType=nand,%
      logicNInput=4]{NAND1}
  \multido{\n=1+1}{4}{%
     \pscircle*[linecolor=red](NAND1\n){2pt}%
  }
  \pscircle*[linecolor=blue](NAND1Q){2pt}
  \end{pspicture}
\end{LTXexample}

\vspace{0.5cm}
Now it is possible to draw a line from the output to the input 

\begin{verbatim}
\ncbar[angleA=0,angleB=180]{<Node A>}{<Node B>}
\end{verbatim}

It may be easier to print a grid since the drawing phase and then comment it out if
all is finished.

\bigskip
\begin{LTXexample}[width=3.5cm](3,3.5)
  \begin{pspicture}(-1,-1)(2.5,3)
  \logic[logicShowNode=true,%
      logicLabelstyle=\footnotesize,%
      logicType=nand,%
      logicWireLength=1,%
      logicNInput=4]{NAND1}
      \pnode(-0.5,0|NAND11){tempA}
      \pnode(2,0|NAND1Q){tempB}
  \end{pspicture}
  \ncbar[angleA=-90,angleB=0,arm=0.75,%
      arrows=*-*, dotsize=0.15]{tempA}{tempB}
\end{LTXexample}

\subsection{Examples}

\begin{LTXexample}[pos=t]
   \begin{pspicture}(-1,0)(5,5)
     \psgrid
     \psset{logicType=nor, logicLabelstyle=\normalsize,%
          logicWidth=1, logicHeight=1.5, dotsize=0.15}
     \logic(1.5,0){nor1}
     \logic(1.5,3){nor2}
     \psline(nor2Q)(4,0|nor2Q)
     \uput[0](4,0|nor2Q){$Q$}
     \psline(nor1Q)(4,0|nor1Q)
     \uput[0](4,0|nor1Q){$\overline{Q}$}
     \psline{*-}(3.50,0|nor2Q)(3.5,2.5)(1.5,2.5)
         (0.5,1.75)(0.5,0|nor12)(nor12)
     \psline{*-}(3.50,0|nor1Q)(3.5,2)(1.5,2)
         (0.5,2.5)(0.5,0|nor21)(nor21)
     \psline(0,0|nor11)(nor11)\uput[180](0,0|nor11){R}
     \psline(0,0|nor22)(nor22)\uput[180](0,0|nor22){S}
   \end{pspicture}
\end{LTXexample}

\bigskip
\begin{LTXexample}[pos=t]
  \begin{pspicture}(-4,0)(5,7)
     \psgrid
     \psset{logicWidth=1, logicHeight=2, dotsize=0.15}
     \logic[logicWireLength=0](-2,0){A0}
     \logic[logicWireLength=0](-2,5){A1}
     \ncbar[angleA=-180,angleB=-180,arm=0.5]{A11}{A02}
     \psline[dotsize=0.15]{-*}(-3.5,3.5)(-2.5,3.5)
     \uput[180](-3.5,3.5){$T$}
     \psline(-3.5,0.5)(A01)\uput[180](-3.5,0.5){$S$}
     \psline(-3.5,6.5)(A12)\uput[180](-3.5,6.5){$R$}
     \psset{logicType=nor, logicLabelstyle=\normalsize}
     \logic(1,0.5){nor1}
     \logic(1,4.5){nor2}
     \psline(nor2Q)(4,0|nor2Q)
     \uput[0](4,0|nor2Q){$Q$}
     \psline(nor1Q)(4,0|nor1Q)
     \uput[0](4,0|nor1Q){$\overline{Q}$}
     \psline{*-}(3,0|nor2Q)(3,4)(1,4)(0,3)(0,0|nor12)(nor12)
     \psline{*-}(3,0|nor1Q)(3,3)(1,3)(0,4)(0,0|nor21)(nor21)
     \psline(A0Q)(nor11)
     \psline(A1Q)(nor22)
  \end{pspicture}
\end{LTXexample}








\section{Adding new components}

Adding new components is not simple. As a matter of fact, because of the complex
mechanism of \cs{multidipole}, there are multiple steps. Nevertheless, it can take some time\ldots

If you want to modify the code, you need to know the following
things. For a dipole, you first
need to define the following items:

\begin{lstlisting}[language=TeX]
  \def\component_name{\@ifnextchar[{\pst@component_name}{\pst@component_name[]}}
  %
  \def\pst@component/_name[#1](#2)(#3)#4{{%
    \pst@draw@dipole{#1}{#2}{#3}{#4}\pst@draw@component_name
    }\ignorespaces}
  %
  \def\pst@multidipole@component_name{\@ifnextchar[{\pst@multidipole@component_name@}%
    {\pst@multidipole@component_name@[]}}
  %
  \def\pst@multidipole@component_name@[#1]#2{%
    \expandafter\def\csname pst@circ@tmp@\number\pst@circ@count@iii\endcsname{#2}%
    {\psset{#1}%
    \ifPst@circ@parallel\aftergroup\advance\aftergroup\pst@circ@count@i\aftergroup\m@ne\fi}%
    \pst@circ@count@ii=\pst@circ@count@i%
    \advance\pst@circ@count@ii\@ne%
    \toks0\expandafter{\pst@multidipole@output}%
    \edef\pst@multidipole@output{%
      \the\toks0%
      \pst@multidipole@def@coor%
      \noexpand\component_name[#1]%
    (! X@\the\pst@circ@count@i\space Y@\the\pst@circ@count@i)%
    (! X@\the\pst@circ@count@ii\space Y@\the\pst@circ@count@ii)%
        {\noexpand\csname pst@circ@tmp@\number\pst@circ@count@iii\endcsname}%
    }%
    \pst@multidipole@
  }
  %
  \def\pst@draw@component_name{%
    % The PSTricks code for your component
    % The center of the component is at (0,0)
    \pnode(component_left_end,0){dipole@1}
    \pnode(component_right_end,0){dipole@2}}
\end{lstlisting}

Then, you have to make some changes in the \cs{multidipole} core code\dots In the definition
of \verb+\pst@multidipole+, look for the last \verb+\ifx+ test
\begin{lstlisting}[language=TeX]
  % ...
  % Extract from \pst@multidipole
                      \else
                        \ifx\circledipole #4%
                          \let\next\pst@multidipole@circledipole
                        \else
                          \ifx\LED #4%
                            \let\next\pst@multidipole@LED
                          \else
                          % Put your modification here
                            \let\next\ignorespaces
                          \fi
                        \fi
                      \fi
  % Extract form \pst@multidipole
  % ...
\end{lstlisting}
and add (marked with \verb+%%%+)
\begin{lstlisting}[language=TeX]
  % ...
  % Extract from \pst@multidipole
                      \else
                        \ifx\circledipole #4%
                          \let\next\pst@multidipole@circledipole
                        \else
                          \ifx\LED #4%
                            \let\next\pst@multidipole@LED
                          \else
                            \ifx\component_name #4%%%
                              \let\next\pst@multidipole@component_name%%%
                            \else%%%
                              \let\next\ignorespaces
                            \fi%%%
                          \fi
                        \fi
                      \fi
  % Extract form \pst@multidipole
  % ...
\end{lstlisting}
Do the same in \verb+\pst@multidipole@+
\begin{lstlisting}[language=TeX]
  % ...
  % Extract from \pst@multidipole@
                      \else
                        \ifx\circledipole #1%
                          \let\next\pst@multidipole@circledipole
                        \else
                          \ifx\LED #1%
                            \let\next\pst@multidipole@LED
                          \else
                            \ifx\component_name #1%%%
                              \let\next\pst@multidipole@component_name%%%
                            \else%%%
                              \let\next\ignorespaces
                              \pst@multidipole@output
                            \fi%%%
                          \fi
                        \fi
                      \fi
  % Extract form \pst@multidipole@
  % ...
\end{lstlisting}
and that's it! All you have to do then is send your modified \texttt{pst-circ.tex} to 
me and  it will become part of the official release of \CircPackage.

\textbf{Important:} Pay attention to the comment character \verb+%+
at the end of lines. They are \emph{very} important in order to avoid spurious blanks.



\section{Acknowledgements}

We thank of course Manuel Luque for his original work on pst-circ and for his circuit
drawings: this wouldn't have been possible without him. As usual, Denis Girou gave us a 
precious hand with some dark tricks of \TeX{} and PSTricks. Jean-C\^ome Charpentier
wrote the outline of \cs{multidipole} (a story about riri, fifi and loulou\dots).

Thanks also to Douglas Waud, Patrick Drechsler (dashpot), Alan Ristow, and Ted Pavlic.


\nocite{*}

{\raggedright
\bibliographystyle{plain}
\bibliography{pst-circ-doc}
}

\end{document}
