\documentclass[dvips,a4paper,english]{article}
\usepackage[T1]{fontenc}
\usepackage[utf8]{inputenc}
\usepackage{mathpazo}%    use this if you have the palatino math font
%\usepackage{arev}%  use this if you do not have the palatino math font
%\usepackage[scaled=0.9]{luximono}%  use this if you do not have the palatino math font
\usepackage{url}
\usepackage{amsmath}
\usepackage{tabularx}
\usepackage{longtable}
%\usepackage{fancyhdr}
%\pagestyle{fancy}
\usepackage[varwidth]{pst-grid}
\let\pstGRIDFV\fileversion
\usepackage{babel}
\usepackage{showexpl}
\lstset{pos=t,wide=true,language=PSTricks,
    morekeywords={psGammaDist,psChiIIDist,psTDist,psFDist,psBetaDist,psPlotImpl}}
\lstdefinestyle{syntax}{backgroundcolor=\color{blue!20},numbers=none,xleftmargin=0pt,xrightmargin=0pt,
    frame=single}
%
\usepackage{xspace}
\def\PS{PostScript\xspace}
\def\CMD#1{{\ttfamily\textbackslash #1}}
\def\dt{\ensuremath{\,\mathrm{d}t}}
\def\Index#1{\index{#1}#1}
%
\def\pshlabel#1{\footnotesize#1}
\def\psvlabel#1{\footnotesize#1}
\usepackage[colorlinks,linktocpage]{hyperref}
%
\begin{document}
\title{\texttt{pst-grid}\\[1cm]
plotting a background grid\\[5mm]
		  {\small v.\pstGRIDFV}}
%\thanks{%
%		This document was written with \texttt{Kile: 1.6a (Qt: 3.1.1; KDE: 3.1.1;}
%		\protect\url{http://sourceforge.net/projects/kile/}) and the PDF output
%		  was build with VTeX/Free (\protect\url{http://www.micropress-inc.com/linux})}\\
\author{Herbert Vo\ss}%\thanks{%
%Thanks to: 
%    Martin Chicoine, 
%}}
\date{\today}

\maketitle

This package allows to place a grid in the background of any text or math. 
It has one optional argument \verb+varwidth+, which enables the package to
use the \verb+varwidth+ package and the environment of the same name instaed
of using the \verb+minipage+ environment.

The package provides only one environment:

\begin{lstlisting}[style=syntax]
\begin{dogrid}[width][<options>]
 ...
\end{dogrid}
\end{lstlisting}

Both arguments are optional, but when using the second one then the first one must be
at least empty but present. Otherwise you'll get an error, because the first optional
argument must contain a length or be empty. If it is missing or empty, then \verb+\columnwidth+
is assumed. The grid is preset with the followinmg arguments:

\begin{lstlisting}
\psset[pst-grid]{Gtrim=0 0}% no trim
\psset[pst-grid]{gridskip=0.5cm}
\psset{linecolor=black!50,linestyle=dotted,linewidth=1pt}
\end{lstlisting}

All settings can be overwritten in the usual way by using the optional argument.
Thge \verb+Gtrim+ option can be used to add or substract some more line of squares
to the grid. Negative values are possible. The \verb+gridskip+ value is added
as vertical space after placing the grid.

\begin{LTXexample}[pos=l]
\begin{dogrid}
\[ \left(
\begin{array}{c@{}c@{}c}
\begin{array}{|cc|}\hline
a_{11} & a_{12} \\
a_{21} & a_{22} \\\hline
\end{array} & 0 & 0 \\
0 & \begin{array}{|ccc|}\hline
    b_{11} & b_{12} & b_{13}\\
    b_{21} & b_{22} & b_{23}\\
    b_{31} & b_{32} & b_{33}\\\hline
    \end{array} & 0 \\
0 & 0 & \begin{array}{|cc|}\hline
        c_{11} & c_{12} \\
        c_{21} & c_{22} \\\hline
        \end{array} \\
\end{array}\right)\]
\end{dogrid}
\end{LTXexample}

\begin{LTXexample}[pos=l]
\begin{dogrid}[][Gtrim=-1 0]
\[ \left(
\begin{array}{c@{}c@{}c}
\begin{array}{|cc|}\hline
a_{11} & a_{12} \\
a_{21} & a_{22} \\\hline
\end{array} & 0 & 0 \\
0 & \begin{array}{|ccc|}\hline
    b_{11} & b_{12} & b_{13}\\
    b_{21} & b_{22} & b_{23}\\
    b_{31} & b_{32} & b_{33}\\\hline
    \end{array} & 0 \\
0 & 0 & \begin{array}{|cc|}\hline
        c_{11} & c_{12} \\
        c_{21} & c_{22} \\\hline
        \end{array} \\
\end{array}\right)\]
\end{dogrid}
\end{LTXexample}

\begin{LTXexample}[pos=l]
\begin{dogrid}[][Gtrim=-1 0,linecolor=red!70]
\[ \left(
\begin{array}{c@{}c@{}c}
\begin{array}{|cc|}\hline
a_{11} & a_{12} \\
a_{21} & a_{22} \\\hline
\end{array} & 0 & 0 \\
0 & \begin{array}{|ccc|}\hline
    b_{11} & b_{12} & b_{13}\\
    b_{21} & b_{22} & b_{23}\\
    b_{31} & b_{32} & b_{33}\\\hline
    \end{array} & 0 \\
0 & 0 & \begin{array}{|cc|}\hline
        c_{11} & c_{12} \\
        c_{21} & c_{22} \\\hline
        \end{array} \\
\end{array}\right)\]
\end{dogrid}
\end{LTXexample}


\begin{LTXexample}[pos=t]
\begin{dogrid}[][linestyle=solid,linecolor=black!20]
\[\begin{array}{rcll}
y & = & x^{2}+bx+c\\
  & = & x^{2}+2\cdot{\displaystyle\frac{b}{2}x+c}\\
  & = & \underbrace{x^{2}+2\cdot\frac{b}{2}x+\left(\frac{b}{2}\right)^{2}}-
	{\displaystyle \left(\frac{b}{2}\right)^{2}+c}\\
 &  & \qquad\left(x+{\displaystyle \frac{b}{2}}\right)^{2}\\
 & = & \left(x+{\displaystyle \frac{b}{2}}\right)^{2}-
	\left({\displaystyle \frac{b}{2}}\right)^{2}+c & 
	\left|+\left({\displaystyle \frac{b}{2}}\right)^{2}-c\right.\\
y+\left({\displaystyle \frac{b}{2}}\right)^{2}-c & = & \left(x+
	{\displaystyle \frac{b}{2}}\right)^{2} & \left|(\textrm{Scheitelpunktform})\right.\\
y-y_{S} & = & (x-x_{S})^{2}\\
S(x_{S};y_{S}) & \,\textrm{bzw.}\, & S\left(-{\displaystyle \frac{b}{2};\,
    \left({\displaystyle \frac{b}{2}}\right)^{2}-c}\right)
\end{array}\]
\end{dogrid}
\end{LTXexample}


\bigskip
\begin{LTXexample}[pos=t]
\begin{dogrid}[.75\linewidth][linecolor=lightgray,linestyle=solid,Gtrim=0 -1]
\begin{enumerate}
\item This is the first sentence
\item This is the second sentence
\item This is the third one
\item And this the last one
\end{enumerate}
\end{dogrid}
\end{LTXexample}


\nocite{*}
\bgroup
\raggedright
\bibliographystyle{plain}
\bibliography{\jobname}
\egroup


\end{document}
