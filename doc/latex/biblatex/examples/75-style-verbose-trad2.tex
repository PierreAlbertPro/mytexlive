%
% This file presents the 'verbose-trad2' style
%
\documentclass[a4paper]{article}
\usepackage[T1]{fontenc}
\usepackage[american]{babel}
\usepackage{csquotes}
\usepackage[style=verbose-trad2]{biblatex}
\addbibresource{biblatex-examples.bib}
% Some generic settings:
\newcommand{\cmd}[1]{\texttt{\textbackslash #1}}
\newenvironment*{pseudofootnotes}
  {\list\labelenumi{%
     \def\makelabel##1{\hss\llap{##1}}%
     \def\labelenumi{\theenumi}%
     \usecounter{enumi}%
     \setlength{\leftmargin}{0pt}%
     \setlength{\labelsep}{0.75em}%
     \setlength{\itemsep}{0pt}%
     \setlength{\parsep}{0pt}}%
   \citereset
   \footnotesize
   \def\footcite##1{\item\Cite{##1}.}%
   \def\footnote##1{\item##1}}
  {\endlist}
\newenvironment*{pseudofootnotes*}
  {\pseudofootnotes
   \def\footnote##1{\item##1\mancite}}
  {\endpseudofootnotes}
\begin{document}

\section*{The \texttt{verbose-trad2} style}

This is another traditional style which uses scholarly abbreviations
like \emph{ibidem} and \emph{idem}. Despite its name, the `logic' of
this style is more closely related to styles like
\texttt{verbose-ibid} and \texttt{verbose-inote} than to the rather
special citation scheme implemented in the \texttt{verbose-trad1}
style.

\subsection*{Additional package options}

\subsubsection*{The \texttt{ibidpage} option}

The scholarly abbreviation \emph{ibidem} is sometimes taken to mean
both `same author~+ same title' and `same author~+ same title~+ same
page' in traditional citation schemes. By default, this is not the
case with this style because it may lead to ambiguous citations. If
you you prefer the wider interpretation of \emph{ibidem}, set the
package option \texttt{ibidpage=true} or simply \texttt{ibidpage} in
the preamble. The default setting is \texttt{ibidpage=false}.

\subsubsection*{The \texttt{strict} option}

A case related to the definition of \emph{ibidem} is the scope of
the \emph{ibidem} and \emph{idem} replacements. By default, this
style will only use such abbreviations if the respective citations
are given in the same footnote or in consecutive footnotes. The
point of this restriction is also to avoid potentially ambiguous
citations. Here's an example:

\begin{verbatim}
...\footcite{aristotle:anima}
...\footcite{aristotle:anima}
...\footnote{Averroes touches upon this issue in his \emph{Epistle
             on the Possibility of Conjunction}.}
...\footcite{aristotle:anima}
\end{verbatim}
%
This could be rendered as follows:

\begin{pseudofootnotes}
\footcite{aristotle:anima}
\footcite{aristotle:anima}
\footnote{Averroes touches upon this issue in his \emph{Epistle on
the Possibility of Conjunction}.}
\footcite{aristotle:anima}
\end{pseudofootnotes}
%
What does the \emph{ibidem} in the last footnote refer to? The last
formal citation, as given in the first and the second footnote
(Aristotle), or the informal reference in the third one (Averroes)?
Too avoid such citations, this style will only use \emph{ibidem} and
\emph{idem} replacements if the respective citations are given in
the same footnote or in consecutive footnotes:

\begin{pseudofootnotes*}
\footcite{aristotle:anima}
\footcite{aristotle:anima}
\footnote{Averroes touches upon this issue in his \emph{Epistle on
the Possibility of Conjunction}.}
\footcite{aristotle:anima}
\end{pseudofootnotes*}
%
Depending on your writing and citing habits, however, you may prefer
the less strict \emph{ibidem} and \emph{idem} handling. You can
force that by setting the package option \texttt{strict=false} in
the preamble. It is still possible to mark a manually inserted
discursive citation with \cmd{mancite} when required:

\begin{verbatim}
...\footcite{aristotle:anima}
...\footnote{\mancite Averroes touches upon this issue in his
             \emph{Epistle on the Possibility of Conjunction}.}
...\footcite{aristotle:anima}
\end{verbatim}
%
This will suppress the \emph{ibidem} in the last footnote.

\subsubsection*{The \texttt{citepages} option}

Use this option to fine-tune the formatting of the \texttt{pages}
and \texttt{pagetotal} fields in verbose citations. When an entry
with a \texttt{pages} field is cited for the first time and the
\texttt{postnote} is a page number or a page range, the citation
will end with two page specifications:

\begin{quote}
Author. \enquote{Title.} In: \emph{Book,} pp.\,100--150, p.\,125.
\end{quote}
%
In this example, \enquote{125} is the \texttt{postnote} and
\enquote{100--150} is the \texttt{pages} field (there are similar
issues with the \texttt{pagetotal} field). This may be confusing to
the reader. The \texttt{citepages} option controls how to deal with
these fields in this case. The option works as follows, given these
citations as an example:

\begin{verbatim}
\cite{key}
\cite[a note]{key}
\cite[125]{key}
\end{verbatim}
%
\texttt{citepages=permit} allows duplicates, i.e., the style will
print both the \texttt{pages}\slash \texttt{pagetotal} and the
\texttt{postnote}. This is the default setting:

\begin{quote}
Author. \enquote{Title.} In: \emph{Book,} pp.\,100--150.

Author. \enquote{Title.} In: \emph{Book,} pp.\,100--150, a note.

Author. \enquote{Title.} In: \emph{Book,} pp.\,100--150, p.\,125.
\end{quote}
%
\texttt{citepages=suppress} unconditionally suppresses the
\texttt{pages}\slash \texttt{pagetotal} fields in citations,
regardless of the \texttt{postnote}:

\begin{quote}
Author. \enquote{Title.} In: \emph{Book.}

Author. \enquote{Title.} In: \emph{Book,} a note.

Author. \enquote{Title.} In: \emph{Book,} p.\,125.
\end{quote}
%
\texttt{citepages=omit} suppresses the \texttt{pages}\slash
\texttt{pagetotal} in the third case only. They are still printed if
there is no \texttt{postnote} or if the \texttt{postnote} is not a
number or range:

\begin{quote}
Author. \enquote{Title.} In: \emph{Book,} pp.\,100--150.

Author. \enquote{Title.} In: \emph{Book,} pp.\,100--150, a note.

Author. \enquote{Title.} In: \emph{Book,} p.\,125.
\end{quote}
%
\texttt{citepages=separate} separates the \texttt{pages}\slash
\texttt{pagetotal} from the \texttt{postnote} in the third case:

\begin{quote}
Author. \enquote{Title.} In: \emph{Book,} pp.\,100--150.

Author. \enquote{Title.} In: \emph{Book,} pp.\,100--150, a note.

Author. \enquote{Title.} In: \emph{Book,} pp.\,100--150, esp. p.\,125.
\end{quote}
%
The string \enquote{especially} in the third case is the bibliography
string \texttt{thiscite}, which may be redefined.

\subsubsection*{The \texttt{dashed} option}

By default, this style replaces recurrent authors/editors in the
bibliography by a dash so that items by the same author or editor
are visually grouped. This feature is controlled by the package
option \texttt{dashed}. Setting \texttt{dashed=false} in the
preamble will disable this feature. The default setting is
\texttt{dashed=true}.

\subsection*{Hints}

If you want terms such as \emph{ibidem} to be printed in italics,
redefine \cmd{mkibid} as follows:

\begin{verbatim}
\renewcommand*{\mkibid}{\emph}
\end{verbatim}

\clearpage

\subsection*{\cmd{footcite} examples}

% The initial citation of an entry includes all the data.
This is just filler text.\footcite{aristotle:anima}
This is just filler text.\footcite{averroes/bland}
% Subsequent citations use a shorter format.
This is just filler text.\footcite[26]{aristotle:anima}
This is just filler text.\footcite[59--61]{averroes/bland}
This is just filler text.\footcite{aristotle:physics}
% Repeated authors are replaced by the abbreviation 'idem':
This is just filler text.\footcite{aristotle:anima}
This is just filler text.\footcite[55]{aristotle:physics}
% Immediately repeated citations are replaced by the
% abbreviation 'ibidem'.
This is just filler text.\footcite[25]{aristotle:physics}
% ibidpage=true would suppress the following page number:
This is just filler text.\footcite[25]{aristotle:physics}

\clearpage

% If the 'shorthand' field is defined, the shorthand is introduced
% on the first citation.
This is just filler text.\footcite{kant:kpv}
This is just filler text.\footcite{kant:ku}
% All subsequent citations will then use the shorthand instead of
% the short author-title format.
This is just filler text.\footcite[24]{kant:kpv}
This is just filler text.\footcite[59--63]{kant:ku}

\clearpage

\subsection*{\cmd{autocite} examples}

% The \autocite command works like \footcite. Note that
% the period is moved and placed before the footnote.

This is just filler text \autocite{aristotle:rhetoric}.
This is just filler text \autocite{averroes/bland}.
This is just filler text \autocite{aristotle:rhetoric}.
This is just filler text \autocite{aristotle:anima}.
This is just filler text \autocite{aristotle:physics}.
This is just filler text \autocite{aristotle:physics}.

\clearpage

% Since all bibliographic data is provided on the first citation,
% this style may be used without a list of references and
% shorthands. Of course these lists may still be printed if desired.

\printshorthands
\printbibliography

\end{document}
