% \iffalse meta-comment
%
% Copyright (C) 2012 by Jean SIMARD
%
% This file may be distributed and/or modified under the conditions of the 
%LaTeX Project Public License, either version 1.2 of this license or (at your 
%option) any later version.  The latest version of this license is in:
%
% http://www.latex-project.org/lppl.txt
%
% and version 1.2 or later is part of all distributions of LaTeX version 
%1999/12/01 or later.
%
% \fi

% \iffalse
%<*driver>
\ProvidesFile{mycv.dtx}
%</driver>
%<class>\NeedsTeXFormat{LaTeX2e}[1999/12/01]
%<class>\ProvidesClass{curve}
%<*class>
	[2012/03/21 v1.0 Curriculum Vitae class]
%</class>
%<*driver>
\documentclass[english]{ltxdoc}
\usepackage{holtxdoc}[2008/08/11]
\usepackage{my}
\usepackage{hyperref}
\EnableCrossrefs
\CodelineIndex 
\RecordChanges
\begin{document}
\DocInput{mycv.dtx}
\end{document}
%</driver>
% \fi
%
% \CheckSum{0}
%
%% \CharacterTable
%%  {Upper-case  \A\B\C\D\E\F\G\H\I\J\K\L\M\N\O\P\Q\R\S\T\U\V\W\X\Y\Z
%%   Lower-case  \a\b\c\d\e\f\g\h\i\j\k\l\m\n\o\p\q\r\s\t\u\v\w\x\y\z
%%   Digits    \0\1\2\3\4\5\6\7\8\9
%%   Exclamation   \!   Double quote  \"   Hash (number) \#
%%   Dollar    \$   Percent     \%   Ampersand   \&
%%   Acute accent  \'   Left paren  \(   Right paren   \)
%%   Asterisk    \*   Plus      \+   Comma     \,
%%   Minus     \-   Point     \.   Solidus     \/
%%   Colon     \:   Semicolon   \;   Less than   \<
%%   Equals    \=   Greater than  \>   Question mark \?
%%   Commercial at \@   Left bracket  \[   Backslash   \\
%%   Right bracket \]   Circumflex  \^   Underscore  \_
%%   Grave accent  \`   Left brace  \{   Vertical bar  \|
%%   Right brace   \}   Tilde     \~}
%
% \changes{v1.0}{2012/03/21}{Initial version}
%
% \GetFileInfo{mycv.dtx}
%
% \DoNotIndex{\begin,\end,\\}
% \DoNotIndex{\RequirePackage}
% \DoNotIndex{\ClassError,\ClassWarning,\ClassInfo}
% \DoNotIndex{\PackageError,\PackageWarning,\PackageInfo}
% \DoNotIndex{\DeclareOption,\ExecuteOptions,\CurrentOption,\ProcessOptions}
% \DoNotIndex{\DeclareOptionX,\ProcessOptionsX}
%
% \title{The \xclass{mycv} class\\create a smart \myemph{curriculum vitae} 
%\thanks{This document corresponds to the \xclass{mycv}~\fileversion, 
%dated~\filedate.}}
% \author{Jean 
%\myname{Simard}\\\href{mailto:juste.lapin@gmail.com}{\xemail{juste.lapin@gmail.com}}}
%
% \maketitle
%
% \begin{abstract}
%   The \xclass{mycv} is providing tools to create a \myemph{curriculum vitae} 
%on a simple page as well as multi-page \myemph{curriculum vitae} with 
%bibliography.
% \end{abstract}
% \tableofcontents
%
% \section{Introduction}
% The \xclass{mycv} class provides different tools to write a 
%\myemph{curriculum vitae}, based on the \xclass{curve} class: 
%\myemph{curriculum vitae} on one page, on multiple page, with a title page, 
%with a table of contents, with a bibliography\dots{}
%
% \section{Usage}
% \subsection{The header}
% \DescribeMacro{\address}
% \DescribeMacro{\postal}
% \DescribeMacro{\city}
% \DescribeMacro{\country}
% \DescribeMacro{\phone}
% \DescribeMacro{\mobile}
% \DescribeMacro{\fax}
% \DescribeMacro{\email}
% \DescribeMacro{\birth}
% Few commands can be defined to describe the identity and contact details of 
%the owner of the \myemph{curriculum vitae}.
% \begin{description}
%   \item[\cs{address}] for the mail address (ex. |\address{13 rue Pascal}|)
%   \item[\cs{postal}] for the postal code (ex. |\postal{8000-8099}|)
%   \item[\cs{city}] for the city (ex. |\city{Zürich}|)
%   \item[\cs{country}] for the country (ex. |\country{Germany}|)
%   \item[\cs{phone}] for the phone number (ex. |\phone{01 23 45 67 89}|)
%   \item[\cs{mobile}] for the mobile number (ex. |\mobile{07 23 45 67 89}|)
%   \item[\cs{fax}] for the fax number (ex. |\fax{01 23 45 67 90}|)
%   \item[\cs{email}] for the email (ex. |\email{albert.einstein@emc2.org}|)
%   \item[\cs{birth}] for the date of birth in format \texttt{dd/mm/aaaa} (ex.  
%|\birth{14/03/1879}|)
% \end{description}
%
% \subsection{The table of contents}
% \DescribeMacro{\tableofcontents}
% The \cs{tableofcontents} has been defined and has the same effect as in the 
%\xclass{article}, \xclass{report} or \xclass{book} classes.
%
% \subsection{Rubrics and entries}
% \begin{declcs}{begin}|{myrubric}|
% \marg{title}\\
% |...|\\
% |\end{myrubric}|
% \end{declcs}
% \DescribeEnv{myrubric}
% The main environment of the \myemph{curriculum vitae} is the |myrubric| 
%environment. It creates a new rubric which will contains multiple entries (see 
%below). A rubric must contain a \meta{title}.
%
% \begin{declcs}{mymakerubric}
% \marg{keyword}
% \marg{title}
% \end{declcs}
% \DescribeMacro{\mymakerubric}
% The \cs{mymakerubric} defines a new rubric in the \myemph{curriculum vitae}.  
%Depending on the \meta{keyword} argument, it includes the file 
%\texttt{keyword.tex} or, if there is a flavor, \texttt{keyword.<flavor>.tex}.  
%The \meta{title} argument is the title of the rubric.
%
% \begin{declcs}{begin}|{myentry}|
% \oarg{duration}
% \marg{date}\\
% |  |\meta{content}\\
% |\end{myentry}|
% \end{declcs}
% \DescribeEnv{myentry}
% In a rubric, there will be multiple entries. Each entry is a experience or a 
%formation described in the \myemph{curriculum vitae}. The \meta{date} should 
%be the year of the formation; the \meta{duration} could be added to inform of 
%the duration of the formation or of the job. The \meta{content} should be used 
%to describe the formation or the job.
%
% \begin{declcs}{begin}|{myheadentry}|
% \marg{date}\\
% |  |\meta{title}\\
% |\end{myheadentry}|
% \end{declcs}
% \DescribeEnv{myheadentry}
% Sometime, a rubric may contain one big entry (with a lot of information). The 
%|myheadentry| may be used to inform of the beginning of a long entry and 
%multiple entries (|myentry|) can be used to describe the formation or the job 
%with two or more points.
%
% \StopEventually{%
%   \PrintChanges
%   \PrintIndex
% }
%
% \section{Implementation}
% \subsection{Initialization}
% Load \xpackage{my} package to load default configuration.
%    \begin{macrocode}
\RequirePackageWithOptions{my}[2011/12/08]
%    \end{macrocode}
% The different following options are transfered to the \xclass{curve} class.
%    \begin{macrocode}
\DeclareOptionX{a4paper}{\PassOptionsToClass{a4paper}{curve}}
\DeclareOptionX{a5paper}{\PassOptionsToClass{a5paper}{curve}}
\DeclareOptionX{b4paper}{\PassOptionsToClass{b4paper}{curve}}
\DeclareOptionX{letterpaper}{\PassOptionsToClass{letterpaper}{curve}}
\DeclareOptionX{legalpaper}{\PassOptionsToClass{legalpaper}{curve}}
\DeclareOptionX{executivepaper}{\PassOptionsToClass{executivepaper}{curve}}
\DeclareOptionX{10pt}{\PassOptionsToClass{10pt}{curve}}
\DeclareOptionX{11pt}{\PassOptionsToClass{11pt}{curve}}
\DeclareOptionX{12pt}{\PassOptionsToClass{12pt}{curve}}
\DeclareOptionX{final}{\PassOptionsToClass{final}{curve}}
\DeclareOptionX{draft}{\PassOptionsToClass{draft}{curve}}
\DeclareOptionX{english}{\PassOptionsToClass{english}{curve}}
\DeclareOptionX{french}{\PassOptionsToClass{french}{curve}}
\DeclareOptionX{francais}{\PassOptionsToClass{francais}{curve}}
\DeclareOptionX{spanish}{\PassOptionsToClass{spanish}{curve}}
\DeclareOptionX{portuguese}{\PassOptionsToClass{portuguese}{curve}}
\DeclareOptionX{portuges}{\PassOptionsToClass{portuges}{curve}}
\DeclareOptionX{brazilian}{\PassOptionsToClass{brazilian}{curve}}
\DeclareOptionX{brazil}{\PassOptionsToClass{brazil}{curve}}
\DeclareOptionX{italian}{\PassOptionsToClass{italian}{curve}}
\DeclareOptionX{german}{\PassOptionsToClass{german}{curve}}
\DeclareOptionX{ngerman}{\PassOptionsToClass{ngerman}{curve}}
\DeclareOptionX{dutch}{\PassOptionsToClass{dutch}{curve}}
\DeclareOptionX{danish}{\PassOptionsToClass{danish}{curve}}
\DeclareOptionX{swedish}{\PassOptionsToClass{swedish}{curve}}
\DeclareOptionX*{%
  \ClassWarning{mycv}{Unknown option `\CurrentOption'}%
}
\ProcessOptionsX\relax
%    \end{macrocode}
%
% Undefined the macro \cs{@makefntext} before loading the \xclass{curve} class 
%to avoid problem of command already defined.
%    \begin{macrocode}
\undef\@makefntext
%    \end{macrocode}
%
% The \xclass{curve} class is now loaded with no default option.
%    \begin{macrocode}
\LoadClass{curve}[2008/02/25]
%    \end{macrocode}
%
% Some useful packages are loaded here.
% \begin{itemize}
%   \item\xpackage{myheadfoot} is used for the header and footer
%   \item\xpackage{mycolor} for a bundle of personal colors
%   \item\xpackage{mydate} for a package about dates
%   \item\xpackage{environ} to define specific environments
%   \item\xpackage{hyperref} for the links in a document
% \end{itemize}
%    \begin{macrocode}
\RequirePackage{myheadfoot}
\RequirePackage{mycolor}
\RequirePackage{mydate}
\RequirePackage{environ}
\RequirePackage{makecell}
\RequirePackage{hyperref}
%    \end{macrocode}
%
% \subsection{The header}
% Define new fields for the header of the \myemph{curriculum vitae}:
% \begin{macro}{\address}
%   The mail address
%    \begin{macrocode}
\newcommand{\address}[1]{\def\@address{#1}}
%    \end{macrocode}
% \end{macro}
% \begin{macro}{\postal}
%   The postal code
%    \begin{macrocode}
\newcommand{\postal}[1]{\def\@postal{#1}}
%    \end{macrocode}
% \end{macro}
% \begin{macro}{\city}
%   The city
%    \begin{macrocode}
\newcommand{\city}[1]{\def\@city{#1}}
%    \end{macrocode}
% \end{macro}
% \begin{macro}{\country}
%   The country
%    \begin{macrocode}
\newcommand{\country}[1]{\def\@country{#1}}
%    \end{macrocode}
% \end{macro}
% \begin{macro}{\phone}
%   The phone number
%    \begin{macrocode}
\newcommand{\phone}[1]{\def\@phone{#1}}
%    \end{macrocode}
% \end{macro}
% \begin{macro}{\mobile}
%   The mobile number
%    \begin{macrocode}
\newcommand{\mobile}[1]{\def\@mobile{#1}}
%    \end{macrocode}
% \end{macro}
% \begin{macro}{\fax}
%   The fax number
%    \begin{macrocode}
\newcommand{\fax}[1]{\def\@fax{#1}}
%    \end{macrocode}
% \end{macro}
% \begin{macro}{\email}
%   The email address
%    \begin{macrocode}
\newcommand{\email}[1]{\def\@email{\href{mailto:#1}{#1}}}
%    \end{macrocode}
% \end{macro}
% \begin{macro}{\birth}
%   The birth date
%    \begin{macrocode}
\newcommand{\birth}[1]{\def\@birth{\mydate[datestyle=short]{#1}}}
%    \end{macrocode}
% \end{macro}
%
% With the previous defined fields, define the \cs{leftheader} and 
%\cs{rightheader} which are used by \xclass{curve}.
%    \begin{macrocode}
\headerscale{1.0}
\photoscale{0.2}
\leftheader{%
  \ifdef{\@author}{\@author\newline}{}%
  \ifdef{\@address}{\@address\newline}{}%
  \ifdef{\@city}{\ifdef{\@postal}{\@postal}{}~\@city\newline}{}%
  \ifdef{\@country}{\@country\newline}{}%
  \ifdef{\@birth}{\@birth\newline}{}%
  \\[-0.5ex]%
  \begin{tabular}{@{}l@{ -- }l@{}}%
    \ifdef{\@phone}{Tel & \@phone\tabularnewline}{}%
    \ifdef{\@mobile}{Mob & \@mobile\tabularnewline}{}%
    \ifdef{\@fax}{Mob & \@fax\tabularnewline}{}%
    \ifdef{\@email}{Email & \@email\tabularnewline}{}%
  \end{tabular}\newline%
}
\rightheader{}
\newrobustcmd\makeheader{\makeheaders}
%    \end{macrocode}
%
% \subsection{The title page}
% Define a new header/footer style for the title page (without header/footer).
%    \begin{macrocode}
\fancypagestyle{firstpage}{%
  \fancyhead{}%
  \fancyfoot{}%
  \renewcommand{\headrulewidth}{0.0pt}%
  \renewcommand{\footrulewidth}{0.0pt}%
  \fancyhead[R]{%
    \ifdef{\@date}{\footnotesize\textcolor{black!70}{\@date}}{}%
  }%
}%
%    \end{macrocode}
% \begin{macro}{\maketitle}
%   The \cs{maketitle} only insert the title and the author on a single blank 
%page. If \cs{date} has been defined, the header/footer style put the date 
%upper-left on the page.
%    \begin{macrocode}
\def\maketitle{%
  \begin{center}%
    \null%
    \thispagestyle{firstpage}%
    \vfill%
    {\LARGE \@title}\par
    \vfill%
    {\Large \@author}\par
    \vfill%
    \null%
  \end{center}%
  \clearpage%
  \pagestyle{plain}%
}
%    \end{macrocode}
% \end{macro}
%
% \subsection{The table of contents}
% Define the \cs{tableofcontents} using the original definition of \LaTeXe.
%    \begin{macrocode}
\setcounter{tocdepth}{1}
\def\@starttoc#1{%
  \begingroup
    \makeatletter
    \@input{\jobname.#1}%
    \if@filesw
      \expandafter\newwrite\csname tf@#1\endcsname
      \immediate\openout \csname tf@#1\endcsname \jobname.#1\relax
    \fi
    \@nobreakfalse
  \endgroup
}
\def\addcontentsline#1#2#3{%
  \addtocontents{#1}{\protect\contentsline{#2}{#3}{\thepage}}%
}
\long\def\addtocontents#1#2{%
  \protected@write\@auxout
  {\let\label\@gobble \let\index\@gobble \let\glossary\@gobble}%
  {\string\@writefile{#1}{#2}}%
}
\def\contentsline#1{\csname l@#1\endcsname}
\newcommand\@pnumwidth{1.55em}
\newcommand\@tocrmarg{2.55em} 
\newcommand\@dotsep{4.5}
\newcommand{\l@toc}{\@dottedtocline{1}{0.5em}{0.5em}}
\newcommand\tableofcontents{%
  \clearpage
  \textbf{\LARGE Sommaire}%
  \vspace{3ex plus 0.5ex minus 0.5ex}
  \addtocontents{toc}{\large\vspace{1ex}}%
  \@starttoc{toc}%
  \clearpage
}
%    \end{macrocode}
%
% \subsection{Rubric and entries}
% \begin{macro}{myrubric}
%   The \texttt{myrubric} environment is based on the \texttt{rubric} 
%environment of the \xclass{curve} class.
%    \begin{macrocode}
\newenvironment{myrubric}[1]{%
  \begin{rubric}{#1}%
}{%
  \end{rubric}%
}
%    \end{macrocode}
% \end{macro}
%
% \begin{macro}{\mymakerubric}
%   Use \cs{makerubric} macro from \xclass{curve} class and add the title of 
%the rubric in the \texttt{.toc} file for the table of contents.
%    \begin{macrocode}
\newcommand{\mymakerubric}[2]{%
  \addcontentsline{toc}{toc}{\large{#2}}%
  \addtocontents{toc}{\large\vspace{1ex}}%
  \makerubric{#1}%
}
%    \end{macrocode}
% \end{macro}
%
% \begin{macro}{myentry}
%   The \texttt{myentry} environment define an new entry based on the 
%\xpackage{environ} package. It is a tricky new environment because he must 
%accept ampersand \texttt{\&} in it.
%    \begin{macrocode}
\NewEnviron{myentry}[2][]{%
  \gdef\@myheadentry{%
    \footnotesize%
    \ifstrempty{#1}%
	  {\makecell[tl]{#2}}%
      {\makecell[tl]{#2\\#1}}%
  }%
  \expandafter\gdef%
  \expandafter\@myentry%
  \expandafter{%
    \expandafter\@myheadentry%
	\expandafter&%
	\expandafter\@prefix%
	\expandafter&\BODY\tabularnewline%
    \vspace{0.5ex plus 0.25ex minus 0.25ex}%
  }%
  \aftergroup\@myentry%
}
%    \end{macrocode}
% \end{macro}
% \begin{macro}{myheadentry}
%   The \texttt{myheadentry} environment is like the \texttt{myentry} 
%environment except it apply bold fonts to every text.
%    \begin{macrocode}
\NewEnviron{myheadentry}[1]{%
  \gdef\@myheadentry{\sffamily\bfseries #1}%
  \expandafter\gdef%
  \expandafter\@myentry%
  \expandafter{%
    \expandafter\@myheadentry%
	\expandafter&%
	\expandafter\@prefix%
	\expandafter&%
	\expandafter\bfseries\BODY\tabularnewline%
    \vspace{0.5ex plus 0.25ex minus 0.25ex}%
  }%
  \aftergroup\@myentry%
}
%    \end{macrocode}
% \end{macro}
%
% The \cs{date} define the \cs{@date} macro. If the user does not want a date, 
%he will not define \cs{date} but the default configuration of \LaTeXe is to 
%define \cs{@date} with |\date{\today}|. The following line forces to undefine 
%\cs{@date}.
%    \begin{macrocode}
\undef{\@date}
%    \end{macrocode}
%
% \Finale
\endinput
