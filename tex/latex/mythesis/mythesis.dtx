% \iffalse meta-comment
%
% Copyright (C) 2012 by Jean SIMARD
%
% This file may be distributed and/or modified under the conditions of the 
%LaTeX Project Public License, either version 1.2 of this license or (at your 
%option) any later version.  The latest version of this license is in:
%
% http://www.latex-project.org/lppl.txt
%
% and version 1.2 or later is part of all distributions of LaTeX version 
%1999/12/01 or later.
%
% \fi

% \iffalse
%<*driver>
\ProvidesFile{mythesis.dtx}
\documentclass[english]{ltxdoc}
\usepackage{holtxdoc}[2008/08/11]
\usepackage{my}
\usepackage{hyperref}
\EnableCrossrefs
\CodelineIndex 
\RecordChanges
\begin{document}
\DocInput{mythesis.dtx}
\end{document}
%</driver>
% \fi
%
% \CheckSum{472}
%
%% \CharacterTable
%%  {Upper-case  \A\B\C\D\E\F\G\H\I\J\K\L\M\N\O\P\Q\R\S\T\U\V\W\X\Y\Z
%%   Lower-case  \a\b\c\d\e\f\g\h\i\j\k\l\m\n\o\p\q\r\s\t\u\v\w\x\y\z
%%   Digits    \0\1\2\3\4\5\6\7\8\9
%%   Exclamation   \!   Double quote  \"   Hash (number) \#
%%   Dollar    \$   Percent     \%   Ampersand   \&
%%   Acute accent  \'   Left paren  \(   Right paren   \)
%%   Asterisk    \*   Plus      \+   Comma     \,
%%   Minus     \-   Point     \.   Solidus     \/
%%   Colon     \:   Semicolon   \;   Less than   \<
%%   Equals    \=   Greater than  \>   Question mark \?
%%   Commercial at \@   Left bracket  \[   Backslash   \\
%%   Right bracket \]   Circumflex  \^   Underscore  \_
%%   Grave accent  \`   Left brace  \{   Vertical bar  \|
%%   Right brace   \}   Tilde     \~}
%
% \changes{v1.0}{2012/04/12}{Initial version}
%
% \GetFileInfo{mythesis.dtx}
%
% \DoNotIndex{\begin,\end,\\}
% \DoNotIndex{\RequirePackage}
% \DoNotIndex{\ClassError,\ClassWarning,\ClassInfo}
% \DoNotIndex{\PackageError,\PackageWarning,\PackageInfo}
% \DoNotIndex{\DeclareOption,\ExecuteOptions,\CurrentOption,\ProcessOptions}
% \DoNotIndex{\DeclareOptionX,\ProcessOptionsX}
%
% \title{The \xclass{mythesis} class\\Writing a memoir is easy!
%\thanks{This document corresponds to the \xclass{mythesis}~\fileversion, 
%dated~\filedate.}}
% \author{Jean 
%\myname{Simard}\\\href{mailto:juste.lapin@gmail.com}{\xemail{juste.lapin@gmail.com}}}
%
% \maketitle
%
% \begin{abstract}
%   The \xclass{mythesis} provide default tools and presentation to create a 
%thesis memoir.
% \end{abstract}
% \tableofcontents
%
% \section{Introduction}
%
% \section{Usage}
%
% \StopEventually{%
%   \PrintChanges
%   \PrintIndex
% }
%
% \section{Implementation}
% \subsection{Initialization}
%\iffalse
%    \begin{macrocode}
%<*mythesis.cls>
%    \end{macrocode}
%\fi
%
% Finally, load the \xclass{book} class.
%    \begin{macrocode}
\NeedsTeXFormat{LaTeX2e}[1999/12/01]
\ProvidesClass{mythesis}[2012/04/18 v1.0 Thesis/memoir class]
%    \end{macrocode}
%
% Load \xpackage{my} package to load default configuration.
%    \begin{macrocode}
\RequirePackageWithOptions{my}[2011/12/08]
%    \end{macrocode}
%
% Define the default options to send to the \xclass{book} class.
%    \begin{macrocode}
\PassOptionsToClass{12pt,a4paper,fleqn,twoside}{book}
%    \end{macrocode}
%
% Redefine all options that can be transmit to the \xclass{book} class.
%    \begin{macrocode}
\DeclareOptionX{10pt}{\PassOptionsToClass{10pt}{book}}
\DeclareOptionX{11pt}{\PassOptionsToClass{11pt}{book}}
\DeclareOptionX{12pt}{\PassOptionsToClass{12pt}{book}}
\DeclareOptionX{a4paper}{\PassOptionsToClass{a4paper}{book}}
\DeclareOptionX{a5paper}{\PassOptionsToClass{a5paper}{book}}
\DeclareOptionX{b5paper}{\PassOptionsToClass{b5paper}{book}}
\DeclareOptionX{letterpaper}{\PassOptionsToClass{letterpaper}{book}}
\DeclareOptionX{executivepaper}{\PassOptionsToClass{executivepaper}{book}}
\DeclareOptionX{legalpaper}{\PassOptionsToClass{legalpaper}{book}}
\DeclareOptionX{fleqn}{\PassOptionsToClass{fleqn}{book}}
\DeclareOptionX{leqno}{\PassOptionsToClass{leqno}{book}}
\DeclareOptionX{titlepage}{\PassOptionsToClass{titlepage}{book}}
\DeclareOptionX{notitlepage}{\PassOptionsToClass{notitlepage}{book}}
\DeclareOptionX{twocolumn}{\PassOptionsToClass{twocolumn}{book}}
\DeclareOptionX{oneside}{\PassOptionsToClass{oneside}{book}}
\DeclareOptionX{twoside}{\PassOptionsToClass{twoside}{book}}
\DeclareOptionX{openright}{\PassOptionsToClass{openright}{book}}
\DeclareOptionX{openany}{\PassOptionsToClass{openany}{book}}
\DeclareOptionX{draft}{\PassOptionsToClass{draft}{book}}
\DeclareOptionX{french}{%
  \input{mythesis-french.def}%
  \ClassInfo{mythesis}{Option `french' transmit to `my' package.}%
}
\DeclareOptionX{english}{%
  \input{mythesis-english.def}%
  \ClassInfo{mythesis}{Option `english' transmit to `my' package.}%
}
\DeclareOptionX{ngerman}{%
  \ClassInfo{mythesis}{Option `ngerman' transmit to `my' package.}%
}
%    \end{macrocode}
%
% Define new options for the \xclass{mythesis} class.
%    \begin{macrocode}
\newtoggle{my@review@bool}
\DeclareOptionX{review}{\toggletrue{my@review@bool}}
\DeclareOptionX{noreview}{\togglefalse{my@review@bool}}
\newtoggle{my@minitoc@bool}
\DeclareOptionX{minitoc}{\toggletrue{my@minitoc@bool}}
\DeclareOptionX{nominitoc}{\togglefalse{my@minitoc@bool}}
\newtoggle{my@smarttab@bool}
\DeclareOptionX{smarttab}{\toggletrue{my@smarttab@bool}}
\DeclareOptionX{nosmarttab}{\togglefalse{my@smarttab@bool}}
%    \end{macrocode}
%
% Process the options.
%    \begin{macrocode}
\DeclareOptionX*{%
  \ClassWarning{mythesis}{Unknown option `\CurrentOption'}%
}
\ProcessOptionsX\relax
%    \end{macrocode}
%
% Load the base class \xclass{book}.
%    \begin{macrocode}
\LoadClass{book}[1995/12/01]
%    \end{macrocode}
%
% Transmit some options to packages.
%    \begin{macrocode}
\iftoggle{my@minitoc@bool}
{
	\PassOptionsToPackage{minitoc}{mytoc}
}{}
\iftoggle{my@smarttab@bool}
{
	\PassOptionsToPackage{smarttab}{myfloat}
}{}
%    \end{macrocode}
%
% Some useful packages are loaded here.
%    \begin{macrocode}
\RequirePackage{myfloat}
\RequirePackage{mytoc}
\RequirePackage{mycolor}
%    \end{macrocode}
%
% Change the default headers and footers with \xpackage{myheadfoot} package.
%    \begin{macrocode}
\RequirePackage{myheadfoot}
%    \end{macrocode}
%
%
% In case of a review version, the \xpackage{lineno} package is loaded. It will 
%put number on lines of the document.
%    \begin{macrocode}
\iftoggle{my@review@bool}
{
  \RequirePackage{lineno}
}{}
%    \end{macrocode}
%
% \subsection{Configuration}
% Configurate the review mode with a number each 5 lines. The counter reset on 
%each page.
%    \begin{macrocode}
\iftoggle{my@review@bool}
{
  \modulolinenumbers[5]
  \AfterEndPreamble{%
    \pagewiselinenumbers%
    \realpagewiselinenumbers%
  }
}{}
%    \end{macrocode}
%
% \subsection{Title page}
% Some new fields to create the title page are needed to define the jury, the 
%university, a subtitle\dots
% \begin{macro}{\subtitle}
%   The \cs{subtitle} macro defines a subtitle for the title page. An optional 
%argument allow to define another language for the subtitle.
%    \begin{macrocode}
\newcommand{\subtitle}[2][english]{\def\@subtitle{\begin{otherlanguage}{#1}#2\end{otherlanguage}}}%
%    \end{macrocode}
% \end{macro}
% \begin{macro}{\logoleft}
%   The \cs{logoleft} macro defines a logo for the left side of the title 
%page.
%    \begin{macrocode}
\newcommand{\logoleft}[1]{\def\@logoleft{#1}}
%    \end{macrocode}
% \end{macro}
% \begin{macro}{\logoright}
%   The \cs{logoright} macro defines a logo for the right side of the title 
%page.
%    \begin{macrocode}
\newcommand{\logoright}[1]{\def\@logoright{#1}}
%    \end{macrocode}
% \end{macro}
% \begin{macro}{\university}
%   The \cs{university} macro defines a university for the title page.
%    \begin{macrocode}
\newcommand{\university}[1]{\def\@university{#1}}
%    \end{macrocode}
% \end{macro}
% \begin{macro}{\laboratory}
%   The \cs{laboratory} macro defines a laboratory for the title page.
%    \begin{macrocode}
\newcommand{\laboratory}[1]{\def\@laboratory{#1}}
%    \end{macrocode}
% \end{macro}
% \begin{macro}{\grade}
%   The \cs{grade} macro defines the grade for which the document is written.
%    \begin{macrocode}
\newcommand{\grade}[1]{\def\@grade{#1}}
%    \end{macrocode}
% \end{macro}
% \begin{macro}{\jury}
%   The \cs{jury} macro defines the jury who evaluate the document.
%    \begin{macrocode}
\newcommand{\jury}[1]{\def\@jury{#1}}
%    \end{macrocode}
% \end{macro}
%
% \begin{macro}{\jurymember}
%   Moreover, the \cs{jury} macro should be filled with multiple member with 
%all firstname, lastname, grade\dots The \cs{jurymember} macro help to define 
%each member.
%    \begin{macrocode}
\def\jurymember#1#2{%
  \@ifnextchar[{\jurymember@i{#1}{#2}}{\jurymember@i{#1}{#2}[]}%
}
\def\jurymember@i#1#2[#3]#4#5#6{%
  #1 & #2~\myname[#3]{#4} & (#5) \\[-0.67ex]%
  \multicolumn{3}{l}{%
    \hspace{0.33\textwidth}%
    \setlength{\parskip}{0.25ex plus 0.5ex minus 0.2ex}%
    \begin{minipage}[t]{0.655\textwidth}%
      \scriptsize\textcolor{black!70}{#6}%
    \end{minipage}%
    \setlength{\parskip}{1ex plus 0.5ex minus 0.2ex}%
  } \\%
}
%    \end{macrocode}
% \end{macro}
%
% At the end, define all default values for the fields of the title page.
%    \begin{macrocode}
\author{}
\title{}
\subtitle{}
\university{}
\laboratory{}
\grade{}
\date{}
\jury{}
\logoleft{}
\logoright{}
%    \end{macrocode}
%
% The title page is splitted into three parts, \cs{my@titlepage@top}, 
%\cs{my@titlepage@bottom} and \cs{my@titlepage@body}. The \cs{my@titlepage@top} 
%contains the optional logo (one on the left and one on the right).
%    \begin{macrocode}
\newcommand{\my@titlepage@top}{%
  \ifx\@logoleft\@empty%
  \else%
    \vbox{\includegraphics[width=3cm]{\@logoleft}}%
  \fi%
  \hspace{\stretch{1}}%
  \ifx\@logoright\@empty%
  \else%
    \vbox{\includegraphics[width=3cm]{\@logoright}}%
  \fi\par%
}
%    \end{macrocode}
%
% The \cs{my@titlepage@bottom} contains the date of the defence and the jury.
%    \begin{macrocode}
\newcommand{\my@titlepage@bottom}{%
  \ifx\@date\@empty%
  \else%
    {\centering\normalsize Thèse soutenue le \@date\par}%
    \ifx\@grade\@empty%
    \else%
      {\centering\small\it pour l'obtention du grade de\par}%
      {\centering\large\bfseries \@grade\par}%
    \fi%
  \fi%
  \ifx\@jury\@empty%
  \else%
    \ifx\@date\@empty%
    {\centering\normalsize\hspace{2cm}Jury constitué de\par}%
    \else%
    {\centering\small\it en présence de\par}%
    \fi%
    \vspace{1ex}%
    {\normalsize\centering\begin{mytabular}{>{\hsize=0.3\textwidth}R>{\hsize=0.425\textwidth}L>{\hsize=0.2\textwidth}R}%
      \@jury%
    \end{mytabular}\par}%
  \vspace{1cm}%
  \fi%
}
%    \end{macrocode}
%
% The body part contains the university, the document type, the author(s), the 
%title and the subtitle.
%    \begin{macrocode}
\newcommand{\my@titlepage@body}{%
  \normalfont%
  \ifx\@university\@empty%
  \else%
  {\raggedright\normalsize\textcolor{black!70}{\@university}\par}%
  \fi%
  {\textcolor{myred}{\rule[0.5ex]{\textwidth}{1pt}}\par}%
  \ifx\@author\@empty%
  \else%
    {\large\raggedleft \@author\par}%
  \fi%
  \vfill%
  \ifx\@title\@empty%
  \else%
    {\LARGE\centering \@title\par}%
  \fi%
  \ifx\@subtitle\@empty%
  \else%
    {\vspace{0.5cm}\large\centering \@subtitle\par}%
  \fi%
}
%    \end{macrocode}
%
% \begin{macro}{\maketitle}
% Finally, the \cs{maketitle} macro is defined, depending the three previous 
%defined macros. The foot/header style page is changed (no number page and no 
%header).
%    \begin{macrocode}
\def\maketitle{%
  \begin{titlepage}%
    \null%
    %\thispagestyle{firstpage}%
    \my@titlepage@top%
    \vfill%
    \my@titlepage@body%
    \vfill%
    \my@titlepage@bottom%
    \null%
  \end{titlepage}%
  \clearpage%
  \thispagestyle{empty}%
}
%    \end{macrocode}
% \end{macro}
%
% \subsection{Preamble of the document}
% The |myabstract| environment is for the abstract of the document. The 
%optional argument is the language of the abstract (multiple abstract in 
%multiple language may be defined).
%    \begin{macrocode}
\newenvironment{myabstract}[1][french]{%
  \begin{otherlanguage}{#1}%
  \ifundef\expandafter{\csname my@lang@\languagename @Abstract\endcsname}{%
    \ClassWarning{mythesis}{Unknown language used in 'myabstract' environment}%
  }{}%
  \begin{mychapter+}[cha-abstract-\languagename]{\csname my@lang@\languagename @Abstract\endcsname}
}{%
  \end{mychapter+}
  \end{otherlanguage}%
}
%    \end{macrocode}
%
% The |mykeywords| environment is to define the keywords of the document. The 
%optional argument is the language of the abstract (multiple abstract in 
%multiple language may be defined).
%    \begin{macrocode}
\newenvironment{mykeywords}{%
  \ifundef\expandafter{\csname my@lang@\languagename @Keywords\endcsname}{%
    \ClassWarning{mythesis}{Unknown language used in 'mykeywords' environment}%
  }{}%
  \begin{myparagraph}[par-keywords-\languagename]{\csname my@lang@\languagename @Keywords\endcsname}
}{%
  \end{myparagraph}
}
%    \end{macrocode}
%
% \begin{macro}{\mykeyword}
% An additional macro is defined to describe the keyword and should be used as 
%a \cs{item} macro to list the keywords.
%    \begin{macrocode}
\newrobustcmd{\mykeyword}{\xspace--\xspace}
%    \end{macrocode}
% \end{macro}
%
% Redefine the default spacing of the paragraphs.
%    \begin{macrocode}
\setlength{\parindent}{0cm}
\setlength{\parskip}{1ex plus 0.5ex minus 0.2ex}
%    \end{macrocode}
%
% Create a default decoration for chapter pages.
%    \begin{macrocode}
\def\@makechapterhead#1{%
	\addtocontents{tcs}{\protect\addvspace{10\p@}}%
	\addtocontents{tdo}{\protect\addvspace{10\p@}}%
	\vspace*{50\p@}%
	{\parindent \z@ \raggedleft \normalfont%
	\ifnum \c@secnumdepth >\m@ne%
	\psset{unit=1cm}%
	\begin{pspicture}(-6,-1)(1,1.5)%
		\psframe[linewidth=0,linecolor=myred,fillstyle=solid,fillcolor=myred](-5,-1)(0,0)%
		\psframe[linewidth=0,framearc=.25,linecolor=myred,fillstyle=solid,fillcolor=myred](-1,-1)(1,1)%
		\psframe[linewidth=0,framearc=.25,linecolor=white,fillstyle=solid,fillcolor=white](-6,-0.95)(-1,1)%
		\uput[45](-5,-1){\large\bfseries\sffamily\textcolor{myred}{\@chapapp}}%
		\rput(0,0){\psscaleboxto(1.5,1.5){\white\selectfont\bfseries\sffamily\thechapter}}%
	\end{pspicture}%
	\par\nobreak%
	\fi%
	\interlinepenalty\@M%
	\Huge\bfseries\sffamily #1\par\nobreak%
	\vskip 40\p@%
	}%
}
%    \end{macrocode}
%
%\iffalse
%    \begin{macrocode}
%</mythesis.cls>
%    \end{macrocode}
%\fi
%
% \subsection{\xclass{mythesis-french} localisation}
%\iffalse
%    \begin{macrocode}
%<*mythesis-french.def>
%    \end{macrocode}
%\fi
%
% Identify file.
%    \begin{macrocode}
\ProvidesFile{mythesis-french.def}[2012/04/18 v1.0 French localization for 'mythesis' class]
%    \end{macrocode}
%
% Define the name which will be used.
%    \begin{macrocode}
\def\my@lang@french@abstract{résumé}
\def\my@lang@francais@abstract{résumé}
\def\my@lang@canadien@abstract{résumé}
\def\my@lang@acadian@abstract{résumé}
\def\my@lang@french@Abstract{Résumé}
\def\my@lang@francais@Abstract{Résumé}
\def\my@lang@canadien@Abstract{Résumé}
\def\my@lang@acadian@Abstract{Résumé}
\def\my@lang@french@keywords{mots-clefs}
\def\my@lang@francais@keywords{mots-clefs}
\def\my@lang@canadien@keywords{mots-clefs}
\def\my@lang@acadian@keywords{mots-clefs}
\def\my@lang@french@Keywords{Mots-clefs}
\def\my@lang@francais@Keywords{Mots-clefs}
\def\my@lang@canadien@Keywords{Mots-clefs}
\def\my@lang@acadian@Keywords{Mots-clefs}
%    \end{macrocode}
%
%\iffalse
%    \begin{macrocode}
%</mythesis-french.def>
%    \end{macrocode}
%\fi
%
% \subsection{\xclass{mythesis-english} localisation}
%\iffalse
%    \begin{macrocode}
%<*mythesis-english.def>
%    \end{macrocode}
%\fi
%
% Identify file.
%    \begin{macrocode}
\ProvidesFile{mythesis-english.def}[2012/04/18 v1.0 English localization for 'mythesis' class]
%    \end{macrocode}
%
% Define the name which will be used.
%    \begin{macrocode}
\def\my@lang@english@abstract{abstract}
\def\my@lang@english@Abstract{Abstract}
\def\my@lang@english@keywords{keywords}
\def\my@lang@english@Keywords{Keywords}
\def\my@lang@USenglish@abstract{abstract}
\def\my@lang@USenglish@Abstract{Abstract}
\def\my@lang@USenglish@keywords{keywords}
\def\my@lang@USenglish@Keywords{Keywords}
\def\my@lang@american@abstract{abstract}
\def\my@lang@american@Abstract{Abstract}
\def\my@lang@american@keywords{keywords}
\def\my@lang@american@Keywords{Keywords}
\def\my@lang@UKenglish@abstract{abstract}
\def\my@lang@UKenglish@Abstract{Abstract}
\def\my@lang@UKenglish@keywords{keywords}
\def\my@lang@UKenglish@Keywords{Keywords}
\def\my@lang@british@abstract{abstract}
\def\my@lang@british@Abstract{Abstract}
\def\my@lang@british@keywords{keywords}
\def\my@lang@british@Keywords{Keywords}
\def\my@lang@canadian@abstract{abstract}
\def\my@lang@canadian@Abstract{Abstract}
\def\my@lang@canadian@keywords{keywords}
\def\my@lang@canadian@Keywords{Keywords}
\def\my@lang@australian@abstract{abstract}
\def\my@lang@australian@Abstract{Abstract}
\def\my@lang@australian@keywords{keywords}
\def\my@lang@australian@Keywords{Keywords}
\def\my@lang@newzealand@abstract{abstract}
\def\my@lang@newzealand@Abstract{Abstract}
\def\my@lang@newzealand@keywords{keywords}
\def\my@lang@newzealand@Keywords{Keywords}
%    \end{macrocode}
%
%\iffalse
%    \begin{macrocode}
%</mythesis-english.def>
%    \end{macrocode}
%\fi
%
% \Finale
\endinput
