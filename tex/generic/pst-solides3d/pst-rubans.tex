\csname PSTRUBANSLoaded\endcsname
\let\PSTRUBANSLoaded\endinput
% Requires PSTricks, pst-xkey and pst-node packages
\ifx\PSTSOLIDESIIIDLoaded\endinput\else\input pst-solides3d.tex\fi
% JPV & ML & AS
\def\fileversion{1.0}
\def\filedate{2007/11/27}
\message{`PSTRUBANS' v\fileversion, \filedate}


\edef\PstAtCode{\the\catcode`\@} \catcode`\@=11\relax
\pst@addfams{pst-solides3d}
%
%%%%%%%%%%%%%%%%%%%%%%%%%%%%%%%%%%%%%%%%%%%%%%%%%%%%%%%%%%%%
%% Definition des differentes variables pour PSTricks
%% ruban en helice circulaire
%%%%%%%%%%%%%%%%%%%%%%%%%%%%%%%%%%%%%%%%%%%%%%%%%%%%%%%%%%%%
\define@key[psset]{pst-solides3d}{dZ}{\def\pst@solides@dZ{#1}} % hauteur du ruban
\psset[pst-solides3d]{dZ=0.5} % hauteur du ruban
\define@key[psset]{pst-solides3d}{spires}{\def\pst@solides@spires{#1}} %
\psset[pst-solides3d]{spires=10} % nombre de spires d'un ruban
\define@key[psset]{pst-solides3d}{rubans}{\def\pst@solides@rubans{#1}} %
\psset[pst-solides3d]{rubans=1} % nombre de spires d'un ruban

\def\pshelices{\pst@object{pshelices}}
%% usage : \pshelices[options]
\def\pshelices@i{\@ifnextchar({\pshelices@ii}{\pshelices@ii(0,0,0)}}
\def\pshelices@ii(#1,#2,#3){%
  \pst@killglue%
  \begingroup%
  \use@par%
\pstVerb{/dz \pst@solides@dZ\space def
/initvar_SpiralHelice {
         /rubans \pst@solides@rubans\space def
         /phase {360 rubans div} bind def
         /Rayon \pst@solides@R\space def
         /Hauteur \pst@solides@h\space def
         /PHI 0 def
         /x0 Rayon PHI cos mul def
         /y0 Rayon PHI sin mul def
         /z0 0 def
         /pulsation {6.28319 Hauteur div \pst@solides@spires\space mul} bind def
         /x1 {Rayon t pulsation mul 57.295 mul PHI add cos mul} def
         /y1 {Rayon t pulsation mul 57.295 mul PHI add sin mul} def
         /z1 {t} def
         /dA {Hauteur \pst@solides@resolution\space div} bind def
         /tableau_des_sommets [
             rubans {
          dA dA Hauteur{/t exch def
                     x0 y0 z0
                     x1 y1 z1
                     x1 y1 z1 dz add
                     x0 y0 z0 dz add
                     /x0 x1 def
                     /y0 y1 def
                     /z0 z1 def
                  } for
          /PHI PHI phase add def
          /x0 Rayon PHI cos mul def
          /y0 Rayon PHI sin mul def
          /z0 0 def
          } repeat
                  ]
           def
         /Sommets {tableau_des_sommets aload pop} def
         /NbrePoints tableau_des_sommets length 3 div cvi def
        /Faces  {0 4 NbrePoints 4 sub {
                 /Ni exch def
               [ Ni Ni 1 add  Ni 2 add Ni 3 add]
                                     } for
              } def
        /Faces_internes {0 4 NbrePoints 4 sub {
                 /Ni exch def
               [Ni 3 add Ni 2 add Ni 1 add Ni]
                                     } for
             } def
} def
}%
\codejps{
/ruban_exterieur {
20 dict begin
   initvar_SpiralHelice
/S [
  Sommets
  ] def
/F [
  Faces
  ] def
  S F generesolid
    RotX 0 ne RotY 0 ne or RotZ 0 ne or  {
        dup {RotX RotY RotZ rotateOpoint3d} solidtransform
               } if
   #1 0 ne #2 0 ne or #3 0 ne or  {
      dup {#1 #2 #3 translatepoint3d} solidtransform
              } if
   solidhue length 0 gt {
      dup solidhue solidputhuecolors
   } {
      dup (fillcolor) outputcolors
   } ifelse
end
} def
%
/ruban_interieur {
20 dict begin
   initvar_SpiralHelice
/S [
  Sommets
  ] def
/F [
  Faces_internes
  ] def
  S F generesolid
    RotX 0 ne RotY 0 ne or RotZ 0 ne or  {
        dup {RotX RotY RotZ rotateOpoint3d} solidtransform
               } if
   #1 0 ne #2 0 ne or #3 0 ne or  {
      dup {#1 #2 #3 translatepoint3d} solidtransform
              } if
  dup
 (fillincolor) outputcolors
end
  } def
ruban_interieur drawsolid**
ruban_exterieur drawsolid**}%
  \endgroup%
  \ignorespaces%
}

%%%%%%%%%%%%%%%%%%%%%%%%%%%%%%%%%%%%%%%%%%%%%%%%%%%%%%%%%%%%
%% ruban en helice torique
%%%%%%%%%%%%%%%%%%%%%%%%%%%%%%%%%%%%%%%%%%%%%%%%%%%%%%%%%%%%

\define@key[psset]{pst-solides3d}{dPHI}{\def\pst@solides@dPHI{#1}} % hauteur du ruban
\psset[pst-solides3d]{dPHI=5} % hauteur du ruban en degr�s

\def\psSpiralRing{\pst@object{psSpiralRing}}
%% usage : \psSpiralRing[options]
\def\psSpiralRing@i{\@ifnextchar({\psSpiralRing@ii}{\psSpiralRing@ii(0,0,0)}}
\def\psSpiralRing@ii(#1,#2,#3){%
%  \pst@killglue%
  \begingroup%
  \use@par%
\pstVerb{%
         /ri \pst@solides@rO\space def % rayon int�rieur
         /rm \pst@solides@rI\space def % rayon moyen
         /spires \pst@solides@spires\space def
         /dPHI \pst@solides@dPHI\space def
/initvar_SpiralRing {
         /THETA {PHI spires 2 mul mul} def
         /x {rm ri THETA cos mul add PHI cos mul} def
         /y {rm ri THETA cos mul add PHI sin mul} def
         /z {ri THETA sin mul} def
         /dt {360 \pst@solides@resolution\space div} bind def
         /PHI 0 def
         x y z /x0 x def /y0 y def /z0 z def % S1
         /tableau_des_sommets [
          0 dt 360 dt sub{/t exch def
                    /PHI t def /THETA0 THETA def
                     x y z
                     /PHI t dt add def /THETA1 THETA def
                     x y z /x1 x def /y1 y def /z1 z def % S2
                     /PHI PHI dPHI add def
                     rm ri THETA1 cos mul add PHI cos mul
                     rm ri THETA1 cos mul add PHI sin mul
                     ri THETA1 sin mul                 % S3
                     /PHI PHI dt sub def
                     rm ri THETA0 cos mul add PHI cos mul
                     rm ri THETA0 cos mul add PHI sin mul
                     ri THETA0 sin mul  % S4
                  } for
                  ]
           def
         /Sommets {tableau_des_sommets aload pop} def
         /NbrePoints tableau_des_sommets length 3 div cvi def
        /Faces  {0 4 NbrePoints 4 sub {
                 /Ni exch def
               [ Ni Ni 1 add  Ni 2 add Ni 3 add]
                                     } for
              } def
        /Faces_internes {0 4 NbrePoints 4 sub {
                 /Ni exch def
               [Ni 3 add Ni 2 add Ni 1 add Ni]
                                     } for
             } def
} def
}%
\codejps{
/ruban_exterieur {
20 dict begin
   initvar_SpiralRing
/S [
  Sommets
  ] def
/F [
  Faces
  ] def
  S F generesolid
    RotX 0 ne RotY 0 ne or RotZ 0 ne or  {
        dup {RotX RotY RotZ rotateOpoint3d} solidtransform
               } if
   #1 0 ne #2 0 ne or #3 0 ne or  {
      dup {#1 #2 #3 translatepoint3d} solidtransform
              } if
   solidhue length 0 gt {
      dup solidhue solidputhuecolors
   } {
      dup (fillcolor) outputcolors
   } ifelse
end
} def
%
/ruban_interieur {
20 dict begin
   initvar_SpiralRing
/S [
  Sommets
  ] def
/F [
  Faces_internes
  ] def
  S F generesolid
    RotX 0 ne RotY 0 ne or RotZ 0 ne or  {
        dup {RotX RotY RotZ rotateOpoint3d} solidtransform
               } if
   #1 0 ne #2 0 ne or #3 0 ne or  {
      dup {#1 #2 #3 translatepoint3d} solidtransform
              } if
  dup
 (fillincolor) outputcolors
end
  } def
ruban_interieur drawsolid**
ruban_exterieur drawsolid**}%
  \endgroup%
  \ignorespaces%
}

%%%%%%%%%%%%%%%%%%%%%%%%%%%%%%%%%%%%%%%%%%%%%%%%%%%%%%%%%%%%
%% ruban en helice spherique
%%%%%%%%%%%%%%%%%%%%%%%%%%%%%%%%%%%%%%%%%%%%%%%%%%%%%%%%%%%%


\def\psSphericalSpiral{\pst@object{psSphericalSpiral}}
%% usage : \psSphericalSpiral[options]
\def\psSphericalSpiral@i{\@ifnextchar({\psSphericalSpiral@ii}{\psSphericalSpiral@ii(0,0,0)}}
\def\psSphericalSpiral@ii(#1,#2,#3){%
  \pst@killglue%
  \begingroup%
  \use@par%
\pstVerb{%
/initvar_SpiralSpherical {
         /Rayon \pst@solides@R\space def
         /spires \pst@solides@spires\space def
         /dPHI \pst@solides@dPHI\space def
         /THETA {PHI spires 2 mul mul} def
         /x {Rayon THETA cos mul PHI cos mul} def
         /y {Rayon THETA sin mul PHI cos mul} def
         /z {Rayon PHI sin mul} def
         /dt {180 \pst@solides@resolution\space div} bind def
         /PHI -90 def
         x y z /x0 x def /y0 y def /z0 z def % S1
         /tableau_des_sommets [
          -90 dt 90 dPHI sub{/t exch def
                    /PHI t def /THETA0 THETA def
                     x y z
                     /PHI t dt add def /THETA1 THETA def
                     x y z /x1 x def /y1 y def /z1 z def % S2
                     /PHI PHI dPHI add def
%                     x1 y1 z                             % S3
                     Rayon THETA1 cos mul PHI cos mul
                     Rayon THETA1 sin mul PHI cos mul
                     z
                     /PHI PHI dt sub def
                     Rayon THETA0 cos mul PHI cos mul
                     Rayon THETA0 sin mul PHI cos mul
                     z  % S4
                  } for
                  ]
           def
         /Sommets {tableau_des_sommets aload pop} def
         /NbrePoints tableau_des_sommets length 3 div cvi def
        /Faces  {0 4 NbrePoints 4 sub {
                 /Ni exch def
               [ Ni Ni 1 add  Ni 2 add Ni 3 add]
                                     } for
              } def
        /Faces_internes {0 4 NbrePoints 4 sub {
                 /Ni exch def
               [Ni 3 add Ni 2 add Ni 1 add Ni]
                                     } for
             } def
} def
}%
\codejps{
/ruban_exterieur {
20 dict begin
   initvar_SpiralSpherical
/S [
  Sommets
  ] def
/F [
  Faces
  ] def
  S F generesolid
   RotX 0 ne RotY 0 ne or RotZ 0 ne or  {
      dup {RotX RotY RotZ rotateOpoint3d} solidtransform
   } if
   #1 0 ne #2 0 ne or #3 0 ne or  {
      dup {#1 #2 #3 translatepoint3d} solidtransform
   } if
   solidhue length 0 gt {
      dup solidhue solidputhuecolors
   } {
      dup (fillcolor) outputcolors
   } ifelse
 end
} def
%
/ruban_interieur {
20 dict begin
   initvar_SpiralSpherical
/S [
  Sommets
  ] def
/F [
  Faces_internes
  ] def
  S F generesolid
   RotX 0 ne RotY 0 ne or RotZ 0 ne or  {
      dup {RotX RotY RotZ rotateOpoint3d} solidtransform
   } if
   #1 0 ne #2 0 ne or #3 0 ne or  {
      dup {#1 #2 #3 translatepoint3d} solidtransform
   } if
  dup
 (fillincolor) outputcolors
 end
  } def
ruban_interieur drawsolid**
ruban_exterieur drawsolid**}%
  \endgroup%
  \ignorespaces%
}

%%%%%%%%%%%%%%%%%%%%%%%%%%%%%%%%%%%%%%%%%%%%%%%%%%%%%%%%%%%%
%% ruban en helice sur un paraboloide
%%%%%%%%%%%%%%%%%%%%%%%%%%%%%%%%%%%%%%%%%%%%%%%%%%%%%%%%%%%%

\define@key[psset]{pst-solides3d}{p}{\def\pst@solides@p{#1}} % parametre de la parabole
\psset[pst-solides3d]{p=2}

\def\psSpiralParaboloid{\pst@object{psSpiralParaboloid}}
%% usage : \psSpiralParaboloid[options]
\def\psSpiralParaboloid@i{\@ifnextchar({\psSpiralParaboloid@ii}{\psSpiralParaboloid@ii(0,0,0)}}
\def\psSpiralParaboloid@ii(#1,#2,#3){%
%  \pst@killglue%
  \begingroup%
  \use@par%
\pstVerb{%
         /dz@ \pst@solides@dZ\space def % hauteur du ruban
         /p@ {\pst@solides@p\space 2 mul} bind def % param�tre de la parabole x 2
         /Hauteur \pst@solides@h\space def
/initvar_SpiralParaboloid {
         /pulsation {6.28319 Hauteur div \pst@solides@spires\space mul} bind def
         /x {z p@ mul sqrt t pulsation mul 57.295 mul cos mul} def
         /y {z p@ mul sqrt t pulsation mul 57.295 mul sin mul} def
         /dH {Hauteur \pst@solides@resolution\space div} bind def
         /tableau_des_sommets [
          0 dH Hauteur {/H exch def
                        /z H def
                        /t H def
                     x y z   % S1
                       /t t dH add def
                       /z t def
                     x y z   % S2
                     /z z dz@ add def
                     x y z  % S3
                       /z H dz@ add def
                       /t H def
                     x y z   % S4
                  } for
                  ]
           def
         /Sommets {tableau_des_sommets aload pop} def
         /NbrePoints tableau_des_sommets length 3 div cvi def
        /Faces  {0 4 NbrePoints 4 sub {
                 /Ni exch def
               [ Ni Ni 1 add  Ni 2 add Ni 3 add]
                                     } for
              } def
        /Faces_internes {0 4 NbrePoints 4 sub {
                 /Ni exch def
               [Ni 3 add Ni 2 add Ni 1 add Ni]
                                     } for
             } def
} def
}%
\codejps{
/ruban_exterieur {
20 dict begin
   initvar_SpiralParaboloid
   /S [
     Sommets
   ] def
   /F [
     Faces
   ] def
   S F generesolid
   RotX 0 ne RotY 0 ne or RotZ 0 ne or  {
      dup {RotX RotY RotZ rotateOpoint3d} solidtransform
   } if
   #1 0 ne #2 0 ne or #3 0 ne or  {
      dup {#1 #2 #3 translatepoint3d} solidtransform
   } if
   solidhue length 0 gt {
      dup solidhue solidputhuecolors
   } {
      dup (fillcolor) outputcolors
   } ifelse
end
} def
%
/ruban_interieur {
20 dict begin
   initvar_SpiralParaboloid
   /S [
     Sommets
   ] def
   /F [
     Faces_internes
   ] def
   S F generesolid
   RotX RotY RotZ add add abs 0 gt {
        dup {RotX RotY RotZ rotateOpoint3d} solidtransform
   } if
   #1 #2 #3 add add abs 0 gt {
      dup {#1 #2 #3 translatepoint3d} solidtransform
   } if
   dup
  (fillincolor) outputcolors
end
} def
ruban_interieur drawsolid**
ruban_exterieur drawsolid**}%
  \endgroup%
  \ignorespaces%
}

%%%%%%%%%%%%%%%%%%%%%%%%%%%%%%%%%%%%%%%%%%%%%%%%%%%%%%
\catcode`\@=\PstAtCode\relax
\endinput
