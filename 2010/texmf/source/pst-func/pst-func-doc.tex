%% $Id: pst-func-doc.tex 621 2012-01-01 15:26:33Z herbert $
\documentclass[11pt,english,BCOR10mm,DIV12,bibliography=totoc,parskip=false,
   smallheadings, headexclude,footexclude,oneside]{pst-doc}
\usepackage[utf8]{inputenc}
\usepackage{pst-func}
\let\pstFuncFV\fileversion
\usepackage{pst-math}
\usepackage{pstricks-add}
\renewcommand\bgImage{%
\psset{yunit=4cm,xunit=3}
\begin{pspicture}(-2,-0.2)(2,1.4)
  \psaxes[Dy=0.25]{->}(0,0)(-2,0)(2,1.25)[$x$,0][$y$,90]
  \rput[lb](1,0.75){\textcolor{red}{$\sigma =0.5$}}
  \rput[lb](1,0.5){\textcolor{blue}{$\sigma =1$}}
  \rput[lb](-2,0.5){$f(x)=\dfrac{1}{\sigma\sqrt{2\pi}}\,e^{-\dfrac{(x-\mu)^2}{2\sigma{}^2}}$}
  \psGauss[linecolor=red, linewidth=2pt]{-1.75}{1.75}%
  \psGaussI[linewidth=1pt]{-2}{2}%
  \psGauss[linecolor=cyan, mue=0.5, linewidth=2pt]{-1.75}{1.75}%
  \psGauss[sigma=1, linecolor=blue, linewidth=2pt]{-1.75}{1.75}
\end{pspicture}}

\lstset{language=PSTricks,
    morekeywords={psGammaDist,psChiIIDist,psTDist,psFDist,psBetaDist,psPlotImpl},basicstyle=\footnotesize\ttfamily}
%
\def\pshlabel#1{\footnotesize#1}
\def\psvlabel#1{\footnotesize#1}
%
\begin{document}

\title{\texttt{pst-func}}
\subtitle{Plotting special mathematical functions; v.\pstFuncFV}
\author{Herbert Vo\ss}
\docauthor{}
\date{\today}
\maketitle

\tableofcontents
\psset{unit=1cm}

\clearpage

\begin{abstract}
\noindent
\LPack{pst-func} loads by default the following packages: \LPack{pst-plot}, 
\LPack{pstricks-add}, \LPack{pst-math}, \LPack{pst-xkey}, and, of course \LPack{pstricks}.
All should be already part of your local \TeX\ installation. If not, or in case
of having older versions, go to \url{http://www.CTAN.org/} and load the newest version.

\vfill\noindent
Thanks to: \\
Rafal Bartczuk,
    Jean-C\^ome Charpentier,
    Martin Chicoine, 
    Gerry Coombes, 
    Denis Girou,
    John Frampton, 
    Attila Gati, 
    Horst Gierhardt,  
    Christophe Jorssen,
    Lars Kotthoff, 
    Buddy Ledger,
    Manuel Luque,
    Patrice Mégret,
    Matthias Rüss,
    Jose-Emilio Vila-Forcen,
Timothy Van Zandt, 
Michael Zedler,
and last but not least \url{http://mathworld.wolfram.com}

\end{abstract}


\section{\nxLcs{psBezier\#}}
This macro can plot a B\'ezier spline from order 1 up to 9 which needs
(order+1) pairs of given coordinates.

Given a set of $n+1$ control points $P_0$, $P_1$, \ldots, $P_n$, the corresponding \Index{B\'ezier} curve 
(or \Index{Bernstein-B\'ezier} curve) is given by 
%
\begin{align}
C(t)=\sum_{i=0}^n P_i B_{i,n}(t)
\end{align}
%
Where $B_{i,n}(t)$ is a Bernstein polynomial $B_{i,n}(t)=\binom{n}{i}t^i(1-t)^{n-i}$, 
 and $t \in [0,1]$. 
The Bézier curve starts through the first and last given point and 
lies within the convex hull of all control points. The curve is tangent 
to $P_1-P_0$ and $P_n-P_{n-1}$ at the endpoint.
Undesirable properties of \Index{Bézier curve}s are their numerical instability for 
large numbers of control points, and the fact that moving a single control 
point changes the global shape of the curve. The former is sometimes avoided 
by smoothly patching together low-order Bézier curves. 

The macro \Lcs{psBezier} (note the upper case B) expects the number of the order 
and $n=order+1$ pairs of coordinates:

\begin{BDef}
\Lcs{psBezier}\Larg{\#}\OptArgs\coord0\coord1\coordn
\end{BDef}

The number of steps between the first and last control points is given
by the keyword \Lkeyword{plotpoints} and preset to 200. It can be
changed in the usual way. 


\begin{lstlisting}
\psset{showpoints=true,linewidth=1.5pt}
\begin{pspicture}(-2,-2)(2,2)% order 1 -- linear
  \psBezier1{<->}(-2,0)(-2,2)
\end{pspicture}\qquad
%
\begin{pspicture}(-2,-2)(2,2)% order 2 -- quadratric
  \psBezier2{<->}(-2,0)(-2,2)(0,2)
\end{pspicture}\qquad
%
\begin{pspicture}(-2,-2)(2,2)% order 3 -- cubic
  \psBezier3{<->}(-2,0)(-2,2)(0,2)(2,2)
\end{pspicture}\qquad

\vspace{1cm}
\begin{pspicture}(-2,-2)(2,2)% order 4 -- quartic
  \psBezier4{<->}(-2,0)(-2,2)(0,2)(2,2)(2,0)
\end{pspicture}\qquad
%
\begin{pspicture}(-2,-2)(2,2)% order 5 -- quintic
  \psBezier5{<->}(-2,0)(-2,2)(0,2)(2,2)(2,0)(2,-2)
\end{pspicture}\qquad
%
\begin{pspicture}(-2,-2)(2,2)% order 6
  \psBezier6{<->}(-2,0)(-2,2)(0,2)(2,2)(2,0)(2,-2)(0,-2)
\end{pspicture}\qquad

\vspace{1cm}
\begin{pspicture}(-2,-2)(2,2)% order 7
  \psBezier7{<->}(-2,0)(-2,2)(0,2)(2,2)(2,0)(2,-2)(0,-2)(-2,-2)
\end{pspicture}\qquad
%
\begin{pspicture}(-2,-2)(2,2)% order 8
  \psBezier8{<->}(-2,0)(-2,2)(0,2)(2,2)(2,0)(2,-2)(0,-2)(-2,-2)(-2,0)
\end{pspicture}\qquad
%
\begin{pspicture}(-2,-2)(2,2)% order 9
  \psBezier9{<->}(-2,0)(-2,2)(0,2)(2,2)(2,0)(2,-2)(0,-2)(-2,-2)(-2,0)(0,0)
\end{pspicture}
\end{lstlisting}


\begingroup
\psset{showpoints=true,linewidth=1.5pt}
\begin{pspicture}(-2,-2)(2,2)% order 1 -- linear
  \psBezier1{<->}(-2,0)(-2,2)
\end{pspicture}\qquad
%
\begin{pspicture}(-2,-2)(2,2)% order 2 -- quadratric
  \psBezier2{<->}(-2,0)(-2,2)(0,2)
\end{pspicture}\qquad
%
\begin{pspicture}(-2,-2)(2,2)% order 3 -- cubic
  \psBezier3{<->}(-2,0)(-2,2)(0,2)(2,2)
\end{pspicture}\qquad

\vspace{1cm}
\begin{pspicture}(-2,-2)(2,2)% order 4 -- quartic
  \psBezier4{<->}(-2,0)(-2,2)(0,2)(2,2)(2,0)
\end{pspicture}\qquad
%
\begin{pspicture}(-2,-2)(2,2)% order 5 -- quintic
  \psBezier5{<->}(-2,0)(-2,2)(0,2)(2,2)(2,0)(2,-2)
\end{pspicture}\qquad
%
\begin{pspicture}(-2,-2)(2,2)% order 6
  \psBezier6{<->}(-2,0)(-2,2)(0,2)(2,2)(2,0)(2,-2)(0,-2)
\end{pspicture}\qquad

\vspace{1cm}
\begin{pspicture}(-2,-2)(2,2)% order 7
  \psBezier7{<->}(-2,0)(-2,2)(0,2)(2,2)(2,0)(2,-2)(0,-2)(-2,-2)
\end{pspicture}\qquad
%
\begin{pspicture}(-2,-2)(2,2)% order 8
  \psBezier8{<->}(-2,0)(-2,2)(0,2)(2,2)(2,0)(2,-2)(0,-2)(-2,-2)(-2,0)
\end{pspicture}\qquad
%
\begin{pspicture}(-2,-2)(2,2)% order 9
  \psBezier9{<->}(-2,0)(-2,2)(0,2)(2,2)(2,0)(2,-2)(0,-2)(-2,-2)(-2,0)(0,0)
\end{pspicture}
\endgroup

\clearpage
\section{Polynomials}

\subsection{Chebyshev polynomials}

The polynomials of the first (\Lps{ChebyshevT}) kind are defined through the identity

\[ T_n(\cos\theta)=\cos(n\theta)\] 

They can be obtained from the generating functions
\begin{align}
  g_1(t,x) &= \frac{1-t^2}{1-2xt+t^2}\\
	   &= T_0(x)+2\sum_{n=1}^\infty T_n(x)t^n
\end{align}

and

\begin{align}
  g_2(t,x) &= \frac{1-xt}{1-2xt+t^2}\\
           &= \sum_{n=0}^\infty T_n(x)t^n
\end{align}

The polynomials of second kind (\Lps{ChebyshevU}) can be generated by

\begin{align}
  g(t,x) &= \frac{1}{1-2xt+t^2}\\
         &= \sum_{n=0}^\infty U_n(x)t^n
\end{align}

\LPack{pst-func} defines the \TeX-macros \Lcs{ChebyshevT} for the
first kind and \Lcs{ChebyshevU} for the second kind of \Index{Chebyshev polynomials}.
These \TeX-macros cannot be used outside of PostScript, they are only wrappers
for \verb+tx@FuncDict begin ChebyshevT end+ and the same for \Lcs{ChebyshevU}.

\begin{center}
\bgroup
\psset{arrowscale=1.5,unit=3cm}
\begin{pspicture}(-1.5,-1.5)(1.5,1.5)
  \psaxes[ticks=none,labels=none]{->}(0,0)(-1.25,-1.25)(1.25,1.25)%
    [Re$\{s_{21}\}$,0][Im$\{s_{21}\}$,90]
  \pscircle(0,0){1}
  \parametricplot[linecolor=blue,plotpoints=10000]{0}{1.5}{
    /N 9 def
    /x 2 N mul t \ChebyshevT def
    /y 2 N mul 1 sub t \ChebyshevU def
    x x 2 exp y 2 exp add div
    y x 2 exp y 2 exp add div
  }
\end{pspicture}
\egroup
\end{center}

\begin{lstlisting}
\psset{arrowscale=1.5,unit=3cm}
\begin{pspicture}(-1.5,-1.5)(1.5,1.5)
  \psaxes[ticks=none,labels=none]{->}(0,0)(-1.25,-1.25)(1.25,1.25)%
    [Re$\{s_{21}\}$,0][Im$\{s_{21}\}$,90]
  \pscircle(0,0){1}
  \parametricplot[linecolor=blue,plotpoints=10000]{0}{1.5}{
    /N 9 def
    /x 2 N mul t \ChebyshevT def
    /y 2 N mul 1 sub t \ChebyshevU def
    x x 2 exp y 2 exp add div
    y x 2 exp y 2 exp add div
  }
\end{pspicture}
\end{lstlisting}

\begin{center}
\bgroup
\psset{xunit=4cm,yunit=3cm,plotpoints=1000}
\begin{pspicture}(-1.2,-2)(2,1.5)
  \psaxes[Dx=0.2]{->}(0,0)(-1.25,-1.2)(1.25,1.2)
  \psset{linewidth=1.5pt}
  \psplot[linestyle=dashed]{-1}{1}{1 x \ChebyshevT}
  \psplot[linecolor=black]{-1}{1}{2 x \ChebyshevT}
  \psplot[linecolor=black]{-1}{1}{3 x \ChebyshevT}
  \psplot[linecolor=blue]{-1}{1}{4 x \ChebyshevT }
  \psplot[linecolor=red]{-1}{1}{5 x \ChebyshevT }
\end{pspicture}
\egroup
\end{center}



\begin{lstlisting}
\psset{xunit=4cm,yunit=3cm,plotpoints=1000}
\begin{pspicture}(-1.2,-2)(2,1.5)
  \psaxes[Dx=0.2]{->}(0,0)(-1.25,-1.2)(1.25,1.2)
  \psset{linewidth=1.5pt}
  \psplot[linestyle=dashed]{-1}{1}{1 x \ChebyshevT}
  \psplot[linecolor=black]{-1}{1}{2 x \ChebyshevT}
  \psplot[linecolor=black]{-1}{1}{3 x \ChebyshevT}
  \psplot[linecolor=blue]{-1}{1}{4 x \ChebyshevT }
  \psplot[linecolor=red]{-1}{1}{5 x \ChebyshevT }
\end{pspicture}
\end{lstlisting}

\begin{center}
\bgroup
\psset{xunit=4cm,yunit=3cm,plotpoints=1000}
\begin{pspicture*}(-1.5,-1.5)(1.5,1.5)
  \psaxes[Dx=0.2]{->}(0,0)(-1.15,-1.1)(1.15,1.1)
  \psset{linewidth=1.5pt}
  \psplot[linecolor=black]{-1}{1}{2 x \ChebyshevU}
  \psplot[linecolor=black]{-1}{1}{3 x \ChebyshevU}
  \psplot[linecolor=blue]{-1}{1}{4 x \ChebyshevU }
  \psplot[linecolor=red]{-1}{1}{5 x \ChebyshevU }
\end{pspicture*}
\egroup
\end{center}



\begin{lstlisting}
\psset{xunit=4cm,yunit=3cm,plotpoints=1000}
\begin{pspicture*}(-1.5,-1.5)(1.5,1.5)
  \psaxes[Dx=0.2]{->}(0,0)(-1.15,-1.1)(1.15,1.1)
  \psaxes[Dx=0.2]{->}(0,0)(-1.25,-1.2)(1.25,1.2)
  \psset{linewidth=1.5pt}
  \psplot[linecolor=black]{-1}{1}{2 x \ChebyshevU}
  \psplot[linecolor=black]{-1}{1}{3 x \ChebyshevU}
  \psplot[linecolor=blue]{-1}{1}{4 x \ChebyshevU }
  \psplot[linecolor=red]{-1}{1}{5 x \ChebyshevU }
\end{pspicture*}
\end{lstlisting}

\begin{center}
\bgroup
\psset{xunit=4cm,yunit=3cm,plotpoints=1000}
\begin{pspicture}(-1.25,-1.2)(1.25,1.2)
  \psaxes[Dx=0.2]{->}(0,0)(-1.25,-1.1)(1.25,1.1)
  \psset{linewidth=1.5pt}
  \psplot[linecolor=black]{-1}{1}{x ACOS 2 mul RadtoDeg cos}
  \psplot[linecolor=black]{-1}{1}{x ACOS 3 mul RadtoDeg cos}
  \psplot[linecolor=blue]{-1}{1}{x ACOS 4 mul RadtoDeg cos}
  \psplot[linecolor=red]{-1}{1}{x ACOS 5 mul RadtoDeg cos}
\end{pspicture}
\egroup
\end{center}

\begin{lstlisting}
\psset{xunit=4cm,yunit=3cm,plotpoints=1000}
\begin{pspicture}(-1.25,-1.2)(1.25,1.2)
  \psaxes[Dx=0.2]{->}(0,0)(-1.25,-1.2)(1.25,1.2)
  \psset{linewidth=1.5pt}
  \psplot[linecolor=black]{-1}{1}{x ACOS 2 mul RadtoDeg cos}
  \psplot[linecolor=black]{-1}{1}{x ACOS 3 mul RadtoDeg cos}
  \psplot[linecolor=blue]{-1}{1}{x ACOS 4 mul RadtoDeg cos}
  \psplot[linecolor=red]{-1}{1}{x ACOS 5 mul RadtoDeg cos}
\end{pspicture}
\end{lstlisting}

\subsection{\Lcs{psPolynomial}}
The polynomial function is defined as
%
\begin{align}
f(x) &= a_0 + a_1x + a_2x^2 + a_3x^3 + \ldots +a_{n-1}x^{n-1} + a_nx^n\\
f^{\prime}(x) &= a_1 + 2a_2x + 3a_3x^2 + \ldots +(n-1)a_{n-1}x^{n-2} + na_nx^{n-1}\\
f^{\prime\prime}(x) &= 2a_2 + 6a_3x + \ldots +(n-1)(n-2)a_{n-1}x^{n-3} + n(n-1)a_nx^{n-2}
\end{align}


\noindent so \LPack{pst-func} needs only the \Index{coefficients} of the
polynomial to calculate the function. The syntax is

\begin{BDef}
\Lcs{psPolynomial}\OptArgs\Largb{xStart}\Largb{xEnd}
\end{BDef}

With the option \Lkeyword{xShift} one can do a horizontal shift to the graph of the function. With another
than the predefined value the macro replaces $x$ by $x-x\mathrm{Shift}$; \Lkeyword{xShift}=1
moves the graph of the \Index{polynomial function} one unit to the right.


\begin{center}
\bgroup
\psset{yunit=0.5cm,xunit=1cm}
\begin{pspicture*}(-3,-5)(5,10)
  \psaxes[Dy=2]{->}(0,0)(-3,-5)(5,10)
  \psset{linewidth=1.5pt}
  \psPolynomial[coeff=6 3 -1,linecolor=red]{-3}{5}
  \psPolynomial[coeff=2 -1 -1 .5 -.1 .025,linecolor=blue]{-2}{4}
  \psPolynomial[coeff=-2 1 -1 .5 .1 .025 .2 ,linecolor=magenta]{-2}{4}
  \psPolynomial[coeff=-2 1 -1 .5 .1 .025 .2 ,linecolor=magenta,xShift=1,linestyle=dashed]{-2}{4}
  \rput[lb](4,4){\textcolor{red}{$f(x)$}}
  \rput[lb](4,8){\textcolor{blue}{$g(x)$}}
  \rput[lb](2,4){\textcolor{magenta}{$h(x)$}}
\end{pspicture*}
\egroup
\end{center}


\begin{lstlisting}
\psset{yunit=0.5cm,xunit=1cm}
\begin{pspicture*}(-3,-5)(5,10)
  \psaxes[Dy=2]{->}(0,0)(-3,-5)(5,10)
  \psset{linewidth=1.5pt}
  \psPolynomial[coeff=6 3 -1,linecolor=red]{-3}{5}
  \psPolynomial[coeff=2 -1 -1 .5 -.1 .025,linecolor=blue]{-2}{4}
  \psPolynomial[coeff=-2 1 -1 .5 .1 .025 .2 ,linecolor=magenta]{-2}{4}
  \psPolynomial[coeff=-2 1 -1 .5 .1 .025 .2 ,linecolor=magenta,xShift=1,linestyle=dashed]{-2}{4}
  \rput[lb](4,4){\textcolor{red}{$f(x)$}}
  \rput[lb](4,8){\textcolor{blue}{$g(x)$}}
  \rput[lb](2,4){\textcolor{magenta}{$h(x)$}}
\end{pspicture*}
\end{lstlisting}


The plot is easily clipped using the star version of the
\Lenv{pspicture} environment, so that points whose coordinates
are outside of the desired range are not plotted.
The plotted polynomials are:
%
\begin{align}
f(x) & = 6 + 3x -x^2 \\
g(x) & = 2 -x -x^2 +0.5x^3 -0.1x^4 +0.025x^5\\
h(x) & = -2 +x -x^2 +0.5x^3 +0.1x^4 +0.025x^5+0.2x^6\\
h^*(x) & = -2 +(x-1) -(x-1)^2 +0.5(x-1)^3 +\nonumber\\
       & \phantom{ = }+0.1(x-1)^4 +0.025(x-1)^5+0.2(x-1)^6
\end{align}
%
There are the following new options:

\noindent\medskip
{\tabcolsep=2pt
\begin{tabularx}{\linewidth}{@{}l>{\ttfamily}l>{\ttfamily}lX@{}}
Name & \textrm{Value}  & \textrm{Default}\\\hline
\Lkeyword{coeff}        & a0 a1 a2 ... & 0 0 1 & The coefficients must have the order $a_0\ a_1\ a_2 \ldots$ and
be separated by \textbf{spaces}. The number of coefficients
is limited only by the memory of the computer ... The default
value of the parameter \Lkeyword{coeff} is \verb+0 0 1+, which gives
the parabola $y=a_0+a_1x+a_2x^2=x^2$.\\
\Lkeyword{xShift} & <number>     & 0     & $(x-xShift)$ for the horizontal shift of the polynomial\\
\Lkeyword{Derivation} & <number>     & 0     & the default is the function itself\\
\Lkeyword{markZeros}    & false|true       & false & dotstyle can be changed\\
\Lkeyword{epsZero}      & <value> &  0.1 & The distance between two zeros, important for
                                  the iteration function to test, if the zero value still
				  exists\\
\Lkeyword{dZero}      & <value> &  0.1 & When searching for all zero values, the function is scanned
                              with this step\\
\Lkeyword{zeroLineTo}    & <number>  & false & plots a line from the zero point to the value of the
                                    zeroLineTo's Derivation of the polynomial function\\
\Lkeyword{zeroLineStyle}    & <line style>  & \Lkeyval{dashed} & the style is one of the for \PST valid styles.\\
\Lkeyword{zeroLineColor}  & <color>  & \Lkeyval{black} & any valid xolor is possible\\
\Lkeyword{zeroLineWidth}  & <width> & \rlap{0.5\textbackslash pslinewidth} & \\
\end{tabularx}
}



\bigskip
The above parameters are only 
valid for the \Lcs{psPolynomial} macro, except \verb+x0+, which can also be used for the Gauss function. All
options can be set in the usual way with \Lcs{psset}. 


\bigskip
\begin{LTXexample}
\psset{yunit=0.5cm,xunit=2cm}
\begin{pspicture*}(-3,-5)(3,10)
  \psaxes[Dy=2]{->}(0,0)(-3,-5)(3,10)
  \psset{linewidth=1.5pt}
  \psPolynomial[coeff=-2 1 -1 .5 .1 .025 .2 ,linecolor=magenta]{-2}{4}
  \psPolynomial[coeff=-2 1 -1 .5 .1 .025 .2 ,linecolor=red,%
    linestyle=dashed,Derivation=1]{-2}{4}
  \psPolynomial[coeff=-2 1 -1 .5 .1 .025 .2 ,linecolor=blue,%
    linestyle=dotted,Derivation=2]{-2}{4}
  \rput[lb](2,4){\textcolor{magenta}{$h(x)$}}
  \rput[lb](1,1){\textcolor{red}{$h^{\prime}(x)$}}
  \rput[lb](-1,6){\textcolor{blue}{$h^{\prime\prime}(x)$}}
\end{pspicture*}
\end{LTXexample}
%$


\begin{LTXexample}
\psset{yunit=0.5cm,xunit=2cm}
\begin{pspicture*}(-3,-5)(3,10)
  \psaxes[Dy=2]{->}(0,0)(-3,-5)(3,10)
  \psset{linewidth=1.5pt}
  \psPolynomial[coeff=0 0 0 1,linecolor=blue]{-2}{4}
  \psPolynomial[coeff=0 0 0 1,linecolor=red,%
    linestyle=dashed,Derivation=2]{-2}{4}
  \psPolynomial[coeff=0 0 0 1,linecolor=cyan,%
    linestyle=dotted,Derivation=3]{-2}{4}
  \rput[lb](1.8,4){\textcolor{blue}{$f(x)=x^3$}}
  \rput[lb](0.2,8){\textcolor{red}{$f^{\prime\prime}(x)=6x$}}
  \rput[lb](-2,5){\textcolor{cyan}{$f^{\prime\prime\prime}(x)=6$}}
\end{pspicture*}
\end{LTXexample}
%$

\begin{LTXexample}
\begin{pspicture*}(-5,-5)(5,5)
  \psaxes{->}(0,0)(-5,-5)(5,5)%
  \psset{dotscale=2}
  \psPolynomial[markZeros,linecolor=red,linewidth=2pt,coeff=-1 1 -1 0 0.15]{-4}{3}%
  \psPolynomial[markZeros,linecolor=blue,linewidth=1pt,linestyle=dashed,%
    coeff=-1 1 -1 0 0.15,Derivation=1,zeroLineTo=0]{-4}{3}%
  \psPolynomial[markZeros,linecolor=magenta,linewidth=1pt,linestyle=dotted,%
    coeff=-1 1 -1 0 0.15,Derivation=2,zeroLineTo=0]{-4}{3}%
  \psPolynomial[markZeros,linecolor=magenta,linewidth=1pt,linestyle=dotted,%
    coeff=-1 1 -1 0 0.15,Derivation=2,zeroLineTo=1]{-4}{3}%
\end{pspicture*}
\end{LTXexample}

\begin{LTXexample}
\psset{xunit=1.5}
\begin{pspicture*}(-5,-5)(5,5)
  \psaxes{->}(0,0)(-5,-5)(5,5)%
  \psset{dotscale=2,dotstyle=x,zeroLineStyle=dotted,zeroLineWidth=1pt}
  \psPolynomial[markZeros,linecolor=red,linewidth=2pt,coeff=-1 1 -1 0 0.15]{-4}{3}%
  \psPolynomial[markZeros,linecolor=blue,linewidth=1pt,linestyle=dashed,%
    coeff=-1 1 -1 0 0.15,Derivation=1,zeroLineTo=0]{-4}{3}%
  \psPolynomial[markZeros,linecolor=magenta,linewidth=1pt,linestyle=dotted,%
    coeff=-1 1 -1 0 0.15,Derivation=2,zeroLineTo=0]{-4}{3}%
  \psPolynomial[markZeros,linecolor=magenta,linewidth=1pt,linestyle=dotted,%
    coeff=-1 1 -1 0 0.15,Derivation=2,zeroLineTo=1]{-4}{3}%
\end{pspicture*}
\end{LTXexample}

\clearpage
\subsection{\Lcs{psBernstein}}
The polynomials defined by
%
\[ B_{i,n}(t)=\binom{n}{i}t^i(1-t)^{n-i} \]
%
where $\tbinom{n}{k}$ is a binomial coefficient are named Bernstein polynomials of degree $n$.
They form a basis for the power polynomials of degree $n$.
The Bernstein polynomials satisfy symmetry
\[B_{i,n}(t)=B_{n-i,n}(1-t)\]
positivity \[B_{i,n}(t)\ge0 \mbox{\qquad for } 0\le t\le1\]
normalization \[\sum_{i=0}^nB_{i,n}(t)=1\] 	
and $B_{i,n}$ with $i!=0$, $n$ has a single unique local maximum of
\[i^in^{-n}(n-i)^{n-i}\binom{n}{i}\]	
occurring at $t=\frac{i}{n}$.
The envelope $f_n(x)$ of the Bernstein polynomials $B_{i,n}(x)$ for $i=0,1,\ldots,n$ 
is given by \[f_n(x)=\frac{1}{\sqrt{\pi n\cdot x(1-x)}}\] 	
illustrated below for $n=20$. 

\begin{BDef}
\Lcs{psBernstein}\OptArgs\Largr{tStart,tEnd}\Largr{i,n}
\end{BDef}

The (\Lkeyword{tStart}, \Lkeyword{tEnd}) are \emph{optional} and preset by \verb=(0,1)=. The only new optional
argument is the boolean key \Lkeyword{envelope}, which plots the envelope curve instead
of the Bernstein polynomial.

\begin{LTXexample}[width=5cm,pos=l]
\psset{xunit=4.5cm,yunit=3cm}
\begin{pspicture}(1,1.1)
  \psaxes{->}(0,0)(1,1)[$t$,0][$B_{0,0}$,90]
  \psBernstein[linecolor=red,linewidth=1pt](0,0)
\end{pspicture}
\end{LTXexample}

\begin{LTXexample}[width=5cm,pos=l]
\psset{xunit=4.5cm,yunit=3cm}
\begin{pspicture}(1,1.1)
  \psaxes{->}(0,0)(1,1)[$t$,0][$B_{i,1}$,90]
  \psBernstein[linecolor=blue,linewidth=1pt](0,1)
  \psBernstein[linecolor=blue,linewidth=1pt](1,1)
\end{pspicture}
\end{LTXexample}

\begin{LTXexample}[width=5cm,pos=l]
\psset{xunit=4.5cm,yunit=3cm}
\begin{pspicture}(1,1.1)
  \psaxes{->}(0,0)(1,1)[$t$,0][$B_{i,2}$,90]
  \multido{\i=0+1}{3}{\psBernstein[linecolor=red,
    linewidth=1pt](\i,2)}
\end{pspicture}
\end{LTXexample}

\begin{LTXexample}[width=5cm,pos=l]
\psset{xunit=4.5cm,yunit=3cm}
\begin{pspicture}(1,1.1)
  \psaxes{->}(0,0)(1,1)[$t$,0][$B_{i,3}$,90]
  \multido{\i=0+1}{4}{\psBernstein[linecolor=magenta,
    linewidth=1pt](\i,3)}
\end{pspicture}
\end{LTXexample}

\begin{LTXexample}[width=5cm,pos=l]
\psset{xunit=4.5cm,yunit=3cm}
\begin{pspicture}(1,1.1)
  \psaxes{->}(0,0)(1,1)[$t$,0][$B_{i,4}$,90]
  \multido{\i=0+1}{5}{\psBernstein[linecolor=cyan,
    linewidth=1pt](\i,4)}
\end{pspicture}
\end{LTXexample}

\begin{LTXexample}[width=5cm,pos=l]
\psset{xunit=4.5cm,yunit=3cm}
\begin{pspicture}(-0.1,-0.05)(1.1,1.1)
  \multido{\i=0+1}{20}{\psBernstein[linecolor=green,
    linewidth=1pt](\i,20)}
  \psBernstein[envelope,linecolor=black](0.02,0.98)(0,20)
  \psaxes{->}(0,0)(1,1)[$t$,0][$B_{i,20}$,180]
\end{pspicture}
\end{LTXexample}

\begin{LTXexample}[width=5cm,pos=l]
\psset{xunit=4.5cm,yunit=3cm}
\begin{pspicture*}(-0.2,-0.05)(1.1,1.1)
  \psaxes{->}(0,0)(1,1)[$t$,0][$B_{env}$,180]
  \multido{\i=2+1}{20}{\psBernstein[envelope,
    linewidth=1pt](0.01,0.99)(0,\i)}
\end{pspicture*}
\end{LTXexample}


\psset{unit=1cm}
\clearpage
\section{\Lcs{psFourier}}

A Fourier sum has the form:
%
\begin{align}
s(x) = \frac{a_0}{2} & + a_1\cos{\omega x} + a_2\cos{2\omega x} +
   a_3\cos{3\omega x} +
	\ldots + a_n\cos{n\omega x}\\
	& + b_1\sin{\omega x} + b_2\sin{2\omega x} + b_3\sin{3\omega x} +
	\ldots + b_m\sin{m\omega x}
\end{align}
%
\noindent The macro \Lcs{psFourier} plots \Index{Fourier sums}. The
syntax is similiar to \Lcs{psPolynomial}, except that there are
two kinds of coefficients:

\begin{BDef}
\Lcs{psFourier}\OptArgs\Largb{xStart}\Largb{xEnd}
\end{BDef}

The coefficients must have the orders $cosCoeff=a_0\ a_1\ a_2\ \ldots$
and $sinCoeff=b_1\ b_2\ b_3\ \ldots$ and be separated by
\textbf{spaces}. The default is \Lkeyword{cosCoeff}=0,\Lkeyword{sinCoeff}=1,
which gives the standard \verb+sin+ function. Note that
%%JF, I think it is better without the angle brackets, but
%%you know the conventions used better than I do, so you
%%may disagree.
%the constant value can only be set with \verb+cosCoeff=<a0>+.
the constant value can only be set with \Lkeyword{cosCoeff}=\verb+a0+.

\begin{LTXexample}
\begin{pspicture}(-5,-3)(5,5.5)
\psaxes{->}(0,0)(-5,-2)(5,4.5)
\psset{plotpoints=500,linewidth=1pt}
\psFourier[cosCoeff=2, linecolor=green]{-4.5}{4.5}
\psFourier[cosCoeff=0 0 2, linecolor=magenta]{-4.5}{4.5}
\psFourier[cosCoeff=2 0 2, linecolor=red]{-4.5}{4.5}
\end{pspicture}
\end{LTXexample}

\begin{LTXexample}
\psset{yunit=0.75}
\begin{pspicture}(-5,-6)(5,7)
\psaxes{->}(0,0)(-5,-6)(5,7)
\psset{plotpoints=500}
\psFourier[linecolor=red,linewidth=1pt]{-4.5}{4.5}
\psFourier[sinCoeff= -1 1 -1 1 -1 1 -1 1,%
	linecolor=blue,linewidth=1.5pt]{-4.5}{4.5}
\end{pspicture}
\end{LTXexample}

\begin{LTXexample}
\begin{pspicture}(-5,-5)(5,5.5)
\psaxes{->}(0,0)(-5,-5)(5,5)
\psset{plotpoints=500,linewidth=1.5pt}
\psFourier[sinCoeff=-.5 1 1 1 1 ,cosCoeff=-.5 1 1 1 1 1,%
	linecolor=blue]{-4.5}{4.5}
\end{pspicture}
\end{LTXexample}

\clearpage
\section{\Lcs{psBessel}}
The Bessel function of order $n$ is defined as
%
\begin{align}
J_n(x) &=\frac{1}{\pi}\int_0^\pi\cos(x\sin t-nt)\dt\\
       &=\sum_{k=0}^{\infty}\frac{(-1)^k \left(\frac{x}{2}\right)^{n+2k}}{k!\Gamma(n+k+1)}
\end{align}
%
\noindent The syntax of the macro is

\begin{BDef}
\Lcs{psBessel}\OptArgs\Largb{order}\Largb{xStart}\Largb{xEnd}
\end{BDef}

There are two special parameters for the Bessel function, and also the
settings of many \LPack{pst-plot} or \LPack{pstricks} parameters
affect the plot. 
These two ,,constants`` have the following meaning:
%
\[
f(t) = constI \cdot J_n + constII
\]
%
\noindent
where \Lkeyword{constI} and \Lkeyword{constII} must be real PostScript expressions, e.g.:

\begin{lstlisting}[style=syntax]
\psset{constI=2.3,constII=t k sin 1.2 mul 0.37 add}
\end{lstlisting}

The Bessel function is plotted with the parametricplot macro, this is the
reason why the variable is named \verb+t+. The internal procedure \verb+k+ 
converts the value t from radian into degrees. The above setting is
the same as
%
\[
f(t) = 2.3 \cdot J_n + 1.2\cdot \sin t + 0.37
\]
%
In particular, note that the default for
\Lkeyword{plotpoints} is $500$. If the plotting computations are too
time consuming at this setting, it can be decreased in the usual
way, at the cost of some reduction in graphics resolution.

\begin{LTXexample}
{
\psset{xunit=0.25,yunit=5}
\begin{pspicture}(-13,-.85)(13,1.25)
\rput(13,0.8){%
	$\displaystyle J_n(x)=\frac{1}{\pi}\int_0^\pi\cos(x\sin t-nt)\dt$%
}
\psaxes[Dy=0.2,Dx=4]{->}(0,0)(-30,-.8)(30,1.2)
\psset{linewidth=1pt}
\psBessel[linecolor=red]{0}{-28}{28}%
\psBessel[linecolor=blue]{1}{-28}{28}%
\psBessel[linecolor=green]{2}{-28}{28}%
\psBessel[linecolor=magenta]{3}{-28}{28}%
\end{pspicture}
}
\end{LTXexample}


\begin{LTXexample}
{
\psset{xunit=0.25,yunit=2.5}
\begin{pspicture}(-13,-1.5)(13,3)
\rput(13,0.8){%
	$\displaystyle f(t) = 2.3 \cdot J_0 + 1.2\cdot \sin t + 0.37$%
}
\psaxes[Dy=0.8,dy=2cm,Dx=4]{->}(0,0)(-30,-1.5)(30,3)
\psset{linewidth=1pt}
\psBessel[linecolor=red,constI=2.3,constII={t k sin 1.2 mul 0.37 add}]{0}{-28}{28}%
\end{pspicture}
}
\end{LTXexample}

\clearpage

\clearpage
\section{Modfied Bessel function of first order}
The modified Bessel function of first order is defined as
%
\begin{align}
I_\nu(x) &= \left(\frac12 x\right)^\nu
  \sum\limits_{k=0}^{\infty}  \frac{{\left(\frac14 x^2\right)}^k}{k!\Gamma(\nu+k+1)}
\end{align}
%
\noindent The syntax of the macro is

\begin{BDef}
\Lcs{psModBessel}\OptArgs\Largb{xStart}\Largb{xEnd}
\end{BDef}

The only valid optional argument for the function is \Lkeyword{nue}, which
is preset to 0, it shows $I_0$.


\begin{LTXexample}
\begin{pspicture}(0,-0.5)(5,5)
\psaxes[ticksize=-5pt 0]{->}(5,5)
\psModBessel[yMaxValue=5,nue=0,linecolor=red]{0}{5}
\psModBessel[yMaxValue=5,nue=1,linecolor=green]{0}{5}
\psModBessel[yMaxValue=5,nue=2,linecolor=blue]{0}{5}
\psModBessel[yMaxValue=5,nue=3,linecolor=cyan]{0}{5}
\end{pspicture}
\end{LTXexample}


\clearpage
\section{\Lcs{psSi}, \Lcs{pssi} and \Lcs{psCi}}
The integral sin  and cosin are defined as
%
\begin{align}
\mathrm{Si}(x) &= \int_0^x\dfrac{\sin t}{t}\dt\\
\mathrm{si}(x) &= - \int_x^{\infty}\dfrac{\sin t}{t}\dt=\mathrm{Si}(x)-\frac{\pi}{2}\\
\mathrm{Ci}(x) &= -\int_x^{\infty}\dfrac{\cos t}{t}\dt=\gamma+\ln x +\int_0^{x}\dfrac{\cos t -1}{t}\dt
\end{align}
%
\noindent The syntax of the macros is

\begin{BDef}
\Lcs{psSi}\OptArgs\Largb{xStart}\Largb{xEnd}\\
\Lcs{pssi}\OptArgs\Largb{xStart}\Largb{xEnd}\\
\Lcs{psCi}\OptArgs\Largb{xStart}\Largb{xEnd}
\end{BDef}


\begin{LTXexample}[pos=t]
\def\pshlabel#1{\footnotesize#1} \def\psvlabel#1{\footnotesize#1}
\psset{xunit=0.5}
\begin{pspicture}(-15,-4.5)(15,2)
  \psaxes[dx=1cm,Dx=2]{->}(0,0)(-15.1,-4)(15,2)
  \psplot[plotpoints=1000]{-14.5}{14.5}{ x RadtoDeg sin x div }
  \psSi[plotpoints=1500,linecolor=red,linewidth=1pt]{-14.5}{14.5}
  \pssi[plotpoints=1500,linecolor=blue,linewidth=1pt]{-14.5}{14.5}
  \rput(-5,1.5){\color{red}$Si(x)=\int\limits_{0}^x \frac{\sin(t)}{t}\dt$}  
  \rput(8,-1.5){\color{blue}$si(x)=-\int\limits_{x}^{\infty} \frac{\sin(t)}{t}\dt=Si(x)-\frac{\pi}{2}$}  
  \rput(8,.5){$f(x)= \frac{\sin(t)}{t}$}
\end{pspicture}
\end{LTXexample}



\begin{LTXexample}[pos=t]
\def\pshlabel#1{\footnotesize#1} \def\psvlabel#1{\footnotesize#1}
\psset{xunit=0.5}
\begin{pspicture*}(-15,-4.2)(15,4.2)
  \psaxes[dx=1cm,Dx=2]{->}(0,0)(-15.1,-4)(15,4)
  \psplot[plotpoints=1000]{-14.5}{14.5}{ x RadtoDeg cos x Div }
  \psCi[plotpoints=500,linecolor=red,linewidth=1pt]{-11.5}{11.5}
  \psci[plotpoints=500,linecolor=blue,linewidth=1pt]{-11.5}{11.5}
  \rput(-8,1.5){\color{red}$Ci(x)=-\int\limits_{x}^{\infty} \frac{\cos(t)}{t}\dt$}  
  \rput(8,1.5){\color{blue}$ci(x)=-Ci(x)+\ln(x)+\gamma$}  
\end{pspicture*}
\end{LTXexample}


\clearpage
\section{\nxLcs{psIntegral}, \nxLcs{psCumIntegral}, and \nxLcs{psConv}}
These new macros\footnote{Created by Jose-Emilio Vila-Forcen}
allows to plot the result of an integral using the Simpson numerical integration rule. 
The first one is the result of the integral of a function with two variables, and 
the integral is performed over one of them. The second one is the cumulative 
integral of a function (similar to \Lcs{psGaussI} but valid for all functions). The third 
one is the result of a convolution. They are defined as:
%
\begin{align}
\text{\Lcs{psIntegral}}(x)    &= \int\limits_a^b f(x,t)\mathrm{d}t \\
\text{\Lcs{psCumIntegral}}(x) &= \int\limits_{\text{xStart}}^{x} f(t)\mathrm{d}t \\
\text{\Lcs{psConv}}(x)        &= \int\limits_a^b f(t)g(x-t)\mathrm{d}t
\end{align}
%
In the first one, the integral is performed from $a$ to $b$ and the function $f$ depends 
on two parameters. In the second one, the function $f$ depends on only one parameter, and the 
integral is performed from the minimum value specified for $x$ (\Lkeyword{xStart}) and the current 
value of $x$ in the plot. The third one uses the \Lcs{psIntegral} macro to perform an approximation 
to the convolution, where the integration is performed from $a$ to $b$.

The syntax of these macros is:

\begin{BDef}
\Lcs{psIntegral}\OptArgs\Largb{xStart}\Largb{xEnd}\Largr{a,b}\Largb{ function }\\
\Lcs{psCumIngegral}\OptArgs\Largb{xStart}\Largb{xEnd}\Largb{ function }\\
\Lcs{psConv}\OptArgs\Largb{xStart}\Largb{xEnd}\Largr{a,b}\Largb{ function f }\Largb{ function g }
\end{BDef}

In the first macro, the function should be created such that it accepts two values: \verb|<x t function>| 
should be a value. For the second and the third functions, they only need to accept one 
parameter: \verb|<x function>| should be a value.

There are no new parameters for these functions. The two most important ones are \Lkeyword{plotpoints}, 
which controls the number of points of the plot (number of divisions on $x$ for the plot) and 
\Lkeyword{Simpson}, which controls the precision of the integration (a larger number means a smallest 
step). The precision and the smoothness of the plot depend strongly on these two parameters.

\bigskip
\begin{LTXexample}
%\usepackage{pst-math}
\psset{xunit=0.5cm,yunit=2cm}
\begin{pspicture}[linewidth=1pt](-10,-.5)(10,1.5)
  \psaxes[dx=1cm,Dx=2]{->}(0,0)(-10,0)(10,1.5)
  \psCumIntegral[plotpoints=200,Simpson=10]{-10}{10}{0 1 GAUSS}
  \psIntegral[plotpoints=200,Simpson=100,linecolor=green]{.1}{10}(-3,3){0 exch GAUSS}
  \psIntegral[plotpoints=200,Simpson=10,linecolor=red,
    fillcolor=red!40,fillstyle=solid,opacity=0.5]{-10}{10}(-4,6){1 GAUSS}
\end{pspicture}
\end{LTXexample}

In the example, the cumulative integral of a Gaussian is presented in black. In red, a 
Gaussian is varying its mean from -10 to 10, and the result is the integral from -4 to 6. 
Finally, in green it is presented the integral of a Gaussian from -3 to 3, where the 
variance is varying from .1 to 10.

\begin{LTXexample}
\psset{xunit=1cm,yunit=4cm}
\begin{pspicture}[linewidth=1pt](-5,-.2)(5,0.75)
  \psaxes[dx=1cm,Dx=1,Dy=0.5]{->}(0,0)(-5,0)(5,0.75)
  \psplot[linecolor=blue,plotpoints=200]{-5}{5}{x abs 2 le {0.25}{0} ifelse}
  \psplot[linecolor=green,plotpoints=200]{-5}{5}{x abs 1 le {.5}{0} ifelse}
  \psConv[plotpoints=100,Simpson=1000,linecolor=red]{-5}{5}(-10,10)%
    {abs 2 le {0.25}{0} ifelse}{abs 1 le {.5} {0} ifelse}
\end{pspicture}
\end{LTXexample}

In the second example, a convolution is performed using two rectangle functions. 
The result (in red) is a \Index{trapezoid function}.

\clearpage
\section{Distributions}
All distributions which use the $\Gamma$- or $\ln\Gamma$-function need the \LPack{pst-math} package,
it defines the PostScript functions \Lps{GAMMA} and \Lps{GAMMALN}. \LPack{pst-func} reads by default the PostScript
file \LFile{pst-math.pro}. It is part of any \TeX\ distribution and should also be on
your system, otherwise install or update it from \textsc{CTAN}. It must the latest version. 

\begin{LTXexample}[pos=l,width=7cm]
\begin{pspicture*}(-0.5,-0.5)(6.2,5.2)
 \psaxes{->}(0,0)(6,5)
 \psset{plotpoints=100,linewidth=1pt}
 \psplot[linecolor=red]{0.01}{4}{ x GAMMA }
 \psplot[linecolor=blue]{0.01}{5}{ x GAMMALN }
\end{pspicture*}
\end{LTXexample}



\clearpage
\subsection{Normal distribution (Gauss)}
The Gauss function is defined as
%
\begin{align}
f(x) &= \dfrac{1}{\sigma\sqrt{2\pi}}\,e^{-\dfrac{\left(x-\mu\right)^2}{2\sigma{}^2}}
\end{align}
%
\noindent The syntax of the macros is

\begin{BDef}
\Lcs{psGauss}\OptArgs\Largb{xStart}\Largb{xEnd}\\
\Lcs{psGaussI}\OptArgs\Largb{xStart}\Largb{xEnd}
\end{BDef}

\noindent where the only new parameter are \Lkeyword{sigma}=<value>+ and  \Lkeyword{mue}=<value>+ for the
horizontal shift,
which can also be set in the usual way with \Lcs{psset}. It is
significant only for the \Lcs{psGauss}- and \Lcs{psGaussI}-macro. The default is
\Lkeyword{sigma}=0.5 and \Lkeyword{mue}=0. The integral is caclulated wuth the Simson algorithm 
and has one special option, called \Lkeyword{Simpson}, which defines the number of intervalls per step
and is predefined with 5.


\begin{LTXexample}[pos=t,preset=\centering,wide=true]
\psset{yunit=4cm,xunit=3}
\begin{pspicture}(-2,-0.2)(2,1.4)
%  \psgrid[griddots=10,gridlabels=0pt, subgriddiv=0]
  \psaxes[Dy=0.25]{->}(0,0)(-2,0)(2,1.25)
  \uput[-90](6,0){x}\uput[0](0,1){y}
  \rput[lb](1,0.75){\textcolor{red}{$\sigma =0.5$}}
  \rput[lb](1,0.5){\textcolor{blue}{$\sigma =1$}}
  \rput[lb](-2,0.5){$f(x)=\dfrac{1}{\sigma\sqrt{2\pi}}\,e^{-\dfrac{(x-\mu)^2}{2\sigma{}^2}}$}
  \psGauss[linecolor=red, linewidth=2pt]{-1.75}{1.75}%
  \psGaussI[linewidth=1pt]{-2}{2}%
  \psGauss[linecolor=cyan, mue=0.5, linewidth=2pt]{-1.75}{1.75}%
  \psGauss[sigma=1, linecolor=blue, linewidth=2pt]{-1.75}{1.75}
\end{pspicture}
\end{LTXexample}



\clearpage
\subsection{Binomial distribution}\label{sec:bindistri}

These two macros plot binomial distribution, \Lcs{psBinomialN} the normalized one. It is always 
done in the $x$-Intervall $[0;1]$.
Rescaling to another one can be done by setting the \Lkeyword{xunit} option
to any other value. 

The binomial distribution gives the discrete probability distribution $P_p(n|N)$ of obtaining
exactly $n$ successes out of $N$ Bernoulli trials (where the result of each 
Bernoulli trial is true with probability $p$ and false with probability
$q=1-p$.  The binomial distribution is therefore given by

\begin{align}
P_p(n|N) &= \binom{N}{n}p^nq^{N-n} \\
         &= \frac{N!}{n!(N-n)!}p^n(1-p)^{N-n},
\end{align}
where $(N; n)$ is a binomial coefficient and $P$ the probability. 

The syntax is quite easy:

\begin{BDef}
\Lcs{psBinomial}\OptArgs\Largb{N}\Largb{probability p}\\
\Lcs{psBinomial}\OptArgs\Largb{m,N}\Largb{probability p}\\
\Lcs{psBinomial}\OptArgs\Largb{m,n,N}\Largb{probability p}\\
\Lcs{psBinomialN}\OptArgs\Largb{N}\Largb{probability p}
\end{BDef}

\begin{itemize}
\item with one argument $N$ the sequence $0\ldots N$ is calculated and plotted
\item with two arguments $m,N$ the sequence $0\ldots N$ is calculated and 
    the sequence $m\ldots N$ is plotted
\item with three arguments $m,n,N$ the sequence $0\ldots N$ is calculated and 
    the sequence $m\ldots n$ is plotted
\end{itemize}

There is a restriction in using the value for N. It depends to the probability, but in general
one should expect problems with $N>100$. PostScript cannot handle such small values and there will
be no graph printed. This happens on PostScript side, so \TeX\ doesn't report any problem in
the log file. The valid options for the macros are \Lkeyword{markZeros} to draw rectangles instead
of a continous line and \Lkeyword{printValue} for printing the $y$-values on top of the lines,
rotated by 90\textdegree. For this option all other options from section~\ref{sec:printValue}
for the macro \Lcs{psPrintValue} are valid, too. The only special option is \Lkeyword{barwidth},
which is a factor (no dimension) and set by default to 1. This option is only valid for
the macro \Lcs{psBinomial} and not for the normalized one!

\psset[pst-func]{barwidth=1}
\begin{LTXexample}[pos=t,preset=\centering]
\psset{xunit=1cm,yunit=5cm}%
\begin{pspicture}(-1,-0.15)(7,0.55)%
\psaxes[Dy=0.2,dy=0.2\psyunit]{->}(0,0)(-1,0)(7,0.5)
\uput[-90](7,0){$k$} \uput[90](0,0.5){$P(X=k)$}
\psBinomial[markZeros,printValue,fillstyle=vlines]{6}{0.4}
\end{pspicture}
\end{LTXexample}

\begin{LTXexample}[pos=t,preset=\centering]
\psset{xunit=1cm,yunit=10cm}%
\begin{pspicture}(-1,-0.05)(8,0.6)%
\psaxes[Dy=0.2,dy=0.2\psyunit]{->}(0,0)(-1,0)(8,0.5)
\uput[-90](8,0){$k$} \uput[90](0,0.5){$P(X=k)$}
\psBinomial[linecolor=red,markZeros,printValue,fillstyle=solid,
	fillcolor=blue,barwidth=0.2]{7}{0.6}
\end{pspicture}
\end{LTXexample}


\begin{LTXexample}[pos=t,preset=\centering]
\psset{xunit=1cm,yunit=10cm}%
\begin{pspicture}(-1,-0.05)(8,0.6)%
\psaxes[Dy=0.2,dy=0.2\psyunit]{->}(0,0)(-1,0)(8,0.5)
\uput[-90](8,0){$k$} \uput[90](0,0.5){$P(X=k)$}
\psBinomial[linecolor=black!30]{0,7}{0.6}
\psBinomial[linecolor=blue,markZeros,printValue,fillstyle=solid,
	fillcolor=blue,barwidth=0.4]{2,5,7}{0.6}
\end{pspicture}
\end{LTXexample}


\begin{LTXexample}[pos=t,preset=\centering]
\psset{xunit=0.25cm,yunit=10cm}
\begin{pspicture*}(-1,-0.05)(61,0.52)
\psaxes[Dx=5,dx=5\psxunit,Dy=0.2,dy=0.2\psyunit]{->}(60,0.5)
\uput[-90](60,0){$k$} \uput[0](0,0.5){$P(X=k)$}
\psBinomial[markZeros,linecolor=red]{4}{.5}
\psset{linewidth=1pt}
\psBinomial[linecolor=green]{5}{.5} \psBinomial[linecolor=blue]{10}{.5}
\psBinomial[linecolor=red]{20}{.5}  \psBinomial[linecolor=magenta]{50}{.5}
\psBinomial[linecolor=cyan]{0,55,75}{.5}
\end{pspicture*}
\end{LTXexample}

The default binomial distribution has the mean of $\mu=E(X)=N\cdot p$ and a variant of $\sigma^2=\mu\cdot(1-p)$.
The normalized distribution has a mean of $0$. Instead of $P(X=k)$ we use $P(Z=z)$ with $Z=\dfrac{X-E(X)}{\sigma(X)}$
and $P\leftarrow P\cdot\sigma$.
The macros use the rekursive definition of the binomial distribution:
%
\begin{align}
P(k) &= P(k-1)\cdot\frac{N-k+1}{k}\cdot\frac{p}{1-p}
\end{align}



\begin{LTXexample}[pos=t,preset=\centering]
\psset{xunit=1cm,yunit=5cm}%
\begin{pspicture}(-3,-0.15)(4,0.55)%
\psaxes[Dy=0.2,dy=0.2\psyunit]{->}(0,0)(-3,0)(4,0.5)
\uput[-90](4,0){$z$} \uput[0](0,0.5){$P(Z=z)$}
\psBinomialN[markZeros,fillstyle=vlines]{6}{0.4}
\end{pspicture}
\end{LTXexample}



\begin{LTXexample}[pos=t,preset=\centering]
\psset{yunit=10}
\begin{pspicture*}(-8,-0.07)(8.1,0.55)
\psaxes[Dy=0.2,dy=0.2\psyunit]{->}(0,0)(-8,0)(8,0.5)
\uput[-90](8,0){$z$} \uput[0](0,0.5){$P(Z=z)$}
\psBinomialN{125}{.5}
\psBinomialN[markZeros,linewidth=1pt,linecolor=red]{4}{.5}
\end{pspicture*}
\end{LTXexample}

\begin{LTXexample}[pos=t,preset=\centering]
\psset{yunit=10}
\begin{pspicture*}(-8,-0.07)(8.1,0.52)
\psaxes[Dy=0.2,dy=0.2\psyunit]{->}(0,0)(-8,0)(8,0.5)
\uput[-90](8,0){$z$} \uput[0](0,0.5){$P(Z=z)$}
\psBinomialN[markZeros,linecolor=red]{4}{.5}
\psset{linewidth=1pt}
\psBinomialN[linecolor=green]{5}{.5}\psBinomialN[linecolor=blue]{10}{.5}
\psBinomialN[linecolor=red]{20}{.5} \psBinomialN[linecolor=gray]{50}{.5}
\end{pspicture*}
\end{LTXexample}



For the normalized distribution the plotstyle can be set to \Lkeyval{curve} (\Lkeyset{plotstyle=curve}), 
then the binomial distribution looks like a normal distribution. This option is only
valid vor \Lcs{psBinomialN}. The option \Lkeyword{showpoints} is valid if \Lkeyval{curve} was chosen.


\begin{LTXexample}[pos=t,preset=\centering]
\psset{xunit=1cm,yunit=10cm}%
\begin{pspicture*}(-4,-0.06)(4.1,0.57)%
\psaxes[Dy=0.2,dy=0.2\psyunit]{->}(0,0)(-4,0)(4,0.5)%
\uput[-90](4,0){$z$} \uput[90](0,0.5){$P(Z=z)$}%
\psBinomialN[linecolor=red,fillstyle=vlines,showpoints=true,markZeros]{36}{0.5}%
\psBinomialN[linecolor=blue,showpoints=true,plotstyle=curve]{36}{0.5}%
\end{pspicture*}
\end{LTXexample}


\begin{LTXexample}[pos=t,preset=\centering]
\psset{xunit=1cm,yunit=10cm}%
\begin{pspicture*}(-4,-0.06)(4.2,0.57)%
\psaxes[Dy=0.2,dy=0.2\psyunit]{->}(0,0)(-4,0)(4,0.5)%
\uput[-90](4,0){$z$} \uput[90](0,0.5){$P(Z=z)$}%
\psBinomialN[linecolor=red]{10}{0.6}%
\psBinomialN[linecolor=blue,showpoints=true,plotstyle=curve]{10}{0.6}%
\end{pspicture*}
\end{LTXexample}


\clearpage
\subsection{Poisson distribution}
Given a Poisson process\footnote{\url{http://mathworld.wolfram.com/PoissonProcess.html}}, 
the probability of obtaining exactly $n$ successes in $N$ trials is given by the 
limit of a binomial distribution (see Section~\ref{sec:bindistri})
%
\begin{align}
P_p(n|N) &= \frac{N!}{n!(N-n)!}\cdot p^n(1-p)^{N-n}\label{eq:normaldistri}
\end{align}
%
Viewing the distribution as a function of the expected number of successes
%	
\begin{align}\label{eq:nu}
\lambda &= n\cdot p
\end{align}
%
instead of the sample size $N$ for fixed $p$, equation (2) then becomes
eq.~\ref{eq:normaldistri}
%	
\begin{align}\label{eq:nuN}
P_{\frac{\lambda}{n}}(n|N) &= \frac{N!}{n!(N-n)!}{\frac{\lambda}{N}}^n {\frac{1-\lambda}{N}}^{N-n}
\end{align}
%
Viewing the distribution as a function of the expected number of successes
%
\[ P_\lambda(X=k)=\frac{\lambda^k}{k!}\,e^{-\lambda} \]
%
Letting the sample size  become large ($N\to\infty$), the distribution then 
approaches (with $p=\frac{\lambda}{n}$)
%
\begin{align}
\lim_{n\to\infty} P(X=k) &= \lim_{n\to\infty}\frac{n!}{(n-k)!\,k!}
	 \left(\frac{\lambda}{n}\right)^k \left(1-\frac{\lambda}{n}\right)^{n-k} \\
  &= \lim_{n\to\infty} \left(\frac{(n-k)!\cdot (n-k+1)\cdots(n-2)(n-1)n}{(n-k)!\,n^k}\right)\cdot\\
  &\qquad \left(\frac{\lambda^k}{k!}\right)\left(1-\frac{\lambda}{n}\right)^n 
	\left(1-\frac{\lambda}{n}\right)^{-k}\\
  &= \frac{\lambda^k}{k!}\cdot \lim_{n\to\infty}
     \underbrace{\left(\frac{n}{n}\cdot \frac{n-1}{n}\cdot\frac{n-2}{n}\cdot\ldots\cdot 
    	\frac{n-k+1}{n}\right)}_{\to 1} \cdot\\
  &\qquad   \underbrace{\left(1-\frac{\lambda}{n}\right)^n}_{\to{e^{-\lambda}}}  
     \underbrace{\left(1-\frac{\lambda}{n}\right)^{-k}}_{\to 1}\\
  &=  \lambda^k e^{\frac{-\lambda}{k!}}
\end{align}
%
which is known as the Poisson distribution and has the follwing syntax:


\begin{BDef}
\Lcs{psPoisson}\OptArgs\Largb{N}\Largb{lambda}\\
\Lcs{psPoisson}\OptArgs\Largb{M,N}\Largb{lambda}
\end{BDef}

in which \texttt{M} is an optional argument with a default of 0.


\begin{LTXexample}[pos=t,preset=\centering]
\psset{xunit=1cm,yunit=20cm}%
\begin{pspicture}(-1,-0.05)(14,0.25)%
\uput[-90](14,0){$k$} \uput[90](0,0.2){$P(X=k)$}
\psPoisson[linecolor=red,markZeros,fillstyle=solid,
	fillcolor=blue!10,printValue,valuewidth=20]{13}{6} % N lambda
\psaxes[Dy=0.1,dy=0.1\psyunit]{->}(0,0)(-1,0)(14,0.2)
\end{pspicture}
\end{LTXexample}

\begin{LTXexample}[pos=t,preset=\centering]
\psset{xunit=1cm,yunit=20cm}%
\begin{pspicture}(-1,-0.05)(14,0.25)%
\uput[-90](14,0){$k$} \uput[90](0,0.2){$P(X=k)$}
\psPoisson[linecolor=blue,markZeros,fillstyle=solid,barwidth=0.4,
	fillcolor=blue!10,printValue,valuewidth=20]{10}{6} % N lambda
\psaxes[Dy=0.1,dy=0.1\psyunit]{->}(0,0)(-1,0)(11,0.2)
\end{pspicture}
\end{LTXexample}


\begin{LTXexample}[pos=t,preset=\centering]
\psset{xunit=1cm,yunit=20cm}%
\begin{pspicture}(-1,-0.05)(14,0.25)%
\uput[-90](14,0){$k$} \uput[90](0,0.2){$P(X=k)$}
\psPoisson[printValue,valuewidth=20]{2,11}{6} % M,N lambda
\psaxes[Dy=0.1,dy=0.1\psyunit]{->}(0,0)(-1,0)(14,0.2)
\end{pspicture}
\end{LTXexample}



\clearpage
\subsection{Gamma distribution}
A gamma distribution is a general type of statistical distribution that is related 
to the beta distribution and arises naturally in processes for which the waiting 
times between Poisson distributed events are relevant. Gamma distributions have 
two free parameters, labeled $alpha$ and $beta$. It is defined as
%
\[
f(x)=\frac{\beta(\beta x)^{\alpha-1}e^{-\beta x}}{\Gamma(\alpha)} \qquad
\text{for $x>0$ and $\alpha$, $\beta>0$}
\]
%
and has the syntax

\begin{BDef}
\Lcs{psGammaDist}\OptArgs\Largb{x0}\Largb{x1}
\end{BDef}

\begin{LTXexample}[pos=t,preset=\centering]
\psset{xunit=1.2cm,yunit=10cm,plotpoints=200}
\begin{pspicture*}(-0.75,-0.05)(9.5,0.6)
 \psGammaDist[linewidth=1pt,linecolor=red]{0.01}{9}
 \psGammaDist[linewidth=1pt,linecolor=blue,alpha=0.3,beta=0.7]{0.01}{9}
 \psaxes[Dy=0.1]{->}(0,0)(9.5,.6)
\end{pspicture*}
\end{LTXexample}


\clearpage
\subsection{$\chi^2$-distribution}
The $\chi^2$-distribution is a continuous probability distribution. It
usually arises when a $k$-dimensional vector's orthogonal components are 
independent and each follow a standard normal distribution. 
The length of the vector will then have a $\chi^2$-distribution.

\iffalse
If Y_i have normal independent  distributions with mean 0 and variance 1, then
chi^2=sum_(i==1)^rY_i^2	
(1)

is distributed as chi^2 with r degrees of freedom. This makes a chi^2 distribution 
a gamma distribution with theta=2 and alpha=r/2, where r is the number of degrees of freedom.

More generally, if chi_i^2 are independently distributed according to a chi^2 
distribution with r_1, r_2, ..., r_k degrees of freedom, then
sum_(j==1)^kchi_j^2	

is distributed according to chi^2 with r=sum_(j==1)^(k)r_j degrees of freedom. 
\fi

The $\chi^2$ with parameter $\nu$ is the same as a Gamma distribution 
 with $\alpha=\nu/2$ and $\beta=1/2$ and the syntax

\begin{BDef}
\Lcs{psChiIIDist}\OptArgs\Largb{x0}\Largb{x1}
\end{BDef}

\begin{LTXexample}[pos=t,preset=\centering]
\psset{xunit=1.2cm,yunit=10cm,plotpoints=200}
\begin{pspicture*}(-0.75,-0.05)(9.5,.65)
 \multido{\rnue=0.5+0.5,\iblue=0+10}{10}{%
   \psChiIIDist[linewidth=1pt,linecolor=blue!\iblue,nue=\rnue]{0.01}{9}}
 \psaxes[Dy=0.1]{->}(0,0)(9.5,.6)
\end{pspicture*}
\end{LTXexample}

\iffalse
The cumulative distribution function is 
%
\begin{align*}
D_r(\chi^2) &=	int_0^{\chi^2}\frac{t^{r/2-1}e^{-t/2}\mathrm{d}t}{\Gamma(1/2r)2^{r/2}}	\\
	    &=	1-\frac{\Gamma(1/2r,1/2\chi^2)}{\Gamma(1/2r)}
\end{align*}
\fi




\clearpage
\subsection{Student's $t$-distribution}

A \Index{statistical distribution} published by \Index{William Gosset} in 1908 under his 
pseudonym ,,Student``. The $t$-distribution with parameter $\nu$ has the \Index{density function}
%
\[
f(x)=\frac1{\sqrt{\nu\pi}}\cdot
 \frac{\Gamma[(\nu+1)/2]}{\Gamma(\nu/2)}\cdot\frac1{[1+(x^2/\nu)]^{(\nu+1)/2}} \qquad
\text{for $-\infty<x<\infty$ and $\nu>0$}
\]
%
and the following syntax

\begin{BDef}
\Lcs{psTDist}\OptArgs\Largb{x0}\Largb{x1}
\end{BDef}


\begin{LTXexample}[pos=t,preset=\centering]
\psset{xunit=1.25cm,yunit=10cm}
\begin{pspicture}(-6,-0.1)(6,.5)
 \psaxes[Dy=0.1]{->}(0,0)(-4.5,0)(5.5,0.5)
 \psset{linewidth=1pt,plotpoints=100}
 \psGauss[mue=0,sigma=1]{-4.5}{4.5}
 \psTDist[linecolor=blue]{-4}{4}
 \psTDist[linecolor=red,nue=4]{-4}{4}
\end{pspicture}
\end{LTXexample}


%The $t_\nu$-distribution has mode 0.

\clearpage
\subsection{$F$-distribution}
A continuous statistical distribution which arises in the testing of 
whether two observed samples have the same variance. 

The F-distribution with parameters $\mu$ and $\nu$ has the probability function
\[
f_{n,m}(x)=\frac{\Gamma[(\mu+\nu)/2]}{\Gamma(\mu/2)\Gamma(\nu/2)}\cdot
 \left(\mu/\nu\right)^{\mu/2}\frac{x^{(\mu/2)-1}}{[1+(\mu x/\nu)]^{(\mu+\nu)/2}}\quad
\text{ for $x>0$ and $\mu$, $\nu>0$}\]
%
and the syntax 

\begin{BDef}
\Lcs{psFDist}\OptArgs\Largb{x0}\Largb{x1}
\end{BDef}
%
The default settings are $\mu=1$ and $\nu=1$.

\begin{LTXexample}[pos=t,preset=\centering]
\psset{xunit=2cm,yunit=10cm,plotpoints=100}
\begin{pspicture*}(-0.5,-0.07)(5.5,0.8)
 \psline[linestyle=dashed](0.5,0)(0.5,0.75)
 \psline[linestyle=dashed](! 2 7 div 0)(! 2 7 div 0.75)
 \psset{linewidth=1pt}
 \psFDist{0.1}{5} 
 \psFDist[linecolor=red,nue=3,mue=12]{0.01}{5}
 \psFDist[linecolor=blue,nue=12,mue=3]{0.01}{5}
 \psaxes[Dy=0.1]{->}(0,0)(5,0.75)
\end{pspicture*}
\end{LTXexample}


\clearpage
\subsection{Beta distribution}
	
A general type of statistical distribution which is related to the gamma distribution. 
Beta distributions have two free parameters, which are labeled according to one of two 
notational conventions. The usual definition calls these $\alpha$ and $\beta$, and the other 
uses $\beta^\prime=\beta-1$ and $\alpha^\prime=\alpha-1$. The beta distribution is 
used as a prior distribution for binomial proportions in \Index{Bayesian analysis}. 
%
%The plots are for various values of ($\alpha,\beta$) with $\alpha=1$ and $\beta$ ranging from 0.25 to 3.00.
%
The domain is $[0,1]$, and the probability function $P(x)$ is given by
%
\[
P(x)	=	\frac{\Gamma(\alpha+\beta)}{\Gamma(\alpha)\Gamma(\beta)}(1-x)^{\beta-1}x^{\alpha-1}
\quad\text{ $\alpha,\beta>0$}
\]
%
and has the syntax (with a default setting of $\alpha=1$ and $\beta=1$):

\begin{BDef}
\Lcs{psBetaDist}\OptArgs\Largb{x0}\Largb{x1}
\end{BDef}
%


\begin{LTXexample}[pos=t,preset=\centering]
\psset{xunit=10cm,yunit=5cm}
\begin{pspicture*}(-0.1,-0.1)(1.1,2.05)
 \psset{linewidth=1pt}
 \multido{\rbeta=0.25+0.25,\ired=0+5,\rblue=50.0+-2.5}{20}{%
   \psBetaDist[beta=\rbeta,linecolor=red!\ired!blue!\rblue]{0.01}{0.99}}
 \psaxes[Dy=0.2,Dx=0.1]{->}(0,0)(1,2.01)
\end{pspicture*}
\end{LTXexample}


\clearpage
\subsection{Cauchy distribution}
The \Index{Cauchy distribution}, also called the \Index{Lorentz distribution}, is a continuous distribution 
describing resonance behavior. It also describes the distribution of horizontal distances at 
which a line segment tilted at a random angle cuts the $x$-axis. 

The general Cauchy distribution and its cumulative distribution can be written as
\begin{align}
P(x) &= \frac{1}{\pi} \frac{b}{\left(x-m\right)^2+b^2}\\
D(x) &= \frac12 +\frac{1}{\pi} \arctan\left(\frac{x-m}{b}\right)
\end{align}

where \Lkeyword{b} is the half width at half maximum and \Lkeyword{m} is the statistical median.
The macro has the syntax (with a default setting of $m=0$ and $b=1$):

\begin{BDef}
\Lcs{psCauchy}\OptArgs\Largb{x0}\Largb{x1}\\
\Lcs{psCauchyI}\OptArgs\Largb{x0}\Largb{x1}\\
\end{BDef}

\Lcs{psCauchyI} is the integral or the cumulative distribution and often named as $D(x)$.

\begin{LTXexample}[pos=t,preset=\centering]
\psset{xunit=2,yunit=3cm}
\begin{pspicture*}(-3,-0.3)(3.1,2.1)
\psset{linewidth=1pt}
\multido{\rb=0.1+0.2,\rm=0.0+0.2}{4}{%
  \psCauchy[b=\rb,m=\rm,linecolor=red]{-2.5}{2.5}
     \psCauchyI[b=\rb,m=\rm,linecolor=blue]{-2.5}{2.5}}
\psaxes[Dy=0.4,dy=0.4,Dx=0.5,dx=0.5]{->}(0,0)(-3,0)(3,2)
\end{pspicture*}
\end{LTXexample}



\iffalse
\clearpage
\subsection{Bose-Einstein distribution}
A distribution which arises in the study of integer \Index{spin particles} in physics,
\[
P(x)=\frac{x^s}{e^{x-mu}-1}\qquad\text{with $s\in\mathbb{Z}$ and $\mu\in\mathbb{R}}
\]
%
and has the syntax (with a default setting of $s=1$ and $\mu=1$):

\begin{BDef}
\Lcs{psBoseEinsteinDist}\OptArgs\Largb{x0}\Largb{x1}
\end{BDef}
\fi


\clearpage
\subsection{Weibull distribution}

In probability theory and statistics, the Weibull distribution is a continuous probability 
distribution. The probability density function of a 
Weibull random variable $x$ is:

\begin{align}
P(x) &= \alpha\beta^{-\alpha} x^{\alpha-1} e^{-\left(\frac{x}{\beta}\right)^\alpha}\\
D(x) &= 1-e^{-\left(\frac{x}{\beta}\right)^\alpha}
\end{align}

or slightly different as

\begin{align}
P(x) &= \frac{\alpha}{\beta}\,x^{\alpha-1} e^{-\frac{x^\alpha}{\beta}}\\
D(x) &= 1 - e^{-\frac{x^\alpha}{\beta}}
\end{align}

always for $x\in[0;\infty)$.
where $\alpha > 0$ is the shape parameter and $\beta > 0$ is the scale parameter of the distribution. 

$D(x)$ is the cumulative distribution function of the Weibull distribution. The values for
$\alpha$ and $\beta$ are preset to 1, but can be changed in the usual way.

The Weibull distribution is related to a number of other probability distributions; in 
particular, it interpolates between the exponential distribution $(\alpha = 1)$ and the 
Rayleigh distribution $(\alpha = 2)$.

\begin{center}
\psset{unit=2}
\begin{pspicture*}(-0.5,-0.5)(2.6,2.6)
\psaxes{->}(0,0)(2.5,2.5)[$x$,-90][$y$,180]
\multido{\rAlpha=0.5+0.5}{5}{%
  \psWeibull[alpha=\rAlpha]{0}{2.5}
  \psWeibullI[alpha=\rAlpha,linestyle=dashed]{0}{2.4}}
\end{pspicture*}
%
\begin{pspicture*}(-0.5,-0.5)(2.6,2.6)
\psaxes{->}(0,0)(2.5,2.5)[$x$,-90][$y$,180]
\multido{\rAlpha=0.5+0.5,\rBeta=0.2+0.2}{5}{%
  \psWeibull[alpha=\rAlpha,beta=\rBeta]{0}{2.5}
  \psWeibullI[alpha=\rAlpha,beta=\rBeta,linestyle=dashed]{0}{2.4}}
\end{pspicture*}
\end{center}

\begin{lstlisting}
\psset{unit=2}
\begin{pspicture*}(-0.5,-0.5)(2.6,2.6)
\psaxes{->}(0,0)(2.5,2.5)[$x$,-90][$y$,180]
\multido{\rAlpha=0.5+0.5}{5}{%
  \psWeibull[alpha=\rAlpha]{0}{2.5}
  \psWeibullI[alpha=\rAlpha,linestyle=dashed]{0}{2.4}}
\end{pspicture*}
%
\begin{pspicture*}(-0.5,-0.5)(2.6,2.6)
\psaxes{->}(0,0)(2.5,2.5)[$x$,-90][$y$,180]
\multido{\rAlpha=0.5+0.5,\rBeta=0.2+0.2}{5}{%
  \psWeibull[alpha=\rAlpha,beta=\rBeta]{0}{2.5}
  \psWeibullI[alpha=\rAlpha,beta=\rBeta,linestyle=dashed]{0}{2.4}}
\end{pspicture*}
\end{lstlisting}
\psset{unit=1cm}

The starting value for $x$ should always be 0 or greater, if it is
less than 0 then the macro draws a line from (\#1,0) to (0,0) and
starts \Lcs{psWeinbull} with 0.

\clearpage
\subsection{Vasicek distribution}

For a homogenous portfolio of infinite granularity the portfolio loss 
distribution is given by 

\[ 
\mathbb{P}(L(P)<x)=1-\mathcal{N}
  \left(\frac{\mathcal{N}^{-1}(PD)-\sqrt{1-R2}\cdot\mathcal{N}^{-1}(x)}{R} 
  \right)
\]
$L(P)$ denotes the portfolio loss in percent. $pd$ is the uniform default 
probability and $R2$ is the uniform asset correlation.

They are preset to $pd=0.22$ and $R2=0.11$ and can be overwritten in the
usual way. The macro uses the PostScript function norminv from the package 
pst-math
which is loaded by default and also shown in the following example.



\begin{LTXexample}[pos=t]
\psset{xunit=5}
\begin{pspicture}(-0.1,-3)(1.1,4)
\psaxes{->}(0,0)(0,-3)(1.1,4)
\psVasicek[plotpoints=200,linecolor=blue]{0}{0.9999}
\psVasicek[plotpoints=200,linecolor=red,pd=0.2,R2=0.3]{0}{0.9999}
\psplot[plotpoints=200,algebraic,linestyle=dashed]{0}{0.9999}{norminv(x)}
\end{pspicture}
\end{LTXexample}



\clearpage
\section{The Lorenz curve}

The so-called \Index{Lorenz curve} is used in economics to describe inequality in 
wealth or size. The Lorenz curve is a function of the cumulative proportion of 
\textit{ordered individuals} mapped onto the corresponding cumulative proportion 
of their size. Given a sample of n ordered individuals with $x_i^{\prime}$ the size of 
individual $i$ and $x_1^{\prime}<x_2^{\prime}<\cdots<x_n^{\prime}$, then the sample Lorenz curve is 
the \textit{polygon} joining the points $(h/n,L_h/L_n)$, where $h=0, 1, 2,\ldots n, L_0=0$, and 
$L_h=sum_(i=1)^(h)x_i^{\prime}$. 

\begin{BDef}
\LcsStar{psLorenz}\OptArgs\Largb{data file}
\end{BDef}


\begin{LTXexample}[pos=t,preset=\centering]
\psset{lly=-6mm,llx=-5mm}
\psgraph[Dx=0.2,Dy=0.2,axesstyle=frame](0,0)(1,1){6cm}{6cm}
\psline[linestyle=dashed](1,1)
\psLorenz*[linecolor=blue!30,linewidth=1.5pt]{0.50 0.10 0.3 0.09 0.01 }
\psLorenz[linecolor=blue!30,plotstyle=bezier]{0.50 0.10 0.3 0.09 0.01 }
\psLorenz[linecolor=red,linewidth=1.5pt]{0.50 0.10 0.3 0.09 0.01 }
\endpsgraph
\end{LTXexample}

There exists an optional argument \Lkeyword{Gini} for the output of the \Index{Gini coefficient}.
It is by default set to \false. With \true the value is caculated and printed below the
origin of the coordinate system. 

\begin{LTXexample}[pos=t,preset=\centering]
\psset{lly=-13mm,llx=-5mm}
\psgraph[Dx=0.2,Dy=0.2,axesstyle=frame](0,0)(1,1){6cm}{6cm}
\psline[linestyle=dashed](1,1)
\psLorenz[linewidth=1.5pt,Gini]{0.025 0.275 0.2 0.270 0.230}
\psLorenz[plotstyle=dots,dotstyle=square,dotscale=1.5]{0.025 0.275 0.2 0.270 0.230}
\endpsgraph
\end{LTXexample}

\clearpage
\section{\nxLcs{psLame} -- Lam\'e Curve, a superellipse}
A superellipse is a curve with Cartesian equation
%	
\begin{align}
\left|\frac{x}{a}\right|^r + \left|\frac{y}{b}\right|^r & =1
\end{align}
%
first discussed in 1818 by Gabriel Lam\'e (1795--1870)%
\footnote{Lam\'e worked on a wide variety of different topics. 
His work on differential geometry and contributions to Fermat's Last Theorem 
are important. He proved the theorem for $n = 7$ in 1839.}. 
A superellipse may be described parametrically by
%
\begin{align}
x = a\cdot\cos^{\frac{2}{r}} t\\
y = b\cdot\sin^{\frac{2}{r}} t
\end{align}
%			
\Index{Superellipses} with $a=b$ are also known as \Index{Lam\'e} curves or Lam\'e ovals and
the restriction to $r>2$ is sometimes also made. The following 
table summarizes a few special cases. \Index{Piet Hein} used $\frac{5}{2}$ with a number of different 
$\frac{a}{b}$ ratios for various of his projects. For example, he used $\frac{a}{b}=\frac{6}{5}$ 
for Sergels Torg 
(Sergel's Square) in Stockholm, and $\frac{a}{b}=\frac{3}{2}$ for his table.

\begin{center}
\begin{tabular}{@{}llm{1.5cm}@{}}
r & curve type & example\\\hline
$\frac{2}{3}$ &	(squashed) astroid 
  & \pspicture(-.5,-.5)(.5,.5)\psLame[radiusA=.5,radiusB=.5]{0.6667}\endpspicture\\
1             & (squashed) diamond
  & \pspicture(-.5,-.5)(.5,.5)\psLame[radiusA=.5,radiusB=.5]{1}\endpspicture\\
2	      & ellipse
  & \pspicture(-.5,-.5)(.5,.5)\psLame[radiusA=.5,radiusB=.5]{2}\endpspicture\\
$\frac{5}{2}$ & Piet Hein's ,,superellipse``
  & \pspicture(-.5,-.5)(.5,.5)\psLame[radiusA=.5,radiusB=.5]{2.5}\endpspicture
\end{tabular}
\end{center}

If $r$ is a rational, then a \Index{superellipse} is algebraic. However, for irrational $r$, 
it is transcendental. For even integers $r=n$, the curve becomes closer to a 
rectangle as $n$ increases. The syntax of the \Lcs{psLame} macro is:

\begin{BDef}
\Lcs{psLame}\OptArgs\Largb{r}
\end{BDef}

It is internally plotted as a \Index{parametric plot} with $0\le\alpha\le360$. Available keywords
are \Lkeyword{radiusA} and \Lkeyword{radiusB}, both are preset to 1, but can have any valid value
and unit.

\bgroup
\begin{LTXexample}[pos=t,preset=\centering]
\definecolorseries{col}{rgb}{last}{red}{blue}
\resetcolorseries[41]{col}
\psset{unit=.5}
\pspicture(-9,-9)(9,9)
  \psaxes[Dx=2,Dy=2,tickstyle=bottom,ticksize=2pt]{->}(0,0)(-9,-9)(9,9)
  \multido{\rA=0.2+0.1,\iA=0+1}{40}{%
    \psLame[radiusA=8,radiusB=7,linecolor={col!![\iA]},linewidth=.5pt]{\rA}}
\endpspicture
\end{LTXexample}
\egroup


\clearpage
\section{\nxLcs{psThomae} -- the popcorn function}

\Index{Thomae's function}, also known as the \Index{popcorn function}, 
the \Index{raindrop function}, the \Index{ruler function} or the 
\Index{Riemann function}, is a modification of the \Index{Dirichlet} function. 
This real-valued function $f(x)$ is defined as follows:
%
\[ f(x)=\begin{cases} 
	    \frac{1}{q}\mbox{ if }x=\frac{p}{q}\mbox{ is a rational number}\\ 
	    0\mbox{ if }x\mbox{ is irrational} 
	\end{cases}
\]
%
It is assumed here that $\mathop{gcd}(p,q) = 1$ and $q > 0$ so that the function is well-defined 
and nonnegative. The syntax is:

\begin{BDef}
\Lcs{psThomae}\OptArgs\Largr{x0,x1}\Largb{points}
\end{BDef}

\verb+(x0,x1)+ is the plotted interval, both values must be grater zero and $x_1>x_0$. 
The plotted number of points is the third parameter. 

\begin{LTXexample}[width=6cm,wide=false]
\psset{unit=4cm}
\begin{pspicture}(-0.1,-0.2)(2.5,1.15)
    \psaxes{->}(0,0)(2.5,1.1)
    \psThomae[dotsize=2.5pt,linecolor=red](0,2){300}
\end{pspicture}
\end{LTXexample}


\clearpage
\section{\nxLcs{psplotImp} -- plotting implicit defined functions}
For a given area, the macro calculates in a
first step row by row for every pixel (1pt) the function $f(x,y)$ and checks for a
changing of the value from $f(x,y)<0$ to $f(x,y)>0$ or vice versa. If this happens,
then the pixel must be  part of the curve of the function $f(x,y)=0$. In a second step the same is
done column by column. This may take some time because an area of $400\times 300$
pixel needs $120$ thousand calculations of the function value. The user still defines
this area in his own coordinates, the translation into pixel (pt) is done internally by the
macro itself.
The only special keyword is \Lkeyword{stepFactor} which is preset to 0.67 and controls the horizontal
and vertical step width.

\begin{BDef}
\Lcs{psplotImp}\OptArgs\Largr{xMin,yMin}\Largr{xMax,yMax}\OptArg{PS code}\Largb{function f(x,y)}
\end{BDef}

The function must be of $f(x,y)=0$ and described in \PS code, or alternatively with
the option \Lkeyword{algebraic} (\LPack{pstricks-add}) in an algebraic form. No other value names than $x$ and $y$
are possible. In general, a starred \Lenv{pspicture*} environment maybe a good choice here.

\medskip
\noindent
\begin{tabularx}{\linewidth}{!{\color{Orange!85!Red}\vrule width 5pt} X @{}} 
The given area for \Lcs{psplotImp} should be \textbf{greater} than the given \Lenv{pspicture} area 
(see examples).
\end{tabularx}

\begin{LTXexample}[preset=\centering]
\begin{pspicture*}(-3,-3.2)(3.5,3.5)
\psaxes{->}(0,0)(-3,-3)(3.2,3)%
\psplotImp[linewidth=2pt,linecolor=red](-5,-2.1)(5,2.1){ x dup mul y dup mul add 4 sub }
\uput[45](0,2){$x^2+y^2-4=0$}
\psplotImp[linewidth=2pt,linecolor=blue,algebraic](-5,-3)(4,2.4){ (x+1)^2+y^2-4 }
\end{pspicture*}
\end{LTXexample}

\begin{LTXexample}[preset=\centering]
\begin{pspicture*}(-3,-2.2)(3.5,2.5)
\psaxes{->}(0,0)(-3,-2)(3.2,2)%
\psplotImp[linewidth=2pt,linecolor=blue](-5,-2.2)(5,2.4){%
      /xqu x dup mul def
      /yqu y dup mul def
      xqu yqu add dup mul 2 dup add 2 mul xqu yqu sub mul sub }
\uput*[0](-3,2){$\left(x^2+y^2\right)^2-8(x^2-y^2)=0$}
\psplotImp[linewidth=1pt,linecolor=red,algebraic](-5,-2.2)(5,2.4){% Lemniskate a =2
    (x^2+y^2)^2-4*(x^2-y^2) }
\end{pspicture*}
\end{LTXexample}

\begin{LTXexample}[preset=\centering]
\begin{pspicture*}(-3,-3.2)(3.5,3.5)
\psaxes{->}(0,0)(-3,-3)(3.2,3)%
\psplotImp[linewidth=2pt,linecolor=green](-6,-6)(4,2.4){%
	 x 3 exp y 3 exp add 4 x y mul mul sub } 
\uput*[45](-2.5,2){$\left(x^3+y^3\right)-4xy=0$}
\end{pspicture*}
\end{LTXexample}


\begin{LTXexample}[preset=\centering]
\begin{pspicture*}(-5,-3.2)(5.5,4.5)
\psaxes{->}(0,0)(-5,-3)(5.2,4)%
\psplotImp[algebraic,linecolor=red](-6,-4)(5,4){ y*cos(x*y)-0.2 }
\psplotImp[algebraic,linecolor=blue](-6,-4)(5,4){ y*cos(x*y)-1.2 }
\end{pspicture*}
\end{LTXexample}



Using the \Lkeyword{polarplot} option implies using the variables $r$ and $phi$ for describing
the function, $y$ and $x$ are not respected in this case. Using the \Lkeyword{algebraic} option
for polar plots are also possible (see next example).

\begin{LTXexample}[preset=\centering]
\begin{pspicture*}(-3,-2.5)(3.75,2.75)\psaxes{->}(0,0)(-3,-2.5)(3.2,2.5)%
\psplotImp[linewidth=2pt,linecolor=cyan,polarplot](-6,-3)(4,2.4){ r 2 sub }% circle r=2
\uput*[45](0.25,2){$f(r,\phi)=r-2=0$}
\psplotImp[polarplot,algebraic](-6,-3)(4,2.4){ r-1 }% circle r=1
\end{pspicture*}
\end{LTXexample}

\begin{LTXexample}[preset=\centering]
\begin{pspicture*}(-5,-2.2)(5.5,3.5)
\pscircle(0,0){1}% 
\psaxes{->}(0,0)(-5,-2)(5.2,3)%
\multido{\rA=0.01+0.2}{5}{%
\psplotImp[linewidth=1pt,linecolor=blue,polarplot](-6,-6)(5,2.4){%
	 r dup mul 1.0 r div sub phi sin dup mul mul \rA\space sub }}%
\uput*[45](0,2){$f(r,\phi)=\left(r^2-\frac{1}{r}\right)\cdot\sin^2\phi=0$}
\end{pspicture*}
\end{LTXexample}

\begin{LTXexample}[preset=\centering]
\begin{pspicture*}(-4,-3.2)(4.5,4.5)
\psaxes{->}(0,0)(-4,-3)(4.2,4)%
\psplotImp[algebraic,polarplot,linecolor=red](-5,-4)(5,4){ r+cos(phi/r)-2 }
\end{pspicture*}
\end{LTXexample}


The data of an implicit plot can be written into an external file for further purposes.
Use the optional argument \Lkeyword[pstricks-add]{saveData} to write the $x|y$ values
into the file \nxLcs{jobname.data}. The file name can be changed with the keyword {\Lkeyword[pstricks-add]{filename}.
When running a \TeX\ file from within a GUI it may be possible that you get a writeaccess error from GhostScript, because
it prevents writing into a file when called from another program. In this case run GhostScript on the \PS-output from
the command line.

\psset{mathLabel}
\begin{LTXexample}[preset=\centering]
\begin{pspicture*}(-3,-3)(3,3)
  \psaxes[linewidth=0.25pt,
   xlabelPos=top,
   labelFontSize=\scriptscriptstyle,
   labelsep=2pt,
   ticksize=0.05]{<->}(0,0)(-2,-1.75)(2,2)[x,0][y,90]
  \psplotImp[linecolor=red,linewidth=1pt,stepFactor=0.2,saveData,
     algebraic](-2.5,-1.75)(2.5,2.5){x^2+(5*y/4-sqrt(abs(x)))^2-2.5}
\end{pspicture*}
\end{LTXexample}

The values are saved pairwise in an array, e.\,g.:
\begin{verbatim}
...
[
-1.53237 0.695058
-1.53237 1.29957
]
[
-1.52534 0.666941
-1.52534 1.32065
]
...
\end{verbatim}

In one array all $y$ values for the same $x$ value are stored.

\iffalse
The data then can be read back to get a continous line of the plot.

\begin{LTXexample}[preset=\centering]
\readdata[nStep=20]{\data}{\jobname.data}
\begin{pspicture*}(-3,-3)(3,3)
  \psaxes[linewidth=0.25pt,
   xlabelPos=top,
   labelFontSize=\scriptscriptstyle,
   labelsep=2pt,
   ticksize=0.05]{<->}(0,0)(-2,-1.75)(2,2)[x,0][y,90]
  \pslistplot[linecolor=red,linewidth=1pt,plotstyle=curve]{\data}
\end{pspicture*}
\end{LTXexample}
\fi


\clearpage
\section{\nxLcs{psVolume} -- Rotating functions around the x-axis}

This macro shows the behaviour of a \Index{rotated function} around the x-axis.

\begin{BDef}
\Lcs{psVolume}\OptArgs\Largr{xMin,xMax}\Largb{steps}\Largb{function $f(x)$}
\end{BDef}

$f(x)$ has to be described as usual for the macro \Lcs{psplot}.

\makebox[\linewidth]{%
\begin{pspicture}(-0.5,-2)(5,2.5)
\psaxes{->}(0,0)(0,-2)(3,2.5)
\psVolume[fillstyle=solid,fillcolor=magenta!30](0,4){1}{x sqrt}
\psline{->}(4,0)(5,0)
\end{pspicture}
%
\begin{pspicture}(-0.5,-2)(5,2.5)
\psaxes{->}(0,0)(0,-2)(3,2.5)
\psVolume[fillstyle=solid,fillcolor=red!40](0,4){2}{x sqrt}
\psline{->}(4,0)(5,0)
\end{pspicture}
%
\begin{pspicture}(-0.5,-2)(5,2.5)
\psaxes{->}(0,0)(0,-2)(3,2.5)
\psVolume[fillstyle=solid,fillcolor=blue!40](0,4){4}{x sqrt}
\psline{->}(4,0)(5,0)
\end{pspicture}
}

\makebox[\linewidth]{%
\begin{pspicture}(-0.5,-2)(5,2.5)
\psaxes{->}(0,0)(0,-2)(3,2.5)
\psVolume[fillstyle=solid,fillcolor=green!40](0,4){8}{x sqrt}
\psline{->}(4,0)(5,0)
\end{pspicture}
%
\begin{pspicture}(-0.5,-2)(5,2.5)
\psaxes{->}(0,0)(0,-2)(3,2.5)
\psVolume[fillstyle=solid,fillcolor=yellow!40](0,4){16}{x sqrt}
\psline{->}(4,0)(5,0)
\end{pspicture}
%
\begin{pspicture}(-0.5,-2)(5,2.5)
\psaxes{->}(0,0)(0,-2)(3,2.5)
\psVolume[fillstyle=solid,fillcolor=cyan!40](0,4){32}{x sqrt}
\psline{->}(4,0)(5,0)
\end{pspicture}
}

\begin{lstlisting} 
\begin{pspicture}(-0.5,-2)(5,2.5)
\psaxes{->}(0,0)(0,-2)(3,2.5)
\psVolume[fillstyle=solid,fillcolor=magenta!30](0,4){1}{x sqrt}
\psline{->}(4,0)(5,0)
\end{pspicture}
%
\begin{pspicture}(-0.5,-2)(5,2.5)
\psaxes{->}(0,0)(0,-2)(3,2.5)
\psVolume[fillstyle=solid,fillcolor=red!40](0,4){2}{x sqrt}
\psline{->}(4,0)(5,0)
\end{pspicture}
%
\begin{pspicture}(-0.5,-2)(5,2.5)
\psaxes{->}(0,0)(0,-2)(3,2.5)
\psVolume[fillstyle=solid,fillcolor=blue!40](0,4){4}{x sqrt}
\psline{->}(4,0)(5,0)
\end{pspicture}

\begin{pspicture}(-0.5,-2)(5,2.5)
\psaxes{->}(0,0)(0,-2)(3,2.5)
\psVolume[fillstyle=solid,fillcolor=green!40](0,4){8}{x sqrt}
\psline{->}(4,0)(5,0)
\end{pspicture}
%
\begin{pspicture}(-0.5,-2)(5,2.5)
\psaxes{->}(0,0)(0,-2)(3,2.5)
\psVolume[fillstyle=solid,fillcolor=yellow!40](0,4){16}{x sqrt}
\psline{->}(4,0)(5,0)
\end{pspicture}
%
\begin{pspicture}(-0.5,-2)(5,2.5)
\psaxes{->}(0,0)(0,-2)(3,2.5)
\psVolume[fillstyle=solid,fillcolor=cyan!40](0,4){32}{x sqrt}
\psline{->}(4,0)(5,0)
\end{pspicture}
\end{lstlisting} 


\psset{xunit=2}
\makebox[\linewidth]{%
\begin{pspicture}(-0.5,-4)(3,4)
  \psaxes{->}(0,0)(0,-4)(3,4)
  \psVolume[fillstyle=solid,fillcolor=cyan!40](0,1){4}{x}
  \psVolume[fillstyle=solid,fillcolor=yellow!40](1,2){4}{x dup mul}
  \psline(2,0)(3,0)
\end{pspicture}
%
\begin{pspicture}(-0.5,-4)(3,4)
  \psaxes{->}(0,0)(0,-4)(3,4)
  \psVolume[fillstyle=solid,fillcolor=cyan!40](0,1){20}{x}
  \psVolume[fillstyle=solid,fillcolor=yellow!40](1,2){20}{x dup mul}
  \psline(2,0)(3,0)
\end{pspicture}
}
\begin{lstlisting} 
\psset{xunit=2}
\begin{pspicture}(-0.5,-4)(3,4)
  \psaxes{->}(0,0)(0,-4)(3,4)
  \psVolume[fillstyle=solid,fillcolor=cyan!40](0,1){4}{x}
  \psVolume[fillstyle=solid,fillcolor=yellow!40](1,2){4}{x dup mul}
  \psline(2,0)(3,0)
\end{pspicture}
%
\begin{pspicture}(-0.5,-4)(3,4)
  \psaxes{->}(0,0)(0,-4)(3,4)
  \psVolume[fillstyle=solid,fillcolor=cyan!40](0,1){20}{x}
  \psVolume[fillstyle=solid,fillcolor=yellow!40](1,2){20}{x dup mul}
  \psline(2,0)(3,0)
\end{pspicture}
\end{lstlisting} 

\clearpage

\section{Examples}

\begin{LTXexample}[preset=\centering]
\psset{xunit=0.5cm,yunit=20cm,arrowscale=1.5}
\begin{pspicture}(-1,-0.1)(21,0.2)
\psChiIIDist[linewidth=1pt,nue=5]{0.01}{19.5}
\psaxes[labels=none,ticks=none]{->}(20,0.2)
\pscustom[fillstyle=solid,fillcolor=red!30]{%
  \psChiIIDist[linewidth=1pt,nue=5]{8}{19.5}%
  \psline(20,0)(8,0)}
\end{pspicture}
\end{LTXexample}

\clearpage
\section{List of all optional arguments for \texttt{pst-func}}

\xkvview{family=pst-func,columns={key,type,default}}




\bgroup
\raggedright
\nocite{*}
\bibliographystyle{plain}
\bibliography{pst-func-doc}
\egroup

\printindex



\end{document}


