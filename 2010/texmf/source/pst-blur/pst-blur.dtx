%\iffalse -*-mode:Latex;tex-command:"latex *;dvips -D600 pst-blur -o"-*- \fi
%\iffalse
%% $Id: pst-blur.dtx,v 2.0 2005/09/08 09:48:33 giese Exp $
%%
%% Copyright 1998 Martin Giese, Martin.Giese@oeaw.ac.at
%%           2005 Herbert Voss, voss@pstricks.de
%% 
%% This file is under the LaTeX Project Public License 
%% See CTAN archives in directory macros/latex/base/lppl.txt.
%%
%% DESCRIPTION:
%%   `pst-blur' is a PSTricks package for blurred shadows
%%
%  
%\fi
%\iffalse
%<*driver>
\NeedsTeXFormat{LaTeX2e}
\documentclass{ltxdoc}
\usepackage{pst-text}
\usepackage{pst-node}
\usepackage{url}
\usepackage{pst-blur}
\AtBeginDocument{
%  \OnlyDescription % comment out for implementation details
  \EnableCrossrefs
  \RecordChanges
  \CodelineIndex}
\AtEndDocument{
  \PrintChanges
  \PrintIndex}
\hbadness=7000            % Over and under full box warnings
\hfuzz=3pt
\MakeShortVerb{\|}
\newcommand\file[1]{\texttt{#1}}
\begin{document}
\DocInput{pst-blur.dtx}
\end{document}
%</driver>
%\fi
%
% \CharacterTable
%  {Upper-case    \A\B\C\D\E\F\G\H\I\J\K\L\M\N\O\P\Q\R\S\T\U\V\W\X\Y\Z
%   Lower-case    \a\b\c\d\e\f\g\h\i\j\k\l\m\n\o\p\q\r\s\t\u\v\w\x\y\z
%   Digits        \0\1\2\3\4\5\6\7\8\9
%   Exclamation   \!     Double quote  \"     Hash (number) \#
%   Dollar        \$     Percent       \%     Ampersand     \&
%   Acute accent  \'     Left paren    \(     Right paren   \)
%   Asterisk      \*     Plus          \+     Comma         \,
%   Minus         \-     Point         \.     Solidus       \/
%   Colon         \:     Semicolon     \;     Less than     \<
%   Equals        \=     Greater than  \>     Question mark \?
%   Commercial at \@     Left bracket  \[     Backslash     \\
%   Right bracket \]     Circumflex    \^     Underscore    \_
%   Grave accent  \`     Left brace    \{     Vertical bar  \|
%   Right brace   \}     Tilde         \~}
% \CheckSum{152}
%
%
% \changes{v2.00}{2005/09/08}{%
%	using the extended pst-xkey instead of the old pst-key package; 
%	creating a dtx file;
%	new \LaTeX\ wrapper file (hv)}
% \changes{v1.80}{2001/02/16}{First public release. (mg)}
%
% \DoNotIndex{\!,\",\#,\$,\%,\&,\',\(,\+,\*,\,,\-,\.,\/,\:,\;,\<,\=,\>,\?}
% \DoNotIndex{\@,\@B,\@K,\@cTq,\@f,\@fPl,\@ifnextchar,\@nameuse,\@oVk}
% \DoNotIndex{\[,\\,\],\^,\_,\ }
% \DoNotIndex{\^,\\^,\\\^,$\^$,$\\^$,$\\^$}
% \DoNotIndex{\0,\2,\4,\5,\6,\7,\8,}
% \DoNotIndex{\A,\a}
% \DoNotIndex{\B,\b,\Bc,\begin,\Bq,\Bqc}
% \DoNotIndex{\C,\c,\catcode,\cJA,\CodelineIndex,\csname}
% \DoNotIndex{\D,\def,\define@key,\Df,\divide,\DocInput,\documentclass,\pst@addfams}
% \DoNotIndex{\eCN,\edef,\else,\eHd,\eMcj,\EnableCrossrefs,\end,\endcsname}
% \DoNotIndex{\endCenterExample,\endExample,\endinput,\endpsclip}
% \DoNotIndex{\PrintIndex,\PrintChanges,\ProvidesFile}
% \DoNotIndex{\endpspicture,\endSideBySideExample,\Example}
% \DoNotIndex{\F,\f,\FdUrr,\fi,\filedate,\fileversion,\FV@Environment}
% \DoNotIndex{\FV@UseKeyValues,\FV@XRightMargin,\FVB@Example,\fvset}
% \DoNotIndex{\G,\g,\GetFileInfo,\gr,\GradientLoaded,\gsFKrbK@o,\gsj,\gsOX}
% \DoNotIndex{\hbadness,\hfuzz,\HLEmphasize,\HLMacro,\HLMacro@i}
% \DoNotIndex{\HLReverse,\HLReverse@i,\hqcu,\HqY}
% \DoNotIndex{\I,\i,\ifx,\input,\Ir,\IU}
% \DoNotIndex{\j,\jl,\JT,\JVodH}
% \DoNotIndex{\K,\k,\kfSlL}
% \DoNotIndex{\L,\let}
% \DoNotIndex{\message,\mHNa,\mIU}
% \DoNotIndex{\N,\nB,\newcmykcolor,\newdimen,\newif,\nW}
% \DoNotIndex{\O,\oCDJDo,\ocQhVI,\OnlyDescription,\oRKJ}
% \DoNotIndex{\P,\p,\ProvidesPackage,\psframe,\pslinewidth,\psset}
% \DoNotIndex{\PstAtCode,\PSTricksLoaded}
% \DoNotIndex{\q,\Qr,\qssRXq,\qu,\qXjFQp,\qYL}
% \DoNotIndex{\R,\r,\RecordChanges,\relax,\RlaYI,\rN,\Rp,\rp,\RPDXNn,\rput}
% \DoNotIndex{\S,\scalebox,\SgY,\SideBySide@Example,\SideBySideExample}
% \DoNotIndex{\SgY,\sk,\Sp,\space,\sZb}
% \DoNotIndex{\T,\the,\tw@}
% \DoNotIndex{\u,\UiSWGEf@,\uJi,\usepackage,\uVQdMM,\UYj}
% \DoNotIndex{\VerbatimEnvironment,\VerbatimInput,\VrC@}
% \DoNotIndex{\WhZ,\WjKCYb,\WNs}
% \DoNotIndex{\XkN,\XW}
% \DoNotIndex{\Z,\ZCM,\Ze}
% \DoNotIndex{\addtocounter,\advance,\alph,\arabic,\AtBeginDocument,\AtEndDocument}
% \DoNotIndex{\AtEndOfPackage,\begingroup,\bfseries,\bgroup,\box,\csname}
% \DoNotIndex{\else,\endcsname,\endgroup,\endinput,\expandafter,\fi}
% \DoNotIndex{\TeX,\z@,\p@,\@one,\xdef,\thr@@,\string,\sixt@@n,\reset,\or,\multiply,\repeat,\RequirePackage}
% \DoNotIndex{\@cclvi,\@ne,\@ehpa,\@nil,\copy,\dp,\global,\hbox,\hss,\ht,\ifodd,\ifdim,\ifcase,\kern}
% \DoNotIndex{\chardef,\loop,\leavevmode,\ifnum,\lower,\next,\wd,\setbox}
% \setcounter{IndexColumns}{2}
%
% \title{\textsf{pst-blur} package \\ version \fileversion}
% \author{Martin Giese\thanks{\protect\url{Martin.Giese@oeaw.ac.at}}}
% \date{\filedate}
% \maketitle
%
%\section{Introduction}
% The ability to paint shadows on arbitrary shapes is a standard 
% feature of PSTricks.  However, these shadows are always `hard':\\
%{\psset{unit=0.6cm,linewidth=0.5pt}
% \begin{pspicture}(-5,0)(10,4)
%   \psset{shadow=true,shadowcolor=gray,shadowsize=0.15cm}
%   \psframe(1,1)(3,3)
%   \pscircle(6,2){1}
%   \rput(9,2){\pscharpath{\fontfamily{ptm}\fontsize{40}{40}\selectfont A}}
%  \end{pspicture}
%}\\
% The |pst-blur| package provides  blurred shadows
% for closed shapes drawn with PSTricks:\\
%{\psset{unit=0.6cm,linewidth=0.5pt,blur=true,blurradius=0.07cm}
% \begin{pspicture}(-5,0)(10,4)
%   \psset{shadow=true,shadowcolor=gray,shadowsize=0.15cm}
%   \psframe(1,1)(3,3)
%   \pscircle(6,2){1}
%   \rput(9,2){\pscharpath{\fontfamily{ptm}\fontsize{40}{40}\selectfont A}}
%  \end{pspicture}
%}\\
% It also provides a new
% box command |\psblurbox|, which is similar to |\psshadowbox|,
% but gives the box a blurred shadow.
% \medskip
%
%The new graphics parameters and macros provided by the package
% are described in
%section 2 of this document.  Section 3, if present, documents the 
%implementation consisting of a generic \TeX\ file and a PostScript
%header for the |dvi|-to-PostScript converter.  You can get section 3
%by calling \LaTeX\ as follows on most relevant systems:
%\begin{verbatim}
%latex '\AtBeginDocument{\AlsoImplementation}%\iffalse -*-mode:Latex;tex-command:"latex *;dvips -D600 pst-blur -o"-*- \fi
%\iffalse
%% $Id: pst-blur.dtx,v 2.0 2005/09/08 09:48:33 giese Exp $
%%
%% Copyright 1998 Martin Giese, Martin.Giese@oeaw.ac.at
%%           2005 Herbert Voss, voss@pstricks.de
%% 
%% This file is under the LaTeX Project Public License 
%% See CTAN archives in directory macros/latex/base/lppl.txt.
%%
%% DESCRIPTION:
%%   `pst-blur' is a PSTricks package for blurred shadows
%%
%  
%\fi
%\iffalse
%<*driver>
\NeedsTeXFormat{LaTeX2e}
\documentclass{ltxdoc}
\usepackage{pst-text}
\usepackage{pst-node}
\usepackage{url}
\usepackage{pst-blur}
\AtBeginDocument{
%  \OnlyDescription % comment out for implementation details
  \EnableCrossrefs
  \RecordChanges
  \CodelineIndex}
\AtEndDocument{
  \PrintChanges
  \PrintIndex}
\hbadness=7000            % Over and under full box warnings
\hfuzz=3pt
\MakeShortVerb{\|}
\newcommand\file[1]{\texttt{#1}}
\begin{document}
\DocInput{pst-blur.dtx}
\end{document}
%</driver>
%\fi
%
% \CharacterTable
%  {Upper-case    \A\B\C\D\E\F\G\H\I\J\K\L\M\N\O\P\Q\R\S\T\U\V\W\X\Y\Z
%   Lower-case    \a\b\c\d\e\f\g\h\i\j\k\l\m\n\o\p\q\r\s\t\u\v\w\x\y\z
%   Digits        \0\1\2\3\4\5\6\7\8\9
%   Exclamation   \!     Double quote  \"     Hash (number) \#
%   Dollar        \$     Percent       \%     Ampersand     \&
%   Acute accent  \'     Left paren    \(     Right paren   \)
%   Asterisk      \*     Plus          \+     Comma         \,
%   Minus         \-     Point         \.     Solidus       \/
%   Colon         \:     Semicolon     \;     Less than     \<
%   Equals        \=     Greater than  \>     Question mark \?
%   Commercial at \@     Left bracket  \[     Backslash     \\
%   Right bracket \]     Circumflex    \^     Underscore    \_
%   Grave accent  \`     Left brace    \{     Vertical bar  \|
%   Right brace   \}     Tilde         \~}
% \CheckSum{152}
%
%
% \changes{v2.00}{2005/09/08}{%
%	using the extended pst-xkey instead of the old pst-key package; 
%	creating a dtx file;
%	new \LaTeX\ wrapper file (hv)}
% \changes{v1.80}{2001/02/16}{First public release. (mg)}
%
% \DoNotIndex{\!,\",\#,\$,\%,\&,\',\(,\+,\*,\,,\-,\.,\/,\:,\;,\<,\=,\>,\?}
% \DoNotIndex{\@,\@B,\@K,\@cTq,\@f,\@fPl,\@ifnextchar,\@nameuse,\@oVk}
% \DoNotIndex{\[,\\,\],\^,\_,\ }
% \DoNotIndex{\^,\\^,\\\^,$\^$,$\\^$,$\\^$}
% \DoNotIndex{\0,\2,\4,\5,\6,\7,\8,}
% \DoNotIndex{\A,\a}
% \DoNotIndex{\B,\b,\Bc,\begin,\Bq,\Bqc}
% \DoNotIndex{\C,\c,\catcode,\cJA,\CodelineIndex,\csname}
% \DoNotIndex{\D,\def,\define@key,\Df,\divide,\DocInput,\documentclass,\pst@addfams}
% \DoNotIndex{\eCN,\edef,\else,\eHd,\eMcj,\EnableCrossrefs,\end,\endcsname}
% \DoNotIndex{\endCenterExample,\endExample,\endinput,\endpsclip}
% \DoNotIndex{\PrintIndex,\PrintChanges,\ProvidesFile}
% \DoNotIndex{\endpspicture,\endSideBySideExample,\Example}
% \DoNotIndex{\F,\f,\FdUrr,\fi,\filedate,\fileversion,\FV@Environment}
% \DoNotIndex{\FV@UseKeyValues,\FV@XRightMargin,\FVB@Example,\fvset}
% \DoNotIndex{\G,\g,\GetFileInfo,\gr,\GradientLoaded,\gsFKrbK@o,\gsj,\gsOX}
% \DoNotIndex{\hbadness,\hfuzz,\HLEmphasize,\HLMacro,\HLMacro@i}
% \DoNotIndex{\HLReverse,\HLReverse@i,\hqcu,\HqY}
% \DoNotIndex{\I,\i,\ifx,\input,\Ir,\IU}
% \DoNotIndex{\j,\jl,\JT,\JVodH}
% \DoNotIndex{\K,\k,\kfSlL}
% \DoNotIndex{\L,\let}
% \DoNotIndex{\message,\mHNa,\mIU}
% \DoNotIndex{\N,\nB,\newcmykcolor,\newdimen,\newif,\nW}
% \DoNotIndex{\O,\oCDJDo,\ocQhVI,\OnlyDescription,\oRKJ}
% \DoNotIndex{\P,\p,\ProvidesPackage,\psframe,\pslinewidth,\psset}
% \DoNotIndex{\PstAtCode,\PSTricksLoaded}
% \DoNotIndex{\q,\Qr,\qssRXq,\qu,\qXjFQp,\qYL}
% \DoNotIndex{\R,\r,\RecordChanges,\relax,\RlaYI,\rN,\Rp,\rp,\RPDXNn,\rput}
% \DoNotIndex{\S,\scalebox,\SgY,\SideBySide@Example,\SideBySideExample}
% \DoNotIndex{\SgY,\sk,\Sp,\space,\sZb}
% \DoNotIndex{\T,\the,\tw@}
% \DoNotIndex{\u,\UiSWGEf@,\uJi,\usepackage,\uVQdMM,\UYj}
% \DoNotIndex{\VerbatimEnvironment,\VerbatimInput,\VrC@}
% \DoNotIndex{\WhZ,\WjKCYb,\WNs}
% \DoNotIndex{\XkN,\XW}
% \DoNotIndex{\Z,\ZCM,\Ze}
% \DoNotIndex{\addtocounter,\advance,\alph,\arabic,\AtBeginDocument,\AtEndDocument}
% \DoNotIndex{\AtEndOfPackage,\begingroup,\bfseries,\bgroup,\box,\csname}
% \DoNotIndex{\else,\endcsname,\endgroup,\endinput,\expandafter,\fi}
% \DoNotIndex{\TeX,\z@,\p@,\@one,\xdef,\thr@@,\string,\sixt@@n,\reset,\or,\multiply,\repeat,\RequirePackage}
% \DoNotIndex{\@cclvi,\@ne,\@ehpa,\@nil,\copy,\dp,\global,\hbox,\hss,\ht,\ifodd,\ifdim,\ifcase,\kern}
% \DoNotIndex{\chardef,\loop,\leavevmode,\ifnum,\lower,\next,\wd,\setbox}
% \setcounter{IndexColumns}{2}
%
% \title{\textsf{pst-blur} package \\ version \fileversion}
% \author{Martin Giese\thanks{\protect\url{Martin.Giese@oeaw.ac.at}}}
% \date{\filedate}
% \maketitle
%
%\section{Introduction}
% The ability to paint shadows on arbitrary shapes is a standard 
% feature of PSTricks.  However, these shadows are always `hard':\\
%{\psset{unit=0.6cm,linewidth=0.5pt}
% \begin{pspicture}(-5,0)(10,4)
%   \psset{shadow=true,shadowcolor=gray,shadowsize=0.15cm}
%   \psframe(1,1)(3,3)
%   \pscircle(6,2){1}
%   \rput(9,2){\pscharpath{\fontfamily{ptm}\fontsize{40}{40}\selectfont A}}
%  \end{pspicture}
%}\\
% The |pst-blur| package provides  blurred shadows
% for closed shapes drawn with PSTricks:\\
%{\psset{unit=0.6cm,linewidth=0.5pt,blur=true,blurradius=0.07cm}
% \begin{pspicture}(-5,0)(10,4)
%   \psset{shadow=true,shadowcolor=gray,shadowsize=0.15cm}
%   \psframe(1,1)(3,3)
%   \pscircle(6,2){1}
%   \rput(9,2){\pscharpath{\fontfamily{ptm}\fontsize{40}{40}\selectfont A}}
%  \end{pspicture}
%}\\
% It also provides a new
% box command |\psblurbox|, which is similar to |\psshadowbox|,
% but gives the box a blurred shadow.
% \medskip
%
%The new graphics parameters and macros provided by the package
% are described in
%section 2 of this document.  Section 3, if present, documents the 
%implementation consisting of a generic \TeX\ file and a PostScript
%header for the |dvi|-to-PostScript converter.  You can get section 3
%by calling \LaTeX\ as follows on most relevant systems:
%\begin{verbatim}
%latex '\AtBeginDocument{\AlsoImplementation}%\iffalse -*-mode:Latex;tex-command:"latex *;dvips -D600 pst-blur -o"-*- \fi
%\iffalse
%% $Id: pst-blur.dtx,v 2.0 2005/09/08 09:48:33 giese Exp $
%%
%% Copyright 1998 Martin Giese, Martin.Giese@oeaw.ac.at
%%           2005 Herbert Voss, voss@pstricks.de
%% 
%% This file is under the LaTeX Project Public License 
%% See CTAN archives in directory macros/latex/base/lppl.txt.
%%
%% DESCRIPTION:
%%   `pst-blur' is a PSTricks package for blurred shadows
%%
%  
%\fi
%\iffalse
%<*driver>
\NeedsTeXFormat{LaTeX2e}
\documentclass{ltxdoc}
\usepackage{pst-text}
\usepackage{pst-node}
\usepackage{url}
\usepackage{pst-blur}
\AtBeginDocument{
%  \OnlyDescription % comment out for implementation details
  \EnableCrossrefs
  \RecordChanges
  \CodelineIndex}
\AtEndDocument{
  \PrintChanges
  \PrintIndex}
\hbadness=7000            % Over and under full box warnings
\hfuzz=3pt
\MakeShortVerb{\|}
\newcommand\file[1]{\texttt{#1}}
\begin{document}
\DocInput{pst-blur.dtx}
\end{document}
%</driver>
%\fi
%
% \CharacterTable
%  {Upper-case    \A\B\C\D\E\F\G\H\I\J\K\L\M\N\O\P\Q\R\S\T\U\V\W\X\Y\Z
%   Lower-case    \a\b\c\d\e\f\g\h\i\j\k\l\m\n\o\p\q\r\s\t\u\v\w\x\y\z
%   Digits        \0\1\2\3\4\5\6\7\8\9
%   Exclamation   \!     Double quote  \"     Hash (number) \#
%   Dollar        \$     Percent       \%     Ampersand     \&
%   Acute accent  \'     Left paren    \(     Right paren   \)
%   Asterisk      \*     Plus          \+     Comma         \,
%   Minus         \-     Point         \.     Solidus       \/
%   Colon         \:     Semicolon     \;     Less than     \<
%   Equals        \=     Greater than  \>     Question mark \?
%   Commercial at \@     Left bracket  \[     Backslash     \\
%   Right bracket \]     Circumflex    \^     Underscore    \_
%   Grave accent  \`     Left brace    \{     Vertical bar  \|
%   Right brace   \}     Tilde         \~}
% \CheckSum{152}
%
%
% \changes{v2.00}{2005/09/08}{%
%	using the extended pst-xkey instead of the old pst-key package; 
%	creating a dtx file;
%	new \LaTeX\ wrapper file (hv)}
% \changes{v1.80}{2001/02/16}{First public release. (mg)}
%
% \DoNotIndex{\!,\",\#,\$,\%,\&,\',\(,\+,\*,\,,\-,\.,\/,\:,\;,\<,\=,\>,\?}
% \DoNotIndex{\@,\@B,\@K,\@cTq,\@f,\@fPl,\@ifnextchar,\@nameuse,\@oVk}
% \DoNotIndex{\[,\\,\],\^,\_,\ }
% \DoNotIndex{\^,\\^,\\\^,$\^$,$\\^$,$\\^$}
% \DoNotIndex{\0,\2,\4,\5,\6,\7,\8,}
% \DoNotIndex{\A,\a}
% \DoNotIndex{\B,\b,\Bc,\begin,\Bq,\Bqc}
% \DoNotIndex{\C,\c,\catcode,\cJA,\CodelineIndex,\csname}
% \DoNotIndex{\D,\def,\define@key,\Df,\divide,\DocInput,\documentclass,\pst@addfams}
% \DoNotIndex{\eCN,\edef,\else,\eHd,\eMcj,\EnableCrossrefs,\end,\endcsname}
% \DoNotIndex{\endCenterExample,\endExample,\endinput,\endpsclip}
% \DoNotIndex{\PrintIndex,\PrintChanges,\ProvidesFile}
% \DoNotIndex{\endpspicture,\endSideBySideExample,\Example}
% \DoNotIndex{\F,\f,\FdUrr,\fi,\filedate,\fileversion,\FV@Environment}
% \DoNotIndex{\FV@UseKeyValues,\FV@XRightMargin,\FVB@Example,\fvset}
% \DoNotIndex{\G,\g,\GetFileInfo,\gr,\GradientLoaded,\gsFKrbK@o,\gsj,\gsOX}
% \DoNotIndex{\hbadness,\hfuzz,\HLEmphasize,\HLMacro,\HLMacro@i}
% \DoNotIndex{\HLReverse,\HLReverse@i,\hqcu,\HqY}
% \DoNotIndex{\I,\i,\ifx,\input,\Ir,\IU}
% \DoNotIndex{\j,\jl,\JT,\JVodH}
% \DoNotIndex{\K,\k,\kfSlL}
% \DoNotIndex{\L,\let}
% \DoNotIndex{\message,\mHNa,\mIU}
% \DoNotIndex{\N,\nB,\newcmykcolor,\newdimen,\newif,\nW}
% \DoNotIndex{\O,\oCDJDo,\ocQhVI,\OnlyDescription,\oRKJ}
% \DoNotIndex{\P,\p,\ProvidesPackage,\psframe,\pslinewidth,\psset}
% \DoNotIndex{\PstAtCode,\PSTricksLoaded}
% \DoNotIndex{\q,\Qr,\qssRXq,\qu,\qXjFQp,\qYL}
% \DoNotIndex{\R,\r,\RecordChanges,\relax,\RlaYI,\rN,\Rp,\rp,\RPDXNn,\rput}
% \DoNotIndex{\S,\scalebox,\SgY,\SideBySide@Example,\SideBySideExample}
% \DoNotIndex{\SgY,\sk,\Sp,\space,\sZb}
% \DoNotIndex{\T,\the,\tw@}
% \DoNotIndex{\u,\UiSWGEf@,\uJi,\usepackage,\uVQdMM,\UYj}
% \DoNotIndex{\VerbatimEnvironment,\VerbatimInput,\VrC@}
% \DoNotIndex{\WhZ,\WjKCYb,\WNs}
% \DoNotIndex{\XkN,\XW}
% \DoNotIndex{\Z,\ZCM,\Ze}
% \DoNotIndex{\addtocounter,\advance,\alph,\arabic,\AtBeginDocument,\AtEndDocument}
% \DoNotIndex{\AtEndOfPackage,\begingroup,\bfseries,\bgroup,\box,\csname}
% \DoNotIndex{\else,\endcsname,\endgroup,\endinput,\expandafter,\fi}
% \DoNotIndex{\TeX,\z@,\p@,\@one,\xdef,\thr@@,\string,\sixt@@n,\reset,\or,\multiply,\repeat,\RequirePackage}
% \DoNotIndex{\@cclvi,\@ne,\@ehpa,\@nil,\copy,\dp,\global,\hbox,\hss,\ht,\ifodd,\ifdim,\ifcase,\kern}
% \DoNotIndex{\chardef,\loop,\leavevmode,\ifnum,\lower,\next,\wd,\setbox}
% \setcounter{IndexColumns}{2}
%
% \title{\textsf{pst-blur} package \\ version \fileversion}
% \author{Martin Giese\thanks{\protect\url{Martin.Giese@oeaw.ac.at}}}
% \date{\filedate}
% \maketitle
%
%\section{Introduction}
% The ability to paint shadows on arbitrary shapes is a standard 
% feature of PSTricks.  However, these shadows are always `hard':\\
%{\psset{unit=0.6cm,linewidth=0.5pt}
% \begin{pspicture}(-5,0)(10,4)
%   \psset{shadow=true,shadowcolor=gray,shadowsize=0.15cm}
%   \psframe(1,1)(3,3)
%   \pscircle(6,2){1}
%   \rput(9,2){\pscharpath{\fontfamily{ptm}\fontsize{40}{40}\selectfont A}}
%  \end{pspicture}
%}\\
% The |pst-blur| package provides  blurred shadows
% for closed shapes drawn with PSTricks:\\
%{\psset{unit=0.6cm,linewidth=0.5pt,blur=true,blurradius=0.07cm}
% \begin{pspicture}(-5,0)(10,4)
%   \psset{shadow=true,shadowcolor=gray,shadowsize=0.15cm}
%   \psframe(1,1)(3,3)
%   \pscircle(6,2){1}
%   \rput(9,2){\pscharpath{\fontfamily{ptm}\fontsize{40}{40}\selectfont A}}
%  \end{pspicture}
%}\\
% It also provides a new
% box command |\psblurbox|, which is similar to |\psshadowbox|,
% but gives the box a blurred shadow.
% \medskip
%
%The new graphics parameters and macros provided by the package
% are described in
%section 2 of this document.  Section 3, if present, documents the 
%implementation consisting of a generic \TeX\ file and a PostScript
%header for the |dvi|-to-PostScript converter.  You can get section 3
%by calling \LaTeX\ as follows on most relevant systems:
%\begin{verbatim}
%latex '\AtBeginDocument{\AlsoImplementation}%\iffalse -*-mode:Latex;tex-command:"latex *;dvips -D600 pst-blur -o"-*- \fi
%\iffalse
%% $Id: pst-blur.dtx,v 2.0 2005/09/08 09:48:33 giese Exp $
%%
%% Copyright 1998 Martin Giese, Martin.Giese@oeaw.ac.at
%%           2005 Herbert Voss, voss@pstricks.de
%% 
%% This file is under the LaTeX Project Public License 
%% See CTAN archives in directory macros/latex/base/lppl.txt.
%%
%% DESCRIPTION:
%%   `pst-blur' is a PSTricks package for blurred shadows
%%
%  
%\fi
%\iffalse
%<*driver>
\NeedsTeXFormat{LaTeX2e}
\documentclass{ltxdoc}
\usepackage{pst-text}
\usepackage{pst-node}
\usepackage{url}
\usepackage{pst-blur}
\AtBeginDocument{
%  \OnlyDescription % comment out for implementation details
  \EnableCrossrefs
  \RecordChanges
  \CodelineIndex}
\AtEndDocument{
  \PrintChanges
  \PrintIndex}
\hbadness=7000            % Over and under full box warnings
\hfuzz=3pt
\MakeShortVerb{\|}
\newcommand\file[1]{\texttt{#1}}
\begin{document}
\DocInput{pst-blur.dtx}
\end{document}
%</driver>
%\fi
%
% \CharacterTable
%  {Upper-case    \A\B\C\D\E\F\G\H\I\J\K\L\M\N\O\P\Q\R\S\T\U\V\W\X\Y\Z
%   Lower-case    \a\b\c\d\e\f\g\h\i\j\k\l\m\n\o\p\q\r\s\t\u\v\w\x\y\z
%   Digits        \0\1\2\3\4\5\6\7\8\9
%   Exclamation   \!     Double quote  \"     Hash (number) \#
%   Dollar        \$     Percent       \%     Ampersand     \&
%   Acute accent  \'     Left paren    \(     Right paren   \)
%   Asterisk      \*     Plus          \+     Comma         \,
%   Minus         \-     Point         \.     Solidus       \/
%   Colon         \:     Semicolon     \;     Less than     \<
%   Equals        \=     Greater than  \>     Question mark \?
%   Commercial at \@     Left bracket  \[     Backslash     \\
%   Right bracket \]     Circumflex    \^     Underscore    \_
%   Grave accent  \`     Left brace    \{     Vertical bar  \|
%   Right brace   \}     Tilde         \~}
% \CheckSum{152}
%
%
% \changes{v2.00}{2005/09/08}{%
%	using the extended pst-xkey instead of the old pst-key package; 
%	creating a dtx file;
%	new \LaTeX\ wrapper file (hv)}
% \changes{v1.80}{2001/02/16}{First public release. (mg)}
%
% \DoNotIndex{\!,\",\#,\$,\%,\&,\',\(,\+,\*,\,,\-,\.,\/,\:,\;,\<,\=,\>,\?}
% \DoNotIndex{\@,\@B,\@K,\@cTq,\@f,\@fPl,\@ifnextchar,\@nameuse,\@oVk}
% \DoNotIndex{\[,\\,\],\^,\_,\ }
% \DoNotIndex{\^,\\^,\\\^,$\^$,$\\^$,$\\^$}
% \DoNotIndex{\0,\2,\4,\5,\6,\7,\8,}
% \DoNotIndex{\A,\a}
% \DoNotIndex{\B,\b,\Bc,\begin,\Bq,\Bqc}
% \DoNotIndex{\C,\c,\catcode,\cJA,\CodelineIndex,\csname}
% \DoNotIndex{\D,\def,\define@key,\Df,\divide,\DocInput,\documentclass,\pst@addfams}
% \DoNotIndex{\eCN,\edef,\else,\eHd,\eMcj,\EnableCrossrefs,\end,\endcsname}
% \DoNotIndex{\endCenterExample,\endExample,\endinput,\endpsclip}
% \DoNotIndex{\PrintIndex,\PrintChanges,\ProvidesFile}
% \DoNotIndex{\endpspicture,\endSideBySideExample,\Example}
% \DoNotIndex{\F,\f,\FdUrr,\fi,\filedate,\fileversion,\FV@Environment}
% \DoNotIndex{\FV@UseKeyValues,\FV@XRightMargin,\FVB@Example,\fvset}
% \DoNotIndex{\G,\g,\GetFileInfo,\gr,\GradientLoaded,\gsFKrbK@o,\gsj,\gsOX}
% \DoNotIndex{\hbadness,\hfuzz,\HLEmphasize,\HLMacro,\HLMacro@i}
% \DoNotIndex{\HLReverse,\HLReverse@i,\hqcu,\HqY}
% \DoNotIndex{\I,\i,\ifx,\input,\Ir,\IU}
% \DoNotIndex{\j,\jl,\JT,\JVodH}
% \DoNotIndex{\K,\k,\kfSlL}
% \DoNotIndex{\L,\let}
% \DoNotIndex{\message,\mHNa,\mIU}
% \DoNotIndex{\N,\nB,\newcmykcolor,\newdimen,\newif,\nW}
% \DoNotIndex{\O,\oCDJDo,\ocQhVI,\OnlyDescription,\oRKJ}
% \DoNotIndex{\P,\p,\ProvidesPackage,\psframe,\pslinewidth,\psset}
% \DoNotIndex{\PstAtCode,\PSTricksLoaded}
% \DoNotIndex{\q,\Qr,\qssRXq,\qu,\qXjFQp,\qYL}
% \DoNotIndex{\R,\r,\RecordChanges,\relax,\RlaYI,\rN,\Rp,\rp,\RPDXNn,\rput}
% \DoNotIndex{\S,\scalebox,\SgY,\SideBySide@Example,\SideBySideExample}
% \DoNotIndex{\SgY,\sk,\Sp,\space,\sZb}
% \DoNotIndex{\T,\the,\tw@}
% \DoNotIndex{\u,\UiSWGEf@,\uJi,\usepackage,\uVQdMM,\UYj}
% \DoNotIndex{\VerbatimEnvironment,\VerbatimInput,\VrC@}
% \DoNotIndex{\WhZ,\WjKCYb,\WNs}
% \DoNotIndex{\XkN,\XW}
% \DoNotIndex{\Z,\ZCM,\Ze}
% \DoNotIndex{\addtocounter,\advance,\alph,\arabic,\AtBeginDocument,\AtEndDocument}
% \DoNotIndex{\AtEndOfPackage,\begingroup,\bfseries,\bgroup,\box,\csname}
% \DoNotIndex{\else,\endcsname,\endgroup,\endinput,\expandafter,\fi}
% \DoNotIndex{\TeX,\z@,\p@,\@one,\xdef,\thr@@,\string,\sixt@@n,\reset,\or,\multiply,\repeat,\RequirePackage}
% \DoNotIndex{\@cclvi,\@ne,\@ehpa,\@nil,\copy,\dp,\global,\hbox,\hss,\ht,\ifodd,\ifdim,\ifcase,\kern}
% \DoNotIndex{\chardef,\loop,\leavevmode,\ifnum,\lower,\next,\wd,\setbox}
% \setcounter{IndexColumns}{2}
%
% \title{\textsf{pst-blur} package \\ version \fileversion}
% \author{Martin Giese\thanks{\protect\url{Martin.Giese@oeaw.ac.at}}}
% \date{\filedate}
% \maketitle
%
%\section{Introduction}
% The ability to paint shadows on arbitrary shapes is a standard 
% feature of PSTricks.  However, these shadows are always `hard':\\
%{\psset{unit=0.6cm,linewidth=0.5pt}
% \begin{pspicture}(-5,0)(10,4)
%   \psset{shadow=true,shadowcolor=gray,shadowsize=0.15cm}
%   \psframe(1,1)(3,3)
%   \pscircle(6,2){1}
%   \rput(9,2){\pscharpath{\fontfamily{ptm}\fontsize{40}{40}\selectfont A}}
%  \end{pspicture}
%}\\
% The |pst-blur| package provides  blurred shadows
% for closed shapes drawn with PSTricks:\\
%{\psset{unit=0.6cm,linewidth=0.5pt,blur=true,blurradius=0.07cm}
% \begin{pspicture}(-5,0)(10,4)
%   \psset{shadow=true,shadowcolor=gray,shadowsize=0.15cm}
%   \psframe(1,1)(3,3)
%   \pscircle(6,2){1}
%   \rput(9,2){\pscharpath{\fontfamily{ptm}\fontsize{40}{40}\selectfont A}}
%  \end{pspicture}
%}\\
% It also provides a new
% box command |\psblurbox|, which is similar to |\psshadowbox|,
% but gives the box a blurred shadow.
% \medskip
%
%The new graphics parameters and macros provided by the package
% are described in
%section 2 of this document.  Section 3, if present, documents the 
%implementation consisting of a generic \TeX\ file and a PostScript
%header for the |dvi|-to-PostScript converter.  You can get section 3
%by calling \LaTeX\ as follows on most relevant systems:
%\begin{verbatim}
%latex '\AtBeginDocument{\AlsoImplementation}\input{pst-blur.dtx}'
%\end{verbatim}
%
%\section{Package Usage}
% To use |pst-blur|, you have to say
% \begin{verbatim}
%   \usepackage{pst-blur}
% \end{verbatim}
% in the document prologue for \LaTeX, and 
% \begin{verbatim}
%   \input pst-blur.tex
% \end{verbatim}
% in ``plain'' \TeX.
%
% \DescribeMacro{blur}
% To paint shapes with blurred shadows,
% set the graphics parameters |shadow| and |blur| to |true|, eg
% \pscircle[shadow=true,blur=true](8,-0.5){0.5}
% \begin{verbatim}
%    \psset{unit=1cm}
%    \pscircle[shadow=true,blur=true](0,0){0.5}
% \end{verbatim}
% \psset{unit=1cm}
% for a circle with a blurred shadow.
% The parameter |blur| has no influence if |shadow| is |false|.
% \medskip
%
% \DescribeMacro{shadowsize}
% \DescribeMacro{shadowangle}
% \DescribeMacro{blurradius}
% The rendering of blurred shadows is controlled by a number of
% additional graphics parameters.  The offset of the shadow is controlled
% by the parameters |shadowsize| and |shadowangle|, which are the same 
% as for ordinary shadows.\footnote{In particular, |shadowangle| has
% to be negative for the usual placement of shadows below and to the
% right of shapes.}  The size of the blurring effect is
% controlled by the parameter |blurradius|, see Fig~\ref{fig:params}.
% The default value for |blurradius| is 1.5pt, which fits nicely with
% the default |shadowsize| of 3pt.
% \medskip
%
% \begin{figure}\caption{Parameters for blurred shadows} 
% \label{fig:params}
% \vskip1cm
% \qquad\qquad\begin{pspicture*}(0,0)(10,6)
%     \psframe[linewidth=4pt,fillcolor=lightgray,fillstyle=crosshatch,
%             	shadow=true,blur=true,shadowsize=2cm,shadowangle=-35,
%		blurradius=1cm,shadowcolor=lightgray](-5,3)(5,10)
%   \pnode(5,3){A}
%   \pnode(5.3,2.9){A1}
%   \pnode(6.64,1.85){B}
%   \pnode(7.51,2.35){C}
%   \pscircle(6.64,1.85){1}
%   \ncline{|-|}{A}{B}
%   \bput(0.2){|shadowsize|}
%   \ncline{->}{B}{C}
%   \bput(1.5){|blurradius|}
%   \psline(5,3)(6,3)
%   \psarcn{->}(5,3){0.5}{0}{-35}
%   \rput(7,3.7){\rnode{D}{$-|shadowangle|$}}
%   \nccurve[linewidth=0.5pt,angleA=-90,angleB=70]{->}{D}{A1}
%   \rput[l](6,5.6){|shadowcolor|}
%   \rput[l](8,5){|blurbg|}
%   {\psset{linestyle=dotted}
%    \psline(0,0.85)(6.64,0.85)
%    \psline(7.64,1.85)(7.64,6)
%    \psset{linestyle=dashed}
%    \psline(0,2.85)(5.64,2.85)
%    \psline(5.64,2.85)(5.64,6)}
%   {\psset{linewidth=0.5pt}
%    \psline{*-}(5.64,5.3)(6,5.4)
%    \psline{*-}(7.64,4.75)(8,4.85)
%    \ncline{F}{F1}
%   }
% \end{pspicture*}
% \end{figure}
%
% \DescribeMacro{shadowcolor}
% \DescribeMacro{blurbg}
% The inner, usually darkest part of the shadow is painted in the
% colour defined by |shadowcolor|.  In the range defined by |blurradius|,
% the colour gradually fades to the background colour set by |blurbg|.
% The default value for |blurbg| is white.  You should change this parameter
% when you want to paint shapes over a coloured background, ie\\
% \begin{minipage}{\textwidth}
% \begin{verbatim}
%   \psframe[fillstyle=solid,fillcolor=yellow](-.7,-.7)(.7,.7)
%   \pscircle[shadow=true,blur=true,blurbg=yellow](0,0){0.4}
% \end{verbatim}
% \rput(11.7,1.4){
%   \psframe[fillstyle=solid,fillcolor=yellow](-.7,-.7)(.7,.7)
%   \pscircle[shadow=true,blur=true,blurbg=yellow](0,0){0.4}}
% \end{minipage}
%
% \DescribeMacro{blursteps}
% The number of distinct colour steps painted between |shadowcolor|
% and |blurbg| is controlled by the parameter |blursteps|.  The default
% value for |blursteps| is 20, which is usually more than sufficient.
% Note, that higher values for |blursteps| result in proportionally slower
% rendering.  This can be very tiresome with complex shapes.
% \medskip
%
% \DescribeMacro{\psblurbox}
% Using a
% \psframebox[shadow=true,blur=true,blurradius=2.5pt,shadowcolor=gray]%
% {|\ttfamily\symbol{92}psframebox|} 
% with a blurred 
% shadow in the middle of some text produces poor results, because \TeX\
% does not know about the extra space taken by the shadow.  For normal
% shadows, this problem is solved by the |\psshadowbox| macro, which
% adds the extra space around the box for the shadow.  For blurred shadows,
% this is not sufficient: an extra |\blurradius| has to be added.  This 
% is done by the macro \psblurbox{\ttfamily\symbol{92}psblurbox}, which is otherwise
% identical to |\psshadowbox|.  Note, that |\psblurbox| shares a
% deficiency of |\psshadowbox|:  It only works correctly 
% with $|shadowangle|=-45$, because \TeX\ does not provide trigonometric
% operations.
%
% \StopEventually{}
%
% \section{The Code}
%
% \subsection{The \file{pst-blur.sty} file}
%    The \file{pst-blur.sty} file is very simple.  It just loads 
%    the generic \file{pst-blur.tex} file. 
%    \begin{macrocode}
%<*stylefile>
\RequirePackage{pstricks}
\ProvidesPackage{pst-blur}[2005/09/08 package wrapper for 
  pst-blur.tex (hv)]
\input{pst-blur.tex}
\ProvidesFile{pst-blur.tex}
  [\filedate\space v\fileversion\space `PST-blur' (hv)]
%</stylefile>
%    \end{macrocode}
%
% \subsection{The \file{pst-blur.tex} file}
%    \file{pst-blur.tex} contains the \TeX-side of things.  We begin
%    by identifying ourselves and setting things up, the same as in 
%    other PSTricks packages.
%    \begin{macrocode}
%<*texfile>
\csname PstBlurLoaded\endcsname
\let\PstBlurLoaded\endinput
\ifx\PSTricksLoaded\endinput\else
  \def\next{\input pstricks.tex }\expandafter\next
\fi
%    \end{macrocode}
% \verb+pst-blur+ uses the extended version of the keyvalue interface.
%    \begin{macrocode}
\ifx\PSTXKeyLoaded\endinput\else\input pst-xkey \fi
%    \end{macrocode}
%%
%    \begin{macrocode}
\def\fileversion{2.0}
\def\filedate{2005/09/08}
\message{ v\fileversion, \filedate}
\edef\TheAtCode{\the\catcode`\@}
\catcode`\@=11
%    \end{macrocode}
% Add the package name to the list of family names of the keyvalue list.
%    \begin{macrocode}
\pst@addfams{pst-blur}
\pstheader{pst-blur.pro}
%    \end{macrocode}
%
% \subsubsection{New graphics parameters}
% \begin{macro}{blur}
% \begin{macro}{blurradius}
% \begin{macro}{blursteps}
% \begin{macro}{blurbg}
% The definitions of the new graphics parameters follow the definitions
% for parameters of the same types found in |pstricks.tex|.
%    \begin{macrocode}
\newif\ifpsblur
\define@key[psset]{pst-blur}{blur}[true]{\@nameuse{psblur#1}\pst@setrepeatarrowsflag}
\psset{blur=false}
%%
\define@key[psset]{pst-blur}{blurradius}{\pst@getlength{#1}\psx@blurradius}
\psset{blurradius=1.5pt}
%%
\define@key[psset]{pst-blur}{blursteps}{\pst@getint{#1}\psx@blursteps}
\psset{blursteps=20}
%%
\define@key[psset]{pst-blur}{blurbg}{\pst@getcolor{#1}\psx@blurbg}
\psset{blurbg=white}
%    \end{macrocode}
% \end{macro}
% \end{macro}
% \end{macro}
% \end{macro}
% \subsection{Hooking into the PSTricks shadow macros}
% \begin{macro}{\pst@closedshadow}
% The macro |\pst@closedshadow| is usually called internally by
% PSTricks to paint a shadow in the shape of the current path.  
% This macro has been renamed |\pst@sharpclosedshadow|.  The
% new |\pst@closedshadow| jumps to either of |\pst@sharpclosedshadow|
% or |\pst@blurclosedshadow|, depending on |\ifpsblur|, which is
% directly related to the graphics parameter |blur|.
%    \begin{macrocode}
\def\pst@closedshadow{%
\ifpsblur\pst@blurclosedshadow\else\pst@sharpclosedshadow\fi
}
\def\pst@sharpclosedshadow{%
  \addto@pscode{%
    gsave
    \psk@shadowsize \psk@shadowangle \tx@PtoC
    \tx@Shadow
    \pst@usecolor\psshadowcolor
    gsave fill grestore
    stroke
    grestore
    gsave
    \pst@usecolor\psfillcolor
    gsave fill grestore
    stroke
    grestore}}
%    \end{macrocode}
% \end{macro}
% \begin{macro}{\pst@blurclosedshadow}
% The PostScript code for blurred shadows is produced by the following
% macro.  It pushes the diverse parameters (|\tx@PtoC| does polar to
% cartesian coordinate transformation for the shadow offset) and calls
% |BlurShadow|.  Afterwards, it fills and strokes the current path,
% same as the original |\pst@closedshadow|.
%    \begin{macrocode}
\def\pst@blurclosedshadow{%
  \addto@pscode{%
    gsave
    gsave \pst@usecolor\psshadowcolor currentrgbcolor grestore
    gsave \pst@usecolor\psx@blurbg currentrgbcolor grestore
    \psx@blurradius\space
    \psx@blursteps\space
    \psk@shadowsize \psk@shadowangle \tx@PtoC
    tx@PstBlurDict begin BlurShadow end
    grestore
    gsave
    \pst@usecolor\psfillcolor
    gsave fill grestore
    stroke
    grestore}}
%    \end{macrocode}
% \end{macro}
% \begin{macro}{\pst@blurclosedshadow}
% This one looks very impressing.  In fact, it is a verbatim copy
% of |\psshadowbox|, with only the line 
% |\advance\pst@dimh\psx@blurradius\p@| added!
%    \begin{macrocode}
\def\psblurbox{%
\def\pst@par{}\pst@object{psblurbox}}
\def\psblurbox@i{\pst@makebox\psblurbox@ii}
\def\psblurbox@ii{%
  \begingroup
  \pst@useboxpar
  \psblurtrue
  \psshadowtrue
  \psboxseptrue
  \setbox\pst@hbox=\hbox{\psframebox@ii}%
  \pst@dimh=\psk@shadowsize\p@
  \pst@dimh=.7071\pst@dimh
  \advance\pst@dimh\psx@blurradius\p@
  \pst@dimg=\dp\pst@hbox
  \advance\pst@dimg\pst@dimh
  \dp\pst@hbox=\pst@dimg
  \pst@dimg=\wd\pst@hbox
  \advance\pst@dimg\pst@dimh
  \wd\pst@hbox=\pst@dimg
  \leavevmode
  \box\pst@hbox
\endgroup}
%%
\catcode`\@=\TheAtCode\relax
%</texfile>
%    \end{macrocode}
% \end{macro}
%
% \subsection{The \file{pst-blur.pro} file}
%    The file \file{pst-blur.pro} contains PostScript definitions
%    to be included  in the PostScript output by the 
%    |dvi|-to-PostScript converter, eg |dvips|. 
%    This is all rather similar to
%    \file{pst-slpe.pro}, and I just don't feel like explaining it,
%    so you'll have to work through it yourself, if you want to
%    know what happens.  The trick is basically to draw the outline
%    repeatedly with varying line widths.  The procedure |Shadow|
%    called in |BlurShadow| is defined in \file{pstricks.pro} and
%    translates the current path based on an $x$- and $y$-displacement
%    taken from the stack.
%    \begin{macrocode}
%<*prolog>
/tx@PstBlurDict 60 dict def
tx@PstBlurDict begin
/Iterate {
  /SegLines ED
  /ThisB ED /ThisG ED /ThisR ED
  /NextB ED /NextG ED /NextR ED
  /W 2.0 BlurRadius mul def
  /WDec W SegLines div def
  /RInc NextR ThisR sub SegLines div def
  /GInc NextG ThisG sub SegLines div def
  /BInc NextB ThisB sub SegLines div def
  /R ThisR def
  /G ThisG def
  /B ThisB def
  SegLines {
    R G B
    sqrt 3 1 roll sqrt 3 1 roll sqrt 3 1 roll
    setrgbcolor
    gsave W setlinewidth
    stroke grestore
    /W W WDec sub def
    /R R RInc add def
    /G G GInc add def
    /B B BInc add def
  } bind repeat
} def
/BlurShadow {
  Shadow
  /BlurSteps ED
  /BlurRadius ED
  dup mul /BEnd ED dup mul /GEnd ED dup mul /REnd ED 
  dup mul /BBeg ED dup mul /GBeg ED dup mul /RBeg ED 
  RBeg REnd add 0.5 mul /RMid ED
  GBeg GEnd add 0.5 mul /GMid ED
  BBeg BEnd add 0.5 mul /BMid ED
  /OuterSteps BlurSteps 2 div cvi def
  /InnerSteps BlurSteps OuterSteps sub def
  1 setlinejoin
  RMid GMid BMid REnd GEnd BEnd OuterSteps Iterate 
  gsave RBeg sqrt GBeg sqrt BBeg sqrt setrgbcolor fill grestore
  clip
  0 setlinejoin
  RMid GMid BMid RBeg GBeg BBeg InnerSteps Iterate 
} def
end
%</prolog>
%    \end{macrocode}
% \Finale
%
'
%\end{verbatim}
%
%\section{Package Usage}
% To use |pst-blur|, you have to say
% \begin{verbatim}
%   \usepackage{pst-blur}
% \end{verbatim}
% in the document prologue for \LaTeX, and 
% \begin{verbatim}
%   \input pst-blur.tex
% \end{verbatim}
% in ``plain'' \TeX.
%
% \DescribeMacro{blur}
% To paint shapes with blurred shadows,
% set the graphics parameters |shadow| and |blur| to |true|, eg
% \pscircle[shadow=true,blur=true](8,-0.5){0.5}
% \begin{verbatim}
%    \psset{unit=1cm}
%    \pscircle[shadow=true,blur=true](0,0){0.5}
% \end{verbatim}
% \psset{unit=1cm}
% for a circle with a blurred shadow.
% The parameter |blur| has no influence if |shadow| is |false|.
% \medskip
%
% \DescribeMacro{shadowsize}
% \DescribeMacro{shadowangle}
% \DescribeMacro{blurradius}
% The rendering of blurred shadows is controlled by a number of
% additional graphics parameters.  The offset of the shadow is controlled
% by the parameters |shadowsize| and |shadowangle|, which are the same 
% as for ordinary shadows.\footnote{In particular, |shadowangle| has
% to be negative for the usual placement of shadows below and to the
% right of shapes.}  The size of the blurring effect is
% controlled by the parameter |blurradius|, see Fig~\ref{fig:params}.
% The default value for |blurradius| is 1.5pt, which fits nicely with
% the default |shadowsize| of 3pt.
% \medskip
%
% \begin{figure}\caption{Parameters for blurred shadows} 
% \label{fig:params}
% \vskip1cm
% \qquad\qquad\begin{pspicture*}(0,0)(10,6)
%     \psframe[linewidth=4pt,fillcolor=lightgray,fillstyle=crosshatch,
%             	shadow=true,blur=true,shadowsize=2cm,shadowangle=-35,
%		blurradius=1cm,shadowcolor=lightgray](-5,3)(5,10)
%   \pnode(5,3){A}
%   \pnode(5.3,2.9){A1}
%   \pnode(6.64,1.85){B}
%   \pnode(7.51,2.35){C}
%   \pscircle(6.64,1.85){1}
%   \ncline{|-|}{A}{B}
%   \bput(0.2){|shadowsize|}
%   \ncline{->}{B}{C}
%   \bput(1.5){|blurradius|}
%   \psline(5,3)(6,3)
%   \psarcn{->}(5,3){0.5}{0}{-35}
%   \rput(7,3.7){\rnode{D}{$-|shadowangle|$}}
%   \nccurve[linewidth=0.5pt,angleA=-90,angleB=70]{->}{D}{A1}
%   \rput[l](6,5.6){|shadowcolor|}
%   \rput[l](8,5){|blurbg|}
%   {\psset{linestyle=dotted}
%    \psline(0,0.85)(6.64,0.85)
%    \psline(7.64,1.85)(7.64,6)
%    \psset{linestyle=dashed}
%    \psline(0,2.85)(5.64,2.85)
%    \psline(5.64,2.85)(5.64,6)}
%   {\psset{linewidth=0.5pt}
%    \psline{*-}(5.64,5.3)(6,5.4)
%    \psline{*-}(7.64,4.75)(8,4.85)
%    \ncline{F}{F1}
%   }
% \end{pspicture*}
% \end{figure}
%
% \DescribeMacro{shadowcolor}
% \DescribeMacro{blurbg}
% The inner, usually darkest part of the shadow is painted in the
% colour defined by |shadowcolor|.  In the range defined by |blurradius|,
% the colour gradually fades to the background colour set by |blurbg|.
% The default value for |blurbg| is white.  You should change this parameter
% when you want to paint shapes over a coloured background, ie\\
% \begin{minipage}{\textwidth}
% \begin{verbatim}
%   \psframe[fillstyle=solid,fillcolor=yellow](-.7,-.7)(.7,.7)
%   \pscircle[shadow=true,blur=true,blurbg=yellow](0,0){0.4}
% \end{verbatim}
% \rput(11.7,1.4){
%   \psframe[fillstyle=solid,fillcolor=yellow](-.7,-.7)(.7,.7)
%   \pscircle[shadow=true,blur=true,blurbg=yellow](0,0){0.4}}
% \end{minipage}
%
% \DescribeMacro{blursteps}
% The number of distinct colour steps painted between |shadowcolor|
% and |blurbg| is controlled by the parameter |blursteps|.  The default
% value for |blursteps| is 20, which is usually more than sufficient.
% Note, that higher values for |blursteps| result in proportionally slower
% rendering.  This can be very tiresome with complex shapes.
% \medskip
%
% \DescribeMacro{\psblurbox}
% Using a
% \psframebox[shadow=true,blur=true,blurradius=2.5pt,shadowcolor=gray]%
% {|\ttfamily\symbol{92}psframebox|} 
% with a blurred 
% shadow in the middle of some text produces poor results, because \TeX\
% does not know about the extra space taken by the shadow.  For normal
% shadows, this problem is solved by the |\psshadowbox| macro, which
% adds the extra space around the box for the shadow.  For blurred shadows,
% this is not sufficient: an extra |\blurradius| has to be added.  This 
% is done by the macro \psblurbox{\ttfamily\symbol{92}psblurbox}, which is otherwise
% identical to |\psshadowbox|.  Note, that |\psblurbox| shares a
% deficiency of |\psshadowbox|:  It only works correctly 
% with $|shadowangle|=-45$, because \TeX\ does not provide trigonometric
% operations.
%
% \StopEventually{}
%
% \section{The Code}
%
% \subsection{The \file{pst-blur.sty} file}
%    The \file{pst-blur.sty} file is very simple.  It just loads 
%    the generic \file{pst-blur.tex} file. 
%    \begin{macrocode}
%<*stylefile>
\RequirePackage{pstricks}
\ProvidesPackage{pst-blur}[2005/09/08 package wrapper for 
  pst-blur.tex (hv)]
%%
%% This is file `pst-blur.tex',
%% generated with the docstrip utility.
%%
%% The original source files were:
%%
%% pst-blur.dtx  (with options: `texfile')
%% 
%% IMPORTANT NOTICE:
%% 
%% For the copyright see the source file.
%% 
%% Any modified versions of this file must be renamed
%% with new filenames distinct from pst-blur.tex.
%% 
%% For distribution of the original source see the terms
%% for copying and modification in the file pst-blur.dtx.
%% 
%% This generated file may be distributed as long as the
%% original source files, as listed above, are part of the
%% same distribution. (The sources need not necessarily be
%% in the same archive or directory.)
%% $Id: pst-blur.dtx,v 2.0 2005/09/08 09:48:33 giese Exp $
%%
%% Copyright 1998 Martin Giese, Martin.Giese@oeaw.ac.at
%%           2005 Herbert Voss, voss@pstricks.de
%%
%% This file is under the LaTeX Project Public License
%% See CTAN archives in directory macros/latex/base/lppl.txt.
%%
%% DESCRIPTION:
%%   `pst-blur' is a PSTricks package for blurred shadows
%%
\csname PstBlurLoaded\endcsname
\let\PstBlurLoaded\endinput
\ifx\PSTricksLoaded\endinput\else
  \def\next{\input pstricks.tex }\expandafter\next
\fi
\ifx\PSTXKeyLoaded\endinput\else\input pst-xkey \fi
%%
\def\fileversion{2.0}
\def\filedate{2005/09/08}
\message{ v\fileversion, \filedate}
\edef\TheAtCode{\the\catcode`\@}
\catcode`\@=11
\pst@addfams{pst-blur}
\pstheader{pst-blur.pro}
\newif\ifpsblur
\define@key[psset]{pst-blur}{blur}[true]{\@nameuse{psblur#1}\pst@setrepeatarrowsflag}
\psset{blur=false}
%%
\define@key[psset]{pst-blur}{blurradius}{\pst@getlength{#1}\psx@blurradius}
\psset{blurradius=1.5pt}
%%
\define@key[psset]{pst-blur}{blursteps}{\pst@getint{#1}\psx@blursteps}
\psset{blursteps=20}
%%
\define@key[psset]{pst-blur}{blurbg}{\pst@getcolor{#1}\psx@blurbg}
\psset{blurbg=white}
\def\pst@closedshadow{%
\ifpsblur\pst@blurclosedshadow\else\pst@sharpclosedshadow\fi
}
\def\pst@sharpclosedshadow{%
  \addto@pscode{%
    gsave
    \psk@shadowsize \psk@shadowangle \tx@PtoC
    \tx@Shadow
    \pst@usecolor\psshadowcolor
    gsave fill grestore
    stroke
    grestore
    gsave
    \pst@usecolor\psfillcolor
    gsave fill grestore
    stroke
    grestore}}
\def\pst@blurclosedshadow{%
  \addto@pscode{%
    gsave
    gsave \pst@usecolor\psshadowcolor currentrgbcolor grestore
    gsave \pst@usecolor\psx@blurbg currentrgbcolor grestore
    \psx@blurradius\space
    \psx@blursteps\space
    \psk@shadowsize \psk@shadowangle \tx@PtoC
    tx@PstBlurDict begin BlurShadow end
    grestore
    gsave
    \pst@usecolor\psfillcolor
    gsave fill grestore
    stroke
    grestore}}
\def\psblurbox{%
\def\pst@par{}\pst@object{psblurbox}}
\def\psblurbox@i{\pst@makebox\psblurbox@ii}
\def\psblurbox@ii{%
  \begingroup
  \pst@useboxpar
  \psblurtrue
  \psshadowtrue
  \psboxseptrue
  \setbox\pst@hbox=\hbox{\psframebox@ii}%
  \pst@dimh=\psk@shadowsize\p@
  \pst@dimh=.7071\pst@dimh
  \advance\pst@dimh\psx@blurradius\p@
  \pst@dimg=\dp\pst@hbox
  \advance\pst@dimg\pst@dimh
  \dp\pst@hbox=\pst@dimg
  \pst@dimg=\wd\pst@hbox
  \advance\pst@dimg\pst@dimh
  \wd\pst@hbox=\pst@dimg
  \leavevmode
  \box\pst@hbox
\endgroup}
%%
\catcode`\@=\TheAtCode\relax
\endinput
%%
%% End of file `pst-blur.tex'.

\ProvidesFile{pst-blur.tex}
  [\filedate\space v\fileversion\space `PST-blur' (hv)]
%</stylefile>
%    \end{macrocode}
%
% \subsection{The \file{pst-blur.tex} file}
%    \file{pst-blur.tex} contains the \TeX-side of things.  We begin
%    by identifying ourselves and setting things up, the same as in 
%    other PSTricks packages.
%    \begin{macrocode}
%<*texfile>
\csname PstBlurLoaded\endcsname
\let\PstBlurLoaded\endinput
\ifx\PSTricksLoaded\endinput\else
  \def\next{\input pstricks.tex }\expandafter\next
\fi
%    \end{macrocode}
% \verb+pst-blur+ uses the extended version of the keyvalue interface.
%    \begin{macrocode}
\ifx\PSTXKeyLoaded\endinput\else\input pst-xkey \fi
%    \end{macrocode}
%%
%    \begin{macrocode}
\def\fileversion{2.0}
\def\filedate{2005/09/08}
\message{ v\fileversion, \filedate}
\edef\TheAtCode{\the\catcode`\@}
\catcode`\@=11
%    \end{macrocode}
% Add the package name to the list of family names of the keyvalue list.
%    \begin{macrocode}
\pst@addfams{pst-blur}
\pstheader{pst-blur.pro}
%    \end{macrocode}
%
% \subsubsection{New graphics parameters}
% \begin{macro}{blur}
% \begin{macro}{blurradius}
% \begin{macro}{blursteps}
% \begin{macro}{blurbg}
% The definitions of the new graphics parameters follow the definitions
% for parameters of the same types found in |pstricks.tex|.
%    \begin{macrocode}
\newif\ifpsblur
\define@key[psset]{pst-blur}{blur}[true]{\@nameuse{psblur#1}\pst@setrepeatarrowsflag}
\psset{blur=false}
%%
\define@key[psset]{pst-blur}{blurradius}{\pst@getlength{#1}\psx@blurradius}
\psset{blurradius=1.5pt}
%%
\define@key[psset]{pst-blur}{blursteps}{\pst@getint{#1}\psx@blursteps}
\psset{blursteps=20}
%%
\define@key[psset]{pst-blur}{blurbg}{\pst@getcolor{#1}\psx@blurbg}
\psset{blurbg=white}
%    \end{macrocode}
% \end{macro}
% \end{macro}
% \end{macro}
% \end{macro}
% \subsection{Hooking into the PSTricks shadow macros}
% \begin{macro}{\pst@closedshadow}
% The macro |\pst@closedshadow| is usually called internally by
% PSTricks to paint a shadow in the shape of the current path.  
% This macro has been renamed |\pst@sharpclosedshadow|.  The
% new |\pst@closedshadow| jumps to either of |\pst@sharpclosedshadow|
% or |\pst@blurclosedshadow|, depending on |\ifpsblur|, which is
% directly related to the graphics parameter |blur|.
%    \begin{macrocode}
\def\pst@closedshadow{%
\ifpsblur\pst@blurclosedshadow\else\pst@sharpclosedshadow\fi
}
\def\pst@sharpclosedshadow{%
  \addto@pscode{%
    gsave
    \psk@shadowsize \psk@shadowangle \tx@PtoC
    \tx@Shadow
    \pst@usecolor\psshadowcolor
    gsave fill grestore
    stroke
    grestore
    gsave
    \pst@usecolor\psfillcolor
    gsave fill grestore
    stroke
    grestore}}
%    \end{macrocode}
% \end{macro}
% \begin{macro}{\pst@blurclosedshadow}
% The PostScript code for blurred shadows is produced by the following
% macro.  It pushes the diverse parameters (|\tx@PtoC| does polar to
% cartesian coordinate transformation for the shadow offset) and calls
% |BlurShadow|.  Afterwards, it fills and strokes the current path,
% same as the original |\pst@closedshadow|.
%    \begin{macrocode}
\def\pst@blurclosedshadow{%
  \addto@pscode{%
    gsave
    gsave \pst@usecolor\psshadowcolor currentrgbcolor grestore
    gsave \pst@usecolor\psx@blurbg currentrgbcolor grestore
    \psx@blurradius\space
    \psx@blursteps\space
    \psk@shadowsize \psk@shadowangle \tx@PtoC
    tx@PstBlurDict begin BlurShadow end
    grestore
    gsave
    \pst@usecolor\psfillcolor
    gsave fill grestore
    stroke
    grestore}}
%    \end{macrocode}
% \end{macro}
% \begin{macro}{\pst@blurclosedshadow}
% This one looks very impressing.  In fact, it is a verbatim copy
% of |\psshadowbox|, with only the line 
% |\advance\pst@dimh\psx@blurradius\p@| added!
%    \begin{macrocode}
\def\psblurbox{%
\def\pst@par{}\pst@object{psblurbox}}
\def\psblurbox@i{\pst@makebox\psblurbox@ii}
\def\psblurbox@ii{%
  \begingroup
  \pst@useboxpar
  \psblurtrue
  \psshadowtrue
  \psboxseptrue
  \setbox\pst@hbox=\hbox{\psframebox@ii}%
  \pst@dimh=\psk@shadowsize\p@
  \pst@dimh=.7071\pst@dimh
  \advance\pst@dimh\psx@blurradius\p@
  \pst@dimg=\dp\pst@hbox
  \advance\pst@dimg\pst@dimh
  \dp\pst@hbox=\pst@dimg
  \pst@dimg=\wd\pst@hbox
  \advance\pst@dimg\pst@dimh
  \wd\pst@hbox=\pst@dimg
  \leavevmode
  \box\pst@hbox
\endgroup}
%%
\catcode`\@=\TheAtCode\relax
%</texfile>
%    \end{macrocode}
% \end{macro}
%
% \subsection{The \file{pst-blur.pro} file}
%    The file \file{pst-blur.pro} contains PostScript definitions
%    to be included  in the PostScript output by the 
%    |dvi|-to-PostScript converter, eg |dvips|. 
%    This is all rather similar to
%    \file{pst-slpe.pro}, and I just don't feel like explaining it,
%    so you'll have to work through it yourself, if you want to
%    know what happens.  The trick is basically to draw the outline
%    repeatedly with varying line widths.  The procedure |Shadow|
%    called in |BlurShadow| is defined in \file{pstricks.pro} and
%    translates the current path based on an $x$- and $y$-displacement
%    taken from the stack.
%    \begin{macrocode}
%<*prolog>
/tx@PstBlurDict 60 dict def
tx@PstBlurDict begin
/Iterate {
  /SegLines ED
  /ThisB ED /ThisG ED /ThisR ED
  /NextB ED /NextG ED /NextR ED
  /W 2.0 BlurRadius mul def
  /WDec W SegLines div def
  /RInc NextR ThisR sub SegLines div def
  /GInc NextG ThisG sub SegLines div def
  /BInc NextB ThisB sub SegLines div def
  /R ThisR def
  /G ThisG def
  /B ThisB def
  SegLines {
    R G B
    sqrt 3 1 roll sqrt 3 1 roll sqrt 3 1 roll
    setrgbcolor
    gsave W setlinewidth
    stroke grestore
    /W W WDec sub def
    /R R RInc add def
    /G G GInc add def
    /B B BInc add def
  } bind repeat
} def
/BlurShadow {
  Shadow
  /BlurSteps ED
  /BlurRadius ED
  dup mul /BEnd ED dup mul /GEnd ED dup mul /REnd ED 
  dup mul /BBeg ED dup mul /GBeg ED dup mul /RBeg ED 
  RBeg REnd add 0.5 mul /RMid ED
  GBeg GEnd add 0.5 mul /GMid ED
  BBeg BEnd add 0.5 mul /BMid ED
  /OuterSteps BlurSteps 2 div cvi def
  /InnerSteps BlurSteps OuterSteps sub def
  1 setlinejoin
  RMid GMid BMid REnd GEnd BEnd OuterSteps Iterate 
  gsave RBeg sqrt GBeg sqrt BBeg sqrt setrgbcolor fill grestore
  clip
  0 setlinejoin
  RMid GMid BMid RBeg GBeg BBeg InnerSteps Iterate 
} def
end
%</prolog>
%    \end{macrocode}
% \Finale
%
'
%\end{verbatim}
%
%\section{Package Usage}
% To use |pst-blur|, you have to say
% \begin{verbatim}
%   \usepackage{pst-blur}
% \end{verbatim}
% in the document prologue for \LaTeX, and 
% \begin{verbatim}
%   \input pst-blur.tex
% \end{verbatim}
% in ``plain'' \TeX.
%
% \DescribeMacro{blur}
% To paint shapes with blurred shadows,
% set the graphics parameters |shadow| and |blur| to |true|, eg
% \pscircle[shadow=true,blur=true](8,-0.5){0.5}
% \begin{verbatim}
%    \psset{unit=1cm}
%    \pscircle[shadow=true,blur=true](0,0){0.5}
% \end{verbatim}
% \psset{unit=1cm}
% for a circle with a blurred shadow.
% The parameter |blur| has no influence if |shadow| is |false|.
% \medskip
%
% \DescribeMacro{shadowsize}
% \DescribeMacro{shadowangle}
% \DescribeMacro{blurradius}
% The rendering of blurred shadows is controlled by a number of
% additional graphics parameters.  The offset of the shadow is controlled
% by the parameters |shadowsize| and |shadowangle|, which are the same 
% as for ordinary shadows.\footnote{In particular, |shadowangle| has
% to be negative for the usual placement of shadows below and to the
% right of shapes.}  The size of the blurring effect is
% controlled by the parameter |blurradius|, see Fig~\ref{fig:params}.
% The default value for |blurradius| is 1.5pt, which fits nicely with
% the default |shadowsize| of 3pt.
% \medskip
%
% \begin{figure}\caption{Parameters for blurred shadows} 
% \label{fig:params}
% \vskip1cm
% \qquad\qquad\begin{pspicture*}(0,0)(10,6)
%     \psframe[linewidth=4pt,fillcolor=lightgray,fillstyle=crosshatch,
%             	shadow=true,blur=true,shadowsize=2cm,shadowangle=-35,
%		blurradius=1cm,shadowcolor=lightgray](-5,3)(5,10)
%   \pnode(5,3){A}
%   \pnode(5.3,2.9){A1}
%   \pnode(6.64,1.85){B}
%   \pnode(7.51,2.35){C}
%   \pscircle(6.64,1.85){1}
%   \ncline{|-|}{A}{B}
%   \bput(0.2){|shadowsize|}
%   \ncline{->}{B}{C}
%   \bput(1.5){|blurradius|}
%   \psline(5,3)(6,3)
%   \psarcn{->}(5,3){0.5}{0}{-35}
%   \rput(7,3.7){\rnode{D}{$-|shadowangle|$}}
%   \nccurve[linewidth=0.5pt,angleA=-90,angleB=70]{->}{D}{A1}
%   \rput[l](6,5.6){|shadowcolor|}
%   \rput[l](8,5){|blurbg|}
%   {\psset{linestyle=dotted}
%    \psline(0,0.85)(6.64,0.85)
%    \psline(7.64,1.85)(7.64,6)
%    \psset{linestyle=dashed}
%    \psline(0,2.85)(5.64,2.85)
%    \psline(5.64,2.85)(5.64,6)}
%   {\psset{linewidth=0.5pt}
%    \psline{*-}(5.64,5.3)(6,5.4)
%    \psline{*-}(7.64,4.75)(8,4.85)
%    \ncline{F}{F1}
%   }
% \end{pspicture*}
% \end{figure}
%
% \DescribeMacro{shadowcolor}
% \DescribeMacro{blurbg}
% The inner, usually darkest part of the shadow is painted in the
% colour defined by |shadowcolor|.  In the range defined by |blurradius|,
% the colour gradually fades to the background colour set by |blurbg|.
% The default value for |blurbg| is white.  You should change this parameter
% when you want to paint shapes over a coloured background, ie\\
% \begin{minipage}{\textwidth}
% \begin{verbatim}
%   \psframe[fillstyle=solid,fillcolor=yellow](-.7,-.7)(.7,.7)
%   \pscircle[shadow=true,blur=true,blurbg=yellow](0,0){0.4}
% \end{verbatim}
% \rput(11.7,1.4){
%   \psframe[fillstyle=solid,fillcolor=yellow](-.7,-.7)(.7,.7)
%   \pscircle[shadow=true,blur=true,blurbg=yellow](0,0){0.4}}
% \end{minipage}
%
% \DescribeMacro{blursteps}
% The number of distinct colour steps painted between |shadowcolor|
% and |blurbg| is controlled by the parameter |blursteps|.  The default
% value for |blursteps| is 20, which is usually more than sufficient.
% Note, that higher values for |blursteps| result in proportionally slower
% rendering.  This can be very tiresome with complex shapes.
% \medskip
%
% \DescribeMacro{\psblurbox}
% Using a
% \psframebox[shadow=true,blur=true,blurradius=2.5pt,shadowcolor=gray]%
% {|\ttfamily\symbol{92}psframebox|} 
% with a blurred 
% shadow in the middle of some text produces poor results, because \TeX\
% does not know about the extra space taken by the shadow.  For normal
% shadows, this problem is solved by the |\psshadowbox| macro, which
% adds the extra space around the box for the shadow.  For blurred shadows,
% this is not sufficient: an extra |\blurradius| has to be added.  This 
% is done by the macro \psblurbox{\ttfamily\symbol{92}psblurbox}, which is otherwise
% identical to |\psshadowbox|.  Note, that |\psblurbox| shares a
% deficiency of |\psshadowbox|:  It only works correctly 
% with $|shadowangle|=-45$, because \TeX\ does not provide trigonometric
% operations.
%
% \StopEventually{}
%
% \section{The Code}
%
% \subsection{The \file{pst-blur.sty} file}
%    The \file{pst-blur.sty} file is very simple.  It just loads 
%    the generic \file{pst-blur.tex} file. 
%    \begin{macrocode}
%<*stylefile>
\RequirePackage{pstricks}
\ProvidesPackage{pst-blur}[2005/09/08 package wrapper for 
  pst-blur.tex (hv)]
%%
%% This is file `pst-blur.tex',
%% generated with the docstrip utility.
%%
%% The original source files were:
%%
%% pst-blur.dtx  (with options: `texfile')
%% 
%% IMPORTANT NOTICE:
%% 
%% For the copyright see the source file.
%% 
%% Any modified versions of this file must be renamed
%% with new filenames distinct from pst-blur.tex.
%% 
%% For distribution of the original source see the terms
%% for copying and modification in the file pst-blur.dtx.
%% 
%% This generated file may be distributed as long as the
%% original source files, as listed above, are part of the
%% same distribution. (The sources need not necessarily be
%% in the same archive or directory.)
%% $Id: pst-blur.dtx,v 2.0 2005/09/08 09:48:33 giese Exp $
%%
%% Copyright 1998 Martin Giese, Martin.Giese@oeaw.ac.at
%%           2005 Herbert Voss, voss@pstricks.de
%%
%% This file is under the LaTeX Project Public License
%% See CTAN archives in directory macros/latex/base/lppl.txt.
%%
%% DESCRIPTION:
%%   `pst-blur' is a PSTricks package for blurred shadows
%%
\csname PstBlurLoaded\endcsname
\let\PstBlurLoaded\endinput
\ifx\PSTricksLoaded\endinput\else
  \def\next{\input pstricks.tex }\expandafter\next
\fi
\ifx\PSTXKeyLoaded\endinput\else\input pst-xkey \fi
%%
\def\fileversion{2.0}
\def\filedate{2005/09/08}
\message{ v\fileversion, \filedate}
\edef\TheAtCode{\the\catcode`\@}
\catcode`\@=11
\pst@addfams{pst-blur}
\pstheader{pst-blur.pro}
\newif\ifpsblur
\define@key[psset]{pst-blur}{blur}[true]{\@nameuse{psblur#1}\pst@setrepeatarrowsflag}
\psset{blur=false}
%%
\define@key[psset]{pst-blur}{blurradius}{\pst@getlength{#1}\psx@blurradius}
\psset{blurradius=1.5pt}
%%
\define@key[psset]{pst-blur}{blursteps}{\pst@getint{#1}\psx@blursteps}
\psset{blursteps=20}
%%
\define@key[psset]{pst-blur}{blurbg}{\pst@getcolor{#1}\psx@blurbg}
\psset{blurbg=white}
\def\pst@closedshadow{%
\ifpsblur\pst@blurclosedshadow\else\pst@sharpclosedshadow\fi
}
\def\pst@sharpclosedshadow{%
  \addto@pscode{%
    gsave
    \psk@shadowsize \psk@shadowangle \tx@PtoC
    \tx@Shadow
    \pst@usecolor\psshadowcolor
    gsave fill grestore
    stroke
    grestore
    gsave
    \pst@usecolor\psfillcolor
    gsave fill grestore
    stroke
    grestore}}
\def\pst@blurclosedshadow{%
  \addto@pscode{%
    gsave
    gsave \pst@usecolor\psshadowcolor currentrgbcolor grestore
    gsave \pst@usecolor\psx@blurbg currentrgbcolor grestore
    \psx@blurradius\space
    \psx@blursteps\space
    \psk@shadowsize \psk@shadowangle \tx@PtoC
    tx@PstBlurDict begin BlurShadow end
    grestore
    gsave
    \pst@usecolor\psfillcolor
    gsave fill grestore
    stroke
    grestore}}
\def\psblurbox{%
\def\pst@par{}\pst@object{psblurbox}}
\def\psblurbox@i{\pst@makebox\psblurbox@ii}
\def\psblurbox@ii{%
  \begingroup
  \pst@useboxpar
  \psblurtrue
  \psshadowtrue
  \psboxseptrue
  \setbox\pst@hbox=\hbox{\psframebox@ii}%
  \pst@dimh=\psk@shadowsize\p@
  \pst@dimh=.7071\pst@dimh
  \advance\pst@dimh\psx@blurradius\p@
  \pst@dimg=\dp\pst@hbox
  \advance\pst@dimg\pst@dimh
  \dp\pst@hbox=\pst@dimg
  \pst@dimg=\wd\pst@hbox
  \advance\pst@dimg\pst@dimh
  \wd\pst@hbox=\pst@dimg
  \leavevmode
  \box\pst@hbox
\endgroup}
%%
\catcode`\@=\TheAtCode\relax
\endinput
%%
%% End of file `pst-blur.tex'.

\ProvidesFile{pst-blur.tex}
  [\filedate\space v\fileversion\space `PST-blur' (hv)]
%</stylefile>
%    \end{macrocode}
%
% \subsection{The \file{pst-blur.tex} file}
%    \file{pst-blur.tex} contains the \TeX-side of things.  We begin
%    by identifying ourselves and setting things up, the same as in 
%    other PSTricks packages.
%    \begin{macrocode}
%<*texfile>
\csname PstBlurLoaded\endcsname
\let\PstBlurLoaded\endinput
\ifx\PSTricksLoaded\endinput\else
  \def\next{\input pstricks.tex }\expandafter\next
\fi
%    \end{macrocode}
% \verb+pst-blur+ uses the extended version of the keyvalue interface.
%    \begin{macrocode}
\ifx\PSTXKeyLoaded\endinput\else\input pst-xkey \fi
%    \end{macrocode}
%%
%    \begin{macrocode}
\def\fileversion{2.0}
\def\filedate{2005/09/08}
\message{ v\fileversion, \filedate}
\edef\TheAtCode{\the\catcode`\@}
\catcode`\@=11
%    \end{macrocode}
% Add the package name to the list of family names of the keyvalue list.
%    \begin{macrocode}
\pst@addfams{pst-blur}
\pstheader{pst-blur.pro}
%    \end{macrocode}
%
% \subsubsection{New graphics parameters}
% \begin{macro}{blur}
% \begin{macro}{blurradius}
% \begin{macro}{blursteps}
% \begin{macro}{blurbg}
% The definitions of the new graphics parameters follow the definitions
% for parameters of the same types found in |pstricks.tex|.
%    \begin{macrocode}
\newif\ifpsblur
\define@key[psset]{pst-blur}{blur}[true]{\@nameuse{psblur#1}\pst@setrepeatarrowsflag}
\psset{blur=false}
%%
\define@key[psset]{pst-blur}{blurradius}{\pst@getlength{#1}\psx@blurradius}
\psset{blurradius=1.5pt}
%%
\define@key[psset]{pst-blur}{blursteps}{\pst@getint{#1}\psx@blursteps}
\psset{blursteps=20}
%%
\define@key[psset]{pst-blur}{blurbg}{\pst@getcolor{#1}\psx@blurbg}
\psset{blurbg=white}
%    \end{macrocode}
% \end{macro}
% \end{macro}
% \end{macro}
% \end{macro}
% \subsection{Hooking into the PSTricks shadow macros}
% \begin{macro}{\pst@closedshadow}
% The macro |\pst@closedshadow| is usually called internally by
% PSTricks to paint a shadow in the shape of the current path.  
% This macro has been renamed |\pst@sharpclosedshadow|.  The
% new |\pst@closedshadow| jumps to either of |\pst@sharpclosedshadow|
% or |\pst@blurclosedshadow|, depending on |\ifpsblur|, which is
% directly related to the graphics parameter |blur|.
%    \begin{macrocode}
\def\pst@closedshadow{%
\ifpsblur\pst@blurclosedshadow\else\pst@sharpclosedshadow\fi
}
\def\pst@sharpclosedshadow{%
  \addto@pscode{%
    gsave
    \psk@shadowsize \psk@shadowangle \tx@PtoC
    \tx@Shadow
    \pst@usecolor\psshadowcolor
    gsave fill grestore
    stroke
    grestore
    gsave
    \pst@usecolor\psfillcolor
    gsave fill grestore
    stroke
    grestore}}
%    \end{macrocode}
% \end{macro}
% \begin{macro}{\pst@blurclosedshadow}
% The PostScript code for blurred shadows is produced by the following
% macro.  It pushes the diverse parameters (|\tx@PtoC| does polar to
% cartesian coordinate transformation for the shadow offset) and calls
% |BlurShadow|.  Afterwards, it fills and strokes the current path,
% same as the original |\pst@closedshadow|.
%    \begin{macrocode}
\def\pst@blurclosedshadow{%
  \addto@pscode{%
    gsave
    gsave \pst@usecolor\psshadowcolor currentrgbcolor grestore
    gsave \pst@usecolor\psx@blurbg currentrgbcolor grestore
    \psx@blurradius\space
    \psx@blursteps\space
    \psk@shadowsize \psk@shadowangle \tx@PtoC
    tx@PstBlurDict begin BlurShadow end
    grestore
    gsave
    \pst@usecolor\psfillcolor
    gsave fill grestore
    stroke
    grestore}}
%    \end{macrocode}
% \end{macro}
% \begin{macro}{\pst@blurclosedshadow}
% This one looks very impressing.  In fact, it is a verbatim copy
% of |\psshadowbox|, with only the line 
% |\advance\pst@dimh\psx@blurradius\p@| added!
%    \begin{macrocode}
\def\psblurbox{%
\def\pst@par{}\pst@object{psblurbox}}
\def\psblurbox@i{\pst@makebox\psblurbox@ii}
\def\psblurbox@ii{%
  \begingroup
  \pst@useboxpar
  \psblurtrue
  \psshadowtrue
  \psboxseptrue
  \setbox\pst@hbox=\hbox{\psframebox@ii}%
  \pst@dimh=\psk@shadowsize\p@
  \pst@dimh=.7071\pst@dimh
  \advance\pst@dimh\psx@blurradius\p@
  \pst@dimg=\dp\pst@hbox
  \advance\pst@dimg\pst@dimh
  \dp\pst@hbox=\pst@dimg
  \pst@dimg=\wd\pst@hbox
  \advance\pst@dimg\pst@dimh
  \wd\pst@hbox=\pst@dimg
  \leavevmode
  \box\pst@hbox
\endgroup}
%%
\catcode`\@=\TheAtCode\relax
%</texfile>
%    \end{macrocode}
% \end{macro}
%
% \subsection{The \file{pst-blur.pro} file}
%    The file \file{pst-blur.pro} contains PostScript definitions
%    to be included  in the PostScript output by the 
%    |dvi|-to-PostScript converter, eg |dvips|. 
%    This is all rather similar to
%    \file{pst-slpe.pro}, and I just don't feel like explaining it,
%    so you'll have to work through it yourself, if you want to
%    know what happens.  The trick is basically to draw the outline
%    repeatedly with varying line widths.  The procedure |Shadow|
%    called in |BlurShadow| is defined in \file{pstricks.pro} and
%    translates the current path based on an $x$- and $y$-displacement
%    taken from the stack.
%    \begin{macrocode}
%<*prolog>
/tx@PstBlurDict 60 dict def
tx@PstBlurDict begin
/Iterate {
  /SegLines ED
  /ThisB ED /ThisG ED /ThisR ED
  /NextB ED /NextG ED /NextR ED
  /W 2.0 BlurRadius mul def
  /WDec W SegLines div def
  /RInc NextR ThisR sub SegLines div def
  /GInc NextG ThisG sub SegLines div def
  /BInc NextB ThisB sub SegLines div def
  /R ThisR def
  /G ThisG def
  /B ThisB def
  SegLines {
    R G B
    sqrt 3 1 roll sqrt 3 1 roll sqrt 3 1 roll
    setrgbcolor
    gsave W setlinewidth
    stroke grestore
    /W W WDec sub def
    /R R RInc add def
    /G G GInc add def
    /B B BInc add def
  } bind repeat
} def
/BlurShadow {
  Shadow
  /BlurSteps ED
  /BlurRadius ED
  dup mul /BEnd ED dup mul /GEnd ED dup mul /REnd ED 
  dup mul /BBeg ED dup mul /GBeg ED dup mul /RBeg ED 
  RBeg REnd add 0.5 mul /RMid ED
  GBeg GEnd add 0.5 mul /GMid ED
  BBeg BEnd add 0.5 mul /BMid ED
  /OuterSteps BlurSteps 2 div cvi def
  /InnerSteps BlurSteps OuterSteps sub def
  1 setlinejoin
  RMid GMid BMid REnd GEnd BEnd OuterSteps Iterate 
  gsave RBeg sqrt GBeg sqrt BBeg sqrt setrgbcolor fill grestore
  clip
  0 setlinejoin
  RMid GMid BMid RBeg GBeg BBeg InnerSteps Iterate 
} def
end
%</prolog>
%    \end{macrocode}
% \Finale
%
'
%\end{verbatim}
%
%\section{Package Usage}
% To use |pst-blur|, you have to say
% \begin{verbatim}
%   \usepackage{pst-blur}
% \end{verbatim}
% in the document prologue for \LaTeX, and 
% \begin{verbatim}
%   \input pst-blur.tex
% \end{verbatim}
% in ``plain'' \TeX.
%
% \DescribeMacro{blur}
% To paint shapes with blurred shadows,
% set the graphics parameters |shadow| and |blur| to |true|, eg
% \pscircle[shadow=true,blur=true](8,-0.5){0.5}
% \begin{verbatim}
%    \psset{unit=1cm}
%    \pscircle[shadow=true,blur=true](0,0){0.5}
% \end{verbatim}
% \psset{unit=1cm}
% for a circle with a blurred shadow.
% The parameter |blur| has no influence if |shadow| is |false|.
% \medskip
%
% \DescribeMacro{shadowsize}
% \DescribeMacro{shadowangle}
% \DescribeMacro{blurradius}
% The rendering of blurred shadows is controlled by a number of
% additional graphics parameters.  The offset of the shadow is controlled
% by the parameters |shadowsize| and |shadowangle|, which are the same 
% as for ordinary shadows.\footnote{In particular, |shadowangle| has
% to be negative for the usual placement of shadows below and to the
% right of shapes.}  The size of the blurring effect is
% controlled by the parameter |blurradius|, see Fig~\ref{fig:params}.
% The default value for |blurradius| is 1.5pt, which fits nicely with
% the default |shadowsize| of 3pt.
% \medskip
%
% \begin{figure}\caption{Parameters for blurred shadows} 
% \label{fig:params}
% \vskip1cm
% \qquad\qquad\begin{pspicture*}(0,0)(10,6)
%     \psframe[linewidth=4pt,fillcolor=lightgray,fillstyle=crosshatch,
%             	shadow=true,blur=true,shadowsize=2cm,shadowangle=-35,
%		blurradius=1cm,shadowcolor=lightgray](-5,3)(5,10)
%   \pnode(5,3){A}
%   \pnode(5.3,2.9){A1}
%   \pnode(6.64,1.85){B}
%   \pnode(7.51,2.35){C}
%   \pscircle(6.64,1.85){1}
%   \ncline{|-|}{A}{B}
%   \bput(0.2){|shadowsize|}
%   \ncline{->}{B}{C}
%   \bput(1.5){|blurradius|}
%   \psline(5,3)(6,3)
%   \psarcn{->}(5,3){0.5}{0}{-35}
%   \rput(7,3.7){\rnode{D}{$-|shadowangle|$}}
%   \nccurve[linewidth=0.5pt,angleA=-90,angleB=70]{->}{D}{A1}
%   \rput[l](6,5.6){|shadowcolor|}
%   \rput[l](8,5){|blurbg|}
%   {\psset{linestyle=dotted}
%    \psline(0,0.85)(6.64,0.85)
%    \psline(7.64,1.85)(7.64,6)
%    \psset{linestyle=dashed}
%    \psline(0,2.85)(5.64,2.85)
%    \psline(5.64,2.85)(5.64,6)}
%   {\psset{linewidth=0.5pt}
%    \psline{*-}(5.64,5.3)(6,5.4)
%    \psline{*-}(7.64,4.75)(8,4.85)
%    \ncline{F}{F1}
%   }
% \end{pspicture*}
% \end{figure}
%
% \DescribeMacro{shadowcolor}
% \DescribeMacro{blurbg}
% The inner, usually darkest part of the shadow is painted in the
% colour defined by |shadowcolor|.  In the range defined by |blurradius|,
% the colour gradually fades to the background colour set by |blurbg|.
% The default value for |blurbg| is white.  You should change this parameter
% when you want to paint shapes over a coloured background, ie\\
% \begin{minipage}{\textwidth}
% \begin{verbatim}
%   \psframe[fillstyle=solid,fillcolor=yellow](-.7,-.7)(.7,.7)
%   \pscircle[shadow=true,blur=true,blurbg=yellow](0,0){0.4}
% \end{verbatim}
% \rput(11.7,1.4){
%   \psframe[fillstyle=solid,fillcolor=yellow](-.7,-.7)(.7,.7)
%   \pscircle[shadow=true,blur=true,blurbg=yellow](0,0){0.4}}
% \end{minipage}
%
% \DescribeMacro{blursteps}
% The number of distinct colour steps painted between |shadowcolor|
% and |blurbg| is controlled by the parameter |blursteps|.  The default
% value for |blursteps| is 20, which is usually more than sufficient.
% Note, that higher values for |blursteps| result in proportionally slower
% rendering.  This can be very tiresome with complex shapes.
% \medskip
%
% \DescribeMacro{\psblurbox}
% Using a
% \psframebox[shadow=true,blur=true,blurradius=2.5pt,shadowcolor=gray]%
% {|\ttfamily\symbol{92}psframebox|} 
% with a blurred 
% shadow in the middle of some text produces poor results, because \TeX\
% does not know about the extra space taken by the shadow.  For normal
% shadows, this problem is solved by the |\psshadowbox| macro, which
% adds the extra space around the box for the shadow.  For blurred shadows,
% this is not sufficient: an extra |\blurradius| has to be added.  This 
% is done by the macro \psblurbox{\ttfamily\symbol{92}psblurbox}, which is otherwise
% identical to |\psshadowbox|.  Note, that |\psblurbox| shares a
% deficiency of |\psshadowbox|:  It only works correctly 
% with $|shadowangle|=-45$, because \TeX\ does not provide trigonometric
% operations.
%
% \StopEventually{}
%
% \section{The Code}
%
% \subsection{The \file{pst-blur.sty} file}
%    The \file{pst-blur.sty} file is very simple.  It just loads 
%    the generic \file{pst-blur.tex} file. 
%    \begin{macrocode}
%<*stylefile>
\RequirePackage{pstricks}
\ProvidesPackage{pst-blur}[2005/09/08 package wrapper for 
  pst-blur.tex (hv)]
%%
%% This is file `pst-blur.tex',
%% generated with the docstrip utility.
%%
%% The original source files were:
%%
%% pst-blur.dtx  (with options: `texfile')
%% 
%% IMPORTANT NOTICE:
%% 
%% For the copyright see the source file.
%% 
%% Any modified versions of this file must be renamed
%% with new filenames distinct from pst-blur.tex.
%% 
%% For distribution of the original source see the terms
%% for copying and modification in the file pst-blur.dtx.
%% 
%% This generated file may be distributed as long as the
%% original source files, as listed above, are part of the
%% same distribution. (The sources need not necessarily be
%% in the same archive or directory.)
%% $Id: pst-blur.dtx,v 2.0 2005/09/08 09:48:33 giese Exp $
%%
%% Copyright 1998 Martin Giese, Martin.Giese@oeaw.ac.at
%%           2005 Herbert Voss, voss@pstricks.de
%%
%% This file is under the LaTeX Project Public License
%% See CTAN archives in directory macros/latex/base/lppl.txt.
%%
%% DESCRIPTION:
%%   `pst-blur' is a PSTricks package for blurred shadows
%%
\csname PstBlurLoaded\endcsname
\let\PstBlurLoaded\endinput
\ifx\PSTricksLoaded\endinput\else
  \def\next{\input pstricks.tex }\expandafter\next
\fi
\ifx\PSTXKeyLoaded\endinput\else\input pst-xkey \fi
%%
\def\fileversion{2.0}
\def\filedate{2005/09/08}
\message{ v\fileversion, \filedate}
\edef\TheAtCode{\the\catcode`\@}
\catcode`\@=11
\pst@addfams{pst-blur}
\pstheader{pst-blur.pro}
\newif\ifpsblur
\define@key[psset]{pst-blur}{blur}[true]{\@nameuse{psblur#1}\pst@setrepeatarrowsflag}
\psset{blur=false}
%%
\define@key[psset]{pst-blur}{blurradius}{\pst@getlength{#1}\psx@blurradius}
\psset{blurradius=1.5pt}
%%
\define@key[psset]{pst-blur}{blursteps}{\pst@getint{#1}\psx@blursteps}
\psset{blursteps=20}
%%
\define@key[psset]{pst-blur}{blurbg}{\pst@getcolor{#1}\psx@blurbg}
\psset{blurbg=white}
\def\pst@closedshadow{%
\ifpsblur\pst@blurclosedshadow\else\pst@sharpclosedshadow\fi
}
\def\pst@sharpclosedshadow{%
  \addto@pscode{%
    gsave
    \psk@shadowsize \psk@shadowangle \tx@PtoC
    \tx@Shadow
    \pst@usecolor\psshadowcolor
    gsave fill grestore
    stroke
    grestore
    gsave
    \pst@usecolor\psfillcolor
    gsave fill grestore
    stroke
    grestore}}
\def\pst@blurclosedshadow{%
  \addto@pscode{%
    gsave
    gsave \pst@usecolor\psshadowcolor currentrgbcolor grestore
    gsave \pst@usecolor\psx@blurbg currentrgbcolor grestore
    \psx@blurradius\space
    \psx@blursteps\space
    \psk@shadowsize \psk@shadowangle \tx@PtoC
    tx@PstBlurDict begin BlurShadow end
    grestore
    gsave
    \pst@usecolor\psfillcolor
    gsave fill grestore
    stroke
    grestore}}
\def\psblurbox{%
\def\pst@par{}\pst@object{psblurbox}}
\def\psblurbox@i{\pst@makebox\psblurbox@ii}
\def\psblurbox@ii{%
  \begingroup
  \pst@useboxpar
  \psblurtrue
  \psshadowtrue
  \psboxseptrue
  \setbox\pst@hbox=\hbox{\psframebox@ii}%
  \pst@dimh=\psk@shadowsize\p@
  \pst@dimh=.7071\pst@dimh
  \advance\pst@dimh\psx@blurradius\p@
  \pst@dimg=\dp\pst@hbox
  \advance\pst@dimg\pst@dimh
  \dp\pst@hbox=\pst@dimg
  \pst@dimg=\wd\pst@hbox
  \advance\pst@dimg\pst@dimh
  \wd\pst@hbox=\pst@dimg
  \leavevmode
  \box\pst@hbox
\endgroup}
%%
\catcode`\@=\TheAtCode\relax
\endinput
%%
%% End of file `pst-blur.tex'.

\ProvidesFile{pst-blur.tex}
  [\filedate\space v\fileversion\space `PST-blur' (hv)]
%</stylefile>
%    \end{macrocode}
%
% \subsection{The \file{pst-blur.tex} file}
%    \file{pst-blur.tex} contains the \TeX-side of things.  We begin
%    by identifying ourselves and setting things up, the same as in 
%    other PSTricks packages.
%    \begin{macrocode}
%<*texfile>
\csname PstBlurLoaded\endcsname
\let\PstBlurLoaded\endinput
\ifx\PSTricksLoaded\endinput\else
  \def\next{\input pstricks.tex }\expandafter\next
\fi
%    \end{macrocode}
% \verb+pst-blur+ uses the extended version of the keyvalue interface.
%    \begin{macrocode}
\ifx\PSTXKeyLoaded\endinput\else\input pst-xkey \fi
%    \end{macrocode}
%%
%    \begin{macrocode}
\def\fileversion{2.0}
\def\filedate{2005/09/08}
\message{ v\fileversion, \filedate}
\edef\TheAtCode{\the\catcode`\@}
\catcode`\@=11
%    \end{macrocode}
% Add the package name to the list of family names of the keyvalue list.
%    \begin{macrocode}
\pst@addfams{pst-blur}
\pstheader{pst-blur.pro}
%    \end{macrocode}
%
% \subsubsection{New graphics parameters}
% \begin{macro}{blur}
% \begin{macro}{blurradius}
% \begin{macro}{blursteps}
% \begin{macro}{blurbg}
% The definitions of the new graphics parameters follow the definitions
% for parameters of the same types found in |pstricks.tex|.
%    \begin{macrocode}
\newif\ifpsblur
\define@key[psset]{pst-blur}{blur}[true]{\@nameuse{psblur#1}\pst@setrepeatarrowsflag}
\psset{blur=false}
%%
\define@key[psset]{pst-blur}{blurradius}{\pst@getlength{#1}\psx@blurradius}
\psset{blurradius=1.5pt}
%%
\define@key[psset]{pst-blur}{blursteps}{\pst@getint{#1}\psx@blursteps}
\psset{blursteps=20}
%%
\define@key[psset]{pst-blur}{blurbg}{\pst@getcolor{#1}\psx@blurbg}
\psset{blurbg=white}
%    \end{macrocode}
% \end{macro}
% \end{macro}
% \end{macro}
% \end{macro}
% \subsection{Hooking into the PSTricks shadow macros}
% \begin{macro}{\pst@closedshadow}
% The macro |\pst@closedshadow| is usually called internally by
% PSTricks to paint a shadow in the shape of the current path.  
% This macro has been renamed |\pst@sharpclosedshadow|.  The
% new |\pst@closedshadow| jumps to either of |\pst@sharpclosedshadow|
% or |\pst@blurclosedshadow|, depending on |\ifpsblur|, which is
% directly related to the graphics parameter |blur|.
%    \begin{macrocode}
\def\pst@closedshadow{%
\ifpsblur\pst@blurclosedshadow\else\pst@sharpclosedshadow\fi
}
\def\pst@sharpclosedshadow{%
  \addto@pscode{%
    gsave
    \psk@shadowsize \psk@shadowangle \tx@PtoC
    \tx@Shadow
    \pst@usecolor\psshadowcolor
    gsave fill grestore
    stroke
    grestore
    gsave
    \pst@usecolor\psfillcolor
    gsave fill grestore
    stroke
    grestore}}
%    \end{macrocode}
% \end{macro}
% \begin{macro}{\pst@blurclosedshadow}
% The PostScript code for blurred shadows is produced by the following
% macro.  It pushes the diverse parameters (|\tx@PtoC| does polar to
% cartesian coordinate transformation for the shadow offset) and calls
% |BlurShadow|.  Afterwards, it fills and strokes the current path,
% same as the original |\pst@closedshadow|.
%    \begin{macrocode}
\def\pst@blurclosedshadow{%
  \addto@pscode{%
    gsave
    gsave \pst@usecolor\psshadowcolor currentrgbcolor grestore
    gsave \pst@usecolor\psx@blurbg currentrgbcolor grestore
    \psx@blurradius\space
    \psx@blursteps\space
    \psk@shadowsize \psk@shadowangle \tx@PtoC
    tx@PstBlurDict begin BlurShadow end
    grestore
    gsave
    \pst@usecolor\psfillcolor
    gsave fill grestore
    stroke
    grestore}}
%    \end{macrocode}
% \end{macro}
% \begin{macro}{\pst@blurclosedshadow}
% This one looks very impressing.  In fact, it is a verbatim copy
% of |\psshadowbox|, with only the line 
% |\advance\pst@dimh\psx@blurradius\p@| added!
%    \begin{macrocode}
\def\psblurbox{%
\def\pst@par{}\pst@object{psblurbox}}
\def\psblurbox@i{\pst@makebox\psblurbox@ii}
\def\psblurbox@ii{%
  \begingroup
  \pst@useboxpar
  \psblurtrue
  \psshadowtrue
  \psboxseptrue
  \setbox\pst@hbox=\hbox{\psframebox@ii}%
  \pst@dimh=\psk@shadowsize\p@
  \pst@dimh=.7071\pst@dimh
  \advance\pst@dimh\psx@blurradius\p@
  \pst@dimg=\dp\pst@hbox
  \advance\pst@dimg\pst@dimh
  \dp\pst@hbox=\pst@dimg
  \pst@dimg=\wd\pst@hbox
  \advance\pst@dimg\pst@dimh
  \wd\pst@hbox=\pst@dimg
  \leavevmode
  \box\pst@hbox
\endgroup}
%%
\catcode`\@=\TheAtCode\relax
%</texfile>
%    \end{macrocode}
% \end{macro}
%
% \subsection{The \file{pst-blur.pro} file}
%    The file \file{pst-blur.pro} contains PostScript definitions
%    to be included  in the PostScript output by the 
%    |dvi|-to-PostScript converter, eg |dvips|. 
%    This is all rather similar to
%    \file{pst-slpe.pro}, and I just don't feel like explaining it,
%    so you'll have to work through it yourself, if you want to
%    know what happens.  The trick is basically to draw the outline
%    repeatedly with varying line widths.  The procedure |Shadow|
%    called in |BlurShadow| is defined in \file{pstricks.pro} and
%    translates the current path based on an $x$- and $y$-displacement
%    taken from the stack.
%    \begin{macrocode}
%<*prolog>
/tx@PstBlurDict 60 dict def
tx@PstBlurDict begin
/Iterate {
  /SegLines ED
  /ThisB ED /ThisG ED /ThisR ED
  /NextB ED /NextG ED /NextR ED
  /W 2.0 BlurRadius mul def
  /WDec W SegLines div def
  /RInc NextR ThisR sub SegLines div def
  /GInc NextG ThisG sub SegLines div def
  /BInc NextB ThisB sub SegLines div def
  /R ThisR def
  /G ThisG def
  /B ThisB def
  SegLines {
    R G B
    sqrt 3 1 roll sqrt 3 1 roll sqrt 3 1 roll
    setrgbcolor
    gsave W setlinewidth
    stroke grestore
    /W W WDec sub def
    /R R RInc add def
    /G G GInc add def
    /B B BInc add def
  } bind repeat
} def
/BlurShadow {
  Shadow
  /BlurSteps ED
  /BlurRadius ED
  dup mul /BEnd ED dup mul /GEnd ED dup mul /REnd ED 
  dup mul /BBeg ED dup mul /GBeg ED dup mul /RBeg ED 
  RBeg REnd add 0.5 mul /RMid ED
  GBeg GEnd add 0.5 mul /GMid ED
  BBeg BEnd add 0.5 mul /BMid ED
  /OuterSteps BlurSteps 2 div cvi def
  /InnerSteps BlurSteps OuterSteps sub def
  1 setlinejoin
  RMid GMid BMid REnd GEnd BEnd OuterSteps Iterate 
  gsave RBeg sqrt GBeg sqrt BBeg sqrt setrgbcolor fill grestore
  clip
  0 setlinejoin
  RMid GMid BMid RBeg GBeg BBeg InnerSteps Iterate 
} def
end
%</prolog>
%    \end{macrocode}
% \Finale
%
