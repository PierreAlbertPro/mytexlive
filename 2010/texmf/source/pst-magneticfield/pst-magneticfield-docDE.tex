%% $Id: pst-magneticfield-docDE.tex 347 2010-06-12 06:33:02Z herbert $
\documentclass[11pt,english,BCOR10mm,DIV12,bibliography=totoc,parskip=false,smallheadings
    headexclude,footexclude,oneside]{pst-doc}
\usepackage[latin1]{inputenc}
\usepackage{pst-magneticfield}
\let\pstMFfv\fileversion

%\newenvironment{postscript}{}{} % uncomment, when running with latex

\lstset{pos=t,language=PSTricks,
    morekeywords={psmagneticfield,psmagneticfieldThreeD},basicstyle=\footnotesize\ttfamily}
\newcommand\Cadre[1]{\psframebox[fillstyle=solid,fillcolor=black,linestyle=none,framesep=0]{#1}}
\def\bgImage{}
%
\begin{document}

\title{\texttt{pst-magneticfield}}
\subtitle{Magnetische Feldlinien einer langgestreckten Spule; v.\pstMFfv}
\author{J\"{u}rgen Gilg\\ Manuel Luque\\Herbert Vo\ss}
%\docauthor{J\"{u}rgen Gilg\\Manuel Luque\\Herbert Vo\ss}
\date{\today}
\maketitle


\clearpage%
\begin{abstract}
Das Paket \LPack{pst-magneticfield} zeichnet magnetische Feldlinien einer langgestreckten Spule. 
Die physikalischen Gr\"{o}{\ss}en sind: Radius der Spule, ihre L\"{a}nge und die Anzahl ihrer 
Windungen. Die voreingestellten Werte sind:

\begin{enumerate}
  \item Anzahl der Windungen: \LKeyset{N=6};
  \item Radius: \LKeyset{R=2};
  \item L\"{a}nge: \LKeyset{L=4}.
\end{enumerate}

Die magnetischen Feldlinien wurden mit dem Runge-Kutta 2 Verfahren angen\"{a}hert, welches sich 
nach einigen anderen Versuchen als der beste Kompromiss zwischen Re\-chen\-ge\-schwin\-dig\-keit und 
Zeichengenauigkeit der Linien erwies. Die Berechnung der notwendigen elliptischen Integrale 
wurden mit einer polynomialen N\"{a}herung aus dem  "Handbook of Mathematical Functions 
With Formulas, Graph, And Mathematical Tables" von Milton Abramowitz und Irene.\,A. Stegun 
(\url{http://www.math.sfu.ca/~cbm/aands/})~\cite{abramowitz} realisiert.
\end{abstract}

\clearpage
\tableofcontents

\clearpage
\section{Einleitung}

Im Folgenden stellen wir die Optionen mit ihren voreingestellten Werten vor:
\begin{enumerate}
  \item Die Maximalzahl von Berechnungspunkten einer jeden Feldlinie um die gesamte Spule: \LKeyset{pointsB=500};
  \item die Maximalzahl von Berechnungspunkten einer jeden Feldlinie um die Windungen: \LKeyset{pointsS=1000};
  \item die Anzahl der Feldlinien um die gesamte Spule: \LKeyset{nL=8};
  \item Schrittweite f\"{u}r die Feldlinien um die gesamte Spule: \LKeyset{PasB=0.02};
  \item Schrittweite f\"{u}r die Feldlinien um die Windungen: \LKeyset{PasS=0.00275};
  \item nur Feldlinien um individuell ausgew\"{a}hlte Windungen: \LKeyset{numSpires=\{\}}, nach dem Gleichheitsszeichen "=" schreiben wir die Nummer(n) der Windung(en) \textsf{1 2 3 etc.} ausgehend von der obersten Windung. Voreingestellt ist, dass bei allen Windungen die Feldlinien gezeichnet werden.
  \item Die Anzahl der Feldlinien um die gew\"{a}hlten Windungen: \LKeyset{nS=1}.
  \item Falls wir die Spule selbst nicht zeichnen m\"{o}chten, erledigt dies die Option \LKeyset{drawSelf=false} (hilfreich bei 3D-Ansichten).
  \item Die Optionen der Spule (Farbe, Linienst\"{a}rke, Pfeile) sind:
  \begin{enumerate}
        \item Die Farbe und Linienst\"{a}rke der Spule: \Lkeyset{styleSpire=styleSpire};
        \item die Stromst\"{a}rkepfeile: \Lkeyset{styleCourant=sensCourant}.
  \end{enumerate}
\begin{verbatim}
\newpsstyle{styleSpire}{linecap=1,linecolor=red,linewidth=2\pslinewidth}
\newpsstyle{sensCourant}{linecolor=red,linewidth=2\pslinewidth,arrowinset=0.1}
\end{verbatim}

 \item Die Farbe und Linienst\"{a}rke der Feldlinien can mit den g\"{a}ngigen Parametern von \LPack{pstricks} eingestellt werden: \Lkeyword{linecolor} und  \Lkeyword{linewidth}
\end{enumerate}

Der Befehl \Lcs{psmagneticfieldThreeD} erlaubt eine 3D-Ansicht der Spule und der magnetischen Feldlinien.

\clearpage
\section{Einfluss der physikalischen Gr\"{o}{\ss}en auf das Erscheinungsbild der Feldlinien}
\subsection{Die L\"{a}nge der Spule}

\begin{center}
\begin{postscript}
\psset{unit=0.5cm}
\begin{pspicture*}[showgrid](-7,-8)(7,8)
\psmagneticfield[linecolor={[HTML]{006633}},N=3,R=2,nS=1](-7,-8)(7,8)
\psframe*[linecolor={[HTML]{99FF66}}](-7,-8)(7,-7)
\rput(0,-7.5){[\Cadre{\textcolor{white}{L=4}},N=3,R=2,nS=1]}
\end{pspicture*}
\begin{pspicture*}[showgrid](-7,-8)(7,8)
\psmagneticfield[linecolor={[HTML]{006633}},L=8,N=3,R=2,nS=1,PasB=0.0025,pointsB=5500](-7,-8)(7,8)
\psframe*[linecolor={[HTML]{99FF66}}](-7,-8)(7,-7)
\rput(0,-7.5){[\Cadre{\textcolor{white}{L=8}},N=3,R=2,nS=1]}
\end{pspicture*}
\end{postscript}
\end{center}

\begin{lstlisting}
\psset{unit=0.5cm}
\begin{pspicture*}[showgrid](-7,-8)(7,8)
\psmagneticfield[linecolor={[HTML]{006633}},N=3,R=2,nS=1](-7,-8)(7,8)
\psframe*[linecolor={[HTML]{99FF66}}](-7,-8)(7,-7)
\rput(0,-7.5){[\Cadre{\textcolor{white}{L=4}},N=3,R=2,nS=1]}
\end{pspicture*}
\begin{pspicture*}[showgrid](-7,-8)(7,8)
\psmagneticfield[linecolor={[HTML]{006633}},L=8,N=3,R=2,nS=1,PasB=0.0025,pointsB=5500](-7,-8)(7,8)
\psframe*[linecolor={[HTML]{99FF66}}](-7,-8)(7,-7)
\rput(0,-7.5){[\Cadre{\textcolor{white}{L=8}},N=3,R=2,nS=1]}
\end{pspicture*}
\end{lstlisting}


\textbf{Anmerkung:} Um das Erscheinungsbild der zweiten Spule zu verbessern, mussten wir die Anzahl 
der Berechungspunkte erh\"{o}hen und die Schrittweite verkleinern, 
 \begin{postscript}
\Cadre{\textcolor{white}{pointsB=5500,PasB=0.0025}}
\end{postscript}, 
was jedoch eine Erh\"{o}hung der Rechenzeit mit sich brachte.


\clearpage

\subsection{Die Anzahl der Windungen}


\begin{center}
\begin{postscript}
\psset{unit=0.5}
\begin{pspicture*}[showgrid](-7,-8)(7,8)
\psmagneticfield[linecolor={[HTML]{006633}},N=1,R=2,nS=0](-7,-8)(7,8)
\psframe*[linecolor={[HTML]{99FF66}}](-7,-8)(7,-7)
\rput(0,-7.5){[\Cadre{\textcolor{white}{N=1}},R=2,nS=0]}
\end{pspicture*}
\begin{pspicture*}[showgrid](-7,-8)(7,8)
\psmagneticfield[linecolor={[HTML]{006633}},N=2,R=2,L=2,PasS=0.003,nS=2](-7,-8)(7,8)
\psframe*[linecolor={[HTML]{99FF66}}](-7,7)(7,8)
\rput(0,7.5){\Cadre{\textcolor{white}{Bobines de Helmholtz}}}
\psframe*[linecolor={[HTML]{99FF66}}](-7,-8)(7,-7)
\rput(0,-7.5){[\Cadre{\textcolor{white}{N=2}},R=2,L=2,PasS=0.003,nS=2]}
\end{pspicture*}
\end{postscript}
\end{center}

\begin{lstlisting}
\psset{unit=0.5}
\begin{pspicture*}[showgrid](-7,-8)(7,8)
\psmagneticfield[linecolor={[HTML]{006633}},N=1,R=2,nS=0](-7,-8)(7,8)
\psframe*[linecolor={[HTML]{99FF66}}](-7,-8)(7,-7)
\rput(0,-7.5){[\Cadre{\textcolor{white}{N=1}},R=2,nS=0]}
\end{pspicture*}
\begin{pspicture*}[showgrid](-7,-8)(7,8)
\psmagneticfield[linecolor={[HTML]{006633}},N=2,R=2,L=2,PasS=0.003,nS=2](-7,-8)(7,8)
\psframe*[linecolor={[HTML]{99FF66}}](-7,7)(7,8)
\rput(0,7.5){\Cadre{\textcolor{white}{Bobines de Helmholtz}}}
\psframe*[linecolor={[HTML]{99FF66}}](-7,-8)(7,-7)
\rput(0,-7.5){[\Cadre{\textcolor{white}{N=2}},R=2,L=2,PasS=0.003,nS=2]}
\end{pspicture*}
\end{lstlisting}


\begin{center}
\begin{postscript}
\psset{unit=0.5}
\begin{pspicture*}[showgrid](-7,-8)(7,8)
\psmagneticfield[linecolor={[HTML]{006633}},N=4,R=2,numSpires=2 3](-7,-8)(7,8)
\psframe*[linecolor={[HTML]{99FF66}}](-7,-8)(7,-7)
\rput(0,-7.5){[\Cadre{\textcolor{white}{N=4}},R=2,L=4]}
\end{pspicture*}
\begin{pspicture*}[showgrid](-7,-8)(7,8)
\psmagneticfield[linecolor={[HTML]{006633}},N=5,R=2,L=5,PasS=0.004,numSpires=2 3 4](-7,-8)(7,8)
\psframe*[linecolor={[HTML]{99FF66}}](-7,-8)(7,-7)
\rput(0,-7.5){[\Cadre{\textcolor{white}{N=5}},R=2,L=5]}
\end{pspicture*}
\end{postscript}
\end{center}

\begin{lstlisting}
\psset{unit=0.5}
\begin{pspicture*}[showgrid](-7,-8)(7,8)
\psmagneticfield[linecolor={[HTML]{006633}},N=4,R=2,numSpires=2 3](-7,-8)(7,8)
\psframe*[linecolor={[HTML]{99FF66}}](-7,-8)(7,-7)
\rput(0,-7.5){[\Cadre{\textcolor{white}{N=4}},R=2,L=4]}
\end{pspicture*}
\begin{pspicture*}[showgrid](-7,-8)(7,8)
\psmagneticfield[linecolor={[HTML]{006633}},N=5,R=2,L=5,PasS=0.004,numSpires=2 3 4](-7,-8)(7,8)
\psframe*[linecolor={[HTML]{99FF66}}](-7,-8)(7,-7)
\rput(0,-7.5){[\Cadre{\textcolor{white}{N=5}},R=2,L=5]}
\end{pspicture*}
\end{lstlisting}



\clearpage
\section{Optionen f\"{u}r die Linien}
\subsection{Die Anzahl der Feldlinien}

Auf Grund der Symmetrie des Problems ist die gew\"{a}hlte Anzahl der Feldlinien \Lkeyword{nL} nur die H\"{a}lfte der tats\"{a}chlich gezeichneten Feldlinien. Hinzu kommt noch eine Feldlinie, die in Richtung der Symmetrieachse der Spule zeigt. Die Anzahl der Feldlinien um die Windungen herum \Lkeyword{nS} kommen auch noch hinzu, diese k\"{o}nnen jedoch mit \Lkeyword{numSpires} individuell ausgew\"{a}hlt werden.



\begin{center}
\begin{postscript}
\psset{unit=0.5}
\begin{pspicture*}[showgrid](-7,-8)(7,8)
\psmagneticfield[linecolor={[HTML]{000099}},N=1,R=2](-7,-8)(7,8)
\psframe*[linecolor={[HTML]{3399FF}}](-7,-8)(7,-7)
\rput(0,-7.5){[\Cadre{\textcolor{white}{nL=8}},N=1,R=2]}
\end{pspicture*}
\begin{pspicture*}[showgrid](-7,-8)(7,8)
\psmagneticfield[linecolor={[HTML]{000099}},N=1,R=2,nL=12](-7,-8)(7,8)
\psframe*[linecolor={[HTML]{3399FF}}](-7,-8)(7,-7)
\rput(0,-7.5){[\Cadre{\textcolor{white}{nL=12}},N=1,R=2]}
\end{pspicture*}
\end{postscript}
\end{center}

\begin{lstlisting}
\psset{unit=0.5}
\begin{pspicture*}[showgrid](-7,-8)(7,8)
\psmagneticfield[linecolor={[HTML]{000099}},N=1,R=2](-7,-8)(7,8)
\psframe*[linecolor={[HTML]{3399FF}}](-7,-8)(7,-7)
\rput(0,-7.5){[\Cadre{\textcolor{white}{nL=8}},N=1,R=2]}
\end{pspicture*}
\begin{pspicture*}[showgrid](-7,-8)(7,8)
\psmagneticfield[linecolor={[HTML]{000099}},N=1,R=2,nL=12](-7,-8)(7,8)
\psframe*[linecolor={[HTML]{3399FF}}](-7,-8)(7,-7)
\rput(0,-7.5){[\Cadre{\textcolor{white}{nL=12}},N=1,R=2]}
\end{pspicture*}
\end{lstlisting}

\clearpage
\subsection{Die Anzahl der Berechnungspunkte und die Schrittweite}

Die Feldlinien wurden mit einem numerischen Verfahren (Runge-Kutta 2) berechnet und dementsprechend 
h\"{a}ngt die Genauigkeit der Linien entscheidend ab von der Schrittweite und der Anzahl der 
Berechnungspunkte, wie in den folgenden zwei Beispielen gezeigt wird:

\begin{center}
\begin{postscript}
\psset{unit=0.5}
\begin{pspicture*}[showgrid](-7,-8)(7,8)
\psmagneticfield[linecolor={[HTML]{660066}},N=2,R=2,L=2,PasB=0.1,nS=0,nL=7,pointsB=100](-7,-8)(7,8)
\psframe*[linecolor={[HTML]{996666}}](-7,7)(7,8)
\rput(0,7.5){\Cadre{\textcolor{white}{Bobines de Helmholtz}}}
\psframe*[linecolor={[HTML]{996666}}](-7,-8)(7,-7)
\rput(0,-7.5){[\Cadre{\textcolor{white}{PasB=0.1,nL=4,pointsB=100}}]}
\end{pspicture*}
\begin{pspicture*}[showgrid](-7,-8)(7,8)
\psmagneticfield[linecolor={[HTML]{660066}},N=2,R=2,L=2,PasB=0.4,nS=0,nL=7,pointsB=100](-7,-8)(7,8)
\psframe*[linecolor={[HTML]{996666}}](-7,7)(7,8)
\rput(0,7.5){\Cadre{\textcolor{white}{Bobines de Helmholtz}}}
\psframe*[linecolor={[HTML]{996666}}](-7,-8)(7,-7)
\rput(0,-7.5){[\Cadre{\textcolor{white}{PasS=0.4,pointsB=100}}]}
\end{pspicture*}
\end{postscript}
\end{center}

\begin{lstlisting}
\psset{unit=0.5}
\begin{pspicture*}[showgrid](-7,-8)(7,8)
\psmagneticfield[linecolor={[HTML]{660066}},N=2,R=2,L=2,PasB=0.1,nS=0,nL=7,pointsB=100](-7,-8)(7,8)
\psframe*[linecolor={[HTML]{996666}}](-7,7)(7,8)
\rput(0,7.5){\Cadre{\textcolor{white}{Bobines de Helmholtz}}}
\psframe*[linecolor={[HTML]{996666}}](-7,-8)(7,-7)
\rput(0,-7.5){[\Cadre{\textcolor{white}{PasB=0.1,nL=4,pointsB=100}}]}
\end{pspicture*}
\begin{pspicture*}[showgrid](-7,-8)(7,8)
\psmagneticfield[linecolor={[HTML]{660066}},N=2,R=2,L=2,PasB=0.4,nS=0,nL=7,pointsB=100](-7,-8)(7,8)
\psframe*[linecolor={[HTML]{996666}}](-7,7)(7,8)
\rput(0,7.5){\Cadre{\textcolor{white}{Bobines de Helmholtz}}}
\psframe*[linecolor={[HTML]{996666}}](-7,-8)(7,-7)
\rput(0,-7.5){[\Cadre{\textcolor{white}{PasS=0.4,pointsB=100}}]}
\end{pspicture*}
\end{lstlisting}


Sollten die voreingestellten Werte f\"{u}r eine individuelle Gestaltung nicht passen, dann muss man 
mit den Werten \Lkeyword{pasB}, \Lkeyword{pointsB} (bzw. \Lkeyword{pasS}, \Lkeyword{pointsS}) spielen, bis es passt.




\clearpage

\section{Der Parameter \nxLkeyword{numSpires}}
\begin{center}
\begin{postscript}
\psset{unit=0.5}
\begin{pspicture*}[showgrid](-8,-10)(8,10)
\psset{linecolor=blue}
\psmagneticfield[R=2,L=12,N=8,pointsS=500,nL=14,nS=1,numSpires=1 3 6 8,PasB=0.075](-8,-10)(8,10)
\psframe*[linecolor={[HTML]{99FF66}}](-8,-10)(8,-9)
\rput(0,-9.5){[\Cadre{\textcolor{white}{numSpires=1 3 6 8}},R=2,L=14]}
\multido{\i=0+1}{8}{\rput[l](!6 6 12 7 div \i\space mul sub){\the\multidocount}}
\end{pspicture*}\quad
\begin{pspicture*}[showgrid](0,-10)(16,10)
\psset{linecolor=blue}
\psmagneticfield[R=2,L=12,N=8,pointsS=500,nL=14,numSpires=,nS=1,PasB=0.075](0,-10)(16,10)
\psframe*[linecolor={[HTML]{99FF66}}](0,-10)(16,-9)
\rput(8,-9.5){[\Cadre{\textcolor{white}{numSpires=all}},R=2,L=14]}
\multido{\i=0+1}{8}{\rput[l](!6 6 12 7 div \i\space mul sub){\the\multidocount}}
\end{pspicture*}
\end{postscript}
\end{center}


\begin{lstlisting}
\psset{unit=0.5}
\begin{pspicture*}[showgrid](-8,-10)(8,10)
\psset{linecolor=blue}
\psmagneticfield[R=2,L=12,N=8,pointsS=500,nL=14,nS=1,numSpires=1 3 6 8,PasB=0.075](-8,-10)(8,10)
\psframe*[linecolor={[HTML]{99FF66}}](-8,-10)(8,-9)
\rput(0,-9.5){[\Cadre{\textcolor{white}{numSpires=1 3 6 8}},R=2,L=14]}
\multido{\i=0+1}{8}{\rput[l](!6 6 12 7 div \i\space mul sub){\the\multidocount}}
\end{pspicture*}\quad
\begin{pspicture*}[showgrid](0,-10)(16,10)
\psset{linecolor=blue}
\psmagneticfield[R=2,L=12,N=8,pointsS=500,nL=14,numSpires=,nS=1,PasB=0.075](0,-10)(16,10)
\psframe*[linecolor={[HTML]{99FF66}}](0,-10)(16,-9)
\rput(8,-9.5){[\Cadre{\textcolor{white}{numSpires=all}},R=2,L=14]}
\multido{\i=0+1}{8}{\rput[l](!6 6 12 7 div \i\space mul sub){\the\multidocount}}
\end{pspicture*}
\end{lstlisting}

\clearpage
\section{Der Parameter \nxLkeyword{AntiHelmholtz}}
\begin{center}
\begin{postscript}
\psset{unit=0.75,AntiHelmholtz,N=2,
  R=2,pointsB=500,pointsS=1000,PasB=0.02,PasS=0.00275,nS=10,
  nL=2,drawSelf=true,styleSpire=styleSpire,styleCourant=sensCourant}
\newpsstyle{grille}{subgriddiv=0,gridcolor=blue!50,griddots=10}
\newpsstyle{cadre}{linecolor=yellow!50}
\begin{pspicture*}[showgrid](-7,-6)(7,6)
\psframe*[linecolor={[HTML]{996666}}](-7,6)(7,6)
\psmagneticfield[linecolor={[HTML]{660066}}]
\end{pspicture*}
\end{postscript}
\end{center}

\begin{lstlisting}
\psset{unit=0.75,AntiHelmholtz,N=2,
  R=2,pointsB=500,pointsS=1000,PasB=0.02,PasS=0.00275,nS=10,
  nL=2,drawSelf=true,styleSpire=styleSpire,styleCourant=sensCourant}
\newpsstyle{grille}{subgriddiv=0,gridcolor=blue!50,griddots=10}
\newpsstyle{cadre}{linecolor=yellow!50}
\begin{pspicture*}[showgrid](-7,-6)(7,6)
\psframe*[linecolor={[HTML]{996666}}](-7,6)(7,6)
\psmagneticfield[linecolor={[HTML]{660066}}]
\end{pspicture*}
\end{lstlisting}


\clearpage
\section{3D-Ansichten}
3D-Ansichten sind mit den zwei folgenden Makros m\"{o}glich

\begin{BDef}
\Lcs{psmagneticfield}\OptArgs\coord1\coord2\\
\Lcs{psmagneticfieldThreeD}\OptArgs\coord1\coord2
\end{BDef}

in denen die in den vorigen Abschnitten besprochenen Parameter die Optionen von \Lcs{psmagneticfield} 
darstellen und mit \verb+(x1,y1)(x2,y2)+ werden die
Koordinaten der linken unteren und rechten oberen Ecke des Gitternetzes festgelegt, welches das 
Feldlinienbild einrahmt wie mit \Lcs{psframe}. Wir k\"{o}nnen die Option \Lkeyword{viewpoint} 
des Pakets \LPack{pst-3d} nutzen, um den Ansichtspunkt zu w\"{a}hlen/\"{a}ndern.
 Die voreingestellten Parameter f\"{u}r das Gitternetz sind:

\begin{verbatim}
\newpsstyle{grille}{subgriddiv=0,gridcolor=lightgray,griddots=10}
\newpsstyle{cadre}{linecolor=green!20}
\end{verbatim}

M\"{o}glichkeiten zur Gestaltung des Gitternetzes zeigen die folgenden zwei Beispiele:


\begin{center}
\begin{postscript}
\psset{unit=0.7cm}
\newpsstyle{grille}{subgriddiv=0,gridcolor=blue!50,griddots=10}
\newpsstyle{cadre}{linecolor=yellow!50}
\begin{pspicture}(-7,-6)(7,6)
\psmagneticfieldThreeD[N=8,R=2,L=8,pointsB=1200,linecolor=blue,pointsS=2000](-7,-6)(7,6)
\end{pspicture}
\end{postscript}
\end{center}

\begin{lstlisting}
\psset{unit=0.7cm}
\newpsstyle{grille}{subgriddiv=0,gridcolor=blue!50,griddots=10}
\newpsstyle{cadre}{linecolor=yellow!50}
\begin{pspicture}(-7,-6)(7,6)
\psmagneticfieldThreeD[N=8,R=2,L=8,pointsB=1200,linecolor=blue,pointsS=2000](-7,-6)(7,6)
\end{pspicture}
\end{lstlisting}


\begin{center}
\begin{postscript}
\psset{unit=0.7cm}
\begin{pspicture}(-7,-6)(7,6)
\psmagneticfieldThreeD[N=2,R=2,L=2,linecolor=blue](-7,-6)(7,6)
\ThreeDput{\rput(0,-7){\textbf{Bobines de HELMHOLTZ}}}
\end{pspicture}
\end{postscript}
\end{center}

\begin{lstlisting}
\psset{unit=0.7cm}
\begin{pspicture}(-7,-6)(7,6)
\psmagneticfieldThreeD[N=2,R=2,L=2,linecolor=blue](-7,-6)(7,6)
\ThreeDput{\rput(0,-7){\textbf{Bobines de HELMHOLTZ}}}
\end{pspicture}
\end{lstlisting}

\begin{center}
\begin{postscript}
\psset{unit=0.75cm,AntiHelmholtz,N=2,
  R=2,pointsB=500,pointsS=1000,PasB=0.02,PasS=0.00275,nS=10,
  nL=2,drawSelf,styleSpire=styleSpire,styleCourant=sensCourant}
\newpsstyle{grille}{subgriddiv=0,gridcolor=blue!50,griddots=10}
\newpsstyle{cadre}{linecolor=yellow!50}
\begin{pspicture}(-7,-6)(7,6)
\psmagneticfieldThreeD[linecolor={[HTML]{660066}}](-7,-6)(7,6)
\end{pspicture}
\end{postscript}
\end{center}

\begin{lstlisting}
\psset{unit=0.75cm,AntiHelmholtz,N=2,
  R=2,pointsB=500,pointsS=1000,PasB=0.02,PasS=0.00275,nS=10,
  nL=2,drawSelf,styleSpire=styleSpire,styleCourant=sensCourant}
\newpsstyle{grille}{subgriddiv=0,gridcolor=blue!50,griddots=10}
\newpsstyle{cadre}{linecolor=yellow!50}
\begin{pspicture}(-7,-6)(7,6)
\psmagneticfieldThreeD[linecolor={[HTML]{660066}}](-7,-6)(7,6)
\end{pspicture}
\end{lstlisting}



\section{Feldst\"arkendichte}

\begin{center}
\begin{postscript}
\begin{pspicture}(-6,-4)(6,4)
\psmagneticfield[N=3,R=2,L=2,StreamDensityPlot](-6,-4)(6,4)
\end{pspicture}
\end{postscript}
\end{center}

\begin{lstlisting}
\begin{pspicture}(-6,-4)(6,4)
\psmagneticfield[N=3,R=2,L=2,StreamDensityPlot](-6,-4)(6,4)
\end{pspicture}
\end{lstlisting}

\begin{center}
\begin{postscript}
\psset{unit=0.75}
\begin{pspicture}(-6,-5)(6,5)
\psmagneticfield[N=2,R=2,L=1,StreamDensityPlot,setgray](-6,-5)(6,5)
\end{pspicture}
\end{postscript}
\end{center}

\begin{lstlisting}
\psset{unit=0.75}
\begin{pspicture}(-6,-5)(6,5)
\psmagneticfield[N=2,R=2,L=1,StreamDensityPlot,setgray](-6,-5)(6,5)
\end{pspicture}
\end{lstlisting}


\begin{center}
\begin{postscript}
\psset{unit=0.75,AntiHelmholtz,
  R=2,pointsB=500,pointsS=2000,PasB=0.02,PasS=0.00275,nS=10,
  nL=2,drawSelf=true,styleSpire=styleSpire,styleCourant=sensCourant}
\begin{pspicture*}(-7,-6)(7,6)
\psmagneticfield[linecolor={[HTML]{660066}},StreamDensityPlot](-7,-6)(7,6)
\end{pspicture*}
\end{postscript}
\end{center}


\begin{lstlisting}
\psset{unit=0.75,AntiHelmholtz,
  R=2,pointsB=500,pointsS=2000,PasB=0.02,PasS=0.00275,nS=10,
  nL=2,drawSelf=true,styleSpire=styleSpire,styleCourant=sensCourant}
\begin{pspicture*}(-7,-6)(7,6)
\psmagneticfield[linecolor={[HTML]{660066}},StreamDensityPlot](-7,-6)(7,6)
\end{pspicture*}
\end{lstlisting}


\clearpage
\section{Liste aller optionalen Parameter von \texttt{pst-magneticfield}}

\xkvview{family=pst-magneticfield,columns={key,type,default}}

\nocite{*}
\bgroup
\raggedright
\bibliographystyle{plain}
\bibliography{pst-magneticfield-doc}
\egroup

\printindex
\end{document}