%% $Id: pstricks-add-doc.tex 612 2011-12-20 11:54:08Z herbert $
\documentclass[11pt,english,BCOR10mm,DIV12,bibliography=totoc,parskip=false,smallheadings
    headexclude,footexclude,oneside]{pst-doc}
\listfiles

\makeatletter
\RequirePackage{ltxcmds}[2010/01/28]
\@ifpackagelater{ltxcmds}{2010/03/09}{}{%
  \def\ltx@pkgextension{sty}%
}
\makeatother

%% $Id: pstricks-add-doc.tex 612 2011-12-20 11:54:08Z herbert $
\documentclass[11pt,english,BCOR10mm,DIV12,bibliography=totoc,parskip=false,smallheadings
    headexclude,footexclude,oneside]{pst-doc}
\listfiles

\makeatletter
\RequirePackage{ltxcmds}[2010/01/28]
\@ifpackagelater{ltxcmds}{2010/03/09}{}{%
  \def\ltx@pkgextension{sty}%
}
\makeatother

%% $Id: pstricks-add-doc.tex 612 2011-12-20 11:54:08Z herbert $
\documentclass[11pt,english,BCOR10mm,DIV12,bibliography=totoc,parskip=false,smallheadings
    headexclude,footexclude,oneside]{pst-doc}
\listfiles

\makeatletter
\RequirePackage{ltxcmds}[2010/01/28]
\@ifpackagelater{ltxcmds}{2010/03/09}{}{%
  \def\ltx@pkgextension{sty}%
}
\makeatother

%% $Id: pstricks-add-doc.tex 612 2011-12-20 11:54:08Z herbert $
\documentclass[11pt,english,BCOR10mm,DIV12,bibliography=totoc,parskip=false,smallheadings
    headexclude,footexclude,oneside]{pst-doc}
\listfiles

\makeatletter
\RequirePackage{ltxcmds}[2010/01/28]
\@ifpackagelater{ltxcmds}{2010/03/09}{}{%
  \def\ltx@pkgextension{sty}%
}
\makeatother

\input{pstricks-add-doc.dat}

\usepackage[utf8]{inputenc}
\usepackage{pstricks-add}
\let\pstricksaddFV\fileversion
\usepackage{pst-eucl,pst-fun,pst-func,multirow}
\usepackage{pifont}
\let\belowcaptionskip\abovecaptionskip
%
\def\textat{\char064}%
\newdimen\fullWidth
\makeatletter
\renewcommand*\l@section{\@dottedtocline{1}{2em}{2.3em}}
\renewcommand*\l@subsection{\@dottedtocline{2}{3.8em}{3.2em}}
\renewcommand*\l@subsubsection{\@dottedtocline{3}{7.0em}{4.1em}}
\renewcommand*\l@paragraph{\@dottedtocline{4}{10em}{5em}}
\makeatother
\lstset{explpreset={pos=l,width=-99pt,overhang=0pt,hsep=\columnsep,vsep=\bigskipamount,rframe={}},
    escapechar=§}

\def\bgImage{\psset{unit=1.5}
\begin{pspicture}(-3,-3)(3,3)
\psChart[userColor={red!30,green!30,blue!40,gray,cyan!50,
    magenta!60,cyan},chartSep=30pt,shadow=true,shadowsize=5pt]{34.5,17.2,20.7,15.5,5.2,6.9}{6}{2}
\psset{nodesepA=5pt,nodesepB=-10pt}
\ncline{psChartO1}{psChart1}\nput{0}{psChartO1}{1000 (34.5\%)}
\ncline{psChartO2}{psChart2}\nput{150}{psChartO2}{500 (17.2\%)}
\ncline{psChartO3}{psChart3}\nput{-90}{psChartO3}{600 (20.7\%)}
\ncline{psChartO4}{psChart4}\nput{0}{psChartO4}{450 (15.5\%)}
\ncline{psChartO5}{psChart5}\nput{0}{psChartO5}{150 (5.2\%)}
\ncline{psChartO6}{psChart6}\nput{0}{psChartO6}{200 (6.9\%)}
\bfseries%
\rput(psChartI1){Taxes}\rput(psChartI2){Rent}\rput(psChartI3){Bills}
\rput(psChartI4){Car}\rput(psChartI5){Gas}\rput(psChartI6){Food}
\end{pspicture}}

\begin{document}
\title{\texttt{pstricks-add}\\additionals Macros for \texttt{pstricks}\\
    \small v.\pstricksaddFV}
%\docauthor{Herbert Vo\ss}
\author{Dominique Rodriguez\\Michael Sharpe\\Herbert Vo\ss}
\date{\today}

\maketitle

\fullWidth=\linewidth
\advance\fullWidth by \marginparsep
\advance\fullWidth by \marginparwidth


\begin{abstract}
This version of \verb+pstricks-add+ needs \verb+pstricks.tex+
version >1.04 from June 2004, otherwise the additional macros may
not work as expected. The ellipsis material and the option
\verb+asolid+ (renamed to \verb+eofill+) are
\index{fillstyle!eofill@\texttt{eofill}} now part of the new
\verb+pstricks.tex+ package, available on CTAN. \LPack{pstricks-add} will for ever be
an experimental and dynamical package, try it at your own risk.

\begin{itemize}
\item It is important to load \LPack{pstricks-add} as the \textbf{last} PSTricks related package, otherwise
a lot of the macros won't work in the expected way.
\item \LPack{pstricks-add} uses the extended version of the keyval package. So be sure that
you have installed \LPack{pst-xkey} which is part of the
\LPack{xkeyval}-package, and that all packages that use the old
keyval interface are loaded \textbf{before} the
\LPack{xkeyval}.\cite{xkeyval}
\item the option \Lkeyword{tickstyle} from \LPack{pst-plot} is no longer supported; use \Lkeyword{ticksize} instead.
\item the option \Lkeyword{xyLabel} is no longer supported; use the option \Lkeyword{labelFontSize} instead.
\item if \LPack{pstricks-add} is loaded together with the package  \LPack{pst-func} then  \Lkeyword{InsideArrow}
    of the \Lcs{psbezier} macro doesn't work!
\end{itemize}

\vfill
\noindent
Thanks to:  
Hendri Adriaens;
Stefano Baroni;
Martin Chicoine;
Gerry Coombes;
Ulrich Dirr;
Christophe Fourey;
Hubert G\"a\ss lein;
J\"urgen Gilg;
Denis Girou;
Pablo Gonzáles;
Peter Hutnick;
Christophe Jorssen;
Uwe Kern;
Manuel Luque;
Jens-Uwe Morawski;
Tobias N\"ahring;
Rolf Niepraschk;
Alan Ristow;
Christine R\"omer;
Arnaud Schmittbuhl;
John Smith;
Timothy Van Zandt
\end{abstract}

\clearpage
\tableofcontents


\clearpage

\section{\nxLcs{psGetSlope} and \nxLcs{psGetDistance}}
%--------------------------------------------------------------------------------------

\begin{BDef}
\Lcs{psGetSlope}\coord1\coord2\Lcs{\Larga{macro}}\\
\Lcs{psGetDistance}\coord1\coord2\Lcs{\Larga{macro}}
\end{BDef}

\begin{LTXexample}[width=4cm]
\psGetSlope(-2,1)(3,1)\SlopeVal \SlopeVal \quad
\psGetDistance(-2,1)(3,1)\DVal \DVal\\
\psGetSlope(-2,1)(-3,-1)\SlopeVal \SlopeVal\quad
\psGetDistance(-2,1)(-3,-1)\DVal \DVal\\
\psGetSlope(-2,0)(3,-1)\SlopeVal \SlopeVal\quad
\psGetDistance(-2,0)(3,-1)\DVal \DVal\\
\psGetSlope(-2111,-12)(3,1)\SlopeVal \SlopeVal\quad
%\psGetDistance(-2111,-12)(3,1)\DVal ==> Overflow!
\end{LTXexample}

\clearpage

%--------------------------------------------------------------------------------------
\section{"`Handmade"' lines :-)}
%--------------------------------------------------------------------------------------

\begin{BDef}
\Lcs{pslineByHand}\OptArgs\coord1\coord2\coord3 \ldots
\end{BDef}

\begin{LTXexample}[width=0.4\linewidth]
\begin{pspicture}(4,6)
\psset{unit=2cm}
  \pslineByHand[linecolor=red](0,0)(0,2)(2,2)(2,0)(0,0)(2,2)(1,3)(0,2)(2,0)
\end{pspicture}
\end{LTXexample}

\iffalse
  \pslineByHand( 1.20, 1.50)( 1.20, 1.51)( 1.20, 1.53)( 1.20, 1.54)( 1.19, 1.55)( 1.19, 1.56)
    ( 1.19, 1.57)( 1.18, 1.59)( 1.18, 1.60)( 1.17, 1.61)( 1.16, 1.62)( 1.15, 1.63)( 1.15, 1.64)
    ( 1.14, 1.65)( 1.13, 1.65)( 1.12, 1.66)( 1.11, 1.67)( 1.10, 1.68)( 1.09, 1.68)( 1.07, 1.69)
    ( 1.06, 1.69)( 1.05, 1.69)( 1.04, 1.70)( 1.03, 1.70)( 1.01, 1.70)( 1.00, 1.70)( 0.99, 1.70)
    ( 0.97, 1.70)( 0.96, 1.70)( 0.95, 1.69)( 0.94, 1.69)( 0.93, 1.69)( 0.91, 1.68)( 0.90, 1.68)
    ( 0.89, 1.67)( 0.88, 1.66)( 0.87, 1.65)( 0.86, 1.65)( 0.85, 1.64)( 0.85, 1.63)( 0.84, 1.62)
    ( 0.83, 1.61)( 0.82, 1.60)( 0.82, 1.59)( 0.81, 1.57)( 0.81, 1.56)( 0.81, 1.55)( 0.80, 1.54)
    ( 0.80, 1.53)( 0.80, 1.51)( 0.80, 1.50)( 0.80, 1.49)( 0.80, 1.47)( 0.80, 1.46)( 0.81, 1.45)
    ( 0.81, 1.44)( 0.81, 1.43)( 0.82, 1.41)( 0.82, 1.40)( 0.83, 1.39)( 0.84, 1.38)( 0.85, 1.37)
    ( 0.85, 1.36)( 0.86, 1.35)( 0.87, 1.35)( 0.88, 1.34)( 0.89, 1.33)( 0.90, 1.32)( 0.91, 1.32)
    ( 0.93, 1.31)( 0.94, 1.31)( 0.95, 1.31)( 0.96, 1.30)( 0.97, 1.30)( 0.99, 1.30)( 1.00, 1.30)
    ( 1.01, 1.30)( 1.03, 1.30)( 1.04, 1.30)( 1.05, 1.31)( 1.06, 1.31)( 1.07, 1.31)( 1.09, 1.32)
    ( 1.10, 1.32)( 1.11, 1.33)( 1.12, 1.34)( 1.13, 1.35)( 1.14, 1.35)( 1.15, 1.36)( 1.15, 1.37)
    ( 1.16, 1.38)( 1.17, 1.39)( 1.18, 1.40)( 1.18, 1.41)( 1.19, 1.43)( 1.19, 1.44)( 1.19, 1.45)
    ( 1.20, 1.46)( 1.20, 1.47)( 1.20, 1.49)( 1.20, 1.50)
\fi

\begin{LTXexample}[pos=t]
\begin{pspicture}(\linewidth,3)
\multido{\rA=0.00+0.25}{12}{\pslineByHand[linecolor=blue](0,\rA)(\linewidth,\rA)}
\end{pspicture}
\end{LTXexample}

The amplitude and the width can be changed by the optional arguments \Lkeyword{varsteptol} and
\Lkeyword{VarStepEpsilon}. Both are preset to \verb+VarStepEpsilon=2,varsteptol=0.8+.


\begin{LTXexample}[pos=t]
\begin{pspicture}(\linewidth,3)
\multido{\rA=0.00+0.25}{12}{%
  \pslineByHand[linecolor=blue,VarStepEpsilon=4,varsteptol=2](0,\rA)(\linewidth,\rA)}
\end{pspicture}
\end{LTXexample}

\clearpage

%--------------------------------------------------------------------------------------
\section{\nxLcs{rmultiput}: a multiple \nxLcs{rput}}
%--------------------------------------------------------------------------------------
\verb+PSTricks+ already has a \Lcs{multirput}, which puts a box n
times with a difference of $dx$ and $dy$ relative to each other.
It is not possible to put it with a different distance from one
point to the next. This is possible with \Lcs{rmultiput}:

\begin{BDef}
\LcsStar{rmultiput}\OptArgs\Largb{any material}\coord1\coord2\ldots\Largr{\coord{n}}
\end{BDef}

\begin{LTXexample}[width=6.2cm]
\psset{unit=0.75}
\begin{pspicture}(-4,-4)(4,4)
\rmultiput[rot=45]{\red\psscalebox{3}{\ding{250}}}%
    (-2,-4)(-2,-3)(-3,-3)(-2,-1)(0,0)(1,2)(1.5,3)(3,3)
\rmultiput[rot=90,ref=lC]{\blue\psscalebox{2}{\ding{253}}}%
    (-2,2.5)(-2,2.5)(-3,2.5)(-2,1)(1,-2)(1.5,-3)(3,-3)
\psgrid[subgriddiv=0,gridcolor=lightgray]
\end{pspicture}
\end{LTXexample}

\clearpage


%--------------------------------------------------------------------------------------
\section{\nxLcs{psVector}: Drawing relative vector lines}
%--------------------------------------------------------------------------------------

The new macros \Lcs{psStartPoint} and \Lcs{psVector} allow to draw a series of
vectors which start point refers to the endpoint of the last drawn vector. The 
coordinates of the endpoint are \emph{always} interpreted relative to the last
the vector. The first vector refers to the coordinates set by \Lcs{psStartPoint}.
With the boolean argument one can draw the horizontal angle of the vector.

The style of the angle arc is saved in \Lkeyval{psMarkAngleStyle} and the style
for the horizontal line in \Lkeyval{psMarkAngleLineStyle} and preset to

\begin{lstlisting}
\newpsstyle{psMarkAngleStyle}{arrows=->,arrowsize=4pt}
\newpsstyle{psMarkAngleLineStyle}{linestyle=dotted}
\end{lstlisting}


\begin{pspicture}[showgrid](10,10)
 \psStartPoint(1,1)
 \psVector(3;30)\psVector(4;60)\psVector[linecolor=red](3;10)
 \psVector[linestyle=dashed](4;110)
 \psStartPoint(1,1)\psset{markAngle}
 \psVector[linestyle=dashed](4;110)\psVector[linecolor=red](3;10)
 \psVector(4;60)\psVector(3;30)
\end{pspicture}

\begin{lstlisting}
\begin{pspicture}[showgrid](10,10)
 \psStartPoint(1,1)
 \psVector(3;30)\psVector(4;60)\psVector[linecolor=red](3;10)
 \psVector[linestyle=dashed](4;110)
 \psStartPoint(1,1)\psset{markAngle}
 \psVector[linestyle=dashed](4;110)\psVector[linecolor=red](3;10)
 \psVector(4;60)\psVector(3;30)
\end{pspicture}
\end{lstlisting}

All end points of the vectors are saved in node names with the preset name \verb=Vector#=,
where \# is the consecutive  number of the nodes. \verb=Vector0= ist the starting point of
the first \Lcs{psVector}. With the macro \Lcs{psStartPoint} one can set the starting point and
with optional argument the name of the nodes. \verb=Vector3= is the default node name of
the endpoint of the third vector or the name of the starting point of the forth vector.

\begin{BDef}
\Lcs{psStartPoint}\OptArg{node basename}\Largr{$x$,$y$}
\end{BDef}

\begin{pspicture}[showgrid,linewidth=1pt](10,10.4)
 \psStartPoint[A](1,1)% nodes have the base name A
 \psVector(3;30)\psVector(4;60)\psVector[linecolor=red](3;10)
 \psVector[linestyle=dashed](4;110)
 \psline{->}(A0)(A4)
 \psStartPoint[B](1,1)\psset{markAngle}% nodes have the base name B
 \psVector[linestyle=dashed](4;110)
 \psVector[linecolor=red](3;10)
 \psVector(4;60)\psVector(3;30)
 \psline[arrows=-D>,arrowscale=2,linewidth=1.5pt,linecolor=red](B2)(A2)
 \psline[arrows=-D>,arrowscale=2,linewidth=1.5pt,linecolor=blue](A3)(B3)
 \multido{\iA=0+1}{5}{\uput[0](A\iA){A\iA}\uput[180](B\iA){B\iA}}
\end{pspicture}

\begin{lstlisting}
\begin{pspicture}[showgrid,linewidth=1pt](10,10.4)
 \psStartPoint[A](1,1)% nodes have the base name A
 \psVector(3;30)\psVector(4;60)\psVector[linecolor=red](3;10)
 \psVector[linestyle=dashed](4;110)
 \psline{->}(A0)(A4)
 \psStartPoint[B](1,1)\psset{markAngle}% nodes have the base name B
 \psVector[linestyle=dashed](4;110)
 \psVector[linecolor=red](3;10)
 \psVector(4;60)\psVector(3;30)
 \psline[arrows=-D>,arrowscale=2,linewidth=1.5pt,linecolor=red](B2)(A2)
 \psline[arrows=-D>,arrowscale=2,linewidth=1.5pt,linecolor=blue](A3)(B3)
 \multido{\iA=0+1}{5}{\uput[0](A\iA){A\iA}\uput[180](B\iA){B\iA}
 \end{pspicture}
\end{lstlisting}

\clearpage


%--------------------------------------------------------------------------------------
\section{\nxLcs{psCircleTangents}: Calculating tangent lines of circles}
%--------------------------------------------------------------------------------------

The macro calculates the points on a circle where tangent lines from another
point or another circle are drawn.

\begin{BDef}
\Lcs{psCircleTangents}\Largr{$x1,y1$}\Largr{$x2,y2$}\Largb{Radius}\\
\Lcs{psCircleTangents}\Largr{$x1,y1$}\Largb{Radius}\Largr{$x2,y2$}\Largb{Radius}
\end{BDef}

In the first case the coordinates of a point and the center and the radius
of a circle must be given. The names of the calculates node names are \verb=CircleT1=
and \verb=CircleT2=.

\bigskip
\begin{pspicture}[showgrid](0,3)(10,10)
\psdot(2,4)\pscircle(7,7){2}
\psCircleTangents(2,4)(7,7){2}
\pcline[nodesep=-1cm,linecolor=blue](2,4)(CircleT1)
\pcline[nodesep=-1cm,linecolor=blue](2,4)(CircleT2)
\psdots(CircleT1)(CircleT2)
\uput[-80](CircleT1){T1}\uput[115](CircleT2){T2}
\end{pspicture}


\begin{lstlisting}
\begin{pspicture}[showgrid](0,3)(10,10)
\psdot(2,4)\pscircle(7,7){2}
\psCircleTangents(2,4)(7,7){2}
\pcline[nodesep=-1cm,linecolor=blue](2,4)(CircleT1)
\pcline[nodesep=-1cm,linecolor=blue](2,4)(CircleT2)
\psdots(CircleT1)(CircleT2)
\uput[-80](CircleT1){T1}\uput[115](CircleT2){T2}
\end{pspicture}
\end{lstlisting}

\bigskip
When using the other variant of the macro two circles must be given. The macro then defines
ten nodes, named \verb=CircleTC1= and \verb=CircleTC2= for the two intersection points,
 \verb=CircleTO1=, \verb=CircleTO2=, \verb=CircleTO3=, and \verb=CircleTO4= for the four
 nodes of the outer tangent lines and 
  \verb=CircleTI1=, \verb=CircleTI2=, \verb=CircleTI3=, and \verb=CircleTI4= for the
  four nodes of the inner tangent lines.

\bigskip
\begin{pspicture}[showgrid](-2,-2)(10,10)
\pscircle(1,1){1}\pscircle(7,7){3}
\psCircleTangents(1,1){1}(7,7){3}
\pcline[nodesep=-1cm,linecolor=blue](CircleTO1)(CircleTO2)
\pcline[nodesep=-1cm,linecolor=blue](CircleTO3)(CircleTO4)
\pcline[nodesep=-1cm,linecolor=red](CircleTI1)(CircleTI2)
\pcline[nodesep=-1cm,linecolor=red](CircleTI3)(CircleTI4)
\psdots(CircleTC1)(CircleTC2)%
  (CircleTO1)(CircleTO2)(CircleTO3)(CircleTO4)%
  (CircleTI1)(CircleTI2)(CircleTI3)(CircleTI4)%
\uput[0](CircleTC1){TC1}\uput[0](CircleTC2){TC2}
\uput[-80](CircleTI1){TI1}\uput[115](CircleTI2){TI2}
\uput[150](CircleTI3){TI3}\uput[-45](CircleTI4){TI4}
\uput[-80](CircleTO1){TO1}\uput[150](CircleTO2){TO2}
\uput[150](CircleTO3){TO3}\uput[-45](CircleTO4){TO4}
\end{pspicture}

\bigskip
\begin{lstlisting}
\begin{pspicture}[showgrid](-2,-2)(10,10)
\pscircle(1,1){1}\pscircle(7,7){3}
\psCircleTangents(1,1){1}(7,7){3}
\pcline[nodesep=-1cm,linecolor=blue](CircleTO1)(CircleTO2)
\pcline[nodesep=-1cm,linecolor=blue](CircleTO3)(CircleTO4)
\pcline[nodesep=-1cm,linecolor=red](CircleTI1)(CircleTI2)
\pcline[nodesep=-1cm,linecolor=red](CircleTI3)(CircleTI4)
\psdots(CircleTC1)\psdots(CircleTC2)%
  (CircleTO1)(CircleTO2)(CircleTO3)(CircleTO4)%
  (CircleTI1)(CircleTI2)(CircleTI3)(CircleTI4)%
\uput[0](CircleTC1){TC1}\uput[0](CircleTC2){TC2}
\uput[-80](CircleTI1){TI1}\uput[115](CircleTI2){TI2}
\uput[150](CircleTI3){TI3}\uput[-45](CircleTI4){TI4}
\uput[-80](CircleTO1){TO1}\uput[150](CircleTO2){TO2}
\uput[150](CircleTO3){TO3}\uput[-45](CircleTO4){TO4}
\end{pspicture}
\end{lstlisting}


\clearpage

%--------------------------------------------------------------------------------------
\section{\nxLcs{psEllipseTangents}: Calculating tangent lines of an ellipse}
%--------------------------------------------------------------------------------------

The macro calculates the two points on an ellipse where tangent lines from an outside  point
 are drawn.

\begin{BDef}
\Lcs{psEllipseTangents}\Largr{$x_0,y_0$}\Largr{$a,b$}\Largr{$x_p,y_p$}\\
\end{BDef}

The first two pairs of coordinates are the same as the ones for the default ellipse.
The names of the calculates node names are \verb=EllipseT1=
and \verb=EllipseT2=.

\bigskip
\begin{pspicture}[showgrid](0,3)(10,10)
\psdot(2,4)\psellipse(7,7)(3,1.5)
\psEllipseTangents(7,7)(3,1.5)(2,4)
\pcline[nodesep=-1cm,linecolor=blue](2,4)(EllipseT1)
\pcline[nodesep=-1cm,linecolor=blue](2,4)(EllipseT2)
\psdots(EllipseT1)(EllipseT2)
\uput[-80](EllipseT1){T1}\uput[115](EllipseT2){T2}
\end{pspicture}


\begin{lstlisting}
\begin{pspicture}[showgrid](0,3)(10,10)
\psdot(2,4)\psellipse(7,7)(3,1.5)
\psEllipseTangents(7,7)(3,1.5)(2,4)
\pcline[nodesep=-1cm,linecolor=blue](2,4)(EllipseT1)
\pcline[nodesep=-1cm,linecolor=blue](2,4)(EllipseT2)
\psdots(EllipseT1)(EllipseT2)
\uput[-80](EllipseT1){T1}\uput[115](EllipseT2){T2}
\end{pspicture}
\end{lstlisting}


\clearpage

%--------------------------------------------------------------------------------------
\section{\nxLcs{psrotate}: Rotating objects}
%--------------------------------------------------------------------------------------
\Lcs{rput} also has an optional argument for rotating objects, but
it always depends on the \Lcs{rput} coordinates. With
\Lcs{psrotate} the rotating center can be placed anywhere. The
rotation is done with \verb+\pscustom+, all optional arguments are
only valid if they are part of the \verb+\pscustom+ macro.

\begin{BDef}
\Lcs{psrotate}\OptArgs\Largr{$x,y$}\Largb{rot angle}\Largb{object}
\end{BDef}

\begin{LTXexample}[width=0.4\linewidth]
\psset{unit=0.75}
\begin{pspicture}(-0.5,-3.5)(8.5,4.5)
  \psaxes{->}(0,0)(-0.5,-3)(8.5,4.5)
  \psdots[linecolor=red,dotscale=1.5](2,1)
  \psarc[linecolor=red,linewidth=0.4pt,showpoints=true]
        {->}(2,1){3}{0}{60}
  \pspolygon[linecolor=green,linewidth=1pt](2,1)(5,1.1)(6,-1)(2,-2)
  \psrotate(2,1){60}{%
    \pspolygon[linecolor=blue,linewidth=1pt](2,1)(5,1.1)(6,-1)(2,-2)}
\end{pspicture}
\end{LTXexample}


\begin{LTXexample}[width=6cm]
\begin{pspicture}(-1,-1)(3,6)
\def\canne{%  Idea by Manuel Luque
  \psgrid[subgriddiv=0](-1,0)(1,5)
  \pscustom[linewidth=2mm]{\psline(0,4)\psarcn(0.3,4){0.3}{180}{360}}%
  \pscircle*(0.6,4){0.1}\pstriangle*(0,0)(0.2,-0.3)}
\def\Object{}
  \canne
  \psrotate(0.3,4){45}{\psset{linecolor=red!50}\canne}
  \psrotate(0.3,4){90}{\psset{linecolor=blue!50}\canne}
  \psrotate(0.3,4){360}{\psset{linecolor=cyan!50}\canne}
  \psdot[linecolor=red](0.3,4)
\end{pspicture}
\end{LTXexample}


\begin{LTXexample}[pos=t]
\begin{pspicture}(0,-6)(15,5)
\def\majorette{\psline[linewidth=0.5mm](0,2)%  Idea by Manuel Luque
               \pscircle[fillstyle=solid]{0.1}
               \pscircle[fillstyle=solid](0,2){0.1}}
  \psaxes[linewidth=0.5pt]{->}(0,0)(0,-5)(15,5)
  \pstVerb{/V0 10 def /Alpha 45 def}% vitesse initiale, angle de lancement
  \multido{\nT=0.0+0.05,\iA=0+40}{41}{%
    \pstVerb{/nT \nT\space def}%
    \rput(!V0 Alpha cos mul nT mul -9.81 2 div nT dup mul mul V0 Alpha sin mul nT mul add){%
       \psrotate(0,1){\iA}{\majorette\psdot[linecolor=red](0,1)\psdot[linecolor=green](0,2)}}}
  \parametricplot[linecolor=red]{0}{2}{% trajectoire du milieu
     V0 Alpha cos mul t mul -9.81 2 div t dup mul mul V0 Alpha sin mul t mul add 1 add}
  \parametricplot[linecolor=green,plotpoints=360]{0}{2}{% d'une extremite
     V0 Alpha cos mul t mul 800 t mul sin sub % x(t)
     -9.81 2 div t dup mul mul V0 Alpha sin mul t mul add 1 add 800 t mul cos add }%y(t)
\end{pspicture}
\end{LTXexample}


\clearpage

%--------------------------------------------------------------------------------------
\section{\nxLcs{psComment}: comments to a graphic}
%--------------------------------------------------------------------------------------

\begin{BDef}
\LcsStar{psComment}\OptArgs\OptArg*{\Largb{arrows}}\coord0\coord1\Largb{Text}\OptArg{line macro}\OptArg{put macro}
\end{BDef}

By default the macro uses the \Lcs{ncline} macro to draw a line from the first to the
second point, it can be changed with the first additional optional argument. The label is
put by default with \Lcs{rput}, which can be changed with the last optional argument.
If this is used, then the line macro has also be defined, eg \verb+\psComment(A)(B){text}[\ncarc][\ncput}+
At least, leave the argument empty.


\begin{LTXexample}[pos=t,wide]
\SpecialCoor\newpsstyle{weiss}{fillstyle=solid,fillcolor=white}
\footnotesize\psset{unit=0.5cm,dimen=middle}
\begin{pspicture}(-12,-4)(6,10)
\psframe*[linecolor=black!20](-5,-3)(5,7) \psframe*[linecolor=black!40](-5,3)(5,6)
\pscircle(-8.19,5.51){0.2}
\psframe[fillcolor=white,fillstyle=solid](-5.8,3.6)(4.3,5.8)
\psframe(-8.98,3.14)(-5.8,6.32)
\multido{\rA=-4.1+1.3}{5}{\rput(\rA,-2.4){\psframe[style=weiss](1.1,6)
  \psline(0,0)(1.1,0.5)(0,1)(1.1,1.6)(0,2.2)(1.1,2.7)(0,3.2)(1.1,3.2)}}
\pspolygon*(-4.1,3.7)(-4.1,3)(-3,3)(-3.01,3.7)(-3.54,4.19)
\pspolygon*(1.09,3.7)(1.1,3)(2.2,3)(2.18,3.7)(1.65,4.24)
\pspolygon*(-2.78,3.7)(-2.8,3)(-1.7,3)(-1.71,3.7)(-2.27,4.04)
\pspolygon*(-1.51,3.7)(-1.5,3)(-0.4,3)(-0.41,3.7)(-1.02,4.17)
\pspolygon*(-0.21,3.7)(-0.2,3)(0.9,3)(0.89,3.7)(0.3,4.04)
\psline(-5,3.83)(-4.15,3.86)(-3.5,4.3)(-2.85,3.81)(-2.22,4.21)(-1.6,3.86)(-0.99,4.33)
       (-0.28,3.83)(0.35,4.19)(0.97,3.83)(1.65,4.39)(2.2,4.01)(3.57,4.89)(2.41,5.8)
  \psline(-5,5.8)(-5.78,5.8)  \psline(-5.78,5.47)(2.85,5.47)
  \psline(-5.8,3.52)(-5,3.5)  \psline(3.57,4.89)(-5.8,4.89)
  \psComment*[ref=r]{->}(-8.14,1.19)(-4.31,3.27){Mantelstift}
  \psComment*[ref=r]{->}(-8.17,-0.56)(-4.37,1.59){Kernstift}[\ncarc]
  \psComment*[ref=r]{->}(-7.91,-2.24)(-4.44,-0.23){Feder}[\ncarc]
  \psComment[npos=-0.1]{->}(-3.48,8.72)(-1.33,5.46){Nur f\"ur Profil}
\end{pspicture}
\end{LTXexample}

\clearpage
%--------------------------------------------------------------------------------------
\section{\nxLcs{psChart}: a pie chart}
%--------------------------------------------------------------------------------------

\begin{BDef}
\Lcs{psChart}\OptArgs\Largb{comma separated value list}\Largb{comma separated value list}\Largb{radius}
\end{BDef}

The special optional arguments for the \Lcs{psChart} macro are as follows:

\noindent
\begin{tabularx}{\linewidth}{@{}>{\ttfamily}lX>{\ttfamily}l@{}}
\textrm{\emph{name}} & \textrm{\emph{description}} & \textrm{\emph{default}}\\\hline
\Lkeyword{chartSep}  & distance from the pie chart center to an outraged pie piece & 10pt\\
\Lkeyword{chartColor} & gray or colored pie (values are: \texttt{gray} or \texttt{color})& gray\\
\Lkeyword{userColor} & a comma separated list of user defined colors for the pie & \{\}\\
\Lkeyword{chartNodeI}& the position of the inner node, relative to the radius & 0.75\\
\Lkeyword{chartNodeO}& the position of the outer node, relative to the radius & 1.5
\end{tabularx}

\bigskip
The first mandatory argument is the list of the values and may not be empty. The second
one is a list of outraged pieces, numbered consecutively from 1 to up the total number
of values. The list of user defined colors must be enclosed in braces!

The macro \Lcs{psChart} defines for every value three nodes at the half angle and
in distances from 0.75, 1, and 1.25 times of the radius from the origin. The nodes
are named as \verb+psChartI?+, \verb+psChart?+, and \verb+psChartO?+, where ? is the number of
the pie. The letter I leads to the inner node and the letter O to the outer node.
The distance can be changed with the optional arguments \Lkeyword{chartNodeI} and
\Lkeyword{chartNodeO} in the usual way with \verb+\psset{chartNodeI=...,chartNodeO=...}+.

The other one is the node on the circle line.
The origin is by default \texttt{(0,0)}. Moving the pie to another position can be done as
usual with the \Lcs{rput}-macro. The used colors are named internally as \Lkeyword{chartFillColor?}
and can be used by the user for coloring lines or text.

\begin{LTXexample}[width=6cm]
\begin{pspicture}(-3,-3)(3,3)
\psChart{ 23, 29, 3, 26, 28, 14 }{}{2}
\multido{\iA=1+1}{6}{%
  \psdot(psChart\iA)\psdot(psChartI\iA)\psdot(psChartO\iA)%
  \psline[linestyle=dashed,linecolor=white](psChart\iA)
  \psline[linestyle=dashed](psChart\iA)(psChartO\iA)}
\end{pspicture}
\end{LTXexample}

\begin{LTXexample}[width=6cm]
\begin{pspicture}(-3,-3)(3,3)
\psChart[chartColor=color]{45,90}{1}{2}
\ncline[linecolor=-chartFillColor1,
  nodesepB=-20pt]{psChartO1}{psChart1}
\rput[l](psChartO1){%
  \textcolor{chartFillColor1}{pie no 1}}
\ncline[linecolor=-chartFillColor2,
  nodesepB=-20pt]{psChartO2}{psChart2}
\rput[lt](psChartO2){%
  \textcolor{chartFillColor2}{pie no 2}}
\end{pspicture}
\end{LTXexample}

\begin{LTXexample}[width=7.5cm]
\psframebox[fillcolor=black!20,
  fillstyle=solid]{%
\begin{pspicture}(-3.5,-3.5)(4.25,3.5)
\psChart[chartColor=color]%
  {23, 29, 3, 26, 28, 14, 17, 4, 9}{}{2}
\multido{\iA=1+1}{9}{%
  \ncline[linecolor=-chartFillColor\iA,
    nodesepB=-10pt]{psChartO\iA}{psChart\iA}
  \rput[l](psChartO\iA){%
    \textcolor{chartFillColor\iA}{pie no \iA}}}
\end{pspicture}}
\end{LTXexample}

\begin{LTXexample}[width=6cm]
\begin{pspicture}(-3,-3)(3,3)
\psChart[userColor={red!30,green!30,
    blue!40,gray,magenta!60,cyan}]%
      { 23, 29, 3, 26, 28, 14 }{1,4}{2}
\end{pspicture}
\end{LTXexample}

\begin{LTXexample}[width=6cm]
\begin{pspicture}(-3,-2.5)(3,2.5)
\psChart{ 23, 29, 3, 26, 28, 14 }{}{2}
\multido{\iA=1+1}{6}{\rput*(psChartI\iA){\iA}}
\end{pspicture}
\end{LTXexample}


%\begin{LTXexample}[pos=t]
\psset{unit=1.5}
\begin{pspicture}(-3,-3)(3,3)
\psChart[userColor={red!30,green!30,blue!40,gray,cyan!50,
    magenta!60,cyan},chartSep=30pt,shadow=true,shadowsize=5pt]{34.5,17.2,20.7,15.5,5.2,6.9}{6}{2}
\psset{nodesepA=5pt,nodesepB=-10pt}
\ncline{psChartO1}{psChart1}\nput{0}{psChartO1}{1000 (34.5\%)}
\ncline{psChartO2}{psChart2}\nput{150}{psChartO2}{500 (17.2\%)}
\ncline{psChartO3}{psChart3}\nput{-90}{psChartO3}{600 (20.7\%)}
\ncline{psChartO4}{psChart4}\nput{0}{psChartO4}{450 (15.5\%)}
\ncline{psChartO5}{psChart5}\nput{0}{psChartO5}{150 (5.2\%)}
\ncline{psChartO6}{psChart6}\nput{0}{psChartO6}{200 (6.9\%)}
\bfseries%
\rput(psChartI1){Taxes}\rput(psChartI2){Rent}\rput(psChartI3){Bills}
\rput(psChartI4){Car}\rput(psChartI5){Gas}\rput(psChartI6){Food}
\end{pspicture}
%\end{LTXexample}
\psset{unit=1cm}

\begin{lstlisting}
\psset{unit=1.5}
\begin{pspicture}(-3,-3)(3,3)
\psChart[userColor={red!30,green!30,blue!40,gray,cyan!50,
    magenta!60,cyan},chartSep=30pt,shadow=true,shadowsize=5pt]{34.5,17.2,20.7,15.5,5.2,6.9}{6}{2}
\psset{nodesepA=5pt,nodesepB=-10pt}
\ncline{psChartO1}{psChart1}\nput{0}{psChartO1}{1000 (34.5\%)}
\ncline{psChartO2}{psChart2}\nput{150}{psChartO2}{500 (17.2\%)}
\ncline{psChartO3}{psChart3}\nput{-90}{psChartO3}{600 (20.7\%)}
\ncline{psChartO4}{psChart4}\nput{0}{psChartO4}{450 (15.5\%)}
\ncline{psChartO5}{psChart5}\nput{0}{psChartO5}{150 (5.2\%)}
\ncline{psChartO6}{psChart6}\nput{0}{psChartO6}{200 (6.9\%)}
\bfseries%
\rput(psChartI1){Taxes}\rput(psChartI2){Rent}\rput(psChartI3){Bills}
\rput(psChartI4){Car}\rput(psChartI5){Gas}\rput(psChartI6){Food}
\end{pspicture}
\end{lstlisting}



\clearpage
%--------------------------------------------------------------------------------------
\section{\nxLcs{psHomothetie}: central dilatation}
%--------------------------------------------------------------------------------------

\begin{BDef}
\Lcs{psHomothetie}\OptArgs\Largr{center}\Largb{factor}\Largb{object}
\end{BDef}

\begin{LTXexample}[width=9cm]
\begin{pspicture}[showgrid=true](-5,-4)(4,8)
\psBill% needs package pst-fun
\psHomothetie[linecolor=blue](4,-3){2}{\psBill}
\psdots[dotsize=3pt,linecolor=red](4,-3)
\psplot[linestyle=dashed,linecolor=red]{-5}{4}%
  [ /m -3 -0.85 sub 4 0.6 sub div def ]
  { m x mul m 4 mul sub 3 sub }%
\psHomothetie[linecolor=green](4,-3){-0.2}{\psBill}
\end{pspicture}
\end{LTXexample}

%\pstVerb{ /m -3 -0.85 sub 4 0.6 sub div def }


\clearpage

%--------------------------------------------------------------------------------------
\section{\nxLcs{psbrace}}
%--------------------------------------------------------------------------------------
\begin{BDef}
\LcsStar{psbrace}\OptArgs\Largr{A}\Largr{B}\Largb{text}
\end{BDef}

Additional to all other available options from \LPack{pstricks} or the other
related packages,  there are two new option, named  \Lkeyword{braceWidth} and
\Lkeyword{bracePos}. All important ones are shown in the following graphics
and table.

\begin{center}
\begin{pspicture}[showgrid=true](10,5)
  \psbrace[braceWidth=1cm,braceWidthInner=1cm,
    braceWidthOuter=1cm,bracePos=0.6,fillcolor=white,
    nodesepA=10mm,nodesepB=10mm](0,5)(10,5){\fbox{Label}}
\pcline{<->}(3,3)(3,4)\ncput*{\footnotesize\ttfamily braceWidth}
\pcline{<->}(3,4)(3,5)\ncput*{\footnotesize\ttfamily braceWidthInner}
\pcline{<->}(3,2)(3,3)\ncput*{\footnotesize\ttfamily braceWidthOuter}
\pcline{<->}(6,1)(6,2)\ncput{\footnotesize\ttfamily nodesepB}
\pcline{<->}(6,1)(7,1)\ncput*{\footnotesize\ttfamily A}
\pcline{<->}(0,0.5)(6,0.5)\ncput*{\footnotesize\ttfamily bracePos}
\psdot[dotscale=2](0,5)\uput[0](0,5){\textbf{A}}
\psdot[dotscale=2](10,5)\uput[180](10,5){\textbf{B}}
\end{pspicture}
\end{center}

A positive value for \Lkeyword{nodesepA} and \Lkeyword{nodesepB} shifts the label to the upper right
and a negative value to the lower left. This does not depends on
the value for the rotating of the label!

\begin{center}
\begin{tabular}{@{}l|l@{}}
name & meaning\\\hline
\Lkeyword{braceWidth} & default is \Lcs{pslinewidth}\\
\Lkeyword{braceWidthInner} & default is \verb+10\pslinewidth+\\
\Lkeyword{braceWidthOuter} & default is \verb+10\pslinewidth+\\
\Lkeyword{bracePos} & relative position (default is $0.5$)\\
\Lkeyword{nodesepA} & x-separation (default is $0pt$)\\
\Lkeyword{nodesepB} & y-separation (default is $0pt$)\\
\Lkeyword{rot} & additional rotating for the text (default is $0$)\\
\Lkeyword{ref} & reference point for the text (default is c)\\
\Lkeyword{fillcolor} & default is black
\end{tabular}
\end{center}

By default the text is written perpendicular to the brace line and
can be changed with the \LPack{pstricks} option \Lkeyword{rot}=\ldots\ The
text parameter can take any object and may also be empty. The
reference point can be any value of the combination of \Lkeyval{l}
(left) or \Lkeyval{r} (right) and \Lkeyval{b} (bottom) or \Lkeyval{B}
(Baseline) or \Lkeyval{C} (center) or \Lkeyval{t} (top), where the
default is \Lkeyval{c}, the center of the object.



\begin{LTXexample}[width=4.5cm]
\begin{pspicture}(4,4)
\psgrid[subgriddiv=0,griddots=10]
\pnode(0,0){A}
\pnode(4,4){B}
\psbrace[linecolor=red,ref=lC](A)(B){Text I}
\psbrace*[linecolor=blue,ref=lC](3,4)(0,1){Text II}
\psbrace[fillcolor=white](3,0)(3,4){III}
\end{pspicture}
\end{LTXexample}

\bigskip
The option \Lcs{specialCoor} is enabled, so that all types of coordinates
are possible, (nodename), ($x,y$), ($nodeA|nodeB$), \ldots
The star version fills the inner of the \Index{brace} with the current linecolor.
With the fillcolor \verb+white+ or any other background color the brace can
be "`unfilled"'.

\begin{LTXexample}
\begin{pspicture}(8,2.5)
\psbrace(0,0)(0,2){\fbox{Text}}%
\psbrace[nodesepA=10pt](2,0)(2,2){\fbox{Text}}
\psbrace[ref=lC](4,0)(4,2){\fbox{Text}}
\psbrace[ref=lt,rot=90,nodesepB=-15pt](6,0)(6,2){\fbox{Text}}
\psbrace[ref=lt,rot=90,nodesepA=-5pt,nodesepB=15pt](8,2)(8,0){\fbox{Text}}
\end{pspicture}
\end{LTXexample}


\begin{LTXexample}
\def\someMath{$\int\limits_1^{\infty}\frac{1}{x^2}\,dx=1$}
\begin{pspicture}(8,2.5)
\psbrace[ref=lC](0,0)(0,2){\someMath}%
\psbrace[rot=90](2,0)(2,2){\someMath}
\psbrace[ref=lC](4,0)(4,2){\someMath}
\psbrace[ref=lt,rot=90,nodesepB=-30pt](6,0)(6,2){\someMath}
\psbrace[ref=lt,rot=90,nodesepB=30pt](8,2)(8,0){\someMath}
\end{pspicture}
\end{LTXexample}

%$

\begin{LTXexample}
\begin{pspicture}(\linewidth,5)
\psbrace(0,0.5)(\linewidth,0.5){\fbox{Text}}%
\psbrace[bracePos=0.25,nodesepB=10pt,rot=90](0,2)(\linewidth,2){\fbox{Text}}
\psbrace[ref=lC,nodesepA=-3.5cm,nodesepB=15pt,rot=90](0,4)(\linewidth,4){%
   \fbox{some very, very long wonderful Text}}
\end{pspicture}
\end{LTXexample}


\begin{LTXexample}[width=8cm]
\psset{unit=0.8}
\begin{pspicture}(10,11)
\psgrid[subgriddiv=0,griddots=10]
\pnode(0,0){A}
\pnode(4,6){B}
\psbrace[ref=lC](A)(B){One}
\psbrace[rot=180,nodesepA=-5pt,ref=rb](B)(A){Two}
\psbrace[linecolor=blue,bracePos=0.25,ref=lB](8,1)(1,7){Three}
\psbrace[braceWidth=-1mm,rot=180,ref=rB](8,1)(1,7){Four}
\psbrace*[linearc=0.5,fillstyle=none,linewidth=1pt,braceWidth=1.5pt,
  bracePos=0.25,ref=lC](8,1)(8,9){A}
\psbrace(4,9)(6,9){}
\psbrace(6,9)(6,7){}
\psbrace(6,7)(4,7){}
\psbrace(4,7)(4,9){}
\psset{linecolor=red}
\psbrace*[ref=lb](7,10)(3,10){I}
\psbrace*[ref=lb,bracePos=0.75](3,10)(3,6){II}
\psbrace*[ref=lb](3,6)(7,6){III}
\psbrace*[ref=lb](7,6)(7,10){IV}
\end{pspicture}
\end{LTXexample}

%$

\begin{LTXexample}[width=5cm]
\[
\begin{pmatrix}
    \Rnode[vref=2ex]{A}{~1} \\
    & \ddots \\
    && \Rnode[href=2]{B}{1} \\
    &&& \Rnode[vref=2ex]{C}{0} \\
    &&&& \ddots \\
    &&&&& \Rnode[href=2]{D}{0}~ \\
\end{pmatrix}
\]
\psbrace[rot=-90,nodesepB=-0.5,nodesepA=-0.2](B)(A){\small n times}
\psbrace[rot=-90,nodesepB=-0.5,nodesepA=-0.2](D)(C){\small n times}
\end{LTXexample}


\clearpage
It is also possible to put a vertical brace around a
default paragraph. This works by setting two invisible nodes at
the beginning and the end of the paragraph. Indentation is
possible with a minipage.

\small
Some nonsense text, which is nothing more than nonsense.
Some nonsense text, which is nothing more than nonsense.

\noindent\rnode{A}{}

\vspace*{-1ex}
Some nonsense text, which is nothing more than nonsense.
Some nonsense text, which is nothing more than nonsense.
Some nonsense text, which is nothing more than nonsense.
Some nonsense text, which is nothing more than nonsense.
Some nonsense text, which is nothing more than nonsense.
Some nonsense text, which is nothing more than nonsense.
Some nonsense text, which is nothing more than nonsense.
Some nonsense text, which is nothing more than nonsense.

\vspace*{-2ex}\noindent\rnode{B}{}\psbrace*[linecolor=red](A)(B){}

Some nonsense text, which is nothing more than nonsense.
Some nonsense text, which is nothing more than nonsense.

\medskip\hfill\begin{minipage}{0.95\linewidth}
\noindent\rnode{A}{}

\vspace*{-1ex}
Some nonsense text, which is nothing more than nonsense.
Some nonsense text, which is nothing more than nonsense.
Some nonsense text, which is nothing more than nonsense.
Some nonsense text, which is nothing more than nonsense.
Some nonsense text, which is nothing more than nonsense.
Some nonsense text, which is nothing more than nonsense.
Some nonsense text, which is nothing more than nonsense.
Some nonsense text, which is nothing more than nonsense.

\vspace*{-2ex}
\noindent\rnode{B}{}\psbrace[linecolor=red](A)(B){}
\end{minipage}

\normalsize

\begin{lstlisting}
Some nonsense text, which is nothing more than nonsense.
Some nonsense text, which is nothing more than nonsense.

\noindent\rnode{A}{}

\vspace*{-1ex}
Some nonsense text, which is nothing more than nonsense.
Some nonsense text, which is nothing more than nonsense.
Some nonsense text, which is nothing more than nonsense.
Some nonsense text, which is nothing more than nonsense.
Some nonsense text, which is nothing more than nonsense.
Some nonsense text, which is nothing more than nonsense.
Some nonsense text, which is nothing more than nonsense.
Some nonsense text, which is nothing more than nonsense.

\vspace*{-2ex}\noindent\rnode{B}{}\psbrace[linecolor=red](A)(B){}

Some nonsense text, which is nothing more than nonsense.
Some nonsense text, which is nothing more than nonsense.

\medskip\hfill\begin{minipage}{0.95\linewidth}
\noindent\rnode{A}{}

\vspace*{-1ex}
Some nonsense text, which is nothing more than nonsense.
Some nonsense text, which is nothing more than nonsense.
Some nonsense text, which is nothing more than nonsense.
Some nonsense text, which is nothing more than nonsense.
Some nonsense text, which is nothing more than nonsense.
Some nonsense text, which is nothing more than nonsense.
Some nonsense text, which is nothing more than nonsense.
Some nonsense text, which is nothing more than nonsense.

\vspace*{-2ex}\noindent\rnode{B}{}\psbrace[linecolor=red](A)(B){}
\end{minipage}
\end{lstlisting}

\clearpage


%--------------------------------------------------------------------------------------
\section{Random dots}
%--------------------------------------------------------------------------------------
The syntax of the new macro \Lcs{psRandom} is:

\begin{BDef}
\Lcs{psRandom}\OptArgs\Largb{}\\
\Lcs{psRandom}\OptArgs\OptArg*{\Largr{$x_{Min},y_{Min}$}}\OptArg*{\Largr{$x_{Max},y_{Max}$}}\Largb{clip path} %$
%\psRandom[<option>](<xMax,yMax>){<clip path>}
%\psRandom[<option>](<xMin,yMin>)(<xMax,yMax>){<clip path>}
\end{BDef}

If there is no area for the dots defined, then \verb+(0,0)(1,1)+ in the current
scale setting is used for placing the dots. If there is only one \Largr{$x_{Max},y_{Max}$} %$
defined, then \verb+(0,0)+ is used for the other point.
This area should be greater than the clipping
path to be sure that the dots are placed over the full area. The clipping path can
be everything. If no clipping path is given, then the frame \verb+(0,0)(1,1)+
in user coordinates is used.  The new options are:

\begin{center}
\begin{tabular}{@{}l|l|l@{}}
name & default\\\hline
\Lkeyword{randomPoints} &   \verb|1000| & number of random dots\tabularnewline
\Lkeyword{color} & \false & random color\tabularnewline
\end{tabular}
\end{center}


\begin{LTXexample}[width=0.3\linewidth]
\psset{unit=5cm}
\begin{pspicture}(1,1)
  \psRandom[dotsize=1pt,fillstyle=solid](1,1){\pscircle(0.5,0.5){0.5}}
\end{pspicture}
\begin{pspicture}(1,1)
  \psRandom[dotsize=2pt,randomPoints=5000,color,%
      fillstyle=solid](1,1){\pscircle(0.5,0.5){0.5}}
\end{pspicture}
\end{LTXexample}

\begin{LTXexample}[width=0.4\linewidth]
\psset{unit=5cm}
\begin{pspicture}(1,1)
  \psRandom[randomPoints=200,dotsize=8pt,dotstyle=+]{}
\end{pspicture}
\begin{pspicture}(1.5,1)
  \psRandom[dotsize=5pt,color](0,0)(1.5,0.8){\psellipse(0.75,0.4)(0.75,0.4)}
\end{pspicture}
\end{LTXexample}

\begin{LTXexample}
\psset{unit=2.5cm}
\begin{pspicture}(0,-1)(3,1)
  \psRandom[dotsize=4pt,dotstyle=o,linecolor=blue,fillcolor=red,%
     fillstyle=solid,randomPoints=1000]%
      (0,-1)(3,1){\psplot{0}{3.14}{ x 114 mul sin }}
\end{pspicture}
\end{LTXexample}

\psset{unit=1cm}


\clearpage
 %--------------------------------------------------------------------------------------
\section{\nxLcs{psDice}}
 %--------------------------------------------------------------------------------------
\Lcs{psdice} creates the view of a dice. The number on the dice is the only parameter.
The optional parameters, like the color can be used as usual. The macro is a box of
dimension zero and is placed
at the current point. Use  the \Lcs{rput} macro to place it anywhere. The optional
argument \Lkeyword{unit} can be used to scale the dice. the default size of
the dice $1\mathrm{cm}\times1\mathrm{cm}$.

\begin{center}
\begin{pspicture}(-1,-1)(8,9)
\multido{\iA=1+1}{6}{%
  \rput(\iA,7.5){\Huge\psdice[unit=0.75,linecolor=red!80]{\iA}}
  \rput(! -0.5 7 \iA\space sub){\Huge\psdice[unit=0.75,linecolor=blue!70]{\iA}}%
  \multido{\iB=1+1}{6}{%
    \rput(! \iA\space 7 \iB\space sub){%
      \rnode[c]{p\iA\iB}{\makebox[1em][l]{\strut\psPrintValue[fontscale=12]{\iA\space \iB\space add}}}%
}}}
\ncbox[linearc=0.35,nodesep=0.2,linestyle=dotted]{p11}{p66}
\ncbox[linearc=0.35,nodesep=0.2,linestyle=dashed]{p15}{p51}
\rput{90}(-1.5,3.5){1. dice}
\rput{0}(3.5,8.5){2. dice}
\psline[linewidth=1.5pt](0.25,0.5)(0.25,8)
\psline[linewidth=1.5pt](-1,6.75)(6.5,6.75)
\end{pspicture}
\end{center}

\begin{lstlisting}
\begin{pspicture}(-1,-1)(8,8)
\multido{\iA=1+1}{6}{%
  \rput(\iA,7.5){\Huge\psdice[unit=0.75,linecolor=red!80]{\iA}}
  \rput(! -0.5 7 \iA\space sub){\Huge\psdice[unit=0.75,linecolor=blue!70]{\iA}}%
  \multido{\iB=1+1}{6}{%
    \rput(! \iA\space 7 \iB\space sub){%
      \rnode[c]{p\iA\iB}{\makebox[1em][l]{\strut\psPrintValue[fontscale=12]{\iA\space \iB\space add}}}%
}}}
\ncbox[linearc=0.35,nodesep=0.2,linestyle=dotted]{p11}{p66}
\ncbox[linearc=0.35,nodesep=0.2,linestyle=dashed]{p15}{p51}
\rput{90}(-1.5,3.5){1. dice}
\rput{0}(3.5,8.5){2. dice}
\psline[linewidth=1.5pt](0.25,0.5)(0.25,8)
\psline[linewidth=1.5pt](-1,6.75)(6.5,6.75)
\end{pspicture}
\end{lstlisting}

\clearpage
%--------------------------------------------------------------------------------------
\section{\nxLcs{psFormatInt}}
%--------------------------------------------------------------------------------------
There exist some packages and a lot of code to format an integer like $1\,000\,000$
or $1,234,567$ (in Europe $1.234.567$). But all packages expect a real number as
argument and cannot handle macros as an argument. For this case \LPack{pstricks-add}
has a macro \Lcs{psFormatInt} which can handle both:

\begin{LTXexample}[width=3cm]
\psFormatInt{1234567}\\
\psFormatInt[intSeparator={,}]{1234567}\\
\psFormatInt[intSeparator=.]{1234567}\\
\psFormatInt[intSeparator=$\cdot$]{1234567}\\
\def\temp{965432}
\psFormatInt{\temp}
\end{LTXexample}

With the option \Lkeyword{intSeparator} the symbol can be changed to any any non-number character.


\clearpage

%--------------------------------------------------------------------------------------
\section{\nxLcs{psRelNode} and \nxLcs{psDefPSPNodes}}
%--------------------------------------------------------------------------------------
With these macros it is possible to put a node relative to a given line or given
\Lenv{pspicture}-environment. In the frist case the parameters are
the angle and the length factor:

\begin{BDef}
\Lcs{psRelNode}\Largs{P0}\Largs{P1}\Largb{length factor}\Largb{end node name}\\
\Lcs{psDefPSPNodes}
\end{BDef}

The length factor relates to the distance $\overline{P_0P_1}$ and
the end node name must be a valid nodename and shouldn't contain
any of the special PostScript characters. There are two valid
options:

\begin{tabularx}{\linewidth}{@{} l|l| X @{} }
name & default & meaning\\\hline 
\Lkeyword{angle} & $0$ & angle between the given line $\overline{P_0P_1}$ and the new one
	$\overline{P_0P_{endNode}}$\tabularnewline 
\Lkeyword{trueAngle} & \false & defines whether the angle refers to the seen line or to
the mathematical one, which respect the scaling factors
\Lkeyword{xunit} and \Lkeyword{yunit}.
\end{tabularx}

\begin{LTXexample}[width=7cm]
\begin{pspicture}[showgrid](7,6)
  \pnode(3,3){A}\pnode(4,2){B}
  \psline[nodesep=-3,linewidth=0.5pt](A)(B)
  \multido{\iA=0+30}{12}{%
    \psRelNode[angle=\iA](A)(B){2}{C}%
    \qdisk(C){2pt}
    \uput[0](C){\iA}}
\end{pspicture}
\end{LTXexample}

In the second case the new macro \Lcs{psDefPSPNodes} defines nine nodes that corresponds to
nine particular points (namely bottom left, bottom center,
bottom right, center left, center center, center right, top left,
top center, top right) of the \Lenv{pspicture} box.

\begin{LTXexample}[width=6cm,wide=false]
\begin{pspicture}[showgrid=true](-1,-1)(4,4)
  \psDefPSPNodes
  \psdots(PSPbl)(PSPbc)(PSPbr)
      (PSPcl)(PSPcc)(PSPcr)(PSPtl)(PSPtc)(PSPtr)
  \uput[90](PSPbl){PSPbl} \uput[90](PSPbc){PSPbc}
  \uput[90](PSPbr){PSPbr} \uput[90](PSPcl){PSPcl}
  \uput[90](PSPcc){PSPcc} \uput[90](PSPcr){PSPcr}
  \uput[90](PSPtl){PSPtl} \uput[90](PSPtc){PSPtc}
  \uput[90](PSPtr){PSPtr}
\end{pspicture}
\end{LTXexample}

The name of the nodes are predefined as:

\begin{lstlisting}[style=syntax]
\psset[pst-PSPNodes]{blName=PSPbl,bcName=PSPbc,brName=PSPbr,
  clName=PSPcl,ccName=PSPcc,crName=PSPcr,tlName=PSPtl,tcName=PSPtc,trName=PSPtr}
\end{lstlisting}

and can be modified in the same way.
%I guess you modified the family to have the pstricks-add one so the
%\xkvview would have to be adapted.

%--------------------------------------------------------------------------------------
\section{\nxLcs{psRelLine}}
%--------------------------------------------------------------------------------------
With this macro it is possible to plot lines relative to a given one. Parameter are
the angle and the length factor:

\begin{BDef}
\Lcs{psRelLine}\Largr{P0}\Largr{P1}\Largb{length factor}\Largb{<end node name>}\\
\Lcs{psRelLine}\OptArg{\Largb{arrows}}\Largr{P0}\Largr{P1}\Largb{length factor}\Largb{end node name}\\
\Lcs{psRelLine}\OptArgs\Largr{P0}\Largr{P1}\Largb{length factor}\Largb{end node name}\\
\Lcs{psRelLine}\OptArgs\OptArg{\Largb{arrows}}\Largr{P0}\Largr{P1}\Largb{length factor}\Largb{end node name}
\end{BDef}

The length factor relates to the distance $\overline{P_0P_1}$ and
the end node name must be a valid nodename and shouldn't contain
any of the special PostScript characters. There are two valid
options which are described in the foregoing section for
\Lcs{psRelNode}.

The following two figures show the same, the first one with a scaling different to $1:1$,
this is the reason why the end points are on an ellipse and not on a circle like in the
second figure.

\begin{LTXexample}[width=5cm]
\psset{yunit=2,xunit=1}
\begin{pspicture}(-2,-2)(3,2)
\psgrid[subgriddiv=2,subgriddots=10,gridcolor=lightgray]
\pnode(-1,0){A}\pnode(3,2){B}
\psline[linecolor=red](A)(B)
\psRelLine[linecolor=blue,angle=30](-1,0)(B){0.5}{EndNode}
\qdisk(EndNode){2pt}
\psRelLine[linecolor=blue,angle=-30](A)(B){0.5}{EndNode}
\qdisk(EndNode){2pt}
\psRelLine[linecolor=magenta,angle=90](-1,0)(3,2){0.5}{EndNode}
\qdisk(EndNode){2pt}
\psRelLine[linecolor=magenta,angle=-90](A)(B){0.5}{EndNode}
\qdisk(EndNode){2pt}
\end{pspicture}
\end{LTXexample}

\begin{LTXexample}[width=5cm]
\begin{pspicture}(-2,-2)(3,2)
\psgrid[subgriddiv=2,subgriddots=10,gridcolor=lightgray]
\pnode(-1,0){A}\pnode(3,2){B}
\psline[linecolor=red](A)(B)
\psarc[linestyle=dashed](A){2.23}{-90}{135}
\psRelLine[linecolor=blue,angle=30](-1,0)(B){0.5}{EndNode}
\qdisk(EndNode){2pt}
\psRelLine[linecolor=blue,angle=-30](A)(B){0.5}{EndNode}
\qdisk(EndNode){2pt}
\psRelLine[linecolor=magenta,angle=90](-1,0)(3,2){0.5}{EndNode}
\qdisk(EndNode){2pt}
\psRelLine[linecolor=magenta,angle=-90](A)(B){0.5}{EndNode}
\qdisk(EndNode){2pt}
\end{pspicture}
\end{LTXexample}

\medskip
The following figure has also a different scaling, but has set the
option \Lkeyword{trueAngle}, all angles refer to "what you see".

\begin{LTXexample}[width=6.5cm]
\psset{yunit=2,xunit=1}
\begin{pspicture}(-3,-1)(3,2)\psgrid[subgridcolor=lightgray]
\pnode(-1,0){A}\pnode(3,2){B}
\psline[linecolor=red](A)(B)
\psarc(A){2.83}{-45}{135}
\psRelLine[linecolor=blue,angle=30,trueAngle](A)(B){0.5}{EndNode}
\qdisk(EndNode){2pt}
\psRelLine[linecolor=blue,angle=-30,trueAngle](A)(B){0.5}{EndNode}
\qdisk(EndNode){2pt}
\psRelLine[linecolor=magenta,angle=90,trueAngle](A)(B){0.5}{EndNode}
\qdisk(EndNode){2pt}
\psRelLine[linecolor=magenta,angle=-90,trueAngle](A)(B){0.5}{EndNode}
\qdisk(EndNode){2pt}
\end{pspicture}
\end{LTXexample}

\medskip
Two examples using \verb+\multido+ to show the behaviour of the
options \verb+trueAngle+ and \verb+angle+.

\medskip
\begin{LTXexample}[width=8cm]
\psset{yunit=4,xunit=2}
\begin{pspicture}(-1,0)(3,2)\psgrid[subgridcolor=lightgray]
\pnode(-1,0){A}\pnode(1,1){B}
\psline[linecolor=red](A)(3,2)
\multido{\iA=0+10}{36}{%
  \psRelLine[linecolor=blue,angle=\iA](B)(A){-0.5}{EndNode}
  \qdisk(EndNode){2pt}
}
\end{pspicture}
\end{LTXexample}

\begin{LTXexample}[width=8cm]
\psset{yunit=4,xunit=2}
\begin{pspicture}(-1,0)(3,2)\psgrid[subgridcolor=lightgray]
\pnode(-1,0){A}\pnode(1,1){B}
\psline[linecolor=red](A)(3,2)
\multido{\iA=0+10}{36}{%
  \psRelLine[linecolor=magenta,angle=\iA,trueAngle]{->}(B)(A){-0.5}{EndNode}
}
\end{pspicture}
\end{LTXexample}

\begin{center}
\bgroup
\psset{xunit=0.75\linewidth,yunit=0.75\linewidth,trueAngle}%
\begin{pspicture}(1,0.6)%\psgrid
  \pnode(.3,.35){Vk} \pnode(.375,.35){D} \pnode(0,.4){DST1} \pnode(1,.18){DST2}
  \pnode(0,.1){A1}   \pnode(1,.31){A1}
  { \psset{linewidth=.02,linestyle=dashed,linecolor=gray}%
    \pcline(DST1)(DST2) % <- Druckseitentangente
    \pcline(A2)(A1) % <- Anstr\"omrichtung
    \lput*{:U}{\small Anstr\"omrichtung $v_{\infty}$} }%
  \psIntersectionPoint(A1)(A2)(DST1)(DST2){Hk}
  \pscurve(Hk)(.4,.38)(Vk)(.36,.33)(.5,.32)(Hk)
  \psParallelLine[linecolor=red!75!green,arrows=->,arrowscale=2](Vk)(Hk)(D){.1}{FtE}
  \psRelLine[linecolor=red!75!green,arrows=->,arrowscale=2,angle=90](D)(FtE){4}{Fn}% why "4"?
  \psParallelLine[linestyle=dashed](D)(FtE)(Fn){.1}{Fnr1}
  \psRelLine[linestyle=dashed,angle=90](FtE)(D){-4}{Fnr2} % why "-4"?
  \psline[linewidth=1.5pt,arrows=->,arrowscale=2](D)(Fnr2)
  \psIntersectionPoint(D)([nodesep=2]D)(Fnr1)([offset=-4]Fnr1){Fh}
  \psIntersectionPoint(D)([offset=2]D)(Fnr1)([nodesep=4]Fnr1){Fv}
  \psline[linecolor=blue,arrows=->,arrowscale=2](D)(Fh)
  \psline[linecolor=blue,arrows=->,arrowscale=2](D)(Fv)
  \psline[linestyle=dotted](Fh)(Fnr1)  \psline[linestyle=dotted](Fv)(Fnr1)
  \uput{.1}[0](Fh){\blue $F_{H}$}   \uput{.1}[180](Fv){\blue $F_{V}$}
  \uput{.1}[-45](Fnr1){$F_{R}$}     \uput{.1}[90](Fn){\color{red!75!green}$F_{N}$}
  \uput{.25}[-90](FtE){\color{red!75!green}$F_{T}$}
\end{pspicture}
\egroup
\end{center}
\begin{lstlisting}
\psset{xunit=0.75\linewidth,yunit=0.75\linewidth,trueAngle}%
\end{center}
\begin{pspicture}(1,0.6)%\psgrid
  \pnode(.3,.35){Vk} \pnode(.375,.35){D} \pnode(0,.4){DST1} \pnode(1,.18){DST2}
  \pnode(0,.1){A1}   \pnode(1,.31){A1}
  { \psset{linewidth=.02,linestyle=dashed,linecolor=gray}%
    \pcline(DST1)(DST2) % <- Druckseitentangente
    \pcline(A2)(A1) % <- Anstr"omrichtung
    \lput*{:U}{\small Anstr"omrichtung $v_{\infty}$} }%
  \psIntersectionPoint(A1)(A2)(DST1)(DST2){Hk}
  \pscurve(Hk)(.4,.38)(Vk)(.36,.33)(.5,.32)(Hk)
  \psParallelLine[linecolor=red!75!green,arrows=->,arrowscale=2](Vk)(Hk)(D){.1}{FtE}
  \psRelLine[linecolor=red!75!green,arrows=->,arrowscale=2,angle=90](D)(FtE){4}{Fn}% why "4"?
  \psParallelLine[linestyle=dashed](D)(FtE)(Fn){.1}{Fnr1}
  \psRelLine[linestyle=dashed,angle=90](FtE)(D){-4}{Fnr2} % why "-4"?
  \psline[linewidth=1.5pt,arrows=->,arrowscale=2](D)(Fnr2)
  \psIntersectionPoint(D)([nodesep=2]D)(Fnr1)([offset=-4]Fnr1){Fh}
  \psIntersectionPoint(D)([offset=2]D)(Fnr1)([nodesep=4]Fnr1){Fv}
  \psline[linecolor=blue,arrows=->,arrowscale=2](D)(Fh)
  \psline[linecolor=blue,arrows=->,arrowscale=2](D)(Fv)
  \psline[linestyle=dotted](Fh)(Fnr1)  \psline[linestyle=dotted](Fv)(Fnr1)
  \uput{.1}[0](Fh){\blue $F_{H}$}   \uput{.1}[180](Fv){\blue $F_{V}$}
  \uput{.1}[-45](Fnr1){$F_{R}$}     \uput{.1}[90](Fn){\color{red!75!green}$F_{N}$}
  \uput{.25}[-90](FtE){\color{red!75!green}$F_{T}$}
\end{pspicture}
\end{lstlisting}


%--------------------------------------------------------------------------------------
\section{\nxLcs{psParallelLine}}
%--------------------------------------------------------------------------------------
With this macro it is possible to plot lines relative to a given one, which is parallel.
There is no special parameter here.

\begin{lstlisting}[style=syntax]
\psParallelLine(<P0>)(<P1>)(<P2>){<length>}{<end node name>}
\psParallelLine{<arrows>}(<P0>)(<P1>)(<P2>){<length>}{<end node name>}
\psParallelLine[<options>](<P0>)(<P1>)(<P2>){<length>}{<end node name>}
\psParallelLine[<options>]{<arrows>}(<P0>)(<P1>)(<P2>){<length>}{<end node name>}
\end{lstlisting}

The line starts at $P_2$, is parallel to $\overline{P_0P_1}$ and
the length of this parallel line depends on the length factor. The
end node name must be a valid nodename and shouldn't contain any
of the special PostScript characters.

\begin{LTXexample}
\begin{pspicture*}(-5,-4)(5,3.5)
  \psgrid[subgriddiv=0,griddots=5]
  \pnode(2,-2){FF}\qdisk(FF){1.5pt}
  \pnode(-5,5){A}\pnode(0,0){O}
  \multido{\nCountA=-2.4+0.4}{9}{%
    \psParallelLine[linecolor=red](O)(A)(0,\nCountA){9}{P1}
    \psline[linecolor=red](0,\nCountA)(FF)
    \psRelLine[linecolor=red](0,\nCountA)(FF){9}{P2}
  }
  \psline[linecolor=blue](A)(FF)
  \psRelLine[linecolor=blue](A)(FF){5}{END1}
  \psline[linewidth=2pt,arrows=->](2,0)(FF)
\end{pspicture*}
\end{LTXexample}


%--------------------------------------------------------------------------------------
\section{\nxLcs{psIntersectionPoint}}
%--------------------------------------------------------------------------------------
This macro calculates the intersection point of two lines, given by the four coordinates.
There is no special parameter here.
\begin{lstlisting}[style=syntax]
\psIntersectionPoint(<P0>)(<P1>)(<P2>)(<P3>){<node name>}
\end{lstlisting}

\begin{LTXexample}[width=5.5cm]
\psset{unit=0.5cm}
\begin{pspicture}(-5,-4)(5,5)
  \psaxes[labelFontSize=\scriptstyle,
    dx=2,Dx=2,dy=2,Dy=2]{->}(0,0)(-5,-4)(5,5)
  \psline[linecolor=red,linewidth=2pt](-5,-1)(5,5)
  \psline[linecolor=blue,linewidth=2pt](-5,3)(5,-4)
  \qdisk(-5,-1){2pt}\uput[-90](-5,-1){A}
  \qdisk(5,5){2pt}\uput[-90](5,5){B}
  \qdisk(-5,3){2pt}\uput[-90](-5,3){C}
  \qdisk(5,-4){2pt}\uput[-90](5,-4){D}
  \psIntersectionPoint(-5,-1)(5,5)(-5,3)(5,-4){IP}
  \qdisk(IP){3pt}\uput{0.3}[90](IP){IP}
  \psline[linestyle=dashed](IP|0,0)(IP)(0,0|IP)
\end{pspicture}
\end{LTXexample}

\clearpage

%--------------------------------------------------------------------------------------
\section[\nxLcs{psCancel}]{\nxLcs{psCancel}\footnotemark}
%--------------------------------------------------------------------------------------
\footnotetext{Thanks to by Stefano Baroni} This macro works like
the \Lcs{cancel} macro from the package of the same name but it
allows as argument any contents, not only letters but also a
complex graphic.

\begin{BDef}
\LcsStar{psCancel}\OptArgs\Largb{contents}%
\end{BDef}

All optional arguments for lines and boxes are valid and can be
used in the usual way. The star option fills the underlying box
rectangle with the linecolor. This can be transparent if
\Lkeyword{opacity} is set to a value less than 1. This can be used
in presentation to strike out words, equations, and graphic
objects. Lines can also be transparent when the option
\Lkeyword{strokeopacity} is used.

\begingroup
\psCancel{A} \psCancel[linecolor=red]{Tikz :-)} \quad
\psCancel[linecolor=blue,doubleline=true]{%
  \readdata{\data}{demo1.data}
  \psset{shift=*,xAxisLabel=x-Axis,yAxisLabel=y-Axis,llx=-13mm,lly=-7mm,
      xAxisLabelPos={c,-1},yAxisLabelPos={-7,c}}
  \pstScalePoints(1,0.00000001){}{}
  \begin{psgraph}[axesstyle=frame,xticksize=0 7.5,yticksize=0 25,subticksize=1,
     ylabelFactor=\cdot 10^8,Dx=5,Dy=1,xsubticks=2](0,0)(25,7.5){5.5cm}{5cm}
  \listplot[linecolor=red, linewidth=2pt, showpoints=true]{\data}
  \end{psgraph}} \qquad% end of Cancel
\psCancel[linewidth=3pt,linecolor=red,
    strokeopacity=0.5]{\tabular[b]{c}first line\\second line\endtabular}\quad
\psCancel*[linecolor=red!50,opacity=0.5]{\tabular[b]{c}first line\\second line\endtabular}
\quad
\psCancel*[linecolor=blue!30,opacity=0.5]{%
  \readdata{\data}{demo1.data}
  \psset{shift=*,xAxisLabel=x-Axis,yAxisLabel=y-Axis,llx=-15mm,lly=-7mm,urx=1mm,
      xAxisLabelPos={c,-1},yAxisLabelPos={-7,c}}
  \pstScalePoints(1,0.00000001){}{}
  \begin{psgraph}[axesstyle=frame,xticksize=0 7.5,yticksize=0 25,subticksize=1,
     ylabelFactor=\cdot 10^8,Dx=5,Dy=1,xsubticks=2](0,0)(25,7.5){5.5cm}{5cm}
  \listplot[linecolor=red, linewidth=2pt, showpoints=true]{\data}
  \end{psgraph}} \quad% end of Cancel
\psCancel[linewidth=4pt,strokeopacity=0.5]{\parbox{8cm}{\[
  \binom{x_R}{y_R} = \underbrace{r\vphantom{\binom{A}{B}}}_{\text{Scaling}}\cdot
    \underbrace{\begin{pmatrix}
        \sin\gamma & -\cos\gamma \\
      \cos \gamma & \sin \gamma \\
      \end{pmatrix}}_{\text{Rotation}} \binom{x_K}{y_K} +
  \underbrace{\binom{t_x}{t_y}}_{\text{Translation}} \]} }% end of psCancel
\endgroup

\bigskip
\begin{lstlisting}
\psCancel{A} \psCancel[linecolor=red]{Tikz :-)} \quad
\psCancel[linecolor=blue,doubleline=true]{%
  \readdata{\data}{demo1.data}
  \psset{shift=*,xAxisLabel=x-Axis,yAxisLabel=y-Axis,llx=-13mm,lly=-7mm,
      xAxisLabelPos={c,-1},yAxisLabelPos={-7,c}}
  \pstScalePoints(1,0.00000001){}{}
  \begin{psgraph}[axesstyle=frame,xticksize=0 7.5,yticksize=0 25,subticksize=1,
     ylabelFactor=\cdot 10^8,Dx=5,Dy=1,xsubticks=2](0,0)(25,7.5){5.5cm}{5cm}
  \listplot[linecolor=red, linewidth=2pt, showpoints=true]{\data}
  \end{psgraph}} \qquad% end of Cancel
\psCancel[linewidth=3pt,linecolor=red,
    strokeopacity=0.5]{\tabular[b]{c}first line\\second line\endtabular}\quad
\psCancel*[linecolor=red!50,opacity=0.5]{\tabular[b]{c}first line\\second line\endtabular}
\quad
\psCancel*[linecolor=blue!30,opacity=0.5]{%
  \readdata{\data}{demo1.data}
  \psset{shift=*,xAxisLabel=x-Axis,yAxisLabel=y-Axis,llx=-15mm,lly=-7mm,urx=1mm,
      xAxisLabelPos={c,-1},yAxisLabelPos={-7,c}}
  \pstScalePoints(1,0.00000001){}{}
  \begin{psgraph}[axesstyle=frame,xticksize=0 7.5,yticksize=0 25,subticksize=1,
     ylabelFactor=\cdot 10^8,Dx=5,Dy=1,xsubticks=2](0,0)(25,7.5){5.5cm}{5cm}
  \listplot[linecolor=red, linewidth=2pt, showpoints=true]{\data}
  \end{psgraph}} \quad% end of Cancel
\psCancel[linewidth=4pt,strokeopacity=0.5]{\parbox{8cm}{\[
  \binom{x_R}{y_R} = \underbrace{r\vphantom{\binom{A}{B}}}_{\text{Scaling}}\cdot
    \underbrace{\begin{pmatrix}
        \sin\gamma & -\cos\gamma \\
      \cos \gamma & \sin \gamma \\
      \end{pmatrix}}_{\text{Rotation}} \binom{x_K}{y_K} +
  \underbrace{\binom{t_x}{t_y}}_{\text{Translation}} \]} }% end of psCancel
\end{lstlisting}

The optional argument \Lkeyword{cancelType} allows to define the lines for the non star version.
Possible values are \Lkeyval{x} for a cross, \Lkeyval{s} for a slash, and \Lkeyval{b}
for a backslash. It is also possible to use the long words for the \Lkeyval{slash} and the \Lkeyval{backslash}.
An empty value is always assumed as a \Lkeyval{x}.

\begin{LTXexample}[pos=t,wide]
\psset{linewidth=3pt,strokeopacity=0.4}
\psCancel{\tabular[b]{c}first line\\second line\endtabular}   \quad
\psCancel[cancelType=x]{\tabular[b]{c}first line\\second line\endtabular}\quad
\psCancel[cancelType=s]{\tabular[b]{c}first line\\second line\endtabular}\quad
\psCancel[cancelType=b]{\tabular[b]{c}first line\\second line\endtabular}
\end{LTXexample}

\clearpage
%--------------------------------------------------------------------------------------
\section{\nxLcs{psStep}}
%--------------------------------------------------------------------------------------
\Lcs{psStep} calculates a step function for the upper or lower
sum or the max/min of the \Index{Riemann} integral definition of a given
function. The available option is

\Lkeyset{StepType=lower}|\Lkeyval{upper}|\Lkeyval{Riemann}|\Lkeyval{infimum}|\Lkeyval{supremum} or alternative
\Lkeyset{StepType=l}|\Lkeyval{u}|\Lkeyval{R}|\Lkeyval{i}|\Lkeyval{s}

with \Lkeyword{lower} as the default setting. The syntax of the function is

\begin{BDef}
\Lcs{psStep}\OptArgs\Largr(x1,x2)\Largb{n}\Largb{function}
\end{BDef}


(x1,x2) is the given interval for the step wise calculated
function, n is the number of the rectangles and \Larg{function} is
the mathematical function in postfix or algebraic=true notation (with
\Lkeyset{algebraic=true}).

\begin{LTXexample}[pos=t,preset=\centering]
\begin{pspicture}(-0.5,-0.5)(10,3)
 \psaxes[labelFontSize=\scriptstyle]{->}(10,3)
 \psplot[plotpoints=100,linewidth=1.5pt,algebraic=true]{0}{10}{sqrt(x)}
 \psStep[linecolor=magenta,StepType=upper,fillstyle=hlines](0,9){9}{x sqrt}
 \psStep[linecolor=blue,fillstyle=vlines](0,9){9}{x sqrt }
\end{pspicture}
\end{LTXexample}

\begin{LTXexample}[pos=t,preset=\centering]
\psset{plotpoints=200}
\begin{pspicture}(-0.5,-2.25)(10,3)
  \psaxes[labelFontSize=\scriptstyle]{->}(0,0)(0,-2.25)(10,3)
 \psplot[linewidth=1.5pt,algebraic=true]{0}{10}{sqrt(x)*sin(x)}
 \psStep[algebraic=true,linecolor=magenta,StepType=upper](0,9){20}{sqrt(x)*sin(x)}
 \psStep[linecolor=blue,linestyle=dashed](0,9){20}{x sqrt x RadtoDeg sin mul}
\end{pspicture}
\end{LTXexample}

\begin{LTXexample}[pos=t,preset=\centering]
\psset{yunit=1.25cm,plotpoints=200}
\begin{pspicture}(-0.5,-1.5)(10,1.5)
 \psaxes[labelFontSize=\scriptstyle]{->}(0,0)(0,-1.5)(10,1.5)
 \psStep[algebraic=true,StepType=Riemann,fillstyle=solid,fillcolor=black!10](0,10){50}%
    {sqrt(x)*cos(x)*sin(x)}
 \psplot[linewidth=1.5pt,algebraic=true]{0}{10}{sqrt(x)*cos(x)*sin(x)}
\end{pspicture}
\end{LTXexample}


\begin{LTXexample}[pos=t,preset=\centering]
\psset{yunit=1.25cm,plotpoints=200}
\begin{pspicture}(-0.5,-1.5)(10,1.5)
 \psaxes[labelFontSize=\scriptstyle]{->}(0,0)(0,-1.5)(10,1.5)
 \psStep[algebraic=true,StepType=infimum,fillstyle=solid,fillcolor=black!10](0,10){50}%
    {sqrt(x)*cos(x)*sin(x)}
 \psplot[linewidth=1.5pt,algebraic=true]{0}{10}{sqrt(x)*cos(x)*sin(x)}
\end{pspicture}
\end{LTXexample}

\begin{LTXexample}[pos=t,preset=\centering]
\psset{yunit=1.25cm,plotpoints=200}
\begin{pspicture}(-0.5,-1.5)(10,1.5)
 \psaxes[labelFontSize=\scriptstyle]{->}(0,0)(0,-1.5)(10,1.5)
 \psStep[algebraic=true,StepType=supremum,fillstyle=solid,fillcolor=black!10](0,10){50}%
    {sqrt(x)*cos(x)*sin(x)}
 \psplot[linewidth=1.5pt,algebraic=true]{0}{10}{sqrt(x)*cos(x)*sin(x)}
\end{pspicture}
\end{LTXexample}

\begin{LTXexample}[pos=t,preset=\centering]
\psset{unit=1.5cm,plotpoints=200}
\begin{pspicture}[plotpoints=200](-0.5,-3)(10,2.5)
  \psStep[algebraic=true,fillstyle=solid,fillcolor=yellow](0.001,9.5){40}{2*sqrt(x)*cos(ln(x))*sin(x)}
  \psStep[algebraic=true,StepType=Riemann,fillstyle=solid,fillcolor=blue](0.001,9.5){40}{2*sqrt(x)*cos(ln(x))*sin(x)}
  \psaxes[labelFontSize=\scriptstyle]{->}(0,0)(0,-2.75)(10,2.5)
  \psplot[algebraic=true,linecolor=white]{0.001}{9.75}{2*sqrt(x)*cos(ln(x))*sin(x)}
  \uput[90](6,1.2){$f(x)=2\cdot\sqrt{x}\cdot\cos{(\ln{x})}\cdot\sin{x}$}
\end{pspicture}
\end{LTXexample}

\clearpage
%--------------------------------------------------------------------------------------

\section{Tangent lines}
There are two macros for plotting a tangent line or the tangent normal line.
The first one is \Lcs{psTangentLine} which expects three pairs of coordinates,
a $x$ and a $dx$ value. The second one is \Lcs{psplotTangent} which expects 
a function for the curve. \xLkeyword{Tnormal}

\subsection{\nxLcs{psTangentLine} and option \nxLkeyword{Tnormal}}

\begin{BDef}
\Lcs{psTangentLine}\OptArgs\coord1\coord2\coord3\Largb{x}\Largb{dx}
\end{BDef}

\begin{LTXexample}[width=0.45\linewidth,wide]
\psset{unit=2}
\begin{pspicture}[showgrid=true](1,-1)(4,1)
  \pscurve[showpoints=true]
    (2.1,-0.2)(2.5,0.2)(3.2,0.235)(3.8,-0.2)
  \psTangentLine[Tnormal,arrows=->,
    linecolor=red](2.5,0.2)(3.2,0.235)%
      (3.8,-0.2){3}{0.1}
  \psTangentLine[arrows=<->,
    linecolor=blue](2.5,0.2)(3.2,0.235)%
      (3.8,-0.2){3}{0.5}
\end{pspicture}
\end{LTXexample}

In special cases one has to use \Lkeyword{curvature}\verb+=1 1 1+ for the macro \Lcs{pscurve}
to get the same equation for the curve as \Lcs{psplotTangentLine} does.

\begin{LTXexample}[pos=t,preset=\centering,wide]
\psset{unit=2}
\begin{pspicture}[showgrid=true](2,-1)(6,2)
\pscurve[showpoints=true,
  curvature=1 1 1](2.1,-0.2)(2.5,0.2)(3.2,0.235)(5.8,2)
\pscurve[showpoints=true,linecolor=green,
  curvature=1 1 1](2.5,0.2)(3.2,0.235)(5.8,2)
\psTangentLine[Tnormal,arrows=->,linecolor=red](2.5,0.2)(3.2,0.235)(5.8,2){4.6}{0.6}
\psTangentLine[arrows=<->,linecolor=blue](2.5,0.2)(3.2,0.235)(5.8,2){4.5}{0.6}
\end{pspicture}
\end{LTXexample}


The end points are saved as nodes \verb=OCurve=, \verb=ETangent=, and \verb=ENormal=. They can
be used in the default ways for nodes:

\begin{LTXexample}[pos=t,preset=\centering,wide]
\psset{unit=4,arrowscale=2}
\begin{pspicture}(0.1,-0.1)(4,1)
\pscurve[showpoints=true](2.1,-0.2)(2.5,0.2)(3.2,0.4)(3.8,-0.2)
\psTangentLine[Tnormal,arrows=->,linecolor=red](2.5,0.2)(3.2,0.4)(3.8,-0.2){3.5}{0.5}
\psTangentLine[arrows=->,linecolor=blue](2.5,0.2)(3.2,0.4)(3.8,-0.2){3.5}{0.5}
\pcline[linestyle=dashed]{->}(OCurve)(ETangent|OCurve)\naput{$v_x$}
\pcline[linestyle=dashed]{->}(ETangent|OCurve)(ETangent)\naput{$v_y$}% double coordinate (x,y|x,y)
\end{pspicture}
\end{LTXexample}


\subsection{\nxLcs{psplotTangent} and option \nxLkeyword{Tnormal}}
%--------------------------------------------------------------------------------------
There is an additional option, named \Lkeyword{Derive} for an
alternative function (see following example) to calculate the
slope of the tangent. This will be in general the first
derivative, but can also be any other function. If this option is
different to to the default value \Lkeyset{Derive=default}, then this
function is taken to calculate the slope. For the other cases,
\LPack{pstricks-add} builds a secant with -0.00005<x<0.00005,
calculates the slope and takes this for the tangent. This may be
problematic in some cases of special functions or $x$ values, then
it may be appropriate to use the Derive option.

\begin{BDef}
\LcsStar{psplotTangent}\OptArgs\Largb{x}\Largb{dx}\Largb{function}
\end{BDef}



The macro expects three parameters:

\begin{description}
\item[$x$]: the $x$ value of the function for which the tangent should be calculated
\item[$dx$]: the $dx$ to both sides of the $x$ value
\item[$f(x)$]: the function in infix (with option \Lkeyword{algebraic}) or the default
postfix (PostScript) notation
\end{description}

The following examples show the use of the algebraic=true option together with the Derive option.
Remember that using the \Lkeyword{algebraic} option implies that the angles have to be in the
radian unit!

\begin{center}
\bgroup
\def\F{x RadtoDeg dup dup cos exch 2 mul cos add exch 3 mul cos add}
\def\Fp{x RadtoDeg dup dup sin exch 2 mul sin 2 mul add exch 3 mul sin 3 mul add neg}
\psset{plotpoints=1001}
\begin{pspicture}(-7.5,-2.5)(7.5,4)%X\psgrid
  \psaxes{->}(0,0)(-7.5,-2)(7.5,3.5)
  \psplot[linewidth=3\pslinewidth]{-7}{7}{\F}
  \psset{linecolor=red, arrows=<->, arrowscale=2}
  \multido{\n=-7+1}{8}{\psplotTangent{\n}{1}{\F}}
  \psset{linecolor=magenta, arrows=<->, arrowscale=2}%
  \multido{\n=0+1}{8}{\psplotTangent[linecolor=blue, Derive=\Fp]{\n}{1}{\F}}
\end{pspicture}
\egroup
\end{center}

\begin{lstlisting}
\def\F{x RadtoDeg dup dup cos exch 2 mul cos add exch 3 mul cos add}
\def\Fp{x RadtoDeg dup dup sin exch 2 mul sin 2 mul add exch 3 mul sin 3 mul add neg}
\psset{plotpoints=1001}
\begin{pspicture}(-7.5,-2.5)(7.5,4)%X\psgrid
  \psaxes{->}(0,0)(-7.5,-2)(7.5,3.5)
  \psplot[linewidth=3\pslinewidth]{-7}{7}{\F}
  \psset{linecolor=red, arrows=<->, arrowscale=2}
  \multido{\n=-7+1}{8}{\psplotTangent{\n}{1}{\F}}
  \psset{linecolor=magenta, arrows=<->, arrowscale=2}%
  \multido{\n=0+1}{8}{\psplotTangent[linecolor=blue, §\ON§Derive=\Fp§\OFF§]{\n}{1}{\F}}
\end{pspicture}
\end{lstlisting}

The star version plots only the tangent line in the positive $x$-direction:

\begin{center}
\bgroup
\def\Falg{cos(x)+cos(2*x)+cos(3*x)}   \def\Fpalg{-sin(x)-2*sin(2*x)-3*sin(3*x)}
\begin{pspicture}(-7.5,-2.5)(7.5,4)%\psgrid
  \psaxes{->}(0,0)(-7.5,-2)(7.5,3.5)
  \psplot[linewidth=1.5pt,algebraic=true,plotpoints=500]{-7.5}{7.5}{\Falg}
  \multido{\n=-7+1}{8}{\psplotTangent*[linecolor=red,arrows=->,arrowscale=2,algebraic=true]{\n}{1}{\Falg}}
  \multido{\n=0+1}{8}{\psplotTangent*[linecolor=magenta,%
     arrows=->,arrowscale=2,algebraic=true,Derive={\Fpalg}]{\n}{1}{\Falg}}
\end{pspicture}
\egroup
\end{center}

\begin{lstlisting}
\def\Falg{cos(x)+cos(2*x)+cos(3*x)}   \def\Fpalg{-sin(x)-2*sin(2*x)-3*sin(3*x)}
\begin{pspicture}(-7.5,-2.5)(7.5,4)%\psgrid
  \psaxes{->}(0,0)(-7.5,-2)(7.5,3.5)
  \psplot[linewidth=1.5pt,algebraic=true,plotpoints=500]{-7.5}{7.5}{\Falg}
  \multido{\n=-7+1}{8}{\psplotTangent*[linecolor=red,arrows=->,arrowscale=2,algebraic=true]{\n}{1}{\Falg}}
  \multido{\n=0+1}{8}{\psplotTangent*[linecolor=magenta,%
     arrows=->,arrowscale=2,algebraic=true,Derive={\Fpalg}]{\n}{1}{\Falg}}
\end{pspicture}
\end{lstlisting}

The next example shows the use of the \Lkeyword{Derive} option to draw
the perpendicular line to the tangent.

\begin{LTXexample}[width=8cm,wide]
\begin{pspicture}(-0.5,-0.5)(7.25,7.25)
  \def\Func{10 x div}
  \psaxes[arrowscale=1.5]{->}(7,7)
  \psplot[linewidth=2pt,algebraic=true]{1.5}{5}{10/x}
  \psplotTangent[linewidth=.5\pslinewidth,linecolor=red,algebraic=true]{3}{2}{10/x}
  \psplotTangent[linewidth=.5\pslinewidth,linecolor=blue,algebraic=true,Derive=(x*x)/10]{3}{2}{10/x}
  \psline[linestyle=dashed](!0 /x 3 def \Func)(!3 /x 3 def \Func)(3,0)
\end{pspicture}
\end{LTXexample}

By setting the optional argument \Lkeyword{Tnormal} one can plot the
normal of the tangent line. It always starts at the given point.

\begin{LTXexample}[width=8cm,wide]
\begin{pspicture}(-0.5,-0.5)(7.25,7.25)
  \def\Func{10 x div}
  \psaxes[arrowscale=1.5]{->}(7,7)
  \psplot[linewidth=2pt]{1.5}{5}{\Func}
  \psplotTangent[linewidth=1.5\pslinewidth,linecolor=red]{3}{2}{\Func}
  \psplotTangent[linewidth=1.5\pslinewidth,linecolor=blue,Tnormal]{3}{2}{\Func}
  \psline[linestyle=dashed](!0 /x 3 def \Func)(!3 /x 3 def \Func)(3,0)
\end{pspicture}
\end{LTXexample}


Let's work with the classical \Index{cardioid}: $r=2(1+\cos(\theta))$ and
$\displaystyle \frac{d r}{d\theta}=-2\sin(\theta)$. The \Lkeyword{Derive}
option always expects the $\frac{d r}{d\theta}$ value and uses
internally the equation for the derivative of implicitly defined
functions:

\[
\frac{dy}{dx}=\frac{r^\prime\cdot\sin\theta + x}{r^\prime\cdot\cos\theta - y}
\]
where $x=r\cdot\cos\theta$ and $y=r\cdot\sin\theta$


\begin{LTXexample}[width=6cm,wide]
\begin{pspicture}(-1,-3)(5,3)%\psgrid[subgridcolor=lightgray]
  \psaxes{->}(0,0)(-1,-3)(5,3)
  \psplot[polarplot,linewidth=3\pslinewidth,linecolor=blue,%
     plotpoints=500]{0}{360}{1 x cos add 2 mul}
\end{pspicture}
\end{LTXexample}

\psset{algebraic=false}
\begin{LTXexample}[width=6cm,wide]
\begin{pspicture}(-1,-3)(5,3)%\psgrid[subgridcolor=lightgray]
  \psaxes{->}(0,0)(-1,-3)(5,3)
  \psplot[polarplot,linewidth=3\pslinewidth,linecolor=blue,plotpoints=500]{0}{360}{1 x cos add 2 mul}
  \multido{\n=0+36}{10}{%
     \psplotTangent[polarplot,linecolor=red,arrows=<->]{\n}{1.5}{1 x cos add 2 mul} }
\end{pspicture}
\end{LTXexample}

\begin{LTXexample}[width=6cm,wide]
\begin{pspicture}(-1,-3)(5,3)%\psgrid[subgridcolor=lightgray]
  \psaxes{->}(0,0)(-1,-3)(5,3)
  \psplot[polarplot,linewidth=3\pslinewidth,linecolor=blue,algebraic=true,plotpoints=500]{0}{6.289}{2*(1+cos(x))}
  \multido{\r=0.000+0.314}{21}{%
     \psplotTangent[polarplot,Derive=-2*sin(x),algebraic=true,linecolor=red,arrows=<->]{\r}{1.5}{2*(1+cos(x))} }
\end{pspicture}
\end{LTXexample}


Let's work with a \Index{Lissajou curve}:
 $\displaystyle\left\{\begin{array}{l}x=3.5\cos(2t)\\y=3.5\sin(6t)\end{array}\right.$
whose derivative is :
 $\displaystyle\left\{\begin{array}{l}x=-7\sin(2t)\\y=21\cos(6t)\end{array}\right.$

The parameter must be the letter $t$ instead of $x$ and when using
the \Lkeyword{algebraic=true} option you must separate the two equations by
a \Lnotation{|} (see example).

\begin{LTXexample}[pos=t,wide]
\def\Lissa{t dup 2 RadtoDeg mul cos 3.5 mul exch 6 mul RadtoDeg sin 3.5 mul}%
\psset{yunit=0.6}
\begin{pspicture}(-4,-4)(4,6)
  \parametricplot[plotpoints=500,linewidth=3\pslinewidth]{0}{3.141592}{\Lissa}
  \multido{\r=0.000+0.314}{11}{%
    \psplotTangent[linecolor=red,arrows=<->]{\r}{1.5}{\Lissa} }
  \multido{\r=0.157+0.314}{11}{%
    \psplotTangent[linecolor=blue,arrows=<->]{\r}{1.5}{\Lissa} }
\end{pspicture}\hfill%
\def\LissaAlg{3.5*cos(2*t)|3.5*sin(6*t)}  \def\LissaAlgDer{-7*sin(2*t)|21*cos(6*t)}%
\begin{pspicture}(-4,-4)(4,6)
  \parametricplot[algebraic=true,plotpoints=500,linewidth=3\pslinewidth]{0}{3.141592}{\LissaAlg}
  \multido{\r=0.000+0.314}{11}{%
    \psplotTangent[algebraic=true,linecolor=red,arrows=<->]{\r}{1.5}{\LissaAlg} }
  \multido{\r=0.157+0.314}{11}{%
    \psplotTangent[algebraic=true,linecolor=blue,arrows=<->,%
       Derive=\LissaAlgDer]{\r}{1.5}{\LissaAlg} }
\end{pspicture}
\end{LTXexample}


\clearpage
\section{Successive derivatives of a function}

The new PostScript function \Lps{Derive} has been added for
plotting successive derivatives of a function. It must be used
with the \Lkeyword{algebraic=true} option. This function has two arguments:

\begin{enumerate}
\item a positive integer which defines the order of the derivative; obviously $0$ means the
  function itself!
\item a function of variable $x$ which can be any function using common operators,
\end{enumerate}

Do not think that the derivative is approximated, the internal PostScript engine will
compute the real derivative using a formal derivative engine.

The following diagram contains the plot of the polynomial:

\[ f(x)=\sum_{i=0}^{14}\frac{(-1)^{i}x^{2i}}{i!}=1-\frac{x^2}{2}+\frac{x^4}{4!}-\frac{x^6}{6!}+\frac{x^8}{8!}-
          \frac{x^{10}}{10!}+\frac{x^{12}}{12!}-\frac{x^{14}}{14!}\]

and of its first 15 derivatives. It is the sequence definition of
the cosine.


\begin{LTXexample}[pos=t,wide,preset=\centering]
\psset{unit=2}
\def\getColor#1{\ifcase#1 Tan\or RedOrange\or magenta\or yellow\or green\or Orange\or blue\or
  DarkOrchid\or BrickRed\or Rhodamine\or OliveGreen\or Goldenrod\or Mahogany\or
  OrangeRed\or CarnationPink\or RoyalPurple\or Lavender\fi}
\begin{pspicture}[showgrid=true](0,-1.2)(7,1.5)
  \psclip{\psframe[linestyle=none](0,-1.1)(7,1.1)}
  \multido{\in=0+1}{16}{%
     \psplot[linewidth=1pt,algebraic=true,linecolor=\getColor{\in}]{0}{7}
      {Derive(\in,1-x^2/2+x^4/24-x^6/720+x^8/40320-x^10/3628800+x^12/479001600-x^14/87178291200)}}
  \endpsclip
\end{pspicture}
\end{LTXexample}

\begin{LTXexample}[width=3.5cm]
\begin{pspicture}[shift=-2.5,showgrid=true,linewidth=1pt](0,-2)(3,3)
  \psplot[algebraic=true]{.001}{3}{x*ln(x)}  % f(x)
  \psplot[algebraic=true,linecolor=red]{.05}{3}{Derive(1,x*ln(x))} % f'(x)=1+ln(x)
\end{pspicture}
\end{LTXexample}


\clearpage
\section{Variable step for plotting a curve}
\subsection{Theory}

As you know with the \Lcs{psplot} macro, the curve is plotted
using a piece-wise linear curve. The step is given by the
parameter \Lkeyword{plotpoints}. For each step between $x_i$ and
$x_{i+1}$, the area defined between the curve and its
approximation (a segment) is majored by this formula :

\begin{minipage}[m]{.5\linewidth}
\[|\varepsilon|\le\frac{M_2(f)(x_{i+1}-x_i)^3}{12}\]

$M_2(f)$ is a majorant of the second derivative of $f$ in the interval $[x_i;x_{i+1}]$.
\end{minipage}
{\psset{unit=1cm, showpoints=false}
\begin{pspicture}[shift=-2,showgrid=true](0,-1)(6,3)
  \pscurve(0,0)(1,1)(3,2.2)(5,2)(6,1)\psline(1,1)(5,2)
  \psline(.5,0)(5.5,0)\psline(1,0)(1,1)\psline(5,0)(5,2)
  \rput[t](1,-.1){$x_n$}\rput[t](5,-.1){$x_{n+1}$}
  \psclip{\pscustom{\psecurve(0,0)(1,1)(3,2.2)(5,2)(6,1)\psline(5,2)}}
    \psframe[fillstyle=solid, fillcolor=gray](0,0)(5,5)
  \endpsclip
  \rput*(3,1.8){$\varepsilon$}
\end{pspicture}}



The parameter \Lkeyword{VarStep} (\false\ by default) activates
the variable step algorithm. It is set to a tolerance defined by
the parameter \Lkeyword{VarStepEpsilon} (\Lkeyval{default} by default,
accept real value). If this parameter is not set by the user, then
it is automatically computed using the default first step given by
the parameter \Lkeyword{plotpoints}. Then, for each step, $f''(x_n)$
and $f''(x_{n+1})$ are computed and the smaller is used as
$M_2(f)$, and then the step is approximated. This means that the
step is constant for second order polynomials.

\subsection{The cosine}

Different value for the tolerance from $0.01$ to $0.000\,1$, a factor $10$ between
each of them. In black, there is the classic \Lcs{psplot} behavior, and in
magenta the default variable step behavior.

\begin{center}
\bgroup
\psset{algebraic=true, VarStep=true, unit=2, showpoints=true, linecolor=red}
\begin{pspicture}(-0,-1)(3.14,2)\psgrid
  \psplot[VarStepEpsilon=.01]{0}{3.14}{cos(x)}
  \psplot[VarStepEpsilon=.001]{0}{3.14}{cos(x)+.15}
  \psplot[VarStepEpsilon=.0001]{0}{3.14}{cos(x)+.3}
  \psplot[linecolor=magenta]{0}{3.14}{cos(x)+.45}
  \psplot[VarStep=false, linewidth=2\pslinewidth, linecolor=black]{-0}{3.14}{cos(x)+.6}
\end{pspicture}
\egroup
\end{center}

\begin{lstlisting}
\psset{algebraic=true, VarStep=true, unit=2, showpoints=true, linecolor=red}
\begin{pspicture}[showgrid=true](-0,-1)(3.14,2)
  \psplot[VarStepEpsilon=.01]{0}{3.14}{cos(x)}
  \psplot[VarStepEpsilon=.001]{0}{3.14}{cos(x)+.15}
  \psplot[VarStepEpsilon=.0001]{0}{3.14}{cos(x)+.3}
  \psplot[linecolor=magenta]{0}{3.14}{cos(x)+.45}
  \psplot[VarStep=false,linewidth=1pt,linecolor=black]{-0}{3.14}{cos(x)+.6}
\end{pspicture}
\end{lstlisting}


\subsection{The Napierian Logarithm}

A really classic example which gives a bad beginning, the tolerance is set to $0.001$.

\begin{center}
\bgroup
\psset{algebraic=true, VarStep=true, linecolor=red, showpoints=true}
\begin{pspicture}[showgrid=true](0,-5)(16,4)
  \psplot[VarStep=false, linecolor=black]{.01}{16}{ln(x)+1}
  \psplot[linecolor=magenta]{.51}{16}{ln(x-1/2)+1/2}
  \psplot[VarStepEpsilon=.001]{1.01}{16}{ln(x-1)}
  \psplot[VarStepEpsilon=.01]{1.51}{16}{ln(x-1.5)-100/200}
\end{pspicture}
\egroup
\end{center}

\begin{lstlisting}
\psset{algebraic=true, VarStep=true, linecolor=red, showpoints=true}
\begin{pspicture}[showgrid=true](0,-5)(16,4)
  \psplot[VarStep=false, linecolor=black]{.01}{16}{ln(x)+1}
  \psplot[linecolor=magenta]{.51}{16}{ln(x-1/2)+1/2}
  \psplot[VarStepEpsilon=.001]{1.01}{16}{ln(x-1)}
  \psplot[VarStepEpsilon=.01]{1.51}{16}{ln(x-1.5)-100/200}
\end{pspicture}
\end{lstlisting}


\clearpage
\subsection{Sine of the inverse of $x$}
Impossible to draw, but let's try!

\begin{center}
\bgroup
\psset{xunit=64,algebraic=true,VarStep,linecolor=red,showpoints=true,linewidth=1pt}
\begin{pspicture}[showgrid=true](0,-1)(.5,1)
  \psplot[VarStepEpsilon=.0001]{.01}{.25}{sin(1/x)}
\end{pspicture}\\
\begin{pspicture}[showgrid=true](0,-1)(.5,1)
  \psplot[VarStepEpsilon=.00001]{.01}{.25}{sin(1/x)}
\end{pspicture}\\
\begin{pspicture}[showgrid=true](0,-1)(.5,1)
  \psplot[VarStepEpsilon=.000001]{.01}{.25}{sin(1/x)}
\end{pspicture}\\
\begin{pspicture}[showgrid=true](0,-1)(.5,1)
  \psplot[VarStep=false, linecolor=black]{.01}{.25}{sin(1/x)}
\end{pspicture}
\egroup
\end{center}

\begin{lstlisting}
\psset{xunit=64,algebraic=true,VarStep,linecolor=red,showpoints=true,linewidth=1pt}
\begin{pspicture}[showgrid=true](0,-1)(.5,1)
  \psplot[VarStepEpsilon=.0001]{.01}{.25}{sin(1/x)}
\end{pspicture}\\
\begin{pspicture}[showgrid=true](0,-1)(.5,1)
  \psplot[VarStepEpsilon=.00001]{.01}{.25}{sin(1/x)}
\end{pspicture}\\
\begin{pspicture}[showgrid=true](0,-1)(.5,1)
  \psplot[VarStepEpsilon=.000001]{.01}{.25}{sin(1/x)}
\end{pspicture}\\
\begin{pspicture}[showgrid=true](0,-1)(.5,1)
  \psplot[VarStep=false, linecolor=black]{.01}{.25}{sin(1/x)}
\end{pspicture}
\end{lstlisting}





\clearpage
\subsection{A really complicated function}

Just appreciate the difference between the normal behavior and the plotting with the
\Lkeyword{varStep} option. The function is:

\[f(x)=x-\frac{x^2}{10}+\ln(x)+\cos(2x)+\sin(x^2)-1\]

\begin{center}
\bgroup
\psset{xunit=3, algebraic=true, VarStep, showpoints=true}
\begin{pspicture}[showgrid=true](0,-2)(5,6)
  \psplot[VarStepEpsilon=.0005, linecolor=red]{.1}{5}{x-x^2/10+ln(x)+cos(2*x)+sin(x^2)}
  \psplot[linecolor=magenta]{.1}{5}{x-x^2/10+ln(x)+cos(2*x)+sin(x^2)+.5}
  \psplot[VarStep=false]{.1}{5}{x-x^2/10+ln(x)+cos(2*x)+sin(x^2)-1}
\end{pspicture}
\egroup
\end{center}

\begin{lstlisting}
\psset{xunit=3, algebraic=true, VarStep, showpoints=true}
\begin{pspicture}[showgrid=true](0,-2)(5,6)
  \psplot[VarStepEpsilon=.0005, linecolor=red]{.1}{5}{x-x^2/10+ln(x)+cos(2*x)+sin(x^2)}
  \psplot[linecolor=magenta]{.1}{5}{x-x^2/10+ln(x)+cos(2*x)+sin(x^2)+.5}
  \psplot[VarStep=false]{.1}{5}{x-x^2/10+ln(x)+cos(2*x)+sin(x^2)-1}
\end{pspicture}
\end{lstlisting}


\clearpage
\subsection{A hyperbola}

\begin{center}
\bgroup
\psset{algebraic=true, showpoints=true, unit=0.75}
\begin{pspicture}(-5,-4)(9,6)
  \psplot[linecolor=black]{-5}{1.8}{(x-1)/(x-2)}
  \psplot[VarStep=true, VarStepEpsilon=.001, linecolor=red]{2.2}{9}{(x-1)/(x-2)}
  \psaxes{->}(0,0)(-5,-4)(9,6)
\end{pspicture}
\egroup
\end{center}

\begin{lstlisting}
\psset{algebraic=true, showpoints=true, unit=0.75}
\begin{pspicture}(-5,-4)(9,6)
  \psplot[linecolor=black]{-5}{1.8}{(x-1)/(x-2)}
  \psplot[VarStep=true, VarStepEpsilon=.001, linecolor=red]{2.2}{9}{(x-1)/(x-2)}
  \psaxes{->}(0,0)(-5,-4)(9,6)
\end{pspicture}
\end{lstlisting}



\clearpage
\subsection{Using \nxLcs{psparametricplot}}

\begin{BDef}
\Lcs{parametricplot}\OptArgs\Largb{t0}\Largb{t1}\OptArg{PS commands}\Largb{x(t) y(t)}
\end{BDef}

\begin{center}
\bgroup
\psset{unit=2.5}
\begin{pspicture}[showgrid=true](-1,-1)(1,1)
\parametricplot[algebraic=true,linecolor=red,VarStep=true, showpoints=true,
                VarStepEpsilon=.0001]
                {-3.14}{3.14}{cos(3*t)|sin(2*t)}
\end{pspicture}
\begin{pspicture}[showgrid=true](-1,-1)(1,1)
\parametricplot[algebraic=true,linecolor=blue,VarStep=true, showpoints=false,
                VarStepEpsilon=.0001]
                {-3.14}{3.14}{cos(3*t)|sin(2*t)}
\end{pspicture}
\egroup
\end{center}

\begin{lstlisting}
\psset{unit=3}
\begin{pspicture}[showgrid=true](-1,-1)(1,1)
\parametricplot[algebraic=true,linecolor=red,VarStep=true, showpoints=true,
                VarStepEpsilon=.0001]
                {-3.14}{3.14}{cos(3*t)|sin(2*t)}
\end{pspicture}
\begin{pspicture}[showgrid=true](-1,-1)(1,1)
\parametricplot[algebraic=true,linecolor=blue,VarStep=true, showpoints=false,
                VarStepEpsilon=.0001]
                {-3.14}{3.14}{cos(3*t)|sin(2*t)}
\end{pspicture}
\end{lstlisting}


\begin{center}
\bgroup
\psset{unit=2.5}
\begin{pspicture}[showgrid=true](-1,-1)(1,1)
\parametricplot[algebraic=true,linecolor=red,VarStep=true, showpoints=true,
                VarStepEpsilon=.0001]
                {0}{47.115}{cos(5*t)|sin(3*t)}
\end{pspicture}
\begin{pspicture}[showgrid=true](-1,-1)(1,1)
\parametricplot[algebraic=true,linecolor=blue,VarStep=true, showpoints=false,
                VarStepEpsilon=.0001]
                {0}{47.115}{cos(5*t)|sin(3*t)}
\end{pspicture}
\egroup
\end{center}

\begin{lstlisting}
\psset{unit=2.5}
\begin{pspicture}[showgrid=true](-1,-1)(1,1)
\parametricplot[algebraic=true,linecolor=red,VarStep=true, showpoints=true,
                VarStepEpsilon=.0001]
                {0}{47.115}{cos(5*t)|sin(3*t)}
\end{pspicture}
\begin{pspicture}[showgrid=true](-1,-1)(1,1)
\parametricplot[algebraic=true,linecolor=blue,VarStep=true, showpoints=false,
                VarStepEpsilon=.0001]
                {0}{47.115}{cos(5*t)|sin(3*t)}
\end{pspicture}
\end{lstlisting}


\begin{center}
\bgroup
\psset{xunit=.5}
\begin{pspicture}[showgrid=true](0,0)(12.566,2)
\parametricplot[algebraic=true,linecolor=red,VarStep, showpoints=true,
        VarStepEpsilon=.01]{0}{12.566}{t+cos(-t-Pi/2)|1+sin(-t-Pi/2)}
\end{pspicture}
%
\begin{pspicture}[showgrid=true](0,0)(12.566,2)
\parametricplot[algebraic=true,linecolor=blue,VarStep, showpoints=false,
        VarStepEpsilon=.001]{0}{12.566}{t+cos(-t-Pi/2)|1+sin(-t-Pi/2)}
\end{pspicture}
\egroup
\end{center}

\begin{lstlisting}
\psset{xunit=.5}
\begin{pspicture}[showgrid=true](0,0)(12.566,2)
\parametricplot[algebraic=true,linecolor=red,VarStep, showpoints=true,
        VarStepEpsilon=.01]{0}{12.566}{t+cos(-t-Pi/2)|1+sin(-t-Pi/2)}
\end{pspicture}
%
\begin{pspicture}[showgrid=true](0,0)(12.566,2)
\parametricplot[algebraic=true,linecolor=blue,VarStep, showpoints=false,
        VarStepEpsilon=.001]{0}{12.566}{t+cos(-t-Pi/2)|1+sin(-t-Pi/2)}
\end{pspicture}
\end{lstlisting}


\section{New math functions and their derivatives}

\subsection{The inverse sine and its derivative}

\begin{center}
\bgroup
\psset{unit=1.5}
\begin{pspicture}[showgrid=true](-1,-2)(1,2)
  \psplot[linecolor=blue,algebraic=true]{-1}{1}{asin(x)}
\end{pspicture}
\hspace{1em}
\psset{algebraic, VarStep, VarStepEpsilon=.001, showpoints=true}
\begin{pspicture}[showgrid=true](-1,-2)(1,2)
  \psplot[linecolor=blue]{-.999}{.999}{asin(x)}
\end{pspicture}
\hspace{1em}
\begin{pspicture}[showgrid=true](-1,0)(1,4)
  \psplot[linecolor=blue]{-.97}{.97}{Derive(1,asin(x))}
\end{pspicture}
\hspace{1em}
\psset{algebraic=true, VarStep, VarStepEpsilon=.0001, showpoints=true}
\begin{pspicture}[showgrid=true](-1,0)(1,4)
  \psplot[linecolor=blue]{-.97}{.97}{Derive(1,asin(x))}
\end{pspicture}
\egroup
\end{center}

\begin{lstlisting}
\psset{unit=1.5}
\begin{pspicture}[showgrid=true](-1,-2)(1,2)
  \psplot[linecolor=blue,algebraic=true]{-1}{1}{asin(x)}
\end{pspicture}
\hspace{1em}
\psset{algebraic=true, VarStep, VarStepEpsilon=.001, showpoints=true}
\begin{pspicture}[showgrid=true](-1,-2)(1,2)
  \psplot[linecolor=blue]{-.999}{.999}{asin(x)}
\end{pspicture}
\hspace{1em}
\begin{pspicture}[showgrid=true](-1,0)(1,4)
  \psplot[linecolor=red]{-.97}{.97}{Derive(1,asin(x))}
\end{pspicture}
\hspace{1em}
\psset{algebraic=true, VarStep, VarStepEpsilon=.0001, showpoints=true}
\begin{pspicture}[showgrid=true](-1,0)(1,4)
  \psplot[linecolor=red]{-.97}{.97}{Derive(1,asin(x))}
\end{pspicture}
\end{lstlisting}


\subsection{The inverse cosine and its derivative}

\begin{center}
\bgroup
\psset{unit=1.5}
\begin{pspicture}[showgrid=true](-1,0)(1,3)
  \psplot[linecolor=blue,algebraic=true]{-1}{1}{acos(x)}
\end{pspicture}
\hspace{1em}
\psset{algebraic=true, VarStep, VarStepEpsilon=.001, showpoints=true}
\begin{pspicture}[showgrid=true](-1,0)(1,3)
  \psplot[linecolor=blue]{-.999}{.999}{acos(x)}
\end{pspicture}
\hspace{1em}
\begin{pspicture}[showgrid=true](-1,-4)(1,-1)
  \psplot[linecolor=blue]{-.97}{.97}{Derive(1,acos(x))}
\end{pspicture}
\hspace{1em}
\psset{algebraic=true, VarStep, VarStepEpsilon=.0001, showpoints=true}
\begin{pspicture}[showgrid=true](-1,-4)(1,-1)
  \psplot[linecolor=blue]{-.97}{.97}{Derive(1,acos(x))}
\end{pspicture}
\egroup
\end{center}

\begin{lstlisting}
\psset{unit=1.5}
\begin{pspicture}[showgrid=true](-1,0)(1,3)
  \psplot[linecolor=blue,algebraic=true]{-1}{1}{acos(x)}
\end{pspicture}
\hspace{1em}
\psset{algebraic=true, VarStep, VarStepEpsilon=.001, showpoints=true}
\begin{pspicture}[showgrid=true](-1,0)(1,3)
  \psplot[linecolor=blue]{-.999}{.999}{acos(x)}
\end{pspicture}
\hspace{1em}
\begin{pspicture}[showgrid=true](-1,-4)(1,-1)
  \psplot[linecolor=red]{-.97}{.97}{Derive(1,acos(x))}
\end{pspicture}
\hspace{1em}
\psset{algebraic=true, VarStep, VarStepEpsilon=.0001, showpoints=true}
\begin{pspicture}[showgrid=true](-1,-4)(1,-1)
  \psplot[linecolor=red]{-.97}{.97}{Derive(1,acos(x))}
\end{pspicture}
\end{lstlisting}



\subsection{The inverse tangent and its derivative}

\begin{center}
\bgroup
\begin{pspicture}[showgrid=true](-4,-2)(4,2)
\psset{algebraic=true}
  \psplot[linecolor=blue,linewidth=1pt]{-4}{4}{atg(x)}
  \psplot[linecolor=red,VarStep, VarStepEpsilon=.0001, showpoints=true]{-4}{4}{Derive(1,atg(x))}
\end{pspicture}
\hspace{1em}
\begin{pspicture}[showgrid=true](-4,-2)(4,2)
\psset{algebraic=true, VarStep, VarStepEpsilon=.001, showpoints=true}
  \psplot[linecolor=blue]{-4}{4}{atg(x)}
  \psplot[linecolor=red]{-4}{4}{Derive(1,atg(x))}
\end{pspicture}
\egroup
\end{center}

\begin{lstlisting}
\begin{pspicture}[showgrid=true](-4,-2)(4,2)
\psset{algebraic=true}
  \psplot[linecolor=blue,linewidth=1pt]{-4}{4}{atg(x)}
  \psplot[linecolor=red,VarStep, VarStepEpsilon=.0001, showpoints=true]{-4}{4}{Derive(1,atg(x))}
\end{pspicture}
\hspace{1em}
\begin{pspicture}[showgrid=true](-4,-2)(4,2)
\psset{algebraic=true, VarStep, VarStepEpsilon=.001, showpoints=true}
  \psplot[linecolor=blue]{-4}{4}{atg(x)}
  \psplot[linecolor=red]{-4}{4}{Derive(1,atg(x))}
\end{pspicture}
\end{lstlisting}

\subsection{Hyperbolic functions}

\begin{center}
\bgroup
\begin{pspicture}(-3,-4)(3,4)
\psset{algebraic=true}
  \psplot[linecolor=red,linewidth=1pt]{-2}{2}{sh(x)}
  \psplot[linecolor=blue,linewidth=1pt]{-2}{2}{ch(x)}
  \psplot[linecolor=green,linewidth=1pt]{-3}{3}{th(x)}
  \psaxes{->}(0,0)(-3,-4)(3,4)
\end{pspicture}
\hspace{1em}
\begin{pspicture}(-3,-4)(3,4)
\psset{algebraic=true, VarStep=true, VarStepEpsilon=.001, showpoints=true}
  \psplot[linecolor=red,linewidth=1pt]{-2}{2}{sh(x)}
  \psplot[linecolor=blue,linewidth=1pt]{-2}{2}{ch(x)}
  \psplot[linecolor=green,linewidth=1pt]{-3}{3}{th(x)}
  \psaxes{->}(0,0)(-3,-4)(3,4)
\end{pspicture}
\egroup
\end{center}

\begin{lstlisting}
\begin{pspicture}(-3,-4)(3,4)
\psset{algebraic=true}
  \psplot[linecolor=red,linewidth=1pt]{-2}{2}{sh(x)}
  \psplot[linecolor=blue,linewidth=1pt]{-2}{2}{ch(x)}
  \psplot[linecolor=green,linewidth=1pt]{-3}{3}{th(x)}
  \psaxes{->}(0,0)(-3,-4)(3,4)
\end{pspicture}
\hspace{1em}
\begin{pspicture}(-3,-4)(3,4)
\psset{algebraic=true, VarStep=true, VarStepEpsilon=.001, showpoints=true}
  \psplot[linecolor=red,linewidth=1pt]{-2}{2}{sh(x)}
  \psplot[linecolor=blue,linewidth=1pt]{-2}{2}{ch(x)}
  \psplot[linecolor=green,linewidth=1pt]{-3}{3}{th(x)}
  \psaxes{->}(0,0)(-3,-4)(3,4)
\end{pspicture}
\end{lstlisting}



\begin{center}
\bgroup
\begin{pspicture}(-3,-4)(3,4)
\psset{algebraic=true}
  \psplot[linecolor=red,linewidth=1pt]{-2}{2}{Derive(1,sh(x))}
  \psplot[linecolor=blue,linewidth=1pt]{-2}{2}{Derive(1,ch(x))}
  \psplot[linecolor=green,linewidth=1pt]{-3}{3}{Derive(1,th(x))}
  \psaxes{->}(0,0)(-3,-4)(3,4)
\end{pspicture}
\hspace{1em}
\begin{pspicture}(-3,-4)(3,4)
\psset{algebraic=true, VarStep=true, VarStepEpsilon=.001, showpoints=true}
  \psplot[linecolor=red,linewidth=1pt]{-2}{2}{Derive(1,sh(x))}
  \psplot[linecolor=blue,linewidth=1pt]{-2}{2}{Derive(1,ch(x))}
  \psplot[linecolor=green,linewidth=1pt]{-3}{3}{Derive(1,th(x))}
  \psaxes{->}(0,0)(-3,-4)(3,4)
\end{pspicture}
\egroup
\end{center}

\begin{lstlisting}
\begin{pspicture}(-3,-4)(3,4)
\psset{algebraic=true,linewidth=1pt}
  \psplot[linecolor=red,linewidth=1pt]{-2}{2}{Derive(1,sh(x))}
  \psplot[linecolor=blue,linewidth=1pt]{-2}{2}{Derive(1,ch(x))}
  \psplot[linecolor=green,linewidth=1pt]{-3}{3}{Derive(1,th(x))}
  \psaxes{->}(0,0)(-3,-4)(3,4)
\end{pspicture}
\hspace{1em}
\begin{pspicture}(-3,-4)(3,4)
\psset{algebraic=true, VarStep=true, VarStepEpsilon=.001, showpoints=true}
  \psplot[linecolor=red,linewidth=1pt]{-2}{2}{Derive(1,sh(x))}
  \psplot[linecolor=blue,linewidth=1pt]{-2}{2}{Derive(1,ch(x))}
  \psplot[linecolor=green,linewidth=1pt]{-3}{3}{Derive(1,th(x))}
  \psaxes{->}(0,0)(-3,-4)(3,4)
\end{pspicture}
\end{lstlisting}



\begin{center}
\bgroup
\begin{pspicture}(-7,-3)(7,3)
\psset{algebraic=true}
  \psplot[linecolor=red,linewidth=1pt]{-7}{7}{Argsh(x)}
  \psplot[linecolor=blue,linewidth=1pt]{1}{7}{Argch(x)}
  \psplot[linecolor=green,linewidth=1pt]{-.99}{.99}{Argth(x)}
  \psaxes{->}(0,0)(-7,-3)(7,3)
\end{pspicture}\\[\baselineskip]
\begin{pspicture}(-7,-3)(7,3)
  \psset{algebraic=true, VarStep, VarStepEpsilon=.001, showpoints=true}
  \psplot[linecolor=red,linewidth=1pt]{-7}{7}{Argsh(x)}
  \psplot[linecolor=blue,linewidth=1pt]{1.001}{7}{Argch(x)}
  \psplot[linecolor=green,linewidth=1pt]{-.99}{.99}{Argth(x)}
  \psaxes{->}(0,0)(-7,-3)(7,3)
\end{pspicture}
\egroup
\end{center}

\begin{lstlisting}
\begin{pspicture}(-7,-3)(7,3)
\psset{algebraic=true}
  \psplot[linecolor=red,linewidth=1pt]{-7}{7}{Argsh(x)}
  \psplot[linecolor=blue,linewidth=1pt]{1}{7}{Argch(x)}
  \psplot[linecolor=green,linewidth=1pt]{-.99}{.99}{Argth(x)}
  \psaxes{->}(0,0)(-7,-3)(7,3)
\end{pspicture}\\[\baselineskip]
\begin{pspicture}(-7,-3)(7,3)
  \psset{algebraic=true, VarStep, VarStepEpsilon=.001, showpoints=true}
  \psplot[linecolor=red,linewidth=1pt]{-7}{7}{Argsh(x)}
  \psplot[linecolor=blue,linewidth=1pt]{1.001}{7}{Argch(x)}
  \psplot[linecolor=green,linewidth=1pt]{-.99}{.99}{Argth(x)}
  \psaxes{->}(0,0)(-7,-3)(7,3)
\end{pspicture}
\end{lstlisting}



\begin{center}
\bgroup
\begin{pspicture}(-7,-0.5)(7,6)
\psset{algebraic=true}
  \psplot[linecolor=red,linewidth=1pt]{-7}{7}{Derive(1,Argsh(x))}
  \psplot[linecolor=blue,linewidth=1pt]{1.014}{7}{Derive(1,Argch(x))}
  \psplot[linecolor=green,linewidth=1pt]{-.9}{.9}{Derive(1,Argth(x))}
  \psaxes{->}(0,0)(-7,0)(7,6)
\end{pspicture}\\[\baselineskip]
\begin{pspicture}(-7,-0.5)(7,6)
\psset{algebraic=true}
  \psset{algebraic=true, VarStep=true, VarStepEpsilon=.001, showpoints=true}
  \psplot[linecolor=red,linewidth=1pt]{-7}{7}{Derive(1,Argsh(x))}
  \psplot[linecolor=blue,linewidth=1pt]{1.014}{7}{Derive(1,Argch(x))}
  \psplot[linecolor=green,linewidth=1pt]{-.9}{.9}{Derive(1,Argth(x))}
  \psaxes{->}(0,0)(-7,0)(7,6)
\end{pspicture}
\egroup
\end{center}

\begin{lstlisting}
\begin{pspicture}(-7,-0.5)(7,6)
\psset{algebraic=true}
  \psplot[linecolor=red,linewidth=1pt]{-7}{7}{Derive(1,Argsh(x))}
  \psplot[linecolor=blue,linewidth=1pt]{1.014}{7}{Derive(1,Argch(x))}
  \psplot[linecolor=green,linewidth=1pt]{-.9}{.9}{Derive(1,Argth(x))}
  \psaxes{->}(0,0)(-7,0)(7,6)
\end{pspicture}\\[\baselineskip]
\begin{pspicture}(-7,-0.5)(7,6)
\psset{algebraic=true}
  \psset{algebraic=true, VarStep=true, VarStepEpsilon=.001, showpoints=true}
  \psplot[linecolor=red,linewidth=1pt]{-7}{7}{Derive(1,Argsh(x))}
  \psplot[linecolor=blue,linewidth=1pt]{1.014}{7}{Derive(1,Argch(x))}
  \psplot[linecolor=green,linewidth=1pt]{-.9}{.9}{Derive(1,Argth(x))}
  \psaxes{->}(0,0)(-7,0)(7,6)
\end{pspicture}
\end{lstlisting}


\clearpage
%--------------------------------------------------------------------------------------
\section[\nxLcs{psplotDiffEqn} -- solving diffential equations]%
  {\nxLcs{psplotDiffEqn} -- solving diffential equations}
%--------------------------------------------------------------------------------------


 A differential equation of first order is like

\begin{align} y^\prime=f(x,y,y^\prime) \end{align}


where $y$ is a function of $x$. We define some vectors $Y=[y, y',
\cdots , y^{(n-1)}]$ and $Y^\prime=[y^\prime, y^{\prime\prime},
\cdots , y^{n}]$, depending on the order $n$. The syntax of the
macro is

\begin{BDef}
\Lcs{psplotDiffEqn}\OptArgs\Largb{x0}\Largb{x1}\Largb{y0}\Largb{f(x,y,y',...)}
\end{BDef}

\begin{itemize}\setlength\itemsep{0pt}\setlength\parsep{0pt}\setlength\parskip{0pt}
\item \verb+options+: the \verb+\psplotDiffEqn+ specific options and all other of PSTricks, which
make sense;
\item $x_0$: the start value;
\item $x_1$: the end value of the definition interval;
\item $y_0$: the initial values for $y(x_0)\ y'(x_0)\ \ldots$;
\item $f(x,y,y',...)$: the differential equation, depending to the number of initial values, e.g.:
    \verb+{0 1}+ for $y_0$ are two initial values, so that we have a differential equation of
    second order $f(x,y,y')$ and the macro leaves $y\ y'$ on the stack.
\end{itemize}

The new options are:


\begin{itemize}\setlength\itemsep{0pt}\setlength\parsep{0pt}\setlength\parskip{0pt}
\item \Lkeyword{method}: integration method (\verb+euler+ for order 1 euler method, \verb+rk4+ for
  4\textsuperscript{th} order Runge-Kutta method);
\item \Lkeyword{whichabs}: select the abscissa for plotting the graph, by default it is
  $x$, but you can specify a number which represent a position in the vector $y$;
\item \Lkeyword{whichord}: same as precedent for the ordinate, by default $y(0)$;
\item \Lkeyword{plotfuncx}: describe a ps function for the abscissa, parameter
  \Lkeyword{whichabs} becomes useless;
\item \Lkeyword{plotfuncy}: idem for the ordinate;
\item \Lkeyword{buildvector}: boolean parameter for specifying the input-output of the
  $f$ description:
  \begin{description}
  \item[\texttt{true}] (default): $y$ is put on the stack element by element, $y'$
    must be given in the same way;
  \item[\texttt{false}]: $y$ is put on the stack as a vector, $y'$ must be returned
  in the same way;
  \end{description}

\item \Lkeyword{algebraic=true}: algebraic=true description for $f$, \Lkeyword{buildvector}
  parameter is useless when activating this option.
\end{itemize}



\clearpage
\subsection{Variable step for differential equations}

A new algorithm has been added for adjusting the step according to the variations of
the curve. The parameter \Lkeyword{method} has a new possible value : \Lkeyword{varrkiv} to
activate the \Index{Runge-Kutta} method with variable step, then the parameter
\Lkeyword{varsteptol} (real value; \verb+.01+ by default) can control the tolerance of
the algortihm.

\begin{center}
\bgroup
\def\Funct{neg}\def\FunctAlg{-y[0]}
\psset{xunit=1.5, yunit=8, showpoints=true}
\begin{pspicture}[showgrid=true](0,0)(10,1.2)
  \psplot[linewidth=6\pslinewidth, linecolor=green, showpoints=false]{0}{10}{Euler x neg exp}
  \psplotDiffEqn[linecolor=magenta, method=varrkiv, varsteptol=.1, plotpoints=2]{0}{10}{1}{\Funct}
  \rput(0,.0){\psplotDiffEqn[linecolor=blue, method=varrkiv, varsteptol=.01, plotpoints=2]{0}{10}{1}{\Funct}}
  \rput(0,.1){\psplotDiffEqn[linecolor=Orange, method=varrkiv, varsteptol=.001, plotpoints=2]{0}{10}{1}{\Funct}}
  \rput(0,.2){\psplotDiffEqn[linecolor=red, method=varrkiv, varsteptol=.0001, plotpoints=2]{0}{10}{1}{\Funct}}
  \psset{linewidth=4\pslinewidth,showpoints=false}
  \rput*(3.3,.9){\psline[linecolor=magenta](-.75cm,0)}
  \rput*[l](3.3,.9){\small RK ordre 4 : $\varepsilon<10^{-1}$}
  \rput*(3.3,.8){\psline[linecolor=blue](-.75cm,0)}
  \rput*[l](3.3,.8){\small RK ordre 4 : $\varepsilon<10^{-2}$}
  \rput*(3.3,.7){\psline[linecolor=Orange](-.75cm,0)}
  \rput*[l](3.3,.7){\small RK ordre 4 : $\varepsilon<10^{-3}$}
  \rput*(3.3,.6){\psline[linecolor=red](-.75cm,0)}
  \rput*[l](3.3,.6){\small RK ordre 4 : $\varepsilon<10^{-4}$}
  \rput*(3.3,.5){\psline[linecolor=green](-.75cm,0)}
  \rput*[l](3.3,.5){\small solution exacte}
\end{pspicture}
{\captionof{figure}{Equation $y'=-y$ with $y_0=1$.}\label{fig:minusexpvarstep}}
\egroup
\end{center}


\begin{lstlisting}[wide=true]
\def\Funct{neg}\def\FunctAlg{-y[0]}
\psset{xunit=1.5, yunit=8, showpoints=true}
\begin{pspicture}[showgrid=true](0,0)(10,1.2)
  \psplot[linewidth=6\pslinewidth, linecolor=green, showpoints=false]{0}{10}{Euler x neg exp}
  \psplotDiffEqn[linecolor=magenta, method=varrkiv, varsteptol=.1, plotpoints=2]{0}{10}{1}{\Funct}
  \rput(0,.0){\psplotDiffEqn[linecolor=blue, method=varrkiv, varsteptol=.01, plotpoints=2]{0}{10}{1}{\Funct}}
  \rput(0,.1){\psplotDiffEqn[linecolor=Orange, method=varrkiv, varsteptol=.001, plotpoints=2]{0}{10}{1}{\Funct}}
  \rput(0,.2){\psplotDiffEqn[linecolor=red, method=varrkiv, varsteptol=.0001, plotpoints=2]{0}{10}{1}{\Funct}}
  \psset{linewidth=4\pslinewidth,showpoints=false}
  \rput*(3.3,.9){\psline[linecolor=magenta](-.75cm,0)}
  \rput*[l](3.3,.9){\small RK ordre 4 : $\varepsilon<10^{-1}$}
  \rput*(3.3,.8){\psline[linecolor=blue](-.75cm,0)}
  \rput*[l](3.3,.8){\small RK ordre 4 : $\varepsilon<10^{-2}$}
  \rput*(3.3,.7){\psline[linecolor=Orange](-.75cm,0)}
  \rput*[l](3.3,.7){\small RK ordre 4 : $\varepsilon<10^{-3}$}
  \rput*(3.3,.6){\psline[linecolor=red](-.75cm,0)}
  \rput*[l](3.3,.6){\small RK ordre 4 : $\varepsilon<10^{-4}$}
  \rput*(3.3,.5){\psline[linecolor=green](-.75cm,0)}
  \rput*[l](3.3,.5){\small solution exacte}
\end{pspicture}
\end{lstlisting}



\begin{center}
\bgroup
\def\Funct{exch neg}
\psset{xunit=1.5, yunit=5, method=varrkiv, showpoints=true}%%
\def\quatrepi{12.5663706144}
\begin{pspicture}(0,-1)(10,1.3)
  \psaxes{->}(0,0)(0,-1)(10,1.3)
  \psplot[linewidth=4\pslinewidth, linecolor=green, algebraic=true]{0}{10}{cos(x)}
  \rput(0,.0){\psplotDiffEqn[linecolor=magenta, plotpoints=7, varsteptol=.1]{0}{10}{1 0}{\Funct}}
  \rput(0,.0){\psplotDiffEqn[linecolor=blue, plotpoints=201, varsteptol=.01]{0}{10}{1 0}{\Funct}}
  \rput(0,.1){\psplotDiffEqn[linewidth=2\pslinewidth, linecolor=red, varsteptol=.001]{0}{10}{1 0}{\Funct}}
  \rput(0,.2){\psplotDiffEqn[linecolor=black, varsteptol=.0001]{0}{10}{1 0}{\Funct}}
  \rput(0,.3){\psplotDiffEqn[linecolor=Orange, varsteptol=.00001]{0}{10}{1 0}{\Funct}}
  \psset{linewidth=4\pslinewidth,showpoints=false}
  \rput*(2.3,.9){\psline[linecolor=magenta](-.75cm,0)}
  \rput*[l](2.3,.9){\small $\varepsilon<10^{-1}$}
  \rput*(2.3,.8){\psline[linecolor=blue](-.75cm,0)}
  \rput*[l](2.3,.8){\small $\varepsilon<10^{-2}$}
  \rput*(2.3,.7){\psline[linecolor=red](-.75cm,0)}
  \rput*[l](2.3,.7){\small $\varepsilon<10^{-3}$}
  \rput*(2.3,.6){\psline[linecolor=black](-.75cm,0)}
  \rput*[l](2.3,.6){\small $\varepsilon<10^{-4}$}
  \rput*(2.3,.5){\psline[linecolor=Orange](-.75cm,0)}
  \rput*[l](2.3,.5){\small $\varepsilon<10^{-5}$}
  \rput*(2.3,.4){\psline[linecolor=green](-.75cm,0)}
  \rput*[l](2.3,.4){\small solution exacte}
\end{pspicture}
{\captionof{figure}{Equation $y''=-y$}\label{fig:trigfunc}}
\egroup
\end{center}

\begin{lstlisting}[wide=true]
\def\Funct{exch neg}
\psset{xunit=1.5, yunit=5, method=varrkiv, showpoints=true}%%
\def\quatrepi{12.5663706144}
\begin{pspicture}(0,-1)(10,1.3)
  \psaxes{->}(0,0)(0,-1)(10,1.3)
  \psplot[linewidth=4\pslinewidth, linecolor=green, algebraic=true]{0}{10}{cos(x)}
  \rput(0,.0){\psplotDiffEqn[linecolor=magenta, plotpoints=7, varsteptol=.1]{0}{10}{1 0}{\Funct}}
  \rput(0,.0){\psplotDiffEqn[linecolor=blue, plotpoints=201, varsteptol=.01]{0}{10}{1 0}{\Funct}}
  \rput(0,.1){\psplotDiffEqn[linewidth=2\pslinewidth, linecolor=red, varsteptol=.001]{0}{10}{1 0}{\Funct}}
  \rput(0,.2){\psplotDiffEqn[linecolor=black, varsteptol=.0001]{0}{10}{1 0}{\Funct}}
  \rput(0,.3){\psplotDiffEqn[linecolor=Orange, varsteptol=.00001]{0}{10}{1 0}{\Funct}}
  \psset{linewidth=4\pslinewidth,showpoints=false}
  \rput*(2.3,.9){\psline[linecolor=magenta](-.75cm,0)}
  \rput*[l](2.3,.9){\small $\varepsilon<10^{-1}$}
  \rput*(2.3,.8){\psline[linecolor=blue](-.75cm,0)}
  \rput*[l](2.3,.8){\small $\varepsilon<10^{-2}$}
  \rput*(2.3,.7){\psline[linecolor=red](-.75cm,0)}
  \rput*[l](2.3,.7){\small $\varepsilon<10^{-3}$}
  \rput*(2.3,.6){\psline[linecolor=black](-.75cm,0)}
  \rput*[l](2.3,.6){\small $\varepsilon<10^{-4}$}
  \rput*(2.3,.5){\psline[linecolor=Orange](-.75cm,0)}
  \rput*[l](2.3,.5){\small $\varepsilon<10^{-5}$}
  \rput*(2.3,.4){\psline[linecolor=green](-.75cm,0)}
  \rput*[l](2.3,.4){\small solution exacte}
\end{pspicture}
\end{lstlisting}




\begin{center}
\bgroup
\def\Funct{exch}
\psset{xunit=4, yunit=1, method=varrkiv, showpoints=true}%%
\def\quatrepi{12.5663706144}
\begin{pspicture}(0,-0.5)(3,11)
  \psaxes{->}(0,0)(3,11)
  \psplot[linewidth=4\pslinewidth, linecolor=green, algebraic=true]{0}{3}{ch(x)}
  \rput(0,.0){\psplotDiffEqn[linecolor=magenta, varsteptol=.1]{0}{3}{1 0}{\Funct}}
  \rput(0,.3){\psplotDiffEqn[linecolor=blue, varsteptol=.01]{0}{3}{1 0}{\Funct}}
  \rput(0,.6){\psplotDiffEqn[linecolor=red, varsteptol=.001]{0}{3}{1 0}{\Funct}}
  \rput(0,.9){\psplotDiffEqn[linecolor=black, varsteptol=.0001]{0}{3}{1 0}{\Funct}}
  \rput(0,1.2){\psplotDiffEqn[linecolor=Orange, varsteptol=.00001]{0}{3}{1 0}{\Funct}}
  \psset{linewidth=4\pslinewidth,showpoints=false}
  \rput*(2.3,.9){\psline[linecolor=magenta](-.75cm,0)}
  \rput*[l](2.3,.9){\small $\varepsilon<10^{-1}$}
  \rput*(2.3,.8){\psline[linecolor=blue](-.75cm,0)}
  \rput*[l](2.3,.8){\small $\varepsilon<10^{-2}$}
  \rput*(2.3,.7){\psline[linecolor=red](-.75cm,0)}
  \rput*[l](2.3,.7){\small $\varepsilon<10^{-3}$}
  \rput*(2.3,.6){\psline[linecolor=black](-.75cm,0)}
  \rput*[l](2.3,.6){\small $\varepsilon<10^{-4}$}
  \rput*(2.3,.5){\psline[linecolor=Orange](-.75cm,0)}
  \rput*[l](2.3,.5){\small $\varepsilon<10^{-5}$}
  \rput*(2.3,.4){\psline[linecolor=green](-.75cm,0)}
  \rput*[l](2.3,.4){\small solution exacte}
\end{pspicture}
\captionof{figure}{Equation $y''=y$}
\egroup
\end{center}

\begin{lstlisting}[wide=true]
\def\Funct{exch}
\psset{xunit=4, yunit=1, method=varrkiv, showpoints=true}%%
\def\quatrepi{12.5663706144}
\begin{pspicture}(0,-0.5)(3,11)
  \psaxes{->}(0,0)(3,11)
  \psplot[linewidth=4\pslinewidth, linecolor=green, algebraic=true]{0}{3}{ch(x)}
  \rput(0,.0){\psplotDiffEqn[linecolor=magenta, varsteptol=.1]{0}{3}{1 0}{\Funct}}
  \rput(0,.3){\psplotDiffEqn[linecolor=blue, varsteptol=.01]{0}{3}{1 0}{\Funct}}
  \rput(0,.6){\psplotDiffEqn[linecolor=red, varsteptol=.001]{0}{3}{1 0}{\Funct}}
  \rput(0,.9){\psplotDiffEqn[linecolor=black, varsteptol=.0001]{0}{3}{1 0}{\Funct}}
  \rput(0,1.2){\psplotDiffEqn[linecolor=Orange, varsteptol=.00001]{0}{3}{1 0}{\Funct}}
  \psset{linewidth=4\pslinewidth,showpoints=false}
  \rput*(2.3,.9){\psline[linecolor=magenta](-.75cm,0)}
  \rput*[l](2.3,.9){\small $\varepsilon<10^{-1}$}
  \rput*(2.3,.8){\psline[linecolor=blue](-.75cm,0)}
  \rput*[l](2.3,.8){\small $\varepsilon<10^{-2}$}
  \rput*(2.3,.7){\psline[linecolor=red](-.75cm,0)}
  \rput*[l](2.3,.7){\small $\varepsilon<10^{-3}$}
  \rput*(2.3,.6){\psline[linecolor=black](-.75cm,0)}
  \rput*[l](2.3,.6){\small $\varepsilon<10^{-4}$}
  \rput*(2.3,.5){\psline[linecolor=Orange](-.75cm,0)}
  \rput*[l](2.3,.5){\small $\varepsilon<10^{-5}$}
  \rput*(2.3,.4){\psline[linecolor=green](-.75cm,0)}
  \rput*[l](2.3,.4){\small solution exacte}
\end{pspicture}
\end{lstlisting}




\clearpage
\subsection{Equation of second order}

Here is the traditional simulation of two stars attracting each
other according to the classical gravitation law in
$\displaystyle\frac{1}{r^2}$. In 2-Dimensions, the system to be
solved is composed of four second order differential equations. In
order to be described, each of them gives two first order
equations, then we obtain a 8 sized vectorial equation. In the
following example the masses of the stars are 1 and 20.

\[
\left\{
\begin{array}[m]{l}
  x''_1=\displaystyle\frac{M_2}{r^2}\cos(\theta)\\
  y''_1=\displaystyle\frac{M_2}{r^2}\sin(\theta)\\
  x''_2=\displaystyle\frac{M_1}{r^2}\cos(\theta)\\
  y''_2=\displaystyle\frac{M_1}{r^2}\sin(\theta)\\
\end{array}
\right.
\mbox{ avec }
\left\{
\begin{array}[m]{l}
  r^2=(x_1-x_2)^2+(y_1-y_2)^2\\
  \cos(\theta)=\displaystyle\frac{(x_1-x_2)}{r}\\
  \sin(\theta)=\displaystyle\frac{(y_1-y_2)}{r}\\
\end{array}
\right.
\mbox{%
\begin{pspicture}[shift=-2](5,4)\psset{arrowscale=2}
  \psframe[linewidth=.75\pslinewidth](5,4)
  \pstGeonode[PosAngle={-90,90}](1,1){M_1}(4,3){M_2}
  \pstHomO[HomCoef=.33, PointSymbol=none]{M_1}{M_2}[F_1]
  \psline[arrows=->](M_1)(F_1)
  \pstHomO[HomCoef=.33, PointSymbol=none]{M_2}{M_1}[F_2]
  \psline[arrows=->, arrowscale=2](M_2)(F_2)
  \pstGeonode[PointSymbol=none, PointName=none](M_2|M_1){A}
  \psline[linewidth=.5\pslinewidth](M_1)(A)
  \pstMarkAngle{A}{M_1}{M_2}{$\theta$}
  \ncline[linewidth=.5\pslinewidth, offset=.5, arrows=<->]{M_1}{M_2}
  \ncput*{$r$}
\end{pspicture}}
\]

\begin{table}[!htbp]
  \centering\small
    \begin{tabular}{|l@{}>{\ttfamily}l@{}>{ \ttfamily \%\% }l|}
      \hline
      && x1 y1 x'1 y'1 x2 y2 x'2 y'2\\
      &/yp2 exch def /xp2 exch def /ay2 exch def /ax2 exch def&mise en variables\\
      &/yp1 exch def /xp1 exch def /ay1 exch def /ax1 exch def&mise en variables\\
      &/ro2 ax2 ax1 sub dup mul ay2 ay1 sub dup mul add def&calcul de r*r\\
      &xp1 yp1&\\
      &ax2 ax1 sub ro2 sqrt div ro2 div&calcul de x''1\\
      &ay2 ay1 sub ro2 sqrt div ro2 div&calcul de y''1\\
      &xp2 yp2&\\
      &3 index -20 mul&calcul de x''2=-20x''1\\
      &3 index -20 mul&calcul de y''2=-20y''1\\
      \hline
    \end{tabular}
    \caption{\PS source code for the gravitational interaction}\label{intgravcode}
\end{table}

\begin{table}[!htbp]
  \centering
    \small\newcommand{\POW}{\symbol{'136}}
    \begin{tabular}{|l@{}>{\ttfamily}l@{}>{ \ttfamily \%\% }l|}
      \hline
      &y[2]|&y'[0]\\
      &y[3]|&y'[1]\\
      &(y[4]-y[0])/((y[4]-y[0])\POW 2+(y[5]-y[1])\POW 2)\POW 1.5|&y'[2]=y''[0]\\
      &(y[5]-y[1])/((y[4]-y[0])\POW 2+(y[5]-y[1])\POW 2)\POW 1.5|&y'[3]=y''[1]\\
      &y[6]|&y'[4]\\
      &y[7]|&y'[5]\\
      &20*(y[0]-y[4])/((y[4]-y[0])\POW 2+(y[5]-y[1])\POW 2)\POW 1.5|&y'[6]=y''[4]\\
      &20*(y[1]-y[5])/((y[4]-y[0])\POW 2+(y[5]-y[1])\POW 2)\POW 1.5&y'[7]=y''[5]\\
      \hline
    \end{tabular}
    \caption{Algebraic description for the gravitational interaction}\label{intgravalgcode}
\end{table}

\newcommand\Grav{%
  /yp2 exch def /xp2 exch def /ay2 exch def /ax2 exch def
  /yp1 exch def /xp1 exch def /ay1 exch def /ax1 exch def
  /ro2 ax2 ax1 sub dup mul ay2 ay1 sub dup mul add def
  xp1 yp1
  ax2 ax1 sub ro2 sqrt div ro2 div
  ay2 ay1 sub ro2 sqrt div ro2 div
  xp2 yp2
  3 index -20 mul
  3 index -20 mul}
\newcommand\GravAlg{%
  y[2]|y[3]|%
  (y[4]-y[0])/((y[4]-y[0])^2+(y[5]-y[1])^2)^1.5|%
  (y[5]-y[1])/((y[4]-y[0])^2+(y[5]-y[1])^2)^1.5|%
  y[6]|y[7]|%
  20*(y[0]-y[4])/((y[4]-y[0])^2+(y[5]-y[1])^2)^1.5|%
  20*(y[1]-y[5])/((y[4]-y[0])^2+(y[5]-y[1])^2)^1.5}
%%  0  1   2   3  4  5   6   7
%% x1 y1 x'1 y'1 x2 y2 x'2 y'2


\begin{LTXexample}[width=5cm,wide]
\def\InitCond{ 1  1  .1  0 -1 -1  -2   0}
\begin{pspicture}[shift=-2,showgrid=true](-3,-1.75)(2,1.5)
  \psplotDiffEqn[whichabs=0, whichord=1, linecolor=blue, method=rk4, plotpoints=100]{0}{3.95}{\InitCond}{\Grav}
  \psset{showpoints=true,whichabs=4, whichord=5}
  \psplotDiffEqn[linecolor=black, method=varrkiv, varsteptol=.0001, plotpoints=200]{0}{3.9}{\InitCond}{\Grav}
\end{pspicture}
\end{LTXexample}
\vspace{-2ex}
{\captionof{figure}{Gravitational interaction: fixed landmark, trajectory of the stars}\label{fig:InterGravRepFix}}



\bigskip
\begin{LTXexample}[width=5cm,wide]
\def\InitCond{ 1  1  .1  0 -1 -1  -2   0}
\begin{pspicture}[shift=-1.5,showgrid=true](-4,-1.75)(1,1)
  \psplotDiffEqn[linecolor=red, plotpoints=200,method=varrkiv, varsteptol=.0001, showpoints=true,
      plotfuncx=y dup 4 get exch 0 get sub,
      plotfuncy=dup 5 get exch 1 get sub ]{0}{3.9}{\InitCond}{\Grav}
\end{pspicture}
\end{LTXexample}
\vspace{-2ex}
{\captionof{figure}{Gravitational interaction : landmark defined by one star}\label{fig:IGnewrep}}


\begin{center}
\bgroup
\def\InitCond{ 1  1  .1   0 -1 -1  -2   0}
\psset{xunit=2}
\begin{pspicture}[showgrid=true](0,0)(8,9)
  \psset{showpoints=true}
  \psplotDiffEqn[linecolor=red, method=varrkiv, plotpoints=2, varsteptol=.0001,
      plotfuncy=dup 6 get dup mul exch 7 get dup mul add sqrt]{0}{8}{\InitCond}{\Grav}
  \psplotDiffEqn[linecolor=blue, method=varrkiv, plotpoints=2, varsteptol=.0001,
      plotfuncy=dup 2 get dup mul exch 3 get dup mul add sqrt]{0}{8}{\InitCond}{\Grav}
\end{pspicture}
\captionof{figure}{Gravitational interaction : speeds of the
stars} \egroup
\end{center}

\begin{lstlisting}
\psset{xunit=2}
\begin{pspicture}[showgrid=true](0,0)(8,9)
  \psset{showpoints=true}
  \psplotDiffEqn[linecolor=red, method=varrkiv, plotpoints=2, varsteptol=.0001,
      plotfuncy=dup 6 get dup mul exch 7 get dup mul add sqrt]{0}{8}{\InitCond}{\Grav}
  \psplotDiffEqn[linecolor=blue, method=varrkiv, plotpoints=2, varsteptol=.0001,
      plotfuncy=dup 2 get dup mul exch 3 get dup mul add sqrt]{0}{8}{\InitCond}{\Grav}
\end{pspicture}
\end{lstlisting}

%--------------------------------------------------------------------------------------
\clearpage
\subsubsection{Simple equation of first order $y'=y$}
%--------------------------------------------------------------------------------------

For the initial value $y(0)=1$ we have the solution $y(x)=e^x$. $y$ is always
on the stack, so we have to do nothing. Using the \Lkeyword{algebraic=true} option, we write it
as \verb$y[0]$. The following example shows different solutions depending to the number of plotpoints
with $y_0=1$:


\begin{center}
\bgroup
\psset{xunit=4, yunit=.4}
\begin{pspicture}(3,19)\psgrid[subgriddiv=1]
  \psplot[linewidth=6\pslinewidth, linecolor=green]{0}{3}{Euler x exp}
  \psplotDiffEqn[linecolor=magenta,plotpoints=16,algebraic=true]{0}{3}{1}{y[0]}
  \psplotDiffEqn[linecolor=blue,plotpoints=151]{0}{3}{1}{}
  \psplotDiffEqn[linecolor=red,method=rk4,plotpoints=15]{0}{3}{1}{}
  \psplotDiffEqn[linecolor=Orange,method=rk4,plotpoints=4]{0}{3}{1}{}
  \psset{linewidth=4\pslinewidth}
  \rput*(0.35,19){\psline[linecolor=magenta](-.75cm,0)}
  \rput*[l](0.35,19){\small Euler order 1 $h=0{,}2$}
  \rput*(0.35,17){\psline[linecolor=blue](-.75cm,0)}
  \rput*[l](0.35,17){\small Euler order 1 $h=0{,}02$}
  \rput*(0.35,15){\psline[linecolor=Orange](-.75cm,0)}
  \rput*[l](0.35,15){\small RK ordre 4 $h=1$}
  \rput*(0.35,13){\psline[linecolor=red](-.75cm,0)}
  \rput*[l](0.35,13){\small RK ordre 4 $h=0{,}2$}
  \rput*(0.35,11){\psline[linecolor=green](-.75cm,0)}
  \rput*[l](0.35,11){\small solution exacte}
\end{pspicture}
\egroup
\end{center}

\begin{lstlisting}
\psset{xunit=4, yunit=.4}
\begin{pspicture}(3,19)\psgrid[subgriddiv=1]
  \psplot[linewidth=6\pslinewidth, linecolor=green]{0}{3}{Euler x exp}
  \psplotDiffEqn[linecolor=magenta,plotpoints=16,algebraic=true]{0}{3}{1}{y[0]}
  \psplotDiffEqn[linecolor=blue,plotpoints=151]{0}{3}{1}{}
  \psplotDiffEqn[linecolor=red,method=rk4,plotpoints=15]{0}{3}{1}{}
  \psplotDiffEqn[linecolor=Orange,method=rk4,plotpoints=4]{0}{3}{1}{}
  \psset{linewidth=4\pslinewidth}
  \rput*(0.35,19){\psline[linecolor=magenta](-.75cm,0)}
  \rput*[l](0.35,19){\small Euler order 1 $h=0{,}2$}
  \rput*(0.35,17){\psline[linecolor=blue](-.75cm,0)}
  \rput*[l](0.35,17){\small Euler order 1 $h=0{,}02$}
  \rput*(0.35,15){\psline[linecolor=Orange](-.75cm,0)}
  \rput*[l](0.35,15){\small RK ordre 4 $h=1$}
  \rput*(0.35,13){\psline[linecolor=red](-.75cm,0)}
  \rput*[l](0.35,13){\small RK ordre 4 $h=0{,}2$}
  \rput*(0.35,11){\psline[linecolor=green](-.75cm,0)}
  \rput*[l](0.35,11){\small solution exacte}
\end{pspicture}
\end{lstlisting}

%--------------------------------------------------------------------------------------
\clearpage
\subsubsection{$y'=\displaystyle\frac{2-ty}{4-t^2}$}% $
%--------------------------------------------------------------------------------------

For the initial value $y(0)=1$ the exact solution is
$y(x)=\displaystyle\frac{t+\sqrt{4-t^2}}{2}$. The function $f$
described in PostScript code is like (y is still on the stack):
\begin{lstlisting}[style=syntax]
x              %% y x
mul            %% x*y
2 exch sub     %% 2-x*y
4 x dup mul    %% 2-x*y 4 x^2
sub            %% 2-x*y 4-x^2
div            %% (2-x*y)/(4-x^2)
\end{lstlisting}
\noindent
The following example uses $y_0=1$.

\begin{lstlisting}[style=syntax]
\newcommand{\InitCond}{1}
\newcommand{\Func}{x mul 2 exch sub 4 x dup mul sub div}
\newcommand{\FuncAlg}{(2-x*y[0])/(4-x^2)}
\end{lstlisting}

\begin{center}
\bgroup
\psset{xunit=6.4, yunit=9.6, showpoints=false}
\begin{pspicture}(0,1)(2,1.5)  \psgrid[griddots=10](0,1)(2,1.5)
  { \psset{linewidth=4\pslinewidth,linecolor=lightgray}
  \psplot{0}{1.8}{x dup dup mul 4 exch sub sqrt add 2 div}
  \psplot{1.8}{2}{x dup dup mul 4 exch sub sqrt add 2 div} }
  \def\InitCond{1}
  \def\Func{x mul 2 exch sub 4 x dup mul sub div}
  \psplotDiffEqn[linecolor=magenta, plotpoints=20]{0}{1.9}{\InitCond}{\Func}
  \psplotDiffEqn[linecolor=blue, plotpoints=191]{0}{1.9}{\InitCond}{\Func}
  \psplotDiffEqn[linecolor=red, method=rk4, plotpoints=11,%
     algebraic=true]{0}{1.9}{\InitCond}{(2-x*y[0])/(4-x^2)}
  \psplotDiffEqn[linecolor=Orange, method=rk4, plotpoints=21,%
     algebraic=true]{0}{1.9}{\InitCond}{(2-x*y[0])/(4-x^2)}
  \psset{linewidth=4\pslinewidth}\small
  \rput*(0,1.4){\psline[linecolor=magenta](-.75cm,0)}\rput*[l](0,1.4){Euler order 1 $h=0{,}1$}
  \rput*(0,1.35){\psline[linecolor=blue](-.75cm,0)}\rput*[l](0,1.35){Euler order 1 $h=0{,}01$}
  \rput*(0,1.3){\psline[linecolor=Orange](-.75cm,0)}\rput*[l](0,1.3){RK order 4 $h=0{,}19$}
  \rput*(0,1.25){\psline[linecolor=red](-.75cm,0)}\rput*[l](0,1.25){RK order 4 $h=0{,}095$}
  \rput*(0,1.2){\psline[linecolor=lightgray](-.75cm,0)}\rput*[l](0,1.2){exactly}
\end{pspicture}
\egroup
\end{center}

\begin{lstlisting}[xrightmargin=-1cm,xleftmargin=-1cm]
\psset{xunit=6.4, yunit=9.6, showpoints=false}
\begin{pspicture}(0,1)(2,1.7)  \psgrid[subgriddiv=5]
  { \psset{linewidth=4\pslinewidth,linecolor=lightgray}
  \psplot{0}{1.8}{x dup dup mul 4 exch sub sqrt add 2 div}
  \psplot{1.8}{2}{x dup dup mul 4 exch sub sqrt add 2 div} }
  \def\InitCond{1}
  \def\Func{x mul 2 exch sub 4 x dup mul sub div}
  \psplotDiffEqn[linecolor=magenta, plotpoints=20]{0}{1.9}{\InitCond}{\Func}
  \psplotDiffEqn[linecolor=blue, plotpoints=191]{0}{1.9}{\InitCond}{\Func}
  \psplotDiffEqn[linecolor=red, method=rk4, plotpoints=11,%
     algebraic=true]{0}{1.9}{\InitCond}{(2-x*y[0])/(4-x^2)}
  \psplotDiffEqn[linecolor=Orange, method=rk4, plotpoints=21,%
     algebraic=true]{0}{1.9}{\InitCond}{(2-x*y[0])/(4-x^2)}
  \psset{linewidth=4\pslinewidth}
  \rput*(0.3,1.6){\psline[linecolor=magenta](-.75cm,0)}\rput*[l](0.3,1.6){\small Euler order 1 $h=0{,}1$}
  \rput*(0.3,1.55){\psline[linecolor=blue](-.75cm,0)}\rput*[l](0.3,1.55){\small Euler order 1 $h=0{,}01$}
  \rput*(0.3,1.5){\psline[linecolor=Orange](-.75cm,0)}\rput*[l](0.3,1.5){\small RK order 4 $h=0{,}19$}
  \rput*(0.3,1.45){\psline[linecolor=red](-.75cm,0)}\rput*[l](0.3,1.45){\small RK order 4 $h=0{,}095$}
  \rput*(0.3,1.4){\psline[linecolor=lightgray](-.75cm,0)}\rput*[l](0.3,1.4){\small exactly}
\end{pspicture}
\end{lstlisting}


%--------------------------------------------------------------------------------------
\clearpage
\subsubsection{$y'=-2xy$}
%--------------------------------------------------------------------------------------

For $y(-1)=\frac{1}{e}$ we get $y(x)=e^{-x^2}$.

\begin{center}
\bgroup
\psset{unit=4}
\begin{pspicture}(-1,0)(3,1.1)\psgrid
  \psplot[linewidth=4\pslinewidth,linecolor=gray]{-1}{3}{Euler x dup mul neg exp}
  \psset{plotpoints=9}
  \psplotDiffEqn[linecolor=cyan]{-1}{3}{1 Euler div}{x -2 mul mul}
  \psplotDiffEqn[linecolor=yellow, method=rk4]{-1}{3}{1 Euler div}{x -2 mul mul}
  \psset{plotpoints=21}
  \psplotDiffEqn[linecolor=blue]{-1}{3}{1 Euler div}{x -2 mul mul}
  \psplotDiffEqn[linecolor=Orange, method=rk4]{-1}{3}{1 Euler div}{x -2 mul mul}
  \psset{linewidth=2\pslinewidth}
  \rput*(2,1){\psline[linecolor=Orange](-0.25,0)}
  \rput*[l](2,1){RK}
  \rput*(2,.9){\psline[linecolor=blue](-0.25,0)}
  \rput*[l](2,.9){\textsc{Euler}-1}
  \rput*(2,.8){\psline[linecolor=gray](-0.25,0)}
  \rput*[l](2,.8){solution}
\end{pspicture}
\egroup
\end{center}


\begin{lstlisting}
\psset{unit=4}
\begin{pspicture}(-1,0)(3,1.1)\psgrid
  \psplot[linewidth=4\pslinewidth,linecolor=gray]{-1}{3}{Euler x dup mul neg exp}
  \psset{plotpoints=9}
  \psplotDiffEqn[linecolor=cyan]{-1}{3}{1 Euler div}{x -2 mul mul}
  \psplotDiffEqn[linecolor=yellow, method=rk4]{-1}{3}{1 Euler div}{x -2 mul mul}
  \psset{plotpoints=21}
  \psplotDiffEqn[linecolor=blue]{-1}{3}{1 Euler div}{x -2 mul mul}
  \psplotDiffEqn[linecolor=Orange, method=rk4]{-1}{3}{1 Euler div}{x -2 mul mul}
  \psset{linewidth=2\pslinewidth}
  \rput*(2,1){\psline[linecolor=Orange](-0.25,0)}
  \rput*[l](2,1){RK}
  \rput*(2,.9){\psline[linecolor=blue](-0.25,0)}
  \rput*[l](2,.9){\textsc{Euler}-1}
  \rput*(2,.8){\psline[linecolor=gray](-0.25,0)}
  \rput*[l](2,.8){solution}
\end{pspicture}
\end{lstlisting}


%--------------------------------------------------------------------------------------
\clearpage
\subsubsection{Spiral of Cornu}
%--------------------------------------------------------------------------------------

The integrals of \Index{Fresnel}:
\begin{align} x & =\int^t_0\cos\frac{\pi t^2}{2}\mathrm{d}t \\
 y & =\int^t_0\sin\frac{\pi t^2}{2}\mathrm{d}t \\
\intertext{with}
 \dot{x} &= \cos\frac{\pi t^2}{2} \\
 \dot{y} & =\sin\frac{\pi t^2}{2}
 \end{align}

\begin{lstlisting}
\psset{unit=8}
\begin{pspicture}(1,1)\psgrid[subgriddiv=5]
  \psplotDiffEqn[whichabs=0,whichord=1,linecolor=red,method=rk4,algebraic=true,%
     plotpoints=500,showpoints=true]{0}{10}{0 0}{cos(Pi*x^2/2)|sin(Pi*x^2/2)}
\end{pspicture}
\end{lstlisting}


\begin{center}
\bgroup
\psset{unit=8}
\begin{pspicture}(1,1)\psgrid[subgriddiv=5]
  \psplotDiffEqn[whichabs=0,whichord=1,linecolor=red,method=rk4,algebraic=true,%
     plotpoints=500,showpoints=true]{0}{10}{0 0}{cos(Pi*x^2/2)|sin(Pi*x^2/2)}
\end{pspicture}
\egroup
\end{center}



%--------------------------------------------------------------------------------------
\clearpage
\subsubsection{Lotka-Volterra}
%--------------------------------------------------------------------------------------

The Lotka-Volterra model describes interactions between two species in an ecosystem, a 
predator and a prey. This represents our first multi-species model. Since we are considering 
two species, the model will involve two equations, one which describes how the prey 
population changes and the second which describes how the predator population changes.

For concreteness let us assume that the prey in our model are rabbits, and that the 
predators are foxes. If we let $R(t)$ and $F(t)$ represent the number of rabbits and 
foxes, respectively, that are alive at time t, then the Lotka-Volterra model is:
%
\begin{align}
\dot R &= a\cdot R - b\cdot R\cdot F\\
\dot F &= e\cdot b\cdot R\cdot F - c\cdot F
\end{align}
%
where the parameters are defined by:
\begin{description}
\item[a] is the natural growth rate of rabbits in the absence of predation,
\item[c] is the natural death rate of foxes in the absence of food (rabbits),
\item[b] is the death rate per encounter of rabbits due to predation,
\item[e] is the efficiency of turning predated rabbits into foxes.
\end{description}

The Stella model representing the \Index{Lotka-Volterra} model will be slightly more complex than the 
single species models we've dealt with before. The main difference is that our model will have 
two stocks (reservoirs), one for each species. Each species will have its own birth and death 
rates. In addition, the Lotka-Volterra model involves four parameters rather than two. All told, 
the Stella representation of the Lotka-Volterra model will use two stocks, four flows, four 
converters and many connectors.

\bgroup
\begin{center}
\def\InitCond{ 0 10 10}%% xa ya xl
\def\Faiglelapin{\Vaigle*(y[2]-y[0])/sqrt(y[1]^2+(y[2]-y[0])^2)|%
                 -\Vaigle*y[1]/sqrt(y[1]^2+(y[2]-y[0])^2)|%
                 -\Vlapin}
\def\Vlapin{1}  \def\Vaigle{1.6}
\psset{unit=.7,subgriddiv=0,gridcolor=lightgray,method=adams,algebraic=true,%
   plotpoints=20,showpoints=true}
\begin{pspicture}[showgrid=true](-3,-3)(10,10)
 \psplotDiffEqn[plotfuncy=pop 0,whichabs=2,linecolor=red]{0}{10}{\InitCond}{\Faiglelapin}
 \psplotDiffEqn[whichabs=0,whichord=1,linecolor=black,method=rk4]{0}{10}{\InitCond}{\Faiglelapin}
  \psplotDiffEqn[whichabs=0,whichord=1,linecolor=blue]{0}{10}{\InitCond}{\Faiglelapin}
\end{pspicture}
\end{center}

\begin{lstlisting}[label={fig:aiglelapin},xrightmargin=-1.5cm]
\def\InitCond{ 0 10 10}%% xa ya xl
\def\Faiglelapin{\Vaigle*(y[2]-y[0])/sqrt(y[1]^2+(y[2]-y[0])^2)|%
                 -\Vaigle*y[1]/sqrt(y[1]^2+(y[2]-y[0])^2)|%
                 -\Vlapin}
\def\Vlapin{1}  \def\Vaigle{1.6}
\psset{unit=.7,subgriddiv=0,gridcolor=lightgray,method=adams,algebraic=true,%
   plotpoints=20,showpoints=true}
\begin{pspicture}[showgrid=true](-3,-3)(10,10)
 \psplotDiffEqn[plotfuncy=pop 0,whichabs=2,linecolor=red]{0}{10}{\InitCond}{\Faiglelapin}
 \psplotDiffEqn[whichabs=0,whichord=1,linecolor=black,method=rk4]{0}{10}{\InitCond}{\Faiglelapin}
  \psplotDiffEqn[whichabs=0,whichord=1,linecolor=blue]{0}{10}{\InitCond}{\Faiglelapin}
\end{pspicture}
\end{lstlisting}


\begin{center}
\def\InitCond{ 0 10 10}%% xa ya xl
\def\Faiglelapin{\Vaigle*(y[2]-y[0])/sqrt(y[1]^2+(y[2]-y[0])^2)|%
                 -\Vaigle*y[1]/sqrt(y[1]^2+(y[2]-y[0])^2)|%
                 -\Vlapin}
\def\Vlapin{1}  \def\Vaigle{1.6}
\psset{unit=.7,subgriddiv=0,gridcolor=lightgray,method=adams,algebraic=true,%
   plotpoints=20,showpoints=true}
\begin{pspicture}[showgrid=true](0,-0.25)(10,14)
 \psplotDiffEqn[plotfuncy=dup 1 get dup mul exch dup 0 get exch 2 get sub dup
    mul add sqrt,linecolor=red,method=rk4]{0}{10}{\InitCond}{\Faiglelapin}
 \psplotDiffEqn[plotfuncy=dup 1 get dup mul exch dup 0 get exch 2 get sub dup
    mul add sqrt,linecolor=blue]{0}{10}{\InitCond}{\Faiglelapin}
 \psplotDiffEqn[plotfuncy=pop Func aload pop pop dup mul exch dup mul add sqrt,
    linecolor=yellow]{0}{10}{\InitCond}{\Faiglelapin}
\end{pspicture}
\end{center}
\egroup

\begin{lstlisting}[label={fig:aiglelapin},xrightmargin=-1.5cm]
\def\InitCond{ 0 10 10}%% xa ya xl
\def\Faiglelapin{\Vaigle*(y[2]-y[0])/sqrt(y[1]^2+(y[2]-y[0])^2)|%
                 -\Vaigle*y[1]/sqrt(y[1]^2+(y[2]-y[0])^2)|%
                 -\Vlapin}
\def\Vlapin{1}  \def\Vaigle{1.6}
\psset{unit=.7,subgriddiv=0,gridcolor=lightgray,method=adams,algebraic=true,%
   plotpoints=20,showpoints=true}
\begin{pspicture}[showgrid=true](10,12)
 \psplotDiffEqn[plotfuncy=dup 1 get dup mul exch dup 0 get exch 2 get sub dup
    mul add sqrt,linecolor=red,method=rk4]{0}{10}{\InitCond}{\Faiglelapin}
 \psplotDiffEqn[plotfuncy=dup 1 get dup mul exch dup 0 get exch 2 get sub dup
    mul add sqrt,linecolor=blue]{0}{10}{\InitCond}{\Faiglelapin}
 \psplotDiffEqn[plotfuncy=pop Func aload pop pop dup mul exch dup mul add sqrt,
    linecolor=yellow]{0}{10}{\InitCond}{\Faiglelapin}
\end{pspicture}
\end{lstlisting}


%--------------------------------------------------------------------------------------
\subsubsection{$y''=y$}
%--------------------------------------------------------------------------------------

Beginning with the initial equation $\displaystyle y(x)=Ae^x+Be^{-x}$ we get the hyperbolic
trigonometrical functions.

\begin{center}
\bgroup
\def\Funct{exch}   \psset{xunit=5cm, yunit=0.75cm}
\begin{pspicture}(0,-0.25)(2,7)\psgrid[subgriddiv=1,griddots=10]
 \psplot[linewidth=4\pslinewidth, linecolor=green]{0}{2}{Euler x exp}  %%e^x
 \psplotDiffEqn[linecolor=magenta, plotpoints=11]{0}{2}{1 1}{\Funct}
 \psplotDiffEqn[linecolor=blue, plotpoints=101]{0}{2}{1 1}{\Funct}
 \psplotDiffEqn[linecolor=red, method=rk4, plotpoints=11]{0}{2}{1 1}{\Funct}
 \psplot[linewidth=4\pslinewidth, linecolor=green]{0}{2}{Euler dup x exp  %%ch(x)
    exch x neg exp add 2 div}
 \psplotDiffEqn[linecolor=magenta, plotpoints=11]{0}{2}{1 0}{\Funct}
 \psplotDiffEqn[linecolor=blue, plotpoints=101]{0}{2}{1 0}{\Funct}
 \psplotDiffEqn[linecolor=red, method=rk4, plotpoints=11]{0}{2}{1 0}{\Funct}
 \psplot[linewidth=4\pslinewidth, linecolor=green]{0}{2}{Euler dup x exp
     exch x neg exp sub 2 div}  %%sh(x)
 \psplotDiffEqn[linecolor=magenta, plotpoints=11]{0}{2}{0 1}{\Funct}
 \psplotDiffEqn[linecolor=blue, plotpoints=101]{0}{2}{0 1}{\Funct}
 \psplotDiffEqn[linecolor=red, method=rk4, plotpoints=11]{0}{2}{0 1}{\Funct}
 \rput*(1.3,.9){\psline[linecolor=magenta](-.75cm,0)}\rput*[l](1.3,.9){\small\textsc{Euler} order 1 $h=1$}
 \rput*(1.3,.8){\psline[linecolor=blue](-.75cm,0)}\rput*[l](1.3,.8){\small\textsc{Euler} order 1 $h=0{,}1$}
 \rput*(1.3,.7){\psline[linecolor=red](-.75cm,0)}\rput*[l](1.3,.7){\small RK order 4 $h=1$}
 \rput*(1.3,.6){\psline[linecolor=green](-.75cm,0)}\rput*[l](1.3,.6){\small exact solution}
\end{pspicture}
\egroup
\end{center}

\begin{lstlisting}[label={fig:minusexp},xrightmargin=-1.5cm]
\def\Funct{exch}   \psset{xunit=5cm, yunit=0.75cm}
\begin{pspicture}(0,-0.25)(2,7)\psgrid[subgriddiv=1,griddots=10]
 \psplot[linewidth=4\pslinewidth, linecolor=green]{0}{2}{Euler x exp}  %%e^x
 \psplotDiffEqn[linecolor=magenta, plotpoints=11]{0}{2}{1 1}{\Funct}
 \psplotDiffEqn[linecolor=blue, plotpoints=101]{0}{2}{1 1}{\Funct}
 \psplotDiffEqn[linecolor=red, method=rk4, plotpoints=11]{0}{2}{1 1}{\Funct}
 \psplot[linewidth=4\pslinewidth, linecolor=green]{0}{2}{Euler dup x exp  %%ch(x)
    exch x neg exp add 2 div}
 \psplotDiffEqn[linecolor=magenta, plotpoints=11]{0}{2}{1 0}{\Funct}
 \psplotDiffEqn[linecolor=blue, plotpoints=101]{0}{2}{1 0}{\Funct}
 \psplotDiffEqn[linecolor=red, method=rk4, plotpoints=11]{0}{2}{1 0}{\Funct}
 \psplot[linewidth=4\pslinewidth, linecolor=green]{0}{2}{Euler dup x exp
     exch x neg exp sub 2 div}  %%sh(x)
 \psplotDiffEqn[linecolor=magenta, plotpoints=11]{0}{2}{0 1}{\Funct}
 \psplotDiffEqn[linecolor=blue, plotpoints=101]{0}{2}{0 1}{\Funct}
 \psplotDiffEqn[linecolor=red, method=rk4, plotpoints=11]{0}{2}{0 1}{\Funct}
 \rput*(1.3,.9){\psline[linecolor=magenta](-.75cm,0)}\rput*[l](1.3,.9){\small\textsc{Euler} order 1 $h=1$}
 \rput*(1.3,.8){\psline[linecolor=blue](-.75cm,0)}\rput*[l](1.3,.8){\small\textsc{Euler} order 1 $h=0{,}1$}
 \rput*(1.3,.7){\psline[linecolor=red](-.75cm,0)}\rput*[l](1.3,.7){\small RK order 4 $h=1$}
 \rput*(1.3,.6){\psline[linecolor=green](-.75cm,0)}\rput*[l](1.3,.6){\small exact solution}
\end{pspicture}
\end{lstlisting}

%--------------------------------------------------------------------------------------
\clearpage
\subsubsection{$y''=-y$}
%--------------------------------------------------------------------------------------
\begin{center}
\bgroup
\def\Funct{exch neg}
\psset{xunit=1, yunit=4}
\def\quatrepi{12.5663706144}%%4pi=12.5663706144
\begin{pspicture}(0,-1.25)(\quatrepi,1.25)\psgrid[subgriddiv=1,griddots=10]
 \psplot[linewidth=4\pslinewidth,linecolor=green]{0}{\quatrepi}{x RadtoDeg cos}%%cos(x)
 \psplotDiffEqn[linecolor=blue, plotpoints=201]{0}{3.1415926}{1 0}{\Funct}
 \psplotDiffEqn[linecolor=red, method=rk4, plotpoints=31]{0}{\quatrepi}{1 0}{\Funct}
 \psplot[linewidth=4\pslinewidth,linecolor=green]{0}{\quatrepi}{x RadtoDeg sin}  %%sin(x)
 \psplotDiffEqn[linecolor=blue,plotpoints=201]{0}{3.1415926}{0 1}{\Funct}
 \psplotDiffEqn[linecolor=red,method=rk4, plotpoints=31]{0}{\quatrepi}{0 1}{\Funct}
 \rput*(3.3,.9){\psline[linecolor=magenta](-.75cm,0)}\rput*[l](3.3,.9){\small Euler order 1 $h=1$}
 \rput*(3.3,.8){\psline[linecolor=blue](-.75cm,0)}\rput*[l](3.3,.8){\small Euler order 1 $h=0{,}1$}
 \rput*(3.3,.7){\psline[linecolor=red](-.75cm,0)}\rput*[l](3.3,.7){\small RK order 4 $h=1$}
 \rput*(3.3,.6){\psline[linecolor=green](-.75cm,0)}\rput*[l](3.3,.6){\small exact solution}
\end{pspicture}
\egroup
\end{center}

\begin{lstlisting}[label={fig:minusexp2}]
\def\Funct{exch neg}
\psset{xunit=1, yunit=4}
\def\quatrepi{12.5663706144}%%4pi=12.5663706144
\begin{pspicture}(0,-1.25)(\quatrepi,1.25)\psgrid[subgriddiv=1,griddots=10]
 \psplot[linewidth=4\pslinewidth,linecolor=green]{0}{\quatrepi}{x RadtoDeg cos}%%cos(x)
 \psplotDiffEqn[linecolor=blue, plotpoints=201]{0}{3.1415926}{1 0}{\Funct}
 \psplotDiffEqn[linecolor=red, method=rk4, plotpoints=31]{0}{\quatrepi}{1 0}{\Funct}
 \psplot[linewidth=4\pslinewidth,linecolor=green]{0}{\quatrepi}{x RadtoDeg sin}  %%sin(x)
 \psplotDiffEqn[linecolor=blue,plotpoints=201]{0}{3.1415926}{0 1}{\Funct}
 \psplotDiffEqn[linecolor=red,method=rk4, plotpoints=31]{0}{\quatrepi}{0 1}{\Funct}
 \rput*(3.3,.9){\psline[linecolor=magenta](-.75cm,0)}\rput*[l](3.3,.9){\small Euler order 1 $h=1$}
 \rput*(3.3,.8){\psline[linecolor=blue](-.75cm,0)}\rput*[l](3.3,.8){\small Euler order 1 $h=0{,}1$}
 \rput*(3.3,.7){\psline[linecolor=red](-.75cm,0)}\rput*[l](3.3,.7){\small RK order 4 $h=1$}
 \rput*(3.3,.6){\psline[linecolor=green](-.75cm,0)}\rput*[l](3.3,.6){\small exact solution}
\end{pspicture}
\end{lstlisting}

%--------------------------------------------------------------------------------------
\clearpage
\subsubsection{The mechanical pendulum: $y''=-\frac{g}{l}\sin(y)$}% $
%--------------------------------------------------------------------------------------

For small \Index{oscillation}s $\sin(y)\simeq y$:

\[ y(x)=y_0\cos\left(\sqrt{\frac{g}{l}}x\right) \]

The function $f$ is written in PostScript code:

\begin{lstlisting}[style=syntax]
exch RadtoDeg sin -9.8 mul %% y' -gsin(y)
\end{lstlisting}

\begin{center}
\bgroup
\def\Func{y[1]|-9.8*sin(y[0])}
\psset{yunit=2,xunit=4,algebraic=true,linewidth=1.5pt}
\begin{pspicture}(0,-2.25)(3,2.25)
  \psaxes{->}(0,0)(0,-2)(3,2)
  \psplot[linewidth=3\pslinewidth, linecolor=Orange]{0}{3}{.1*cos(sqrt(9.8)*x)}
  \psset{method=rk4,plotpoints=50,linecolor=blue}
  \psplotDiffEqn{0}{3}{.1 0}{\Func}
  \psplot[linewidth=3\pslinewidth,linecolor=Orange]{0}{3}{.25*cos(sqrt(9.8)*x)}
  \psplotDiffEqn{0}{3}{.25 0}{\Func}
  \psplotDiffEqn{0}{3}{.5 0}{\Func}
  \psplotDiffEqn{0}{3}{1 0}{\Func}
  \psplotDiffEqn[plotpoints=100]{0}{3}{Pi 2 div 0}{\Func}
\end{pspicture}
\egroup
\end{center}

\begin{lstlisting}[label=fig:second]
\def\Func{y[1]|-9.8*sin(y[0])}
\psset{yunit=2,xunit=4,algebraic=true,linewidth=1.5pt}
\begin{pspicture}(0,-2.25)(3,2.25)
  \psaxes{->}(0,0)(0,-2)(3,2)
  \psplot[linewidth=3\pslinewidth, linecolor=Orange]{0}{3}{.1*cos(sqrt(9.8)*x)}
  \psset{method=rk4,plotpoints=50,linecolor=blue}
  \psplotDiffEqn{0}{3}{.1 0}{\Func}
  \psplot[linewidth=3\pslinewidth,linecolor=Orange]{0}{3}{.25*cos(sqrt(9.8)*x)}
  \psplotDiffEqn{0}{3}{.25 0}{\Func}
  \psplotDiffEqn{0}{3}{.5 0}{\Func}
  \psplotDiffEqn{0}{3}{1 0}{\Func}
  \psplotDiffEqn[plotpoints=100]{0}{3}{Pi 2 div 0}{\Func}
\end{pspicture}
\end{lstlisting}

%--------------------------------------------------------------------------------------
\clearpage
\subsubsection{$y''=-\frac{y'}{4}-2y$}% $
%--------------------------------------------------------------------------------------

For $y_0=5$ and $y'_0=0$ the solution is:

\[
5e^{-\frac{x}{8}}\left(\cos\left(\omega x\right)+\frac{\sin(\omega x)}{8\omega}\right)
\mbox{ avec } \omega=\frac{\sqrt{127}}{8}
\]

\begin{center}
\bgroup
\psset{xunit=.6,yunit=0.8,plotpoints=500}
\begin{pspicture}(0,-4.25)(26,5.25)
  \psaxes{->}(0,0)(0,-4)(26,5)
  \psplot[plotpoints=200,linewidth=4\pslinewidth,linecolor=gray]{0}{26}{%
     Euler x -8 div exp x 127 sqrt 8 div mul RadtoDeg dup cos 5 mul exch sin 127 sqrt div 5 mul add mul}
  \psplotDiffEqn[linecolor=red,linewidth=5\pslinewidth]{0}{26}{5 0}
     {dup 3 1 roll -4 div exch 2 mul sub}
  \psplotDiffEqn[linecolor=black,algebraic=true]{0}{26}{5 0} {y[1]|-y[1]/4-2*y[0]}
  \psset{method=rk4, plotpoints=50}
  \psplotDiffEqn[linecolor=blue,linewidth=5\pslinewidth]{0}{26}{5 0}{%
      dup 3 1 roll -4 div exch 2 mul sub}
  \psplotDiffEqn[linecolor=black,algebraic=true]{0}{26}{5 0}{y[1]|-y[1]/4-2*y[0]}
\end{pspicture}
\egroup
\end{center}

\begin{lstlisting}
\psset{xunit=.6,yunit=0.8,plotpoints=500}
\begin{pspicture}(0,-4.25)(26,5.25)
  \psaxes{->}(0,0)(0,-4)(26,5)
  \psplot[plotpoints=200,linewidth=4\pslinewidth,linecolor=gray]{0}{26}{%
     Euler x -8 div exp x 127 sqrt 8 div mul RadtoDeg dup cos 5 mul exch sin 127 sqrt div 5 mul add mul}
  \psplotDiffEqn[linecolor=red,linewidth=5\pslinewidth]{0}{26}{5 0}
     {dup 3 1 roll -4 div exch 2 mul sub}
  \psplotDiffEqn[linecolor=black,algebraic=true]{0}{26}{5 0} {y[1]|-y[1]/4-2*y[0]}
  \psset{method=rk4, plotpoints=50}
  \psplotDiffEqn[linecolor=blue,linewidth=5\pslinewidth]{0}{26}{5 0}{%
      dup 3 1 roll -4 div exch 2 mul sub}
  \psplotDiffEqn[linecolor=black,algebraic=true]{0}{26}{5 0}{y[1]|-y[1]/4-2*y[0]}
\end{pspicture}
\end{lstlisting}


\clearpage
\subsection{Save final state of a equation}
With the macros \Lcs{BeginSaveFinalState} and \Lcs{EndSaveFinalState} the
end values of a differential equation
can be saved and then used with the optional argument \Lkeyword{GetFinalState}  
as starting values for another equation.

\begin{lstlisting}
\psset{unit=10cm,linewidth=2pt}
\begin{pspicture}(1,1)\psgrid[subgridcolor=black!20,subgriddiv=20]
\BeginSaveFinalState
 \psplotDiffEqn[
   whichabs=0,whichord=1,linecolor=red,method=rk4,
   plotpoints=10,showpoints=true]{0}{1}{0 0}{
   pop pop
   x dup mul 2 div 180 mul cos %% dx/dt
   x dup mul 2 div 180 mul sin %% dy/dt
 }
 \psplotDiffEqn[GetFinalState,
   whichabs=0,whichord=1,linecolor=blue,method=rk4,%SaveFinalState,
   plotpoints=10,showpoints=true]{1}{2}{0 0}{
   pop pop
   x dup mul 2 div 180 mul cos %% dx/dt
   x dup mul 2 div 180 mul sin %% dy/dt
 }
 \psplotDiffEqn[GetFinalState,
   whichabs=0,whichord=1,linecolor=cyan,method=rk4,%SaveFinalState,
   plotpoints=19,showpoints=true]{2}{3}{0 0 }{
   pop pop
   x dup mul 2 div 180 mul cos %% dx/dt
   x dup mul 2 div 180 mul sin %% dy/dt
 }
\EndSaveFinalState
\end{pspicture}
\end{lstlisting}


\bigskip
\begin{center}
\psset{unit=6cm,linewidth=2pt}
\begin{pspicture}(1,1)\psgrid[subgridcolor=black!20,subgriddiv=20]
\BeginSaveFinalState
 \psplotDiffEqn[
   whichabs=0,whichord=1,linecolor=red,method=rk4,
   plotpoints=10,showpoints=true]{0}{1}{0 0}{
   pop pop
   x dup mul 2 div 180 mul cos %% dx/dt
   x dup mul 2 div 180 mul sin %% dy/dt
 }
 \psplotDiffEqn[GetFinalState,
   whichabs=0,whichord=1,linecolor=blue,method=rk4,%SaveFinalState,
   plotpoints=10,showpoints=true]{1}{2}{0 0}{
   pop pop
   x dup mul 2 div 180 mul cos %% dx/dt
   x dup mul 2 div 180 mul sin %% dy/dt
 }
 \psplotDiffEqn[GetFinalState,
   whichabs=0,whichord=1,linecolor=cyan,method=rk4,%SaveFinalState,
   plotpoints=19,showpoints=true]{2}{3}{0 0 }{
   pop pop
   x dup mul 2 div 180 mul cos %% dx/dt
   x dup mul 2 div 180 mul sin %% dy/dt
 }
\EndSaveFinalState
\end{pspicture}
\end{center}

\psset{unit=1cm,linewidth=0.75pt}


%--------------------------------------------------------------------------------------
\clearpage
\section{\nxLcs{psMatrixPlot}}\label{sec:psMatrix}
%--------------------------------------------------------------------------------------
\begin{filecontents}{matrix.data}
/dotmatrix [ %
0  1  1  0  0  0  0  1  1  1
0  1  1  0  1  1  1  0  1  0
1  0  1  1  0  0  0  1  1  0
0  0  1  0  0  0  0  0  1  1
1  1  1  1  1  0  1  0  0  1
0  0  1  1  0  1  0  1  1  1
1  0  0  0  1  1  0  0  0  1
0  0  0  1  1  1  0  1  1  0
1  1  0  0  0  0  1  0  0  1
1  0  1  0  0  1  1  1  0  0
] def
\end{filecontents}


This macro allows you to visualize a matrix. The datafile must be
defined as a PostScript matrix named \Lps{dotmatrix}:
\begin{lstlisting}[style=syntax]
/dotmatrix [ %  <------------ important line
0  1  1  0  0  0  0  1  1  1
0  1  1  0  1  1  1  0  1  0
1  0  1  1  0  0  0  1  1  0
0  0  1  0  0  0  0  0  1  1
1  1  1  1  1  0  1  0  0  1
0  0  1  1  0  1  0  1  1  1
1  0  0  0  1  1  0  0  0  1
0  0  0  1  1  1  0  1  1  0
1  1  0  0  0  0  1  0  0  1
1  0  1  0  0  1  1  1  0  0
] def        %  <------------ important line
\end{lstlisting}

Only the value 0 is important, in which case nothing happens, and
for all other cases a dot is printed. The syntax of the macro is:

\begin{BDef}
\Lcs{psMatrixPlot}\OptArgs\Largb{rows}\Largb{columns}\Largb{data file}
\end{BDef}

The \Index{matrix} is scanned line by line from the the first one to the
last. In general it appears as a bottom-to-top version of the
above listed matrix, the first row $0\,1\,1\,0\,0\,0\,0\,1\,1\,1$
is the first plotted line ($y=1$). With the option
\Lkeyword{ChangeOrder}=\true\ it looks exactly like the above view.

\bgroup
\begin{center}
\psscalebox{0.6}{%
\begin{pspicture}(-0.5,-0.75)(11,11)
  \psaxes{->}(11,11)
  \psMatrixPlot[dotsize=1.1cm,dotstyle=square*,linecolor=magenta]%
    {10}{10}{matrix.data}
  \psMatrixPlot[dotsize=.5cm,dotstyle=o,ChangeOrder]{10}{10}{matrix.data}
\end{pspicture}}\quad
\psscalebox{0.6}{%
\begin{pspicture}(-0.5,-0.75)(11,11)
  \psaxes[ticksize=-5pt 0]{->}(11,11)
  \psMatrixPlot[dotsize=1.1cm,dotstyle=square*,linecolor=magenta,XYoffset=-0.5]%
    {10}{10}{matrix.data}
  \psMatrixPlot[dotsize=.5cm,dotstyle=o,ChangeOrder,XYoffset=-0.5]{10}{10}{matrix.data}
\end{pspicture}}
\end{center}

\begin{lstlisting}
\psscalebox{0.6}{%
\begin{pspicture}(-0.5,-0.75)(11,11)
  \psaxes[ticksize=-5pt 0]{->}(11,11)
  \psMatrixPlot[dotsize=1.1cm,dotstyle=square*,linecolor=magenta]%
    {10}{10}{matrix.data}
  \psMatrixPlot[dotsize=.5cm,dotstyle=o,ChangeOrder]{10}{10}{matrix.data}
\end{pspicture}}\quad
\psscalebox{0.6}{%
\begin{pspicture}(-0.5,-0.75)(11,11)
  \psaxes{->}(11,11)
  \psMatrixPlot[dotsize=1.1cm,dotstyle=square*,linecolor=magenta,XYoffset=-0.5]%
    {10}{10}{matrix.data}
  \psMatrixPlot[dotsize=.5cm,dotstyle=o,ChangeOrder,XYoffset=-0.5]{10}{10}{matrix.data}
\end{pspicture}}
\end{lstlisting}

\begin{LTXexample}[pos=t,preset=\centering]
\begin{pspicture}(-0.5,-0.75)(11,11)
  \psaxes[ticksize=-5pt 0]{->}(11,11)
  \psMatrixPlot[dotscale=3,dotstyle=*,linecolor=blue]{10}{8}{matrix.data}
\end{pspicture}
\end{LTXexample}

\clearpage
With the \Lkeyword{colorType}=1 the data is printed as continous color
in the range of the wavelength. The smallest value of the data array
is set to red and the biggest value is set to violett. All other values
are substituted by the corresponding color of the wavlength.
\Lkeyword{colorType}=2 ist the same, but vice versa
with the color, from violet to red. \Lkeyword{colorType}=3 is the grayscale
image and \Lkeyword{colorType}=4 the same invers.

The following examples use a 200$\times$200
matrix data, which is saved as /dotmatrix [...] in the file \LFile{pstricks-add-doc.dat}.

\begin{LTXexample}[pos=t,preset=\centering]
\begin{pspicture}(10,10)
  \psMatrixPlot[colorType=1,xStep=0.05,yStep=0.05]{200}{200}{dotmatrix.data}
\end{pspicture}
\end{LTXexample}

\begin{LTXexample}[pos=t,preset=\centering]
\begin{pspicture}(10,10)
  \psMatrixPlot[colorType=2,xStep=0.05,yStep=0.05]{200}{200}{dotmatrix.data}
\end{pspicture}
\end{LTXexample}

\begin{LTXexample}[pos=t,preset=\centering]
\begin{pspicture}(10,10)
  \psMatrixPlot[colorType=3,xStep=0.05,yStep=0.05]{200}{200}{dotmatrix.data}
\end{pspicture}
\end{LTXexample}

\begin{LTXexample}[pos=t,preset=\centering]
\begin{pspicture}(10,10)
  \psMatrixPlot[colorType=4,xStep=0.05,yStep=0.05]{200}{200}{dotmatrix.data}
\end{pspicture}
\end{LTXexample}
\egroup

\clearpage
With the \Lkeyword{colorType}=5 the color setting can be user defined by the
optional argument \Lkeyword{colorTypeDef}. On the stack is the current value
which can be used for the setting but must be left on the stack when everything
is finished. The following example prints the 0 as color white, the value 1 as
black and all other values depending to the corresponding gray value.

\begin{filecontents*}{matrix1.data}
/dotmatrix [ % <------------ important line
3 0 0 0 0 0 0 0 1 2
0 0 0 0 0 0 0 1 2 1
8 0 0 0 0 0 1 2 1 0
0 0 0 0 0 1 2 1 0 0
0 0 0 0 1 2 1 0 0 0
9 0 0 1 2 1 3 0 0 0
0 0 1 2 1 4 0 0 0 0
0 1 2 1 5 0 0 0 0 0
1 2 1 6 0 0 0 0 0 0
2 1 7 0 0 0 0 0 0 3
] def % <------------ important line
\end{filecontents*}

\begin{center}
\psscalebox{0.7}{%
\begin{pspicture}(-0.5,-0.75)(11,11)
\psaxes[ticksize=-5pt 0]{->}(11,11)
\psMatrixPlot[
  colorType=5,
  colorTypeDef={
    dup /value exch def % save value and leave one on the stack
    value Min sub dMaxMin div neg 1 add 300 mul 400 add \pswavelengthToGRAY 
    value 0 eq \pslbrace 1 \psrbrace if % 
    value 1 eq \pslbrace 0 \psrbrace if  
    setgray 
  },
  dotsize=1.1cm,xStep=1,yStep=1,dotstyle=square*]{10}{10}{matrix1.data}
\end{pspicture}}
\end{center}


\begin{lstlisting}
\begin{filecontents}{matrix1.data}
/dotmatrix [ % <------------ important line
3 0 0 0 0 0 0 0 1 2
0 0 0 0 0 0 0 1 2 1
8 0 0 0 0 0 1 2 1 0
0 0 0 0 0 1 2 1 0 0
0 0 0 0 1 2 1 0 0 0
9 0 0 1 2 1 3 0 0 0
0 0 1 2 1 4 0 0 0 0
0 1 2 1 5 0 0 0 0 0
1 2 1 6 0 0 0 0 0 0
2 1 7 0 0 0 0 0 0 3
] def % <------------ important line
\end{filecontents}
\psscalebox{0.7}{%
\begin{pspicture}(-0.5,-0.75)(11,11)
\psaxes[ticksize=-5pt 0]{->}(11,11)
\psMatrixPlot[
  colorType=5,
  colorTypeDef={
    dup /value exch def % save value and leave one on the stack
    value Min sub dMaxMin div neg 1 add 300 mul 400 add \pswavelengthToGRAY 
    value 0 eq \pslbrace 1 \psrbrace if % 
    value 1 eq \pslbrace 0 \psrbrace if  
    setgray 
  },
  dotsize=1.1cm,xStep=1,yStep=1,dotstyle=square*]{10}{10}{matrix1.data}
\end{pspicture}}
\end{lstlisting}


\Lps{if} statements in the color definition must be enclosed with \Lcs{pslbrace} and \Lcs{psrbrace}
when they are parentheses used in PostScript. In the above example the color definition should be
modified when the matrix is a real big one, in such a case a nested \Lps{ifelse} makes more sense:

\begin{lstlisting}
  colorTypeDef={
    dup /value exch def 
    value 0 eq 
      \pslbrace 1 setgray \psrbrace
      \pslbrace value 1 eq 
        \pslbrace 0 setgray \psrbrace
        \pslbrace Min sub dMaxMin div neg 1 add 300 mul 400 add
          \pswavelengthToGRAY setgray \psrbrace ifelse
      \psrbrace ifelse 
  },
\end{lstlisting}

Replace the \Lcs{pslbrace} and \Lcs{psrbrace} with \{ and \} if it maybe confusing to read:

\begin{lstlisting}
    dup /value exch def 
    value 0 eq 
      { 1 setgray }
      { value 1 eq 
        { 0 setgray }
        { Min sub dMaxMin div neg 1 add 300 mul 400 add
          \pswavelengthToGRAY setgray } ifelse
      } ifelse 
\end{lstlisting}

Another possibility is to define the color procedure onside the data file, where
it \emph{must} be named \Lps{colorTypeDef}. If such a definition exists, the one from
the optional argument \Lkeyword{colorTypeDef} will be ignored. There can be no
\TeX-specific code inside this definition because it is read on PostScript level,
the reason why \Lcs{pswavelengthToGRAY} cannot be used.

\begin{center}
\begin{filecontents}{matrix1.data}
/colorTypeDef {
  dup /value exch def 
  value 0 eq 
    { 1 setgray }
    { value 1 eq 
      { 0 setgray }
      { Min sub dMaxMin div neg 1 add 300 mul 400 add
%        \pswavelengthToGRAY not possible
         tx@addDict begin wavelengthToRGB Red Green Blue end 
        setrgbcolor
      } ifelse
    } ifelse 
} def
/dotmatrix [ % <------------ important line
3 0 0 0 0 0 0 0 1 2
0 0 0 0 0 0 0 1 2 1
8 0 0 0 0 0 1 2 1 0
0 0 0 0 0 1 2 1 0 0
0 0 0 0 1 2 1 0 0 0
9 0 0 1 2 1 3 0 0 0
0 0 1 2 1 4 0 0 0 0
0 1 2 1 5 0 0 0 0 0
1 2 1 6 0 0 0 0 0 0
2 1 7 0 0 0 0 0 0 3
] def % <------------ important line
\end{filecontents}
\psscalebox{0.7}{%
\begin{pspicture}(-0.5,-0.75)(11,11)
\psaxes[ticksize=-5pt 0]{->}(11,11)
\psMatrixPlot[
  colorType=5,dotsize=1.1cm,xStep=1,yStep=1,dotstyle=square*]{10}{10}{matrix1.data}
\end{pspicture}}
\end{center}

\begin{lstlisting}
\begin{filecontents}{matrix1.data}
/colorTypeDef {
  dup /value exch def 
  value 0 eq 
    { 1 setgray }
    { value 1 eq 
      { 0 setgray }
      { Min sub dMaxMin div neg 1 add 300 mul 400 add
%        \pswavelengthToRGB not possible
         tx@addDict begin wavelengthToRGB Red Green Blue end 
        setrgbcolor
      } ifelse
    } ifelse 
} def
/dotmatrix [ % <------------ important line
3 0 0 0 0 0 0 0 1 2
0 0 0 0 0 0 0 1 2 1
8 0 0 0 0 0 1 2 1 0
0 0 0 0 0 1 2 1 0 0
0 0 0 0 1 2 1 0 0 0
9 0 0 1 2 1 3 0 0 0
0 0 1 2 1 4 0 0 0 0
0 1 2 1 5 0 0 0 0 0
1 2 1 6 0 0 0 0 0 0
2 1 7 0 0 0 0 0 0 3
] def % <------------ important line
\end{filecontents}
\psscalebox{0.7}{%
\begin{pspicture}(-0.5,-0.75)(11,11)
\psaxes[ticksize=-5pt 0]{->}(11,11)
\psMatrixPlot[colorType=5,dotsize=1.1cm,xStep=1,yStep=1,
  dotstyle=square*]{10}{10}{matrix1.data}
\end{pspicture}}
\end{lstlisting}



%--------------------------------------------------------------------------------------
\section{Dashed Lines}
%--------------------------------------------------------------------------------------
Tobias Nähring has implemented an enhanced feature for dashed
lines. The number of arguments is no longer limited.

\begin{BDef}
\Lkeyword{dash}=value1\OptArg*{unit} value2\OptArg*{unit} \ldots
\end{BDef}

\begin{LTXexample}[width=0.4\linewidth]
\psset{linewidth=2.5pt,unit=0.6}
\begin{pspicture}(-5,-4)(5,4)
 \psgrid[subgriddiv=0,griddots=10,gridlabels=0pt]
  \psset{linestyle=dashed}
  \pscurve[dash=5mm 1mm 1mm 1mm,linewidth=0.1](-5,4)(-4,3)(-3,4)(-2,3)
  \psline[dash=5mm 1mm 1mm 1mm 1mm 1mm 1mm 1mm 1mm 1mm](-5,0.9)(5,0.9)
  \psccurve[linestyle=solid](0,0)(1,0)(1,1)(0,1)
  \psccurve[linestyle=dashed,dash=5mm 2mm 0.1 0.2,linetype=0](0,0)(-2.5,0)(-2.5,-2.5)(0,-2.5)
  \pscurve[dash=3mm 3mm 1mm 1mm,linecolor=red,linewidth=2pt](5,-4)(5,2)(4.5,3.5)(3,4)(-5,4)
\end{pspicture}
\end{LTXexample}



\clearpage
%--------------------------------------------------------------------------------------
\section{Arrows}
%--------------------------------------------------------------------------------------
\subsection{Definition}
%--------------------------------------------------------------------------------------
\LPack{pstricks-add} defines the following "`arrows"':

\begin{center}
  \bgroup
  \def\myline#1{\psline[linecolor=red,linewidth=0.5pt,arrowscale=1.5]{#1}(0,1ex)(1.3,1ex)}%
  \psset{arrowscale=1.5}
  \begin{tabular}{@{} c @{\qquad} p{3cm} l @{}}%
    Value & Example & Name \\[2pt]\hline
    \Lnotation{-}      & \myline{-}      & None\\
    \Lnotation{<->}    & \myline{<->}    & Arrowheads.\\
    \Lnotation{>-<}    & \myline{>-<}    & Reverse arrowheads.\\
    \Lnotation{<{<}-{>}>}  & \myline{<<->>}  & Double arrowheads.\\
    \Lnotation{{>}>-{<}<}  & \myline{>>-<<}  & Double reverse arrowheads.\\
    \Lnotation{{|}-{|}}    & \myline{|-|}    & T-bars, flush to endpoints.\\
    \Lnotation{{|}*-{|}*}  & \myline{|*-|*}  & T-bars, centered on endpoints.\\
    \Lnotation{[-]}    & \myline{[-]}    & Square brackets.\\
    \Lnotation{]-[}    & \myline{]-[}    & Reversed square brackets.\\
    \Lnotation{(-)}    & \myline{(-)}    & Rounded brackets.\\
    \Lnotation{)-(}    & \myline{)-(}    & Reversed rounded brackets.\\
    \Lnotation{o-o}    & \myline{o-o}    & Circles, centered on endpoints.\\
    \Lnotation{*-*}    & \myline{*-*}    & Disks, centered on endpoints.\\
    \Lnotation{oo-oo}  & \myline{oo-oo}  & Circles, flush to endpoints.\\
    \Lnotation{**-**}  & \myline{**-**}  & Disks, flush to endpoints.\\
    \Lnotation{{|}<->{|}}  & \myline{|<->|}  & T-bars and arrows.\\
    \Lnotation{{|}>-<{|}}  & \myline{|>-<|}  & T-bars and reverse arrows.\\
    \Lnotation{h-h{|}}   & \myline{h-h}    & left/right hook arrows.\\
    \Lnotation{H-H{|}}   & \myline{H-H}    & left/right hook arrows.\\
    \Lnotation{v-v|}   & \myline{v-v}    & left/right inside vee arrows.\\
    \Lnotation{V-V|}   & \myline{V-V}    & left/right outside vee arrows.\\
    \Lnotation{f-f|}   & \myline{f-f}    & left/right inside filled arrows.\\
    \Lnotation{F-F|}   & \myline{F-F}    & left/right outside filled arrows.\\
    \Lnotation{t-t|}   & \myline{t-t}    & left/right inside slash arrows.\\[5pt]
    \Lnotation{T-T|}   & \myline{T-T}    & left/right outside slash arrows.\\
  \end{tabular}
  \egroup
\end{center}



You can also mix and match, e.g., \Lnotation{->}, \Lnotation{*-)} and \Lnotation{[->} are all valid values
of the \Lkeyword{arrows} parameter. The parameter can be set with

\begin{BDef}
\Lcs{psset}\Largb{arrows=<type>}
\end{BDef}

\noindent or for some macros with a special option, like\\[5pt]
\noindent\verb|\psline[<general options>]{<arrow type>}(A)(B)|\\
\noindent\verb/\psline[linecolor=red,linewidth=2pt]{|->}(0,0)(0,2)/ \ \psline[linecolor=red,linewidth=2pt]{|->}(0,0)(0,2)

\subsection{Multiple arrows}
There are two new options which are only valid for the arrow type \verb+<<+ or \verb+>>+.
\verb+nArrow+ sets both, the \verb+nArrowA+ and the  \verb+nArrowB+ parameter. The meaning
is declared in the following tables. Without setting one of these parameters the behaviour
is like the one described in the old PSTricks manual.

\begin{center}
\begin{tabular}{@{}lc@{}}%
    Value & Meaning \\[2pt]\hline
    \Lnotation{-{>}>}   & \ -A \\
    \Lnotation{{<}<-{>}>} & A-A\\
    \Lnotation{{<}<-}   & A-\ \\
    \Lnotation{{>}>-}   & B-\ \\
    \Lnotation{-{<}<}   & \ -B\\
    \Lnotation{{>}>-{<}<} & B-B\\
    \Lnotation{{>}>-{>}>} & B-A\\
    \Lnotation{{<}<-{<}<} & A-B
  \end{tabular}
\end{center}




\begin{center}
  \bgroup
  \psset{linecolor=red,linewidth=1pt,arrowscale=2}%
  \begin{tabular}{lp{2.8cm}}%
    Value & Example \\[2pt]\hline
    \verb+\psline{->>}(0,1ex)(2.3,1ex)+  & \psline{->>}(0,1ex)(2.3,1ex) \\
    \verb+\psline[nArrowsA=3]{->>}(0,1ex)(2.3,1ex)+  & \psline[nArrowsA=3]{->>}(0,1ex)(2.3,1ex)\\
    \verb+\psline[nArrowsA=5]{->>}(0,1ex)(2.3,1ex)+  & \psline[nArrowsA=5]{->>}(0,1ex)(2.3,1ex)\\
    \verb+\psline{<<-}(0,1ex)(2.3,1ex)+  & \psline{<<-}(0,1ex)(2.3,1ex)\\
    \verb+\psline[nArrowsA=3]{<<-}(0,1ex)(2.3,1ex)+  & \psline[nArrowsA=3]{<<-}(0,1ex)(2.3,1ex)\\
    \verb+\psline[nArrowsA=5]{<<-}(0,1ex)(2.3,1ex)+  & \psline[nArrowsA=5]{<<-}(0,1ex)(2.3,1ex)\\
    \verb+\psline{<<->>}(0,1ex)(2.3,1ex)+  & \psline{<<->>}(0,1ex)(2.3,1ex)\\
    \verb+\psline[nArrowsA=3]{<<->>}(0,1ex)(2.3,1ex)+  & \psline[nArrowsA=3]{<<->>}(0,1ex)(2.3,1ex)\\
    \verb+\psline[nArrowsA=5]{<<->>}(0,1ex)(2.3,1ex)+  & \psline[nArrowsA=5]{<<->>}(0,1ex)(2.3,1ex)\\
    \verb+\psline{<<-|}(0,1ex)(2.3,1ex)+  & \psline{<<-|}(0,1ex)(2.3,1ex)\\
    \verb+\psline[nArrowsA=3]{<<-<<}(0,1ex)(2.3,1ex)+  & \psline[nArrowsA=3]{<<-<<}(0,1ex)(2.3,1ex)\\
    \verb+\psline[nArrowsA=5]{<<-o}(0,1ex)(2.3,1ex)+  & \psline[nArrowsA=5]{<<-o}(0,1ex)(2.3,1ex)\\
    \verb+\psline[nArrowsA=3,nArrowsB=4]{<<-<<}(0,1ex)(2.3,1ex)+  & \psline[nArrowsA=3,nArrowsB=4]{<<-<<}(0,1ex)(2.3,1ex)\\
    \verb+\psline[nArrowsA=3,nArrowsB=4]{>>->>}(0,1ex)(2.3,1ex)+  & \psline[nArrowsA=3,nArrowsB=4]{>>->>}(0,1ex)(2.3,1ex)\\
    \verb+\psline[nArrowsA=1,nArrowsB=4]{>>->>}(0,1ex)(2.3,1ex)+  & \psline[nArrowsA=1,nArrowsB=4]{>>->>}(0,1ex)(2.3,1ex)\\
  \end{tabular}
  \egroup
\end{center}



\subsection{\texttt{hookarrow}}
%\begin{LTXexample}
\bgroup
\psset{arrowsize=8pt,arrowlength=1,linewidth=1pt,nodesep=2pt,shortput=tablr}
\large
\begin{psmatrix}[colsep=12mm,rowsep=10mm]
        &   & $R_2$            \\
        &   &   0   &   & $R_3$\\
$e_b:S$ & 1 &       & 1 & 0    \\
        &   &   0              \\
        &   &   $R_1$          \\
\end{psmatrix}
\ncline{h-}{1,3}{2,3}<{$e_{r2}$}>{$f_{r2}$}
\ncline{-h}{2,3}{3,2}<{$e_1$}
\ncline{-h}{3,1}{3,2}^{$e_s$}_{$f_{s}$}
\ncline{-h}{3,2}{4,3}>{$e_3$}<{$f_3$}
\ncline{-h}{4,3}{3,4}>{$e_4$}<{$f_4$}
\ncline{-h}{3,4}{2,3}>{$e_2$}<{$f_2$}
\ncline{-h}{3,4}{3,5}^{$e_5$}
\ncline{-h}{3,5}{2,5}<{$e_{r3}$}>{$f_{r3}$}
\ncline{-h}{4,3}{5,3}<{$e_{r1}$}>{$f_{r1}$}
%\end{LTXexample}
\egroup

\begin{lstlisting}
\psset{arrowsize=8pt,arrowlength=1,linewidth=1pt,nodesep=2pt,shortput=tablr}
\large
\begin{psmatrix}[colsep=12mm,rowsep=10mm]
        &   & $R_2$            \\
        &   &   0   &   & $R_3$\\
$e_b:S$ & 1 &       & 1 & 0    \\
        &   &   0              \\
        &   &   $R_1$          \\
\end{psmatrix}
\ncline{h-}{1,3}{2,3}<{$e_{r2}$}>{$f_{r2}$}\ncline{-h}{2,3}{3,2}<{$e_1$}
\ncline{-h}{3,1}{3,2}^{$e_s$}_{$f_{s}$}    \ncline{-h}{3,2}{4,3}>{$e_3$}<{$f_3$}
\ncline{-h}{4,3}{3,4}>{$e_4$}<{$f_4$}      \ncline{-h}{3,4}{2,3}>{$e_2$}<{$f_2$}
\ncline{-h}{3,4}{3,5}^{$e_5$}              
\ncline{-h}{3,5}{2,5}<{$e_{r3}$}>{$f_{r3}$}
\ncline{-h}{4,3}{5,3}<{$e_{r1}$}>{$f_{r1}$}
\end{lstlisting}



\subsection{\texttt{hookrightarrow} and \texttt{hookleftarrow}}
This is another type of arrow and is abbreviated with \Lnotation{H}.
The length and width of the hook is set by the new options
\Lkeyword{hooklength} and \Lkeyword{hookwidth}, which are by default set
to
%
\begin{BDef}
\Lcs{psset}\Largb{hooklength=3mm,hookwidth=1mm}
\end{BDef}
%
If the line begins with a right hook then the line ends with a left hook and vice versa:

\begin{LTXexample}[width=3cm]
\begin{pspicture}(3,4)
\psline[linewidth=5pt,linecolor=blue,hooklength=5mm,hookwidth=-3mm]{H->}(0,3.5)(3,3.5)
\psline[linewidth=5pt,linecolor=red,hooklength=5mm,hookwidth=3mm]{H->}(0,2.5)(3,2.5)
\psline[linewidth=5pt,hooklength=5mm,hookwidth=3mm]{H-H}(0,1.5)(3,1.5)
\psline[linewidth=1pt]{H-H}(0,0.5)(3,0.5)
\end{pspicture}
\end{LTXexample}


\begin{LTXexample}[width=7.25cm]
$\begin{psmatrix}
E&W_i(X)&&Y\\
&&W_j(X)
\psset{arrows=->,nodesep=3pt,linewidth=2pt}
\everypsbox{\scriptstyle}
\ncline[linecolor=red,arrows=H->,%
  hooklength=4mm,hookwidth=2mm]{1,1}{1,2}
\ncline{1,2}{1,4}^{\tilde{t}}
\ncline{1,2}{2,3}<{W_{ij}}
\ncline{2,3}{1,4}>{\tilde{s}}
\end{psmatrix}$
\end{LTXexample}


%--------------------------------------------------------------------------------------
\subsection{\nxLkeyword{ArrowInside} Option}
%--------------------------------------------------------------------------------------

It is now possible to have arrows inside lines and not only at the
beginning or the end. The new defined options

\psset{arrowscale=2,linecolor=red,unit=1cm,linewidth=1.5pt}
\begin{longtable}{l|>{\RaggedRight}p{8.5cm}|p{2.2cm}}
Name & Example & Output\\\hline
\endfirsthead
Name & Example & Output\\\hline
\endhead
\Lkeyword{ArrowInside} &
  \texttt{\textbackslash psline[ArrowInside=->](0,0)(2,0)} &
  \psline[ArrowInside=->](0,0.1)(2,0.1) \\
\Lkeyword{ArrowInsidePos} & \texttt{\textbackslash psline[ArrowInside=->,\%}
  \hspace*{20pt}\texttt{ArrowInsidePos=0.25](0,0)(2,0)}
& \psline[ArrowInside=->, ArrowInsidePos=0.25](0,0.1)(2,0.1) \\
\Lkeyword{ArrowInsidePos} & \texttt{\textbackslash psline[ArrowInside=->,\%}
  \hspace*{20pt}\texttt{ArrowInsidePos=10](0,0)(2,0)}
& \psline[ArrowInside=->, ArrowInsidePos=10](0,0.1)(2,0.1) \\
\Lkeyword{ArrowInsideNo} & \texttt{\textbackslash psline[ArrowInside=->,\%}
  \hspace*{20pt}\texttt{ArrowInsideNo=2](0,0)(2,0)}
& \psline[ArrowInside=->, ArrowInsideNo=2](0,0.1)(2,0.1) \\
\Lkeyword{ArrowInsideOffset} & \texttt{\textbackslash psline[ArrowInside=->,\%}
  \hspace*{20pt}\texttt{ArrowInsideNo=2,\%}\newline
  \hspace*{20pt}\texttt{ArrowInsideOffset=0.1](0,0)(2,0)}
& \psline[ArrowInside=->, ArrowInsideNo=2,ArrowInsideOffset=0.1](0,0.1)(2,0.1) \\
%
\Lkeyword{ArrowInside} & \texttt{\textbackslash psline[ArrowInside=->]\{->\}(0,0)(2,0)} &
  \psline[ArrowInside=->]{->}(0,0)(2,0)\\
\Lkeyword{ArrowInsidePos} & \texttt{\textbackslash psline[ArrowInside=->,\%}
  \hspace*{20pt}\texttt{ArrowInsidePos=0.25]\{->\}(0,0)(2,0)}
  & \psline[ArrowInside=->, ArrowInsidePos=0.25]{->}(0,0)(2,0) \\
\Lkeyword{ArrowInsidePos} & \texttt{\textbackslash psline[ArrowInside=->,\%}
  \hspace*{20pt}\texttt{ArrowInsidePos=10]\{->\}(0,0)(2,0)}
  & \psline[ArrowInside=->, ArrowInsidePos=10]{->}(0,0)(2,0) \\
\Lkeyword{ArrowInsideNo} & \texttt{\textbackslash psline[ArrowInside=->,\%}
  \hspace*{20pt}\texttt{ArrowInsideNo=2]\{->\}(0,0)(2,0)}
  & \psline[ArrowInside=->, ArrowInsideNo=2]{->}(0,0)(2,0) \\
\Lkeyword{ArrowInsideOffset} & \texttt{\textbackslash psline[ArrowInside=->,\%}
  \hspace*{20pt}\texttt{ArrowInsideNo=2,\%}\newline
  \hspace*{20pt}\texttt{ArrowInsideOffset=0.1]\{->\}(0,0)(2,0)}
  & \psline[ArrowInside=->, ArrowInsideNo=2,ArrowInsideOffset=0.1]{->}(0,0)(2,0) \\
%
\Lkeyword{ArrowFill} & \texttt{\textbackslash psline[ArrowFill=false,\%}
  \hspace*{20pt}\texttt{arrowinset=0]\{->\}(0,0)(2,0)} &
  \psline[ArrowFill=false,arrowinset=0]{->}(0,0)(2,0)\\
\Lkeyword{ArrowFill} & \texttt{\textbackslash psline[ArrowFill=false,\%}
  \hspace*{20pt}\texttt{arrowinset=0]\{<<->>\}(0,0)(2,0)} &
  \psline[ArrowFill=false,arrowinset=0]{<<->>}(0,0)(2,0)\\
\Lkeyword{ArrowFill} & \texttt{\textbackslash psline[ArrowInside=->,\%}\newline
  \hspace*{20pt}\texttt{arrowinset=0,\%}\newline
  \hspace*{20pt}\texttt{ArrowFill=false,\%}\newline
  \hspace*{20pt}\texttt{ArrowInsideNo=2,\%}\newline
  \hspace*{20pt}\texttt{ArrowInsideOffset=0.1]\{->\}(0,0)(2,0)}
  & \psline[ArrowInside=->, ArrowFill=false,ArrowInsideNo=2,ArrowInsideOffset=0.1]{->}(0,0)(2,0) \\
\end{longtable}

\medskip
Without the default arrow definition there is only the one inside
the line, defined by the type and the position. The position is
relative to the length of the whole line. $0.25$ means at $25\%$
of the line length. The peak of the arrow gets the coordinates
which are calculated by the macro. If you want arrows with an
absolute position difference, then choose a value greater than
\verb|1|, e.\,g. \verb|10| which places an arrow every 10~pt. The
default unit \verb|pt| cannot be changed.

\medskip
\noindent
\begin{tabularx}{\linewidth}{@{\color{red}\vrule width 2pt}lX@{}}
& The \Lkeyword{ArrowInside} takes only arrow definitions like \Lnotation{->} into account.
Arrows from right to left (\Lnotation{<-}) are not possible and ignored. If you need
such arrows, change the order of the pairs of coordinates for the line or curve macro.
\end{tabularx}

%--------------------------------------------------------------------------------------
\subsection{\nxLkeyword{ArrowFill} Option}
%--------------------------------------------------------------------------------------

By default all arrows are filled polygons. With the option
\Lkeyset{ArrowFill=false} there are ''white`` arrows. Only for the
beginning/end arrows are they empty, the inside arrows are
overpainted by the line.


\psset{arrowscale=1}
\begin{LTXexample}[width=3.5cm]
\psset{arrowscale=2.5}
\psline[linecolor=red,arrowinset=0]{<->}(-1,0)(2,0)
\end{LTXexample}

\begin{LTXexample}[width=3.5cm]
\psset{arrowscale=2.5}
\psline[linecolor=red,arrowinset=0,ArrowFill=false]{<->}(-1,0)(2,0)
\end{LTXexample}

\begin{LTXexample}[width=3.5cm]
\psset{arrowscale=2.5}
\psline[linecolor=red,arrowinset=0,arrowsize=0.2,
  ArrowFill=false]{<->}(-1,0)(2,0)
\end{LTXexample}

\begin{LTXexample}[width=3.5cm]
\psline[linecolor=blue,arrowscale=4,
  ArrowFill]{>>->>}(-1,0)(2,0)
\end{LTXexample}

\begin{LTXexample}[width=3.5cm]
\psline[linecolor=blue,arrowscale=4,
  ArrowFill=false]{>>->>}(-1,0)(2,0)
\rule{3cm}{0pt}\\[30pt]
\end{LTXexample}

\begin{LTXexample}[width=3.5cm]
\psline[linecolor=blue,arrowscale=4,
  ArrowFill]{>|->|}(-1,0)(2,0)
\end{LTXexample}

\begin{LTXexample}[width=3.5cm]
\psline[linecolor=blue,arrowscale=4,
  ArrowFill=false]{>|->|}(-1,0)(2,0)%
\end{LTXexample}


%--------------------------------------------------------------------------------------
\subsection{Examples}
%--------------------------------------------------------------------------------------

All examples are printed with \verb|\psset{arrowscale=2,linecolor=red}|.
\subsubsection{\nxLcs{psline}}

\bigskip
\begin{LTXexample}[width=2.5cm]
\begin{pspicture}(2,2)
\psset{arrowscale=2,ArrowFill=true}
\psline[ArrowInside=->]{|<->|}(2,1)
\end{pspicture}
\end{LTXexample}

\begin{LTXexample}[width=2.5cm]
\begin{pspicture}(2,2)
\psset{arrowscale=2,ArrowFill=true}
\psline[ArrowInside=-|]{|-|}(2,1)
\end{pspicture}
\end{LTXexample}

\begin{LTXexample}[width=2.5cm]
\begin{pspicture}(2,2)
\psset{arrowscale=2,ArrowFill=true}
\psline[ArrowInside=->,ArrowInsideNo=2]{->}(2,1)
\end{pspicture}
\end{LTXexample}

\begin{LTXexample}[width=2.5cm]
\begin{pspicture}(2,2)
\psset{arrowscale=2,ArrowFill=true}
\psline[ArrowInside=->,ArrowInsideNo=2,ArrowInsideOffset=0.1]{->}(2,1)
\end{pspicture}
\end{LTXexample}

\begin{LTXexample}[width=6.5cm]
\begin{pspicture}(6,2)
\psset{arrowscale=2,ArrowFill=true}
\psline[ArrowInside=-*]{->}(0,0)(2,1)(3,0)(4,0)(6,2)
\end{pspicture}
\end{LTXexample}

\begin{LTXexample}[width=6.5cm]
\begin{pspicture}(6,2)
\psset{arrowscale=2,ArrowFill=true}
\psline[ArrowInside=-*,ArrowInsidePos=0.25]{->}(0,0)(2,1)(3,0)(4,0)(6,2)
\end{pspicture}
\end{LTXexample}

\begin{LTXexample}[width=6.5cm]
\begin{pspicture}(6,2)
\psset{arrowscale=2,ArrowFill=true}
\psline[ArrowInside=-*,ArrowInsidePos=0.25,ArrowInsideNo=2]{->}%
   (0,0)(2,1)(3,0)(4,0)(6,2)
\end{pspicture}
\end{LTXexample}

\begin{LTXexample}[width=6.5cm]
\begin{pspicture}(6,2)
\psset{arrowscale=2,ArrowFill=true}
\psline[ArrowInside=->, ArrowInsidePos=0.25]{->}%
        (0,0)(2,1)(3,0)(4,0)(6,2)
\end{pspicture}
\end{LTXexample}

\begin{LTXexample}[width=6.5cm]
\begin{pspicture}(6,2)
\psset{arrowscale=2,ArrowFill=true}
\psline[linestyle=none,ArrowInside=->,ArrowInsidePos=0.25]{->}%
        (0,0)(2,1)(3,0)(4,0)(6,2)
\end{pspicture}
\end{LTXexample}

\begin{LTXexample}[width=6.5cm]
\begin{pspicture}(6,2)
\psset{arrowscale=2,ArrowFill=true}
\psline[ArrowInside=-<, ArrowInsidePos=0.75]{->}%
     (0,0)(2,1)(3,0)(4,0)(6,2)
\end{pspicture}
\end{LTXexample}

\begin{LTXexample}[width=6.5cm]
\begin{pspicture}(6,2)
\psset{arrowscale=2,ArrowFill=true,ArrowInside=-*}
\psline(0,0)(2,1)(3,0)(4,0)(6,2)
\psset{linestyle=none}
\psline[ArrowInsidePos=0](0,0)(2,1)(3,0)(4,0)(6,2)
\psline[ArrowInsidePos=1](0,0)(2,1)(3,0)(4,0)(6,2)
\end{pspicture}
\end{LTXexample}

\begin{LTXexample}[width=6.5cm]
\begin{pspicture}(6,5)
\psset{arrowscale=2,ArrowFill=true}
\psline[ArrowInside=->,ArrowInsidePos=20](0,0)(3,0)%
       (3,3)(1,3)(1,5)(5,5)(5,0)(7,0)(6,3)
\end{pspicture}
\end{LTXexample}

\begin{LTXexample}[width=6.5cm]
\begin{pspicture}(6,2)
\psset{arrowscale=2,ArrowFill=true}
\psline[ArrowInside=-|]{<->}(0,2)(2,0)(3,2)(4,0)(6,2)
\end{pspicture}
\end{LTXexample}

%--------------------------------------------------------------------------------------
\subsubsection{\nxLcs{pspolygon}}
%--------------------------------------------------------------------------------------
% Polygons (\pspolygon macro)

\begin{LTXexample}[width=6.5cm]
\begin{pspicture}(6,3)
\psset{arrowscale=2}
\pspolygon[ArrowInside=-|](0,0)(3,3)(6,3)(6,1)
\end{pspicture}
\end{LTXexample}

\begin{LTXexample}[width=6.5cm]
\begin{pspicture}(6,3)
\psset{arrowscale=2}
\pspolygon[ArrowInside=->,ArrowInsidePos=0.25]%
     (0,0)(3,3)(6,3)(6,1)
\end{pspicture}
\end{LTXexample}

\begin{LTXexample}[width=6.5cm]
\begin{pspicture}(6,3)
\psset{arrowscale=2}
\pspolygon[ArrowInside=->,ArrowInsideNo=4]%
       (0,0)(3,3)(6,3)(6,1)
\end{pspicture}
\end{LTXexample}

\begin{LTXexample}[width=6.5cm]
\begin{pspicture}(6,3)
\psset{arrowscale=2}
\pspolygon[ArrowInside=->,ArrowInsideNo=4,%
   ArrowInsideOffset=0.1](0,0)(3,3)(6,3)(6,1)
\end{pspicture}
\end{LTXexample}

\begin{LTXexample}[width=6.5cm]
\begin{pspicture}(6,3)
\psset{arrowscale=2}
 \pspolygon[ArrowInside=-|](0,0)(3,3)(6,3)(6,1)
 \psset{linestyle=none,ArrowInside=-*}
 \pspolygon[ArrowInsidePos=0](0,0)(3,3)(6,3)(6,1)
 \pspolygon[ArrowInsidePos=1](0,0)(3,3)(6,3)(6,1)
 \psset{ArrowInside=-o}
 \pspolygon[ArrowInsidePos=0.25](0,0)(3,3)(6,3)(6,1)
 \pspolygon[ArrowInsidePos=0.75](0,0)(3,3)(6,3)(6,1)
\end{pspicture}
\end{LTXexample}

\psset{linestyle=solid}

\begin{LTXexample}[width=6.5cm]
\begin{pspicture}(6,5)
\psset{arrowscale=2}
  \pspolygon[ArrowInside=->,ArrowInsidePos=20]%
    (0,0)(3,0)(3,3)(1,3)(1,5)(5,5)(5,0)(7,0)(6,3)
\end{pspicture}
\end{LTXexample}


%--------------------------------------------------------------------------------------
\subsubsection{\nxLcs{psbezier}}
%--------------------------------------------------------------------------------------
% Bezier curves (\psbezier macro)


\begin{LTXexample}[width=3.5cm]
\begin{pspicture}(3,3)
\psset{arrowscale=2}
  \psbezier[ArrowInside=-|](0,1)(1,0)(2,1)(3,3)
  \psset{linestyle=none,ArrowInside=-o}
  \psbezier[ArrowInsidePos=0.25](0,1)(1,0)(2,1)(3,3)
  \psbezier[ArrowInsidePos=0.75](0,1)(1,0)(2,1)(3,3)
  \psset{linestyle=none,ArrowInside=-*}
  \psbezier[ArrowInsidePos=0](0,1)(1,0)(2,1)(3,3)
  \psbezier[ArrowInsidePos=1](0,1)(1,0)(2,1)(3,3)
\end{pspicture}
\end{LTXexample}



\resetOptions
\begin{LTXexample}[width=4.5cm]
\begin{pspicture}(4,3)
\psset{arrowscale=2}
\psbezier[ArrowInside=->,showpoints]%
  {*-*}(0,0)(2,3)(3,0)(4,2)
\end{pspicture}
\end{LTXexample}




\begin{LTXexample}[width=4.5cm]
\begin{pspicture}(4,3)
\psset{arrowscale=2}
  \psbezier[ArrowInside=->,showpoints=true,
      ArrowInsideNo=2](0,0)(2,3)(3,0)(4,2)
\end{pspicture}
\end{LTXexample}


\begin{LTXexample}[width=4.5cm]
\begin{pspicture}(4,3)
\psset{arrowscale=2}
  \psbezier[ArrowInside=->,showpoints=true,
     ArrowInsideNo=2,ArrowInsideOffset=-0.2]%
      {->}(0,0)(2,3)(3,0)(4,2)
\end{pspicture}
\end{LTXexample}


\begin{LTXexample}[width=5.5cm]
\begin{pspicture}(5,3)
\psset{arrowscale=2}
  \psbezier[ArrowInsideNo=9,ArrowInside=-|,%
    showpoints=true]{*-*}(0,0)(1,3)(3,0)(5,3)
\end{pspicture}
\end{LTXexample}

\begin{LTXexample}[width=4.5cm]
\begin{pspicture}(4,3)
\psset{arrowscale=2}
  \psset{ArrowInside=-|}
  \psbezier[ArrowInsidePos=0.25,showpoints=true]{*-*}(2,3)(3,0)(4,2)
  \psset{linestyle=none}
  \psbezier[ArrowInsidePos=0.75](0,0)(2,3)(3,0)(4,2)
\end{pspicture}
\end{LTXexample}

\begin{LTXexample}[width=5.5cm]
\begin{pspicture}(5,6)
\psset{arrowscale=2}
  \pnode(3,4){A}\pnode(5,6){B}\pnode(5,0){C}
  \psbezier[ArrowInside=->,%
     showpoints=true](A)(B)(C)
  \psset{linestyle=none,ArrowInside=-<}
  \psbezier[ArrowInsideNo=4](0,0)(A)(B)(C)
  \psset{ArrowInside=-o}
  \psbezier[ArrowInsidePos=0.1](0,0)(A)(B)(C)
  \psbezier[ArrowInsidePos=0.9](0,0)(A)(B)(C)
  \psset{ArrowInside=-*}
  \psbezier[ArrowInsidePos=0.3](0,0)(A)(B)(C)
  \psbezier[ArrowInsidePos=0.7](0,0)(A)(B)(C)
\end{pspicture}
\end{LTXexample}

\psset{linestyle=solid}

\begin{LTXexample}[pos=t]
\begin{pspicture}(-3,-5)(15,5)
  \psbezier[ArrowInsideNo=19,%
      ArrowInside=->,ArrowFill=false,%
      showpoints=true]{->}(-3,0)(5,-5)(8,5)(15,-5)
\end{pspicture}
\end{LTXexample}



%--------------------------------------------------------------------------------------
\subsubsection{\nxLcs{pcline}}
%--------------------------------------------------------------------------------------
These examples need the package \verb|pst-node|.

% Lines (\pcline macro)
\begin{LTXexample}[width=2.5cm]
\begin{pspicture}(2,1)
\psset{arrowscale=2}
\pcline[ArrowInside=->](0,0)(2,1)
\end{pspicture}
\end{LTXexample}


\begin{LTXexample}[width=2.5cm]
\begin{pspicture}(2,1)
\psset{arrowscale=2}
\pcline[ArrowInside=->]{<->}(0,0)(2,1)
\end{pspicture}
\end{LTXexample}


\begin{LTXexample}[width=2.5cm]
\begin{pspicture}(2,1)
\psset{arrowscale=2}
\pcline[ArrowInside=-|,ArrowInsidePos=0.75]{|-|}(0,0)(2,1)
\end{pspicture}
\end{LTXexample}


\begin{LTXexample}[width=2.5cm]
\psset{arrowscale=2}
\pcline[ArrowInside=->,ArrowInsidePos=0.65]{*-*}(0,0)(2,0)
\naput[labelsep=0.3]{\large$g$}
\end{LTXexample}


\begin{LTXexample}[width=2.5cm]
\psset{arrowscale=2}
\pcline[ArrowInside=->,ArrowInsidePos=10]{|-|}(0,0)(2,0)
\naput[labelsep=0.3]{\large$l$}
\end{LTXexample}



%--------------------------------------------------------------------------------------
\subsubsection{\nxLcs{pccurve}}
%--------------------------------------------------------------------------------------
These examples also need the package \verb|pst-node|.

\begin{LTXexample}[width=2.5cm]
\begin{pspicture}(2,2)
\psset{arrowscale=2}
\pccurve[ArrowInside=->,ArrowInsidePos=0.65,showpoints=true]{*-*}(0,0)(2,2)
\naput[labelsep=0.3]{\large$h$}
\end{pspicture}
\end{LTXexample}


\begin{LTXexample}[width=2.5cm]
\begin{pspicture}(2,2)
\psset{arrowscale=2}
\pccurve[ArrowInside=->,ArrowInsideNo=3,showpoints=true]{|->}(0,0)(2,2)
\naput[labelsep=0.3]{\large$i$}
\end{pspicture}
\end{LTXexample}


\begin{LTXexample}[width=4.5cm]
\begin{pspicture}(4,4)
\psset{arrowscale=2}
\pccurve[ArrowInside=->,ArrowInsidePos=20]{|-|}(0,0)(4,4)
\naput[labelsep=0.3]{\large$k$}
\end{pspicture}
\end{LTXexample}

\clearpage

\subsection{Special arrows \texttt{v--V},\texttt{t--T}, and \texttt{f--F}}

Possible optional arguments are

\psset{linecolor=black}

\begin{center}
\begin{tabular}{@{}l|l@{}}\toprule
\emph{name} & \emph{meaning}\\\hline
\Lkeyword{veearrowlength} & default is 3mm\\
\Lkeyword{veearrowangle} & default is 30\\
\Lkeyword{veearrowlinewidth} & default is 0.35mm\\
\Lkeyword{filledveearrowlength} & default is 3mm\\
\Lkeyword{filledveearrowangle} & default is 15\\
\Lkeyword{filledveearrowlinewidth} & default is 0.35mm\\
\Lkeyword{tickarrowlength} & default is 1.5mm\\
\Lkeyword{tickarrowlinewidth} & default is 0.35mm\\
\Lkeyword{arrowlinestyle}     & default is solid\\\bottomrule
\end{tabular}
\end{center}


\begin{LTXexample}[width=4cm]
\psset{unit=5mm}
\begin{pspicture}(4,6)
  \psset{dimen=middle,arrows=c-c,
    arrowscale=2,linewidth=.25mm}
  \psline[linecolor=red,linewidth=.05mm](0,0)(0,6)
  \psline[linecolor=red,linewidth=.05mm](4,0)(4,6)
  \psline{v-v}(0,6)(4,6)
  \psline{v-V}(0,4)(4,4)
  \psline{V-v}(0,2)(4,2)
  \psline{V-V}(0,0)(4,0)
\end{pspicture}
\end{LTXexample}


\begin{LTXexample}[width=4cm]
\psset{unit=5mm}
\begin{pspicture}(4,6)
  \psset{dimen=middle,arrows=c-c,
    arrowscale=2,linewidth=.25mm}
  \psline[linecolor=red,linewidth=.05mm](0,0)(0,6)
  \psline[linecolor=red,linewidth=.05mm](4,0)(4,6)
  \psline{f-f}(0,6)(4,6)
  \psline{f-F}(0,4)(4,4)
  \psline{F-f}(0,2)(4,2)
  \psline{F-F}(0,0)(4,0)
\end{pspicture}
\end{LTXexample}


\begin{LTXexample}[width=4cm]
\psset{unit=5mm}
\begin{pspicture}(4,6)
  \psset{dimen=middle,arrows=c-c,linewidth=.25mm}
  \psline[linecolor=red,linewidth=.05mm](0,0)(0,6)
  \psline[linecolor=red,linewidth=.05mm](4,0)(4,6)
  \psline{t-t}(0,6)(4,6)
  \psline{t-T}(0,4)(4,4)
  \psline{T-t}(0,2)(4,2)
  \psline{T-T}(0,0)(4,0)
\end{pspicture}
\end{LTXexample}

\begin{LTXexample}[pos=t,vsep=5mm]
\psset{unit=5mm}
 \begin{pspicture}(10,6)
 \psset{dimen=middle,arrows=c-c,arrowscale=2,linewidth=.25mm,
        arrowlinestyle=dashed,dash=1.5pt 1pt}
 \psline[linecolor=red,linewidth=.05mm](0,0)(0,6)
 \psline[linecolor=red,linewidth=.05mm](4,0)(4,6)
 \psline{v-v}(0,6)(4,6) \psline{v-V}(0,4)(4,4)
 \psline{V-v}(0,2)(4,2) \psline{V-V}(0,0)(4,0)
 \psline[linecolor=red,linewidth=.05mm](6,0)(6,6)
 \psline[linecolor=red,linewidth=.05mm](10,0)(10,6)
 \psset{arrowlinestyle=dotted,dotsep=0.8pt}
 \psline{v-v}(6,6)(10,6) \psline{v-V}(6,4)(10,4)
 \psline{V-v}(6,2)(10,2) \psline{V-V}(6,0)(10,0)
\end{pspicture}
\end{LTXexample}

\begin{LTXexample}[pos=t,vsep=5mm]
\psset{unit=5mm}
 \begin{pspicture}(10,7)
 \psset{dimen=middle,arrows=c-c,arrowscale=2,linewidth=.25mm,
        arrowlinestyle=dashed,dash=1.5pt 1pt}
 \psline[linecolor=red,linewidth=.05mm](0,0)(0,6)
 \psline[linecolor=red,linewidth=.05mm](4,0)(4,6)
 \psline{t-t}(0,6)(4,6) \psline{t-T}(0,4)(4,4)
 \psline{T-t}(0,2)(4,2) \psline{T-T}(0,0)(4,0)
 \psline[linecolor=red,linewidth=.05mm](6,0)(6,6)
 \psline[linecolor=red,linewidth=.05mm](10,0)(10,6)
 \psset{arrowlinestyle=dotted,dotsep=0.8pt}
 \psline{t-t}(6,6)(10,6) \psline{t-T}(6,4)(10,4)
 \psline{T-t}(6,2)(10,2) \psline{T-T}(6,0)(10,0)
\end{pspicture}
\end{LTXexample}




\subsection{Special arrow option \texttt{arrowLW}}

Only for the arrowtype \Lnotation{o} and \Lnotation{*} it is possible to
set the arrowlinewidth with the optional keyword \Lkeyword{arrowLW}.
When scaling an arrow by the keyword \Lkeyword{arrowscale} the width
of the borderline is also scaled. With the optional argument
\Lkeyword{arrowLW} the line width can be set separately and is not
taken into account by the scaling value.

\begin{LTXexample}[width=4cm]
\begin{pspicture}(4,6)
\psline[arrowscale=3,arrows=*-o](0,5)(4,5)
\psline[arrowscale=3,arrows=*-o,
  arrowLW=0.5pt](0,3)(4,3)
\psline[arrowscale=3,arrows=*-o,
  arrowLW=0.3333\pslinewidth](0,1)(4,1)
\end{pspicture}
\end{LTXexample}



\section{Ticks and other marks along a curve}
\subsection{Quick overview}

The macros described below allow you to place tick and other marks along an arbitrary 
parametric curve with placement rules similar to those used by \Lcs{psaxes} in 
the \LPack{pst-plot} package. You have to define a metric function along the curve to 
govern tick placement. That function can be a specified function of {\tt x,y} which 
should increase along the curve, or it can be an function whose increment is a specified 
positive function of {\tt x, y, dx, dy, ds} where the last term is the arc-length element 
that you could specify alternately as {\tt dx dup mul dy dup mul add sqrt}.
% start new material


In addition, a new command \Lcs{Put} is proposed, expanding as appropriate to \Lcs{rput} or \Lcs{uput}. Its syntax is

\begin{BDef}
\LcsStar{Put}\OptArgs\OptArg*{\Largb{<ref>}}\Largr{<position>}\Largb{<stuff>}
\end{BDef}

where the optional {\tt *} blanks the background, the optional \OptArgs\ may be used to specify a rotation 
using any form acceptable to \Lcs{SpecialCoor} (eg, \nxLkeyword{rot=45} or \Lkeyword{rot}\verb|={(1,1)}| 
or \Lkeyword{rot}\verb|=(P)|, and \Larg{ref} takes one of 
two forms: \verb=(a)= a refpt such as {\tt Bl}, in which case \Lcs{rput} is called; (b) a polar form of offset 
(eg, \verb=7pt;30=, or \verb=;(P)= --- in the latter case, \Ldim{pslabelsep} is substituted for the missing 
radius), in which case a modified form of \Lcs{uput} is called. The idea of \Lcs{Put} is to allow  {\tt position}, 
{\tt ref} and {\tt rot} to be specified in any of the forms acceptable to \Lcs{SpecialCoor} and to do so with 
the same output no matter what form is used. The cost of this consistency is that \Lcs{Put} can lead to results 
that differ from \Lcs{uput} in some special cases. 


\subsection{Details}
Suppose you have drawn a parametric curve using \Lcs{psparametricplot}, and you wish to 
indicate some points on the curve using tick-marks like those  on the axes. This is a 
two-step process, the first of which serves to define at the PostScript level a 
number of data arrays containing information about the curve. Those arrays are used 
in the second step to compute tick positions and draw the ticks. The first step is 
to run the macro \Lcs{pscurvepoints}. For example,

\begin{verbatim}
\pscurvepoints[plotpoints=20]{0}{6}{t t t mul 12 div}{Pt}%
\end{verbatim}
makes a virtual (ie, data only---nothing is rendered) polyline with 20 vertices approximating 
the curve $x(t)=t, y(t)=t^2/12$, $0\le t\le 6$. The last argument {\tt Pt} is the root name 
given to the data arrays.  PostScript arrays will be created with the following names: {\tt Pt.X, Pt.Y} 
for the coordinates of the vertices, {\tt PtDelta.X, PtDelta.Y} for the increments between the 
vertices (using, eg, {\tt PtDelta.X[2]=Pt.X[2]-Pt.X[1]}) and {\tt PtNormal.X, PtNormal.Y} for 
a vector normal to {\tt PtDelta.X, PtDelta.Y} in the visual, not mathematical, sense. 
(Both senses are the same if the scales on the axes are identical.) The {\tt Normal} is 
always constructed so as to point ``upward'' (ie, to your left) as you traverse the curve 
in the positive direction. The PostScript variable {\tt unitratio} provides the ratio of 
the unit on the y axis to that on x axis, and {\tt unitratiosq} is its square. All of 
these PostScript objects are stored in the main {\tt pstricks} dictionary \Lps{tx@Dict} 
which should be automatically made available when using many {\tt pstricks} macros. 
If {\tt gs} returns you an error message like
\begin{verbatim}
Error: /undefined in Pt.X
\end{verbatim}
then you may need to enclose the offending PostScript code within a block of the form
\begin{verbatim}
tx@Dict begin ... end
\end{verbatim}
so that the dictionary is made available.

With this preparation, the main tick-making macro may be run. For example,
\begin{verbatim}
\pspolylineticks{Pt}{ dx dy add 3 div }{1}{2}%
\end{verbatim}
looks for data arrays made using \Lcs{pscurvepoints} with the root name {\tt Pt}. The next argument, 
{\tt dx dy add 3 div}, specifies the (PostScript) function of increments that should be used to 
construct the metric. If the keyword \verb|metricInitValue| is defined, eg, with 
\Lcs{psset}\Largb{\Lkeyword{metricInitValue}=2.5}, it is used as the initial value of the metric, 
otherwise it is defined to be 0. In the previous example, the increment function is always 
positive, and care should be taken to guarantee this is so or the results will not be meaningful. 
(If we wanted to use arc-length, the function would have been {\tt ds}, assuming equal scales on 
the axes.) The last two arguments determine the index of the first tick and the number of ticks. 
Tick numbering begins with index 0, so the example says to drop the first tick and draw the 
next 2 ticks. In this example, where all keywords take their default values, ticks are 
potentially located at values on the curve where the metric takes a positive integer value. 
In the arc-length example, the tick with index 0 is at the beginning of the curve, and subsequent 
ticks are at unit distance, measured along the curve. At each index where a tick is drawn, a 
\Lcs{pnode} is created: In this example, you create nodes {\tt PtTick1, PtTick2} on the curve 
where the ticks are located. This is handy for placing labels using, eg, \Lcs{Put}. In 
addition, PostScript data arrays (in this example, {\tt PtTickN.X, PtTickN.Y} of the normals 
at these nodes are stored in the dictionary {\tt TDict}. More importantly, the tangent and 
normal vectors at {\tt PtTick0} etc are constructed as nodes with names {\tt PtTangent0, PtNormal0} 
etc. See the last example below for typical usage.

The shape of the ticks is governed by the keywords \Lkeyword{ticksize} (default value {\tt -4pt 4pt})
 and \Lkeyword{tickwidth} (default value \verb|.5\linewidth|.) With the default settings, ticks 
 are drawn perpendicular the the curve extending {\tt 4pt} to each side. The line
\begin{verbatim}
\pspolylineticks[ticksize=-6pt 6pt]{Pt}{ dx dy add 3 div }{1}{2}%
\end{verbatim}
would draw longer ticks than the default.

Placement of the ticks is governed by the keywords \Lkeyword{Ds} and \Lkeyword{Os}, whose meaning for the 
curve is similar to (but not the same as) the meanings of \Lkeyword{Dx} and \Lkeyword{Ox} with respect to the x axis. 
That is, if {\tt Ds=2} and {\tt Os=0}, ticks will be drawn where the metric takes 
values 0, 2, 4 and so on. More generally, ticks are placed where the metric takes 
value {\tt Os, Os+Ds, Os+2*Ds,...}, as long as those positions are on the curve. If \Lkeyword{Os} 
has an empty value as a result, say, of \verb|\psset{Os=}|, then \Lkeyword{Os} is set internally 
to the initial metric value. If \Lkeyword{Ds} has an empty value, it is set internally to the 
final metric value less the initial metric value, divided by 10. 

To draw major and minor ticks requires two passes---one to draw the minor ticks and then one to draw the major ticks.

Note that a ticks may be placed at arbitrary metric values on the curve by running the macro once for each point, like:
\begin{verbatim}
\pspolylineticks[ticksize=-6pt 6pt,Os=1.3]{Pt}{ dx dy add 3 div }{0}{1}%
\pspolylineticks[ticksize=-6pt 6pt,Os=2.4]{Pt}{ dx dy add 3 div }{0}{1}%
\end{verbatim}

You may also dispense entirely with the tick and use the macro to generate a node sequence 
that can be used to place other graphic objects. For example:
\begin{verbatim}
\pspolylineticks[ticksize=0pt 0pt]{Pt}{ dx dy add 3 div }{0}{3}%
%This defines nodes PtTick0..PtTick2
\multido{\iA=0+1}{3}{\psdot(PtTick\iA)}
\end{verbatim}


There is another way to define a metric function without using increments. If the keywork \Lkeyword{metricFunction} is set to \true, 
then the function you present as an argument to \Lcs{pspolylineticks} must be a function of 
$x$ and $y$ only, and must be designed to increase along the curve. It is useful only in 
those cases where, in essence, the increment function can be explicitly integrated. 
For example, in the elliptical motion of planets and comets around the sun, it is not hard 
to integrate the area function explicitly, and this provides a convenient metric, being proportional to time elapsed.

There is some useful information left in the log by these macros. 
They report the starting and ending values of the metric function, 
the the range of indices for the Tick related arrays.

\subsection{Examples}
The examples in this section make use of very recent (as of May, 2010) versions 
of \LPack{pstricks} and related packages. 
%If the {\tt pst-grapha} package is not available on CTAN,  download it from
%\begin{verbatim}
%http://math.ucsd.edu/~msharpe/pst-grapha.dmg
%\end{verbatim}

The first couple of examples are constructed entirely by hand, and have no interest 
other than to illustrate what is going on under the surface in the simplest case.

\begin{LTXexample}[pos=t]
\begin{pspicture}(-1,-1)(10,4)
\psline[showpoints=true](1,2)(4,0)(9,3)%
\uput[180](1,2){$s=0$}%
\uput[-90](4,0){$s=1$}%
\uput[0](9,3){$s=2$}%
\makeatletter% need to use macro names containing @
\pstVerb{tx@Dict begin %the pstricks dictionary
% declare arrays of length 3 (indices 0,1,2) to hold points, 
% differences and normals
/unitratiosq 1 def % yunit=xunit
/P.X [ 1 4 9 ] def %array of x coords
/P.Y [ 2 0 3 ] def %array of y coords
/PDelta.X [ 0 3 5 ] def % 3=4-1, 5=9-4, 0 never used
/PDelta.Y [ 0 -2 3 ] def % -2=0-2, 3=3-0, 0 never used
% normal to (3,-2) is (2,3), normal to (5,3) is (-3,5)
/PNormal.X [ 2 2 -3 ] def % index 0 =index 1
/PNormal.Y [ 3 3 5 ] def % index 0 = index 1
end }
\def\Ppointcount{2}
\makeatother
% make ticks using metric function with values 0,1,2
\pspolylineticks[Os=.5,Ds=1]{P}{1}{0}{2}
% ticks at s=0.5,1.5 (increment function =1)
\uput[-135](PTick0){$s=0.5$}%
\uput[-45](PTick1){$s=1.5$}%
\end{pspicture}
\end{LTXexample}

\clearpage
Now the same data, but with arc-length as metric. We change the last few lines:

\begin{LTXexample}[pos=t]
\begin{pspicture}(-1,-1)(10,4)
\psline[showpoints=true](1,2)(4,0)(9,3)%
%\uput[180](1,2){$s=0$}%
%\uput[-90](4,0){$s=1$}%
%\uput[0](9,3){$s=2$}%
\makeatletter% need to use macro names containing @
\pstVerb{tx@Dict begin %the pstricks dictionary
% declare arrays of length 3 (indices 0,1,2) to hold points, 
% differences and normals
/unitratiosq 1 def % yunit=xunit
/P.X [ 1 4 9 ] def %array of x coords
/P.Y [ 2 0 3 ] def %array of y coords
/PDelta.X [ 0 3 5 ] def % 3=4-1, 5=9-4, 0 never used
/PDelta.Y [ 0 -2 3 ] def % -2=0-2, 3=3-0, 0 never used
% normal to (3,-2) is (2,3), normal to (5,3) is (-3,5)
/PNormal.X [ 2 2 -3 ] def % index 0 =index 1
/PNormal.Y [ 3 3 5 ] def % index 0 = index 1
end }
\def\Ppointcount{2}
\makeatother
% make ticks using metric function arc-length
\pspolylineticks[Os=1,Ds=1]{P}{ ds }{0}{9}
% ticks at s=1,2... (increment function = distance)
\uput[-135](PTick0){$s=1$}%
\uput[-135](PTick1){$s=2$}%
\end{pspicture}
\end{LTXexample}

\clearpage
Once again the same data, but with metric equal to the x coordinate. Change the last few lines to:

\begin{LTXexample}[pos=t]
\begin{pspicture}(-1,-1)(10,4)
\psline[showpoints=true](1,2)(4,0)(9,3)%
%\uput[180](1,2){$s=0$}%
%\uput[-90](4,0){$s=1$}%
%\uput[0](9,3){$s=2$}%
\makeatletter% need to use macro names containing @
\pstVerb{tx@Dict begin %the pstricks dictionary
% declare arrays of length 3 (indices 0,1,2) to hold points, 
% differences and normals
/unitratiosq 1 def % yunit=xunit
/P.X [ 1 4 9 ] def %array of x coords
/P.Y [ 2 0 3 ] def %array of y coords
/PDelta.X [ 0 3 5 ] def % 3=4-1, 5=9-4, 0 never used
/PDelta.Y [ 0 -2 3 ] def % -2=0-2, 3=3-0, 0 never used
% normal to (3,-2) is (2,3), normal to (5,3) is (-3,5)
/PNormal.X [ 2 2 -3 ] def % index 0 =index 1
/PNormal.Y [ 3 3 5 ] def % index 0 = index 1
end }
\def\Ppointcount{2}
\makeatother
% make ticks using metric function arc-length
\pspolylineticks[metricFunction,Os=1,Ds=2]{P}{ x }{0}{5}
% ticks at x=1,3,... , start at tick index 0, draw 5 ticks
% the tick at s=1 has index 0
% ticks at s=1,2... (increment function = distance)
\uput[-135](PTick0){$s=1$}%
\uput[-135](PTick1){$s=3$}%
\end{pspicture}
\end{LTXexample}

\clearpage
The next example is a smooth path where subticks are drawn first, followed by major ticks. 
The metric is arc-length with initial value $s=1$.
\begin{LTXexample}[pos=t]
\begin{pspicture}(-1,-1)(10,4)
%\parametricplot[algebraic]{0}{9}{(t^2)/9 | 4*Ex(-t)*(1+t+(t^{2})/2+(t^{3})/6)}
\psparametricplot[algebraic]{0}{9}{t^2/9 | sin(t)+1}%
\pscurvepoints{0}{9}{(t^2)/9 | sin(t)+1}{P}%
% make ticks using  arc-length metric
\pspolylineticks[metricInitValue=1,ticksize=-2pt 2pt,Os=1,Ds=.2]{P}{ ds }{1}{56}%
\pspolylineticks[metricInitValue=1,Os=1,Ds=2]{P}{ ds }{0}{6}%
\multido{\iA=1+1,\iB=3+2}{5}{\Put{6pt;(PNormal\iA)}(PTick\iA){\tiny \iB}}
%\nodexn{(PTick\iA)+(10pt;{(PNormal\iA)})}{Q}\rput(Q){\tiny \iB}}%
%\multido{\iA=1+1,\iB=3+2}{5}{\uput{6pt}[{(PNormal\iA)}](PTick\iA){\iB}}%
% ticks at x=1,3,... , start at tick index 0, draw 5 ticks
% the tick at s=1 has index 0
% ticks at s=1,2... (increment function = distance)
\end{pspicture}
\end{LTXexample}

\clearpage
Suppose for the next example that we have an ellipse $x^2/a^2+y^2/b^2=1$ ($a>b$) with 
eccentricity $\epsilon=(1-b^2/a^2)^{1/2}$. With planetary motion in mind, a natural metric 
for the ellipse is the area swept out by the radial line from the focus $(\epsilon a,0)$ 
starting from $(a,0)$ around to an arbitrary location $(x,y)$, where $y>0$, as this quantity 
is proportional to the time elapsed since perihelion. A routine calculation gives the following formula:
\[A=\frac{ab}{2}\arccos\bigg(\frac{x}{a}\bigg)-\frac{\epsilon a y}{2}.\]
Remembering that PostScript's {\tt acos} gives  its result in degrees, not radians, we have the 
following, drawn for the case $a=4$, $b=3$.

\begin{LTXexample}[pos=t]
\begin{pspicture}(-4.5,-.5)(4.5,3.5)
\pstVerb{ /smajor 4 def /sminor 3 def % define semimajor, semiminor 
/ecc 1 sminor smajor div dup mul sub sqrt def % compute eccentricity
/ab smajor sminor mul 2 div def %first coeff
/ea smajor ecc mul 2 div def }% second coeff
\psparametricplot[algebraic]{0}{3.142}{smajor*cos(t) | sminor*sin(t)}%
\pscurvepoints{0}{3.142}{smajor*cos(t) | sminor*sin(t)}{P}%
\pspolylineticks[metricFunction,Ds=2,ticksize=-1.5pt 0]{P}{ ab x smajor div acos %
180 div PI mul mul  ea y mul sub }{1}{9}%
\pnode(! ecc smajor mul 0){S}% focus
\psline[linecolor=lightgray](S)(!smajor 0)%
\multido{\i=1+1}{9}{\psline[linecolor=lightgray](S)(PTick\i)}
\psdot(S)
\end{pspicture}
\end{LTXexample}

\clearpage
The next examples works without visible ticks, using the macros to construct nodes at which other objects will be placed.

\begin{LTXexample}[pos=t]
\begin{pspicture}(-1,-1)(10,4)
\psparametricplot[algebraic]{0}{9}{t| 3*Ex(-t)*(1+t+t^2/2+t^3/6)}
\pscurvepoints{0}{9}{t| 3*Ex(-t)*(1+t+t^2/2+t^3/6)}{P}%
\pspolylineticks[Os=1,Ds=1,ticksize=0 0]{P}{ ds }{0}{9}%
\multido{\i=0+1}{9}{\psdot[dotscale=1.5,dotstyle=o](PTick\i)}%
% ticks at s=1,2,... , start at tick index 0, set 9 ticks
% the tick at s=1 has index 0
% ticks at s=1,2... (increment function = distance)
\multido{\i=0+3}{3}{\Put[rot=(PTangent\i)]{7pt;(PNormal\i)}(PTick\i){PTick\i}}%
\uput[-135](PTick1){$s=2$}%
\end{pspicture}
\end{LTXexample}

This variant also has no visible ticks, but makes a color gradient along the curve based on arc-length from the start.

\begin{LTXexample}[pos=t]
\begin{pspicture}(-1,-1)(10,4)
\psparametricplot[plotpoints=200,linecolor=white]{0}{360}{ t cos 1 add 4 mul t 1 add 20 div ln 2 div 1 add }
\pscurvepoints[plotpoints=200]{0}{360}{ t cos 1 add 4 mul t 1 add 20 div ln 2 div 1 add }{P}%
\pspolylineticks[Os=0,Ds=.2,ticksize=0 0]{P}{ ds }{0}{90}%
\definecolorseries{ctest}{hsb}{last}{green}{violet}
\resetcolorseries[88]{ctest}%
\multido{\iA=0+1,\iB=1+1}{87}{\psline[linewidth=2pt,linecolor=ctest!![\iB](PTick\iA)(PTick\iB)}%
\end{pspicture}
\end{LTXexample}

\clearpage
Here is a another variant of this technique which allows arrows to be placed at locations 
on the curve where the metric takes particular values.

\begin{LTXexample}[pos=t]
\begin{pspicture}(-1,-1)(10,4.5)
\psparametricplot[plotpoints=100]{0}{360}{t cos 1 add 5 mul t sin 1 add 2 mul}
\pscurvepoints[plotpoints=100]{0}{360}{t cos 1 add 5 mul t sin 1 add 2 mul}{P}%
\pspolylineticks[Os=0,Ds=2.3,ticksize=0 0]{P}%
{ ds }{0}{10}% distance
\multido{\i=0+1}{10}{\psrline[arrows=->,arrowscale=1.5](PTick\i)(2pt;{(PTangent\i)})}%
\end{pspicture}
\end{LTXexample}

\section{Troubleshooting}
If you get PostScript errors when you process your file, the  most likely culprit is the 
function you specified to define the metric. There are some  things to look out for:
\begin{itemize}
\item If \Lkeyword{metricFunction}, the function you specify in PostScript code must 
involve only {\tt x} and {\tt y}, and must leave exactly one real value on the stack as a result of 
substituting specific values for {\tt x} and {\tt y}. The function must be strictly increasing on the curve.
\item If \Lkeyword{metricFunction}=\false (the default), the function you specify in PostScript 
code must involve only the variables {\tt x}, {\tt y}, {\tt dx}, {\tt dy}, {\tt ds} (where {\tt ds} 
is defined to be the arc-length element {\tt dx dup mul dy dup mul add sqrt}, and must leave exactly 
one strictly positive real value on the stack when specific values are substituted for those variables. 
The constant function {\tt 1} gives equal weight to each segment in the curve, so in effect it gives 
you the original parametrization, up to a constant factor.
\item If the function you specify in \Lcs{parametricplot} and \Lcs{pscurvepoints} is \Lkeyword{algebraic}, 
make sure you follow precisely the syntax it understands. In complex cases, PostScript may be the safer solution.
\item It is unwise to use a different resolution for \Lcs{psparametricplot} and \Lcs{pscurvepoints}. 
The default value of \Lkeyword{plotpoints}=50 is marginal except for modest curve segments, and 200 should 
suffice for most smooth curves.
\end{itemize}


%--------------------------------------------------------------------------------------
\section{Transparent colors}
%--------------------------------------------------------------------------------------

Transparency is now part of the main \LPack{pstricks} package.
But pay attention, the names and syntax have changed and you need
to run \Lprog{ps2pdf} with the option
\Loption{-dCompatibilityLevel}=1.4.


%--------------------------------------------------------------------------------------
\section{,,Manipulating transparent colors''}
%--------------------------------------------------------------------------------------

\LPack{pstricks-add} supports real transparency and a simulated one with hatch lines:
\begin{lstlisting}
\def\defineTColor{\@ifnextchar[{\defineTColor@i}{\defineTColor@i[]}}
\def\defineTColor@i[#1]#2#3{%     transparency "Colors"
  \newpsstyle{#2}{%
     fillstyle=vlines,hatchwidth=0.1\pslinewidth,
     hatchsep=1\pslinewidth,hatchcolor=#3,#1%
  }%
}
\defineTColor{TRed}{red}
\defineTColor{TGreen}{green}
\defineTColor{TBlue}{blue}
\end{lstlisting}

There are three predefined "'transparent"` colors \verb+TRed+,
\verb+TGreen+, \verb+TBlue+. They are used as \PST{} styles and
not as colors:

\bgroup
\begin{LTXexample}[pos=t,preset=\centering]
\begin{pspicture}(-3,-5)(5,5)
\psframe(-1,-3)(5,5) % objet de base
\psrotate(2,-2){15}{%
  \psframe[style=TRed](-1,-3)(5,5)}
\psrotate(2,-2){30}{%
  \psframe[style=TGreen](-1,-3)(5,5)}
\psrotate(2,-2){45}{%
  \psframe[style=TBlue](-1,-3)(5,5)}
\psframe[linewidth=3pt](-1,-3)(5,5)
\psdots[dotstyle=+,dotangle=45,dotscale=3](2,-2) % centre de la rotation
\end{pspicture}
\end{LTXexample}
\egroup

%--------------------------------------------------------------------------------------
\section{Calculated colors}
%--------------------------------------------------------------------------------------
The \verb+xcolor+ package (version 2.6) has a new feature for defining colors:
\begin{lstlisting}[style=syntax]
  \definecolor[ps]{<name>}{<model>}{< PS code >}
\end{lstlisting}

\verb+model+ can be one of the color models, which \PS will
understand, e.g. \verb+rgb+. With this definition the color is
calculated on the \PS side.
\begin{LTXexample}[pos=t,preset=\centering]
\definecolor[ps]{bl}{rgb}{tx@addDict begin  Red Green Blue end}%
\psset{unit=1bp}
\begin{pspicture}(0,-30)(400,100)
\multido{\iLAMBDA=0+1}{400}{%
  \pstVerb{
    \iLAMBDA\space 379 add dup /lambda exch def
    tx@addDict begin  wavelengthToRGB end
  }%
  \psline[linecolor=bl](\iLAMBDA,0)(\iLAMBDA,100)%
}
\psaxes[yAxis=false,Ox=350,dx=50bp,Dx=50]{->}(-29,-10)(420,100)
\uput[-90](420,-10){$\lambda$[\textsf{nm}]}
\end{pspicture}
\end{LTXexample}


\begin{center}
\newcommand{\Touch}{%
\psframe[linestyle=none,fillstyle=solid,fillcolor=bl,dimen=middle](0.1,0.75)}
\definecolor[ps]{bl}{rgb}{tx@addDict begin Red Green Blue end}%
% Echelle 1cm <-> 40 nm
%         1 nm <-> 0.025 cm
\psframebox[fillstyle=solid,fillcolor=black]{%
\begin{pspicture}(-1,-0.5)(12,1.5)
\multido{\iLAMBDA=380+2}{200}{%
  \pstVerb{
    /lambda \iLAMBDA\space def
    lambda
    tx@addDict begin  wavelengthToRGB end
  }%
 \rput(! lambda 0.025 mul 9.5 sub 0){\Touch}
}
\multido{\n=0+1,\iDiv=380+40}{11}{%
    \psline[linecolor=white](\n,0.1)(\n,-0.1)
    \uput[270](\n,0){\textbf{\white\iDiv}}}
    \psline[linecolor=white]{->}(11,0)
    \uput[270](11,0){\textbf{\white$\lambda$(nm)}}
\end{pspicture}}

\psframebox[fillstyle=solid,fillcolor=black]{%
\begin{pspicture}(-1,-0.5)(12,1)
  \pstVerb{
    /lambda 656 def
    lambda
    tx@addDict begin  wavelengthToRGB end
  }%
 \rput(! 656 0.025 mul 9.5 sub 0){\Touch}
  \pstVerb{
    /lambda 486 def
    lambda
    tx@addDict begin  wavelengthToRGB end
  }%
 \rput(! 486 0.025 mul 9.5 sub 0){\Touch}
   \pstVerb{
    /lambda 434 def
    lambda
    tx@addDict begin  wavelengthToRGB end
  }%
 \rput(! 434 0.025 mul 9.5 sub 0){\Touch}
  \pstVerb{
    /lambda 410 def
    lambda
    tx@addDict begin  wavelengthToRGB end
  }%
 \rput(! 410 0.025 mul 9.5 sub 0){\Touch}
\multido{\n=0+1,\iDiv=380+40}{11}{%
    \psline[linecolor=white](\n,0.1)(\n,-0.1)
    \uput[270](\n,0){\textbf{\white\iDiv}}}
    \psline[linecolor=white]{->}(11,0)
    \uput[270](11,0){\textbf{\white$\lambda$(nm)}}
\end{pspicture}}

\Index{Spectrum} of \Index{hydrogen} emission (Manuel Luque)
\end{center}

\begin{lstlisting}
\newcommand\Touch{%
\psframe[linestyle=none,fillstyle=solid,fillcolor=bl,dimen=middle](0.1,0.75)}
\definecolor[ps]{bl}{rgb}{tx@addDict begin Red Green Blue end}%
% Echelle 1cm <-> 40 nm
%         1 nm <-> 0.025 cm
\psframebox[fillstyle=solid,fillcolor=black]{%
\begin{pspicture}(-1,-0.5)(12,1.5)
\multido{\iLAMBDA=380+2}{200}{%
  \pstVerb{
    /lambda \iLAMBDA\space def
    lambda
    tx@addDict begin  wavelengthToRGB end
  }%
 \rput(! lambda 0.025 mul 9.5 sub 0){\Touch}
}
\multido{\n=0+1,\iDiv=380+40}{11}{%
    \psline[linecolor=white](\n,0.1)(\n,-0.1)
    \uput[270](\n,0){\textbf{\white\iDiv}}}
    \psline[linecolor=white]{->}(11,0)
    \uput[270](11,0){\textbf{\white$\lambda$(nm)}}
\end{pspicture}}

\psframebox[fillstyle=solid,fillcolor=black]{%
\begin{pspicture}(-1,-0.5)(12,1)
  \pstVerb{
    /lambda 656 def
    lambda
    tx@addDict begin  wavelengthToRGB end
  }%
 \rput(! 656 0.025 mul 9.5 sub 0){\Touch}
  \pstVerb{
    /lambda 486 def
    lambda
    tx@addDict begin  wavelengthToRGB end
  }%
 \rput(! 486 0.025 mul 9.5 sub 0){\Touch}
   \pstVerb{
    /lambda 434 def
    lambda
    tx@addDict begin  wavelengthToRGB end
  }%
 \rput(! 434 0.025 mul 9.5 sub 0){\Touch}
  \pstVerb{
    /lambda 410 def
    lambda
    tx@addDict begin  wavelengthToRGB end
  }%
 \rput(! 410 0.025 mul 9.5 sub 0){\Touch}
\multido{\n=0+1,\iDiv=380+40}{11}{%
    \psline[linecolor=white](\n,0.1)(\n,-0.1)
    \uput[270](\n,0){\textbf{\white\iDiv}}}
    \psline[linecolor=white]{->}(11,0)
    \uput[270](11,0){\textbf{\white$\lambda$(nm)}}
\end{pspicture}}

Spectrum of hydrogen emission (Manuel Luque)
\end{lstlisting}



%--------------------------------------------------------------------------------------
\section{Gouraud shading}
%--------------------------------------------------------------------------------------
\begin{quotation}
\Index{Gouraud} shading is a method used in computer graphics to simulate the differing effects of
light and colour across the surface of an object. In practice, Gouraud shading is used to
achieve smooth lighting on low-polygon surfaces without the heavy computational requirements
of calculating lighting for each pixel. The technique was first presented by Henri Gouraud in 1971.\\
~\hfill{\small \url{http://www.wikipedia.org}}
\end{quotation}

PostScript level 3 supports this kind of shading and it can only
be seen with Acroread 7 or later. The syntax is easy:

\begin{lstlisting}[style=syntax]
  \psGTriangle(x1,y1)(x2,y2)(x3,y3){color1}{color2}{color3}
\end{lstlisting}

\psset{unit=0.75cm}

\begin{LTXexample}[pos=t,preset=\centering]
\begin{pspicture}(0,-.25)(10,10)
  \psGTriangle(0,0)(5,10)(10,0){red}{green}{blue}
\end{pspicture}
\end{LTXexample}

\begin{LTXexample}[pos=t,preset=\centering]
\begin{pspicture}(0,-.25)(10,10)
  \psGTriangle*(0,0)(9,10)(10,3){black}{white!50}{red!50!green!95}
\end{pspicture}
\end{LTXexample}

\begin{LTXexample}[pos=t,preset=\centering]
\begin{pspicture}(0,-.25)(10,10)
  \psGTriangle*(0,0)(5,10)(10,0){-red!100!green!84!blue!86}
                               {-red!80!green!100!blue!40}
                               {-red!60!green!30!blue!100}
\end{pspicture}
\end{LTXexample}

\begin{LTXexample}[pos=t,preset=\centering]
\definecolor{rose}{rgb}{1.00, 0.84, 0.88}
\definecolor{vertpommepasmure}{rgb}{0.80, 1.0, 0.40}
\definecolor{fushia}{rgb}{0.60, 0.30, 1.0}
\begin{pspicture}(0,-.25)(10,10)
  \psGTriangle(0,0)(5,10)(10,0){rose}{vertpommepasmure}{fushia}
\end{pspicture}
\end{LTXexample}

\section{Internal color macros}
The internal macros \Lcs{pswavelengthToRGB} and \Lcs{pswavelengthToRGB} can be used for own purposed.
They are defines as follows:

\begin{lstlisting}
\def\pswavelengthToGRAY{ tx@addDict begin wavelengthToGRAY end }
\def\pswavelengthToRGB{ tx@addDict begin wavelengthToRGB Red Green Blue end }
\end{lstlisting}

both macros leave the value(s) on the stack which then can be used for further
manipulating or setting the color with \Lps{setgray} or \Lps{setrgbcolor}. 
For an example see Section~\ref{sec:psMatrix}.

\appendix


%--------------------------------------------------------------------------------------
\clearpage
\section{\nxLcs{resetOptions}}
%--------------------------------------------------------------------------------------

Sometimes it is difficult to know what options, which are changed
inside a long document, are different to the default ones. With
this macro all options belonging to \LPack{pst-plot} can be reset.
This refers to all options of the packages \LPack{pstricks},
\LPack{pst-plot} and \LPack{pst-node}.



%--------------------------------------------------------------------------------------
\section{PostScript}
%--------------------------------------------------------------------------------------

\Index{PostScript} uses the stack system and the LIFO system, "'Last In, First Out"`.

\newlength{\Li}\settowidth{\Li}{Function}
\begin{table}[htbp]
\caption{Some primitive PostScript macros}\label{tab:primpost}
\centering
\ttfamily
    \begin{tabular}{@{} l | r@{ $\rightarrow$ } l @{}}\hline
    \multirow{2}{\Li}{\normalfont\emph{Function}} & \multicolumn{2}{ c }{\normalfont\emph{Meaning}}\\
    &\normalfont\emph{on stack before} & \normalfont\emph{after}\\\hline
    \Lps{add} & $x\quad y$&$x+y$\\ 
    \Lps{sub} & $x\quad y$&$x-y$\\ 
    \Lps{mul} & $x\quad y$&$x\times y$\\ 
    \Lps{div} & $x\quad y$&$x\div y$\\ 
    \Lps{sqrt} & $x$&$\sqrt{x}$\\ 
    \Lps{abs} & $x$&$|x|$\\ 
    \Lps{neg} & $x$&$-x$\\ 
    \Lps{cos} & $x$&$\cos(x)$ ($x$ in degrees)\\ 
    \Lps{sin} & $x$&$\sin(x)$ ($x$  in degrees)\\ 
    \Lps{tan} & $x$&$\tan(x)$ ($x$  in degrees)\\ 
    \Lps{atan} & $y\quad x$&$\angle{(\vec{Ox};\vec{OM})}$ (in degrees of $M(x,y)$)\\ 
    \Lps{ln} & $x$&$\ln(x)$\\ 
    \Lps{log} & $x$&$\log(x)$\\ 
    \Lps{array} & $n$&\normalfont$v$ (of dimension $n$)\\ 
    \Lps{aload} & $v$&$x_1\quad x_2\quad \cdots\quad x_n\quad v$\\ 
    \Lps{astore} & $x_1\quad x_2\quad \cdots\quad x_n\quad v$ & $[v]$\\ 
    \Lps{pop} & $x$ & --\\ 
    \Lps{dup} & $x$ & $x\quad x$ \\\hline
%    \Lps{roll} & $x_1\quad x_2\quad \cdots\quad x_n\quad n p$ &\\\hline
  \end{tabular}
\end{table}


\clearpage
\section{List of all optional arguments for \texttt{pstricks-add}}

\xkvview{family=pstricks-add,columns={key,type,default}}





\nocite{*}
\bgroup
\RaggedRight
\bibliographystyle{plain}
\bibliography{pstricks-add-doc}
\egroup

\printindex




\end{document}


\usepackage[utf8]{inputenc}
\usepackage{pstricks-add}
\let\pstricksaddFV\fileversion
\usepackage{pst-eucl,pst-fun,pst-func,multirow}
\usepackage{pifont}
\let\belowcaptionskip\abovecaptionskip
%
\def\textat{\char064}%
\newdimen\fullWidth
\makeatletter
\renewcommand*\l@section{\@dottedtocline{1}{2em}{2.3em}}
\renewcommand*\l@subsection{\@dottedtocline{2}{3.8em}{3.2em}}
\renewcommand*\l@subsubsection{\@dottedtocline{3}{7.0em}{4.1em}}
\renewcommand*\l@paragraph{\@dottedtocline{4}{10em}{5em}}
\makeatother
\lstset{explpreset={pos=l,width=-99pt,overhang=0pt,hsep=\columnsep,vsep=\bigskipamount,rframe={}},
    escapechar=§}

\def\bgImage{\psset{unit=1.5}
\begin{pspicture}(-3,-3)(3,3)
\psChart[userColor={red!30,green!30,blue!40,gray,cyan!50,
    magenta!60,cyan},chartSep=30pt,shadow=true,shadowsize=5pt]{34.5,17.2,20.7,15.5,5.2,6.9}{6}{2}
\psset{nodesepA=5pt,nodesepB=-10pt}
\ncline{psChartO1}{psChart1}\nput{0}{psChartO1}{1000 (34.5\%)}
\ncline{psChartO2}{psChart2}\nput{150}{psChartO2}{500 (17.2\%)}
\ncline{psChartO3}{psChart3}\nput{-90}{psChartO3}{600 (20.7\%)}
\ncline{psChartO4}{psChart4}\nput{0}{psChartO4}{450 (15.5\%)}
\ncline{psChartO5}{psChart5}\nput{0}{psChartO5}{150 (5.2\%)}
\ncline{psChartO6}{psChart6}\nput{0}{psChartO6}{200 (6.9\%)}
\bfseries%
\rput(psChartI1){Taxes}\rput(psChartI2){Rent}\rput(psChartI3){Bills}
\rput(psChartI4){Car}\rput(psChartI5){Gas}\rput(psChartI6){Food}
\end{pspicture}}

\begin{document}
\title{\texttt{pstricks-add}\\additionals Macros for \texttt{pstricks}\\
    \small v.\pstricksaddFV}
%\docauthor{Herbert Vo\ss}
\author{Dominique Rodriguez\\Michael Sharpe\\Herbert Vo\ss}
\date{\today}

\maketitle

\fullWidth=\linewidth
\advance\fullWidth by \marginparsep
\advance\fullWidth by \marginparwidth


\begin{abstract}
This version of \verb+pstricks-add+ needs \verb+pstricks.tex+
version >1.04 from June 2004, otherwise the additional macros may
not work as expected. The ellipsis material and the option
\verb+asolid+ (renamed to \verb+eofill+) are
\index{fillstyle!eofill@\texttt{eofill}} now part of the new
\verb+pstricks.tex+ package, available on CTAN. \LPack{pstricks-add} will for ever be
an experimental and dynamical package, try it at your own risk.

\begin{itemize}
\item It is important to load \LPack{pstricks-add} as the \textbf{last} PSTricks related package, otherwise
a lot of the macros won't work in the expected way.
\item \LPack{pstricks-add} uses the extended version of the keyval package. So be sure that
you have installed \LPack{pst-xkey} which is part of the
\LPack{xkeyval}-package, and that all packages that use the old
keyval interface are loaded \textbf{before} the
\LPack{xkeyval}.\cite{xkeyval}
\item the option \Lkeyword{tickstyle} from \LPack{pst-plot} is no longer supported; use \Lkeyword{ticksize} instead.
\item the option \Lkeyword{xyLabel} is no longer supported; use the option \Lkeyword{labelFontSize} instead.
\item if \LPack{pstricks-add} is loaded together with the package  \LPack{pst-func} then  \Lkeyword{InsideArrow}
    of the \Lcs{psbezier} macro doesn't work!
\end{itemize}

\vfill
\noindent
Thanks to:  
Hendri Adriaens;
Stefano Baroni;
Martin Chicoine;
Gerry Coombes;
Ulrich Dirr;
Christophe Fourey;
Hubert G\"a\ss lein;
J\"urgen Gilg;
Denis Girou;
Pablo Gonzáles;
Peter Hutnick;
Christophe Jorssen;
Uwe Kern;
Manuel Luque;
Jens-Uwe Morawski;
Tobias N\"ahring;
Rolf Niepraschk;
Alan Ristow;
Christine R\"omer;
Arnaud Schmittbuhl;
John Smith;
Timothy Van Zandt
\end{abstract}

\clearpage
\tableofcontents


\clearpage

\section{\nxLcs{psGetSlope} and \nxLcs{psGetDistance}}
%--------------------------------------------------------------------------------------

\begin{BDef}
\Lcs{psGetSlope}\coord1\coord2\Lcs{\Larga{macro}}\\
\Lcs{psGetDistance}\coord1\coord2\Lcs{\Larga{macro}}
\end{BDef}

\begin{LTXexample}[width=4cm]
\psGetSlope(-2,1)(3,1)\SlopeVal \SlopeVal \quad
\psGetDistance(-2,1)(3,1)\DVal \DVal\\
\psGetSlope(-2,1)(-3,-1)\SlopeVal \SlopeVal\quad
\psGetDistance(-2,1)(-3,-1)\DVal \DVal\\
\psGetSlope(-2,0)(3,-1)\SlopeVal \SlopeVal\quad
\psGetDistance(-2,0)(3,-1)\DVal \DVal\\
\psGetSlope(-2111,-12)(3,1)\SlopeVal \SlopeVal\quad
%\psGetDistance(-2111,-12)(3,1)\DVal ==> Overflow!
\end{LTXexample}

\clearpage

%--------------------------------------------------------------------------------------
\section{"`Handmade"' lines :-)}
%--------------------------------------------------------------------------------------

\begin{BDef}
\Lcs{pslineByHand}\OptArgs\coord1\coord2\coord3 \ldots
\end{BDef}

\begin{LTXexample}[width=0.4\linewidth]
\begin{pspicture}(4,6)
\psset{unit=2cm}
  \pslineByHand[linecolor=red](0,0)(0,2)(2,2)(2,0)(0,0)(2,2)(1,3)(0,2)(2,0)
\end{pspicture}
\end{LTXexample}

\iffalse
  \pslineByHand( 1.20, 1.50)( 1.20, 1.51)( 1.20, 1.53)( 1.20, 1.54)( 1.19, 1.55)( 1.19, 1.56)
    ( 1.19, 1.57)( 1.18, 1.59)( 1.18, 1.60)( 1.17, 1.61)( 1.16, 1.62)( 1.15, 1.63)( 1.15, 1.64)
    ( 1.14, 1.65)( 1.13, 1.65)( 1.12, 1.66)( 1.11, 1.67)( 1.10, 1.68)( 1.09, 1.68)( 1.07, 1.69)
    ( 1.06, 1.69)( 1.05, 1.69)( 1.04, 1.70)( 1.03, 1.70)( 1.01, 1.70)( 1.00, 1.70)( 0.99, 1.70)
    ( 0.97, 1.70)( 0.96, 1.70)( 0.95, 1.69)( 0.94, 1.69)( 0.93, 1.69)( 0.91, 1.68)( 0.90, 1.68)
    ( 0.89, 1.67)( 0.88, 1.66)( 0.87, 1.65)( 0.86, 1.65)( 0.85, 1.64)( 0.85, 1.63)( 0.84, 1.62)
    ( 0.83, 1.61)( 0.82, 1.60)( 0.82, 1.59)( 0.81, 1.57)( 0.81, 1.56)( 0.81, 1.55)( 0.80, 1.54)
    ( 0.80, 1.53)( 0.80, 1.51)( 0.80, 1.50)( 0.80, 1.49)( 0.80, 1.47)( 0.80, 1.46)( 0.81, 1.45)
    ( 0.81, 1.44)( 0.81, 1.43)( 0.82, 1.41)( 0.82, 1.40)( 0.83, 1.39)( 0.84, 1.38)( 0.85, 1.37)
    ( 0.85, 1.36)( 0.86, 1.35)( 0.87, 1.35)( 0.88, 1.34)( 0.89, 1.33)( 0.90, 1.32)( 0.91, 1.32)
    ( 0.93, 1.31)( 0.94, 1.31)( 0.95, 1.31)( 0.96, 1.30)( 0.97, 1.30)( 0.99, 1.30)( 1.00, 1.30)
    ( 1.01, 1.30)( 1.03, 1.30)( 1.04, 1.30)( 1.05, 1.31)( 1.06, 1.31)( 1.07, 1.31)( 1.09, 1.32)
    ( 1.10, 1.32)( 1.11, 1.33)( 1.12, 1.34)( 1.13, 1.35)( 1.14, 1.35)( 1.15, 1.36)( 1.15, 1.37)
    ( 1.16, 1.38)( 1.17, 1.39)( 1.18, 1.40)( 1.18, 1.41)( 1.19, 1.43)( 1.19, 1.44)( 1.19, 1.45)
    ( 1.20, 1.46)( 1.20, 1.47)( 1.20, 1.49)( 1.20, 1.50)
\fi

\begin{LTXexample}[pos=t]
\begin{pspicture}(\linewidth,3)
\multido{\rA=0.00+0.25}{12}{\pslineByHand[linecolor=blue](0,\rA)(\linewidth,\rA)}
\end{pspicture}
\end{LTXexample}

The amplitude and the width can be changed by the optional arguments \Lkeyword{varsteptol} and
\Lkeyword{VarStepEpsilon}. Both are preset to \verb+VarStepEpsilon=2,varsteptol=0.8+.


\begin{LTXexample}[pos=t]
\begin{pspicture}(\linewidth,3)
\multido{\rA=0.00+0.25}{12}{%
  \pslineByHand[linecolor=blue,VarStepEpsilon=4,varsteptol=2](0,\rA)(\linewidth,\rA)}
\end{pspicture}
\end{LTXexample}

\clearpage

%--------------------------------------------------------------------------------------
\section{\nxLcs{rmultiput}: a multiple \nxLcs{rput}}
%--------------------------------------------------------------------------------------
\verb+PSTricks+ already has a \Lcs{multirput}, which puts a box n
times with a difference of $dx$ and $dy$ relative to each other.
It is not possible to put it with a different distance from one
point to the next. This is possible with \Lcs{rmultiput}:

\begin{BDef}
\LcsStar{rmultiput}\OptArgs\Largb{any material}\coord1\coord2\ldots\Largr{\coord{n}}
\end{BDef}

\begin{LTXexample}[width=6.2cm]
\psset{unit=0.75}
\begin{pspicture}(-4,-4)(4,4)
\rmultiput[rot=45]{\red\psscalebox{3}{\ding{250}}}%
    (-2,-4)(-2,-3)(-3,-3)(-2,-1)(0,0)(1,2)(1.5,3)(3,3)
\rmultiput[rot=90,ref=lC]{\blue\psscalebox{2}{\ding{253}}}%
    (-2,2.5)(-2,2.5)(-3,2.5)(-2,1)(1,-2)(1.5,-3)(3,-3)
\psgrid[subgriddiv=0,gridcolor=lightgray]
\end{pspicture}
\end{LTXexample}

\clearpage


%--------------------------------------------------------------------------------------
\section{\nxLcs{psVector}: Drawing relative vector lines}
%--------------------------------------------------------------------------------------

The new macros \Lcs{psStartPoint} and \Lcs{psVector} allow to draw a series of
vectors which start point refers to the endpoint of the last drawn vector. The 
coordinates of the endpoint are \emph{always} interpreted relative to the last
the vector. The first vector refers to the coordinates set by \Lcs{psStartPoint}.
With the boolean argument one can draw the horizontal angle of the vector.

The style of the angle arc is saved in \Lkeyval{psMarkAngleStyle} and the style
for the horizontal line in \Lkeyval{psMarkAngleLineStyle} and preset to

\begin{lstlisting}
\newpsstyle{psMarkAngleStyle}{arrows=->,arrowsize=4pt}
\newpsstyle{psMarkAngleLineStyle}{linestyle=dotted}
\end{lstlisting}


\begin{pspicture}[showgrid](10,10)
 \psStartPoint(1,1)
 \psVector(3;30)\psVector(4;60)\psVector[linecolor=red](3;10)
 \psVector[linestyle=dashed](4;110)
 \psStartPoint(1,1)\psset{markAngle}
 \psVector[linestyle=dashed](4;110)\psVector[linecolor=red](3;10)
 \psVector(4;60)\psVector(3;30)
\end{pspicture}

\begin{lstlisting}
\begin{pspicture}[showgrid](10,10)
 \psStartPoint(1,1)
 \psVector(3;30)\psVector(4;60)\psVector[linecolor=red](3;10)
 \psVector[linestyle=dashed](4;110)
 \psStartPoint(1,1)\psset{markAngle}
 \psVector[linestyle=dashed](4;110)\psVector[linecolor=red](3;10)
 \psVector(4;60)\psVector(3;30)
\end{pspicture}
\end{lstlisting}

All end points of the vectors are saved in node names with the preset name \verb=Vector#=,
where \# is the consecutive  number of the nodes. \verb=Vector0= ist the starting point of
the first \Lcs{psVector}. With the macro \Lcs{psStartPoint} one can set the starting point and
with optional argument the name of the nodes. \verb=Vector3= is the default node name of
the endpoint of the third vector or the name of the starting point of the forth vector.

\begin{BDef}
\Lcs{psStartPoint}\OptArg{node basename}\Largr{$x$,$y$}
\end{BDef}

\begin{pspicture}[showgrid,linewidth=1pt](10,10.4)
 \psStartPoint[A](1,1)% nodes have the base name A
 \psVector(3;30)\psVector(4;60)\psVector[linecolor=red](3;10)
 \psVector[linestyle=dashed](4;110)
 \psline{->}(A0)(A4)
 \psStartPoint[B](1,1)\psset{markAngle}% nodes have the base name B
 \psVector[linestyle=dashed](4;110)
 \psVector[linecolor=red](3;10)
 \psVector(4;60)\psVector(3;30)
 \psline[arrows=-D>,arrowscale=2,linewidth=1.5pt,linecolor=red](B2)(A2)
 \psline[arrows=-D>,arrowscale=2,linewidth=1.5pt,linecolor=blue](A3)(B3)
 \multido{\iA=0+1}{5}{\uput[0](A\iA){A\iA}\uput[180](B\iA){B\iA}}
\end{pspicture}

\begin{lstlisting}
\begin{pspicture}[showgrid,linewidth=1pt](10,10.4)
 \psStartPoint[A](1,1)% nodes have the base name A
 \psVector(3;30)\psVector(4;60)\psVector[linecolor=red](3;10)
 \psVector[linestyle=dashed](4;110)
 \psline{->}(A0)(A4)
 \psStartPoint[B](1,1)\psset{markAngle}% nodes have the base name B
 \psVector[linestyle=dashed](4;110)
 \psVector[linecolor=red](3;10)
 \psVector(4;60)\psVector(3;30)
 \psline[arrows=-D>,arrowscale=2,linewidth=1.5pt,linecolor=red](B2)(A2)
 \psline[arrows=-D>,arrowscale=2,linewidth=1.5pt,linecolor=blue](A3)(B3)
 \multido{\iA=0+1}{5}{\uput[0](A\iA){A\iA}\uput[180](B\iA){B\iA}
 \end{pspicture}
\end{lstlisting}

\clearpage


%--------------------------------------------------------------------------------------
\section{\nxLcs{psCircleTangents}: Calculating tangent lines of circles}
%--------------------------------------------------------------------------------------

The macro calculates the points on a circle where tangent lines from another
point or another circle are drawn.

\begin{BDef}
\Lcs{psCircleTangents}\Largr{$x1,y1$}\Largr{$x2,y2$}\Largb{Radius}\\
\Lcs{psCircleTangents}\Largr{$x1,y1$}\Largb{Radius}\Largr{$x2,y2$}\Largb{Radius}
\end{BDef}

In the first case the coordinates of a point and the center and the radius
of a circle must be given. The names of the calculates node names are \verb=CircleT1=
and \verb=CircleT2=.

\bigskip
\begin{pspicture}[showgrid](0,3)(10,10)
\psdot(2,4)\pscircle(7,7){2}
\psCircleTangents(2,4)(7,7){2}
\pcline[nodesep=-1cm,linecolor=blue](2,4)(CircleT1)
\pcline[nodesep=-1cm,linecolor=blue](2,4)(CircleT2)
\psdots(CircleT1)(CircleT2)
\uput[-80](CircleT1){T1}\uput[115](CircleT2){T2}
\end{pspicture}


\begin{lstlisting}
\begin{pspicture}[showgrid](0,3)(10,10)
\psdot(2,4)\pscircle(7,7){2}
\psCircleTangents(2,4)(7,7){2}
\pcline[nodesep=-1cm,linecolor=blue](2,4)(CircleT1)
\pcline[nodesep=-1cm,linecolor=blue](2,4)(CircleT2)
\psdots(CircleT1)(CircleT2)
\uput[-80](CircleT1){T1}\uput[115](CircleT2){T2}
\end{pspicture}
\end{lstlisting}

\bigskip
When using the other variant of the macro two circles must be given. The macro then defines
ten nodes, named \verb=CircleTC1= and \verb=CircleTC2= for the two intersection points,
 \verb=CircleTO1=, \verb=CircleTO2=, \verb=CircleTO3=, and \verb=CircleTO4= for the four
 nodes of the outer tangent lines and 
  \verb=CircleTI1=, \verb=CircleTI2=, \verb=CircleTI3=, and \verb=CircleTI4= for the
  four nodes of the inner tangent lines.

\bigskip
\begin{pspicture}[showgrid](-2,-2)(10,10)
\pscircle(1,1){1}\pscircle(7,7){3}
\psCircleTangents(1,1){1}(7,7){3}
\pcline[nodesep=-1cm,linecolor=blue](CircleTO1)(CircleTO2)
\pcline[nodesep=-1cm,linecolor=blue](CircleTO3)(CircleTO4)
\pcline[nodesep=-1cm,linecolor=red](CircleTI1)(CircleTI2)
\pcline[nodesep=-1cm,linecolor=red](CircleTI3)(CircleTI4)
\psdots(CircleTC1)(CircleTC2)%
  (CircleTO1)(CircleTO2)(CircleTO3)(CircleTO4)%
  (CircleTI1)(CircleTI2)(CircleTI3)(CircleTI4)%
\uput[0](CircleTC1){TC1}\uput[0](CircleTC2){TC2}
\uput[-80](CircleTI1){TI1}\uput[115](CircleTI2){TI2}
\uput[150](CircleTI3){TI3}\uput[-45](CircleTI4){TI4}
\uput[-80](CircleTO1){TO1}\uput[150](CircleTO2){TO2}
\uput[150](CircleTO3){TO3}\uput[-45](CircleTO4){TO4}
\end{pspicture}

\bigskip
\begin{lstlisting}
\begin{pspicture}[showgrid](-2,-2)(10,10)
\pscircle(1,1){1}\pscircle(7,7){3}
\psCircleTangents(1,1){1}(7,7){3}
\pcline[nodesep=-1cm,linecolor=blue](CircleTO1)(CircleTO2)
\pcline[nodesep=-1cm,linecolor=blue](CircleTO3)(CircleTO4)
\pcline[nodesep=-1cm,linecolor=red](CircleTI1)(CircleTI2)
\pcline[nodesep=-1cm,linecolor=red](CircleTI3)(CircleTI4)
\psdots(CircleTC1)\psdots(CircleTC2)%
  (CircleTO1)(CircleTO2)(CircleTO3)(CircleTO4)%
  (CircleTI1)(CircleTI2)(CircleTI3)(CircleTI4)%
\uput[0](CircleTC1){TC1}\uput[0](CircleTC2){TC2}
\uput[-80](CircleTI1){TI1}\uput[115](CircleTI2){TI2}
\uput[150](CircleTI3){TI3}\uput[-45](CircleTI4){TI4}
\uput[-80](CircleTO1){TO1}\uput[150](CircleTO2){TO2}
\uput[150](CircleTO3){TO3}\uput[-45](CircleTO4){TO4}
\end{pspicture}
\end{lstlisting}


\clearpage

%--------------------------------------------------------------------------------------
\section{\nxLcs{psEllipseTangents}: Calculating tangent lines of an ellipse}
%--------------------------------------------------------------------------------------

The macro calculates the two points on an ellipse where tangent lines from an outside  point
 are drawn.

\begin{BDef}
\Lcs{psEllipseTangents}\Largr{$x_0,y_0$}\Largr{$a,b$}\Largr{$x_p,y_p$}\\
\end{BDef}

The first two pairs of coordinates are the same as the ones for the default ellipse.
The names of the calculates node names are \verb=EllipseT1=
and \verb=EllipseT2=.

\bigskip
\begin{pspicture}[showgrid](0,3)(10,10)
\psdot(2,4)\psellipse(7,7)(3,1.5)
\psEllipseTangents(7,7)(3,1.5)(2,4)
\pcline[nodesep=-1cm,linecolor=blue](2,4)(EllipseT1)
\pcline[nodesep=-1cm,linecolor=blue](2,4)(EllipseT2)
\psdots(EllipseT1)(EllipseT2)
\uput[-80](EllipseT1){T1}\uput[115](EllipseT2){T2}
\end{pspicture}


\begin{lstlisting}
\begin{pspicture}[showgrid](0,3)(10,10)
\psdot(2,4)\psellipse(7,7)(3,1.5)
\psEllipseTangents(7,7)(3,1.5)(2,4)
\pcline[nodesep=-1cm,linecolor=blue](2,4)(EllipseT1)
\pcline[nodesep=-1cm,linecolor=blue](2,4)(EllipseT2)
\psdots(EllipseT1)(EllipseT2)
\uput[-80](EllipseT1){T1}\uput[115](EllipseT2){T2}
\end{pspicture}
\end{lstlisting}


\clearpage

%--------------------------------------------------------------------------------------
\section{\nxLcs{psrotate}: Rotating objects}
%--------------------------------------------------------------------------------------
\Lcs{rput} also has an optional argument for rotating objects, but
it always depends on the \Lcs{rput} coordinates. With
\Lcs{psrotate} the rotating center can be placed anywhere. The
rotation is done with \verb+\pscustom+, all optional arguments are
only valid if they are part of the \verb+\pscustom+ macro.

\begin{BDef}
\Lcs{psrotate}\OptArgs\Largr{$x,y$}\Largb{rot angle}\Largb{object}
\end{BDef}

\begin{LTXexample}[width=0.4\linewidth]
\psset{unit=0.75}
\begin{pspicture}(-0.5,-3.5)(8.5,4.5)
  \psaxes{->}(0,0)(-0.5,-3)(8.5,4.5)
  \psdots[linecolor=red,dotscale=1.5](2,1)
  \psarc[linecolor=red,linewidth=0.4pt,showpoints=true]
        {->}(2,1){3}{0}{60}
  \pspolygon[linecolor=green,linewidth=1pt](2,1)(5,1.1)(6,-1)(2,-2)
  \psrotate(2,1){60}{%
    \pspolygon[linecolor=blue,linewidth=1pt](2,1)(5,1.1)(6,-1)(2,-2)}
\end{pspicture}
\end{LTXexample}


\begin{LTXexample}[width=6cm]
\begin{pspicture}(-1,-1)(3,6)
\def\canne{%  Idea by Manuel Luque
  \psgrid[subgriddiv=0](-1,0)(1,5)
  \pscustom[linewidth=2mm]{\psline(0,4)\psarcn(0.3,4){0.3}{180}{360}}%
  \pscircle*(0.6,4){0.1}\pstriangle*(0,0)(0.2,-0.3)}
\def\Object{}
  \canne
  \psrotate(0.3,4){45}{\psset{linecolor=red!50}\canne}
  \psrotate(0.3,4){90}{\psset{linecolor=blue!50}\canne}
  \psrotate(0.3,4){360}{\psset{linecolor=cyan!50}\canne}
  \psdot[linecolor=red](0.3,4)
\end{pspicture}
\end{LTXexample}


\begin{LTXexample}[pos=t]
\begin{pspicture}(0,-6)(15,5)
\def\majorette{\psline[linewidth=0.5mm](0,2)%  Idea by Manuel Luque
               \pscircle[fillstyle=solid]{0.1}
               \pscircle[fillstyle=solid](0,2){0.1}}
  \psaxes[linewidth=0.5pt]{->}(0,0)(0,-5)(15,5)
  \pstVerb{/V0 10 def /Alpha 45 def}% vitesse initiale, angle de lancement
  \multido{\nT=0.0+0.05,\iA=0+40}{41}{%
    \pstVerb{/nT \nT\space def}%
    \rput(!V0 Alpha cos mul nT mul -9.81 2 div nT dup mul mul V0 Alpha sin mul nT mul add){%
       \psrotate(0,1){\iA}{\majorette\psdot[linecolor=red](0,1)\psdot[linecolor=green](0,2)}}}
  \parametricplot[linecolor=red]{0}{2}{% trajectoire du milieu
     V0 Alpha cos mul t mul -9.81 2 div t dup mul mul V0 Alpha sin mul t mul add 1 add}
  \parametricplot[linecolor=green,plotpoints=360]{0}{2}{% d'une extremite
     V0 Alpha cos mul t mul 800 t mul sin sub % x(t)
     -9.81 2 div t dup mul mul V0 Alpha sin mul t mul add 1 add 800 t mul cos add }%y(t)
\end{pspicture}
\end{LTXexample}


\clearpage

%--------------------------------------------------------------------------------------
\section{\nxLcs{psComment}: comments to a graphic}
%--------------------------------------------------------------------------------------

\begin{BDef}
\LcsStar{psComment}\OptArgs\OptArg*{\Largb{arrows}}\coord0\coord1\Largb{Text}\OptArg{line macro}\OptArg{put macro}
\end{BDef}

By default the macro uses the \Lcs{ncline} macro to draw a line from the first to the
second point, it can be changed with the first additional optional argument. The label is
put by default with \Lcs{rput}, which can be changed with the last optional argument.
If this is used, then the line macro has also be defined, eg \verb+\psComment(A)(B){text}[\ncarc][\ncput}+
At least, leave the argument empty.


\begin{LTXexample}[pos=t,wide]
\SpecialCoor\newpsstyle{weiss}{fillstyle=solid,fillcolor=white}
\footnotesize\psset{unit=0.5cm,dimen=middle}
\begin{pspicture}(-12,-4)(6,10)
\psframe*[linecolor=black!20](-5,-3)(5,7) \psframe*[linecolor=black!40](-5,3)(5,6)
\pscircle(-8.19,5.51){0.2}
\psframe[fillcolor=white,fillstyle=solid](-5.8,3.6)(4.3,5.8)
\psframe(-8.98,3.14)(-5.8,6.32)
\multido{\rA=-4.1+1.3}{5}{\rput(\rA,-2.4){\psframe[style=weiss](1.1,6)
  \psline(0,0)(1.1,0.5)(0,1)(1.1,1.6)(0,2.2)(1.1,2.7)(0,3.2)(1.1,3.2)}}
\pspolygon*(-4.1,3.7)(-4.1,3)(-3,3)(-3.01,3.7)(-3.54,4.19)
\pspolygon*(1.09,3.7)(1.1,3)(2.2,3)(2.18,3.7)(1.65,4.24)
\pspolygon*(-2.78,3.7)(-2.8,3)(-1.7,3)(-1.71,3.7)(-2.27,4.04)
\pspolygon*(-1.51,3.7)(-1.5,3)(-0.4,3)(-0.41,3.7)(-1.02,4.17)
\pspolygon*(-0.21,3.7)(-0.2,3)(0.9,3)(0.89,3.7)(0.3,4.04)
\psline(-5,3.83)(-4.15,3.86)(-3.5,4.3)(-2.85,3.81)(-2.22,4.21)(-1.6,3.86)(-0.99,4.33)
       (-0.28,3.83)(0.35,4.19)(0.97,3.83)(1.65,4.39)(2.2,4.01)(3.57,4.89)(2.41,5.8)
  \psline(-5,5.8)(-5.78,5.8)  \psline(-5.78,5.47)(2.85,5.47)
  \psline(-5.8,3.52)(-5,3.5)  \psline(3.57,4.89)(-5.8,4.89)
  \psComment*[ref=r]{->}(-8.14,1.19)(-4.31,3.27){Mantelstift}
  \psComment*[ref=r]{->}(-8.17,-0.56)(-4.37,1.59){Kernstift}[\ncarc]
  \psComment*[ref=r]{->}(-7.91,-2.24)(-4.44,-0.23){Feder}[\ncarc]
  \psComment[npos=-0.1]{->}(-3.48,8.72)(-1.33,5.46){Nur f\"ur Profil}
\end{pspicture}
\end{LTXexample}

\clearpage
%--------------------------------------------------------------------------------------
\section{\nxLcs{psChart}: a pie chart}
%--------------------------------------------------------------------------------------

\begin{BDef}
\Lcs{psChart}\OptArgs\Largb{comma separated value list}\Largb{comma separated value list}\Largb{radius}
\end{BDef}

The special optional arguments for the \Lcs{psChart} macro are as follows:

\noindent
\begin{tabularx}{\linewidth}{@{}>{\ttfamily}lX>{\ttfamily}l@{}}
\textrm{\emph{name}} & \textrm{\emph{description}} & \textrm{\emph{default}}\\\hline
\Lkeyword{chartSep}  & distance from the pie chart center to an outraged pie piece & 10pt\\
\Lkeyword{chartColor} & gray or colored pie (values are: \texttt{gray} or \texttt{color})& gray\\
\Lkeyword{userColor} & a comma separated list of user defined colors for the pie & \{\}\\
\Lkeyword{chartNodeI}& the position of the inner node, relative to the radius & 0.75\\
\Lkeyword{chartNodeO}& the position of the outer node, relative to the radius & 1.5
\end{tabularx}

\bigskip
The first mandatory argument is the list of the values and may not be empty. The second
one is a list of outraged pieces, numbered consecutively from 1 to up the total number
of values. The list of user defined colors must be enclosed in braces!

The macro \Lcs{psChart} defines for every value three nodes at the half angle and
in distances from 0.75, 1, and 1.25 times of the radius from the origin. The nodes
are named as \verb+psChartI?+, \verb+psChart?+, and \verb+psChartO?+, where ? is the number of
the pie. The letter I leads to the inner node and the letter O to the outer node.
The distance can be changed with the optional arguments \Lkeyword{chartNodeI} and
\Lkeyword{chartNodeO} in the usual way with \verb+\psset{chartNodeI=...,chartNodeO=...}+.

The other one is the node on the circle line.
The origin is by default \texttt{(0,0)}. Moving the pie to another position can be done as
usual with the \Lcs{rput}-macro. The used colors are named internally as \Lkeyword{chartFillColor?}
and can be used by the user for coloring lines or text.

\begin{LTXexample}[width=6cm]
\begin{pspicture}(-3,-3)(3,3)
\psChart{ 23, 29, 3, 26, 28, 14 }{}{2}
\multido{\iA=1+1}{6}{%
  \psdot(psChart\iA)\psdot(psChartI\iA)\psdot(psChartO\iA)%
  \psline[linestyle=dashed,linecolor=white](psChart\iA)
  \psline[linestyle=dashed](psChart\iA)(psChartO\iA)}
\end{pspicture}
\end{LTXexample}

\begin{LTXexample}[width=6cm]
\begin{pspicture}(-3,-3)(3,3)
\psChart[chartColor=color]{45,90}{1}{2}
\ncline[linecolor=-chartFillColor1,
  nodesepB=-20pt]{psChartO1}{psChart1}
\rput[l](psChartO1){%
  \textcolor{chartFillColor1}{pie no 1}}
\ncline[linecolor=-chartFillColor2,
  nodesepB=-20pt]{psChartO2}{psChart2}
\rput[lt](psChartO2){%
  \textcolor{chartFillColor2}{pie no 2}}
\end{pspicture}
\end{LTXexample}

\begin{LTXexample}[width=7.5cm]
\psframebox[fillcolor=black!20,
  fillstyle=solid]{%
\begin{pspicture}(-3.5,-3.5)(4.25,3.5)
\psChart[chartColor=color]%
  {23, 29, 3, 26, 28, 14, 17, 4, 9}{}{2}
\multido{\iA=1+1}{9}{%
  \ncline[linecolor=-chartFillColor\iA,
    nodesepB=-10pt]{psChartO\iA}{psChart\iA}
  \rput[l](psChartO\iA){%
    \textcolor{chartFillColor\iA}{pie no \iA}}}
\end{pspicture}}
\end{LTXexample}

\begin{LTXexample}[width=6cm]
\begin{pspicture}(-3,-3)(3,3)
\psChart[userColor={red!30,green!30,
    blue!40,gray,magenta!60,cyan}]%
      { 23, 29, 3, 26, 28, 14 }{1,4}{2}
\end{pspicture}
\end{LTXexample}

\begin{LTXexample}[width=6cm]
\begin{pspicture}(-3,-2.5)(3,2.5)
\psChart{ 23, 29, 3, 26, 28, 14 }{}{2}
\multido{\iA=1+1}{6}{\rput*(psChartI\iA){\iA}}
\end{pspicture}
\end{LTXexample}


%\begin{LTXexample}[pos=t]
\psset{unit=1.5}
\begin{pspicture}(-3,-3)(3,3)
\psChart[userColor={red!30,green!30,blue!40,gray,cyan!50,
    magenta!60,cyan},chartSep=30pt,shadow=true,shadowsize=5pt]{34.5,17.2,20.7,15.5,5.2,6.9}{6}{2}
\psset{nodesepA=5pt,nodesepB=-10pt}
\ncline{psChartO1}{psChart1}\nput{0}{psChartO1}{1000 (34.5\%)}
\ncline{psChartO2}{psChart2}\nput{150}{psChartO2}{500 (17.2\%)}
\ncline{psChartO3}{psChart3}\nput{-90}{psChartO3}{600 (20.7\%)}
\ncline{psChartO4}{psChart4}\nput{0}{psChartO4}{450 (15.5\%)}
\ncline{psChartO5}{psChart5}\nput{0}{psChartO5}{150 (5.2\%)}
\ncline{psChartO6}{psChart6}\nput{0}{psChartO6}{200 (6.9\%)}
\bfseries%
\rput(psChartI1){Taxes}\rput(psChartI2){Rent}\rput(psChartI3){Bills}
\rput(psChartI4){Car}\rput(psChartI5){Gas}\rput(psChartI6){Food}
\end{pspicture}
%\end{LTXexample}
\psset{unit=1cm}

\begin{lstlisting}
\psset{unit=1.5}
\begin{pspicture}(-3,-3)(3,3)
\psChart[userColor={red!30,green!30,blue!40,gray,cyan!50,
    magenta!60,cyan},chartSep=30pt,shadow=true,shadowsize=5pt]{34.5,17.2,20.7,15.5,5.2,6.9}{6}{2}
\psset{nodesepA=5pt,nodesepB=-10pt}
\ncline{psChartO1}{psChart1}\nput{0}{psChartO1}{1000 (34.5\%)}
\ncline{psChartO2}{psChart2}\nput{150}{psChartO2}{500 (17.2\%)}
\ncline{psChartO3}{psChart3}\nput{-90}{psChartO3}{600 (20.7\%)}
\ncline{psChartO4}{psChart4}\nput{0}{psChartO4}{450 (15.5\%)}
\ncline{psChartO5}{psChart5}\nput{0}{psChartO5}{150 (5.2\%)}
\ncline{psChartO6}{psChart6}\nput{0}{psChartO6}{200 (6.9\%)}
\bfseries%
\rput(psChartI1){Taxes}\rput(psChartI2){Rent}\rput(psChartI3){Bills}
\rput(psChartI4){Car}\rput(psChartI5){Gas}\rput(psChartI6){Food}
\end{pspicture}
\end{lstlisting}



\clearpage
%--------------------------------------------------------------------------------------
\section{\nxLcs{psHomothetie}: central dilatation}
%--------------------------------------------------------------------------------------

\begin{BDef}
\Lcs{psHomothetie}\OptArgs\Largr{center}\Largb{factor}\Largb{object}
\end{BDef}

\begin{LTXexample}[width=9cm]
\begin{pspicture}[showgrid=true](-5,-4)(4,8)
\psBill% needs package pst-fun
\psHomothetie[linecolor=blue](4,-3){2}{\psBill}
\psdots[dotsize=3pt,linecolor=red](4,-3)
\psplot[linestyle=dashed,linecolor=red]{-5}{4}%
  [ /m -3 -0.85 sub 4 0.6 sub div def ]
  { m x mul m 4 mul sub 3 sub }%
\psHomothetie[linecolor=green](4,-3){-0.2}{\psBill}
\end{pspicture}
\end{LTXexample}

%\pstVerb{ /m -3 -0.85 sub 4 0.6 sub div def }


\clearpage

%--------------------------------------------------------------------------------------
\section{\nxLcs{psbrace}}
%--------------------------------------------------------------------------------------
\begin{BDef}
\LcsStar{psbrace}\OptArgs\Largr{A}\Largr{B}\Largb{text}
\end{BDef}

Additional to all other available options from \LPack{pstricks} or the other
related packages,  there are two new option, named  \Lkeyword{braceWidth} and
\Lkeyword{bracePos}. All important ones are shown in the following graphics
and table.

\begin{center}
\begin{pspicture}[showgrid=true](10,5)
  \psbrace[braceWidth=1cm,braceWidthInner=1cm,
    braceWidthOuter=1cm,bracePos=0.6,fillcolor=white,
    nodesepA=10mm,nodesepB=10mm](0,5)(10,5){\fbox{Label}}
\pcline{<->}(3,3)(3,4)\ncput*{\footnotesize\ttfamily braceWidth}
\pcline{<->}(3,4)(3,5)\ncput*{\footnotesize\ttfamily braceWidthInner}
\pcline{<->}(3,2)(3,3)\ncput*{\footnotesize\ttfamily braceWidthOuter}
\pcline{<->}(6,1)(6,2)\ncput{\footnotesize\ttfamily nodesepB}
\pcline{<->}(6,1)(7,1)\ncput*{\footnotesize\ttfamily A}
\pcline{<->}(0,0.5)(6,0.5)\ncput*{\footnotesize\ttfamily bracePos}
\psdot[dotscale=2](0,5)\uput[0](0,5){\textbf{A}}
\psdot[dotscale=2](10,5)\uput[180](10,5){\textbf{B}}
\end{pspicture}
\end{center}

A positive value for \Lkeyword{nodesepA} and \Lkeyword{nodesepB} shifts the label to the upper right
and a negative value to the lower left. This does not depends on
the value for the rotating of the label!

\begin{center}
\begin{tabular}{@{}l|l@{}}
name & meaning\\\hline
\Lkeyword{braceWidth} & default is \Lcs{pslinewidth}\\
\Lkeyword{braceWidthInner} & default is \verb+10\pslinewidth+\\
\Lkeyword{braceWidthOuter} & default is \verb+10\pslinewidth+\\
\Lkeyword{bracePos} & relative position (default is $0.5$)\\
\Lkeyword{nodesepA} & x-separation (default is $0pt$)\\
\Lkeyword{nodesepB} & y-separation (default is $0pt$)\\
\Lkeyword{rot} & additional rotating for the text (default is $0$)\\
\Lkeyword{ref} & reference point for the text (default is c)\\
\Lkeyword{fillcolor} & default is black
\end{tabular}
\end{center}

By default the text is written perpendicular to the brace line and
can be changed with the \LPack{pstricks} option \Lkeyword{rot}=\ldots\ The
text parameter can take any object and may also be empty. The
reference point can be any value of the combination of \Lkeyval{l}
(left) or \Lkeyval{r} (right) and \Lkeyval{b} (bottom) or \Lkeyval{B}
(Baseline) or \Lkeyval{C} (center) or \Lkeyval{t} (top), where the
default is \Lkeyval{c}, the center of the object.



\begin{LTXexample}[width=4.5cm]
\begin{pspicture}(4,4)
\psgrid[subgriddiv=0,griddots=10]
\pnode(0,0){A}
\pnode(4,4){B}
\psbrace[linecolor=red,ref=lC](A)(B){Text I}
\psbrace*[linecolor=blue,ref=lC](3,4)(0,1){Text II}
\psbrace[fillcolor=white](3,0)(3,4){III}
\end{pspicture}
\end{LTXexample}

\bigskip
The option \Lcs{specialCoor} is enabled, so that all types of coordinates
are possible, (nodename), ($x,y$), ($nodeA|nodeB$), \ldots
The star version fills the inner of the \Index{brace} with the current linecolor.
With the fillcolor \verb+white+ or any other background color the brace can
be "`unfilled"'.

\begin{LTXexample}
\begin{pspicture}(8,2.5)
\psbrace(0,0)(0,2){\fbox{Text}}%
\psbrace[nodesepA=10pt](2,0)(2,2){\fbox{Text}}
\psbrace[ref=lC](4,0)(4,2){\fbox{Text}}
\psbrace[ref=lt,rot=90,nodesepB=-15pt](6,0)(6,2){\fbox{Text}}
\psbrace[ref=lt,rot=90,nodesepA=-5pt,nodesepB=15pt](8,2)(8,0){\fbox{Text}}
\end{pspicture}
\end{LTXexample}


\begin{LTXexample}
\def\someMath{$\int\limits_1^{\infty}\frac{1}{x^2}\,dx=1$}
\begin{pspicture}(8,2.5)
\psbrace[ref=lC](0,0)(0,2){\someMath}%
\psbrace[rot=90](2,0)(2,2){\someMath}
\psbrace[ref=lC](4,0)(4,2){\someMath}
\psbrace[ref=lt,rot=90,nodesepB=-30pt](6,0)(6,2){\someMath}
\psbrace[ref=lt,rot=90,nodesepB=30pt](8,2)(8,0){\someMath}
\end{pspicture}
\end{LTXexample}

%$

\begin{LTXexample}
\begin{pspicture}(\linewidth,5)
\psbrace(0,0.5)(\linewidth,0.5){\fbox{Text}}%
\psbrace[bracePos=0.25,nodesepB=10pt,rot=90](0,2)(\linewidth,2){\fbox{Text}}
\psbrace[ref=lC,nodesepA=-3.5cm,nodesepB=15pt,rot=90](0,4)(\linewidth,4){%
   \fbox{some very, very long wonderful Text}}
\end{pspicture}
\end{LTXexample}


\begin{LTXexample}[width=8cm]
\psset{unit=0.8}
\begin{pspicture}(10,11)
\psgrid[subgriddiv=0,griddots=10]
\pnode(0,0){A}
\pnode(4,6){B}
\psbrace[ref=lC](A)(B){One}
\psbrace[rot=180,nodesepA=-5pt,ref=rb](B)(A){Two}
\psbrace[linecolor=blue,bracePos=0.25,ref=lB](8,1)(1,7){Three}
\psbrace[braceWidth=-1mm,rot=180,ref=rB](8,1)(1,7){Four}
\psbrace*[linearc=0.5,fillstyle=none,linewidth=1pt,braceWidth=1.5pt,
  bracePos=0.25,ref=lC](8,1)(8,9){A}
\psbrace(4,9)(6,9){}
\psbrace(6,9)(6,7){}
\psbrace(6,7)(4,7){}
\psbrace(4,7)(4,9){}
\psset{linecolor=red}
\psbrace*[ref=lb](7,10)(3,10){I}
\psbrace*[ref=lb,bracePos=0.75](3,10)(3,6){II}
\psbrace*[ref=lb](3,6)(7,6){III}
\psbrace*[ref=lb](7,6)(7,10){IV}
\end{pspicture}
\end{LTXexample}

%$

\begin{LTXexample}[width=5cm]
\[
\begin{pmatrix}
    \Rnode[vref=2ex]{A}{~1} \\
    & \ddots \\
    && \Rnode[href=2]{B}{1} \\
    &&& \Rnode[vref=2ex]{C}{0} \\
    &&&& \ddots \\
    &&&&& \Rnode[href=2]{D}{0}~ \\
\end{pmatrix}
\]
\psbrace[rot=-90,nodesepB=-0.5,nodesepA=-0.2](B)(A){\small n times}
\psbrace[rot=-90,nodesepB=-0.5,nodesepA=-0.2](D)(C){\small n times}
\end{LTXexample}


\clearpage
It is also possible to put a vertical brace around a
default paragraph. This works by setting two invisible nodes at
the beginning and the end of the paragraph. Indentation is
possible with a minipage.

\small
Some nonsense text, which is nothing more than nonsense.
Some nonsense text, which is nothing more than nonsense.

\noindent\rnode{A}{}

\vspace*{-1ex}
Some nonsense text, which is nothing more than nonsense.
Some nonsense text, which is nothing more than nonsense.
Some nonsense text, which is nothing more than nonsense.
Some nonsense text, which is nothing more than nonsense.
Some nonsense text, which is nothing more than nonsense.
Some nonsense text, which is nothing more than nonsense.
Some nonsense text, which is nothing more than nonsense.
Some nonsense text, which is nothing more than nonsense.

\vspace*{-2ex}\noindent\rnode{B}{}\psbrace*[linecolor=red](A)(B){}

Some nonsense text, which is nothing more than nonsense.
Some nonsense text, which is nothing more than nonsense.

\medskip\hfill\begin{minipage}{0.95\linewidth}
\noindent\rnode{A}{}

\vspace*{-1ex}
Some nonsense text, which is nothing more than nonsense.
Some nonsense text, which is nothing more than nonsense.
Some nonsense text, which is nothing more than nonsense.
Some nonsense text, which is nothing more than nonsense.
Some nonsense text, which is nothing more than nonsense.
Some nonsense text, which is nothing more than nonsense.
Some nonsense text, which is nothing more than nonsense.
Some nonsense text, which is nothing more than nonsense.

\vspace*{-2ex}
\noindent\rnode{B}{}\psbrace[linecolor=red](A)(B){}
\end{minipage}

\normalsize

\begin{lstlisting}
Some nonsense text, which is nothing more than nonsense.
Some nonsense text, which is nothing more than nonsense.

\noindent\rnode{A}{}

\vspace*{-1ex}
Some nonsense text, which is nothing more than nonsense.
Some nonsense text, which is nothing more than nonsense.
Some nonsense text, which is nothing more than nonsense.
Some nonsense text, which is nothing more than nonsense.
Some nonsense text, which is nothing more than nonsense.
Some nonsense text, which is nothing more than nonsense.
Some nonsense text, which is nothing more than nonsense.
Some nonsense text, which is nothing more than nonsense.

\vspace*{-2ex}\noindent\rnode{B}{}\psbrace[linecolor=red](A)(B){}

Some nonsense text, which is nothing more than nonsense.
Some nonsense text, which is nothing more than nonsense.

\medskip\hfill\begin{minipage}{0.95\linewidth}
\noindent\rnode{A}{}

\vspace*{-1ex}
Some nonsense text, which is nothing more than nonsense.
Some nonsense text, which is nothing more than nonsense.
Some nonsense text, which is nothing more than nonsense.
Some nonsense text, which is nothing more than nonsense.
Some nonsense text, which is nothing more than nonsense.
Some nonsense text, which is nothing more than nonsense.
Some nonsense text, which is nothing more than nonsense.
Some nonsense text, which is nothing more than nonsense.

\vspace*{-2ex}\noindent\rnode{B}{}\psbrace[linecolor=red](A)(B){}
\end{minipage}
\end{lstlisting}

\clearpage


%--------------------------------------------------------------------------------------
\section{Random dots}
%--------------------------------------------------------------------------------------
The syntax of the new macro \Lcs{psRandom} is:

\begin{BDef}
\Lcs{psRandom}\OptArgs\Largb{}\\
\Lcs{psRandom}\OptArgs\OptArg*{\Largr{$x_{Min},y_{Min}$}}\OptArg*{\Largr{$x_{Max},y_{Max}$}}\Largb{clip path} %$
%\psRandom[<option>](<xMax,yMax>){<clip path>}
%\psRandom[<option>](<xMin,yMin>)(<xMax,yMax>){<clip path>}
\end{BDef}

If there is no area for the dots defined, then \verb+(0,0)(1,1)+ in the current
scale setting is used for placing the dots. If there is only one \Largr{$x_{Max},y_{Max}$} %$
defined, then \verb+(0,0)+ is used for the other point.
This area should be greater than the clipping
path to be sure that the dots are placed over the full area. The clipping path can
be everything. If no clipping path is given, then the frame \verb+(0,0)(1,1)+
in user coordinates is used.  The new options are:

\begin{center}
\begin{tabular}{@{}l|l|l@{}}
name & default\\\hline
\Lkeyword{randomPoints} &   \verb|1000| & number of random dots\tabularnewline
\Lkeyword{color} & \false & random color\tabularnewline
\end{tabular}
\end{center}


\begin{LTXexample}[width=0.3\linewidth]
\psset{unit=5cm}
\begin{pspicture}(1,1)
  \psRandom[dotsize=1pt,fillstyle=solid](1,1){\pscircle(0.5,0.5){0.5}}
\end{pspicture}
\begin{pspicture}(1,1)
  \psRandom[dotsize=2pt,randomPoints=5000,color,%
      fillstyle=solid](1,1){\pscircle(0.5,0.5){0.5}}
\end{pspicture}
\end{LTXexample}

\begin{LTXexample}[width=0.4\linewidth]
\psset{unit=5cm}
\begin{pspicture}(1,1)
  \psRandom[randomPoints=200,dotsize=8pt,dotstyle=+]{}
\end{pspicture}
\begin{pspicture}(1.5,1)
  \psRandom[dotsize=5pt,color](0,0)(1.5,0.8){\psellipse(0.75,0.4)(0.75,0.4)}
\end{pspicture}
\end{LTXexample}

\begin{LTXexample}
\psset{unit=2.5cm}
\begin{pspicture}(0,-1)(3,1)
  \psRandom[dotsize=4pt,dotstyle=o,linecolor=blue,fillcolor=red,%
     fillstyle=solid,randomPoints=1000]%
      (0,-1)(3,1){\psplot{0}{3.14}{ x 114 mul sin }}
\end{pspicture}
\end{LTXexample}

\psset{unit=1cm}


\clearpage
 %--------------------------------------------------------------------------------------
\section{\nxLcs{psDice}}
 %--------------------------------------------------------------------------------------
\Lcs{psdice} creates the view of a dice. The number on the dice is the only parameter.
The optional parameters, like the color can be used as usual. The macro is a box of
dimension zero and is placed
at the current point. Use  the \Lcs{rput} macro to place it anywhere. The optional
argument \Lkeyword{unit} can be used to scale the dice. the default size of
the dice $1\mathrm{cm}\times1\mathrm{cm}$.

\begin{center}
\begin{pspicture}(-1,-1)(8,9)
\multido{\iA=1+1}{6}{%
  \rput(\iA,7.5){\Huge\psdice[unit=0.75,linecolor=red!80]{\iA}}
  \rput(! -0.5 7 \iA\space sub){\Huge\psdice[unit=0.75,linecolor=blue!70]{\iA}}%
  \multido{\iB=1+1}{6}{%
    \rput(! \iA\space 7 \iB\space sub){%
      \rnode[c]{p\iA\iB}{\makebox[1em][l]{\strut\psPrintValue[fontscale=12]{\iA\space \iB\space add}}}%
}}}
\ncbox[linearc=0.35,nodesep=0.2,linestyle=dotted]{p11}{p66}
\ncbox[linearc=0.35,nodesep=0.2,linestyle=dashed]{p15}{p51}
\rput{90}(-1.5,3.5){1. dice}
\rput{0}(3.5,8.5){2. dice}
\psline[linewidth=1.5pt](0.25,0.5)(0.25,8)
\psline[linewidth=1.5pt](-1,6.75)(6.5,6.75)
\end{pspicture}
\end{center}

\begin{lstlisting}
\begin{pspicture}(-1,-1)(8,8)
\multido{\iA=1+1}{6}{%
  \rput(\iA,7.5){\Huge\psdice[unit=0.75,linecolor=red!80]{\iA}}
  \rput(! -0.5 7 \iA\space sub){\Huge\psdice[unit=0.75,linecolor=blue!70]{\iA}}%
  \multido{\iB=1+1}{6}{%
    \rput(! \iA\space 7 \iB\space sub){%
      \rnode[c]{p\iA\iB}{\makebox[1em][l]{\strut\psPrintValue[fontscale=12]{\iA\space \iB\space add}}}%
}}}
\ncbox[linearc=0.35,nodesep=0.2,linestyle=dotted]{p11}{p66}
\ncbox[linearc=0.35,nodesep=0.2,linestyle=dashed]{p15}{p51}
\rput{90}(-1.5,3.5){1. dice}
\rput{0}(3.5,8.5){2. dice}
\psline[linewidth=1.5pt](0.25,0.5)(0.25,8)
\psline[linewidth=1.5pt](-1,6.75)(6.5,6.75)
\end{pspicture}
\end{lstlisting}

\clearpage
%--------------------------------------------------------------------------------------
\section{\nxLcs{psFormatInt}}
%--------------------------------------------------------------------------------------
There exist some packages and a lot of code to format an integer like $1\,000\,000$
or $1,234,567$ (in Europe $1.234.567$). But all packages expect a real number as
argument and cannot handle macros as an argument. For this case \LPack{pstricks-add}
has a macro \Lcs{psFormatInt} which can handle both:

\begin{LTXexample}[width=3cm]
\psFormatInt{1234567}\\
\psFormatInt[intSeparator={,}]{1234567}\\
\psFormatInt[intSeparator=.]{1234567}\\
\psFormatInt[intSeparator=$\cdot$]{1234567}\\
\def\temp{965432}
\psFormatInt{\temp}
\end{LTXexample}

With the option \Lkeyword{intSeparator} the symbol can be changed to any any non-number character.


\clearpage

%--------------------------------------------------------------------------------------
\section{\nxLcs{psRelNode} and \nxLcs{psDefPSPNodes}}
%--------------------------------------------------------------------------------------
With these macros it is possible to put a node relative to a given line or given
\Lenv{pspicture}-environment. In the frist case the parameters are
the angle and the length factor:

\begin{BDef}
\Lcs{psRelNode}\Largs{P0}\Largs{P1}\Largb{length factor}\Largb{end node name}\\
\Lcs{psDefPSPNodes}
\end{BDef}

The length factor relates to the distance $\overline{P_0P_1}$ and
the end node name must be a valid nodename and shouldn't contain
any of the special PostScript characters. There are two valid
options:

\begin{tabularx}{\linewidth}{@{} l|l| X @{} }
name & default & meaning\\\hline 
\Lkeyword{angle} & $0$ & angle between the given line $\overline{P_0P_1}$ and the new one
	$\overline{P_0P_{endNode}}$\tabularnewline 
\Lkeyword{trueAngle} & \false & defines whether the angle refers to the seen line or to
the mathematical one, which respect the scaling factors
\Lkeyword{xunit} and \Lkeyword{yunit}.
\end{tabularx}

\begin{LTXexample}[width=7cm]
\begin{pspicture}[showgrid](7,6)
  \pnode(3,3){A}\pnode(4,2){B}
  \psline[nodesep=-3,linewidth=0.5pt](A)(B)
  \multido{\iA=0+30}{12}{%
    \psRelNode[angle=\iA](A)(B){2}{C}%
    \qdisk(C){2pt}
    \uput[0](C){\iA}}
\end{pspicture}
\end{LTXexample}

In the second case the new macro \Lcs{psDefPSPNodes} defines nine nodes that corresponds to
nine particular points (namely bottom left, bottom center,
bottom right, center left, center center, center right, top left,
top center, top right) of the \Lenv{pspicture} box.

\begin{LTXexample}[width=6cm,wide=false]
\begin{pspicture}[showgrid=true](-1,-1)(4,4)
  \psDefPSPNodes
  \psdots(PSPbl)(PSPbc)(PSPbr)
      (PSPcl)(PSPcc)(PSPcr)(PSPtl)(PSPtc)(PSPtr)
  \uput[90](PSPbl){PSPbl} \uput[90](PSPbc){PSPbc}
  \uput[90](PSPbr){PSPbr} \uput[90](PSPcl){PSPcl}
  \uput[90](PSPcc){PSPcc} \uput[90](PSPcr){PSPcr}
  \uput[90](PSPtl){PSPtl} \uput[90](PSPtc){PSPtc}
  \uput[90](PSPtr){PSPtr}
\end{pspicture}
\end{LTXexample}

The name of the nodes are predefined as:

\begin{lstlisting}[style=syntax]
\psset[pst-PSPNodes]{blName=PSPbl,bcName=PSPbc,brName=PSPbr,
  clName=PSPcl,ccName=PSPcc,crName=PSPcr,tlName=PSPtl,tcName=PSPtc,trName=PSPtr}
\end{lstlisting}

and can be modified in the same way.
%I guess you modified the family to have the pstricks-add one so the
%\xkvview would have to be adapted.

%--------------------------------------------------------------------------------------
\section{\nxLcs{psRelLine}}
%--------------------------------------------------------------------------------------
With this macro it is possible to plot lines relative to a given one. Parameter are
the angle and the length factor:

\begin{BDef}
\Lcs{psRelLine}\Largr{P0}\Largr{P1}\Largb{length factor}\Largb{<end node name>}\\
\Lcs{psRelLine}\OptArg{\Largb{arrows}}\Largr{P0}\Largr{P1}\Largb{length factor}\Largb{end node name}\\
\Lcs{psRelLine}\OptArgs\Largr{P0}\Largr{P1}\Largb{length factor}\Largb{end node name}\\
\Lcs{psRelLine}\OptArgs\OptArg{\Largb{arrows}}\Largr{P0}\Largr{P1}\Largb{length factor}\Largb{end node name}
\end{BDef}

The length factor relates to the distance $\overline{P_0P_1}$ and
the end node name must be a valid nodename and shouldn't contain
any of the special PostScript characters. There are two valid
options which are described in the foregoing section for
\Lcs{psRelNode}.

The following two figures show the same, the first one with a scaling different to $1:1$,
this is the reason why the end points are on an ellipse and not on a circle like in the
second figure.

\begin{LTXexample}[width=5cm]
\psset{yunit=2,xunit=1}
\begin{pspicture}(-2,-2)(3,2)
\psgrid[subgriddiv=2,subgriddots=10,gridcolor=lightgray]
\pnode(-1,0){A}\pnode(3,2){B}
\psline[linecolor=red](A)(B)
\psRelLine[linecolor=blue,angle=30](-1,0)(B){0.5}{EndNode}
\qdisk(EndNode){2pt}
\psRelLine[linecolor=blue,angle=-30](A)(B){0.5}{EndNode}
\qdisk(EndNode){2pt}
\psRelLine[linecolor=magenta,angle=90](-1,0)(3,2){0.5}{EndNode}
\qdisk(EndNode){2pt}
\psRelLine[linecolor=magenta,angle=-90](A)(B){0.5}{EndNode}
\qdisk(EndNode){2pt}
\end{pspicture}
\end{LTXexample}

\begin{LTXexample}[width=5cm]
\begin{pspicture}(-2,-2)(3,2)
\psgrid[subgriddiv=2,subgriddots=10,gridcolor=lightgray]
\pnode(-1,0){A}\pnode(3,2){B}
\psline[linecolor=red](A)(B)
\psarc[linestyle=dashed](A){2.23}{-90}{135}
\psRelLine[linecolor=blue,angle=30](-1,0)(B){0.5}{EndNode}
\qdisk(EndNode){2pt}
\psRelLine[linecolor=blue,angle=-30](A)(B){0.5}{EndNode}
\qdisk(EndNode){2pt}
\psRelLine[linecolor=magenta,angle=90](-1,0)(3,2){0.5}{EndNode}
\qdisk(EndNode){2pt}
\psRelLine[linecolor=magenta,angle=-90](A)(B){0.5}{EndNode}
\qdisk(EndNode){2pt}
\end{pspicture}
\end{LTXexample}

\medskip
The following figure has also a different scaling, but has set the
option \Lkeyword{trueAngle}, all angles refer to "what you see".

\begin{LTXexample}[width=6.5cm]
\psset{yunit=2,xunit=1}
\begin{pspicture}(-3,-1)(3,2)\psgrid[subgridcolor=lightgray]
\pnode(-1,0){A}\pnode(3,2){B}
\psline[linecolor=red](A)(B)
\psarc(A){2.83}{-45}{135}
\psRelLine[linecolor=blue,angle=30,trueAngle](A)(B){0.5}{EndNode}
\qdisk(EndNode){2pt}
\psRelLine[linecolor=blue,angle=-30,trueAngle](A)(B){0.5}{EndNode}
\qdisk(EndNode){2pt}
\psRelLine[linecolor=magenta,angle=90,trueAngle](A)(B){0.5}{EndNode}
\qdisk(EndNode){2pt}
\psRelLine[linecolor=magenta,angle=-90,trueAngle](A)(B){0.5}{EndNode}
\qdisk(EndNode){2pt}
\end{pspicture}
\end{LTXexample}

\medskip
Two examples using \verb+\multido+ to show the behaviour of the
options \verb+trueAngle+ and \verb+angle+.

\medskip
\begin{LTXexample}[width=8cm]
\psset{yunit=4,xunit=2}
\begin{pspicture}(-1,0)(3,2)\psgrid[subgridcolor=lightgray]
\pnode(-1,0){A}\pnode(1,1){B}
\psline[linecolor=red](A)(3,2)
\multido{\iA=0+10}{36}{%
  \psRelLine[linecolor=blue,angle=\iA](B)(A){-0.5}{EndNode}
  \qdisk(EndNode){2pt}
}
\end{pspicture}
\end{LTXexample}

\begin{LTXexample}[width=8cm]
\psset{yunit=4,xunit=2}
\begin{pspicture}(-1,0)(3,2)\psgrid[subgridcolor=lightgray]
\pnode(-1,0){A}\pnode(1,1){B}
\psline[linecolor=red](A)(3,2)
\multido{\iA=0+10}{36}{%
  \psRelLine[linecolor=magenta,angle=\iA,trueAngle]{->}(B)(A){-0.5}{EndNode}
}
\end{pspicture}
\end{LTXexample}

\begin{center}
\bgroup
\psset{xunit=0.75\linewidth,yunit=0.75\linewidth,trueAngle}%
\begin{pspicture}(1,0.6)%\psgrid
  \pnode(.3,.35){Vk} \pnode(.375,.35){D} \pnode(0,.4){DST1} \pnode(1,.18){DST2}
  \pnode(0,.1){A1}   \pnode(1,.31){A1}
  { \psset{linewidth=.02,linestyle=dashed,linecolor=gray}%
    \pcline(DST1)(DST2) % <- Druckseitentangente
    \pcline(A2)(A1) % <- Anstr\"omrichtung
    \lput*{:U}{\small Anstr\"omrichtung $v_{\infty}$} }%
  \psIntersectionPoint(A1)(A2)(DST1)(DST2){Hk}
  \pscurve(Hk)(.4,.38)(Vk)(.36,.33)(.5,.32)(Hk)
  \psParallelLine[linecolor=red!75!green,arrows=->,arrowscale=2](Vk)(Hk)(D){.1}{FtE}
  \psRelLine[linecolor=red!75!green,arrows=->,arrowscale=2,angle=90](D)(FtE){4}{Fn}% why "4"?
  \psParallelLine[linestyle=dashed](D)(FtE)(Fn){.1}{Fnr1}
  \psRelLine[linestyle=dashed,angle=90](FtE)(D){-4}{Fnr2} % why "-4"?
  \psline[linewidth=1.5pt,arrows=->,arrowscale=2](D)(Fnr2)
  \psIntersectionPoint(D)([nodesep=2]D)(Fnr1)([offset=-4]Fnr1){Fh}
  \psIntersectionPoint(D)([offset=2]D)(Fnr1)([nodesep=4]Fnr1){Fv}
  \psline[linecolor=blue,arrows=->,arrowscale=2](D)(Fh)
  \psline[linecolor=blue,arrows=->,arrowscale=2](D)(Fv)
  \psline[linestyle=dotted](Fh)(Fnr1)  \psline[linestyle=dotted](Fv)(Fnr1)
  \uput{.1}[0](Fh){\blue $F_{H}$}   \uput{.1}[180](Fv){\blue $F_{V}$}
  \uput{.1}[-45](Fnr1){$F_{R}$}     \uput{.1}[90](Fn){\color{red!75!green}$F_{N}$}
  \uput{.25}[-90](FtE){\color{red!75!green}$F_{T}$}
\end{pspicture}
\egroup
\end{center}
\begin{lstlisting}
\psset{xunit=0.75\linewidth,yunit=0.75\linewidth,trueAngle}%
\end{center}
\begin{pspicture}(1,0.6)%\psgrid
  \pnode(.3,.35){Vk} \pnode(.375,.35){D} \pnode(0,.4){DST1} \pnode(1,.18){DST2}
  \pnode(0,.1){A1}   \pnode(1,.31){A1}
  { \psset{linewidth=.02,linestyle=dashed,linecolor=gray}%
    \pcline(DST1)(DST2) % <- Druckseitentangente
    \pcline(A2)(A1) % <- Anstr"omrichtung
    \lput*{:U}{\small Anstr"omrichtung $v_{\infty}$} }%
  \psIntersectionPoint(A1)(A2)(DST1)(DST2){Hk}
  \pscurve(Hk)(.4,.38)(Vk)(.36,.33)(.5,.32)(Hk)
  \psParallelLine[linecolor=red!75!green,arrows=->,arrowscale=2](Vk)(Hk)(D){.1}{FtE}
  \psRelLine[linecolor=red!75!green,arrows=->,arrowscale=2,angle=90](D)(FtE){4}{Fn}% why "4"?
  \psParallelLine[linestyle=dashed](D)(FtE)(Fn){.1}{Fnr1}
  \psRelLine[linestyle=dashed,angle=90](FtE)(D){-4}{Fnr2} % why "-4"?
  \psline[linewidth=1.5pt,arrows=->,arrowscale=2](D)(Fnr2)
  \psIntersectionPoint(D)([nodesep=2]D)(Fnr1)([offset=-4]Fnr1){Fh}
  \psIntersectionPoint(D)([offset=2]D)(Fnr1)([nodesep=4]Fnr1){Fv}
  \psline[linecolor=blue,arrows=->,arrowscale=2](D)(Fh)
  \psline[linecolor=blue,arrows=->,arrowscale=2](D)(Fv)
  \psline[linestyle=dotted](Fh)(Fnr1)  \psline[linestyle=dotted](Fv)(Fnr1)
  \uput{.1}[0](Fh){\blue $F_{H}$}   \uput{.1}[180](Fv){\blue $F_{V}$}
  \uput{.1}[-45](Fnr1){$F_{R}$}     \uput{.1}[90](Fn){\color{red!75!green}$F_{N}$}
  \uput{.25}[-90](FtE){\color{red!75!green}$F_{T}$}
\end{pspicture}
\end{lstlisting}


%--------------------------------------------------------------------------------------
\section{\nxLcs{psParallelLine}}
%--------------------------------------------------------------------------------------
With this macro it is possible to plot lines relative to a given one, which is parallel.
There is no special parameter here.

\begin{lstlisting}[style=syntax]
\psParallelLine(<P0>)(<P1>)(<P2>){<length>}{<end node name>}
\psParallelLine{<arrows>}(<P0>)(<P1>)(<P2>){<length>}{<end node name>}
\psParallelLine[<options>](<P0>)(<P1>)(<P2>){<length>}{<end node name>}
\psParallelLine[<options>]{<arrows>}(<P0>)(<P1>)(<P2>){<length>}{<end node name>}
\end{lstlisting}

The line starts at $P_2$, is parallel to $\overline{P_0P_1}$ and
the length of this parallel line depends on the length factor. The
end node name must be a valid nodename and shouldn't contain any
of the special PostScript characters.

\begin{LTXexample}
\begin{pspicture*}(-5,-4)(5,3.5)
  \psgrid[subgriddiv=0,griddots=5]
  \pnode(2,-2){FF}\qdisk(FF){1.5pt}
  \pnode(-5,5){A}\pnode(0,0){O}
  \multido{\nCountA=-2.4+0.4}{9}{%
    \psParallelLine[linecolor=red](O)(A)(0,\nCountA){9}{P1}
    \psline[linecolor=red](0,\nCountA)(FF)
    \psRelLine[linecolor=red](0,\nCountA)(FF){9}{P2}
  }
  \psline[linecolor=blue](A)(FF)
  \psRelLine[linecolor=blue](A)(FF){5}{END1}
  \psline[linewidth=2pt,arrows=->](2,0)(FF)
\end{pspicture*}
\end{LTXexample}


%--------------------------------------------------------------------------------------
\section{\nxLcs{psIntersectionPoint}}
%--------------------------------------------------------------------------------------
This macro calculates the intersection point of two lines, given by the four coordinates.
There is no special parameter here.
\begin{lstlisting}[style=syntax]
\psIntersectionPoint(<P0>)(<P1>)(<P2>)(<P3>){<node name>}
\end{lstlisting}

\begin{LTXexample}[width=5.5cm]
\psset{unit=0.5cm}
\begin{pspicture}(-5,-4)(5,5)
  \psaxes[labelFontSize=\scriptstyle,
    dx=2,Dx=2,dy=2,Dy=2]{->}(0,0)(-5,-4)(5,5)
  \psline[linecolor=red,linewidth=2pt](-5,-1)(5,5)
  \psline[linecolor=blue,linewidth=2pt](-5,3)(5,-4)
  \qdisk(-5,-1){2pt}\uput[-90](-5,-1){A}
  \qdisk(5,5){2pt}\uput[-90](5,5){B}
  \qdisk(-5,3){2pt}\uput[-90](-5,3){C}
  \qdisk(5,-4){2pt}\uput[-90](5,-4){D}
  \psIntersectionPoint(-5,-1)(5,5)(-5,3)(5,-4){IP}
  \qdisk(IP){3pt}\uput{0.3}[90](IP){IP}
  \psline[linestyle=dashed](IP|0,0)(IP)(0,0|IP)
\end{pspicture}
\end{LTXexample}

\clearpage

%--------------------------------------------------------------------------------------
\section[\nxLcs{psCancel}]{\nxLcs{psCancel}\footnotemark}
%--------------------------------------------------------------------------------------
\footnotetext{Thanks to by Stefano Baroni} This macro works like
the \Lcs{cancel} macro from the package of the same name but it
allows as argument any contents, not only letters but also a
complex graphic.

\begin{BDef}
\LcsStar{psCancel}\OptArgs\Largb{contents}%
\end{BDef}

All optional arguments for lines and boxes are valid and can be
used in the usual way. The star option fills the underlying box
rectangle with the linecolor. This can be transparent if
\Lkeyword{opacity} is set to a value less than 1. This can be used
in presentation to strike out words, equations, and graphic
objects. Lines can also be transparent when the option
\Lkeyword{strokeopacity} is used.

\begingroup
\psCancel{A} \psCancel[linecolor=red]{Tikz :-)} \quad
\psCancel[linecolor=blue,doubleline=true]{%
  \readdata{\data}{demo1.data}
  \psset{shift=*,xAxisLabel=x-Axis,yAxisLabel=y-Axis,llx=-13mm,lly=-7mm,
      xAxisLabelPos={c,-1},yAxisLabelPos={-7,c}}
  \pstScalePoints(1,0.00000001){}{}
  \begin{psgraph}[axesstyle=frame,xticksize=0 7.5,yticksize=0 25,subticksize=1,
     ylabelFactor=\cdot 10^8,Dx=5,Dy=1,xsubticks=2](0,0)(25,7.5){5.5cm}{5cm}
  \listplot[linecolor=red, linewidth=2pt, showpoints=true]{\data}
  \end{psgraph}} \qquad% end of Cancel
\psCancel[linewidth=3pt,linecolor=red,
    strokeopacity=0.5]{\tabular[b]{c}first line\\second line\endtabular}\quad
\psCancel*[linecolor=red!50,opacity=0.5]{\tabular[b]{c}first line\\second line\endtabular}
\quad
\psCancel*[linecolor=blue!30,opacity=0.5]{%
  \readdata{\data}{demo1.data}
  \psset{shift=*,xAxisLabel=x-Axis,yAxisLabel=y-Axis,llx=-15mm,lly=-7mm,urx=1mm,
      xAxisLabelPos={c,-1},yAxisLabelPos={-7,c}}
  \pstScalePoints(1,0.00000001){}{}
  \begin{psgraph}[axesstyle=frame,xticksize=0 7.5,yticksize=0 25,subticksize=1,
     ylabelFactor=\cdot 10^8,Dx=5,Dy=1,xsubticks=2](0,0)(25,7.5){5.5cm}{5cm}
  \listplot[linecolor=red, linewidth=2pt, showpoints=true]{\data}
  \end{psgraph}} \quad% end of Cancel
\psCancel[linewidth=4pt,strokeopacity=0.5]{\parbox{8cm}{\[
  \binom{x_R}{y_R} = \underbrace{r\vphantom{\binom{A}{B}}}_{\text{Scaling}}\cdot
    \underbrace{\begin{pmatrix}
        \sin\gamma & -\cos\gamma \\
      \cos \gamma & \sin \gamma \\
      \end{pmatrix}}_{\text{Rotation}} \binom{x_K}{y_K} +
  \underbrace{\binom{t_x}{t_y}}_{\text{Translation}} \]} }% end of psCancel
\endgroup

\bigskip
\begin{lstlisting}
\psCancel{A} \psCancel[linecolor=red]{Tikz :-)} \quad
\psCancel[linecolor=blue,doubleline=true]{%
  \readdata{\data}{demo1.data}
  \psset{shift=*,xAxisLabel=x-Axis,yAxisLabel=y-Axis,llx=-13mm,lly=-7mm,
      xAxisLabelPos={c,-1},yAxisLabelPos={-7,c}}
  \pstScalePoints(1,0.00000001){}{}
  \begin{psgraph}[axesstyle=frame,xticksize=0 7.5,yticksize=0 25,subticksize=1,
     ylabelFactor=\cdot 10^8,Dx=5,Dy=1,xsubticks=2](0,0)(25,7.5){5.5cm}{5cm}
  \listplot[linecolor=red, linewidth=2pt, showpoints=true]{\data}
  \end{psgraph}} \qquad% end of Cancel
\psCancel[linewidth=3pt,linecolor=red,
    strokeopacity=0.5]{\tabular[b]{c}first line\\second line\endtabular}\quad
\psCancel*[linecolor=red!50,opacity=0.5]{\tabular[b]{c}first line\\second line\endtabular}
\quad
\psCancel*[linecolor=blue!30,opacity=0.5]{%
  \readdata{\data}{demo1.data}
  \psset{shift=*,xAxisLabel=x-Axis,yAxisLabel=y-Axis,llx=-15mm,lly=-7mm,urx=1mm,
      xAxisLabelPos={c,-1},yAxisLabelPos={-7,c}}
  \pstScalePoints(1,0.00000001){}{}
  \begin{psgraph}[axesstyle=frame,xticksize=0 7.5,yticksize=0 25,subticksize=1,
     ylabelFactor=\cdot 10^8,Dx=5,Dy=1,xsubticks=2](0,0)(25,7.5){5.5cm}{5cm}
  \listplot[linecolor=red, linewidth=2pt, showpoints=true]{\data}
  \end{psgraph}} \quad% end of Cancel
\psCancel[linewidth=4pt,strokeopacity=0.5]{\parbox{8cm}{\[
  \binom{x_R}{y_R} = \underbrace{r\vphantom{\binom{A}{B}}}_{\text{Scaling}}\cdot
    \underbrace{\begin{pmatrix}
        \sin\gamma & -\cos\gamma \\
      \cos \gamma & \sin \gamma \\
      \end{pmatrix}}_{\text{Rotation}} \binom{x_K}{y_K} +
  \underbrace{\binom{t_x}{t_y}}_{\text{Translation}} \]} }% end of psCancel
\end{lstlisting}

The optional argument \Lkeyword{cancelType} allows to define the lines for the non star version.
Possible values are \Lkeyval{x} for a cross, \Lkeyval{s} for a slash, and \Lkeyval{b}
for a backslash. It is also possible to use the long words for the \Lkeyval{slash} and the \Lkeyval{backslash}.
An empty value is always assumed as a \Lkeyval{x}.

\begin{LTXexample}[pos=t,wide]
\psset{linewidth=3pt,strokeopacity=0.4}
\psCancel{\tabular[b]{c}first line\\second line\endtabular}   \quad
\psCancel[cancelType=x]{\tabular[b]{c}first line\\second line\endtabular}\quad
\psCancel[cancelType=s]{\tabular[b]{c}first line\\second line\endtabular}\quad
\psCancel[cancelType=b]{\tabular[b]{c}first line\\second line\endtabular}
\end{LTXexample}

\clearpage
%--------------------------------------------------------------------------------------
\section{\nxLcs{psStep}}
%--------------------------------------------------------------------------------------
\Lcs{psStep} calculates a step function for the upper or lower
sum or the max/min of the \Index{Riemann} integral definition of a given
function. The available option is

\Lkeyset{StepType=lower}|\Lkeyval{upper}|\Lkeyval{Riemann}|\Lkeyval{infimum}|\Lkeyval{supremum} or alternative
\Lkeyset{StepType=l}|\Lkeyval{u}|\Lkeyval{R}|\Lkeyval{i}|\Lkeyval{s}

with \Lkeyword{lower} as the default setting. The syntax of the function is

\begin{BDef}
\Lcs{psStep}\OptArgs\Largr(x1,x2)\Largb{n}\Largb{function}
\end{BDef}


(x1,x2) is the given interval for the step wise calculated
function, n is the number of the rectangles and \Larg{function} is
the mathematical function in postfix or algebraic=true notation (with
\Lkeyset{algebraic=true}).

\begin{LTXexample}[pos=t,preset=\centering]
\begin{pspicture}(-0.5,-0.5)(10,3)
 \psaxes[labelFontSize=\scriptstyle]{->}(10,3)
 \psplot[plotpoints=100,linewidth=1.5pt,algebraic=true]{0}{10}{sqrt(x)}
 \psStep[linecolor=magenta,StepType=upper,fillstyle=hlines](0,9){9}{x sqrt}
 \psStep[linecolor=blue,fillstyle=vlines](0,9){9}{x sqrt }
\end{pspicture}
\end{LTXexample}

\begin{LTXexample}[pos=t,preset=\centering]
\psset{plotpoints=200}
\begin{pspicture}(-0.5,-2.25)(10,3)
  \psaxes[labelFontSize=\scriptstyle]{->}(0,0)(0,-2.25)(10,3)
 \psplot[linewidth=1.5pt,algebraic=true]{0}{10}{sqrt(x)*sin(x)}
 \psStep[algebraic=true,linecolor=magenta,StepType=upper](0,9){20}{sqrt(x)*sin(x)}
 \psStep[linecolor=blue,linestyle=dashed](0,9){20}{x sqrt x RadtoDeg sin mul}
\end{pspicture}
\end{LTXexample}

\begin{LTXexample}[pos=t,preset=\centering]
\psset{yunit=1.25cm,plotpoints=200}
\begin{pspicture}(-0.5,-1.5)(10,1.5)
 \psaxes[labelFontSize=\scriptstyle]{->}(0,0)(0,-1.5)(10,1.5)
 \psStep[algebraic=true,StepType=Riemann,fillstyle=solid,fillcolor=black!10](0,10){50}%
    {sqrt(x)*cos(x)*sin(x)}
 \psplot[linewidth=1.5pt,algebraic=true]{0}{10}{sqrt(x)*cos(x)*sin(x)}
\end{pspicture}
\end{LTXexample}


\begin{LTXexample}[pos=t,preset=\centering]
\psset{yunit=1.25cm,plotpoints=200}
\begin{pspicture}(-0.5,-1.5)(10,1.5)
 \psaxes[labelFontSize=\scriptstyle]{->}(0,0)(0,-1.5)(10,1.5)
 \psStep[algebraic=true,StepType=infimum,fillstyle=solid,fillcolor=black!10](0,10){50}%
    {sqrt(x)*cos(x)*sin(x)}
 \psplot[linewidth=1.5pt,algebraic=true]{0}{10}{sqrt(x)*cos(x)*sin(x)}
\end{pspicture}
\end{LTXexample}

\begin{LTXexample}[pos=t,preset=\centering]
\psset{yunit=1.25cm,plotpoints=200}
\begin{pspicture}(-0.5,-1.5)(10,1.5)
 \psaxes[labelFontSize=\scriptstyle]{->}(0,0)(0,-1.5)(10,1.5)
 \psStep[algebraic=true,StepType=supremum,fillstyle=solid,fillcolor=black!10](0,10){50}%
    {sqrt(x)*cos(x)*sin(x)}
 \psplot[linewidth=1.5pt,algebraic=true]{0}{10}{sqrt(x)*cos(x)*sin(x)}
\end{pspicture}
\end{LTXexample}

\begin{LTXexample}[pos=t,preset=\centering]
\psset{unit=1.5cm,plotpoints=200}
\begin{pspicture}[plotpoints=200](-0.5,-3)(10,2.5)
  \psStep[algebraic=true,fillstyle=solid,fillcolor=yellow](0.001,9.5){40}{2*sqrt(x)*cos(ln(x))*sin(x)}
  \psStep[algebraic=true,StepType=Riemann,fillstyle=solid,fillcolor=blue](0.001,9.5){40}{2*sqrt(x)*cos(ln(x))*sin(x)}
  \psaxes[labelFontSize=\scriptstyle]{->}(0,0)(0,-2.75)(10,2.5)
  \psplot[algebraic=true,linecolor=white]{0.001}{9.75}{2*sqrt(x)*cos(ln(x))*sin(x)}
  \uput[90](6,1.2){$f(x)=2\cdot\sqrt{x}\cdot\cos{(\ln{x})}\cdot\sin{x}$}
\end{pspicture}
\end{LTXexample}

\clearpage
%--------------------------------------------------------------------------------------

\section{Tangent lines}
There are two macros for plotting a tangent line or the tangent normal line.
The first one is \Lcs{psTangentLine} which expects three pairs of coordinates,
a $x$ and a $dx$ value. The second one is \Lcs{psplotTangent} which expects 
a function for the curve. \xLkeyword{Tnormal}

\subsection{\nxLcs{psTangentLine} and option \nxLkeyword{Tnormal}}

\begin{BDef}
\Lcs{psTangentLine}\OptArgs\coord1\coord2\coord3\Largb{x}\Largb{dx}
\end{BDef}

\begin{LTXexample}[width=0.45\linewidth,wide]
\psset{unit=2}
\begin{pspicture}[showgrid=true](1,-1)(4,1)
  \pscurve[showpoints=true]
    (2.1,-0.2)(2.5,0.2)(3.2,0.235)(3.8,-0.2)
  \psTangentLine[Tnormal,arrows=->,
    linecolor=red](2.5,0.2)(3.2,0.235)%
      (3.8,-0.2){3}{0.1}
  \psTangentLine[arrows=<->,
    linecolor=blue](2.5,0.2)(3.2,0.235)%
      (3.8,-0.2){3}{0.5}
\end{pspicture}
\end{LTXexample}

In special cases one has to use \Lkeyword{curvature}\verb+=1 1 1+ for the macro \Lcs{pscurve}
to get the same equation for the curve as \Lcs{psplotTangentLine} does.

\begin{LTXexample}[pos=t,preset=\centering,wide]
\psset{unit=2}
\begin{pspicture}[showgrid=true](2,-1)(6,2)
\pscurve[showpoints=true,
  curvature=1 1 1](2.1,-0.2)(2.5,0.2)(3.2,0.235)(5.8,2)
\pscurve[showpoints=true,linecolor=green,
  curvature=1 1 1](2.5,0.2)(3.2,0.235)(5.8,2)
\psTangentLine[Tnormal,arrows=->,linecolor=red](2.5,0.2)(3.2,0.235)(5.8,2){4.6}{0.6}
\psTangentLine[arrows=<->,linecolor=blue](2.5,0.2)(3.2,0.235)(5.8,2){4.5}{0.6}
\end{pspicture}
\end{LTXexample}


The end points are saved as nodes \verb=OCurve=, \verb=ETangent=, and \verb=ENormal=. They can
be used in the default ways for nodes:

\begin{LTXexample}[pos=t,preset=\centering,wide]
\psset{unit=4,arrowscale=2}
\begin{pspicture}(0.1,-0.1)(4,1)
\pscurve[showpoints=true](2.1,-0.2)(2.5,0.2)(3.2,0.4)(3.8,-0.2)
\psTangentLine[Tnormal,arrows=->,linecolor=red](2.5,0.2)(3.2,0.4)(3.8,-0.2){3.5}{0.5}
\psTangentLine[arrows=->,linecolor=blue](2.5,0.2)(3.2,0.4)(3.8,-0.2){3.5}{0.5}
\pcline[linestyle=dashed]{->}(OCurve)(ETangent|OCurve)\naput{$v_x$}
\pcline[linestyle=dashed]{->}(ETangent|OCurve)(ETangent)\naput{$v_y$}% double coordinate (x,y|x,y)
\end{pspicture}
\end{LTXexample}


\subsection{\nxLcs{psplotTangent} and option \nxLkeyword{Tnormal}}
%--------------------------------------------------------------------------------------
There is an additional option, named \Lkeyword{Derive} for an
alternative function (see following example) to calculate the
slope of the tangent. This will be in general the first
derivative, but can also be any other function. If this option is
different to to the default value \Lkeyset{Derive=default}, then this
function is taken to calculate the slope. For the other cases,
\LPack{pstricks-add} builds a secant with -0.00005<x<0.00005,
calculates the slope and takes this for the tangent. This may be
problematic in some cases of special functions or $x$ values, then
it may be appropriate to use the Derive option.

\begin{BDef}
\LcsStar{psplotTangent}\OptArgs\Largb{x}\Largb{dx}\Largb{function}
\end{BDef}



The macro expects three parameters:

\begin{description}
\item[$x$]: the $x$ value of the function for which the tangent should be calculated
\item[$dx$]: the $dx$ to both sides of the $x$ value
\item[$f(x)$]: the function in infix (with option \Lkeyword{algebraic}) or the default
postfix (PostScript) notation
\end{description}

The following examples show the use of the algebraic=true option together with the Derive option.
Remember that using the \Lkeyword{algebraic} option implies that the angles have to be in the
radian unit!

\begin{center}
\bgroup
\def\F{x RadtoDeg dup dup cos exch 2 mul cos add exch 3 mul cos add}
\def\Fp{x RadtoDeg dup dup sin exch 2 mul sin 2 mul add exch 3 mul sin 3 mul add neg}
\psset{plotpoints=1001}
\begin{pspicture}(-7.5,-2.5)(7.5,4)%X\psgrid
  \psaxes{->}(0,0)(-7.5,-2)(7.5,3.5)
  \psplot[linewidth=3\pslinewidth]{-7}{7}{\F}
  \psset{linecolor=red, arrows=<->, arrowscale=2}
  \multido{\n=-7+1}{8}{\psplotTangent{\n}{1}{\F}}
  \psset{linecolor=magenta, arrows=<->, arrowscale=2}%
  \multido{\n=0+1}{8}{\psplotTangent[linecolor=blue, Derive=\Fp]{\n}{1}{\F}}
\end{pspicture}
\egroup
\end{center}

\begin{lstlisting}
\def\F{x RadtoDeg dup dup cos exch 2 mul cos add exch 3 mul cos add}
\def\Fp{x RadtoDeg dup dup sin exch 2 mul sin 2 mul add exch 3 mul sin 3 mul add neg}
\psset{plotpoints=1001}
\begin{pspicture}(-7.5,-2.5)(7.5,4)%X\psgrid
  \psaxes{->}(0,0)(-7.5,-2)(7.5,3.5)
  \psplot[linewidth=3\pslinewidth]{-7}{7}{\F}
  \psset{linecolor=red, arrows=<->, arrowscale=2}
  \multido{\n=-7+1}{8}{\psplotTangent{\n}{1}{\F}}
  \psset{linecolor=magenta, arrows=<->, arrowscale=2}%
  \multido{\n=0+1}{8}{\psplotTangent[linecolor=blue, §\ON§Derive=\Fp§\OFF§]{\n}{1}{\F}}
\end{pspicture}
\end{lstlisting}

The star version plots only the tangent line in the positive $x$-direction:

\begin{center}
\bgroup
\def\Falg{cos(x)+cos(2*x)+cos(3*x)}   \def\Fpalg{-sin(x)-2*sin(2*x)-3*sin(3*x)}
\begin{pspicture}(-7.5,-2.5)(7.5,4)%\psgrid
  \psaxes{->}(0,0)(-7.5,-2)(7.5,3.5)
  \psplot[linewidth=1.5pt,algebraic=true,plotpoints=500]{-7.5}{7.5}{\Falg}
  \multido{\n=-7+1}{8}{\psplotTangent*[linecolor=red,arrows=->,arrowscale=2,algebraic=true]{\n}{1}{\Falg}}
  \multido{\n=0+1}{8}{\psplotTangent*[linecolor=magenta,%
     arrows=->,arrowscale=2,algebraic=true,Derive={\Fpalg}]{\n}{1}{\Falg}}
\end{pspicture}
\egroup
\end{center}

\begin{lstlisting}
\def\Falg{cos(x)+cos(2*x)+cos(3*x)}   \def\Fpalg{-sin(x)-2*sin(2*x)-3*sin(3*x)}
\begin{pspicture}(-7.5,-2.5)(7.5,4)%\psgrid
  \psaxes{->}(0,0)(-7.5,-2)(7.5,3.5)
  \psplot[linewidth=1.5pt,algebraic=true,plotpoints=500]{-7.5}{7.5}{\Falg}
  \multido{\n=-7+1}{8}{\psplotTangent*[linecolor=red,arrows=->,arrowscale=2,algebraic=true]{\n}{1}{\Falg}}
  \multido{\n=0+1}{8}{\psplotTangent*[linecolor=magenta,%
     arrows=->,arrowscale=2,algebraic=true,Derive={\Fpalg}]{\n}{1}{\Falg}}
\end{pspicture}
\end{lstlisting}

The next example shows the use of the \Lkeyword{Derive} option to draw
the perpendicular line to the tangent.

\begin{LTXexample}[width=8cm,wide]
\begin{pspicture}(-0.5,-0.5)(7.25,7.25)
  \def\Func{10 x div}
  \psaxes[arrowscale=1.5]{->}(7,7)
  \psplot[linewidth=2pt,algebraic=true]{1.5}{5}{10/x}
  \psplotTangent[linewidth=.5\pslinewidth,linecolor=red,algebraic=true]{3}{2}{10/x}
  \psplotTangent[linewidth=.5\pslinewidth,linecolor=blue,algebraic=true,Derive=(x*x)/10]{3}{2}{10/x}
  \psline[linestyle=dashed](!0 /x 3 def \Func)(!3 /x 3 def \Func)(3,0)
\end{pspicture}
\end{LTXexample}

By setting the optional argument \Lkeyword{Tnormal} one can plot the
normal of the tangent line. It always starts at the given point.

\begin{LTXexample}[width=8cm,wide]
\begin{pspicture}(-0.5,-0.5)(7.25,7.25)
  \def\Func{10 x div}
  \psaxes[arrowscale=1.5]{->}(7,7)
  \psplot[linewidth=2pt]{1.5}{5}{\Func}
  \psplotTangent[linewidth=1.5\pslinewidth,linecolor=red]{3}{2}{\Func}
  \psplotTangent[linewidth=1.5\pslinewidth,linecolor=blue,Tnormal]{3}{2}{\Func}
  \psline[linestyle=dashed](!0 /x 3 def \Func)(!3 /x 3 def \Func)(3,0)
\end{pspicture}
\end{LTXexample}


Let's work with the classical \Index{cardioid}: $r=2(1+\cos(\theta))$ and
$\displaystyle \frac{d r}{d\theta}=-2\sin(\theta)$. The \Lkeyword{Derive}
option always expects the $\frac{d r}{d\theta}$ value and uses
internally the equation for the derivative of implicitly defined
functions:

\[
\frac{dy}{dx}=\frac{r^\prime\cdot\sin\theta + x}{r^\prime\cdot\cos\theta - y}
\]
where $x=r\cdot\cos\theta$ and $y=r\cdot\sin\theta$


\begin{LTXexample}[width=6cm,wide]
\begin{pspicture}(-1,-3)(5,3)%\psgrid[subgridcolor=lightgray]
  \psaxes{->}(0,0)(-1,-3)(5,3)
  \psplot[polarplot,linewidth=3\pslinewidth,linecolor=blue,%
     plotpoints=500]{0}{360}{1 x cos add 2 mul}
\end{pspicture}
\end{LTXexample}

\psset{algebraic=false}
\begin{LTXexample}[width=6cm,wide]
\begin{pspicture}(-1,-3)(5,3)%\psgrid[subgridcolor=lightgray]
  \psaxes{->}(0,0)(-1,-3)(5,3)
  \psplot[polarplot,linewidth=3\pslinewidth,linecolor=blue,plotpoints=500]{0}{360}{1 x cos add 2 mul}
  \multido{\n=0+36}{10}{%
     \psplotTangent[polarplot,linecolor=red,arrows=<->]{\n}{1.5}{1 x cos add 2 mul} }
\end{pspicture}
\end{LTXexample}

\begin{LTXexample}[width=6cm,wide]
\begin{pspicture}(-1,-3)(5,3)%\psgrid[subgridcolor=lightgray]
  \psaxes{->}(0,0)(-1,-3)(5,3)
  \psplot[polarplot,linewidth=3\pslinewidth,linecolor=blue,algebraic=true,plotpoints=500]{0}{6.289}{2*(1+cos(x))}
  \multido{\r=0.000+0.314}{21}{%
     \psplotTangent[polarplot,Derive=-2*sin(x),algebraic=true,linecolor=red,arrows=<->]{\r}{1.5}{2*(1+cos(x))} }
\end{pspicture}
\end{LTXexample}


Let's work with a \Index{Lissajou curve}:
 $\displaystyle\left\{\begin{array}{l}x=3.5\cos(2t)\\y=3.5\sin(6t)\end{array}\right.$
whose derivative is :
 $\displaystyle\left\{\begin{array}{l}x=-7\sin(2t)\\y=21\cos(6t)\end{array}\right.$

The parameter must be the letter $t$ instead of $x$ and when using
the \Lkeyword{algebraic=true} option you must separate the two equations by
a \Lnotation{|} (see example).

\begin{LTXexample}[pos=t,wide]
\def\Lissa{t dup 2 RadtoDeg mul cos 3.5 mul exch 6 mul RadtoDeg sin 3.5 mul}%
\psset{yunit=0.6}
\begin{pspicture}(-4,-4)(4,6)
  \parametricplot[plotpoints=500,linewidth=3\pslinewidth]{0}{3.141592}{\Lissa}
  \multido{\r=0.000+0.314}{11}{%
    \psplotTangent[linecolor=red,arrows=<->]{\r}{1.5}{\Lissa} }
  \multido{\r=0.157+0.314}{11}{%
    \psplotTangent[linecolor=blue,arrows=<->]{\r}{1.5}{\Lissa} }
\end{pspicture}\hfill%
\def\LissaAlg{3.5*cos(2*t)|3.5*sin(6*t)}  \def\LissaAlgDer{-7*sin(2*t)|21*cos(6*t)}%
\begin{pspicture}(-4,-4)(4,6)
  \parametricplot[algebraic=true,plotpoints=500,linewidth=3\pslinewidth]{0}{3.141592}{\LissaAlg}
  \multido{\r=0.000+0.314}{11}{%
    \psplotTangent[algebraic=true,linecolor=red,arrows=<->]{\r}{1.5}{\LissaAlg} }
  \multido{\r=0.157+0.314}{11}{%
    \psplotTangent[algebraic=true,linecolor=blue,arrows=<->,%
       Derive=\LissaAlgDer]{\r}{1.5}{\LissaAlg} }
\end{pspicture}
\end{LTXexample}


\clearpage
\section{Successive derivatives of a function}

The new PostScript function \Lps{Derive} has been added for
plotting successive derivatives of a function. It must be used
with the \Lkeyword{algebraic=true} option. This function has two arguments:

\begin{enumerate}
\item a positive integer which defines the order of the derivative; obviously $0$ means the
  function itself!
\item a function of variable $x$ which can be any function using common operators,
\end{enumerate}

Do not think that the derivative is approximated, the internal PostScript engine will
compute the real derivative using a formal derivative engine.

The following diagram contains the plot of the polynomial:

\[ f(x)=\sum_{i=0}^{14}\frac{(-1)^{i}x^{2i}}{i!}=1-\frac{x^2}{2}+\frac{x^4}{4!}-\frac{x^6}{6!}+\frac{x^8}{8!}-
          \frac{x^{10}}{10!}+\frac{x^{12}}{12!}-\frac{x^{14}}{14!}\]

and of its first 15 derivatives. It is the sequence definition of
the cosine.


\begin{LTXexample}[pos=t,wide,preset=\centering]
\psset{unit=2}
\def\getColor#1{\ifcase#1 Tan\or RedOrange\or magenta\or yellow\or green\or Orange\or blue\or
  DarkOrchid\or BrickRed\or Rhodamine\or OliveGreen\or Goldenrod\or Mahogany\or
  OrangeRed\or CarnationPink\or RoyalPurple\or Lavender\fi}
\begin{pspicture}[showgrid=true](0,-1.2)(7,1.5)
  \psclip{\psframe[linestyle=none](0,-1.1)(7,1.1)}
  \multido{\in=0+1}{16}{%
     \psplot[linewidth=1pt,algebraic=true,linecolor=\getColor{\in}]{0}{7}
      {Derive(\in,1-x^2/2+x^4/24-x^6/720+x^8/40320-x^10/3628800+x^12/479001600-x^14/87178291200)}}
  \endpsclip
\end{pspicture}
\end{LTXexample}

\begin{LTXexample}[width=3.5cm]
\begin{pspicture}[shift=-2.5,showgrid=true,linewidth=1pt](0,-2)(3,3)
  \psplot[algebraic=true]{.001}{3}{x*ln(x)}  % f(x)
  \psplot[algebraic=true,linecolor=red]{.05}{3}{Derive(1,x*ln(x))} % f'(x)=1+ln(x)
\end{pspicture}
\end{LTXexample}


\clearpage
\section{Variable step for plotting a curve}
\subsection{Theory}

As you know with the \Lcs{psplot} macro, the curve is plotted
using a piece-wise linear curve. The step is given by the
parameter \Lkeyword{plotpoints}. For each step between $x_i$ and
$x_{i+1}$, the area defined between the curve and its
approximation (a segment) is majored by this formula :

\begin{minipage}[m]{.5\linewidth}
\[|\varepsilon|\le\frac{M_2(f)(x_{i+1}-x_i)^3}{12}\]

$M_2(f)$ is a majorant of the second derivative of $f$ in the interval $[x_i;x_{i+1}]$.
\end{minipage}
{\psset{unit=1cm, showpoints=false}
\begin{pspicture}[shift=-2,showgrid=true](0,-1)(6,3)
  \pscurve(0,0)(1,1)(3,2.2)(5,2)(6,1)\psline(1,1)(5,2)
  \psline(.5,0)(5.5,0)\psline(1,0)(1,1)\psline(5,0)(5,2)
  \rput[t](1,-.1){$x_n$}\rput[t](5,-.1){$x_{n+1}$}
  \psclip{\pscustom{\psecurve(0,0)(1,1)(3,2.2)(5,2)(6,1)\psline(5,2)}}
    \psframe[fillstyle=solid, fillcolor=gray](0,0)(5,5)
  \endpsclip
  \rput*(3,1.8){$\varepsilon$}
\end{pspicture}}



The parameter \Lkeyword{VarStep} (\false\ by default) activates
the variable step algorithm. It is set to a tolerance defined by
the parameter \Lkeyword{VarStepEpsilon} (\Lkeyval{default} by default,
accept real value). If this parameter is not set by the user, then
it is automatically computed using the default first step given by
the parameter \Lkeyword{plotpoints}. Then, for each step, $f''(x_n)$
and $f''(x_{n+1})$ are computed and the smaller is used as
$M_2(f)$, and then the step is approximated. This means that the
step is constant for second order polynomials.

\subsection{The cosine}

Different value for the tolerance from $0.01$ to $0.000\,1$, a factor $10$ between
each of them. In black, there is the classic \Lcs{psplot} behavior, and in
magenta the default variable step behavior.

\begin{center}
\bgroup
\psset{algebraic=true, VarStep=true, unit=2, showpoints=true, linecolor=red}
\begin{pspicture}(-0,-1)(3.14,2)\psgrid
  \psplot[VarStepEpsilon=.01]{0}{3.14}{cos(x)}
  \psplot[VarStepEpsilon=.001]{0}{3.14}{cos(x)+.15}
  \psplot[VarStepEpsilon=.0001]{0}{3.14}{cos(x)+.3}
  \psplot[linecolor=magenta]{0}{3.14}{cos(x)+.45}
  \psplot[VarStep=false, linewidth=2\pslinewidth, linecolor=black]{-0}{3.14}{cos(x)+.6}
\end{pspicture}
\egroup
\end{center}

\begin{lstlisting}
\psset{algebraic=true, VarStep=true, unit=2, showpoints=true, linecolor=red}
\begin{pspicture}[showgrid=true](-0,-1)(3.14,2)
  \psplot[VarStepEpsilon=.01]{0}{3.14}{cos(x)}
  \psplot[VarStepEpsilon=.001]{0}{3.14}{cos(x)+.15}
  \psplot[VarStepEpsilon=.0001]{0}{3.14}{cos(x)+.3}
  \psplot[linecolor=magenta]{0}{3.14}{cos(x)+.45}
  \psplot[VarStep=false,linewidth=1pt,linecolor=black]{-0}{3.14}{cos(x)+.6}
\end{pspicture}
\end{lstlisting}


\subsection{The Napierian Logarithm}

A really classic example which gives a bad beginning, the tolerance is set to $0.001$.

\begin{center}
\bgroup
\psset{algebraic=true, VarStep=true, linecolor=red, showpoints=true}
\begin{pspicture}[showgrid=true](0,-5)(16,4)
  \psplot[VarStep=false, linecolor=black]{.01}{16}{ln(x)+1}
  \psplot[linecolor=magenta]{.51}{16}{ln(x-1/2)+1/2}
  \psplot[VarStepEpsilon=.001]{1.01}{16}{ln(x-1)}
  \psplot[VarStepEpsilon=.01]{1.51}{16}{ln(x-1.5)-100/200}
\end{pspicture}
\egroup
\end{center}

\begin{lstlisting}
\psset{algebraic=true, VarStep=true, linecolor=red, showpoints=true}
\begin{pspicture}[showgrid=true](0,-5)(16,4)
  \psplot[VarStep=false, linecolor=black]{.01}{16}{ln(x)+1}
  \psplot[linecolor=magenta]{.51}{16}{ln(x-1/2)+1/2}
  \psplot[VarStepEpsilon=.001]{1.01}{16}{ln(x-1)}
  \psplot[VarStepEpsilon=.01]{1.51}{16}{ln(x-1.5)-100/200}
\end{pspicture}
\end{lstlisting}


\clearpage
\subsection{Sine of the inverse of $x$}
Impossible to draw, but let's try!

\begin{center}
\bgroup
\psset{xunit=64,algebraic=true,VarStep,linecolor=red,showpoints=true,linewidth=1pt}
\begin{pspicture}[showgrid=true](0,-1)(.5,1)
  \psplot[VarStepEpsilon=.0001]{.01}{.25}{sin(1/x)}
\end{pspicture}\\
\begin{pspicture}[showgrid=true](0,-1)(.5,1)
  \psplot[VarStepEpsilon=.00001]{.01}{.25}{sin(1/x)}
\end{pspicture}\\
\begin{pspicture}[showgrid=true](0,-1)(.5,1)
  \psplot[VarStepEpsilon=.000001]{.01}{.25}{sin(1/x)}
\end{pspicture}\\
\begin{pspicture}[showgrid=true](0,-1)(.5,1)
  \psplot[VarStep=false, linecolor=black]{.01}{.25}{sin(1/x)}
\end{pspicture}
\egroup
\end{center}

\begin{lstlisting}
\psset{xunit=64,algebraic=true,VarStep,linecolor=red,showpoints=true,linewidth=1pt}
\begin{pspicture}[showgrid=true](0,-1)(.5,1)
  \psplot[VarStepEpsilon=.0001]{.01}{.25}{sin(1/x)}
\end{pspicture}\\
\begin{pspicture}[showgrid=true](0,-1)(.5,1)
  \psplot[VarStepEpsilon=.00001]{.01}{.25}{sin(1/x)}
\end{pspicture}\\
\begin{pspicture}[showgrid=true](0,-1)(.5,1)
  \psplot[VarStepEpsilon=.000001]{.01}{.25}{sin(1/x)}
\end{pspicture}\\
\begin{pspicture}[showgrid=true](0,-1)(.5,1)
  \psplot[VarStep=false, linecolor=black]{.01}{.25}{sin(1/x)}
\end{pspicture}
\end{lstlisting}





\clearpage
\subsection{A really complicated function}

Just appreciate the difference between the normal behavior and the plotting with the
\Lkeyword{varStep} option. The function is:

\[f(x)=x-\frac{x^2}{10}+\ln(x)+\cos(2x)+\sin(x^2)-1\]

\begin{center}
\bgroup
\psset{xunit=3, algebraic=true, VarStep, showpoints=true}
\begin{pspicture}[showgrid=true](0,-2)(5,6)
  \psplot[VarStepEpsilon=.0005, linecolor=red]{.1}{5}{x-x^2/10+ln(x)+cos(2*x)+sin(x^2)}
  \psplot[linecolor=magenta]{.1}{5}{x-x^2/10+ln(x)+cos(2*x)+sin(x^2)+.5}
  \psplot[VarStep=false]{.1}{5}{x-x^2/10+ln(x)+cos(2*x)+sin(x^2)-1}
\end{pspicture}
\egroup
\end{center}

\begin{lstlisting}
\psset{xunit=3, algebraic=true, VarStep, showpoints=true}
\begin{pspicture}[showgrid=true](0,-2)(5,6)
  \psplot[VarStepEpsilon=.0005, linecolor=red]{.1}{5}{x-x^2/10+ln(x)+cos(2*x)+sin(x^2)}
  \psplot[linecolor=magenta]{.1}{5}{x-x^2/10+ln(x)+cos(2*x)+sin(x^2)+.5}
  \psplot[VarStep=false]{.1}{5}{x-x^2/10+ln(x)+cos(2*x)+sin(x^2)-1}
\end{pspicture}
\end{lstlisting}


\clearpage
\subsection{A hyperbola}

\begin{center}
\bgroup
\psset{algebraic=true, showpoints=true, unit=0.75}
\begin{pspicture}(-5,-4)(9,6)
  \psplot[linecolor=black]{-5}{1.8}{(x-1)/(x-2)}
  \psplot[VarStep=true, VarStepEpsilon=.001, linecolor=red]{2.2}{9}{(x-1)/(x-2)}
  \psaxes{->}(0,0)(-5,-4)(9,6)
\end{pspicture}
\egroup
\end{center}

\begin{lstlisting}
\psset{algebraic=true, showpoints=true, unit=0.75}
\begin{pspicture}(-5,-4)(9,6)
  \psplot[linecolor=black]{-5}{1.8}{(x-1)/(x-2)}
  \psplot[VarStep=true, VarStepEpsilon=.001, linecolor=red]{2.2}{9}{(x-1)/(x-2)}
  \psaxes{->}(0,0)(-5,-4)(9,6)
\end{pspicture}
\end{lstlisting}



\clearpage
\subsection{Using \nxLcs{psparametricplot}}

\begin{BDef}
\Lcs{parametricplot}\OptArgs\Largb{t0}\Largb{t1}\OptArg{PS commands}\Largb{x(t) y(t)}
\end{BDef}

\begin{center}
\bgroup
\psset{unit=2.5}
\begin{pspicture}[showgrid=true](-1,-1)(1,1)
\parametricplot[algebraic=true,linecolor=red,VarStep=true, showpoints=true,
                VarStepEpsilon=.0001]
                {-3.14}{3.14}{cos(3*t)|sin(2*t)}
\end{pspicture}
\begin{pspicture}[showgrid=true](-1,-1)(1,1)
\parametricplot[algebraic=true,linecolor=blue,VarStep=true, showpoints=false,
                VarStepEpsilon=.0001]
                {-3.14}{3.14}{cos(3*t)|sin(2*t)}
\end{pspicture}
\egroup
\end{center}

\begin{lstlisting}
\psset{unit=3}
\begin{pspicture}[showgrid=true](-1,-1)(1,1)
\parametricplot[algebraic=true,linecolor=red,VarStep=true, showpoints=true,
                VarStepEpsilon=.0001]
                {-3.14}{3.14}{cos(3*t)|sin(2*t)}
\end{pspicture}
\begin{pspicture}[showgrid=true](-1,-1)(1,1)
\parametricplot[algebraic=true,linecolor=blue,VarStep=true, showpoints=false,
                VarStepEpsilon=.0001]
                {-3.14}{3.14}{cos(3*t)|sin(2*t)}
\end{pspicture}
\end{lstlisting}


\begin{center}
\bgroup
\psset{unit=2.5}
\begin{pspicture}[showgrid=true](-1,-1)(1,1)
\parametricplot[algebraic=true,linecolor=red,VarStep=true, showpoints=true,
                VarStepEpsilon=.0001]
                {0}{47.115}{cos(5*t)|sin(3*t)}
\end{pspicture}
\begin{pspicture}[showgrid=true](-1,-1)(1,1)
\parametricplot[algebraic=true,linecolor=blue,VarStep=true, showpoints=false,
                VarStepEpsilon=.0001]
                {0}{47.115}{cos(5*t)|sin(3*t)}
\end{pspicture}
\egroup
\end{center}

\begin{lstlisting}
\psset{unit=2.5}
\begin{pspicture}[showgrid=true](-1,-1)(1,1)
\parametricplot[algebraic=true,linecolor=red,VarStep=true, showpoints=true,
                VarStepEpsilon=.0001]
                {0}{47.115}{cos(5*t)|sin(3*t)}
\end{pspicture}
\begin{pspicture}[showgrid=true](-1,-1)(1,1)
\parametricplot[algebraic=true,linecolor=blue,VarStep=true, showpoints=false,
                VarStepEpsilon=.0001]
                {0}{47.115}{cos(5*t)|sin(3*t)}
\end{pspicture}
\end{lstlisting}


\begin{center}
\bgroup
\psset{xunit=.5}
\begin{pspicture}[showgrid=true](0,0)(12.566,2)
\parametricplot[algebraic=true,linecolor=red,VarStep, showpoints=true,
        VarStepEpsilon=.01]{0}{12.566}{t+cos(-t-Pi/2)|1+sin(-t-Pi/2)}
\end{pspicture}
%
\begin{pspicture}[showgrid=true](0,0)(12.566,2)
\parametricplot[algebraic=true,linecolor=blue,VarStep, showpoints=false,
        VarStepEpsilon=.001]{0}{12.566}{t+cos(-t-Pi/2)|1+sin(-t-Pi/2)}
\end{pspicture}
\egroup
\end{center}

\begin{lstlisting}
\psset{xunit=.5}
\begin{pspicture}[showgrid=true](0,0)(12.566,2)
\parametricplot[algebraic=true,linecolor=red,VarStep, showpoints=true,
        VarStepEpsilon=.01]{0}{12.566}{t+cos(-t-Pi/2)|1+sin(-t-Pi/2)}
\end{pspicture}
%
\begin{pspicture}[showgrid=true](0,0)(12.566,2)
\parametricplot[algebraic=true,linecolor=blue,VarStep, showpoints=false,
        VarStepEpsilon=.001]{0}{12.566}{t+cos(-t-Pi/2)|1+sin(-t-Pi/2)}
\end{pspicture}
\end{lstlisting}


\section{New math functions and their derivatives}

\subsection{The inverse sine and its derivative}

\begin{center}
\bgroup
\psset{unit=1.5}
\begin{pspicture}[showgrid=true](-1,-2)(1,2)
  \psplot[linecolor=blue,algebraic=true]{-1}{1}{asin(x)}
\end{pspicture}
\hspace{1em}
\psset{algebraic, VarStep, VarStepEpsilon=.001, showpoints=true}
\begin{pspicture}[showgrid=true](-1,-2)(1,2)
  \psplot[linecolor=blue]{-.999}{.999}{asin(x)}
\end{pspicture}
\hspace{1em}
\begin{pspicture}[showgrid=true](-1,0)(1,4)
  \psplot[linecolor=blue]{-.97}{.97}{Derive(1,asin(x))}
\end{pspicture}
\hspace{1em}
\psset{algebraic=true, VarStep, VarStepEpsilon=.0001, showpoints=true}
\begin{pspicture}[showgrid=true](-1,0)(1,4)
  \psplot[linecolor=blue]{-.97}{.97}{Derive(1,asin(x))}
\end{pspicture}
\egroup
\end{center}

\begin{lstlisting}
\psset{unit=1.5}
\begin{pspicture}[showgrid=true](-1,-2)(1,2)
  \psplot[linecolor=blue,algebraic=true]{-1}{1}{asin(x)}
\end{pspicture}
\hspace{1em}
\psset{algebraic=true, VarStep, VarStepEpsilon=.001, showpoints=true}
\begin{pspicture}[showgrid=true](-1,-2)(1,2)
  \psplot[linecolor=blue]{-.999}{.999}{asin(x)}
\end{pspicture}
\hspace{1em}
\begin{pspicture}[showgrid=true](-1,0)(1,4)
  \psplot[linecolor=red]{-.97}{.97}{Derive(1,asin(x))}
\end{pspicture}
\hspace{1em}
\psset{algebraic=true, VarStep, VarStepEpsilon=.0001, showpoints=true}
\begin{pspicture}[showgrid=true](-1,0)(1,4)
  \psplot[linecolor=red]{-.97}{.97}{Derive(1,asin(x))}
\end{pspicture}
\end{lstlisting}


\subsection{The inverse cosine and its derivative}

\begin{center}
\bgroup
\psset{unit=1.5}
\begin{pspicture}[showgrid=true](-1,0)(1,3)
  \psplot[linecolor=blue,algebraic=true]{-1}{1}{acos(x)}
\end{pspicture}
\hspace{1em}
\psset{algebraic=true, VarStep, VarStepEpsilon=.001, showpoints=true}
\begin{pspicture}[showgrid=true](-1,0)(1,3)
  \psplot[linecolor=blue]{-.999}{.999}{acos(x)}
\end{pspicture}
\hspace{1em}
\begin{pspicture}[showgrid=true](-1,-4)(1,-1)
  \psplot[linecolor=blue]{-.97}{.97}{Derive(1,acos(x))}
\end{pspicture}
\hspace{1em}
\psset{algebraic=true, VarStep, VarStepEpsilon=.0001, showpoints=true}
\begin{pspicture}[showgrid=true](-1,-4)(1,-1)
  \psplot[linecolor=blue]{-.97}{.97}{Derive(1,acos(x))}
\end{pspicture}
\egroup
\end{center}

\begin{lstlisting}
\psset{unit=1.5}
\begin{pspicture}[showgrid=true](-1,0)(1,3)
  \psplot[linecolor=blue,algebraic=true]{-1}{1}{acos(x)}
\end{pspicture}
\hspace{1em}
\psset{algebraic=true, VarStep, VarStepEpsilon=.001, showpoints=true}
\begin{pspicture}[showgrid=true](-1,0)(1,3)
  \psplot[linecolor=blue]{-.999}{.999}{acos(x)}
\end{pspicture}
\hspace{1em}
\begin{pspicture}[showgrid=true](-1,-4)(1,-1)
  \psplot[linecolor=red]{-.97}{.97}{Derive(1,acos(x))}
\end{pspicture}
\hspace{1em}
\psset{algebraic=true, VarStep, VarStepEpsilon=.0001, showpoints=true}
\begin{pspicture}[showgrid=true](-1,-4)(1,-1)
  \psplot[linecolor=red]{-.97}{.97}{Derive(1,acos(x))}
\end{pspicture}
\end{lstlisting}



\subsection{The inverse tangent and its derivative}

\begin{center}
\bgroup
\begin{pspicture}[showgrid=true](-4,-2)(4,2)
\psset{algebraic=true}
  \psplot[linecolor=blue,linewidth=1pt]{-4}{4}{atg(x)}
  \psplot[linecolor=red,VarStep, VarStepEpsilon=.0001, showpoints=true]{-4}{4}{Derive(1,atg(x))}
\end{pspicture}
\hspace{1em}
\begin{pspicture}[showgrid=true](-4,-2)(4,2)
\psset{algebraic=true, VarStep, VarStepEpsilon=.001, showpoints=true}
  \psplot[linecolor=blue]{-4}{4}{atg(x)}
  \psplot[linecolor=red]{-4}{4}{Derive(1,atg(x))}
\end{pspicture}
\egroup
\end{center}

\begin{lstlisting}
\begin{pspicture}[showgrid=true](-4,-2)(4,2)
\psset{algebraic=true}
  \psplot[linecolor=blue,linewidth=1pt]{-4}{4}{atg(x)}
  \psplot[linecolor=red,VarStep, VarStepEpsilon=.0001, showpoints=true]{-4}{4}{Derive(1,atg(x))}
\end{pspicture}
\hspace{1em}
\begin{pspicture}[showgrid=true](-4,-2)(4,2)
\psset{algebraic=true, VarStep, VarStepEpsilon=.001, showpoints=true}
  \psplot[linecolor=blue]{-4}{4}{atg(x)}
  \psplot[linecolor=red]{-4}{4}{Derive(1,atg(x))}
\end{pspicture}
\end{lstlisting}

\subsection{Hyperbolic functions}

\begin{center}
\bgroup
\begin{pspicture}(-3,-4)(3,4)
\psset{algebraic=true}
  \psplot[linecolor=red,linewidth=1pt]{-2}{2}{sh(x)}
  \psplot[linecolor=blue,linewidth=1pt]{-2}{2}{ch(x)}
  \psplot[linecolor=green,linewidth=1pt]{-3}{3}{th(x)}
  \psaxes{->}(0,0)(-3,-4)(3,4)
\end{pspicture}
\hspace{1em}
\begin{pspicture}(-3,-4)(3,4)
\psset{algebraic=true, VarStep=true, VarStepEpsilon=.001, showpoints=true}
  \psplot[linecolor=red,linewidth=1pt]{-2}{2}{sh(x)}
  \psplot[linecolor=blue,linewidth=1pt]{-2}{2}{ch(x)}
  \psplot[linecolor=green,linewidth=1pt]{-3}{3}{th(x)}
  \psaxes{->}(0,0)(-3,-4)(3,4)
\end{pspicture}
\egroup
\end{center}

\begin{lstlisting}
\begin{pspicture}(-3,-4)(3,4)
\psset{algebraic=true}
  \psplot[linecolor=red,linewidth=1pt]{-2}{2}{sh(x)}
  \psplot[linecolor=blue,linewidth=1pt]{-2}{2}{ch(x)}
  \psplot[linecolor=green,linewidth=1pt]{-3}{3}{th(x)}
  \psaxes{->}(0,0)(-3,-4)(3,4)
\end{pspicture}
\hspace{1em}
\begin{pspicture}(-3,-4)(3,4)
\psset{algebraic=true, VarStep=true, VarStepEpsilon=.001, showpoints=true}
  \psplot[linecolor=red,linewidth=1pt]{-2}{2}{sh(x)}
  \psplot[linecolor=blue,linewidth=1pt]{-2}{2}{ch(x)}
  \psplot[linecolor=green,linewidth=1pt]{-3}{3}{th(x)}
  \psaxes{->}(0,0)(-3,-4)(3,4)
\end{pspicture}
\end{lstlisting}



\begin{center}
\bgroup
\begin{pspicture}(-3,-4)(3,4)
\psset{algebraic=true}
  \psplot[linecolor=red,linewidth=1pt]{-2}{2}{Derive(1,sh(x))}
  \psplot[linecolor=blue,linewidth=1pt]{-2}{2}{Derive(1,ch(x))}
  \psplot[linecolor=green,linewidth=1pt]{-3}{3}{Derive(1,th(x))}
  \psaxes{->}(0,0)(-3,-4)(3,4)
\end{pspicture}
\hspace{1em}
\begin{pspicture}(-3,-4)(3,4)
\psset{algebraic=true, VarStep=true, VarStepEpsilon=.001, showpoints=true}
  \psplot[linecolor=red,linewidth=1pt]{-2}{2}{Derive(1,sh(x))}
  \psplot[linecolor=blue,linewidth=1pt]{-2}{2}{Derive(1,ch(x))}
  \psplot[linecolor=green,linewidth=1pt]{-3}{3}{Derive(1,th(x))}
  \psaxes{->}(0,0)(-3,-4)(3,4)
\end{pspicture}
\egroup
\end{center}

\begin{lstlisting}
\begin{pspicture}(-3,-4)(3,4)
\psset{algebraic=true,linewidth=1pt}
  \psplot[linecolor=red,linewidth=1pt]{-2}{2}{Derive(1,sh(x))}
  \psplot[linecolor=blue,linewidth=1pt]{-2}{2}{Derive(1,ch(x))}
  \psplot[linecolor=green,linewidth=1pt]{-3}{3}{Derive(1,th(x))}
  \psaxes{->}(0,0)(-3,-4)(3,4)
\end{pspicture}
\hspace{1em}
\begin{pspicture}(-3,-4)(3,4)
\psset{algebraic=true, VarStep=true, VarStepEpsilon=.001, showpoints=true}
  \psplot[linecolor=red,linewidth=1pt]{-2}{2}{Derive(1,sh(x))}
  \psplot[linecolor=blue,linewidth=1pt]{-2}{2}{Derive(1,ch(x))}
  \psplot[linecolor=green,linewidth=1pt]{-3}{3}{Derive(1,th(x))}
  \psaxes{->}(0,0)(-3,-4)(3,4)
\end{pspicture}
\end{lstlisting}



\begin{center}
\bgroup
\begin{pspicture}(-7,-3)(7,3)
\psset{algebraic=true}
  \psplot[linecolor=red,linewidth=1pt]{-7}{7}{Argsh(x)}
  \psplot[linecolor=blue,linewidth=1pt]{1}{7}{Argch(x)}
  \psplot[linecolor=green,linewidth=1pt]{-.99}{.99}{Argth(x)}
  \psaxes{->}(0,0)(-7,-3)(7,3)
\end{pspicture}\\[\baselineskip]
\begin{pspicture}(-7,-3)(7,3)
  \psset{algebraic=true, VarStep, VarStepEpsilon=.001, showpoints=true}
  \psplot[linecolor=red,linewidth=1pt]{-7}{7}{Argsh(x)}
  \psplot[linecolor=blue,linewidth=1pt]{1.001}{7}{Argch(x)}
  \psplot[linecolor=green,linewidth=1pt]{-.99}{.99}{Argth(x)}
  \psaxes{->}(0,0)(-7,-3)(7,3)
\end{pspicture}
\egroup
\end{center}

\begin{lstlisting}
\begin{pspicture}(-7,-3)(7,3)
\psset{algebraic=true}
  \psplot[linecolor=red,linewidth=1pt]{-7}{7}{Argsh(x)}
  \psplot[linecolor=blue,linewidth=1pt]{1}{7}{Argch(x)}
  \psplot[linecolor=green,linewidth=1pt]{-.99}{.99}{Argth(x)}
  \psaxes{->}(0,0)(-7,-3)(7,3)
\end{pspicture}\\[\baselineskip]
\begin{pspicture}(-7,-3)(7,3)
  \psset{algebraic=true, VarStep, VarStepEpsilon=.001, showpoints=true}
  \psplot[linecolor=red,linewidth=1pt]{-7}{7}{Argsh(x)}
  \psplot[linecolor=blue,linewidth=1pt]{1.001}{7}{Argch(x)}
  \psplot[linecolor=green,linewidth=1pt]{-.99}{.99}{Argth(x)}
  \psaxes{->}(0,0)(-7,-3)(7,3)
\end{pspicture}
\end{lstlisting}



\begin{center}
\bgroup
\begin{pspicture}(-7,-0.5)(7,6)
\psset{algebraic=true}
  \psplot[linecolor=red,linewidth=1pt]{-7}{7}{Derive(1,Argsh(x))}
  \psplot[linecolor=blue,linewidth=1pt]{1.014}{7}{Derive(1,Argch(x))}
  \psplot[linecolor=green,linewidth=1pt]{-.9}{.9}{Derive(1,Argth(x))}
  \psaxes{->}(0,0)(-7,0)(7,6)
\end{pspicture}\\[\baselineskip]
\begin{pspicture}(-7,-0.5)(7,6)
\psset{algebraic=true}
  \psset{algebraic=true, VarStep=true, VarStepEpsilon=.001, showpoints=true}
  \psplot[linecolor=red,linewidth=1pt]{-7}{7}{Derive(1,Argsh(x))}
  \psplot[linecolor=blue,linewidth=1pt]{1.014}{7}{Derive(1,Argch(x))}
  \psplot[linecolor=green,linewidth=1pt]{-.9}{.9}{Derive(1,Argth(x))}
  \psaxes{->}(0,0)(-7,0)(7,6)
\end{pspicture}
\egroup
\end{center}

\begin{lstlisting}
\begin{pspicture}(-7,-0.5)(7,6)
\psset{algebraic=true}
  \psplot[linecolor=red,linewidth=1pt]{-7}{7}{Derive(1,Argsh(x))}
  \psplot[linecolor=blue,linewidth=1pt]{1.014}{7}{Derive(1,Argch(x))}
  \psplot[linecolor=green,linewidth=1pt]{-.9}{.9}{Derive(1,Argth(x))}
  \psaxes{->}(0,0)(-7,0)(7,6)
\end{pspicture}\\[\baselineskip]
\begin{pspicture}(-7,-0.5)(7,6)
\psset{algebraic=true}
  \psset{algebraic=true, VarStep=true, VarStepEpsilon=.001, showpoints=true}
  \psplot[linecolor=red,linewidth=1pt]{-7}{7}{Derive(1,Argsh(x))}
  \psplot[linecolor=blue,linewidth=1pt]{1.014}{7}{Derive(1,Argch(x))}
  \psplot[linecolor=green,linewidth=1pt]{-.9}{.9}{Derive(1,Argth(x))}
  \psaxes{->}(0,0)(-7,0)(7,6)
\end{pspicture}
\end{lstlisting}


\clearpage
%--------------------------------------------------------------------------------------
\section[\nxLcs{psplotDiffEqn} -- solving diffential equations]%
  {\nxLcs{psplotDiffEqn} -- solving diffential equations}
%--------------------------------------------------------------------------------------


 A differential equation of first order is like

\begin{align} y^\prime=f(x,y,y^\prime) \end{align}


where $y$ is a function of $x$. We define some vectors $Y=[y, y',
\cdots , y^{(n-1)}]$ and $Y^\prime=[y^\prime, y^{\prime\prime},
\cdots , y^{n}]$, depending on the order $n$. The syntax of the
macro is

\begin{BDef}
\Lcs{psplotDiffEqn}\OptArgs\Largb{x0}\Largb{x1}\Largb{y0}\Largb{f(x,y,y',...)}
\end{BDef}

\begin{itemize}\setlength\itemsep{0pt}\setlength\parsep{0pt}\setlength\parskip{0pt}
\item \verb+options+: the \verb+\psplotDiffEqn+ specific options and all other of PSTricks, which
make sense;
\item $x_0$: the start value;
\item $x_1$: the end value of the definition interval;
\item $y_0$: the initial values for $y(x_0)\ y'(x_0)\ \ldots$;
\item $f(x,y,y',...)$: the differential equation, depending to the number of initial values, e.g.:
    \verb+{0 1}+ for $y_0$ are two initial values, so that we have a differential equation of
    second order $f(x,y,y')$ and the macro leaves $y\ y'$ on the stack.
\end{itemize}

The new options are:


\begin{itemize}\setlength\itemsep{0pt}\setlength\parsep{0pt}\setlength\parskip{0pt}
\item \Lkeyword{method}: integration method (\verb+euler+ for order 1 euler method, \verb+rk4+ for
  4\textsuperscript{th} order Runge-Kutta method);
\item \Lkeyword{whichabs}: select the abscissa for plotting the graph, by default it is
  $x$, but you can specify a number which represent a position in the vector $y$;
\item \Lkeyword{whichord}: same as precedent for the ordinate, by default $y(0)$;
\item \Lkeyword{plotfuncx}: describe a ps function for the abscissa, parameter
  \Lkeyword{whichabs} becomes useless;
\item \Lkeyword{plotfuncy}: idem for the ordinate;
\item \Lkeyword{buildvector}: boolean parameter for specifying the input-output of the
  $f$ description:
  \begin{description}
  \item[\texttt{true}] (default): $y$ is put on the stack element by element, $y'$
    must be given in the same way;
  \item[\texttt{false}]: $y$ is put on the stack as a vector, $y'$ must be returned
  in the same way;
  \end{description}

\item \Lkeyword{algebraic=true}: algebraic=true description for $f$, \Lkeyword{buildvector}
  parameter is useless when activating this option.
\end{itemize}



\clearpage
\subsection{Variable step for differential equations}

A new algorithm has been added for adjusting the step according to the variations of
the curve. The parameter \Lkeyword{method} has a new possible value : \Lkeyword{varrkiv} to
activate the \Index{Runge-Kutta} method with variable step, then the parameter
\Lkeyword{varsteptol} (real value; \verb+.01+ by default) can control the tolerance of
the algortihm.

\begin{center}
\bgroup
\def\Funct{neg}\def\FunctAlg{-y[0]}
\psset{xunit=1.5, yunit=8, showpoints=true}
\begin{pspicture}[showgrid=true](0,0)(10,1.2)
  \psplot[linewidth=6\pslinewidth, linecolor=green, showpoints=false]{0}{10}{Euler x neg exp}
  \psplotDiffEqn[linecolor=magenta, method=varrkiv, varsteptol=.1, plotpoints=2]{0}{10}{1}{\Funct}
  \rput(0,.0){\psplotDiffEqn[linecolor=blue, method=varrkiv, varsteptol=.01, plotpoints=2]{0}{10}{1}{\Funct}}
  \rput(0,.1){\psplotDiffEqn[linecolor=Orange, method=varrkiv, varsteptol=.001, plotpoints=2]{0}{10}{1}{\Funct}}
  \rput(0,.2){\psplotDiffEqn[linecolor=red, method=varrkiv, varsteptol=.0001, plotpoints=2]{0}{10}{1}{\Funct}}
  \psset{linewidth=4\pslinewidth,showpoints=false}
  \rput*(3.3,.9){\psline[linecolor=magenta](-.75cm,0)}
  \rput*[l](3.3,.9){\small RK ordre 4 : $\varepsilon<10^{-1}$}
  \rput*(3.3,.8){\psline[linecolor=blue](-.75cm,0)}
  \rput*[l](3.3,.8){\small RK ordre 4 : $\varepsilon<10^{-2}$}
  \rput*(3.3,.7){\psline[linecolor=Orange](-.75cm,0)}
  \rput*[l](3.3,.7){\small RK ordre 4 : $\varepsilon<10^{-3}$}
  \rput*(3.3,.6){\psline[linecolor=red](-.75cm,0)}
  \rput*[l](3.3,.6){\small RK ordre 4 : $\varepsilon<10^{-4}$}
  \rput*(3.3,.5){\psline[linecolor=green](-.75cm,0)}
  \rput*[l](3.3,.5){\small solution exacte}
\end{pspicture}
{\captionof{figure}{Equation $y'=-y$ with $y_0=1$.}\label{fig:minusexpvarstep}}
\egroup
\end{center}


\begin{lstlisting}[wide=true]
\def\Funct{neg}\def\FunctAlg{-y[0]}
\psset{xunit=1.5, yunit=8, showpoints=true}
\begin{pspicture}[showgrid=true](0,0)(10,1.2)
  \psplot[linewidth=6\pslinewidth, linecolor=green, showpoints=false]{0}{10}{Euler x neg exp}
  \psplotDiffEqn[linecolor=magenta, method=varrkiv, varsteptol=.1, plotpoints=2]{0}{10}{1}{\Funct}
  \rput(0,.0){\psplotDiffEqn[linecolor=blue, method=varrkiv, varsteptol=.01, plotpoints=2]{0}{10}{1}{\Funct}}
  \rput(0,.1){\psplotDiffEqn[linecolor=Orange, method=varrkiv, varsteptol=.001, plotpoints=2]{0}{10}{1}{\Funct}}
  \rput(0,.2){\psplotDiffEqn[linecolor=red, method=varrkiv, varsteptol=.0001, plotpoints=2]{0}{10}{1}{\Funct}}
  \psset{linewidth=4\pslinewidth,showpoints=false}
  \rput*(3.3,.9){\psline[linecolor=magenta](-.75cm,0)}
  \rput*[l](3.3,.9){\small RK ordre 4 : $\varepsilon<10^{-1}$}
  \rput*(3.3,.8){\psline[linecolor=blue](-.75cm,0)}
  \rput*[l](3.3,.8){\small RK ordre 4 : $\varepsilon<10^{-2}$}
  \rput*(3.3,.7){\psline[linecolor=Orange](-.75cm,0)}
  \rput*[l](3.3,.7){\small RK ordre 4 : $\varepsilon<10^{-3}$}
  \rput*(3.3,.6){\psline[linecolor=red](-.75cm,0)}
  \rput*[l](3.3,.6){\small RK ordre 4 : $\varepsilon<10^{-4}$}
  \rput*(3.3,.5){\psline[linecolor=green](-.75cm,0)}
  \rput*[l](3.3,.5){\small solution exacte}
\end{pspicture}
\end{lstlisting}



\begin{center}
\bgroup
\def\Funct{exch neg}
\psset{xunit=1.5, yunit=5, method=varrkiv, showpoints=true}%%
\def\quatrepi{12.5663706144}
\begin{pspicture}(0,-1)(10,1.3)
  \psaxes{->}(0,0)(0,-1)(10,1.3)
  \psplot[linewidth=4\pslinewidth, linecolor=green, algebraic=true]{0}{10}{cos(x)}
  \rput(0,.0){\psplotDiffEqn[linecolor=magenta, plotpoints=7, varsteptol=.1]{0}{10}{1 0}{\Funct}}
  \rput(0,.0){\psplotDiffEqn[linecolor=blue, plotpoints=201, varsteptol=.01]{0}{10}{1 0}{\Funct}}
  \rput(0,.1){\psplotDiffEqn[linewidth=2\pslinewidth, linecolor=red, varsteptol=.001]{0}{10}{1 0}{\Funct}}
  \rput(0,.2){\psplotDiffEqn[linecolor=black, varsteptol=.0001]{0}{10}{1 0}{\Funct}}
  \rput(0,.3){\psplotDiffEqn[linecolor=Orange, varsteptol=.00001]{0}{10}{1 0}{\Funct}}
  \psset{linewidth=4\pslinewidth,showpoints=false}
  \rput*(2.3,.9){\psline[linecolor=magenta](-.75cm,0)}
  \rput*[l](2.3,.9){\small $\varepsilon<10^{-1}$}
  \rput*(2.3,.8){\psline[linecolor=blue](-.75cm,0)}
  \rput*[l](2.3,.8){\small $\varepsilon<10^{-2}$}
  \rput*(2.3,.7){\psline[linecolor=red](-.75cm,0)}
  \rput*[l](2.3,.7){\small $\varepsilon<10^{-3}$}
  \rput*(2.3,.6){\psline[linecolor=black](-.75cm,0)}
  \rput*[l](2.3,.6){\small $\varepsilon<10^{-4}$}
  \rput*(2.3,.5){\psline[linecolor=Orange](-.75cm,0)}
  \rput*[l](2.3,.5){\small $\varepsilon<10^{-5}$}
  \rput*(2.3,.4){\psline[linecolor=green](-.75cm,0)}
  \rput*[l](2.3,.4){\small solution exacte}
\end{pspicture}
{\captionof{figure}{Equation $y''=-y$}\label{fig:trigfunc}}
\egroup
\end{center}

\begin{lstlisting}[wide=true]
\def\Funct{exch neg}
\psset{xunit=1.5, yunit=5, method=varrkiv, showpoints=true}%%
\def\quatrepi{12.5663706144}
\begin{pspicture}(0,-1)(10,1.3)
  \psaxes{->}(0,0)(0,-1)(10,1.3)
  \psplot[linewidth=4\pslinewidth, linecolor=green, algebraic=true]{0}{10}{cos(x)}
  \rput(0,.0){\psplotDiffEqn[linecolor=magenta, plotpoints=7, varsteptol=.1]{0}{10}{1 0}{\Funct}}
  \rput(0,.0){\psplotDiffEqn[linecolor=blue, plotpoints=201, varsteptol=.01]{0}{10}{1 0}{\Funct}}
  \rput(0,.1){\psplotDiffEqn[linewidth=2\pslinewidth, linecolor=red, varsteptol=.001]{0}{10}{1 0}{\Funct}}
  \rput(0,.2){\psplotDiffEqn[linecolor=black, varsteptol=.0001]{0}{10}{1 0}{\Funct}}
  \rput(0,.3){\psplotDiffEqn[linecolor=Orange, varsteptol=.00001]{0}{10}{1 0}{\Funct}}
  \psset{linewidth=4\pslinewidth,showpoints=false}
  \rput*(2.3,.9){\psline[linecolor=magenta](-.75cm,0)}
  \rput*[l](2.3,.9){\small $\varepsilon<10^{-1}$}
  \rput*(2.3,.8){\psline[linecolor=blue](-.75cm,0)}
  \rput*[l](2.3,.8){\small $\varepsilon<10^{-2}$}
  \rput*(2.3,.7){\psline[linecolor=red](-.75cm,0)}
  \rput*[l](2.3,.7){\small $\varepsilon<10^{-3}$}
  \rput*(2.3,.6){\psline[linecolor=black](-.75cm,0)}
  \rput*[l](2.3,.6){\small $\varepsilon<10^{-4}$}
  \rput*(2.3,.5){\psline[linecolor=Orange](-.75cm,0)}
  \rput*[l](2.3,.5){\small $\varepsilon<10^{-5}$}
  \rput*(2.3,.4){\psline[linecolor=green](-.75cm,0)}
  \rput*[l](2.3,.4){\small solution exacte}
\end{pspicture}
\end{lstlisting}




\begin{center}
\bgroup
\def\Funct{exch}
\psset{xunit=4, yunit=1, method=varrkiv, showpoints=true}%%
\def\quatrepi{12.5663706144}
\begin{pspicture}(0,-0.5)(3,11)
  \psaxes{->}(0,0)(3,11)
  \psplot[linewidth=4\pslinewidth, linecolor=green, algebraic=true]{0}{3}{ch(x)}
  \rput(0,.0){\psplotDiffEqn[linecolor=magenta, varsteptol=.1]{0}{3}{1 0}{\Funct}}
  \rput(0,.3){\psplotDiffEqn[linecolor=blue, varsteptol=.01]{0}{3}{1 0}{\Funct}}
  \rput(0,.6){\psplotDiffEqn[linecolor=red, varsteptol=.001]{0}{3}{1 0}{\Funct}}
  \rput(0,.9){\psplotDiffEqn[linecolor=black, varsteptol=.0001]{0}{3}{1 0}{\Funct}}
  \rput(0,1.2){\psplotDiffEqn[linecolor=Orange, varsteptol=.00001]{0}{3}{1 0}{\Funct}}
  \psset{linewidth=4\pslinewidth,showpoints=false}
  \rput*(2.3,.9){\psline[linecolor=magenta](-.75cm,0)}
  \rput*[l](2.3,.9){\small $\varepsilon<10^{-1}$}
  \rput*(2.3,.8){\psline[linecolor=blue](-.75cm,0)}
  \rput*[l](2.3,.8){\small $\varepsilon<10^{-2}$}
  \rput*(2.3,.7){\psline[linecolor=red](-.75cm,0)}
  \rput*[l](2.3,.7){\small $\varepsilon<10^{-3}$}
  \rput*(2.3,.6){\psline[linecolor=black](-.75cm,0)}
  \rput*[l](2.3,.6){\small $\varepsilon<10^{-4}$}
  \rput*(2.3,.5){\psline[linecolor=Orange](-.75cm,0)}
  \rput*[l](2.3,.5){\small $\varepsilon<10^{-5}$}
  \rput*(2.3,.4){\psline[linecolor=green](-.75cm,0)}
  \rput*[l](2.3,.4){\small solution exacte}
\end{pspicture}
\captionof{figure}{Equation $y''=y$}
\egroup
\end{center}

\begin{lstlisting}[wide=true]
\def\Funct{exch}
\psset{xunit=4, yunit=1, method=varrkiv, showpoints=true}%%
\def\quatrepi{12.5663706144}
\begin{pspicture}(0,-0.5)(3,11)
  \psaxes{->}(0,0)(3,11)
  \psplot[linewidth=4\pslinewidth, linecolor=green, algebraic=true]{0}{3}{ch(x)}
  \rput(0,.0){\psplotDiffEqn[linecolor=magenta, varsteptol=.1]{0}{3}{1 0}{\Funct}}
  \rput(0,.3){\psplotDiffEqn[linecolor=blue, varsteptol=.01]{0}{3}{1 0}{\Funct}}
  \rput(0,.6){\psplotDiffEqn[linecolor=red, varsteptol=.001]{0}{3}{1 0}{\Funct}}
  \rput(0,.9){\psplotDiffEqn[linecolor=black, varsteptol=.0001]{0}{3}{1 0}{\Funct}}
  \rput(0,1.2){\psplotDiffEqn[linecolor=Orange, varsteptol=.00001]{0}{3}{1 0}{\Funct}}
  \psset{linewidth=4\pslinewidth,showpoints=false}
  \rput*(2.3,.9){\psline[linecolor=magenta](-.75cm,0)}
  \rput*[l](2.3,.9){\small $\varepsilon<10^{-1}$}
  \rput*(2.3,.8){\psline[linecolor=blue](-.75cm,0)}
  \rput*[l](2.3,.8){\small $\varepsilon<10^{-2}$}
  \rput*(2.3,.7){\psline[linecolor=red](-.75cm,0)}
  \rput*[l](2.3,.7){\small $\varepsilon<10^{-3}$}
  \rput*(2.3,.6){\psline[linecolor=black](-.75cm,0)}
  \rput*[l](2.3,.6){\small $\varepsilon<10^{-4}$}
  \rput*(2.3,.5){\psline[linecolor=Orange](-.75cm,0)}
  \rput*[l](2.3,.5){\small $\varepsilon<10^{-5}$}
  \rput*(2.3,.4){\psline[linecolor=green](-.75cm,0)}
  \rput*[l](2.3,.4){\small solution exacte}
\end{pspicture}
\end{lstlisting}




\clearpage
\subsection{Equation of second order}

Here is the traditional simulation of two stars attracting each
other according to the classical gravitation law in
$\displaystyle\frac{1}{r^2}$. In 2-Dimensions, the system to be
solved is composed of four second order differential equations. In
order to be described, each of them gives two first order
equations, then we obtain a 8 sized vectorial equation. In the
following example the masses of the stars are 1 and 20.

\[
\left\{
\begin{array}[m]{l}
  x''_1=\displaystyle\frac{M_2}{r^2}\cos(\theta)\\
  y''_1=\displaystyle\frac{M_2}{r^2}\sin(\theta)\\
  x''_2=\displaystyle\frac{M_1}{r^2}\cos(\theta)\\
  y''_2=\displaystyle\frac{M_1}{r^2}\sin(\theta)\\
\end{array}
\right.
\mbox{ avec }
\left\{
\begin{array}[m]{l}
  r^2=(x_1-x_2)^2+(y_1-y_2)^2\\
  \cos(\theta)=\displaystyle\frac{(x_1-x_2)}{r}\\
  \sin(\theta)=\displaystyle\frac{(y_1-y_2)}{r}\\
\end{array}
\right.
\mbox{%
\begin{pspicture}[shift=-2](5,4)\psset{arrowscale=2}
  \psframe[linewidth=.75\pslinewidth](5,4)
  \pstGeonode[PosAngle={-90,90}](1,1){M_1}(4,3){M_2}
  \pstHomO[HomCoef=.33, PointSymbol=none]{M_1}{M_2}[F_1]
  \psline[arrows=->](M_1)(F_1)
  \pstHomO[HomCoef=.33, PointSymbol=none]{M_2}{M_1}[F_2]
  \psline[arrows=->, arrowscale=2](M_2)(F_2)
  \pstGeonode[PointSymbol=none, PointName=none](M_2|M_1){A}
  \psline[linewidth=.5\pslinewidth](M_1)(A)
  \pstMarkAngle{A}{M_1}{M_2}{$\theta$}
  \ncline[linewidth=.5\pslinewidth, offset=.5, arrows=<->]{M_1}{M_2}
  \ncput*{$r$}
\end{pspicture}}
\]

\begin{table}[!htbp]
  \centering\small
    \begin{tabular}{|l@{}>{\ttfamily}l@{}>{ \ttfamily \%\% }l|}
      \hline
      && x1 y1 x'1 y'1 x2 y2 x'2 y'2\\
      &/yp2 exch def /xp2 exch def /ay2 exch def /ax2 exch def&mise en variables\\
      &/yp1 exch def /xp1 exch def /ay1 exch def /ax1 exch def&mise en variables\\
      &/ro2 ax2 ax1 sub dup mul ay2 ay1 sub dup mul add def&calcul de r*r\\
      &xp1 yp1&\\
      &ax2 ax1 sub ro2 sqrt div ro2 div&calcul de x''1\\
      &ay2 ay1 sub ro2 sqrt div ro2 div&calcul de y''1\\
      &xp2 yp2&\\
      &3 index -20 mul&calcul de x''2=-20x''1\\
      &3 index -20 mul&calcul de y''2=-20y''1\\
      \hline
    \end{tabular}
    \caption{\PS source code for the gravitational interaction}\label{intgravcode}
\end{table}

\begin{table}[!htbp]
  \centering
    \small\newcommand{\POW}{\symbol{'136}}
    \begin{tabular}{|l@{}>{\ttfamily}l@{}>{ \ttfamily \%\% }l|}
      \hline
      &y[2]|&y'[0]\\
      &y[3]|&y'[1]\\
      &(y[4]-y[0])/((y[4]-y[0])\POW 2+(y[5]-y[1])\POW 2)\POW 1.5|&y'[2]=y''[0]\\
      &(y[5]-y[1])/((y[4]-y[0])\POW 2+(y[5]-y[1])\POW 2)\POW 1.5|&y'[3]=y''[1]\\
      &y[6]|&y'[4]\\
      &y[7]|&y'[5]\\
      &20*(y[0]-y[4])/((y[4]-y[0])\POW 2+(y[5]-y[1])\POW 2)\POW 1.5|&y'[6]=y''[4]\\
      &20*(y[1]-y[5])/((y[4]-y[0])\POW 2+(y[5]-y[1])\POW 2)\POW 1.5&y'[7]=y''[5]\\
      \hline
    \end{tabular}
    \caption{Algebraic description for the gravitational interaction}\label{intgravalgcode}
\end{table}

\newcommand\Grav{%
  /yp2 exch def /xp2 exch def /ay2 exch def /ax2 exch def
  /yp1 exch def /xp1 exch def /ay1 exch def /ax1 exch def
  /ro2 ax2 ax1 sub dup mul ay2 ay1 sub dup mul add def
  xp1 yp1
  ax2 ax1 sub ro2 sqrt div ro2 div
  ay2 ay1 sub ro2 sqrt div ro2 div
  xp2 yp2
  3 index -20 mul
  3 index -20 mul}
\newcommand\GravAlg{%
  y[2]|y[3]|%
  (y[4]-y[0])/((y[4]-y[0])^2+(y[5]-y[1])^2)^1.5|%
  (y[5]-y[1])/((y[4]-y[0])^2+(y[5]-y[1])^2)^1.5|%
  y[6]|y[7]|%
  20*(y[0]-y[4])/((y[4]-y[0])^2+(y[5]-y[1])^2)^1.5|%
  20*(y[1]-y[5])/((y[4]-y[0])^2+(y[5]-y[1])^2)^1.5}
%%  0  1   2   3  4  5   6   7
%% x1 y1 x'1 y'1 x2 y2 x'2 y'2


\begin{LTXexample}[width=5cm,wide]
\def\InitCond{ 1  1  .1  0 -1 -1  -2   0}
\begin{pspicture}[shift=-2,showgrid=true](-3,-1.75)(2,1.5)
  \psplotDiffEqn[whichabs=0, whichord=1, linecolor=blue, method=rk4, plotpoints=100]{0}{3.95}{\InitCond}{\Grav}
  \psset{showpoints=true,whichabs=4, whichord=5}
  \psplotDiffEqn[linecolor=black, method=varrkiv, varsteptol=.0001, plotpoints=200]{0}{3.9}{\InitCond}{\Grav}
\end{pspicture}
\end{LTXexample}
\vspace{-2ex}
{\captionof{figure}{Gravitational interaction: fixed landmark, trajectory of the stars}\label{fig:InterGravRepFix}}



\bigskip
\begin{LTXexample}[width=5cm,wide]
\def\InitCond{ 1  1  .1  0 -1 -1  -2   0}
\begin{pspicture}[shift=-1.5,showgrid=true](-4,-1.75)(1,1)
  \psplotDiffEqn[linecolor=red, plotpoints=200,method=varrkiv, varsteptol=.0001, showpoints=true,
      plotfuncx=y dup 4 get exch 0 get sub,
      plotfuncy=dup 5 get exch 1 get sub ]{0}{3.9}{\InitCond}{\Grav}
\end{pspicture}
\end{LTXexample}
\vspace{-2ex}
{\captionof{figure}{Gravitational interaction : landmark defined by one star}\label{fig:IGnewrep}}


\begin{center}
\bgroup
\def\InitCond{ 1  1  .1   0 -1 -1  -2   0}
\psset{xunit=2}
\begin{pspicture}[showgrid=true](0,0)(8,9)
  \psset{showpoints=true}
  \psplotDiffEqn[linecolor=red, method=varrkiv, plotpoints=2, varsteptol=.0001,
      plotfuncy=dup 6 get dup mul exch 7 get dup mul add sqrt]{0}{8}{\InitCond}{\Grav}
  \psplotDiffEqn[linecolor=blue, method=varrkiv, plotpoints=2, varsteptol=.0001,
      plotfuncy=dup 2 get dup mul exch 3 get dup mul add sqrt]{0}{8}{\InitCond}{\Grav}
\end{pspicture}
\captionof{figure}{Gravitational interaction : speeds of the
stars} \egroup
\end{center}

\begin{lstlisting}
\psset{xunit=2}
\begin{pspicture}[showgrid=true](0,0)(8,9)
  \psset{showpoints=true}
  \psplotDiffEqn[linecolor=red, method=varrkiv, plotpoints=2, varsteptol=.0001,
      plotfuncy=dup 6 get dup mul exch 7 get dup mul add sqrt]{0}{8}{\InitCond}{\Grav}
  \psplotDiffEqn[linecolor=blue, method=varrkiv, plotpoints=2, varsteptol=.0001,
      plotfuncy=dup 2 get dup mul exch 3 get dup mul add sqrt]{0}{8}{\InitCond}{\Grav}
\end{pspicture}
\end{lstlisting}

%--------------------------------------------------------------------------------------
\clearpage
\subsubsection{Simple equation of first order $y'=y$}
%--------------------------------------------------------------------------------------

For the initial value $y(0)=1$ we have the solution $y(x)=e^x$. $y$ is always
on the stack, so we have to do nothing. Using the \Lkeyword{algebraic=true} option, we write it
as \verb$y[0]$. The following example shows different solutions depending to the number of plotpoints
with $y_0=1$:


\begin{center}
\bgroup
\psset{xunit=4, yunit=.4}
\begin{pspicture}(3,19)\psgrid[subgriddiv=1]
  \psplot[linewidth=6\pslinewidth, linecolor=green]{0}{3}{Euler x exp}
  \psplotDiffEqn[linecolor=magenta,plotpoints=16,algebraic=true]{0}{3}{1}{y[0]}
  \psplotDiffEqn[linecolor=blue,plotpoints=151]{0}{3}{1}{}
  \psplotDiffEqn[linecolor=red,method=rk4,plotpoints=15]{0}{3}{1}{}
  \psplotDiffEqn[linecolor=Orange,method=rk4,plotpoints=4]{0}{3}{1}{}
  \psset{linewidth=4\pslinewidth}
  \rput*(0.35,19){\psline[linecolor=magenta](-.75cm,0)}
  \rput*[l](0.35,19){\small Euler order 1 $h=0{,}2$}
  \rput*(0.35,17){\psline[linecolor=blue](-.75cm,0)}
  \rput*[l](0.35,17){\small Euler order 1 $h=0{,}02$}
  \rput*(0.35,15){\psline[linecolor=Orange](-.75cm,0)}
  \rput*[l](0.35,15){\small RK ordre 4 $h=1$}
  \rput*(0.35,13){\psline[linecolor=red](-.75cm,0)}
  \rput*[l](0.35,13){\small RK ordre 4 $h=0{,}2$}
  \rput*(0.35,11){\psline[linecolor=green](-.75cm,0)}
  \rput*[l](0.35,11){\small solution exacte}
\end{pspicture}
\egroup
\end{center}

\begin{lstlisting}
\psset{xunit=4, yunit=.4}
\begin{pspicture}(3,19)\psgrid[subgriddiv=1]
  \psplot[linewidth=6\pslinewidth, linecolor=green]{0}{3}{Euler x exp}
  \psplotDiffEqn[linecolor=magenta,plotpoints=16,algebraic=true]{0}{3}{1}{y[0]}
  \psplotDiffEqn[linecolor=blue,plotpoints=151]{0}{3}{1}{}
  \psplotDiffEqn[linecolor=red,method=rk4,plotpoints=15]{0}{3}{1}{}
  \psplotDiffEqn[linecolor=Orange,method=rk4,plotpoints=4]{0}{3}{1}{}
  \psset{linewidth=4\pslinewidth}
  \rput*(0.35,19){\psline[linecolor=magenta](-.75cm,0)}
  \rput*[l](0.35,19){\small Euler order 1 $h=0{,}2$}
  \rput*(0.35,17){\psline[linecolor=blue](-.75cm,0)}
  \rput*[l](0.35,17){\small Euler order 1 $h=0{,}02$}
  \rput*(0.35,15){\psline[linecolor=Orange](-.75cm,0)}
  \rput*[l](0.35,15){\small RK ordre 4 $h=1$}
  \rput*(0.35,13){\psline[linecolor=red](-.75cm,0)}
  \rput*[l](0.35,13){\small RK ordre 4 $h=0{,}2$}
  \rput*(0.35,11){\psline[linecolor=green](-.75cm,0)}
  \rput*[l](0.35,11){\small solution exacte}
\end{pspicture}
\end{lstlisting}

%--------------------------------------------------------------------------------------
\clearpage
\subsubsection{$y'=\displaystyle\frac{2-ty}{4-t^2}$}% $
%--------------------------------------------------------------------------------------

For the initial value $y(0)=1$ the exact solution is
$y(x)=\displaystyle\frac{t+\sqrt{4-t^2}}{2}$. The function $f$
described in PostScript code is like (y is still on the stack):
\begin{lstlisting}[style=syntax]
x              %% y x
mul            %% x*y
2 exch sub     %% 2-x*y
4 x dup mul    %% 2-x*y 4 x^2
sub            %% 2-x*y 4-x^2
div            %% (2-x*y)/(4-x^2)
\end{lstlisting}
\noindent
The following example uses $y_0=1$.

\begin{lstlisting}[style=syntax]
\newcommand{\InitCond}{1}
\newcommand{\Func}{x mul 2 exch sub 4 x dup mul sub div}
\newcommand{\FuncAlg}{(2-x*y[0])/(4-x^2)}
\end{lstlisting}

\begin{center}
\bgroup
\psset{xunit=6.4, yunit=9.6, showpoints=false}
\begin{pspicture}(0,1)(2,1.5)  \psgrid[griddots=10](0,1)(2,1.5)
  { \psset{linewidth=4\pslinewidth,linecolor=lightgray}
  \psplot{0}{1.8}{x dup dup mul 4 exch sub sqrt add 2 div}
  \psplot{1.8}{2}{x dup dup mul 4 exch sub sqrt add 2 div} }
  \def\InitCond{1}
  \def\Func{x mul 2 exch sub 4 x dup mul sub div}
  \psplotDiffEqn[linecolor=magenta, plotpoints=20]{0}{1.9}{\InitCond}{\Func}
  \psplotDiffEqn[linecolor=blue, plotpoints=191]{0}{1.9}{\InitCond}{\Func}
  \psplotDiffEqn[linecolor=red, method=rk4, plotpoints=11,%
     algebraic=true]{0}{1.9}{\InitCond}{(2-x*y[0])/(4-x^2)}
  \psplotDiffEqn[linecolor=Orange, method=rk4, plotpoints=21,%
     algebraic=true]{0}{1.9}{\InitCond}{(2-x*y[0])/(4-x^2)}
  \psset{linewidth=4\pslinewidth}\small
  \rput*(0,1.4){\psline[linecolor=magenta](-.75cm,0)}\rput*[l](0,1.4){Euler order 1 $h=0{,}1$}
  \rput*(0,1.35){\psline[linecolor=blue](-.75cm,0)}\rput*[l](0,1.35){Euler order 1 $h=0{,}01$}
  \rput*(0,1.3){\psline[linecolor=Orange](-.75cm,0)}\rput*[l](0,1.3){RK order 4 $h=0{,}19$}
  \rput*(0,1.25){\psline[linecolor=red](-.75cm,0)}\rput*[l](0,1.25){RK order 4 $h=0{,}095$}
  \rput*(0,1.2){\psline[linecolor=lightgray](-.75cm,0)}\rput*[l](0,1.2){exactly}
\end{pspicture}
\egroup
\end{center}

\begin{lstlisting}[xrightmargin=-1cm,xleftmargin=-1cm]
\psset{xunit=6.4, yunit=9.6, showpoints=false}
\begin{pspicture}(0,1)(2,1.7)  \psgrid[subgriddiv=5]
  { \psset{linewidth=4\pslinewidth,linecolor=lightgray}
  \psplot{0}{1.8}{x dup dup mul 4 exch sub sqrt add 2 div}
  \psplot{1.8}{2}{x dup dup mul 4 exch sub sqrt add 2 div} }
  \def\InitCond{1}
  \def\Func{x mul 2 exch sub 4 x dup mul sub div}
  \psplotDiffEqn[linecolor=magenta, plotpoints=20]{0}{1.9}{\InitCond}{\Func}
  \psplotDiffEqn[linecolor=blue, plotpoints=191]{0}{1.9}{\InitCond}{\Func}
  \psplotDiffEqn[linecolor=red, method=rk4, plotpoints=11,%
     algebraic=true]{0}{1.9}{\InitCond}{(2-x*y[0])/(4-x^2)}
  \psplotDiffEqn[linecolor=Orange, method=rk4, plotpoints=21,%
     algebraic=true]{0}{1.9}{\InitCond}{(2-x*y[0])/(4-x^2)}
  \psset{linewidth=4\pslinewidth}
  \rput*(0.3,1.6){\psline[linecolor=magenta](-.75cm,0)}\rput*[l](0.3,1.6){\small Euler order 1 $h=0{,}1$}
  \rput*(0.3,1.55){\psline[linecolor=blue](-.75cm,0)}\rput*[l](0.3,1.55){\small Euler order 1 $h=0{,}01$}
  \rput*(0.3,1.5){\psline[linecolor=Orange](-.75cm,0)}\rput*[l](0.3,1.5){\small RK order 4 $h=0{,}19$}
  \rput*(0.3,1.45){\psline[linecolor=red](-.75cm,0)}\rput*[l](0.3,1.45){\small RK order 4 $h=0{,}095$}
  \rput*(0.3,1.4){\psline[linecolor=lightgray](-.75cm,0)}\rput*[l](0.3,1.4){\small exactly}
\end{pspicture}
\end{lstlisting}


%--------------------------------------------------------------------------------------
\clearpage
\subsubsection{$y'=-2xy$}
%--------------------------------------------------------------------------------------

For $y(-1)=\frac{1}{e}$ we get $y(x)=e^{-x^2}$.

\begin{center}
\bgroup
\psset{unit=4}
\begin{pspicture}(-1,0)(3,1.1)\psgrid
  \psplot[linewidth=4\pslinewidth,linecolor=gray]{-1}{3}{Euler x dup mul neg exp}
  \psset{plotpoints=9}
  \psplotDiffEqn[linecolor=cyan]{-1}{3}{1 Euler div}{x -2 mul mul}
  \psplotDiffEqn[linecolor=yellow, method=rk4]{-1}{3}{1 Euler div}{x -2 mul mul}
  \psset{plotpoints=21}
  \psplotDiffEqn[linecolor=blue]{-1}{3}{1 Euler div}{x -2 mul mul}
  \psplotDiffEqn[linecolor=Orange, method=rk4]{-1}{3}{1 Euler div}{x -2 mul mul}
  \psset{linewidth=2\pslinewidth}
  \rput*(2,1){\psline[linecolor=Orange](-0.25,0)}
  \rput*[l](2,1){RK}
  \rput*(2,.9){\psline[linecolor=blue](-0.25,0)}
  \rput*[l](2,.9){\textsc{Euler}-1}
  \rput*(2,.8){\psline[linecolor=gray](-0.25,0)}
  \rput*[l](2,.8){solution}
\end{pspicture}
\egroup
\end{center}


\begin{lstlisting}
\psset{unit=4}
\begin{pspicture}(-1,0)(3,1.1)\psgrid
  \psplot[linewidth=4\pslinewidth,linecolor=gray]{-1}{3}{Euler x dup mul neg exp}
  \psset{plotpoints=9}
  \psplotDiffEqn[linecolor=cyan]{-1}{3}{1 Euler div}{x -2 mul mul}
  \psplotDiffEqn[linecolor=yellow, method=rk4]{-1}{3}{1 Euler div}{x -2 mul mul}
  \psset{plotpoints=21}
  \psplotDiffEqn[linecolor=blue]{-1}{3}{1 Euler div}{x -2 mul mul}
  \psplotDiffEqn[linecolor=Orange, method=rk4]{-1}{3}{1 Euler div}{x -2 mul mul}
  \psset{linewidth=2\pslinewidth}
  \rput*(2,1){\psline[linecolor=Orange](-0.25,0)}
  \rput*[l](2,1){RK}
  \rput*(2,.9){\psline[linecolor=blue](-0.25,0)}
  \rput*[l](2,.9){\textsc{Euler}-1}
  \rput*(2,.8){\psline[linecolor=gray](-0.25,0)}
  \rput*[l](2,.8){solution}
\end{pspicture}
\end{lstlisting}


%--------------------------------------------------------------------------------------
\clearpage
\subsubsection{Spiral of Cornu}
%--------------------------------------------------------------------------------------

The integrals of \Index{Fresnel}:
\begin{align} x & =\int^t_0\cos\frac{\pi t^2}{2}\mathrm{d}t \\
 y & =\int^t_0\sin\frac{\pi t^2}{2}\mathrm{d}t \\
\intertext{with}
 \dot{x} &= \cos\frac{\pi t^2}{2} \\
 \dot{y} & =\sin\frac{\pi t^2}{2}
 \end{align}

\begin{lstlisting}
\psset{unit=8}
\begin{pspicture}(1,1)\psgrid[subgriddiv=5]
  \psplotDiffEqn[whichabs=0,whichord=1,linecolor=red,method=rk4,algebraic=true,%
     plotpoints=500,showpoints=true]{0}{10}{0 0}{cos(Pi*x^2/2)|sin(Pi*x^2/2)}
\end{pspicture}
\end{lstlisting}


\begin{center}
\bgroup
\psset{unit=8}
\begin{pspicture}(1,1)\psgrid[subgriddiv=5]
  \psplotDiffEqn[whichabs=0,whichord=1,linecolor=red,method=rk4,algebraic=true,%
     plotpoints=500,showpoints=true]{0}{10}{0 0}{cos(Pi*x^2/2)|sin(Pi*x^2/2)}
\end{pspicture}
\egroup
\end{center}



%--------------------------------------------------------------------------------------
\clearpage
\subsubsection{Lotka-Volterra}
%--------------------------------------------------------------------------------------

The Lotka-Volterra model describes interactions between two species in an ecosystem, a 
predator and a prey. This represents our first multi-species model. Since we are considering 
two species, the model will involve two equations, one which describes how the prey 
population changes and the second which describes how the predator population changes.

For concreteness let us assume that the prey in our model are rabbits, and that the 
predators are foxes. If we let $R(t)$ and $F(t)$ represent the number of rabbits and 
foxes, respectively, that are alive at time t, then the Lotka-Volterra model is:
%
\begin{align}
\dot R &= a\cdot R - b\cdot R\cdot F\\
\dot F &= e\cdot b\cdot R\cdot F - c\cdot F
\end{align}
%
where the parameters are defined by:
\begin{description}
\item[a] is the natural growth rate of rabbits in the absence of predation,
\item[c] is the natural death rate of foxes in the absence of food (rabbits),
\item[b] is the death rate per encounter of rabbits due to predation,
\item[e] is the efficiency of turning predated rabbits into foxes.
\end{description}

The Stella model representing the \Index{Lotka-Volterra} model will be slightly more complex than the 
single species models we've dealt with before. The main difference is that our model will have 
two stocks (reservoirs), one for each species. Each species will have its own birth and death 
rates. In addition, the Lotka-Volterra model involves four parameters rather than two. All told, 
the Stella representation of the Lotka-Volterra model will use two stocks, four flows, four 
converters and many connectors.

\bgroup
\begin{center}
\def\InitCond{ 0 10 10}%% xa ya xl
\def\Faiglelapin{\Vaigle*(y[2]-y[0])/sqrt(y[1]^2+(y[2]-y[0])^2)|%
                 -\Vaigle*y[1]/sqrt(y[1]^2+(y[2]-y[0])^2)|%
                 -\Vlapin}
\def\Vlapin{1}  \def\Vaigle{1.6}
\psset{unit=.7,subgriddiv=0,gridcolor=lightgray,method=adams,algebraic=true,%
   plotpoints=20,showpoints=true}
\begin{pspicture}[showgrid=true](-3,-3)(10,10)
 \psplotDiffEqn[plotfuncy=pop 0,whichabs=2,linecolor=red]{0}{10}{\InitCond}{\Faiglelapin}
 \psplotDiffEqn[whichabs=0,whichord=1,linecolor=black,method=rk4]{0}{10}{\InitCond}{\Faiglelapin}
  \psplotDiffEqn[whichabs=0,whichord=1,linecolor=blue]{0}{10}{\InitCond}{\Faiglelapin}
\end{pspicture}
\end{center}

\begin{lstlisting}[label={fig:aiglelapin},xrightmargin=-1.5cm]
\def\InitCond{ 0 10 10}%% xa ya xl
\def\Faiglelapin{\Vaigle*(y[2]-y[0])/sqrt(y[1]^2+(y[2]-y[0])^2)|%
                 -\Vaigle*y[1]/sqrt(y[1]^2+(y[2]-y[0])^2)|%
                 -\Vlapin}
\def\Vlapin{1}  \def\Vaigle{1.6}
\psset{unit=.7,subgriddiv=0,gridcolor=lightgray,method=adams,algebraic=true,%
   plotpoints=20,showpoints=true}
\begin{pspicture}[showgrid=true](-3,-3)(10,10)
 \psplotDiffEqn[plotfuncy=pop 0,whichabs=2,linecolor=red]{0}{10}{\InitCond}{\Faiglelapin}
 \psplotDiffEqn[whichabs=0,whichord=1,linecolor=black,method=rk4]{0}{10}{\InitCond}{\Faiglelapin}
  \psplotDiffEqn[whichabs=0,whichord=1,linecolor=blue]{0}{10}{\InitCond}{\Faiglelapin}
\end{pspicture}
\end{lstlisting}


\begin{center}
\def\InitCond{ 0 10 10}%% xa ya xl
\def\Faiglelapin{\Vaigle*(y[2]-y[0])/sqrt(y[1]^2+(y[2]-y[0])^2)|%
                 -\Vaigle*y[1]/sqrt(y[1]^2+(y[2]-y[0])^2)|%
                 -\Vlapin}
\def\Vlapin{1}  \def\Vaigle{1.6}
\psset{unit=.7,subgriddiv=0,gridcolor=lightgray,method=adams,algebraic=true,%
   plotpoints=20,showpoints=true}
\begin{pspicture}[showgrid=true](0,-0.25)(10,14)
 \psplotDiffEqn[plotfuncy=dup 1 get dup mul exch dup 0 get exch 2 get sub dup
    mul add sqrt,linecolor=red,method=rk4]{0}{10}{\InitCond}{\Faiglelapin}
 \psplotDiffEqn[plotfuncy=dup 1 get dup mul exch dup 0 get exch 2 get sub dup
    mul add sqrt,linecolor=blue]{0}{10}{\InitCond}{\Faiglelapin}
 \psplotDiffEqn[plotfuncy=pop Func aload pop pop dup mul exch dup mul add sqrt,
    linecolor=yellow]{0}{10}{\InitCond}{\Faiglelapin}
\end{pspicture}
\end{center}
\egroup

\begin{lstlisting}[label={fig:aiglelapin},xrightmargin=-1.5cm]
\def\InitCond{ 0 10 10}%% xa ya xl
\def\Faiglelapin{\Vaigle*(y[2]-y[0])/sqrt(y[1]^2+(y[2]-y[0])^2)|%
                 -\Vaigle*y[1]/sqrt(y[1]^2+(y[2]-y[0])^2)|%
                 -\Vlapin}
\def\Vlapin{1}  \def\Vaigle{1.6}
\psset{unit=.7,subgriddiv=0,gridcolor=lightgray,method=adams,algebraic=true,%
   plotpoints=20,showpoints=true}
\begin{pspicture}[showgrid=true](10,12)
 \psplotDiffEqn[plotfuncy=dup 1 get dup mul exch dup 0 get exch 2 get sub dup
    mul add sqrt,linecolor=red,method=rk4]{0}{10}{\InitCond}{\Faiglelapin}
 \psplotDiffEqn[plotfuncy=dup 1 get dup mul exch dup 0 get exch 2 get sub dup
    mul add sqrt,linecolor=blue]{0}{10}{\InitCond}{\Faiglelapin}
 \psplotDiffEqn[plotfuncy=pop Func aload pop pop dup mul exch dup mul add sqrt,
    linecolor=yellow]{0}{10}{\InitCond}{\Faiglelapin}
\end{pspicture}
\end{lstlisting}


%--------------------------------------------------------------------------------------
\subsubsection{$y''=y$}
%--------------------------------------------------------------------------------------

Beginning with the initial equation $\displaystyle y(x)=Ae^x+Be^{-x}$ we get the hyperbolic
trigonometrical functions.

\begin{center}
\bgroup
\def\Funct{exch}   \psset{xunit=5cm, yunit=0.75cm}
\begin{pspicture}(0,-0.25)(2,7)\psgrid[subgriddiv=1,griddots=10]
 \psplot[linewidth=4\pslinewidth, linecolor=green]{0}{2}{Euler x exp}  %%e^x
 \psplotDiffEqn[linecolor=magenta, plotpoints=11]{0}{2}{1 1}{\Funct}
 \psplotDiffEqn[linecolor=blue, plotpoints=101]{0}{2}{1 1}{\Funct}
 \psplotDiffEqn[linecolor=red, method=rk4, plotpoints=11]{0}{2}{1 1}{\Funct}
 \psplot[linewidth=4\pslinewidth, linecolor=green]{0}{2}{Euler dup x exp  %%ch(x)
    exch x neg exp add 2 div}
 \psplotDiffEqn[linecolor=magenta, plotpoints=11]{0}{2}{1 0}{\Funct}
 \psplotDiffEqn[linecolor=blue, plotpoints=101]{0}{2}{1 0}{\Funct}
 \psplotDiffEqn[linecolor=red, method=rk4, plotpoints=11]{0}{2}{1 0}{\Funct}
 \psplot[linewidth=4\pslinewidth, linecolor=green]{0}{2}{Euler dup x exp
     exch x neg exp sub 2 div}  %%sh(x)
 \psplotDiffEqn[linecolor=magenta, plotpoints=11]{0}{2}{0 1}{\Funct}
 \psplotDiffEqn[linecolor=blue, plotpoints=101]{0}{2}{0 1}{\Funct}
 \psplotDiffEqn[linecolor=red, method=rk4, plotpoints=11]{0}{2}{0 1}{\Funct}
 \rput*(1.3,.9){\psline[linecolor=magenta](-.75cm,0)}\rput*[l](1.3,.9){\small\textsc{Euler} order 1 $h=1$}
 \rput*(1.3,.8){\psline[linecolor=blue](-.75cm,0)}\rput*[l](1.3,.8){\small\textsc{Euler} order 1 $h=0{,}1$}
 \rput*(1.3,.7){\psline[linecolor=red](-.75cm,0)}\rput*[l](1.3,.7){\small RK order 4 $h=1$}
 \rput*(1.3,.6){\psline[linecolor=green](-.75cm,0)}\rput*[l](1.3,.6){\small exact solution}
\end{pspicture}
\egroup
\end{center}

\begin{lstlisting}[label={fig:minusexp},xrightmargin=-1.5cm]
\def\Funct{exch}   \psset{xunit=5cm, yunit=0.75cm}
\begin{pspicture}(0,-0.25)(2,7)\psgrid[subgriddiv=1,griddots=10]
 \psplot[linewidth=4\pslinewidth, linecolor=green]{0}{2}{Euler x exp}  %%e^x
 \psplotDiffEqn[linecolor=magenta, plotpoints=11]{0}{2}{1 1}{\Funct}
 \psplotDiffEqn[linecolor=blue, plotpoints=101]{0}{2}{1 1}{\Funct}
 \psplotDiffEqn[linecolor=red, method=rk4, plotpoints=11]{0}{2}{1 1}{\Funct}
 \psplot[linewidth=4\pslinewidth, linecolor=green]{0}{2}{Euler dup x exp  %%ch(x)
    exch x neg exp add 2 div}
 \psplotDiffEqn[linecolor=magenta, plotpoints=11]{0}{2}{1 0}{\Funct}
 \psplotDiffEqn[linecolor=blue, plotpoints=101]{0}{2}{1 0}{\Funct}
 \psplotDiffEqn[linecolor=red, method=rk4, plotpoints=11]{0}{2}{1 0}{\Funct}
 \psplot[linewidth=4\pslinewidth, linecolor=green]{0}{2}{Euler dup x exp
     exch x neg exp sub 2 div}  %%sh(x)
 \psplotDiffEqn[linecolor=magenta, plotpoints=11]{0}{2}{0 1}{\Funct}
 \psplotDiffEqn[linecolor=blue, plotpoints=101]{0}{2}{0 1}{\Funct}
 \psplotDiffEqn[linecolor=red, method=rk4, plotpoints=11]{0}{2}{0 1}{\Funct}
 \rput*(1.3,.9){\psline[linecolor=magenta](-.75cm,0)}\rput*[l](1.3,.9){\small\textsc{Euler} order 1 $h=1$}
 \rput*(1.3,.8){\psline[linecolor=blue](-.75cm,0)}\rput*[l](1.3,.8){\small\textsc{Euler} order 1 $h=0{,}1$}
 \rput*(1.3,.7){\psline[linecolor=red](-.75cm,0)}\rput*[l](1.3,.7){\small RK order 4 $h=1$}
 \rput*(1.3,.6){\psline[linecolor=green](-.75cm,0)}\rput*[l](1.3,.6){\small exact solution}
\end{pspicture}
\end{lstlisting}

%--------------------------------------------------------------------------------------
\clearpage
\subsubsection{$y''=-y$}
%--------------------------------------------------------------------------------------
\begin{center}
\bgroup
\def\Funct{exch neg}
\psset{xunit=1, yunit=4}
\def\quatrepi{12.5663706144}%%4pi=12.5663706144
\begin{pspicture}(0,-1.25)(\quatrepi,1.25)\psgrid[subgriddiv=1,griddots=10]
 \psplot[linewidth=4\pslinewidth,linecolor=green]{0}{\quatrepi}{x RadtoDeg cos}%%cos(x)
 \psplotDiffEqn[linecolor=blue, plotpoints=201]{0}{3.1415926}{1 0}{\Funct}
 \psplotDiffEqn[linecolor=red, method=rk4, plotpoints=31]{0}{\quatrepi}{1 0}{\Funct}
 \psplot[linewidth=4\pslinewidth,linecolor=green]{0}{\quatrepi}{x RadtoDeg sin}  %%sin(x)
 \psplotDiffEqn[linecolor=blue,plotpoints=201]{0}{3.1415926}{0 1}{\Funct}
 \psplotDiffEqn[linecolor=red,method=rk4, plotpoints=31]{0}{\quatrepi}{0 1}{\Funct}
 \rput*(3.3,.9){\psline[linecolor=magenta](-.75cm,0)}\rput*[l](3.3,.9){\small Euler order 1 $h=1$}
 \rput*(3.3,.8){\psline[linecolor=blue](-.75cm,0)}\rput*[l](3.3,.8){\small Euler order 1 $h=0{,}1$}
 \rput*(3.3,.7){\psline[linecolor=red](-.75cm,0)}\rput*[l](3.3,.7){\small RK order 4 $h=1$}
 \rput*(3.3,.6){\psline[linecolor=green](-.75cm,0)}\rput*[l](3.3,.6){\small exact solution}
\end{pspicture}
\egroup
\end{center}

\begin{lstlisting}[label={fig:minusexp2}]
\def\Funct{exch neg}
\psset{xunit=1, yunit=4}
\def\quatrepi{12.5663706144}%%4pi=12.5663706144
\begin{pspicture}(0,-1.25)(\quatrepi,1.25)\psgrid[subgriddiv=1,griddots=10]
 \psplot[linewidth=4\pslinewidth,linecolor=green]{0}{\quatrepi}{x RadtoDeg cos}%%cos(x)
 \psplotDiffEqn[linecolor=blue, plotpoints=201]{0}{3.1415926}{1 0}{\Funct}
 \psplotDiffEqn[linecolor=red, method=rk4, plotpoints=31]{0}{\quatrepi}{1 0}{\Funct}
 \psplot[linewidth=4\pslinewidth,linecolor=green]{0}{\quatrepi}{x RadtoDeg sin}  %%sin(x)
 \psplotDiffEqn[linecolor=blue,plotpoints=201]{0}{3.1415926}{0 1}{\Funct}
 \psplotDiffEqn[linecolor=red,method=rk4, plotpoints=31]{0}{\quatrepi}{0 1}{\Funct}
 \rput*(3.3,.9){\psline[linecolor=magenta](-.75cm,0)}\rput*[l](3.3,.9){\small Euler order 1 $h=1$}
 \rput*(3.3,.8){\psline[linecolor=blue](-.75cm,0)}\rput*[l](3.3,.8){\small Euler order 1 $h=0{,}1$}
 \rput*(3.3,.7){\psline[linecolor=red](-.75cm,0)}\rput*[l](3.3,.7){\small RK order 4 $h=1$}
 \rput*(3.3,.6){\psline[linecolor=green](-.75cm,0)}\rput*[l](3.3,.6){\small exact solution}
\end{pspicture}
\end{lstlisting}

%--------------------------------------------------------------------------------------
\clearpage
\subsubsection{The mechanical pendulum: $y''=-\frac{g}{l}\sin(y)$}% $
%--------------------------------------------------------------------------------------

For small \Index{oscillation}s $\sin(y)\simeq y$:

\[ y(x)=y_0\cos\left(\sqrt{\frac{g}{l}}x\right) \]

The function $f$ is written in PostScript code:

\begin{lstlisting}[style=syntax]
exch RadtoDeg sin -9.8 mul %% y' -gsin(y)
\end{lstlisting}

\begin{center}
\bgroup
\def\Func{y[1]|-9.8*sin(y[0])}
\psset{yunit=2,xunit=4,algebraic=true,linewidth=1.5pt}
\begin{pspicture}(0,-2.25)(3,2.25)
  \psaxes{->}(0,0)(0,-2)(3,2)
  \psplot[linewidth=3\pslinewidth, linecolor=Orange]{0}{3}{.1*cos(sqrt(9.8)*x)}
  \psset{method=rk4,plotpoints=50,linecolor=blue}
  \psplotDiffEqn{0}{3}{.1 0}{\Func}
  \psplot[linewidth=3\pslinewidth,linecolor=Orange]{0}{3}{.25*cos(sqrt(9.8)*x)}
  \psplotDiffEqn{0}{3}{.25 0}{\Func}
  \psplotDiffEqn{0}{3}{.5 0}{\Func}
  \psplotDiffEqn{0}{3}{1 0}{\Func}
  \psplotDiffEqn[plotpoints=100]{0}{3}{Pi 2 div 0}{\Func}
\end{pspicture}
\egroup
\end{center}

\begin{lstlisting}[label=fig:second]
\def\Func{y[1]|-9.8*sin(y[0])}
\psset{yunit=2,xunit=4,algebraic=true,linewidth=1.5pt}
\begin{pspicture}(0,-2.25)(3,2.25)
  \psaxes{->}(0,0)(0,-2)(3,2)
  \psplot[linewidth=3\pslinewidth, linecolor=Orange]{0}{3}{.1*cos(sqrt(9.8)*x)}
  \psset{method=rk4,plotpoints=50,linecolor=blue}
  \psplotDiffEqn{0}{3}{.1 0}{\Func}
  \psplot[linewidth=3\pslinewidth,linecolor=Orange]{0}{3}{.25*cos(sqrt(9.8)*x)}
  \psplotDiffEqn{0}{3}{.25 0}{\Func}
  \psplotDiffEqn{0}{3}{.5 0}{\Func}
  \psplotDiffEqn{0}{3}{1 0}{\Func}
  \psplotDiffEqn[plotpoints=100]{0}{3}{Pi 2 div 0}{\Func}
\end{pspicture}
\end{lstlisting}

%--------------------------------------------------------------------------------------
\clearpage
\subsubsection{$y''=-\frac{y'}{4}-2y$}% $
%--------------------------------------------------------------------------------------

For $y_0=5$ and $y'_0=0$ the solution is:

\[
5e^{-\frac{x}{8}}\left(\cos\left(\omega x\right)+\frac{\sin(\omega x)}{8\omega}\right)
\mbox{ avec } \omega=\frac{\sqrt{127}}{8}
\]

\begin{center}
\bgroup
\psset{xunit=.6,yunit=0.8,plotpoints=500}
\begin{pspicture}(0,-4.25)(26,5.25)
  \psaxes{->}(0,0)(0,-4)(26,5)
  \psplot[plotpoints=200,linewidth=4\pslinewidth,linecolor=gray]{0}{26}{%
     Euler x -8 div exp x 127 sqrt 8 div mul RadtoDeg dup cos 5 mul exch sin 127 sqrt div 5 mul add mul}
  \psplotDiffEqn[linecolor=red,linewidth=5\pslinewidth]{0}{26}{5 0}
     {dup 3 1 roll -4 div exch 2 mul sub}
  \psplotDiffEqn[linecolor=black,algebraic=true]{0}{26}{5 0} {y[1]|-y[1]/4-2*y[0]}
  \psset{method=rk4, plotpoints=50}
  \psplotDiffEqn[linecolor=blue,linewidth=5\pslinewidth]{0}{26}{5 0}{%
      dup 3 1 roll -4 div exch 2 mul sub}
  \psplotDiffEqn[linecolor=black,algebraic=true]{0}{26}{5 0}{y[1]|-y[1]/4-2*y[0]}
\end{pspicture}
\egroup
\end{center}

\begin{lstlisting}
\psset{xunit=.6,yunit=0.8,plotpoints=500}
\begin{pspicture}(0,-4.25)(26,5.25)
  \psaxes{->}(0,0)(0,-4)(26,5)
  \psplot[plotpoints=200,linewidth=4\pslinewidth,linecolor=gray]{0}{26}{%
     Euler x -8 div exp x 127 sqrt 8 div mul RadtoDeg dup cos 5 mul exch sin 127 sqrt div 5 mul add mul}
  \psplotDiffEqn[linecolor=red,linewidth=5\pslinewidth]{0}{26}{5 0}
     {dup 3 1 roll -4 div exch 2 mul sub}
  \psplotDiffEqn[linecolor=black,algebraic=true]{0}{26}{5 0} {y[1]|-y[1]/4-2*y[0]}
  \psset{method=rk4, plotpoints=50}
  \psplotDiffEqn[linecolor=blue,linewidth=5\pslinewidth]{0}{26}{5 0}{%
      dup 3 1 roll -4 div exch 2 mul sub}
  \psplotDiffEqn[linecolor=black,algebraic=true]{0}{26}{5 0}{y[1]|-y[1]/4-2*y[0]}
\end{pspicture}
\end{lstlisting}


\clearpage
\subsection{Save final state of a equation}
With the macros \Lcs{BeginSaveFinalState} and \Lcs{EndSaveFinalState} the
end values of a differential equation
can be saved and then used with the optional argument \Lkeyword{GetFinalState}  
as starting values for another equation.

\begin{lstlisting}
\psset{unit=10cm,linewidth=2pt}
\begin{pspicture}(1,1)\psgrid[subgridcolor=black!20,subgriddiv=20]
\BeginSaveFinalState
 \psplotDiffEqn[
   whichabs=0,whichord=1,linecolor=red,method=rk4,
   plotpoints=10,showpoints=true]{0}{1}{0 0}{
   pop pop
   x dup mul 2 div 180 mul cos %% dx/dt
   x dup mul 2 div 180 mul sin %% dy/dt
 }
 \psplotDiffEqn[GetFinalState,
   whichabs=0,whichord=1,linecolor=blue,method=rk4,%SaveFinalState,
   plotpoints=10,showpoints=true]{1}{2}{0 0}{
   pop pop
   x dup mul 2 div 180 mul cos %% dx/dt
   x dup mul 2 div 180 mul sin %% dy/dt
 }
 \psplotDiffEqn[GetFinalState,
   whichabs=0,whichord=1,linecolor=cyan,method=rk4,%SaveFinalState,
   plotpoints=19,showpoints=true]{2}{3}{0 0 }{
   pop pop
   x dup mul 2 div 180 mul cos %% dx/dt
   x dup mul 2 div 180 mul sin %% dy/dt
 }
\EndSaveFinalState
\end{pspicture}
\end{lstlisting}


\bigskip
\begin{center}
\psset{unit=6cm,linewidth=2pt}
\begin{pspicture}(1,1)\psgrid[subgridcolor=black!20,subgriddiv=20]
\BeginSaveFinalState
 \psplotDiffEqn[
   whichabs=0,whichord=1,linecolor=red,method=rk4,
   plotpoints=10,showpoints=true]{0}{1}{0 0}{
   pop pop
   x dup mul 2 div 180 mul cos %% dx/dt
   x dup mul 2 div 180 mul sin %% dy/dt
 }
 \psplotDiffEqn[GetFinalState,
   whichabs=0,whichord=1,linecolor=blue,method=rk4,%SaveFinalState,
   plotpoints=10,showpoints=true]{1}{2}{0 0}{
   pop pop
   x dup mul 2 div 180 mul cos %% dx/dt
   x dup mul 2 div 180 mul sin %% dy/dt
 }
 \psplotDiffEqn[GetFinalState,
   whichabs=0,whichord=1,linecolor=cyan,method=rk4,%SaveFinalState,
   plotpoints=19,showpoints=true]{2}{3}{0 0 }{
   pop pop
   x dup mul 2 div 180 mul cos %% dx/dt
   x dup mul 2 div 180 mul sin %% dy/dt
 }
\EndSaveFinalState
\end{pspicture}
\end{center}

\psset{unit=1cm,linewidth=0.75pt}


%--------------------------------------------------------------------------------------
\clearpage
\section{\nxLcs{psMatrixPlot}}\label{sec:psMatrix}
%--------------------------------------------------------------------------------------
\begin{filecontents}{matrix.data}
/dotmatrix [ %
0  1  1  0  0  0  0  1  1  1
0  1  1  0  1  1  1  0  1  0
1  0  1  1  0  0  0  1  1  0
0  0  1  0  0  0  0  0  1  1
1  1  1  1  1  0  1  0  0  1
0  0  1  1  0  1  0  1  1  1
1  0  0  0  1  1  0  0  0  1
0  0  0  1  1  1  0  1  1  0
1  1  0  0  0  0  1  0  0  1
1  0  1  0  0  1  1  1  0  0
] def
\end{filecontents}


This macro allows you to visualize a matrix. The datafile must be
defined as a PostScript matrix named \Lps{dotmatrix}:
\begin{lstlisting}[style=syntax]
/dotmatrix [ %  <------------ important line
0  1  1  0  0  0  0  1  1  1
0  1  1  0  1  1  1  0  1  0
1  0  1  1  0  0  0  1  1  0
0  0  1  0  0  0  0  0  1  1
1  1  1  1  1  0  1  0  0  1
0  0  1  1  0  1  0  1  1  1
1  0  0  0  1  1  0  0  0  1
0  0  0  1  1  1  0  1  1  0
1  1  0  0  0  0  1  0  0  1
1  0  1  0  0  1  1  1  0  0
] def        %  <------------ important line
\end{lstlisting}

Only the value 0 is important, in which case nothing happens, and
for all other cases a dot is printed. The syntax of the macro is:

\begin{BDef}
\Lcs{psMatrixPlot}\OptArgs\Largb{rows}\Largb{columns}\Largb{data file}
\end{BDef}

The \Index{matrix} is scanned line by line from the the first one to the
last. In general it appears as a bottom-to-top version of the
above listed matrix, the first row $0\,1\,1\,0\,0\,0\,0\,1\,1\,1$
is the first plotted line ($y=1$). With the option
\Lkeyword{ChangeOrder}=\true\ it looks exactly like the above view.

\bgroup
\begin{center}
\psscalebox{0.6}{%
\begin{pspicture}(-0.5,-0.75)(11,11)
  \psaxes{->}(11,11)
  \psMatrixPlot[dotsize=1.1cm,dotstyle=square*,linecolor=magenta]%
    {10}{10}{matrix.data}
  \psMatrixPlot[dotsize=.5cm,dotstyle=o,ChangeOrder]{10}{10}{matrix.data}
\end{pspicture}}\quad
\psscalebox{0.6}{%
\begin{pspicture}(-0.5,-0.75)(11,11)
  \psaxes[ticksize=-5pt 0]{->}(11,11)
  \psMatrixPlot[dotsize=1.1cm,dotstyle=square*,linecolor=magenta,XYoffset=-0.5]%
    {10}{10}{matrix.data}
  \psMatrixPlot[dotsize=.5cm,dotstyle=o,ChangeOrder,XYoffset=-0.5]{10}{10}{matrix.data}
\end{pspicture}}
\end{center}

\begin{lstlisting}
\psscalebox{0.6}{%
\begin{pspicture}(-0.5,-0.75)(11,11)
  \psaxes[ticksize=-5pt 0]{->}(11,11)
  \psMatrixPlot[dotsize=1.1cm,dotstyle=square*,linecolor=magenta]%
    {10}{10}{matrix.data}
  \psMatrixPlot[dotsize=.5cm,dotstyle=o,ChangeOrder]{10}{10}{matrix.data}
\end{pspicture}}\quad
\psscalebox{0.6}{%
\begin{pspicture}(-0.5,-0.75)(11,11)
  \psaxes{->}(11,11)
  \psMatrixPlot[dotsize=1.1cm,dotstyle=square*,linecolor=magenta,XYoffset=-0.5]%
    {10}{10}{matrix.data}
  \psMatrixPlot[dotsize=.5cm,dotstyle=o,ChangeOrder,XYoffset=-0.5]{10}{10}{matrix.data}
\end{pspicture}}
\end{lstlisting}

\begin{LTXexample}[pos=t,preset=\centering]
\begin{pspicture}(-0.5,-0.75)(11,11)
  \psaxes[ticksize=-5pt 0]{->}(11,11)
  \psMatrixPlot[dotscale=3,dotstyle=*,linecolor=blue]{10}{8}{matrix.data}
\end{pspicture}
\end{LTXexample}

\clearpage
With the \Lkeyword{colorType}=1 the data is printed as continous color
in the range of the wavelength. The smallest value of the data array
is set to red and the biggest value is set to violett. All other values
are substituted by the corresponding color of the wavlength.
\Lkeyword{colorType}=2 ist the same, but vice versa
with the color, from violet to red. \Lkeyword{colorType}=3 is the grayscale
image and \Lkeyword{colorType}=4 the same invers.

The following examples use a 200$\times$200
matrix data, which is saved as /dotmatrix [...] in the file \LFile{pstricks-add-doc.dat}.

\begin{LTXexample}[pos=t,preset=\centering]
\begin{pspicture}(10,10)
  \psMatrixPlot[colorType=1,xStep=0.05,yStep=0.05]{200}{200}{dotmatrix.data}
\end{pspicture}
\end{LTXexample}

\begin{LTXexample}[pos=t,preset=\centering]
\begin{pspicture}(10,10)
  \psMatrixPlot[colorType=2,xStep=0.05,yStep=0.05]{200}{200}{dotmatrix.data}
\end{pspicture}
\end{LTXexample}

\begin{LTXexample}[pos=t,preset=\centering]
\begin{pspicture}(10,10)
  \psMatrixPlot[colorType=3,xStep=0.05,yStep=0.05]{200}{200}{dotmatrix.data}
\end{pspicture}
\end{LTXexample}

\begin{LTXexample}[pos=t,preset=\centering]
\begin{pspicture}(10,10)
  \psMatrixPlot[colorType=4,xStep=0.05,yStep=0.05]{200}{200}{dotmatrix.data}
\end{pspicture}
\end{LTXexample}
\egroup

\clearpage
With the \Lkeyword{colorType}=5 the color setting can be user defined by the
optional argument \Lkeyword{colorTypeDef}. On the stack is the current value
which can be used for the setting but must be left on the stack when everything
is finished. The following example prints the 0 as color white, the value 1 as
black and all other values depending to the corresponding gray value.

\begin{filecontents*}{matrix1.data}
/dotmatrix [ % <------------ important line
3 0 0 0 0 0 0 0 1 2
0 0 0 0 0 0 0 1 2 1
8 0 0 0 0 0 1 2 1 0
0 0 0 0 0 1 2 1 0 0
0 0 0 0 1 2 1 0 0 0
9 0 0 1 2 1 3 0 0 0
0 0 1 2 1 4 0 0 0 0
0 1 2 1 5 0 0 0 0 0
1 2 1 6 0 0 0 0 0 0
2 1 7 0 0 0 0 0 0 3
] def % <------------ important line
\end{filecontents*}

\begin{center}
\psscalebox{0.7}{%
\begin{pspicture}(-0.5,-0.75)(11,11)
\psaxes[ticksize=-5pt 0]{->}(11,11)
\psMatrixPlot[
  colorType=5,
  colorTypeDef={
    dup /value exch def % save value and leave one on the stack
    value Min sub dMaxMin div neg 1 add 300 mul 400 add \pswavelengthToGRAY 
    value 0 eq \pslbrace 1 \psrbrace if % 
    value 1 eq \pslbrace 0 \psrbrace if  
    setgray 
  },
  dotsize=1.1cm,xStep=1,yStep=1,dotstyle=square*]{10}{10}{matrix1.data}
\end{pspicture}}
\end{center}


\begin{lstlisting}
\begin{filecontents}{matrix1.data}
/dotmatrix [ % <------------ important line
3 0 0 0 0 0 0 0 1 2
0 0 0 0 0 0 0 1 2 1
8 0 0 0 0 0 1 2 1 0
0 0 0 0 0 1 2 1 0 0
0 0 0 0 1 2 1 0 0 0
9 0 0 1 2 1 3 0 0 0
0 0 1 2 1 4 0 0 0 0
0 1 2 1 5 0 0 0 0 0
1 2 1 6 0 0 0 0 0 0
2 1 7 0 0 0 0 0 0 3
] def % <------------ important line
\end{filecontents}
\psscalebox{0.7}{%
\begin{pspicture}(-0.5,-0.75)(11,11)
\psaxes[ticksize=-5pt 0]{->}(11,11)
\psMatrixPlot[
  colorType=5,
  colorTypeDef={
    dup /value exch def % save value and leave one on the stack
    value Min sub dMaxMin div neg 1 add 300 mul 400 add \pswavelengthToGRAY 
    value 0 eq \pslbrace 1 \psrbrace if % 
    value 1 eq \pslbrace 0 \psrbrace if  
    setgray 
  },
  dotsize=1.1cm,xStep=1,yStep=1,dotstyle=square*]{10}{10}{matrix1.data}
\end{pspicture}}
\end{lstlisting}


\Lps{if} statements in the color definition must be enclosed with \Lcs{pslbrace} and \Lcs{psrbrace}
when they are parentheses used in PostScript. In the above example the color definition should be
modified when the matrix is a real big one, in such a case a nested \Lps{ifelse} makes more sense:

\begin{lstlisting}
  colorTypeDef={
    dup /value exch def 
    value 0 eq 
      \pslbrace 1 setgray \psrbrace
      \pslbrace value 1 eq 
        \pslbrace 0 setgray \psrbrace
        \pslbrace Min sub dMaxMin div neg 1 add 300 mul 400 add
          \pswavelengthToGRAY setgray \psrbrace ifelse
      \psrbrace ifelse 
  },
\end{lstlisting}

Replace the \Lcs{pslbrace} and \Lcs{psrbrace} with \{ and \} if it maybe confusing to read:

\begin{lstlisting}
    dup /value exch def 
    value 0 eq 
      { 1 setgray }
      { value 1 eq 
        { 0 setgray }
        { Min sub dMaxMin div neg 1 add 300 mul 400 add
          \pswavelengthToGRAY setgray } ifelse
      } ifelse 
\end{lstlisting}

Another possibility is to define the color procedure onside the data file, where
it \emph{must} be named \Lps{colorTypeDef}. If such a definition exists, the one from
the optional argument \Lkeyword{colorTypeDef} will be ignored. There can be no
\TeX-specific code inside this definition because it is read on PostScript level,
the reason why \Lcs{pswavelengthToGRAY} cannot be used.

\begin{center}
\begin{filecontents}{matrix1.data}
/colorTypeDef {
  dup /value exch def 
  value 0 eq 
    { 1 setgray }
    { value 1 eq 
      { 0 setgray }
      { Min sub dMaxMin div neg 1 add 300 mul 400 add
%        \pswavelengthToGRAY not possible
         tx@addDict begin wavelengthToRGB Red Green Blue end 
        setrgbcolor
      } ifelse
    } ifelse 
} def
/dotmatrix [ % <------------ important line
3 0 0 0 0 0 0 0 1 2
0 0 0 0 0 0 0 1 2 1
8 0 0 0 0 0 1 2 1 0
0 0 0 0 0 1 2 1 0 0
0 0 0 0 1 2 1 0 0 0
9 0 0 1 2 1 3 0 0 0
0 0 1 2 1 4 0 0 0 0
0 1 2 1 5 0 0 0 0 0
1 2 1 6 0 0 0 0 0 0
2 1 7 0 0 0 0 0 0 3
] def % <------------ important line
\end{filecontents}
\psscalebox{0.7}{%
\begin{pspicture}(-0.5,-0.75)(11,11)
\psaxes[ticksize=-5pt 0]{->}(11,11)
\psMatrixPlot[
  colorType=5,dotsize=1.1cm,xStep=1,yStep=1,dotstyle=square*]{10}{10}{matrix1.data}
\end{pspicture}}
\end{center}

\begin{lstlisting}
\begin{filecontents}{matrix1.data}
/colorTypeDef {
  dup /value exch def 
  value 0 eq 
    { 1 setgray }
    { value 1 eq 
      { 0 setgray }
      { Min sub dMaxMin div neg 1 add 300 mul 400 add
%        \pswavelengthToRGB not possible
         tx@addDict begin wavelengthToRGB Red Green Blue end 
        setrgbcolor
      } ifelse
    } ifelse 
} def
/dotmatrix [ % <------------ important line
3 0 0 0 0 0 0 0 1 2
0 0 0 0 0 0 0 1 2 1
8 0 0 0 0 0 1 2 1 0
0 0 0 0 0 1 2 1 0 0
0 0 0 0 1 2 1 0 0 0
9 0 0 1 2 1 3 0 0 0
0 0 1 2 1 4 0 0 0 0
0 1 2 1 5 0 0 0 0 0
1 2 1 6 0 0 0 0 0 0
2 1 7 0 0 0 0 0 0 3
] def % <------------ important line
\end{filecontents}
\psscalebox{0.7}{%
\begin{pspicture}(-0.5,-0.75)(11,11)
\psaxes[ticksize=-5pt 0]{->}(11,11)
\psMatrixPlot[colorType=5,dotsize=1.1cm,xStep=1,yStep=1,
  dotstyle=square*]{10}{10}{matrix1.data}
\end{pspicture}}
\end{lstlisting}



%--------------------------------------------------------------------------------------
\section{Dashed Lines}
%--------------------------------------------------------------------------------------
Tobias Nähring has implemented an enhanced feature for dashed
lines. The number of arguments is no longer limited.

\begin{BDef}
\Lkeyword{dash}=value1\OptArg*{unit} value2\OptArg*{unit} \ldots
\end{BDef}

\begin{LTXexample}[width=0.4\linewidth]
\psset{linewidth=2.5pt,unit=0.6}
\begin{pspicture}(-5,-4)(5,4)
 \psgrid[subgriddiv=0,griddots=10,gridlabels=0pt]
  \psset{linestyle=dashed}
  \pscurve[dash=5mm 1mm 1mm 1mm,linewidth=0.1](-5,4)(-4,3)(-3,4)(-2,3)
  \psline[dash=5mm 1mm 1mm 1mm 1mm 1mm 1mm 1mm 1mm 1mm](-5,0.9)(5,0.9)
  \psccurve[linestyle=solid](0,0)(1,0)(1,1)(0,1)
  \psccurve[linestyle=dashed,dash=5mm 2mm 0.1 0.2,linetype=0](0,0)(-2.5,0)(-2.5,-2.5)(0,-2.5)
  \pscurve[dash=3mm 3mm 1mm 1mm,linecolor=red,linewidth=2pt](5,-4)(5,2)(4.5,3.5)(3,4)(-5,4)
\end{pspicture}
\end{LTXexample}



\clearpage
%--------------------------------------------------------------------------------------
\section{Arrows}
%--------------------------------------------------------------------------------------
\subsection{Definition}
%--------------------------------------------------------------------------------------
\LPack{pstricks-add} defines the following "`arrows"':

\begin{center}
  \bgroup
  \def\myline#1{\psline[linecolor=red,linewidth=0.5pt,arrowscale=1.5]{#1}(0,1ex)(1.3,1ex)}%
  \psset{arrowscale=1.5}
  \begin{tabular}{@{} c @{\qquad} p{3cm} l @{}}%
    Value & Example & Name \\[2pt]\hline
    \Lnotation{-}      & \myline{-}      & None\\
    \Lnotation{<->}    & \myline{<->}    & Arrowheads.\\
    \Lnotation{>-<}    & \myline{>-<}    & Reverse arrowheads.\\
    \Lnotation{<{<}-{>}>}  & \myline{<<->>}  & Double arrowheads.\\
    \Lnotation{{>}>-{<}<}  & \myline{>>-<<}  & Double reverse arrowheads.\\
    \Lnotation{{|}-{|}}    & \myline{|-|}    & T-bars, flush to endpoints.\\
    \Lnotation{{|}*-{|}*}  & \myline{|*-|*}  & T-bars, centered on endpoints.\\
    \Lnotation{[-]}    & \myline{[-]}    & Square brackets.\\
    \Lnotation{]-[}    & \myline{]-[}    & Reversed square brackets.\\
    \Lnotation{(-)}    & \myline{(-)}    & Rounded brackets.\\
    \Lnotation{)-(}    & \myline{)-(}    & Reversed rounded brackets.\\
    \Lnotation{o-o}    & \myline{o-o}    & Circles, centered on endpoints.\\
    \Lnotation{*-*}    & \myline{*-*}    & Disks, centered on endpoints.\\
    \Lnotation{oo-oo}  & \myline{oo-oo}  & Circles, flush to endpoints.\\
    \Lnotation{**-**}  & \myline{**-**}  & Disks, flush to endpoints.\\
    \Lnotation{{|}<->{|}}  & \myline{|<->|}  & T-bars and arrows.\\
    \Lnotation{{|}>-<{|}}  & \myline{|>-<|}  & T-bars and reverse arrows.\\
    \Lnotation{h-h{|}}   & \myline{h-h}    & left/right hook arrows.\\
    \Lnotation{H-H{|}}   & \myline{H-H}    & left/right hook arrows.\\
    \Lnotation{v-v|}   & \myline{v-v}    & left/right inside vee arrows.\\
    \Lnotation{V-V|}   & \myline{V-V}    & left/right outside vee arrows.\\
    \Lnotation{f-f|}   & \myline{f-f}    & left/right inside filled arrows.\\
    \Lnotation{F-F|}   & \myline{F-F}    & left/right outside filled arrows.\\
    \Lnotation{t-t|}   & \myline{t-t}    & left/right inside slash arrows.\\[5pt]
    \Lnotation{T-T|}   & \myline{T-T}    & left/right outside slash arrows.\\
  \end{tabular}
  \egroup
\end{center}



You can also mix and match, e.g., \Lnotation{->}, \Lnotation{*-)} and \Lnotation{[->} are all valid values
of the \Lkeyword{arrows} parameter. The parameter can be set with

\begin{BDef}
\Lcs{psset}\Largb{arrows=<type>}
\end{BDef}

\noindent or for some macros with a special option, like\\[5pt]
\noindent\verb|\psline[<general options>]{<arrow type>}(A)(B)|\\
\noindent\verb/\psline[linecolor=red,linewidth=2pt]{|->}(0,0)(0,2)/ \ \psline[linecolor=red,linewidth=2pt]{|->}(0,0)(0,2)

\subsection{Multiple arrows}
There are two new options which are only valid for the arrow type \verb+<<+ or \verb+>>+.
\verb+nArrow+ sets both, the \verb+nArrowA+ and the  \verb+nArrowB+ parameter. The meaning
is declared in the following tables. Without setting one of these parameters the behaviour
is like the one described in the old PSTricks manual.

\begin{center}
\begin{tabular}{@{}lc@{}}%
    Value & Meaning \\[2pt]\hline
    \Lnotation{-{>}>}   & \ -A \\
    \Lnotation{{<}<-{>}>} & A-A\\
    \Lnotation{{<}<-}   & A-\ \\
    \Lnotation{{>}>-}   & B-\ \\
    \Lnotation{-{<}<}   & \ -B\\
    \Lnotation{{>}>-{<}<} & B-B\\
    \Lnotation{{>}>-{>}>} & B-A\\
    \Lnotation{{<}<-{<}<} & A-B
  \end{tabular}
\end{center}




\begin{center}
  \bgroup
  \psset{linecolor=red,linewidth=1pt,arrowscale=2}%
  \begin{tabular}{lp{2.8cm}}%
    Value & Example \\[2pt]\hline
    \verb+\psline{->>}(0,1ex)(2.3,1ex)+  & \psline{->>}(0,1ex)(2.3,1ex) \\
    \verb+\psline[nArrowsA=3]{->>}(0,1ex)(2.3,1ex)+  & \psline[nArrowsA=3]{->>}(0,1ex)(2.3,1ex)\\
    \verb+\psline[nArrowsA=5]{->>}(0,1ex)(2.3,1ex)+  & \psline[nArrowsA=5]{->>}(0,1ex)(2.3,1ex)\\
    \verb+\psline{<<-}(0,1ex)(2.3,1ex)+  & \psline{<<-}(0,1ex)(2.3,1ex)\\
    \verb+\psline[nArrowsA=3]{<<-}(0,1ex)(2.3,1ex)+  & \psline[nArrowsA=3]{<<-}(0,1ex)(2.3,1ex)\\
    \verb+\psline[nArrowsA=5]{<<-}(0,1ex)(2.3,1ex)+  & \psline[nArrowsA=5]{<<-}(0,1ex)(2.3,1ex)\\
    \verb+\psline{<<->>}(0,1ex)(2.3,1ex)+  & \psline{<<->>}(0,1ex)(2.3,1ex)\\
    \verb+\psline[nArrowsA=3]{<<->>}(0,1ex)(2.3,1ex)+  & \psline[nArrowsA=3]{<<->>}(0,1ex)(2.3,1ex)\\
    \verb+\psline[nArrowsA=5]{<<->>}(0,1ex)(2.3,1ex)+  & \psline[nArrowsA=5]{<<->>}(0,1ex)(2.3,1ex)\\
    \verb+\psline{<<-|}(0,1ex)(2.3,1ex)+  & \psline{<<-|}(0,1ex)(2.3,1ex)\\
    \verb+\psline[nArrowsA=3]{<<-<<}(0,1ex)(2.3,1ex)+  & \psline[nArrowsA=3]{<<-<<}(0,1ex)(2.3,1ex)\\
    \verb+\psline[nArrowsA=5]{<<-o}(0,1ex)(2.3,1ex)+  & \psline[nArrowsA=5]{<<-o}(0,1ex)(2.3,1ex)\\
    \verb+\psline[nArrowsA=3,nArrowsB=4]{<<-<<}(0,1ex)(2.3,1ex)+  & \psline[nArrowsA=3,nArrowsB=4]{<<-<<}(0,1ex)(2.3,1ex)\\
    \verb+\psline[nArrowsA=3,nArrowsB=4]{>>->>}(0,1ex)(2.3,1ex)+  & \psline[nArrowsA=3,nArrowsB=4]{>>->>}(0,1ex)(2.3,1ex)\\
    \verb+\psline[nArrowsA=1,nArrowsB=4]{>>->>}(0,1ex)(2.3,1ex)+  & \psline[nArrowsA=1,nArrowsB=4]{>>->>}(0,1ex)(2.3,1ex)\\
  \end{tabular}
  \egroup
\end{center}



\subsection{\texttt{hookarrow}}
%\begin{LTXexample}
\bgroup
\psset{arrowsize=8pt,arrowlength=1,linewidth=1pt,nodesep=2pt,shortput=tablr}
\large
\begin{psmatrix}[colsep=12mm,rowsep=10mm]
        &   & $R_2$            \\
        &   &   0   &   & $R_3$\\
$e_b:S$ & 1 &       & 1 & 0    \\
        &   &   0              \\
        &   &   $R_1$          \\
\end{psmatrix}
\ncline{h-}{1,3}{2,3}<{$e_{r2}$}>{$f_{r2}$}
\ncline{-h}{2,3}{3,2}<{$e_1$}
\ncline{-h}{3,1}{3,2}^{$e_s$}_{$f_{s}$}
\ncline{-h}{3,2}{4,3}>{$e_3$}<{$f_3$}
\ncline{-h}{4,3}{3,4}>{$e_4$}<{$f_4$}
\ncline{-h}{3,4}{2,3}>{$e_2$}<{$f_2$}
\ncline{-h}{3,4}{3,5}^{$e_5$}
\ncline{-h}{3,5}{2,5}<{$e_{r3}$}>{$f_{r3}$}
\ncline{-h}{4,3}{5,3}<{$e_{r1}$}>{$f_{r1}$}
%\end{LTXexample}
\egroup

\begin{lstlisting}
\psset{arrowsize=8pt,arrowlength=1,linewidth=1pt,nodesep=2pt,shortput=tablr}
\large
\begin{psmatrix}[colsep=12mm,rowsep=10mm]
        &   & $R_2$            \\
        &   &   0   &   & $R_3$\\
$e_b:S$ & 1 &       & 1 & 0    \\
        &   &   0              \\
        &   &   $R_1$          \\
\end{psmatrix}
\ncline{h-}{1,3}{2,3}<{$e_{r2}$}>{$f_{r2}$}\ncline{-h}{2,3}{3,2}<{$e_1$}
\ncline{-h}{3,1}{3,2}^{$e_s$}_{$f_{s}$}    \ncline{-h}{3,2}{4,3}>{$e_3$}<{$f_3$}
\ncline{-h}{4,3}{3,4}>{$e_4$}<{$f_4$}      \ncline{-h}{3,4}{2,3}>{$e_2$}<{$f_2$}
\ncline{-h}{3,4}{3,5}^{$e_5$}              
\ncline{-h}{3,5}{2,5}<{$e_{r3}$}>{$f_{r3}$}
\ncline{-h}{4,3}{5,3}<{$e_{r1}$}>{$f_{r1}$}
\end{lstlisting}



\subsection{\texttt{hookrightarrow} and \texttt{hookleftarrow}}
This is another type of arrow and is abbreviated with \Lnotation{H}.
The length and width of the hook is set by the new options
\Lkeyword{hooklength} and \Lkeyword{hookwidth}, which are by default set
to
%
\begin{BDef}
\Lcs{psset}\Largb{hooklength=3mm,hookwidth=1mm}
\end{BDef}
%
If the line begins with a right hook then the line ends with a left hook and vice versa:

\begin{LTXexample}[width=3cm]
\begin{pspicture}(3,4)
\psline[linewidth=5pt,linecolor=blue,hooklength=5mm,hookwidth=-3mm]{H->}(0,3.5)(3,3.5)
\psline[linewidth=5pt,linecolor=red,hooklength=5mm,hookwidth=3mm]{H->}(0,2.5)(3,2.5)
\psline[linewidth=5pt,hooklength=5mm,hookwidth=3mm]{H-H}(0,1.5)(3,1.5)
\psline[linewidth=1pt]{H-H}(0,0.5)(3,0.5)
\end{pspicture}
\end{LTXexample}


\begin{LTXexample}[width=7.25cm]
$\begin{psmatrix}
E&W_i(X)&&Y\\
&&W_j(X)
\psset{arrows=->,nodesep=3pt,linewidth=2pt}
\everypsbox{\scriptstyle}
\ncline[linecolor=red,arrows=H->,%
  hooklength=4mm,hookwidth=2mm]{1,1}{1,2}
\ncline{1,2}{1,4}^{\tilde{t}}
\ncline{1,2}{2,3}<{W_{ij}}
\ncline{2,3}{1,4}>{\tilde{s}}
\end{psmatrix}$
\end{LTXexample}


%--------------------------------------------------------------------------------------
\subsection{\nxLkeyword{ArrowInside} Option}
%--------------------------------------------------------------------------------------

It is now possible to have arrows inside lines and not only at the
beginning or the end. The new defined options

\psset{arrowscale=2,linecolor=red,unit=1cm,linewidth=1.5pt}
\begin{longtable}{l|>{\RaggedRight}p{8.5cm}|p{2.2cm}}
Name & Example & Output\\\hline
\endfirsthead
Name & Example & Output\\\hline
\endhead
\Lkeyword{ArrowInside} &
  \texttt{\textbackslash psline[ArrowInside=->](0,0)(2,0)} &
  \psline[ArrowInside=->](0,0.1)(2,0.1) \\
\Lkeyword{ArrowInsidePos} & \texttt{\textbackslash psline[ArrowInside=->,\%}
  \hspace*{20pt}\texttt{ArrowInsidePos=0.25](0,0)(2,0)}
& \psline[ArrowInside=->, ArrowInsidePos=0.25](0,0.1)(2,0.1) \\
\Lkeyword{ArrowInsidePos} & \texttt{\textbackslash psline[ArrowInside=->,\%}
  \hspace*{20pt}\texttt{ArrowInsidePos=10](0,0)(2,0)}
& \psline[ArrowInside=->, ArrowInsidePos=10](0,0.1)(2,0.1) \\
\Lkeyword{ArrowInsideNo} & \texttt{\textbackslash psline[ArrowInside=->,\%}
  \hspace*{20pt}\texttt{ArrowInsideNo=2](0,0)(2,0)}
& \psline[ArrowInside=->, ArrowInsideNo=2](0,0.1)(2,0.1) \\
\Lkeyword{ArrowInsideOffset} & \texttt{\textbackslash psline[ArrowInside=->,\%}
  \hspace*{20pt}\texttt{ArrowInsideNo=2,\%}\newline
  \hspace*{20pt}\texttt{ArrowInsideOffset=0.1](0,0)(2,0)}
& \psline[ArrowInside=->, ArrowInsideNo=2,ArrowInsideOffset=0.1](0,0.1)(2,0.1) \\
%
\Lkeyword{ArrowInside} & \texttt{\textbackslash psline[ArrowInside=->]\{->\}(0,0)(2,0)} &
  \psline[ArrowInside=->]{->}(0,0)(2,0)\\
\Lkeyword{ArrowInsidePos} & \texttt{\textbackslash psline[ArrowInside=->,\%}
  \hspace*{20pt}\texttt{ArrowInsidePos=0.25]\{->\}(0,0)(2,0)}
  & \psline[ArrowInside=->, ArrowInsidePos=0.25]{->}(0,0)(2,0) \\
\Lkeyword{ArrowInsidePos} & \texttt{\textbackslash psline[ArrowInside=->,\%}
  \hspace*{20pt}\texttt{ArrowInsidePos=10]\{->\}(0,0)(2,0)}
  & \psline[ArrowInside=->, ArrowInsidePos=10]{->}(0,0)(2,0) \\
\Lkeyword{ArrowInsideNo} & \texttt{\textbackslash psline[ArrowInside=->,\%}
  \hspace*{20pt}\texttt{ArrowInsideNo=2]\{->\}(0,0)(2,0)}
  & \psline[ArrowInside=->, ArrowInsideNo=2]{->}(0,0)(2,0) \\
\Lkeyword{ArrowInsideOffset} & \texttt{\textbackslash psline[ArrowInside=->,\%}
  \hspace*{20pt}\texttt{ArrowInsideNo=2,\%}\newline
  \hspace*{20pt}\texttt{ArrowInsideOffset=0.1]\{->\}(0,0)(2,0)}
  & \psline[ArrowInside=->, ArrowInsideNo=2,ArrowInsideOffset=0.1]{->}(0,0)(2,0) \\
%
\Lkeyword{ArrowFill} & \texttt{\textbackslash psline[ArrowFill=false,\%}
  \hspace*{20pt}\texttt{arrowinset=0]\{->\}(0,0)(2,0)} &
  \psline[ArrowFill=false,arrowinset=0]{->}(0,0)(2,0)\\
\Lkeyword{ArrowFill} & \texttt{\textbackslash psline[ArrowFill=false,\%}
  \hspace*{20pt}\texttt{arrowinset=0]\{<<->>\}(0,0)(2,0)} &
  \psline[ArrowFill=false,arrowinset=0]{<<->>}(0,0)(2,0)\\
\Lkeyword{ArrowFill} & \texttt{\textbackslash psline[ArrowInside=->,\%}\newline
  \hspace*{20pt}\texttt{arrowinset=0,\%}\newline
  \hspace*{20pt}\texttt{ArrowFill=false,\%}\newline
  \hspace*{20pt}\texttt{ArrowInsideNo=2,\%}\newline
  \hspace*{20pt}\texttt{ArrowInsideOffset=0.1]\{->\}(0,0)(2,0)}
  & \psline[ArrowInside=->, ArrowFill=false,ArrowInsideNo=2,ArrowInsideOffset=0.1]{->}(0,0)(2,0) \\
\end{longtable}

\medskip
Without the default arrow definition there is only the one inside
the line, defined by the type and the position. The position is
relative to the length of the whole line. $0.25$ means at $25\%$
of the line length. The peak of the arrow gets the coordinates
which are calculated by the macro. If you want arrows with an
absolute position difference, then choose a value greater than
\verb|1|, e.\,g. \verb|10| which places an arrow every 10~pt. The
default unit \verb|pt| cannot be changed.

\medskip
\noindent
\begin{tabularx}{\linewidth}{@{\color{red}\vrule width 2pt}lX@{}}
& The \Lkeyword{ArrowInside} takes only arrow definitions like \Lnotation{->} into account.
Arrows from right to left (\Lnotation{<-}) are not possible and ignored. If you need
such arrows, change the order of the pairs of coordinates for the line or curve macro.
\end{tabularx}

%--------------------------------------------------------------------------------------
\subsection{\nxLkeyword{ArrowFill} Option}
%--------------------------------------------------------------------------------------

By default all arrows are filled polygons. With the option
\Lkeyset{ArrowFill=false} there are ''white`` arrows. Only for the
beginning/end arrows are they empty, the inside arrows are
overpainted by the line.


\psset{arrowscale=1}
\begin{LTXexample}[width=3.5cm]
\psset{arrowscale=2.5}
\psline[linecolor=red,arrowinset=0]{<->}(-1,0)(2,0)
\end{LTXexample}

\begin{LTXexample}[width=3.5cm]
\psset{arrowscale=2.5}
\psline[linecolor=red,arrowinset=0,ArrowFill=false]{<->}(-1,0)(2,0)
\end{LTXexample}

\begin{LTXexample}[width=3.5cm]
\psset{arrowscale=2.5}
\psline[linecolor=red,arrowinset=0,arrowsize=0.2,
  ArrowFill=false]{<->}(-1,0)(2,0)
\end{LTXexample}

\begin{LTXexample}[width=3.5cm]
\psline[linecolor=blue,arrowscale=4,
  ArrowFill]{>>->>}(-1,0)(2,0)
\end{LTXexample}

\begin{LTXexample}[width=3.5cm]
\psline[linecolor=blue,arrowscale=4,
  ArrowFill=false]{>>->>}(-1,0)(2,0)
\rule{3cm}{0pt}\\[30pt]
\end{LTXexample}

\begin{LTXexample}[width=3.5cm]
\psline[linecolor=blue,arrowscale=4,
  ArrowFill]{>|->|}(-1,0)(2,0)
\end{LTXexample}

\begin{LTXexample}[width=3.5cm]
\psline[linecolor=blue,arrowscale=4,
  ArrowFill=false]{>|->|}(-1,0)(2,0)%
\end{LTXexample}


%--------------------------------------------------------------------------------------
\subsection{Examples}
%--------------------------------------------------------------------------------------

All examples are printed with \verb|\psset{arrowscale=2,linecolor=red}|.
\subsubsection{\nxLcs{psline}}

\bigskip
\begin{LTXexample}[width=2.5cm]
\begin{pspicture}(2,2)
\psset{arrowscale=2,ArrowFill=true}
\psline[ArrowInside=->]{|<->|}(2,1)
\end{pspicture}
\end{LTXexample}

\begin{LTXexample}[width=2.5cm]
\begin{pspicture}(2,2)
\psset{arrowscale=2,ArrowFill=true}
\psline[ArrowInside=-|]{|-|}(2,1)
\end{pspicture}
\end{LTXexample}

\begin{LTXexample}[width=2.5cm]
\begin{pspicture}(2,2)
\psset{arrowscale=2,ArrowFill=true}
\psline[ArrowInside=->,ArrowInsideNo=2]{->}(2,1)
\end{pspicture}
\end{LTXexample}

\begin{LTXexample}[width=2.5cm]
\begin{pspicture}(2,2)
\psset{arrowscale=2,ArrowFill=true}
\psline[ArrowInside=->,ArrowInsideNo=2,ArrowInsideOffset=0.1]{->}(2,1)
\end{pspicture}
\end{LTXexample}

\begin{LTXexample}[width=6.5cm]
\begin{pspicture}(6,2)
\psset{arrowscale=2,ArrowFill=true}
\psline[ArrowInside=-*]{->}(0,0)(2,1)(3,0)(4,0)(6,2)
\end{pspicture}
\end{LTXexample}

\begin{LTXexample}[width=6.5cm]
\begin{pspicture}(6,2)
\psset{arrowscale=2,ArrowFill=true}
\psline[ArrowInside=-*,ArrowInsidePos=0.25]{->}(0,0)(2,1)(3,0)(4,0)(6,2)
\end{pspicture}
\end{LTXexample}

\begin{LTXexample}[width=6.5cm]
\begin{pspicture}(6,2)
\psset{arrowscale=2,ArrowFill=true}
\psline[ArrowInside=-*,ArrowInsidePos=0.25,ArrowInsideNo=2]{->}%
   (0,0)(2,1)(3,0)(4,0)(6,2)
\end{pspicture}
\end{LTXexample}

\begin{LTXexample}[width=6.5cm]
\begin{pspicture}(6,2)
\psset{arrowscale=2,ArrowFill=true}
\psline[ArrowInside=->, ArrowInsidePos=0.25]{->}%
        (0,0)(2,1)(3,0)(4,0)(6,2)
\end{pspicture}
\end{LTXexample}

\begin{LTXexample}[width=6.5cm]
\begin{pspicture}(6,2)
\psset{arrowscale=2,ArrowFill=true}
\psline[linestyle=none,ArrowInside=->,ArrowInsidePos=0.25]{->}%
        (0,0)(2,1)(3,0)(4,0)(6,2)
\end{pspicture}
\end{LTXexample}

\begin{LTXexample}[width=6.5cm]
\begin{pspicture}(6,2)
\psset{arrowscale=2,ArrowFill=true}
\psline[ArrowInside=-<, ArrowInsidePos=0.75]{->}%
     (0,0)(2,1)(3,0)(4,0)(6,2)
\end{pspicture}
\end{LTXexample}

\begin{LTXexample}[width=6.5cm]
\begin{pspicture}(6,2)
\psset{arrowscale=2,ArrowFill=true,ArrowInside=-*}
\psline(0,0)(2,1)(3,0)(4,0)(6,2)
\psset{linestyle=none}
\psline[ArrowInsidePos=0](0,0)(2,1)(3,0)(4,0)(6,2)
\psline[ArrowInsidePos=1](0,0)(2,1)(3,0)(4,0)(6,2)
\end{pspicture}
\end{LTXexample}

\begin{LTXexample}[width=6.5cm]
\begin{pspicture}(6,5)
\psset{arrowscale=2,ArrowFill=true}
\psline[ArrowInside=->,ArrowInsidePos=20](0,0)(3,0)%
       (3,3)(1,3)(1,5)(5,5)(5,0)(7,0)(6,3)
\end{pspicture}
\end{LTXexample}

\begin{LTXexample}[width=6.5cm]
\begin{pspicture}(6,2)
\psset{arrowscale=2,ArrowFill=true}
\psline[ArrowInside=-|]{<->}(0,2)(2,0)(3,2)(4,0)(6,2)
\end{pspicture}
\end{LTXexample}

%--------------------------------------------------------------------------------------
\subsubsection{\nxLcs{pspolygon}}
%--------------------------------------------------------------------------------------
% Polygons (\pspolygon macro)

\begin{LTXexample}[width=6.5cm]
\begin{pspicture}(6,3)
\psset{arrowscale=2}
\pspolygon[ArrowInside=-|](0,0)(3,3)(6,3)(6,1)
\end{pspicture}
\end{LTXexample}

\begin{LTXexample}[width=6.5cm]
\begin{pspicture}(6,3)
\psset{arrowscale=2}
\pspolygon[ArrowInside=->,ArrowInsidePos=0.25]%
     (0,0)(3,3)(6,3)(6,1)
\end{pspicture}
\end{LTXexample}

\begin{LTXexample}[width=6.5cm]
\begin{pspicture}(6,3)
\psset{arrowscale=2}
\pspolygon[ArrowInside=->,ArrowInsideNo=4]%
       (0,0)(3,3)(6,3)(6,1)
\end{pspicture}
\end{LTXexample}

\begin{LTXexample}[width=6.5cm]
\begin{pspicture}(6,3)
\psset{arrowscale=2}
\pspolygon[ArrowInside=->,ArrowInsideNo=4,%
   ArrowInsideOffset=0.1](0,0)(3,3)(6,3)(6,1)
\end{pspicture}
\end{LTXexample}

\begin{LTXexample}[width=6.5cm]
\begin{pspicture}(6,3)
\psset{arrowscale=2}
 \pspolygon[ArrowInside=-|](0,0)(3,3)(6,3)(6,1)
 \psset{linestyle=none,ArrowInside=-*}
 \pspolygon[ArrowInsidePos=0](0,0)(3,3)(6,3)(6,1)
 \pspolygon[ArrowInsidePos=1](0,0)(3,3)(6,3)(6,1)
 \psset{ArrowInside=-o}
 \pspolygon[ArrowInsidePos=0.25](0,0)(3,3)(6,3)(6,1)
 \pspolygon[ArrowInsidePos=0.75](0,0)(3,3)(6,3)(6,1)
\end{pspicture}
\end{LTXexample}

\psset{linestyle=solid}

\begin{LTXexample}[width=6.5cm]
\begin{pspicture}(6,5)
\psset{arrowscale=2}
  \pspolygon[ArrowInside=->,ArrowInsidePos=20]%
    (0,0)(3,0)(3,3)(1,3)(1,5)(5,5)(5,0)(7,0)(6,3)
\end{pspicture}
\end{LTXexample}


%--------------------------------------------------------------------------------------
\subsubsection{\nxLcs{psbezier}}
%--------------------------------------------------------------------------------------
% Bezier curves (\psbezier macro)


\begin{LTXexample}[width=3.5cm]
\begin{pspicture}(3,3)
\psset{arrowscale=2}
  \psbezier[ArrowInside=-|](0,1)(1,0)(2,1)(3,3)
  \psset{linestyle=none,ArrowInside=-o}
  \psbezier[ArrowInsidePos=0.25](0,1)(1,0)(2,1)(3,3)
  \psbezier[ArrowInsidePos=0.75](0,1)(1,0)(2,1)(3,3)
  \psset{linestyle=none,ArrowInside=-*}
  \psbezier[ArrowInsidePos=0](0,1)(1,0)(2,1)(3,3)
  \psbezier[ArrowInsidePos=1](0,1)(1,0)(2,1)(3,3)
\end{pspicture}
\end{LTXexample}



\resetOptions
\begin{LTXexample}[width=4.5cm]
\begin{pspicture}(4,3)
\psset{arrowscale=2}
\psbezier[ArrowInside=->,showpoints]%
  {*-*}(0,0)(2,3)(3,0)(4,2)
\end{pspicture}
\end{LTXexample}




\begin{LTXexample}[width=4.5cm]
\begin{pspicture}(4,3)
\psset{arrowscale=2}
  \psbezier[ArrowInside=->,showpoints=true,
      ArrowInsideNo=2](0,0)(2,3)(3,0)(4,2)
\end{pspicture}
\end{LTXexample}


\begin{LTXexample}[width=4.5cm]
\begin{pspicture}(4,3)
\psset{arrowscale=2}
  \psbezier[ArrowInside=->,showpoints=true,
     ArrowInsideNo=2,ArrowInsideOffset=-0.2]%
      {->}(0,0)(2,3)(3,0)(4,2)
\end{pspicture}
\end{LTXexample}


\begin{LTXexample}[width=5.5cm]
\begin{pspicture}(5,3)
\psset{arrowscale=2}
  \psbezier[ArrowInsideNo=9,ArrowInside=-|,%
    showpoints=true]{*-*}(0,0)(1,3)(3,0)(5,3)
\end{pspicture}
\end{LTXexample}

\begin{LTXexample}[width=4.5cm]
\begin{pspicture}(4,3)
\psset{arrowscale=2}
  \psset{ArrowInside=-|}
  \psbezier[ArrowInsidePos=0.25,showpoints=true]{*-*}(2,3)(3,0)(4,2)
  \psset{linestyle=none}
  \psbezier[ArrowInsidePos=0.75](0,0)(2,3)(3,0)(4,2)
\end{pspicture}
\end{LTXexample}

\begin{LTXexample}[width=5.5cm]
\begin{pspicture}(5,6)
\psset{arrowscale=2}
  \pnode(3,4){A}\pnode(5,6){B}\pnode(5,0){C}
  \psbezier[ArrowInside=->,%
     showpoints=true](A)(B)(C)
  \psset{linestyle=none,ArrowInside=-<}
  \psbezier[ArrowInsideNo=4](0,0)(A)(B)(C)
  \psset{ArrowInside=-o}
  \psbezier[ArrowInsidePos=0.1](0,0)(A)(B)(C)
  \psbezier[ArrowInsidePos=0.9](0,0)(A)(B)(C)
  \psset{ArrowInside=-*}
  \psbezier[ArrowInsidePos=0.3](0,0)(A)(B)(C)
  \psbezier[ArrowInsidePos=0.7](0,0)(A)(B)(C)
\end{pspicture}
\end{LTXexample}

\psset{linestyle=solid}

\begin{LTXexample}[pos=t]
\begin{pspicture}(-3,-5)(15,5)
  \psbezier[ArrowInsideNo=19,%
      ArrowInside=->,ArrowFill=false,%
      showpoints=true]{->}(-3,0)(5,-5)(8,5)(15,-5)
\end{pspicture}
\end{LTXexample}



%--------------------------------------------------------------------------------------
\subsubsection{\nxLcs{pcline}}
%--------------------------------------------------------------------------------------
These examples need the package \verb|pst-node|.

% Lines (\pcline macro)
\begin{LTXexample}[width=2.5cm]
\begin{pspicture}(2,1)
\psset{arrowscale=2}
\pcline[ArrowInside=->](0,0)(2,1)
\end{pspicture}
\end{LTXexample}


\begin{LTXexample}[width=2.5cm]
\begin{pspicture}(2,1)
\psset{arrowscale=2}
\pcline[ArrowInside=->]{<->}(0,0)(2,1)
\end{pspicture}
\end{LTXexample}


\begin{LTXexample}[width=2.5cm]
\begin{pspicture}(2,1)
\psset{arrowscale=2}
\pcline[ArrowInside=-|,ArrowInsidePos=0.75]{|-|}(0,0)(2,1)
\end{pspicture}
\end{LTXexample}


\begin{LTXexample}[width=2.5cm]
\psset{arrowscale=2}
\pcline[ArrowInside=->,ArrowInsidePos=0.65]{*-*}(0,0)(2,0)
\naput[labelsep=0.3]{\large$g$}
\end{LTXexample}


\begin{LTXexample}[width=2.5cm]
\psset{arrowscale=2}
\pcline[ArrowInside=->,ArrowInsidePos=10]{|-|}(0,0)(2,0)
\naput[labelsep=0.3]{\large$l$}
\end{LTXexample}



%--------------------------------------------------------------------------------------
\subsubsection{\nxLcs{pccurve}}
%--------------------------------------------------------------------------------------
These examples also need the package \verb|pst-node|.

\begin{LTXexample}[width=2.5cm]
\begin{pspicture}(2,2)
\psset{arrowscale=2}
\pccurve[ArrowInside=->,ArrowInsidePos=0.65,showpoints=true]{*-*}(0,0)(2,2)
\naput[labelsep=0.3]{\large$h$}
\end{pspicture}
\end{LTXexample}


\begin{LTXexample}[width=2.5cm]
\begin{pspicture}(2,2)
\psset{arrowscale=2}
\pccurve[ArrowInside=->,ArrowInsideNo=3,showpoints=true]{|->}(0,0)(2,2)
\naput[labelsep=0.3]{\large$i$}
\end{pspicture}
\end{LTXexample}


\begin{LTXexample}[width=4.5cm]
\begin{pspicture}(4,4)
\psset{arrowscale=2}
\pccurve[ArrowInside=->,ArrowInsidePos=20]{|-|}(0,0)(4,4)
\naput[labelsep=0.3]{\large$k$}
\end{pspicture}
\end{LTXexample}

\clearpage

\subsection{Special arrows \texttt{v--V},\texttt{t--T}, and \texttt{f--F}}

Possible optional arguments are

\psset{linecolor=black}

\begin{center}
\begin{tabular}{@{}l|l@{}}\toprule
\emph{name} & \emph{meaning}\\\hline
\Lkeyword{veearrowlength} & default is 3mm\\
\Lkeyword{veearrowangle} & default is 30\\
\Lkeyword{veearrowlinewidth} & default is 0.35mm\\
\Lkeyword{filledveearrowlength} & default is 3mm\\
\Lkeyword{filledveearrowangle} & default is 15\\
\Lkeyword{filledveearrowlinewidth} & default is 0.35mm\\
\Lkeyword{tickarrowlength} & default is 1.5mm\\
\Lkeyword{tickarrowlinewidth} & default is 0.35mm\\
\Lkeyword{arrowlinestyle}     & default is solid\\\bottomrule
\end{tabular}
\end{center}


\begin{LTXexample}[width=4cm]
\psset{unit=5mm}
\begin{pspicture}(4,6)
  \psset{dimen=middle,arrows=c-c,
    arrowscale=2,linewidth=.25mm}
  \psline[linecolor=red,linewidth=.05mm](0,0)(0,6)
  \psline[linecolor=red,linewidth=.05mm](4,0)(4,6)
  \psline{v-v}(0,6)(4,6)
  \psline{v-V}(0,4)(4,4)
  \psline{V-v}(0,2)(4,2)
  \psline{V-V}(0,0)(4,0)
\end{pspicture}
\end{LTXexample}


\begin{LTXexample}[width=4cm]
\psset{unit=5mm}
\begin{pspicture}(4,6)
  \psset{dimen=middle,arrows=c-c,
    arrowscale=2,linewidth=.25mm}
  \psline[linecolor=red,linewidth=.05mm](0,0)(0,6)
  \psline[linecolor=red,linewidth=.05mm](4,0)(4,6)
  \psline{f-f}(0,6)(4,6)
  \psline{f-F}(0,4)(4,4)
  \psline{F-f}(0,2)(4,2)
  \psline{F-F}(0,0)(4,0)
\end{pspicture}
\end{LTXexample}


\begin{LTXexample}[width=4cm]
\psset{unit=5mm}
\begin{pspicture}(4,6)
  \psset{dimen=middle,arrows=c-c,linewidth=.25mm}
  \psline[linecolor=red,linewidth=.05mm](0,0)(0,6)
  \psline[linecolor=red,linewidth=.05mm](4,0)(4,6)
  \psline{t-t}(0,6)(4,6)
  \psline{t-T}(0,4)(4,4)
  \psline{T-t}(0,2)(4,2)
  \psline{T-T}(0,0)(4,0)
\end{pspicture}
\end{LTXexample}

\begin{LTXexample}[pos=t,vsep=5mm]
\psset{unit=5mm}
 \begin{pspicture}(10,6)
 \psset{dimen=middle,arrows=c-c,arrowscale=2,linewidth=.25mm,
        arrowlinestyle=dashed,dash=1.5pt 1pt}
 \psline[linecolor=red,linewidth=.05mm](0,0)(0,6)
 \psline[linecolor=red,linewidth=.05mm](4,0)(4,6)
 \psline{v-v}(0,6)(4,6) \psline{v-V}(0,4)(4,4)
 \psline{V-v}(0,2)(4,2) \psline{V-V}(0,0)(4,0)
 \psline[linecolor=red,linewidth=.05mm](6,0)(6,6)
 \psline[linecolor=red,linewidth=.05mm](10,0)(10,6)
 \psset{arrowlinestyle=dotted,dotsep=0.8pt}
 \psline{v-v}(6,6)(10,6) \psline{v-V}(6,4)(10,4)
 \psline{V-v}(6,2)(10,2) \psline{V-V}(6,0)(10,0)
\end{pspicture}
\end{LTXexample}

\begin{LTXexample}[pos=t,vsep=5mm]
\psset{unit=5mm}
 \begin{pspicture}(10,7)
 \psset{dimen=middle,arrows=c-c,arrowscale=2,linewidth=.25mm,
        arrowlinestyle=dashed,dash=1.5pt 1pt}
 \psline[linecolor=red,linewidth=.05mm](0,0)(0,6)
 \psline[linecolor=red,linewidth=.05mm](4,0)(4,6)
 \psline{t-t}(0,6)(4,6) \psline{t-T}(0,4)(4,4)
 \psline{T-t}(0,2)(4,2) \psline{T-T}(0,0)(4,0)
 \psline[linecolor=red,linewidth=.05mm](6,0)(6,6)
 \psline[linecolor=red,linewidth=.05mm](10,0)(10,6)
 \psset{arrowlinestyle=dotted,dotsep=0.8pt}
 \psline{t-t}(6,6)(10,6) \psline{t-T}(6,4)(10,4)
 \psline{T-t}(6,2)(10,2) \psline{T-T}(6,0)(10,0)
\end{pspicture}
\end{LTXexample}




\subsection{Special arrow option \texttt{arrowLW}}

Only for the arrowtype \Lnotation{o} and \Lnotation{*} it is possible to
set the arrowlinewidth with the optional keyword \Lkeyword{arrowLW}.
When scaling an arrow by the keyword \Lkeyword{arrowscale} the width
of the borderline is also scaled. With the optional argument
\Lkeyword{arrowLW} the line width can be set separately and is not
taken into account by the scaling value.

\begin{LTXexample}[width=4cm]
\begin{pspicture}(4,6)
\psline[arrowscale=3,arrows=*-o](0,5)(4,5)
\psline[arrowscale=3,arrows=*-o,
  arrowLW=0.5pt](0,3)(4,3)
\psline[arrowscale=3,arrows=*-o,
  arrowLW=0.3333\pslinewidth](0,1)(4,1)
\end{pspicture}
\end{LTXexample}



\section{Ticks and other marks along a curve}
\subsection{Quick overview}

The macros described below allow you to place tick and other marks along an arbitrary 
parametric curve with placement rules similar to those used by \Lcs{psaxes} in 
the \LPack{pst-plot} package. You have to define a metric function along the curve to 
govern tick placement. That function can be a specified function of {\tt x,y} which 
should increase along the curve, or it can be an function whose increment is a specified 
positive function of {\tt x, y, dx, dy, ds} where the last term is the arc-length element 
that you could specify alternately as {\tt dx dup mul dy dup mul add sqrt}.
% start new material


In addition, a new command \Lcs{Put} is proposed, expanding as appropriate to \Lcs{rput} or \Lcs{uput}. Its syntax is

\begin{BDef}
\LcsStar{Put}\OptArgs\OptArg*{\Largb{<ref>}}\Largr{<position>}\Largb{<stuff>}
\end{BDef}

where the optional {\tt *} blanks the background, the optional \OptArgs\ may be used to specify a rotation 
using any form acceptable to \Lcs{SpecialCoor} (eg, \nxLkeyword{rot=45} or \Lkeyword{rot}\verb|={(1,1)}| 
or \Lkeyword{rot}\verb|=(P)|, and \Larg{ref} takes one of 
two forms: \verb=(a)= a refpt such as {\tt Bl}, in which case \Lcs{rput} is called; (b) a polar form of offset 
(eg, \verb=7pt;30=, or \verb=;(P)= --- in the latter case, \Ldim{pslabelsep} is substituted for the missing 
radius), in which case a modified form of \Lcs{uput} is called. The idea of \Lcs{Put} is to allow  {\tt position}, 
{\tt ref} and {\tt rot} to be specified in any of the forms acceptable to \Lcs{SpecialCoor} and to do so with 
the same output no matter what form is used. The cost of this consistency is that \Lcs{Put} can lead to results 
that differ from \Lcs{uput} in some special cases. 


\subsection{Details}
Suppose you have drawn a parametric curve using \Lcs{psparametricplot}, and you wish to 
indicate some points on the curve using tick-marks like those  on the axes. This is a 
two-step process, the first of which serves to define at the PostScript level a 
number of data arrays containing information about the curve. Those arrays are used 
in the second step to compute tick positions and draw the ticks. The first step is 
to run the macro \Lcs{pscurvepoints}. For example,

\begin{verbatim}
\pscurvepoints[plotpoints=20]{0}{6}{t t t mul 12 div}{Pt}%
\end{verbatim}
makes a virtual (ie, data only---nothing is rendered) polyline with 20 vertices approximating 
the curve $x(t)=t, y(t)=t^2/12$, $0\le t\le 6$. The last argument {\tt Pt} is the root name 
given to the data arrays.  PostScript arrays will be created with the following names: {\tt Pt.X, Pt.Y} 
for the coordinates of the vertices, {\tt PtDelta.X, PtDelta.Y} for the increments between the 
vertices (using, eg, {\tt PtDelta.X[2]=Pt.X[2]-Pt.X[1]}) and {\tt PtNormal.X, PtNormal.Y} for 
a vector normal to {\tt PtDelta.X, PtDelta.Y} in the visual, not mathematical, sense. 
(Both senses are the same if the scales on the axes are identical.) The {\tt Normal} is 
always constructed so as to point ``upward'' (ie, to your left) as you traverse the curve 
in the positive direction. The PostScript variable {\tt unitratio} provides the ratio of 
the unit on the y axis to that on x axis, and {\tt unitratiosq} is its square. All of 
these PostScript objects are stored in the main {\tt pstricks} dictionary \Lps{tx@Dict} 
which should be automatically made available when using many {\tt pstricks} macros. 
If {\tt gs} returns you an error message like
\begin{verbatim}
Error: /undefined in Pt.X
\end{verbatim}
then you may need to enclose the offending PostScript code within a block of the form
\begin{verbatim}
tx@Dict begin ... end
\end{verbatim}
so that the dictionary is made available.

With this preparation, the main tick-making macro may be run. For example,
\begin{verbatim}
\pspolylineticks{Pt}{ dx dy add 3 div }{1}{2}%
\end{verbatim}
looks for data arrays made using \Lcs{pscurvepoints} with the root name {\tt Pt}. The next argument, 
{\tt dx dy add 3 div}, specifies the (PostScript) function of increments that should be used to 
construct the metric. If the keyword \verb|metricInitValue| is defined, eg, with 
\Lcs{psset}\Largb{\Lkeyword{metricInitValue}=2.5}, it is used as the initial value of the metric, 
otherwise it is defined to be 0. In the previous example, the increment function is always 
positive, and care should be taken to guarantee this is so or the results will not be meaningful. 
(If we wanted to use arc-length, the function would have been {\tt ds}, assuming equal scales on 
the axes.) The last two arguments determine the index of the first tick and the number of ticks. 
Tick numbering begins with index 0, so the example says to drop the first tick and draw the 
next 2 ticks. In this example, where all keywords take their default values, ticks are 
potentially located at values on the curve where the metric takes a positive integer value. 
In the arc-length example, the tick with index 0 is at the beginning of the curve, and subsequent 
ticks are at unit distance, measured along the curve. At each index where a tick is drawn, a 
\Lcs{pnode} is created: In this example, you create nodes {\tt PtTick1, PtTick2} on the curve 
where the ticks are located. This is handy for placing labels using, eg, \Lcs{Put}. In 
addition, PostScript data arrays (in this example, {\tt PtTickN.X, PtTickN.Y} of the normals 
at these nodes are stored in the dictionary {\tt TDict}. More importantly, the tangent and 
normal vectors at {\tt PtTick0} etc are constructed as nodes with names {\tt PtTangent0, PtNormal0} 
etc. See the last example below for typical usage.

The shape of the ticks is governed by the keywords \Lkeyword{ticksize} (default value {\tt -4pt 4pt})
 and \Lkeyword{tickwidth} (default value \verb|.5\linewidth|.) With the default settings, ticks 
 are drawn perpendicular the the curve extending {\tt 4pt} to each side. The line
\begin{verbatim}
\pspolylineticks[ticksize=-6pt 6pt]{Pt}{ dx dy add 3 div }{1}{2}%
\end{verbatim}
would draw longer ticks than the default.

Placement of the ticks is governed by the keywords \Lkeyword{Ds} and \Lkeyword{Os}, whose meaning for the 
curve is similar to (but not the same as) the meanings of \Lkeyword{Dx} and \Lkeyword{Ox} with respect to the x axis. 
That is, if {\tt Ds=2} and {\tt Os=0}, ticks will be drawn where the metric takes 
values 0, 2, 4 and so on. More generally, ticks are placed where the metric takes 
value {\tt Os, Os+Ds, Os+2*Ds,...}, as long as those positions are on the curve. If \Lkeyword{Os} 
has an empty value as a result, say, of \verb|\psset{Os=}|, then \Lkeyword{Os} is set internally 
to the initial metric value. If \Lkeyword{Ds} has an empty value, it is set internally to the 
final metric value less the initial metric value, divided by 10. 

To draw major and minor ticks requires two passes---one to draw the minor ticks and then one to draw the major ticks.

Note that a ticks may be placed at arbitrary metric values on the curve by running the macro once for each point, like:
\begin{verbatim}
\pspolylineticks[ticksize=-6pt 6pt,Os=1.3]{Pt}{ dx dy add 3 div }{0}{1}%
\pspolylineticks[ticksize=-6pt 6pt,Os=2.4]{Pt}{ dx dy add 3 div }{0}{1}%
\end{verbatim}

You may also dispense entirely with the tick and use the macro to generate a node sequence 
that can be used to place other graphic objects. For example:
\begin{verbatim}
\pspolylineticks[ticksize=0pt 0pt]{Pt}{ dx dy add 3 div }{0}{3}%
%This defines nodes PtTick0..PtTick2
\multido{\iA=0+1}{3}{\psdot(PtTick\iA)}
\end{verbatim}


There is another way to define a metric function without using increments. If the keywork \Lkeyword{metricFunction} is set to \true, 
then the function you present as an argument to \Lcs{pspolylineticks} must be a function of 
$x$ and $y$ only, and must be designed to increase along the curve. It is useful only in 
those cases where, in essence, the increment function can be explicitly integrated. 
For example, in the elliptical motion of planets and comets around the sun, it is not hard 
to integrate the area function explicitly, and this provides a convenient metric, being proportional to time elapsed.

There is some useful information left in the log by these macros. 
They report the starting and ending values of the metric function, 
the the range of indices for the Tick related arrays.

\subsection{Examples}
The examples in this section make use of very recent (as of May, 2010) versions 
of \LPack{pstricks} and related packages. 
%If the {\tt pst-grapha} package is not available on CTAN,  download it from
%\begin{verbatim}
%http://math.ucsd.edu/~msharpe/pst-grapha.dmg
%\end{verbatim}

The first couple of examples are constructed entirely by hand, and have no interest 
other than to illustrate what is going on under the surface in the simplest case.

\begin{LTXexample}[pos=t]
\begin{pspicture}(-1,-1)(10,4)
\psline[showpoints=true](1,2)(4,0)(9,3)%
\uput[180](1,2){$s=0$}%
\uput[-90](4,0){$s=1$}%
\uput[0](9,3){$s=2$}%
\makeatletter% need to use macro names containing @
\pstVerb{tx@Dict begin %the pstricks dictionary
% declare arrays of length 3 (indices 0,1,2) to hold points, 
% differences and normals
/unitratiosq 1 def % yunit=xunit
/P.X [ 1 4 9 ] def %array of x coords
/P.Y [ 2 0 3 ] def %array of y coords
/PDelta.X [ 0 3 5 ] def % 3=4-1, 5=9-4, 0 never used
/PDelta.Y [ 0 -2 3 ] def % -2=0-2, 3=3-0, 0 never used
% normal to (3,-2) is (2,3), normal to (5,3) is (-3,5)
/PNormal.X [ 2 2 -3 ] def % index 0 =index 1
/PNormal.Y [ 3 3 5 ] def % index 0 = index 1
end }
\def\Ppointcount{2}
\makeatother
% make ticks using metric function with values 0,1,2
\pspolylineticks[Os=.5,Ds=1]{P}{1}{0}{2}
% ticks at s=0.5,1.5 (increment function =1)
\uput[-135](PTick0){$s=0.5$}%
\uput[-45](PTick1){$s=1.5$}%
\end{pspicture}
\end{LTXexample}

\clearpage
Now the same data, but with arc-length as metric. We change the last few lines:

\begin{LTXexample}[pos=t]
\begin{pspicture}(-1,-1)(10,4)
\psline[showpoints=true](1,2)(4,0)(9,3)%
%\uput[180](1,2){$s=0$}%
%\uput[-90](4,0){$s=1$}%
%\uput[0](9,3){$s=2$}%
\makeatletter% need to use macro names containing @
\pstVerb{tx@Dict begin %the pstricks dictionary
% declare arrays of length 3 (indices 0,1,2) to hold points, 
% differences and normals
/unitratiosq 1 def % yunit=xunit
/P.X [ 1 4 9 ] def %array of x coords
/P.Y [ 2 0 3 ] def %array of y coords
/PDelta.X [ 0 3 5 ] def % 3=4-1, 5=9-4, 0 never used
/PDelta.Y [ 0 -2 3 ] def % -2=0-2, 3=3-0, 0 never used
% normal to (3,-2) is (2,3), normal to (5,3) is (-3,5)
/PNormal.X [ 2 2 -3 ] def % index 0 =index 1
/PNormal.Y [ 3 3 5 ] def % index 0 = index 1
end }
\def\Ppointcount{2}
\makeatother
% make ticks using metric function arc-length
\pspolylineticks[Os=1,Ds=1]{P}{ ds }{0}{9}
% ticks at s=1,2... (increment function = distance)
\uput[-135](PTick0){$s=1$}%
\uput[-135](PTick1){$s=2$}%
\end{pspicture}
\end{LTXexample}

\clearpage
Once again the same data, but with metric equal to the x coordinate. Change the last few lines to:

\begin{LTXexample}[pos=t]
\begin{pspicture}(-1,-1)(10,4)
\psline[showpoints=true](1,2)(4,0)(9,3)%
%\uput[180](1,2){$s=0$}%
%\uput[-90](4,0){$s=1$}%
%\uput[0](9,3){$s=2$}%
\makeatletter% need to use macro names containing @
\pstVerb{tx@Dict begin %the pstricks dictionary
% declare arrays of length 3 (indices 0,1,2) to hold points, 
% differences and normals
/unitratiosq 1 def % yunit=xunit
/P.X [ 1 4 9 ] def %array of x coords
/P.Y [ 2 0 3 ] def %array of y coords
/PDelta.X [ 0 3 5 ] def % 3=4-1, 5=9-4, 0 never used
/PDelta.Y [ 0 -2 3 ] def % -2=0-2, 3=3-0, 0 never used
% normal to (3,-2) is (2,3), normal to (5,3) is (-3,5)
/PNormal.X [ 2 2 -3 ] def % index 0 =index 1
/PNormal.Y [ 3 3 5 ] def % index 0 = index 1
end }
\def\Ppointcount{2}
\makeatother
% make ticks using metric function arc-length
\pspolylineticks[metricFunction,Os=1,Ds=2]{P}{ x }{0}{5}
% ticks at x=1,3,... , start at tick index 0, draw 5 ticks
% the tick at s=1 has index 0
% ticks at s=1,2... (increment function = distance)
\uput[-135](PTick0){$s=1$}%
\uput[-135](PTick1){$s=3$}%
\end{pspicture}
\end{LTXexample}

\clearpage
The next example is a smooth path where subticks are drawn first, followed by major ticks. 
The metric is arc-length with initial value $s=1$.
\begin{LTXexample}[pos=t]
\begin{pspicture}(-1,-1)(10,4)
%\parametricplot[algebraic]{0}{9}{(t^2)/9 | 4*Ex(-t)*(1+t+(t^{2})/2+(t^{3})/6)}
\psparametricplot[algebraic]{0}{9}{t^2/9 | sin(t)+1}%
\pscurvepoints{0}{9}{(t^2)/9 | sin(t)+1}{P}%
% make ticks using  arc-length metric
\pspolylineticks[metricInitValue=1,ticksize=-2pt 2pt,Os=1,Ds=.2]{P}{ ds }{1}{56}%
\pspolylineticks[metricInitValue=1,Os=1,Ds=2]{P}{ ds }{0}{6}%
\multido{\iA=1+1,\iB=3+2}{5}{\Put{6pt;(PNormal\iA)}(PTick\iA){\tiny \iB}}
%\nodexn{(PTick\iA)+(10pt;{(PNormal\iA)})}{Q}\rput(Q){\tiny \iB}}%
%\multido{\iA=1+1,\iB=3+2}{5}{\uput{6pt}[{(PNormal\iA)}](PTick\iA){\iB}}%
% ticks at x=1,3,... , start at tick index 0, draw 5 ticks
% the tick at s=1 has index 0
% ticks at s=1,2... (increment function = distance)
\end{pspicture}
\end{LTXexample}

\clearpage
Suppose for the next example that we have an ellipse $x^2/a^2+y^2/b^2=1$ ($a>b$) with 
eccentricity $\epsilon=(1-b^2/a^2)^{1/2}$. With planetary motion in mind, a natural metric 
for the ellipse is the area swept out by the radial line from the focus $(\epsilon a,0)$ 
starting from $(a,0)$ around to an arbitrary location $(x,y)$, where $y>0$, as this quantity 
is proportional to the time elapsed since perihelion. A routine calculation gives the following formula:
\[A=\frac{ab}{2}\arccos\bigg(\frac{x}{a}\bigg)-\frac{\epsilon a y}{2}.\]
Remembering that PostScript's {\tt acos} gives  its result in degrees, not radians, we have the 
following, drawn for the case $a=4$, $b=3$.

\begin{LTXexample}[pos=t]
\begin{pspicture}(-4.5,-.5)(4.5,3.5)
\pstVerb{ /smajor 4 def /sminor 3 def % define semimajor, semiminor 
/ecc 1 sminor smajor div dup mul sub sqrt def % compute eccentricity
/ab smajor sminor mul 2 div def %first coeff
/ea smajor ecc mul 2 div def }% second coeff
\psparametricplot[algebraic]{0}{3.142}{smajor*cos(t) | sminor*sin(t)}%
\pscurvepoints{0}{3.142}{smajor*cos(t) | sminor*sin(t)}{P}%
\pspolylineticks[metricFunction,Ds=2,ticksize=-1.5pt 0]{P}{ ab x smajor div acos %
180 div PI mul mul  ea y mul sub }{1}{9}%
\pnode(! ecc smajor mul 0){S}% focus
\psline[linecolor=lightgray](S)(!smajor 0)%
\multido{\i=1+1}{9}{\psline[linecolor=lightgray](S)(PTick\i)}
\psdot(S)
\end{pspicture}
\end{LTXexample}

\clearpage
The next examples works without visible ticks, using the macros to construct nodes at which other objects will be placed.

\begin{LTXexample}[pos=t]
\begin{pspicture}(-1,-1)(10,4)
\psparametricplot[algebraic]{0}{9}{t| 3*Ex(-t)*(1+t+t^2/2+t^3/6)}
\pscurvepoints{0}{9}{t| 3*Ex(-t)*(1+t+t^2/2+t^3/6)}{P}%
\pspolylineticks[Os=1,Ds=1,ticksize=0 0]{P}{ ds }{0}{9}%
\multido{\i=0+1}{9}{\psdot[dotscale=1.5,dotstyle=o](PTick\i)}%
% ticks at s=1,2,... , start at tick index 0, set 9 ticks
% the tick at s=1 has index 0
% ticks at s=1,2... (increment function = distance)
\multido{\i=0+3}{3}{\Put[rot=(PTangent\i)]{7pt;(PNormal\i)}(PTick\i){PTick\i}}%
\uput[-135](PTick1){$s=2$}%
\end{pspicture}
\end{LTXexample}

This variant also has no visible ticks, but makes a color gradient along the curve based on arc-length from the start.

\begin{LTXexample}[pos=t]
\begin{pspicture}(-1,-1)(10,4)
\psparametricplot[plotpoints=200,linecolor=white]{0}{360}{ t cos 1 add 4 mul t 1 add 20 div ln 2 div 1 add }
\pscurvepoints[plotpoints=200]{0}{360}{ t cos 1 add 4 mul t 1 add 20 div ln 2 div 1 add }{P}%
\pspolylineticks[Os=0,Ds=.2,ticksize=0 0]{P}{ ds }{0}{90}%
\definecolorseries{ctest}{hsb}{last}{green}{violet}
\resetcolorseries[88]{ctest}%
\multido{\iA=0+1,\iB=1+1}{87}{\psline[linewidth=2pt,linecolor=ctest!![\iB](PTick\iA)(PTick\iB)}%
\end{pspicture}
\end{LTXexample}

\clearpage
Here is a another variant of this technique which allows arrows to be placed at locations 
on the curve where the metric takes particular values.

\begin{LTXexample}[pos=t]
\begin{pspicture}(-1,-1)(10,4.5)
\psparametricplot[plotpoints=100]{0}{360}{t cos 1 add 5 mul t sin 1 add 2 mul}
\pscurvepoints[plotpoints=100]{0}{360}{t cos 1 add 5 mul t sin 1 add 2 mul}{P}%
\pspolylineticks[Os=0,Ds=2.3,ticksize=0 0]{P}%
{ ds }{0}{10}% distance
\multido{\i=0+1}{10}{\psrline[arrows=->,arrowscale=1.5](PTick\i)(2pt;{(PTangent\i)})}%
\end{pspicture}
\end{LTXexample}

\section{Troubleshooting}
If you get PostScript errors when you process your file, the  most likely culprit is the 
function you specified to define the metric. There are some  things to look out for:
\begin{itemize}
\item If \Lkeyword{metricFunction}, the function you specify in PostScript code must 
involve only {\tt x} and {\tt y}, and must leave exactly one real value on the stack as a result of 
substituting specific values for {\tt x} and {\tt y}. The function must be strictly increasing on the curve.
\item If \Lkeyword{metricFunction}=\false (the default), the function you specify in PostScript 
code must involve only the variables {\tt x}, {\tt y}, {\tt dx}, {\tt dy}, {\tt ds} (where {\tt ds} 
is defined to be the arc-length element {\tt dx dup mul dy dup mul add sqrt}, and must leave exactly 
one strictly positive real value on the stack when specific values are substituted for those variables. 
The constant function {\tt 1} gives equal weight to each segment in the curve, so in effect it gives 
you the original parametrization, up to a constant factor.
\item If the function you specify in \Lcs{parametricplot} and \Lcs{pscurvepoints} is \Lkeyword{algebraic}, 
make sure you follow precisely the syntax it understands. In complex cases, PostScript may be the safer solution.
\item It is unwise to use a different resolution for \Lcs{psparametricplot} and \Lcs{pscurvepoints}. 
The default value of \Lkeyword{plotpoints}=50 is marginal except for modest curve segments, and 200 should 
suffice for most smooth curves.
\end{itemize}


%--------------------------------------------------------------------------------------
\section{Transparent colors}
%--------------------------------------------------------------------------------------

Transparency is now part of the main \LPack{pstricks} package.
But pay attention, the names and syntax have changed and you need
to run \Lprog{ps2pdf} with the option
\Loption{-dCompatibilityLevel}=1.4.


%--------------------------------------------------------------------------------------
\section{,,Manipulating transparent colors''}
%--------------------------------------------------------------------------------------

\LPack{pstricks-add} supports real transparency and a simulated one with hatch lines:
\begin{lstlisting}
\def\defineTColor{\@ifnextchar[{\defineTColor@i}{\defineTColor@i[]}}
\def\defineTColor@i[#1]#2#3{%     transparency "Colors"
  \newpsstyle{#2}{%
     fillstyle=vlines,hatchwidth=0.1\pslinewidth,
     hatchsep=1\pslinewidth,hatchcolor=#3,#1%
  }%
}
\defineTColor{TRed}{red}
\defineTColor{TGreen}{green}
\defineTColor{TBlue}{blue}
\end{lstlisting}

There are three predefined "'transparent"` colors \verb+TRed+,
\verb+TGreen+, \verb+TBlue+. They are used as \PST{} styles and
not as colors:

\bgroup
\begin{LTXexample}[pos=t,preset=\centering]
\begin{pspicture}(-3,-5)(5,5)
\psframe(-1,-3)(5,5) % objet de base
\psrotate(2,-2){15}{%
  \psframe[style=TRed](-1,-3)(5,5)}
\psrotate(2,-2){30}{%
  \psframe[style=TGreen](-1,-3)(5,5)}
\psrotate(2,-2){45}{%
  \psframe[style=TBlue](-1,-3)(5,5)}
\psframe[linewidth=3pt](-1,-3)(5,5)
\psdots[dotstyle=+,dotangle=45,dotscale=3](2,-2) % centre de la rotation
\end{pspicture}
\end{LTXexample}
\egroup

%--------------------------------------------------------------------------------------
\section{Calculated colors}
%--------------------------------------------------------------------------------------
The \verb+xcolor+ package (version 2.6) has a new feature for defining colors:
\begin{lstlisting}[style=syntax]
  \definecolor[ps]{<name>}{<model>}{< PS code >}
\end{lstlisting}

\verb+model+ can be one of the color models, which \PS will
understand, e.g. \verb+rgb+. With this definition the color is
calculated on the \PS side.
\begin{LTXexample}[pos=t,preset=\centering]
\definecolor[ps]{bl}{rgb}{tx@addDict begin  Red Green Blue end}%
\psset{unit=1bp}
\begin{pspicture}(0,-30)(400,100)
\multido{\iLAMBDA=0+1}{400}{%
  \pstVerb{
    \iLAMBDA\space 379 add dup /lambda exch def
    tx@addDict begin  wavelengthToRGB end
  }%
  \psline[linecolor=bl](\iLAMBDA,0)(\iLAMBDA,100)%
}
\psaxes[yAxis=false,Ox=350,dx=50bp,Dx=50]{->}(-29,-10)(420,100)
\uput[-90](420,-10){$\lambda$[\textsf{nm}]}
\end{pspicture}
\end{LTXexample}


\begin{center}
\newcommand{\Touch}{%
\psframe[linestyle=none,fillstyle=solid,fillcolor=bl,dimen=middle](0.1,0.75)}
\definecolor[ps]{bl}{rgb}{tx@addDict begin Red Green Blue end}%
% Echelle 1cm <-> 40 nm
%         1 nm <-> 0.025 cm
\psframebox[fillstyle=solid,fillcolor=black]{%
\begin{pspicture}(-1,-0.5)(12,1.5)
\multido{\iLAMBDA=380+2}{200}{%
  \pstVerb{
    /lambda \iLAMBDA\space def
    lambda
    tx@addDict begin  wavelengthToRGB end
  }%
 \rput(! lambda 0.025 mul 9.5 sub 0){\Touch}
}
\multido{\n=0+1,\iDiv=380+40}{11}{%
    \psline[linecolor=white](\n,0.1)(\n,-0.1)
    \uput[270](\n,0){\textbf{\white\iDiv}}}
    \psline[linecolor=white]{->}(11,0)
    \uput[270](11,0){\textbf{\white$\lambda$(nm)}}
\end{pspicture}}

\psframebox[fillstyle=solid,fillcolor=black]{%
\begin{pspicture}(-1,-0.5)(12,1)
  \pstVerb{
    /lambda 656 def
    lambda
    tx@addDict begin  wavelengthToRGB end
  }%
 \rput(! 656 0.025 mul 9.5 sub 0){\Touch}
  \pstVerb{
    /lambda 486 def
    lambda
    tx@addDict begin  wavelengthToRGB end
  }%
 \rput(! 486 0.025 mul 9.5 sub 0){\Touch}
   \pstVerb{
    /lambda 434 def
    lambda
    tx@addDict begin  wavelengthToRGB end
  }%
 \rput(! 434 0.025 mul 9.5 sub 0){\Touch}
  \pstVerb{
    /lambda 410 def
    lambda
    tx@addDict begin  wavelengthToRGB end
  }%
 \rput(! 410 0.025 mul 9.5 sub 0){\Touch}
\multido{\n=0+1,\iDiv=380+40}{11}{%
    \psline[linecolor=white](\n,0.1)(\n,-0.1)
    \uput[270](\n,0){\textbf{\white\iDiv}}}
    \psline[linecolor=white]{->}(11,0)
    \uput[270](11,0){\textbf{\white$\lambda$(nm)}}
\end{pspicture}}

\Index{Spectrum} of \Index{hydrogen} emission (Manuel Luque)
\end{center}

\begin{lstlisting}
\newcommand\Touch{%
\psframe[linestyle=none,fillstyle=solid,fillcolor=bl,dimen=middle](0.1,0.75)}
\definecolor[ps]{bl}{rgb}{tx@addDict begin Red Green Blue end}%
% Echelle 1cm <-> 40 nm
%         1 nm <-> 0.025 cm
\psframebox[fillstyle=solid,fillcolor=black]{%
\begin{pspicture}(-1,-0.5)(12,1.5)
\multido{\iLAMBDA=380+2}{200}{%
  \pstVerb{
    /lambda \iLAMBDA\space def
    lambda
    tx@addDict begin  wavelengthToRGB end
  }%
 \rput(! lambda 0.025 mul 9.5 sub 0){\Touch}
}
\multido{\n=0+1,\iDiv=380+40}{11}{%
    \psline[linecolor=white](\n,0.1)(\n,-0.1)
    \uput[270](\n,0){\textbf{\white\iDiv}}}
    \psline[linecolor=white]{->}(11,0)
    \uput[270](11,0){\textbf{\white$\lambda$(nm)}}
\end{pspicture}}

\psframebox[fillstyle=solid,fillcolor=black]{%
\begin{pspicture}(-1,-0.5)(12,1)
  \pstVerb{
    /lambda 656 def
    lambda
    tx@addDict begin  wavelengthToRGB end
  }%
 \rput(! 656 0.025 mul 9.5 sub 0){\Touch}
  \pstVerb{
    /lambda 486 def
    lambda
    tx@addDict begin  wavelengthToRGB end
  }%
 \rput(! 486 0.025 mul 9.5 sub 0){\Touch}
   \pstVerb{
    /lambda 434 def
    lambda
    tx@addDict begin  wavelengthToRGB end
  }%
 \rput(! 434 0.025 mul 9.5 sub 0){\Touch}
  \pstVerb{
    /lambda 410 def
    lambda
    tx@addDict begin  wavelengthToRGB end
  }%
 \rput(! 410 0.025 mul 9.5 sub 0){\Touch}
\multido{\n=0+1,\iDiv=380+40}{11}{%
    \psline[linecolor=white](\n,0.1)(\n,-0.1)
    \uput[270](\n,0){\textbf{\white\iDiv}}}
    \psline[linecolor=white]{->}(11,0)
    \uput[270](11,0){\textbf{\white$\lambda$(nm)}}
\end{pspicture}}

Spectrum of hydrogen emission (Manuel Luque)
\end{lstlisting}



%--------------------------------------------------------------------------------------
\section{Gouraud shading}
%--------------------------------------------------------------------------------------
\begin{quotation}
\Index{Gouraud} shading is a method used in computer graphics to simulate the differing effects of
light and colour across the surface of an object. In practice, Gouraud shading is used to
achieve smooth lighting on low-polygon surfaces without the heavy computational requirements
of calculating lighting for each pixel. The technique was first presented by Henri Gouraud in 1971.\\
~\hfill{\small \url{http://www.wikipedia.org}}
\end{quotation}

PostScript level 3 supports this kind of shading and it can only
be seen with Acroread 7 or later. The syntax is easy:

\begin{lstlisting}[style=syntax]
  \psGTriangle(x1,y1)(x2,y2)(x3,y3){color1}{color2}{color3}
\end{lstlisting}

\psset{unit=0.75cm}

\begin{LTXexample}[pos=t,preset=\centering]
\begin{pspicture}(0,-.25)(10,10)
  \psGTriangle(0,0)(5,10)(10,0){red}{green}{blue}
\end{pspicture}
\end{LTXexample}

\begin{LTXexample}[pos=t,preset=\centering]
\begin{pspicture}(0,-.25)(10,10)
  \psGTriangle*(0,0)(9,10)(10,3){black}{white!50}{red!50!green!95}
\end{pspicture}
\end{LTXexample}

\begin{LTXexample}[pos=t,preset=\centering]
\begin{pspicture}(0,-.25)(10,10)
  \psGTriangle*(0,0)(5,10)(10,0){-red!100!green!84!blue!86}
                               {-red!80!green!100!blue!40}
                               {-red!60!green!30!blue!100}
\end{pspicture}
\end{LTXexample}

\begin{LTXexample}[pos=t,preset=\centering]
\definecolor{rose}{rgb}{1.00, 0.84, 0.88}
\definecolor{vertpommepasmure}{rgb}{0.80, 1.0, 0.40}
\definecolor{fushia}{rgb}{0.60, 0.30, 1.0}
\begin{pspicture}(0,-.25)(10,10)
  \psGTriangle(0,0)(5,10)(10,0){rose}{vertpommepasmure}{fushia}
\end{pspicture}
\end{LTXexample}

\section{Internal color macros}
The internal macros \Lcs{pswavelengthToRGB} and \Lcs{pswavelengthToRGB} can be used for own purposed.
They are defines as follows:

\begin{lstlisting}
\def\pswavelengthToGRAY{ tx@addDict begin wavelengthToGRAY end }
\def\pswavelengthToRGB{ tx@addDict begin wavelengthToRGB Red Green Blue end }
\end{lstlisting}

both macros leave the value(s) on the stack which then can be used for further
manipulating or setting the color with \Lps{setgray} or \Lps{setrgbcolor}. 
For an example see Section~\ref{sec:psMatrix}.

\appendix


%--------------------------------------------------------------------------------------
\clearpage
\section{\nxLcs{resetOptions}}
%--------------------------------------------------------------------------------------

Sometimes it is difficult to know what options, which are changed
inside a long document, are different to the default ones. With
this macro all options belonging to \LPack{pst-plot} can be reset.
This refers to all options of the packages \LPack{pstricks},
\LPack{pst-plot} and \LPack{pst-node}.



%--------------------------------------------------------------------------------------
\section{PostScript}
%--------------------------------------------------------------------------------------

\Index{PostScript} uses the stack system and the LIFO system, "'Last In, First Out"`.

\newlength{\Li}\settowidth{\Li}{Function}
\begin{table}[htbp]
\caption{Some primitive PostScript macros}\label{tab:primpost}
\centering
\ttfamily
    \begin{tabular}{@{} l | r@{ $\rightarrow$ } l @{}}\hline
    \multirow{2}{\Li}{\normalfont\emph{Function}} & \multicolumn{2}{ c }{\normalfont\emph{Meaning}}\\
    &\normalfont\emph{on stack before} & \normalfont\emph{after}\\\hline
    \Lps{add} & $x\quad y$&$x+y$\\ 
    \Lps{sub} & $x\quad y$&$x-y$\\ 
    \Lps{mul} & $x\quad y$&$x\times y$\\ 
    \Lps{div} & $x\quad y$&$x\div y$\\ 
    \Lps{sqrt} & $x$&$\sqrt{x}$\\ 
    \Lps{abs} & $x$&$|x|$\\ 
    \Lps{neg} & $x$&$-x$\\ 
    \Lps{cos} & $x$&$\cos(x)$ ($x$ in degrees)\\ 
    \Lps{sin} & $x$&$\sin(x)$ ($x$  in degrees)\\ 
    \Lps{tan} & $x$&$\tan(x)$ ($x$  in degrees)\\ 
    \Lps{atan} & $y\quad x$&$\angle{(\vec{Ox};\vec{OM})}$ (in degrees of $M(x,y)$)\\ 
    \Lps{ln} & $x$&$\ln(x)$\\ 
    \Lps{log} & $x$&$\log(x)$\\ 
    \Lps{array} & $n$&\normalfont$v$ (of dimension $n$)\\ 
    \Lps{aload} & $v$&$x_1\quad x_2\quad \cdots\quad x_n\quad v$\\ 
    \Lps{astore} & $x_1\quad x_2\quad \cdots\quad x_n\quad v$ & $[v]$\\ 
    \Lps{pop} & $x$ & --\\ 
    \Lps{dup} & $x$ & $x\quad x$ \\\hline
%    \Lps{roll} & $x_1\quad x_2\quad \cdots\quad x_n\quad n p$ &\\\hline
  \end{tabular}
\end{table}


\clearpage
\section{List of all optional arguments for \texttt{pstricks-add}}

\xkvview{family=pstricks-add,columns={key,type,default}}





\nocite{*}
\bgroup
\RaggedRight
\bibliographystyle{plain}
\bibliography{pstricks-add-doc}
\egroup

\printindex




\end{document}


\usepackage[utf8]{inputenc}
\usepackage{pstricks-add}
\let\pstricksaddFV\fileversion
\usepackage{pst-eucl,pst-fun,pst-func,multirow}
\usepackage{pifont}
\let\belowcaptionskip\abovecaptionskip
%
\def\textat{\char064}%
\newdimen\fullWidth
\makeatletter
\renewcommand*\l@section{\@dottedtocline{1}{2em}{2.3em}}
\renewcommand*\l@subsection{\@dottedtocline{2}{3.8em}{3.2em}}
\renewcommand*\l@subsubsection{\@dottedtocline{3}{7.0em}{4.1em}}
\renewcommand*\l@paragraph{\@dottedtocline{4}{10em}{5em}}
\makeatother
\lstset{explpreset={pos=l,width=-99pt,overhang=0pt,hsep=\columnsep,vsep=\bigskipamount,rframe={}},
    escapechar=§}

\def\bgImage{\psset{unit=1.5}
\begin{pspicture}(-3,-3)(3,3)
\psChart[userColor={red!30,green!30,blue!40,gray,cyan!50,
    magenta!60,cyan},chartSep=30pt,shadow=true,shadowsize=5pt]{34.5,17.2,20.7,15.5,5.2,6.9}{6}{2}
\psset{nodesepA=5pt,nodesepB=-10pt}
\ncline{psChartO1}{psChart1}\nput{0}{psChartO1}{1000 (34.5\%)}
\ncline{psChartO2}{psChart2}\nput{150}{psChartO2}{500 (17.2\%)}
\ncline{psChartO3}{psChart3}\nput{-90}{psChartO3}{600 (20.7\%)}
\ncline{psChartO4}{psChart4}\nput{0}{psChartO4}{450 (15.5\%)}
\ncline{psChartO5}{psChart5}\nput{0}{psChartO5}{150 (5.2\%)}
\ncline{psChartO6}{psChart6}\nput{0}{psChartO6}{200 (6.9\%)}
\bfseries%
\rput(psChartI1){Taxes}\rput(psChartI2){Rent}\rput(psChartI3){Bills}
\rput(psChartI4){Car}\rput(psChartI5){Gas}\rput(psChartI6){Food}
\end{pspicture}}

\begin{document}
\title{\texttt{pstricks-add}\\additionals Macros for \texttt{pstricks}\\
    \small v.\pstricksaddFV}
%\docauthor{Herbert Vo\ss}
\author{Dominique Rodriguez\\Michael Sharpe\\Herbert Vo\ss}
\date{\today}

\maketitle

\fullWidth=\linewidth
\advance\fullWidth by \marginparsep
\advance\fullWidth by \marginparwidth


\begin{abstract}
This version of \verb+pstricks-add+ needs \verb+pstricks.tex+
version >1.04 from June 2004, otherwise the additional macros may
not work as expected. The ellipsis material and the option
\verb+asolid+ (renamed to \verb+eofill+) are
\index{fillstyle!eofill@\texttt{eofill}} now part of the new
\verb+pstricks.tex+ package, available on CTAN. \LPack{pstricks-add} will for ever be
an experimental and dynamical package, try it at your own risk.

\begin{itemize}
\item It is important to load \LPack{pstricks-add} as the \textbf{last} PSTricks related package, otherwise
a lot of the macros won't work in the expected way.
\item \LPack{pstricks-add} uses the extended version of the keyval package. So be sure that
you have installed \LPack{pst-xkey} which is part of the
\LPack{xkeyval}-package, and that all packages that use the old
keyval interface are loaded \textbf{before} the
\LPack{xkeyval}.\cite{xkeyval}
\item the option \Lkeyword{tickstyle} from \LPack{pst-plot} is no longer supported; use \Lkeyword{ticksize} instead.
\item the option \Lkeyword{xyLabel} is no longer supported; use the option \Lkeyword{labelFontSize} instead.
\item if \LPack{pstricks-add} is loaded together with the package  \LPack{pst-func} then  \Lkeyword{InsideArrow}
    of the \Lcs{psbezier} macro doesn't work!
\end{itemize}

\vfill
\noindent
Thanks to:  
Hendri Adriaens;
Stefano Baroni;
Martin Chicoine;
Gerry Coombes;
Ulrich Dirr;
Christophe Fourey;
Hubert G\"a\ss lein;
J\"urgen Gilg;
Denis Girou;
Pablo Gonzáles;
Peter Hutnick;
Christophe Jorssen;
Uwe Kern;
Manuel Luque;
Jens-Uwe Morawski;
Tobias N\"ahring;
Rolf Niepraschk;
Alan Ristow;
Christine R\"omer;
Arnaud Schmittbuhl;
John Smith;
Timothy Van Zandt
\end{abstract}

\clearpage
\tableofcontents


\clearpage

\section{\nxLcs{psGetSlope} and \nxLcs{psGetDistance}}
%--------------------------------------------------------------------------------------

\begin{BDef}
\Lcs{psGetSlope}\coord1\coord2\Lcs{\Larga{macro}}\\
\Lcs{psGetDistance}\coord1\coord2\Lcs{\Larga{macro}}
\end{BDef}

\begin{LTXexample}[width=4cm]
\psGetSlope(-2,1)(3,1)\SlopeVal \SlopeVal \quad
\psGetDistance(-2,1)(3,1)\DVal \DVal\\
\psGetSlope(-2,1)(-3,-1)\SlopeVal \SlopeVal\quad
\psGetDistance(-2,1)(-3,-1)\DVal \DVal\\
\psGetSlope(-2,0)(3,-1)\SlopeVal \SlopeVal\quad
\psGetDistance(-2,0)(3,-1)\DVal \DVal\\
\psGetSlope(-2111,-12)(3,1)\SlopeVal \SlopeVal\quad
%\psGetDistance(-2111,-12)(3,1)\DVal ==> Overflow!
\end{LTXexample}

\clearpage

%--------------------------------------------------------------------------------------
\section{"`Handmade"' lines :-)}
%--------------------------------------------------------------------------------------

\begin{BDef}
\Lcs{pslineByHand}\OptArgs\coord1\coord2\coord3 \ldots
\end{BDef}

\begin{LTXexample}[width=0.4\linewidth]
\begin{pspicture}(4,6)
\psset{unit=2cm}
  \pslineByHand[linecolor=red](0,0)(0,2)(2,2)(2,0)(0,0)(2,2)(1,3)(0,2)(2,0)
\end{pspicture}
\end{LTXexample}

\iffalse
  \pslineByHand( 1.20, 1.50)( 1.20, 1.51)( 1.20, 1.53)( 1.20, 1.54)( 1.19, 1.55)( 1.19, 1.56)
    ( 1.19, 1.57)( 1.18, 1.59)( 1.18, 1.60)( 1.17, 1.61)( 1.16, 1.62)( 1.15, 1.63)( 1.15, 1.64)
    ( 1.14, 1.65)( 1.13, 1.65)( 1.12, 1.66)( 1.11, 1.67)( 1.10, 1.68)( 1.09, 1.68)( 1.07, 1.69)
    ( 1.06, 1.69)( 1.05, 1.69)( 1.04, 1.70)( 1.03, 1.70)( 1.01, 1.70)( 1.00, 1.70)( 0.99, 1.70)
    ( 0.97, 1.70)( 0.96, 1.70)( 0.95, 1.69)( 0.94, 1.69)( 0.93, 1.69)( 0.91, 1.68)( 0.90, 1.68)
    ( 0.89, 1.67)( 0.88, 1.66)( 0.87, 1.65)( 0.86, 1.65)( 0.85, 1.64)( 0.85, 1.63)( 0.84, 1.62)
    ( 0.83, 1.61)( 0.82, 1.60)( 0.82, 1.59)( 0.81, 1.57)( 0.81, 1.56)( 0.81, 1.55)( 0.80, 1.54)
    ( 0.80, 1.53)( 0.80, 1.51)( 0.80, 1.50)( 0.80, 1.49)( 0.80, 1.47)( 0.80, 1.46)( 0.81, 1.45)
    ( 0.81, 1.44)( 0.81, 1.43)( 0.82, 1.41)( 0.82, 1.40)( 0.83, 1.39)( 0.84, 1.38)( 0.85, 1.37)
    ( 0.85, 1.36)( 0.86, 1.35)( 0.87, 1.35)( 0.88, 1.34)( 0.89, 1.33)( 0.90, 1.32)( 0.91, 1.32)
    ( 0.93, 1.31)( 0.94, 1.31)( 0.95, 1.31)( 0.96, 1.30)( 0.97, 1.30)( 0.99, 1.30)( 1.00, 1.30)
    ( 1.01, 1.30)( 1.03, 1.30)( 1.04, 1.30)( 1.05, 1.31)( 1.06, 1.31)( 1.07, 1.31)( 1.09, 1.32)
    ( 1.10, 1.32)( 1.11, 1.33)( 1.12, 1.34)( 1.13, 1.35)( 1.14, 1.35)( 1.15, 1.36)( 1.15, 1.37)
    ( 1.16, 1.38)( 1.17, 1.39)( 1.18, 1.40)( 1.18, 1.41)( 1.19, 1.43)( 1.19, 1.44)( 1.19, 1.45)
    ( 1.20, 1.46)( 1.20, 1.47)( 1.20, 1.49)( 1.20, 1.50)
\fi

\begin{LTXexample}[pos=t]
\begin{pspicture}(\linewidth,3)
\multido{\rA=0.00+0.25}{12}{\pslineByHand[linecolor=blue](0,\rA)(\linewidth,\rA)}
\end{pspicture}
\end{LTXexample}

The amplitude and the width can be changed by the optional arguments \Lkeyword{varsteptol} and
\Lkeyword{VarStepEpsilon}. Both are preset to \verb+VarStepEpsilon=2,varsteptol=0.8+.


\begin{LTXexample}[pos=t]
\begin{pspicture}(\linewidth,3)
\multido{\rA=0.00+0.25}{12}{%
  \pslineByHand[linecolor=blue,VarStepEpsilon=4,varsteptol=2](0,\rA)(\linewidth,\rA)}
\end{pspicture}
\end{LTXexample}

\clearpage

%--------------------------------------------------------------------------------------
\section{\nxLcs{rmultiput}: a multiple \nxLcs{rput}}
%--------------------------------------------------------------------------------------
\verb+PSTricks+ already has a \Lcs{multirput}, which puts a box n
times with a difference of $dx$ and $dy$ relative to each other.
It is not possible to put it with a different distance from one
point to the next. This is possible with \Lcs{rmultiput}:

\begin{BDef}
\LcsStar{rmultiput}\OptArgs\Largb{any material}\coord1\coord2\ldots\Largr{\coord{n}}
\end{BDef}

\begin{LTXexample}[width=6.2cm]
\psset{unit=0.75}
\begin{pspicture}(-4,-4)(4,4)
\rmultiput[rot=45]{\red\psscalebox{3}{\ding{250}}}%
    (-2,-4)(-2,-3)(-3,-3)(-2,-1)(0,0)(1,2)(1.5,3)(3,3)
\rmultiput[rot=90,ref=lC]{\blue\psscalebox{2}{\ding{253}}}%
    (-2,2.5)(-2,2.5)(-3,2.5)(-2,1)(1,-2)(1.5,-3)(3,-3)
\psgrid[subgriddiv=0,gridcolor=lightgray]
\end{pspicture}
\end{LTXexample}

\clearpage


%--------------------------------------------------------------------------------------
\section{\nxLcs{psVector}: Drawing relative vector lines}
%--------------------------------------------------------------------------------------

The new macros \Lcs{psStartPoint} and \Lcs{psVector} allow to draw a series of
vectors which start point refers to the endpoint of the last drawn vector. The 
coordinates of the endpoint are \emph{always} interpreted relative to the last
the vector. The first vector refers to the coordinates set by \Lcs{psStartPoint}.
With the boolean argument one can draw the horizontal angle of the vector.

The style of the angle arc is saved in \Lkeyval{psMarkAngleStyle} and the style
for the horizontal line in \Lkeyval{psMarkAngleLineStyle} and preset to

\begin{lstlisting}
\newpsstyle{psMarkAngleStyle}{arrows=->,arrowsize=4pt}
\newpsstyle{psMarkAngleLineStyle}{linestyle=dotted}
\end{lstlisting}


\begin{pspicture}[showgrid](10,10)
 \psStartPoint(1,1)
 \psVector(3;30)\psVector(4;60)\psVector[linecolor=red](3;10)
 \psVector[linestyle=dashed](4;110)
 \psStartPoint(1,1)\psset{markAngle}
 \psVector[linestyle=dashed](4;110)\psVector[linecolor=red](3;10)
 \psVector(4;60)\psVector(3;30)
\end{pspicture}

\begin{lstlisting}
\begin{pspicture}[showgrid](10,10)
 \psStartPoint(1,1)
 \psVector(3;30)\psVector(4;60)\psVector[linecolor=red](3;10)
 \psVector[linestyle=dashed](4;110)
 \psStartPoint(1,1)\psset{markAngle}
 \psVector[linestyle=dashed](4;110)\psVector[linecolor=red](3;10)
 \psVector(4;60)\psVector(3;30)
\end{pspicture}
\end{lstlisting}

All end points of the vectors are saved in node names with the preset name \verb=Vector#=,
where \# is the consecutive  number of the nodes. \verb=Vector0= ist the starting point of
the first \Lcs{psVector}. With the macro \Lcs{psStartPoint} one can set the starting point and
with optional argument the name of the nodes. \verb=Vector3= is the default node name of
the endpoint of the third vector or the name of the starting point of the forth vector.

\begin{BDef}
\Lcs{psStartPoint}\OptArg{node basename}\Largr{$x$,$y$}
\end{BDef}

\begin{pspicture}[showgrid,linewidth=1pt](10,10.4)
 \psStartPoint[A](1,1)% nodes have the base name A
 \psVector(3;30)\psVector(4;60)\psVector[linecolor=red](3;10)
 \psVector[linestyle=dashed](4;110)
 \psline{->}(A0)(A4)
 \psStartPoint[B](1,1)\psset{markAngle}% nodes have the base name B
 \psVector[linestyle=dashed](4;110)
 \psVector[linecolor=red](3;10)
 \psVector(4;60)\psVector(3;30)
 \psline[arrows=-D>,arrowscale=2,linewidth=1.5pt,linecolor=red](B2)(A2)
 \psline[arrows=-D>,arrowscale=2,linewidth=1.5pt,linecolor=blue](A3)(B3)
 \multido{\iA=0+1}{5}{\uput[0](A\iA){A\iA}\uput[180](B\iA){B\iA}}
\end{pspicture}

\begin{lstlisting}
\begin{pspicture}[showgrid,linewidth=1pt](10,10.4)
 \psStartPoint[A](1,1)% nodes have the base name A
 \psVector(3;30)\psVector(4;60)\psVector[linecolor=red](3;10)
 \psVector[linestyle=dashed](4;110)
 \psline{->}(A0)(A4)
 \psStartPoint[B](1,1)\psset{markAngle}% nodes have the base name B
 \psVector[linestyle=dashed](4;110)
 \psVector[linecolor=red](3;10)
 \psVector(4;60)\psVector(3;30)
 \psline[arrows=-D>,arrowscale=2,linewidth=1.5pt,linecolor=red](B2)(A2)
 \psline[arrows=-D>,arrowscale=2,linewidth=1.5pt,linecolor=blue](A3)(B3)
 \multido{\iA=0+1}{5}{\uput[0](A\iA){A\iA}\uput[180](B\iA){B\iA}
 \end{pspicture}
\end{lstlisting}

\clearpage


%--------------------------------------------------------------------------------------
\section{\nxLcs{psCircleTangents}: Calculating tangent lines of circles}
%--------------------------------------------------------------------------------------

The macro calculates the points on a circle where tangent lines from another
point or another circle are drawn.

\begin{BDef}
\Lcs{psCircleTangents}\Largr{$x1,y1$}\Largr{$x2,y2$}\Largb{Radius}\\
\Lcs{psCircleTangents}\Largr{$x1,y1$}\Largb{Radius}\Largr{$x2,y2$}\Largb{Radius}
\end{BDef}

In the first case the coordinates of a point and the center and the radius
of a circle must be given. The names of the calculates node names are \verb=CircleT1=
and \verb=CircleT2=.

\bigskip
\begin{pspicture}[showgrid](0,3)(10,10)
\psdot(2,4)\pscircle(7,7){2}
\psCircleTangents(2,4)(7,7){2}
\pcline[nodesep=-1cm,linecolor=blue](2,4)(CircleT1)
\pcline[nodesep=-1cm,linecolor=blue](2,4)(CircleT2)
\psdots(CircleT1)(CircleT2)
\uput[-80](CircleT1){T1}\uput[115](CircleT2){T2}
\end{pspicture}


\begin{lstlisting}
\begin{pspicture}[showgrid](0,3)(10,10)
\psdot(2,4)\pscircle(7,7){2}
\psCircleTangents(2,4)(7,7){2}
\pcline[nodesep=-1cm,linecolor=blue](2,4)(CircleT1)
\pcline[nodesep=-1cm,linecolor=blue](2,4)(CircleT2)
\psdots(CircleT1)(CircleT2)
\uput[-80](CircleT1){T1}\uput[115](CircleT2){T2}
\end{pspicture}
\end{lstlisting}

\bigskip
When using the other variant of the macro two circles must be given. The macro then defines
ten nodes, named \verb=CircleTC1= and \verb=CircleTC2= for the two intersection points,
 \verb=CircleTO1=, \verb=CircleTO2=, \verb=CircleTO3=, and \verb=CircleTO4= for the four
 nodes of the outer tangent lines and 
  \verb=CircleTI1=, \verb=CircleTI2=, \verb=CircleTI3=, and \verb=CircleTI4= for the
  four nodes of the inner tangent lines.

\bigskip
\begin{pspicture}[showgrid](-2,-2)(10,10)
\pscircle(1,1){1}\pscircle(7,7){3}
\psCircleTangents(1,1){1}(7,7){3}
\pcline[nodesep=-1cm,linecolor=blue](CircleTO1)(CircleTO2)
\pcline[nodesep=-1cm,linecolor=blue](CircleTO3)(CircleTO4)
\pcline[nodesep=-1cm,linecolor=red](CircleTI1)(CircleTI2)
\pcline[nodesep=-1cm,linecolor=red](CircleTI3)(CircleTI4)
\psdots(CircleTC1)(CircleTC2)%
  (CircleTO1)(CircleTO2)(CircleTO3)(CircleTO4)%
  (CircleTI1)(CircleTI2)(CircleTI3)(CircleTI4)%
\uput[0](CircleTC1){TC1}\uput[0](CircleTC2){TC2}
\uput[-80](CircleTI1){TI1}\uput[115](CircleTI2){TI2}
\uput[150](CircleTI3){TI3}\uput[-45](CircleTI4){TI4}
\uput[-80](CircleTO1){TO1}\uput[150](CircleTO2){TO2}
\uput[150](CircleTO3){TO3}\uput[-45](CircleTO4){TO4}
\end{pspicture}

\bigskip
\begin{lstlisting}
\begin{pspicture}[showgrid](-2,-2)(10,10)
\pscircle(1,1){1}\pscircle(7,7){3}
\psCircleTangents(1,1){1}(7,7){3}
\pcline[nodesep=-1cm,linecolor=blue](CircleTO1)(CircleTO2)
\pcline[nodesep=-1cm,linecolor=blue](CircleTO3)(CircleTO4)
\pcline[nodesep=-1cm,linecolor=red](CircleTI1)(CircleTI2)
\pcline[nodesep=-1cm,linecolor=red](CircleTI3)(CircleTI4)
\psdots(CircleTC1)\psdots(CircleTC2)%
  (CircleTO1)(CircleTO2)(CircleTO3)(CircleTO4)%
  (CircleTI1)(CircleTI2)(CircleTI3)(CircleTI4)%
\uput[0](CircleTC1){TC1}\uput[0](CircleTC2){TC2}
\uput[-80](CircleTI1){TI1}\uput[115](CircleTI2){TI2}
\uput[150](CircleTI3){TI3}\uput[-45](CircleTI4){TI4}
\uput[-80](CircleTO1){TO1}\uput[150](CircleTO2){TO2}
\uput[150](CircleTO3){TO3}\uput[-45](CircleTO4){TO4}
\end{pspicture}
\end{lstlisting}


\clearpage

%--------------------------------------------------------------------------------------
\section{\nxLcs{psEllipseTangents}: Calculating tangent lines of an ellipse}
%--------------------------------------------------------------------------------------

The macro calculates the two points on an ellipse where tangent lines from an outside  point
 are drawn.

\begin{BDef}
\Lcs{psEllipseTangents}\Largr{$x_0,y_0$}\Largr{$a,b$}\Largr{$x_p,y_p$}\\
\end{BDef}

The first two pairs of coordinates are the same as the ones for the default ellipse.
The names of the calculates node names are \verb=EllipseT1=
and \verb=EllipseT2=.

\bigskip
\begin{pspicture}[showgrid](0,3)(10,10)
\psdot(2,4)\psellipse(7,7)(3,1.5)
\psEllipseTangents(7,7)(3,1.5)(2,4)
\pcline[nodesep=-1cm,linecolor=blue](2,4)(EllipseT1)
\pcline[nodesep=-1cm,linecolor=blue](2,4)(EllipseT2)
\psdots(EllipseT1)(EllipseT2)
\uput[-80](EllipseT1){T1}\uput[115](EllipseT2){T2}
\end{pspicture}


\begin{lstlisting}
\begin{pspicture}[showgrid](0,3)(10,10)
\psdot(2,4)\psellipse(7,7)(3,1.5)
\psEllipseTangents(7,7)(3,1.5)(2,4)
\pcline[nodesep=-1cm,linecolor=blue](2,4)(EllipseT1)
\pcline[nodesep=-1cm,linecolor=blue](2,4)(EllipseT2)
\psdots(EllipseT1)(EllipseT2)
\uput[-80](EllipseT1){T1}\uput[115](EllipseT2){T2}
\end{pspicture}
\end{lstlisting}


\clearpage

%--------------------------------------------------------------------------------------
\section{\nxLcs{psrotate}: Rotating objects}
%--------------------------------------------------------------------------------------
\Lcs{rput} also has an optional argument for rotating objects, but
it always depends on the \Lcs{rput} coordinates. With
\Lcs{psrotate} the rotating center can be placed anywhere. The
rotation is done with \verb+\pscustom+, all optional arguments are
only valid if they are part of the \verb+\pscustom+ macro.

\begin{BDef}
\Lcs{psrotate}\OptArgs\Largr{$x,y$}\Largb{rot angle}\Largb{object}
\end{BDef}

\begin{LTXexample}[width=0.4\linewidth]
\psset{unit=0.75}
\begin{pspicture}(-0.5,-3.5)(8.5,4.5)
  \psaxes{->}(0,0)(-0.5,-3)(8.5,4.5)
  \psdots[linecolor=red,dotscale=1.5](2,1)
  \psarc[linecolor=red,linewidth=0.4pt,showpoints=true]
        {->}(2,1){3}{0}{60}
  \pspolygon[linecolor=green,linewidth=1pt](2,1)(5,1.1)(6,-1)(2,-2)
  \psrotate(2,1){60}{%
    \pspolygon[linecolor=blue,linewidth=1pt](2,1)(5,1.1)(6,-1)(2,-2)}
\end{pspicture}
\end{LTXexample}


\begin{LTXexample}[width=6cm]
\begin{pspicture}(-1,-1)(3,6)
\def\canne{%  Idea by Manuel Luque
  \psgrid[subgriddiv=0](-1,0)(1,5)
  \pscustom[linewidth=2mm]{\psline(0,4)\psarcn(0.3,4){0.3}{180}{360}}%
  \pscircle*(0.6,4){0.1}\pstriangle*(0,0)(0.2,-0.3)}
\def\Object{}
  \canne
  \psrotate(0.3,4){45}{\psset{linecolor=red!50}\canne}
  \psrotate(0.3,4){90}{\psset{linecolor=blue!50}\canne}
  \psrotate(0.3,4){360}{\psset{linecolor=cyan!50}\canne}
  \psdot[linecolor=red](0.3,4)
\end{pspicture}
\end{LTXexample}


\begin{LTXexample}[pos=t]
\begin{pspicture}(0,-6)(15,5)
\def\majorette{\psline[linewidth=0.5mm](0,2)%  Idea by Manuel Luque
               \pscircle[fillstyle=solid]{0.1}
               \pscircle[fillstyle=solid](0,2){0.1}}
  \psaxes[linewidth=0.5pt]{->}(0,0)(0,-5)(15,5)
  \pstVerb{/V0 10 def /Alpha 45 def}% vitesse initiale, angle de lancement
  \multido{\nT=0.0+0.05,\iA=0+40}{41}{%
    \pstVerb{/nT \nT\space def}%
    \rput(!V0 Alpha cos mul nT mul -9.81 2 div nT dup mul mul V0 Alpha sin mul nT mul add){%
       \psrotate(0,1){\iA}{\majorette\psdot[linecolor=red](0,1)\psdot[linecolor=green](0,2)}}}
  \parametricplot[linecolor=red]{0}{2}{% trajectoire du milieu
     V0 Alpha cos mul t mul -9.81 2 div t dup mul mul V0 Alpha sin mul t mul add 1 add}
  \parametricplot[linecolor=green,plotpoints=360]{0}{2}{% d'une extremite
     V0 Alpha cos mul t mul 800 t mul sin sub % x(t)
     -9.81 2 div t dup mul mul V0 Alpha sin mul t mul add 1 add 800 t mul cos add }%y(t)
\end{pspicture}
\end{LTXexample}


\clearpage

%--------------------------------------------------------------------------------------
\section{\nxLcs{psComment}: comments to a graphic}
%--------------------------------------------------------------------------------------

\begin{BDef}
\LcsStar{psComment}\OptArgs\OptArg*{\Largb{arrows}}\coord0\coord1\Largb{Text}\OptArg{line macro}\OptArg{put macro}
\end{BDef}

By default the macro uses the \Lcs{ncline} macro to draw a line from the first to the
second point, it can be changed with the first additional optional argument. The label is
put by default with \Lcs{rput}, which can be changed with the last optional argument.
If this is used, then the line macro has also be defined, eg \verb+\psComment(A)(B){text}[\ncarc][\ncput}+
At least, leave the argument empty.


\begin{LTXexample}[pos=t,wide]
\SpecialCoor\newpsstyle{weiss}{fillstyle=solid,fillcolor=white}
\footnotesize\psset{unit=0.5cm,dimen=middle}
\begin{pspicture}(-12,-4)(6,10)
\psframe*[linecolor=black!20](-5,-3)(5,7) \psframe*[linecolor=black!40](-5,3)(5,6)
\pscircle(-8.19,5.51){0.2}
\psframe[fillcolor=white,fillstyle=solid](-5.8,3.6)(4.3,5.8)
\psframe(-8.98,3.14)(-5.8,6.32)
\multido{\rA=-4.1+1.3}{5}{\rput(\rA,-2.4){\psframe[style=weiss](1.1,6)
  \psline(0,0)(1.1,0.5)(0,1)(1.1,1.6)(0,2.2)(1.1,2.7)(0,3.2)(1.1,3.2)}}
\pspolygon*(-4.1,3.7)(-4.1,3)(-3,3)(-3.01,3.7)(-3.54,4.19)
\pspolygon*(1.09,3.7)(1.1,3)(2.2,3)(2.18,3.7)(1.65,4.24)
\pspolygon*(-2.78,3.7)(-2.8,3)(-1.7,3)(-1.71,3.7)(-2.27,4.04)
\pspolygon*(-1.51,3.7)(-1.5,3)(-0.4,3)(-0.41,3.7)(-1.02,4.17)
\pspolygon*(-0.21,3.7)(-0.2,3)(0.9,3)(0.89,3.7)(0.3,4.04)
\psline(-5,3.83)(-4.15,3.86)(-3.5,4.3)(-2.85,3.81)(-2.22,4.21)(-1.6,3.86)(-0.99,4.33)
       (-0.28,3.83)(0.35,4.19)(0.97,3.83)(1.65,4.39)(2.2,4.01)(3.57,4.89)(2.41,5.8)
  \psline(-5,5.8)(-5.78,5.8)  \psline(-5.78,5.47)(2.85,5.47)
  \psline(-5.8,3.52)(-5,3.5)  \psline(3.57,4.89)(-5.8,4.89)
  \psComment*[ref=r]{->}(-8.14,1.19)(-4.31,3.27){Mantelstift}
  \psComment*[ref=r]{->}(-8.17,-0.56)(-4.37,1.59){Kernstift}[\ncarc]
  \psComment*[ref=r]{->}(-7.91,-2.24)(-4.44,-0.23){Feder}[\ncarc]
  \psComment[npos=-0.1]{->}(-3.48,8.72)(-1.33,5.46){Nur f\"ur Profil}
\end{pspicture}
\end{LTXexample}

\clearpage
%--------------------------------------------------------------------------------------
\section{\nxLcs{psChart}: a pie chart}
%--------------------------------------------------------------------------------------

\begin{BDef}
\Lcs{psChart}\OptArgs\Largb{comma separated value list}\Largb{comma separated value list}\Largb{radius}
\end{BDef}

The special optional arguments for the \Lcs{psChart} macro are as follows:

\noindent
\begin{tabularx}{\linewidth}{@{}>{\ttfamily}lX>{\ttfamily}l@{}}
\textrm{\emph{name}} & \textrm{\emph{description}} & \textrm{\emph{default}}\\\hline
\Lkeyword{chartSep}  & distance from the pie chart center to an outraged pie piece & 10pt\\
\Lkeyword{chartColor} & gray or colored pie (values are: \texttt{gray} or \texttt{color})& gray\\
\Lkeyword{userColor} & a comma separated list of user defined colors for the pie & \{\}\\
\Lkeyword{chartNodeI}& the position of the inner node, relative to the radius & 0.75\\
\Lkeyword{chartNodeO}& the position of the outer node, relative to the radius & 1.5
\end{tabularx}

\bigskip
The first mandatory argument is the list of the values and may not be empty. The second
one is a list of outraged pieces, numbered consecutively from 1 to up the total number
of values. The list of user defined colors must be enclosed in braces!

The macro \Lcs{psChart} defines for every value three nodes at the half angle and
in distances from 0.75, 1, and 1.25 times of the radius from the origin. The nodes
are named as \verb+psChartI?+, \verb+psChart?+, and \verb+psChartO?+, where ? is the number of
the pie. The letter I leads to the inner node and the letter O to the outer node.
The distance can be changed with the optional arguments \Lkeyword{chartNodeI} and
\Lkeyword{chartNodeO} in the usual way with \verb+\psset{chartNodeI=...,chartNodeO=...}+.

The other one is the node on the circle line.
The origin is by default \texttt{(0,0)}. Moving the pie to another position can be done as
usual with the \Lcs{rput}-macro. The used colors are named internally as \Lkeyword{chartFillColor?}
and can be used by the user for coloring lines or text.

\begin{LTXexample}[width=6cm]
\begin{pspicture}(-3,-3)(3,3)
\psChart{ 23, 29, 3, 26, 28, 14 }{}{2}
\multido{\iA=1+1}{6}{%
  \psdot(psChart\iA)\psdot(psChartI\iA)\psdot(psChartO\iA)%
  \psline[linestyle=dashed,linecolor=white](psChart\iA)
  \psline[linestyle=dashed](psChart\iA)(psChartO\iA)}
\end{pspicture}
\end{LTXexample}

\begin{LTXexample}[width=6cm]
\begin{pspicture}(-3,-3)(3,3)
\psChart[chartColor=color]{45,90}{1}{2}
\ncline[linecolor=-chartFillColor1,
  nodesepB=-20pt]{psChartO1}{psChart1}
\rput[l](psChartO1){%
  \textcolor{chartFillColor1}{pie no 1}}
\ncline[linecolor=-chartFillColor2,
  nodesepB=-20pt]{psChartO2}{psChart2}
\rput[lt](psChartO2){%
  \textcolor{chartFillColor2}{pie no 2}}
\end{pspicture}
\end{LTXexample}

\begin{LTXexample}[width=7.5cm]
\psframebox[fillcolor=black!20,
  fillstyle=solid]{%
\begin{pspicture}(-3.5,-3.5)(4.25,3.5)
\psChart[chartColor=color]%
  {23, 29, 3, 26, 28, 14, 17, 4, 9}{}{2}
\multido{\iA=1+1}{9}{%
  \ncline[linecolor=-chartFillColor\iA,
    nodesepB=-10pt]{psChartO\iA}{psChart\iA}
  \rput[l](psChartO\iA){%
    \textcolor{chartFillColor\iA}{pie no \iA}}}
\end{pspicture}}
\end{LTXexample}

\begin{LTXexample}[width=6cm]
\begin{pspicture}(-3,-3)(3,3)
\psChart[userColor={red!30,green!30,
    blue!40,gray,magenta!60,cyan}]%
      { 23, 29, 3, 26, 28, 14 }{1,4}{2}
\end{pspicture}
\end{LTXexample}

\begin{LTXexample}[width=6cm]
\begin{pspicture}(-3,-2.5)(3,2.5)
\psChart{ 23, 29, 3, 26, 28, 14 }{}{2}
\multido{\iA=1+1}{6}{\rput*(psChartI\iA){\iA}}
\end{pspicture}
\end{LTXexample}


%\begin{LTXexample}[pos=t]
\psset{unit=1.5}
\begin{pspicture}(-3,-3)(3,3)
\psChart[userColor={red!30,green!30,blue!40,gray,cyan!50,
    magenta!60,cyan},chartSep=30pt,shadow=true,shadowsize=5pt]{34.5,17.2,20.7,15.5,5.2,6.9}{6}{2}
\psset{nodesepA=5pt,nodesepB=-10pt}
\ncline{psChartO1}{psChart1}\nput{0}{psChartO1}{1000 (34.5\%)}
\ncline{psChartO2}{psChart2}\nput{150}{psChartO2}{500 (17.2\%)}
\ncline{psChartO3}{psChart3}\nput{-90}{psChartO3}{600 (20.7\%)}
\ncline{psChartO4}{psChart4}\nput{0}{psChartO4}{450 (15.5\%)}
\ncline{psChartO5}{psChart5}\nput{0}{psChartO5}{150 (5.2\%)}
\ncline{psChartO6}{psChart6}\nput{0}{psChartO6}{200 (6.9\%)}
\bfseries%
\rput(psChartI1){Taxes}\rput(psChartI2){Rent}\rput(psChartI3){Bills}
\rput(psChartI4){Car}\rput(psChartI5){Gas}\rput(psChartI6){Food}
\end{pspicture}
%\end{LTXexample}
\psset{unit=1cm}

\begin{lstlisting}
\psset{unit=1.5}
\begin{pspicture}(-3,-3)(3,3)
\psChart[userColor={red!30,green!30,blue!40,gray,cyan!50,
    magenta!60,cyan},chartSep=30pt,shadow=true,shadowsize=5pt]{34.5,17.2,20.7,15.5,5.2,6.9}{6}{2}
\psset{nodesepA=5pt,nodesepB=-10pt}
\ncline{psChartO1}{psChart1}\nput{0}{psChartO1}{1000 (34.5\%)}
\ncline{psChartO2}{psChart2}\nput{150}{psChartO2}{500 (17.2\%)}
\ncline{psChartO3}{psChart3}\nput{-90}{psChartO3}{600 (20.7\%)}
\ncline{psChartO4}{psChart4}\nput{0}{psChartO4}{450 (15.5\%)}
\ncline{psChartO5}{psChart5}\nput{0}{psChartO5}{150 (5.2\%)}
\ncline{psChartO6}{psChart6}\nput{0}{psChartO6}{200 (6.9\%)}
\bfseries%
\rput(psChartI1){Taxes}\rput(psChartI2){Rent}\rput(psChartI3){Bills}
\rput(psChartI4){Car}\rput(psChartI5){Gas}\rput(psChartI6){Food}
\end{pspicture}
\end{lstlisting}



\clearpage
%--------------------------------------------------------------------------------------
\section{\nxLcs{psHomothetie}: central dilatation}
%--------------------------------------------------------------------------------------

\begin{BDef}
\Lcs{psHomothetie}\OptArgs\Largr{center}\Largb{factor}\Largb{object}
\end{BDef}

\begin{LTXexample}[width=9cm]
\begin{pspicture}[showgrid=true](-5,-4)(4,8)
\psBill% needs package pst-fun
\psHomothetie[linecolor=blue](4,-3){2}{\psBill}
\psdots[dotsize=3pt,linecolor=red](4,-3)
\psplot[linestyle=dashed,linecolor=red]{-5}{4}%
  [ /m -3 -0.85 sub 4 0.6 sub div def ]
  { m x mul m 4 mul sub 3 sub }%
\psHomothetie[linecolor=green](4,-3){-0.2}{\psBill}
\end{pspicture}
\end{LTXexample}

%\pstVerb{ /m -3 -0.85 sub 4 0.6 sub div def }


\clearpage

%--------------------------------------------------------------------------------------
\section{\nxLcs{psbrace}}
%--------------------------------------------------------------------------------------
\begin{BDef}
\LcsStar{psbrace}\OptArgs\Largr{A}\Largr{B}\Largb{text}
\end{BDef}

Additional to all other available options from \LPack{pstricks} or the other
related packages,  there are two new option, named  \Lkeyword{braceWidth} and
\Lkeyword{bracePos}. All important ones are shown in the following graphics
and table.

\begin{center}
\begin{pspicture}[showgrid=true](10,5)
  \psbrace[braceWidth=1cm,braceWidthInner=1cm,
    braceWidthOuter=1cm,bracePos=0.6,fillcolor=white,
    nodesepA=10mm,nodesepB=10mm](0,5)(10,5){\fbox{Label}}
\pcline{<->}(3,3)(3,4)\ncput*{\footnotesize\ttfamily braceWidth}
\pcline{<->}(3,4)(3,5)\ncput*{\footnotesize\ttfamily braceWidthInner}
\pcline{<->}(3,2)(3,3)\ncput*{\footnotesize\ttfamily braceWidthOuter}
\pcline{<->}(6,1)(6,2)\ncput{\footnotesize\ttfamily nodesepB}
\pcline{<->}(6,1)(7,1)\ncput*{\footnotesize\ttfamily A}
\pcline{<->}(0,0.5)(6,0.5)\ncput*{\footnotesize\ttfamily bracePos}
\psdot[dotscale=2](0,5)\uput[0](0,5){\textbf{A}}
\psdot[dotscale=2](10,5)\uput[180](10,5){\textbf{B}}
\end{pspicture}
\end{center}

A positive value for \Lkeyword{nodesepA} and \Lkeyword{nodesepB} shifts the label to the upper right
and a negative value to the lower left. This does not depends on
the value for the rotating of the label!

\begin{center}
\begin{tabular}{@{}l|l@{}}
name & meaning\\\hline
\Lkeyword{braceWidth} & default is \Lcs{pslinewidth}\\
\Lkeyword{braceWidthInner} & default is \verb+10\pslinewidth+\\
\Lkeyword{braceWidthOuter} & default is \verb+10\pslinewidth+\\
\Lkeyword{bracePos} & relative position (default is $0.5$)\\
\Lkeyword{nodesepA} & x-separation (default is $0pt$)\\
\Lkeyword{nodesepB} & y-separation (default is $0pt$)\\
\Lkeyword{rot} & additional rotating for the text (default is $0$)\\
\Lkeyword{ref} & reference point for the text (default is c)\\
\Lkeyword{fillcolor} & default is black
\end{tabular}
\end{center}

By default the text is written perpendicular to the brace line and
can be changed with the \LPack{pstricks} option \Lkeyword{rot}=\ldots\ The
text parameter can take any object and may also be empty. The
reference point can be any value of the combination of \Lkeyval{l}
(left) or \Lkeyval{r} (right) and \Lkeyval{b} (bottom) or \Lkeyval{B}
(Baseline) or \Lkeyval{C} (center) or \Lkeyval{t} (top), where the
default is \Lkeyval{c}, the center of the object.



\begin{LTXexample}[width=4.5cm]
\begin{pspicture}(4,4)
\psgrid[subgriddiv=0,griddots=10]
\pnode(0,0){A}
\pnode(4,4){B}
\psbrace[linecolor=red,ref=lC](A)(B){Text I}
\psbrace*[linecolor=blue,ref=lC](3,4)(0,1){Text II}
\psbrace[fillcolor=white](3,0)(3,4){III}
\end{pspicture}
\end{LTXexample}

\bigskip
The option \Lcs{specialCoor} is enabled, so that all types of coordinates
are possible, (nodename), ($x,y$), ($nodeA|nodeB$), \ldots
The star version fills the inner of the \Index{brace} with the current linecolor.
With the fillcolor \verb+white+ or any other background color the brace can
be "`unfilled"'.

\begin{LTXexample}
\begin{pspicture}(8,2.5)
\psbrace(0,0)(0,2){\fbox{Text}}%
\psbrace[nodesepA=10pt](2,0)(2,2){\fbox{Text}}
\psbrace[ref=lC](4,0)(4,2){\fbox{Text}}
\psbrace[ref=lt,rot=90,nodesepB=-15pt](6,0)(6,2){\fbox{Text}}
\psbrace[ref=lt,rot=90,nodesepA=-5pt,nodesepB=15pt](8,2)(8,0){\fbox{Text}}
\end{pspicture}
\end{LTXexample}


\begin{LTXexample}
\def\someMath{$\int\limits_1^{\infty}\frac{1}{x^2}\,dx=1$}
\begin{pspicture}(8,2.5)
\psbrace[ref=lC](0,0)(0,2){\someMath}%
\psbrace[rot=90](2,0)(2,2){\someMath}
\psbrace[ref=lC](4,0)(4,2){\someMath}
\psbrace[ref=lt,rot=90,nodesepB=-30pt](6,0)(6,2){\someMath}
\psbrace[ref=lt,rot=90,nodesepB=30pt](8,2)(8,0){\someMath}
\end{pspicture}
\end{LTXexample}

%$

\begin{LTXexample}
\begin{pspicture}(\linewidth,5)
\psbrace(0,0.5)(\linewidth,0.5){\fbox{Text}}%
\psbrace[bracePos=0.25,nodesepB=10pt,rot=90](0,2)(\linewidth,2){\fbox{Text}}
\psbrace[ref=lC,nodesepA=-3.5cm,nodesepB=15pt,rot=90](0,4)(\linewidth,4){%
   \fbox{some very, very long wonderful Text}}
\end{pspicture}
\end{LTXexample}


\begin{LTXexample}[width=8cm]
\psset{unit=0.8}
\begin{pspicture}(10,11)
\psgrid[subgriddiv=0,griddots=10]
\pnode(0,0){A}
\pnode(4,6){B}
\psbrace[ref=lC](A)(B){One}
\psbrace[rot=180,nodesepA=-5pt,ref=rb](B)(A){Two}
\psbrace[linecolor=blue,bracePos=0.25,ref=lB](8,1)(1,7){Three}
\psbrace[braceWidth=-1mm,rot=180,ref=rB](8,1)(1,7){Four}
\psbrace*[linearc=0.5,fillstyle=none,linewidth=1pt,braceWidth=1.5pt,
  bracePos=0.25,ref=lC](8,1)(8,9){A}
\psbrace(4,9)(6,9){}
\psbrace(6,9)(6,7){}
\psbrace(6,7)(4,7){}
\psbrace(4,7)(4,9){}
\psset{linecolor=red}
\psbrace*[ref=lb](7,10)(3,10){I}
\psbrace*[ref=lb,bracePos=0.75](3,10)(3,6){II}
\psbrace*[ref=lb](3,6)(7,6){III}
\psbrace*[ref=lb](7,6)(7,10){IV}
\end{pspicture}
\end{LTXexample}

%$

\begin{LTXexample}[width=5cm]
\[
\begin{pmatrix}
    \Rnode[vref=2ex]{A}{~1} \\
    & \ddots \\
    && \Rnode[href=2]{B}{1} \\
    &&& \Rnode[vref=2ex]{C}{0} \\
    &&&& \ddots \\
    &&&&& \Rnode[href=2]{D}{0}~ \\
\end{pmatrix}
\]
\psbrace[rot=-90,nodesepB=-0.5,nodesepA=-0.2](B)(A){\small n times}
\psbrace[rot=-90,nodesepB=-0.5,nodesepA=-0.2](D)(C){\small n times}
\end{LTXexample}


\clearpage
It is also possible to put a vertical brace around a
default paragraph. This works by setting two invisible nodes at
the beginning and the end of the paragraph. Indentation is
possible with a minipage.

\small
Some nonsense text, which is nothing more than nonsense.
Some nonsense text, which is nothing more than nonsense.

\noindent\rnode{A}{}

\vspace*{-1ex}
Some nonsense text, which is nothing more than nonsense.
Some nonsense text, which is nothing more than nonsense.
Some nonsense text, which is nothing more than nonsense.
Some nonsense text, which is nothing more than nonsense.
Some nonsense text, which is nothing more than nonsense.
Some nonsense text, which is nothing more than nonsense.
Some nonsense text, which is nothing more than nonsense.
Some nonsense text, which is nothing more than nonsense.

\vspace*{-2ex}\noindent\rnode{B}{}\psbrace*[linecolor=red](A)(B){}

Some nonsense text, which is nothing more than nonsense.
Some nonsense text, which is nothing more than nonsense.

\medskip\hfill\begin{minipage}{0.95\linewidth}
\noindent\rnode{A}{}

\vspace*{-1ex}
Some nonsense text, which is nothing more than nonsense.
Some nonsense text, which is nothing more than nonsense.
Some nonsense text, which is nothing more than nonsense.
Some nonsense text, which is nothing more than nonsense.
Some nonsense text, which is nothing more than nonsense.
Some nonsense text, which is nothing more than nonsense.
Some nonsense text, which is nothing more than nonsense.
Some nonsense text, which is nothing more than nonsense.

\vspace*{-2ex}
\noindent\rnode{B}{}\psbrace[linecolor=red](A)(B){}
\end{minipage}

\normalsize

\begin{lstlisting}
Some nonsense text, which is nothing more than nonsense.
Some nonsense text, which is nothing more than nonsense.

\noindent\rnode{A}{}

\vspace*{-1ex}
Some nonsense text, which is nothing more than nonsense.
Some nonsense text, which is nothing more than nonsense.
Some nonsense text, which is nothing more than nonsense.
Some nonsense text, which is nothing more than nonsense.
Some nonsense text, which is nothing more than nonsense.
Some nonsense text, which is nothing more than nonsense.
Some nonsense text, which is nothing more than nonsense.
Some nonsense text, which is nothing more than nonsense.

\vspace*{-2ex}\noindent\rnode{B}{}\psbrace[linecolor=red](A)(B){}

Some nonsense text, which is nothing more than nonsense.
Some nonsense text, which is nothing more than nonsense.

\medskip\hfill\begin{minipage}{0.95\linewidth}
\noindent\rnode{A}{}

\vspace*{-1ex}
Some nonsense text, which is nothing more than nonsense.
Some nonsense text, which is nothing more than nonsense.
Some nonsense text, which is nothing more than nonsense.
Some nonsense text, which is nothing more than nonsense.
Some nonsense text, which is nothing more than nonsense.
Some nonsense text, which is nothing more than nonsense.
Some nonsense text, which is nothing more than nonsense.
Some nonsense text, which is nothing more than nonsense.

\vspace*{-2ex}\noindent\rnode{B}{}\psbrace[linecolor=red](A)(B){}
\end{minipage}
\end{lstlisting}

\clearpage


%--------------------------------------------------------------------------------------
\section{Random dots}
%--------------------------------------------------------------------------------------
The syntax of the new macro \Lcs{psRandom} is:

\begin{BDef}
\Lcs{psRandom}\OptArgs\Largb{}\\
\Lcs{psRandom}\OptArgs\OptArg*{\Largr{$x_{Min},y_{Min}$}}\OptArg*{\Largr{$x_{Max},y_{Max}$}}\Largb{clip path} %$
%\psRandom[<option>](<xMax,yMax>){<clip path>}
%\psRandom[<option>](<xMin,yMin>)(<xMax,yMax>){<clip path>}
\end{BDef}

If there is no area for the dots defined, then \verb+(0,0)(1,1)+ in the current
scale setting is used for placing the dots. If there is only one \Largr{$x_{Max},y_{Max}$} %$
defined, then \verb+(0,0)+ is used for the other point.
This area should be greater than the clipping
path to be sure that the dots are placed over the full area. The clipping path can
be everything. If no clipping path is given, then the frame \verb+(0,0)(1,1)+
in user coordinates is used.  The new options are:

\begin{center}
\begin{tabular}{@{}l|l|l@{}}
name & default\\\hline
\Lkeyword{randomPoints} &   \verb|1000| & number of random dots\tabularnewline
\Lkeyword{color} & \false & random color\tabularnewline
\end{tabular}
\end{center}


\begin{LTXexample}[width=0.3\linewidth]
\psset{unit=5cm}
\begin{pspicture}(1,1)
  \psRandom[dotsize=1pt,fillstyle=solid](1,1){\pscircle(0.5,0.5){0.5}}
\end{pspicture}
\begin{pspicture}(1,1)
  \psRandom[dotsize=2pt,randomPoints=5000,color,%
      fillstyle=solid](1,1){\pscircle(0.5,0.5){0.5}}
\end{pspicture}
\end{LTXexample}

\begin{LTXexample}[width=0.4\linewidth]
\psset{unit=5cm}
\begin{pspicture}(1,1)
  \psRandom[randomPoints=200,dotsize=8pt,dotstyle=+]{}
\end{pspicture}
\begin{pspicture}(1.5,1)
  \psRandom[dotsize=5pt,color](0,0)(1.5,0.8){\psellipse(0.75,0.4)(0.75,0.4)}
\end{pspicture}
\end{LTXexample}

\begin{LTXexample}
\psset{unit=2.5cm}
\begin{pspicture}(0,-1)(3,1)
  \psRandom[dotsize=4pt,dotstyle=o,linecolor=blue,fillcolor=red,%
     fillstyle=solid,randomPoints=1000]%
      (0,-1)(3,1){\psplot{0}{3.14}{ x 114 mul sin }}
\end{pspicture}
\end{LTXexample}

\psset{unit=1cm}


\clearpage
 %--------------------------------------------------------------------------------------
\section{\nxLcs{psDice}}
 %--------------------------------------------------------------------------------------
\Lcs{psdice} creates the view of a dice. The number on the dice is the only parameter.
The optional parameters, like the color can be used as usual. The macro is a box of
dimension zero and is placed
at the current point. Use  the \Lcs{rput} macro to place it anywhere. The optional
argument \Lkeyword{unit} can be used to scale the dice. the default size of
the dice $1\mathrm{cm}\times1\mathrm{cm}$.

\begin{center}
\begin{pspicture}(-1,-1)(8,9)
\multido{\iA=1+1}{6}{%
  \rput(\iA,7.5){\Huge\psdice[unit=0.75,linecolor=red!80]{\iA}}
  \rput(! -0.5 7 \iA\space sub){\Huge\psdice[unit=0.75,linecolor=blue!70]{\iA}}%
  \multido{\iB=1+1}{6}{%
    \rput(! \iA\space 7 \iB\space sub){%
      \rnode[c]{p\iA\iB}{\makebox[1em][l]{\strut\psPrintValue[fontscale=12]{\iA\space \iB\space add}}}%
}}}
\ncbox[linearc=0.35,nodesep=0.2,linestyle=dotted]{p11}{p66}
\ncbox[linearc=0.35,nodesep=0.2,linestyle=dashed]{p15}{p51}
\rput{90}(-1.5,3.5){1. dice}
\rput{0}(3.5,8.5){2. dice}
\psline[linewidth=1.5pt](0.25,0.5)(0.25,8)
\psline[linewidth=1.5pt](-1,6.75)(6.5,6.75)
\end{pspicture}
\end{center}

\begin{lstlisting}
\begin{pspicture}(-1,-1)(8,8)
\multido{\iA=1+1}{6}{%
  \rput(\iA,7.5){\Huge\psdice[unit=0.75,linecolor=red!80]{\iA}}
  \rput(! -0.5 7 \iA\space sub){\Huge\psdice[unit=0.75,linecolor=blue!70]{\iA}}%
  \multido{\iB=1+1}{6}{%
    \rput(! \iA\space 7 \iB\space sub){%
      \rnode[c]{p\iA\iB}{\makebox[1em][l]{\strut\psPrintValue[fontscale=12]{\iA\space \iB\space add}}}%
}}}
\ncbox[linearc=0.35,nodesep=0.2,linestyle=dotted]{p11}{p66}
\ncbox[linearc=0.35,nodesep=0.2,linestyle=dashed]{p15}{p51}
\rput{90}(-1.5,3.5){1. dice}
\rput{0}(3.5,8.5){2. dice}
\psline[linewidth=1.5pt](0.25,0.5)(0.25,8)
\psline[linewidth=1.5pt](-1,6.75)(6.5,6.75)
\end{pspicture}
\end{lstlisting}

\clearpage
%--------------------------------------------------------------------------------------
\section{\nxLcs{psFormatInt}}
%--------------------------------------------------------------------------------------
There exist some packages and a lot of code to format an integer like $1\,000\,000$
or $1,234,567$ (in Europe $1.234.567$). But all packages expect a real number as
argument and cannot handle macros as an argument. For this case \LPack{pstricks-add}
has a macro \Lcs{psFormatInt} which can handle both:

\begin{LTXexample}[width=3cm]
\psFormatInt{1234567}\\
\psFormatInt[intSeparator={,}]{1234567}\\
\psFormatInt[intSeparator=.]{1234567}\\
\psFormatInt[intSeparator=$\cdot$]{1234567}\\
\def\temp{965432}
\psFormatInt{\temp}
\end{LTXexample}

With the option \Lkeyword{intSeparator} the symbol can be changed to any any non-number character.


\clearpage

%--------------------------------------------------------------------------------------
\section{\nxLcs{psRelNode} and \nxLcs{psDefPSPNodes}}
%--------------------------------------------------------------------------------------
With these macros it is possible to put a node relative to a given line or given
\Lenv{pspicture}-environment. In the frist case the parameters are
the angle and the length factor:

\begin{BDef}
\Lcs{psRelNode}\Largs{P0}\Largs{P1}\Largb{length factor}\Largb{end node name}\\
\Lcs{psDefPSPNodes}
\end{BDef}

The length factor relates to the distance $\overline{P_0P_1}$ and
the end node name must be a valid nodename and shouldn't contain
any of the special PostScript characters. There are two valid
options:

\begin{tabularx}{\linewidth}{@{} l|l| X @{} }
name & default & meaning\\\hline 
\Lkeyword{angle} & $0$ & angle between the given line $\overline{P_0P_1}$ and the new one
	$\overline{P_0P_{endNode}}$\tabularnewline 
\Lkeyword{trueAngle} & \false & defines whether the angle refers to the seen line or to
the mathematical one, which respect the scaling factors
\Lkeyword{xunit} and \Lkeyword{yunit}.
\end{tabularx}

\begin{LTXexample}[width=7cm]
\begin{pspicture}[showgrid](7,6)
  \pnode(3,3){A}\pnode(4,2){B}
  \psline[nodesep=-3,linewidth=0.5pt](A)(B)
  \multido{\iA=0+30}{12}{%
    \psRelNode[angle=\iA](A)(B){2}{C}%
    \qdisk(C){2pt}
    \uput[0](C){\iA}}
\end{pspicture}
\end{LTXexample}

In the second case the new macro \Lcs{psDefPSPNodes} defines nine nodes that corresponds to
nine particular points (namely bottom left, bottom center,
bottom right, center left, center center, center right, top left,
top center, top right) of the \Lenv{pspicture} box.

\begin{LTXexample}[width=6cm,wide=false]
\begin{pspicture}[showgrid=true](-1,-1)(4,4)
  \psDefPSPNodes
  \psdots(PSPbl)(PSPbc)(PSPbr)
      (PSPcl)(PSPcc)(PSPcr)(PSPtl)(PSPtc)(PSPtr)
  \uput[90](PSPbl){PSPbl} \uput[90](PSPbc){PSPbc}
  \uput[90](PSPbr){PSPbr} \uput[90](PSPcl){PSPcl}
  \uput[90](PSPcc){PSPcc} \uput[90](PSPcr){PSPcr}
  \uput[90](PSPtl){PSPtl} \uput[90](PSPtc){PSPtc}
  \uput[90](PSPtr){PSPtr}
\end{pspicture}
\end{LTXexample}

The name of the nodes are predefined as:

\begin{lstlisting}[style=syntax]
\psset[pst-PSPNodes]{blName=PSPbl,bcName=PSPbc,brName=PSPbr,
  clName=PSPcl,ccName=PSPcc,crName=PSPcr,tlName=PSPtl,tcName=PSPtc,trName=PSPtr}
\end{lstlisting}

and can be modified in the same way.
%I guess you modified the family to have the pstricks-add one so the
%\xkvview would have to be adapted.

%--------------------------------------------------------------------------------------
\section{\nxLcs{psRelLine}}
%--------------------------------------------------------------------------------------
With this macro it is possible to plot lines relative to a given one. Parameter are
the angle and the length factor:

\begin{BDef}
\Lcs{psRelLine}\Largr{P0}\Largr{P1}\Largb{length factor}\Largb{<end node name>}\\
\Lcs{psRelLine}\OptArg{\Largb{arrows}}\Largr{P0}\Largr{P1}\Largb{length factor}\Largb{end node name}\\
\Lcs{psRelLine}\OptArgs\Largr{P0}\Largr{P1}\Largb{length factor}\Largb{end node name}\\
\Lcs{psRelLine}\OptArgs\OptArg{\Largb{arrows}}\Largr{P0}\Largr{P1}\Largb{length factor}\Largb{end node name}
\end{BDef}

The length factor relates to the distance $\overline{P_0P_1}$ and
the end node name must be a valid nodename and shouldn't contain
any of the special PostScript characters. There are two valid
options which are described in the foregoing section for
\Lcs{psRelNode}.

The following two figures show the same, the first one with a scaling different to $1:1$,
this is the reason why the end points are on an ellipse and not on a circle like in the
second figure.

\begin{LTXexample}[width=5cm]
\psset{yunit=2,xunit=1}
\begin{pspicture}(-2,-2)(3,2)
\psgrid[subgriddiv=2,subgriddots=10,gridcolor=lightgray]
\pnode(-1,0){A}\pnode(3,2){B}
\psline[linecolor=red](A)(B)
\psRelLine[linecolor=blue,angle=30](-1,0)(B){0.5}{EndNode}
\qdisk(EndNode){2pt}
\psRelLine[linecolor=blue,angle=-30](A)(B){0.5}{EndNode}
\qdisk(EndNode){2pt}
\psRelLine[linecolor=magenta,angle=90](-1,0)(3,2){0.5}{EndNode}
\qdisk(EndNode){2pt}
\psRelLine[linecolor=magenta,angle=-90](A)(B){0.5}{EndNode}
\qdisk(EndNode){2pt}
\end{pspicture}
\end{LTXexample}

\begin{LTXexample}[width=5cm]
\begin{pspicture}(-2,-2)(3,2)
\psgrid[subgriddiv=2,subgriddots=10,gridcolor=lightgray]
\pnode(-1,0){A}\pnode(3,2){B}
\psline[linecolor=red](A)(B)
\psarc[linestyle=dashed](A){2.23}{-90}{135}
\psRelLine[linecolor=blue,angle=30](-1,0)(B){0.5}{EndNode}
\qdisk(EndNode){2pt}
\psRelLine[linecolor=blue,angle=-30](A)(B){0.5}{EndNode}
\qdisk(EndNode){2pt}
\psRelLine[linecolor=magenta,angle=90](-1,0)(3,2){0.5}{EndNode}
\qdisk(EndNode){2pt}
\psRelLine[linecolor=magenta,angle=-90](A)(B){0.5}{EndNode}
\qdisk(EndNode){2pt}
\end{pspicture}
\end{LTXexample}

\medskip
The following figure has also a different scaling, but has set the
option \Lkeyword{trueAngle}, all angles refer to "what you see".

\begin{LTXexample}[width=6.5cm]
\psset{yunit=2,xunit=1}
\begin{pspicture}(-3,-1)(3,2)\psgrid[subgridcolor=lightgray]
\pnode(-1,0){A}\pnode(3,2){B}
\psline[linecolor=red](A)(B)
\psarc(A){2.83}{-45}{135}
\psRelLine[linecolor=blue,angle=30,trueAngle](A)(B){0.5}{EndNode}
\qdisk(EndNode){2pt}
\psRelLine[linecolor=blue,angle=-30,trueAngle](A)(B){0.5}{EndNode}
\qdisk(EndNode){2pt}
\psRelLine[linecolor=magenta,angle=90,trueAngle](A)(B){0.5}{EndNode}
\qdisk(EndNode){2pt}
\psRelLine[linecolor=magenta,angle=-90,trueAngle](A)(B){0.5}{EndNode}
\qdisk(EndNode){2pt}
\end{pspicture}
\end{LTXexample}

\medskip
Two examples using \verb+\multido+ to show the behaviour of the
options \verb+trueAngle+ and \verb+angle+.

\medskip
\begin{LTXexample}[width=8cm]
\psset{yunit=4,xunit=2}
\begin{pspicture}(-1,0)(3,2)\psgrid[subgridcolor=lightgray]
\pnode(-1,0){A}\pnode(1,1){B}
\psline[linecolor=red](A)(3,2)
\multido{\iA=0+10}{36}{%
  \psRelLine[linecolor=blue,angle=\iA](B)(A){-0.5}{EndNode}
  \qdisk(EndNode){2pt}
}
\end{pspicture}
\end{LTXexample}

\begin{LTXexample}[width=8cm]
\psset{yunit=4,xunit=2}
\begin{pspicture}(-1,0)(3,2)\psgrid[subgridcolor=lightgray]
\pnode(-1,0){A}\pnode(1,1){B}
\psline[linecolor=red](A)(3,2)
\multido{\iA=0+10}{36}{%
  \psRelLine[linecolor=magenta,angle=\iA,trueAngle]{->}(B)(A){-0.5}{EndNode}
}
\end{pspicture}
\end{LTXexample}

\begin{center}
\bgroup
\psset{xunit=0.75\linewidth,yunit=0.75\linewidth,trueAngle}%
\begin{pspicture}(1,0.6)%\psgrid
  \pnode(.3,.35){Vk} \pnode(.375,.35){D} \pnode(0,.4){DST1} \pnode(1,.18){DST2}
  \pnode(0,.1){A1}   \pnode(1,.31){A1}
  { \psset{linewidth=.02,linestyle=dashed,linecolor=gray}%
    \pcline(DST1)(DST2) % <- Druckseitentangente
    \pcline(A2)(A1) % <- Anstr\"omrichtung
    \lput*{:U}{\small Anstr\"omrichtung $v_{\infty}$} }%
  \psIntersectionPoint(A1)(A2)(DST1)(DST2){Hk}
  \pscurve(Hk)(.4,.38)(Vk)(.36,.33)(.5,.32)(Hk)
  \psParallelLine[linecolor=red!75!green,arrows=->,arrowscale=2](Vk)(Hk)(D){.1}{FtE}
  \psRelLine[linecolor=red!75!green,arrows=->,arrowscale=2,angle=90](D)(FtE){4}{Fn}% why "4"?
  \psParallelLine[linestyle=dashed](D)(FtE)(Fn){.1}{Fnr1}
  \psRelLine[linestyle=dashed,angle=90](FtE)(D){-4}{Fnr2} % why "-4"?
  \psline[linewidth=1.5pt,arrows=->,arrowscale=2](D)(Fnr2)
  \psIntersectionPoint(D)([nodesep=2]D)(Fnr1)([offset=-4]Fnr1){Fh}
  \psIntersectionPoint(D)([offset=2]D)(Fnr1)([nodesep=4]Fnr1){Fv}
  \psline[linecolor=blue,arrows=->,arrowscale=2](D)(Fh)
  \psline[linecolor=blue,arrows=->,arrowscale=2](D)(Fv)
  \psline[linestyle=dotted](Fh)(Fnr1)  \psline[linestyle=dotted](Fv)(Fnr1)
  \uput{.1}[0](Fh){\blue $F_{H}$}   \uput{.1}[180](Fv){\blue $F_{V}$}
  \uput{.1}[-45](Fnr1){$F_{R}$}     \uput{.1}[90](Fn){\color{red!75!green}$F_{N}$}
  \uput{.25}[-90](FtE){\color{red!75!green}$F_{T}$}
\end{pspicture}
\egroup
\end{center}
\begin{lstlisting}
\psset{xunit=0.75\linewidth,yunit=0.75\linewidth,trueAngle}%
\end{center}
\begin{pspicture}(1,0.6)%\psgrid
  \pnode(.3,.35){Vk} \pnode(.375,.35){D} \pnode(0,.4){DST1} \pnode(1,.18){DST2}
  \pnode(0,.1){A1}   \pnode(1,.31){A1}
  { \psset{linewidth=.02,linestyle=dashed,linecolor=gray}%
    \pcline(DST1)(DST2) % <- Druckseitentangente
    \pcline(A2)(A1) % <- Anstr"omrichtung
    \lput*{:U}{\small Anstr"omrichtung $v_{\infty}$} }%
  \psIntersectionPoint(A1)(A2)(DST1)(DST2){Hk}
  \pscurve(Hk)(.4,.38)(Vk)(.36,.33)(.5,.32)(Hk)
  \psParallelLine[linecolor=red!75!green,arrows=->,arrowscale=2](Vk)(Hk)(D){.1}{FtE}
  \psRelLine[linecolor=red!75!green,arrows=->,arrowscale=2,angle=90](D)(FtE){4}{Fn}% why "4"?
  \psParallelLine[linestyle=dashed](D)(FtE)(Fn){.1}{Fnr1}
  \psRelLine[linestyle=dashed,angle=90](FtE)(D){-4}{Fnr2} % why "-4"?
  \psline[linewidth=1.5pt,arrows=->,arrowscale=2](D)(Fnr2)
  \psIntersectionPoint(D)([nodesep=2]D)(Fnr1)([offset=-4]Fnr1){Fh}
  \psIntersectionPoint(D)([offset=2]D)(Fnr1)([nodesep=4]Fnr1){Fv}
  \psline[linecolor=blue,arrows=->,arrowscale=2](D)(Fh)
  \psline[linecolor=blue,arrows=->,arrowscale=2](D)(Fv)
  \psline[linestyle=dotted](Fh)(Fnr1)  \psline[linestyle=dotted](Fv)(Fnr1)
  \uput{.1}[0](Fh){\blue $F_{H}$}   \uput{.1}[180](Fv){\blue $F_{V}$}
  \uput{.1}[-45](Fnr1){$F_{R}$}     \uput{.1}[90](Fn){\color{red!75!green}$F_{N}$}
  \uput{.25}[-90](FtE){\color{red!75!green}$F_{T}$}
\end{pspicture}
\end{lstlisting}


%--------------------------------------------------------------------------------------
\section{\nxLcs{psParallelLine}}
%--------------------------------------------------------------------------------------
With this macro it is possible to plot lines relative to a given one, which is parallel.
There is no special parameter here.

\begin{lstlisting}[style=syntax]
\psParallelLine(<P0>)(<P1>)(<P2>){<length>}{<end node name>}
\psParallelLine{<arrows>}(<P0>)(<P1>)(<P2>){<length>}{<end node name>}
\psParallelLine[<options>](<P0>)(<P1>)(<P2>){<length>}{<end node name>}
\psParallelLine[<options>]{<arrows>}(<P0>)(<P1>)(<P2>){<length>}{<end node name>}
\end{lstlisting}

The line starts at $P_2$, is parallel to $\overline{P_0P_1}$ and
the length of this parallel line depends on the length factor. The
end node name must be a valid nodename and shouldn't contain any
of the special PostScript characters.

\begin{LTXexample}
\begin{pspicture*}(-5,-4)(5,3.5)
  \psgrid[subgriddiv=0,griddots=5]
  \pnode(2,-2){FF}\qdisk(FF){1.5pt}
  \pnode(-5,5){A}\pnode(0,0){O}
  \multido{\nCountA=-2.4+0.4}{9}{%
    \psParallelLine[linecolor=red](O)(A)(0,\nCountA){9}{P1}
    \psline[linecolor=red](0,\nCountA)(FF)
    \psRelLine[linecolor=red](0,\nCountA)(FF){9}{P2}
  }
  \psline[linecolor=blue](A)(FF)
  \psRelLine[linecolor=blue](A)(FF){5}{END1}
  \psline[linewidth=2pt,arrows=->](2,0)(FF)
\end{pspicture*}
\end{LTXexample}


%--------------------------------------------------------------------------------------
\section{\nxLcs{psIntersectionPoint}}
%--------------------------------------------------------------------------------------
This macro calculates the intersection point of two lines, given by the four coordinates.
There is no special parameter here.
\begin{lstlisting}[style=syntax]
\psIntersectionPoint(<P0>)(<P1>)(<P2>)(<P3>){<node name>}
\end{lstlisting}

\begin{LTXexample}[width=5.5cm]
\psset{unit=0.5cm}
\begin{pspicture}(-5,-4)(5,5)
  \psaxes[labelFontSize=\scriptstyle,
    dx=2,Dx=2,dy=2,Dy=2]{->}(0,0)(-5,-4)(5,5)
  \psline[linecolor=red,linewidth=2pt](-5,-1)(5,5)
  \psline[linecolor=blue,linewidth=2pt](-5,3)(5,-4)
  \qdisk(-5,-1){2pt}\uput[-90](-5,-1){A}
  \qdisk(5,5){2pt}\uput[-90](5,5){B}
  \qdisk(-5,3){2pt}\uput[-90](-5,3){C}
  \qdisk(5,-4){2pt}\uput[-90](5,-4){D}
  \psIntersectionPoint(-5,-1)(5,5)(-5,3)(5,-4){IP}
  \qdisk(IP){3pt}\uput{0.3}[90](IP){IP}
  \psline[linestyle=dashed](IP|0,0)(IP)(0,0|IP)
\end{pspicture}
\end{LTXexample}

\clearpage

%--------------------------------------------------------------------------------------
\section[\nxLcs{psCancel}]{\nxLcs{psCancel}\footnotemark}
%--------------------------------------------------------------------------------------
\footnotetext{Thanks to by Stefano Baroni} This macro works like
the \Lcs{cancel} macro from the package of the same name but it
allows as argument any contents, not only letters but also a
complex graphic.

\begin{BDef}
\LcsStar{psCancel}\OptArgs\Largb{contents}%
\end{BDef}

All optional arguments for lines and boxes are valid and can be
used in the usual way. The star option fills the underlying box
rectangle with the linecolor. This can be transparent if
\Lkeyword{opacity} is set to a value less than 1. This can be used
in presentation to strike out words, equations, and graphic
objects. Lines can also be transparent when the option
\Lkeyword{strokeopacity} is used.

\begingroup
\psCancel{A} \psCancel[linecolor=red]{Tikz :-)} \quad
\psCancel[linecolor=blue,doubleline=true]{%
  \readdata{\data}{demo1.data}
  \psset{shift=*,xAxisLabel=x-Axis,yAxisLabel=y-Axis,llx=-13mm,lly=-7mm,
      xAxisLabelPos={c,-1},yAxisLabelPos={-7,c}}
  \pstScalePoints(1,0.00000001){}{}
  \begin{psgraph}[axesstyle=frame,xticksize=0 7.5,yticksize=0 25,subticksize=1,
     ylabelFactor=\cdot 10^8,Dx=5,Dy=1,xsubticks=2](0,0)(25,7.5){5.5cm}{5cm}
  \listplot[linecolor=red, linewidth=2pt, showpoints=true]{\data}
  \end{psgraph}} \qquad% end of Cancel
\psCancel[linewidth=3pt,linecolor=red,
    strokeopacity=0.5]{\tabular[b]{c}first line\\second line\endtabular}\quad
\psCancel*[linecolor=red!50,opacity=0.5]{\tabular[b]{c}first line\\second line\endtabular}
\quad
\psCancel*[linecolor=blue!30,opacity=0.5]{%
  \readdata{\data}{demo1.data}
  \psset{shift=*,xAxisLabel=x-Axis,yAxisLabel=y-Axis,llx=-15mm,lly=-7mm,urx=1mm,
      xAxisLabelPos={c,-1},yAxisLabelPos={-7,c}}
  \pstScalePoints(1,0.00000001){}{}
  \begin{psgraph}[axesstyle=frame,xticksize=0 7.5,yticksize=0 25,subticksize=1,
     ylabelFactor=\cdot 10^8,Dx=5,Dy=1,xsubticks=2](0,0)(25,7.5){5.5cm}{5cm}
  \listplot[linecolor=red, linewidth=2pt, showpoints=true]{\data}
  \end{psgraph}} \quad% end of Cancel
\psCancel[linewidth=4pt,strokeopacity=0.5]{\parbox{8cm}{\[
  \binom{x_R}{y_R} = \underbrace{r\vphantom{\binom{A}{B}}}_{\text{Scaling}}\cdot
    \underbrace{\begin{pmatrix}
        \sin\gamma & -\cos\gamma \\
      \cos \gamma & \sin \gamma \\
      \end{pmatrix}}_{\text{Rotation}} \binom{x_K}{y_K} +
  \underbrace{\binom{t_x}{t_y}}_{\text{Translation}} \]} }% end of psCancel
\endgroup

\bigskip
\begin{lstlisting}
\psCancel{A} \psCancel[linecolor=red]{Tikz :-)} \quad
\psCancel[linecolor=blue,doubleline=true]{%
  \readdata{\data}{demo1.data}
  \psset{shift=*,xAxisLabel=x-Axis,yAxisLabel=y-Axis,llx=-13mm,lly=-7mm,
      xAxisLabelPos={c,-1},yAxisLabelPos={-7,c}}
  \pstScalePoints(1,0.00000001){}{}
  \begin{psgraph}[axesstyle=frame,xticksize=0 7.5,yticksize=0 25,subticksize=1,
     ylabelFactor=\cdot 10^8,Dx=5,Dy=1,xsubticks=2](0,0)(25,7.5){5.5cm}{5cm}
  \listplot[linecolor=red, linewidth=2pt, showpoints=true]{\data}
  \end{psgraph}} \qquad% end of Cancel
\psCancel[linewidth=3pt,linecolor=red,
    strokeopacity=0.5]{\tabular[b]{c}first line\\second line\endtabular}\quad
\psCancel*[linecolor=red!50,opacity=0.5]{\tabular[b]{c}first line\\second line\endtabular}
\quad
\psCancel*[linecolor=blue!30,opacity=0.5]{%
  \readdata{\data}{demo1.data}
  \psset{shift=*,xAxisLabel=x-Axis,yAxisLabel=y-Axis,llx=-15mm,lly=-7mm,urx=1mm,
      xAxisLabelPos={c,-1},yAxisLabelPos={-7,c}}
  \pstScalePoints(1,0.00000001){}{}
  \begin{psgraph}[axesstyle=frame,xticksize=0 7.5,yticksize=0 25,subticksize=1,
     ylabelFactor=\cdot 10^8,Dx=5,Dy=1,xsubticks=2](0,0)(25,7.5){5.5cm}{5cm}
  \listplot[linecolor=red, linewidth=2pt, showpoints=true]{\data}
  \end{psgraph}} \quad% end of Cancel
\psCancel[linewidth=4pt,strokeopacity=0.5]{\parbox{8cm}{\[
  \binom{x_R}{y_R} = \underbrace{r\vphantom{\binom{A}{B}}}_{\text{Scaling}}\cdot
    \underbrace{\begin{pmatrix}
        \sin\gamma & -\cos\gamma \\
      \cos \gamma & \sin \gamma \\
      \end{pmatrix}}_{\text{Rotation}} \binom{x_K}{y_K} +
  \underbrace{\binom{t_x}{t_y}}_{\text{Translation}} \]} }% end of psCancel
\end{lstlisting}

The optional argument \Lkeyword{cancelType} allows to define the lines for the non star version.
Possible values are \Lkeyval{x} for a cross, \Lkeyval{s} for a slash, and \Lkeyval{b}
for a backslash. It is also possible to use the long words for the \Lkeyval{slash} and the \Lkeyval{backslash}.
An empty value is always assumed as a \Lkeyval{x}.

\begin{LTXexample}[pos=t,wide]
\psset{linewidth=3pt,strokeopacity=0.4}
\psCancel{\tabular[b]{c}first line\\second line\endtabular}   \quad
\psCancel[cancelType=x]{\tabular[b]{c}first line\\second line\endtabular}\quad
\psCancel[cancelType=s]{\tabular[b]{c}first line\\second line\endtabular}\quad
\psCancel[cancelType=b]{\tabular[b]{c}first line\\second line\endtabular}
\end{LTXexample}

\clearpage
%--------------------------------------------------------------------------------------
\section{\nxLcs{psStep}}
%--------------------------------------------------------------------------------------
\Lcs{psStep} calculates a step function for the upper or lower
sum or the max/min of the \Index{Riemann} integral definition of a given
function. The available option is

\Lkeyset{StepType=lower}|\Lkeyval{upper}|\Lkeyval{Riemann}|\Lkeyval{infimum}|\Lkeyval{supremum} or alternative
\Lkeyset{StepType=l}|\Lkeyval{u}|\Lkeyval{R}|\Lkeyval{i}|\Lkeyval{s}

with \Lkeyword{lower} as the default setting. The syntax of the function is

\begin{BDef}
\Lcs{psStep}\OptArgs\Largr(x1,x2)\Largb{n}\Largb{function}
\end{BDef}


(x1,x2) is the given interval for the step wise calculated
function, n is the number of the rectangles and \Larg{function} is
the mathematical function in postfix or algebraic=true notation (with
\Lkeyset{algebraic=true}).

\begin{LTXexample}[pos=t,preset=\centering]
\begin{pspicture}(-0.5,-0.5)(10,3)
 \psaxes[labelFontSize=\scriptstyle]{->}(10,3)
 \psplot[plotpoints=100,linewidth=1.5pt,algebraic=true]{0}{10}{sqrt(x)}
 \psStep[linecolor=magenta,StepType=upper,fillstyle=hlines](0,9){9}{x sqrt}
 \psStep[linecolor=blue,fillstyle=vlines](0,9){9}{x sqrt }
\end{pspicture}
\end{LTXexample}

\begin{LTXexample}[pos=t,preset=\centering]
\psset{plotpoints=200}
\begin{pspicture}(-0.5,-2.25)(10,3)
  \psaxes[labelFontSize=\scriptstyle]{->}(0,0)(0,-2.25)(10,3)
 \psplot[linewidth=1.5pt,algebraic=true]{0}{10}{sqrt(x)*sin(x)}
 \psStep[algebraic=true,linecolor=magenta,StepType=upper](0,9){20}{sqrt(x)*sin(x)}
 \psStep[linecolor=blue,linestyle=dashed](0,9){20}{x sqrt x RadtoDeg sin mul}
\end{pspicture}
\end{LTXexample}

\begin{LTXexample}[pos=t,preset=\centering]
\psset{yunit=1.25cm,plotpoints=200}
\begin{pspicture}(-0.5,-1.5)(10,1.5)
 \psaxes[labelFontSize=\scriptstyle]{->}(0,0)(0,-1.5)(10,1.5)
 \psStep[algebraic=true,StepType=Riemann,fillstyle=solid,fillcolor=black!10](0,10){50}%
    {sqrt(x)*cos(x)*sin(x)}
 \psplot[linewidth=1.5pt,algebraic=true]{0}{10}{sqrt(x)*cos(x)*sin(x)}
\end{pspicture}
\end{LTXexample}


\begin{LTXexample}[pos=t,preset=\centering]
\psset{yunit=1.25cm,plotpoints=200}
\begin{pspicture}(-0.5,-1.5)(10,1.5)
 \psaxes[labelFontSize=\scriptstyle]{->}(0,0)(0,-1.5)(10,1.5)
 \psStep[algebraic=true,StepType=infimum,fillstyle=solid,fillcolor=black!10](0,10){50}%
    {sqrt(x)*cos(x)*sin(x)}
 \psplot[linewidth=1.5pt,algebraic=true]{0}{10}{sqrt(x)*cos(x)*sin(x)}
\end{pspicture}
\end{LTXexample}

\begin{LTXexample}[pos=t,preset=\centering]
\psset{yunit=1.25cm,plotpoints=200}
\begin{pspicture}(-0.5,-1.5)(10,1.5)
 \psaxes[labelFontSize=\scriptstyle]{->}(0,0)(0,-1.5)(10,1.5)
 \psStep[algebraic=true,StepType=supremum,fillstyle=solid,fillcolor=black!10](0,10){50}%
    {sqrt(x)*cos(x)*sin(x)}
 \psplot[linewidth=1.5pt,algebraic=true]{0}{10}{sqrt(x)*cos(x)*sin(x)}
\end{pspicture}
\end{LTXexample}

\begin{LTXexample}[pos=t,preset=\centering]
\psset{unit=1.5cm,plotpoints=200}
\begin{pspicture}[plotpoints=200](-0.5,-3)(10,2.5)
  \psStep[algebraic=true,fillstyle=solid,fillcolor=yellow](0.001,9.5){40}{2*sqrt(x)*cos(ln(x))*sin(x)}
  \psStep[algebraic=true,StepType=Riemann,fillstyle=solid,fillcolor=blue](0.001,9.5){40}{2*sqrt(x)*cos(ln(x))*sin(x)}
  \psaxes[labelFontSize=\scriptstyle]{->}(0,0)(0,-2.75)(10,2.5)
  \psplot[algebraic=true,linecolor=white]{0.001}{9.75}{2*sqrt(x)*cos(ln(x))*sin(x)}
  \uput[90](6,1.2){$f(x)=2\cdot\sqrt{x}\cdot\cos{(\ln{x})}\cdot\sin{x}$}
\end{pspicture}
\end{LTXexample}

\clearpage
%--------------------------------------------------------------------------------------

\section{Tangent lines}
There are two macros for plotting a tangent line or the tangent normal line.
The first one is \Lcs{psTangentLine} which expects three pairs of coordinates,
a $x$ and a $dx$ value. The second one is \Lcs{psplotTangent} which expects 
a function for the curve. \xLkeyword{Tnormal}

\subsection{\nxLcs{psTangentLine} and option \nxLkeyword{Tnormal}}

\begin{BDef}
\Lcs{psTangentLine}\OptArgs\coord1\coord2\coord3\Largb{x}\Largb{dx}
\end{BDef}

\begin{LTXexample}[width=0.45\linewidth,wide]
\psset{unit=2}
\begin{pspicture}[showgrid=true](1,-1)(4,1)
  \pscurve[showpoints=true]
    (2.1,-0.2)(2.5,0.2)(3.2,0.235)(3.8,-0.2)
  \psTangentLine[Tnormal,arrows=->,
    linecolor=red](2.5,0.2)(3.2,0.235)%
      (3.8,-0.2){3}{0.1}
  \psTangentLine[arrows=<->,
    linecolor=blue](2.5,0.2)(3.2,0.235)%
      (3.8,-0.2){3}{0.5}
\end{pspicture}
\end{LTXexample}

In special cases one has to use \Lkeyword{curvature}\verb+=1 1 1+ for the macro \Lcs{pscurve}
to get the same equation for the curve as \Lcs{psplotTangentLine} does.

\begin{LTXexample}[pos=t,preset=\centering,wide]
\psset{unit=2}
\begin{pspicture}[showgrid=true](2,-1)(6,2)
\pscurve[showpoints=true,
  curvature=1 1 1](2.1,-0.2)(2.5,0.2)(3.2,0.235)(5.8,2)
\pscurve[showpoints=true,linecolor=green,
  curvature=1 1 1](2.5,0.2)(3.2,0.235)(5.8,2)
\psTangentLine[Tnormal,arrows=->,linecolor=red](2.5,0.2)(3.2,0.235)(5.8,2){4.6}{0.6}
\psTangentLine[arrows=<->,linecolor=blue](2.5,0.2)(3.2,0.235)(5.8,2){4.5}{0.6}
\end{pspicture}
\end{LTXexample}


The end points are saved as nodes \verb=OCurve=, \verb=ETangent=, and \verb=ENormal=. They can
be used in the default ways for nodes:

\begin{LTXexample}[pos=t,preset=\centering,wide]
\psset{unit=4,arrowscale=2}
\begin{pspicture}(0.1,-0.1)(4,1)
\pscurve[showpoints=true](2.1,-0.2)(2.5,0.2)(3.2,0.4)(3.8,-0.2)
\psTangentLine[Tnormal,arrows=->,linecolor=red](2.5,0.2)(3.2,0.4)(3.8,-0.2){3.5}{0.5}
\psTangentLine[arrows=->,linecolor=blue](2.5,0.2)(3.2,0.4)(3.8,-0.2){3.5}{0.5}
\pcline[linestyle=dashed]{->}(OCurve)(ETangent|OCurve)\naput{$v_x$}
\pcline[linestyle=dashed]{->}(ETangent|OCurve)(ETangent)\naput{$v_y$}% double coordinate (x,y|x,y)
\end{pspicture}
\end{LTXexample}


\subsection{\nxLcs{psplotTangent} and option \nxLkeyword{Tnormal}}
%--------------------------------------------------------------------------------------
There is an additional option, named \Lkeyword{Derive} for an
alternative function (see following example) to calculate the
slope of the tangent. This will be in general the first
derivative, but can also be any other function. If this option is
different to to the default value \Lkeyset{Derive=default}, then this
function is taken to calculate the slope. For the other cases,
\LPack{pstricks-add} builds a secant with -0.00005<x<0.00005,
calculates the slope and takes this for the tangent. This may be
problematic in some cases of special functions or $x$ values, then
it may be appropriate to use the Derive option.

\begin{BDef}
\LcsStar{psplotTangent}\OptArgs\Largb{x}\Largb{dx}\Largb{function}
\end{BDef}



The macro expects three parameters:

\begin{description}
\item[$x$]: the $x$ value of the function for which the tangent should be calculated
\item[$dx$]: the $dx$ to both sides of the $x$ value
\item[$f(x)$]: the function in infix (with option \Lkeyword{algebraic}) or the default
postfix (PostScript) notation
\end{description}

The following examples show the use of the algebraic=true option together with the Derive option.
Remember that using the \Lkeyword{algebraic} option implies that the angles have to be in the
radian unit!

\begin{center}
\bgroup
\def\F{x RadtoDeg dup dup cos exch 2 mul cos add exch 3 mul cos add}
\def\Fp{x RadtoDeg dup dup sin exch 2 mul sin 2 mul add exch 3 mul sin 3 mul add neg}
\psset{plotpoints=1001}
\begin{pspicture}(-7.5,-2.5)(7.5,4)%X\psgrid
  \psaxes{->}(0,0)(-7.5,-2)(7.5,3.5)
  \psplot[linewidth=3\pslinewidth]{-7}{7}{\F}
  \psset{linecolor=red, arrows=<->, arrowscale=2}
  \multido{\n=-7+1}{8}{\psplotTangent{\n}{1}{\F}}
  \psset{linecolor=magenta, arrows=<->, arrowscale=2}%
  \multido{\n=0+1}{8}{\psplotTangent[linecolor=blue, Derive=\Fp]{\n}{1}{\F}}
\end{pspicture}
\egroup
\end{center}

\begin{lstlisting}
\def\F{x RadtoDeg dup dup cos exch 2 mul cos add exch 3 mul cos add}
\def\Fp{x RadtoDeg dup dup sin exch 2 mul sin 2 mul add exch 3 mul sin 3 mul add neg}
\psset{plotpoints=1001}
\begin{pspicture}(-7.5,-2.5)(7.5,4)%X\psgrid
  \psaxes{->}(0,0)(-7.5,-2)(7.5,3.5)
  \psplot[linewidth=3\pslinewidth]{-7}{7}{\F}
  \psset{linecolor=red, arrows=<->, arrowscale=2}
  \multido{\n=-7+1}{8}{\psplotTangent{\n}{1}{\F}}
  \psset{linecolor=magenta, arrows=<->, arrowscale=2}%
  \multido{\n=0+1}{8}{\psplotTangent[linecolor=blue, §\ON§Derive=\Fp§\OFF§]{\n}{1}{\F}}
\end{pspicture}
\end{lstlisting}

The star version plots only the tangent line in the positive $x$-direction:

\begin{center}
\bgroup
\def\Falg{cos(x)+cos(2*x)+cos(3*x)}   \def\Fpalg{-sin(x)-2*sin(2*x)-3*sin(3*x)}
\begin{pspicture}(-7.5,-2.5)(7.5,4)%\psgrid
  \psaxes{->}(0,0)(-7.5,-2)(7.5,3.5)
  \psplot[linewidth=1.5pt,algebraic=true,plotpoints=500]{-7.5}{7.5}{\Falg}
  \multido{\n=-7+1}{8}{\psplotTangent*[linecolor=red,arrows=->,arrowscale=2,algebraic=true]{\n}{1}{\Falg}}
  \multido{\n=0+1}{8}{\psplotTangent*[linecolor=magenta,%
     arrows=->,arrowscale=2,algebraic=true,Derive={\Fpalg}]{\n}{1}{\Falg}}
\end{pspicture}
\egroup
\end{center}

\begin{lstlisting}
\def\Falg{cos(x)+cos(2*x)+cos(3*x)}   \def\Fpalg{-sin(x)-2*sin(2*x)-3*sin(3*x)}
\begin{pspicture}(-7.5,-2.5)(7.5,4)%\psgrid
  \psaxes{->}(0,0)(-7.5,-2)(7.5,3.5)
  \psplot[linewidth=1.5pt,algebraic=true,plotpoints=500]{-7.5}{7.5}{\Falg}
  \multido{\n=-7+1}{8}{\psplotTangent*[linecolor=red,arrows=->,arrowscale=2,algebraic=true]{\n}{1}{\Falg}}
  \multido{\n=0+1}{8}{\psplotTangent*[linecolor=magenta,%
     arrows=->,arrowscale=2,algebraic=true,Derive={\Fpalg}]{\n}{1}{\Falg}}
\end{pspicture}
\end{lstlisting}

The next example shows the use of the \Lkeyword{Derive} option to draw
the perpendicular line to the tangent.

\begin{LTXexample}[width=8cm,wide]
\begin{pspicture}(-0.5,-0.5)(7.25,7.25)
  \def\Func{10 x div}
  \psaxes[arrowscale=1.5]{->}(7,7)
  \psplot[linewidth=2pt,algebraic=true]{1.5}{5}{10/x}
  \psplotTangent[linewidth=.5\pslinewidth,linecolor=red,algebraic=true]{3}{2}{10/x}
  \psplotTangent[linewidth=.5\pslinewidth,linecolor=blue,algebraic=true,Derive=(x*x)/10]{3}{2}{10/x}
  \psline[linestyle=dashed](!0 /x 3 def \Func)(!3 /x 3 def \Func)(3,0)
\end{pspicture}
\end{LTXexample}

By setting the optional argument \Lkeyword{Tnormal} one can plot the
normal of the tangent line. It always starts at the given point.

\begin{LTXexample}[width=8cm,wide]
\begin{pspicture}(-0.5,-0.5)(7.25,7.25)
  \def\Func{10 x div}
  \psaxes[arrowscale=1.5]{->}(7,7)
  \psplot[linewidth=2pt]{1.5}{5}{\Func}
  \psplotTangent[linewidth=1.5\pslinewidth,linecolor=red]{3}{2}{\Func}
  \psplotTangent[linewidth=1.5\pslinewidth,linecolor=blue,Tnormal]{3}{2}{\Func}
  \psline[linestyle=dashed](!0 /x 3 def \Func)(!3 /x 3 def \Func)(3,0)
\end{pspicture}
\end{LTXexample}


Let's work with the classical \Index{cardioid}: $r=2(1+\cos(\theta))$ and
$\displaystyle \frac{d r}{d\theta}=-2\sin(\theta)$. The \Lkeyword{Derive}
option always expects the $\frac{d r}{d\theta}$ value and uses
internally the equation for the derivative of implicitly defined
functions:

\[
\frac{dy}{dx}=\frac{r^\prime\cdot\sin\theta + x}{r^\prime\cdot\cos\theta - y}
\]
where $x=r\cdot\cos\theta$ and $y=r\cdot\sin\theta$


\begin{LTXexample}[width=6cm,wide]
\begin{pspicture}(-1,-3)(5,3)%\psgrid[subgridcolor=lightgray]
  \psaxes{->}(0,0)(-1,-3)(5,3)
  \psplot[polarplot,linewidth=3\pslinewidth,linecolor=blue,%
     plotpoints=500]{0}{360}{1 x cos add 2 mul}
\end{pspicture}
\end{LTXexample}

\psset{algebraic=false}
\begin{LTXexample}[width=6cm,wide]
\begin{pspicture}(-1,-3)(5,3)%\psgrid[subgridcolor=lightgray]
  \psaxes{->}(0,0)(-1,-3)(5,3)
  \psplot[polarplot,linewidth=3\pslinewidth,linecolor=blue,plotpoints=500]{0}{360}{1 x cos add 2 mul}
  \multido{\n=0+36}{10}{%
     \psplotTangent[polarplot,linecolor=red,arrows=<->]{\n}{1.5}{1 x cos add 2 mul} }
\end{pspicture}
\end{LTXexample}

\begin{LTXexample}[width=6cm,wide]
\begin{pspicture}(-1,-3)(5,3)%\psgrid[subgridcolor=lightgray]
  \psaxes{->}(0,0)(-1,-3)(5,3)
  \psplot[polarplot,linewidth=3\pslinewidth,linecolor=blue,algebraic=true,plotpoints=500]{0}{6.289}{2*(1+cos(x))}
  \multido{\r=0.000+0.314}{21}{%
     \psplotTangent[polarplot,Derive=-2*sin(x),algebraic=true,linecolor=red,arrows=<->]{\r}{1.5}{2*(1+cos(x))} }
\end{pspicture}
\end{LTXexample}


Let's work with a \Index{Lissajou curve}:
 $\displaystyle\left\{\begin{array}{l}x=3.5\cos(2t)\\y=3.5\sin(6t)\end{array}\right.$
whose derivative is :
 $\displaystyle\left\{\begin{array}{l}x=-7\sin(2t)\\y=21\cos(6t)\end{array}\right.$

The parameter must be the letter $t$ instead of $x$ and when using
the \Lkeyword{algebraic=true} option you must separate the two equations by
a \Lnotation{|} (see example).

\begin{LTXexample}[pos=t,wide]
\def\Lissa{t dup 2 RadtoDeg mul cos 3.5 mul exch 6 mul RadtoDeg sin 3.5 mul}%
\psset{yunit=0.6}
\begin{pspicture}(-4,-4)(4,6)
  \parametricplot[plotpoints=500,linewidth=3\pslinewidth]{0}{3.141592}{\Lissa}
  \multido{\r=0.000+0.314}{11}{%
    \psplotTangent[linecolor=red,arrows=<->]{\r}{1.5}{\Lissa} }
  \multido{\r=0.157+0.314}{11}{%
    \psplotTangent[linecolor=blue,arrows=<->]{\r}{1.5}{\Lissa} }
\end{pspicture}\hfill%
\def\LissaAlg{3.5*cos(2*t)|3.5*sin(6*t)}  \def\LissaAlgDer{-7*sin(2*t)|21*cos(6*t)}%
\begin{pspicture}(-4,-4)(4,6)
  \parametricplot[algebraic=true,plotpoints=500,linewidth=3\pslinewidth]{0}{3.141592}{\LissaAlg}
  \multido{\r=0.000+0.314}{11}{%
    \psplotTangent[algebraic=true,linecolor=red,arrows=<->]{\r}{1.5}{\LissaAlg} }
  \multido{\r=0.157+0.314}{11}{%
    \psplotTangent[algebraic=true,linecolor=blue,arrows=<->,%
       Derive=\LissaAlgDer]{\r}{1.5}{\LissaAlg} }
\end{pspicture}
\end{LTXexample}


\clearpage
\section{Successive derivatives of a function}

The new PostScript function \Lps{Derive} has been added for
plotting successive derivatives of a function. It must be used
with the \Lkeyword{algebraic=true} option. This function has two arguments:

\begin{enumerate}
\item a positive integer which defines the order of the derivative; obviously $0$ means the
  function itself!
\item a function of variable $x$ which can be any function using common operators,
\end{enumerate}

Do not think that the derivative is approximated, the internal PostScript engine will
compute the real derivative using a formal derivative engine.

The following diagram contains the plot of the polynomial:

\[ f(x)=\sum_{i=0}^{14}\frac{(-1)^{i}x^{2i}}{i!}=1-\frac{x^2}{2}+\frac{x^4}{4!}-\frac{x^6}{6!}+\frac{x^8}{8!}-
          \frac{x^{10}}{10!}+\frac{x^{12}}{12!}-\frac{x^{14}}{14!}\]

and of its first 15 derivatives. It is the sequence definition of
the cosine.


\begin{LTXexample}[pos=t,wide,preset=\centering]
\psset{unit=2}
\def\getColor#1{\ifcase#1 Tan\or RedOrange\or magenta\or yellow\or green\or Orange\or blue\or
  DarkOrchid\or BrickRed\or Rhodamine\or OliveGreen\or Goldenrod\or Mahogany\or
  OrangeRed\or CarnationPink\or RoyalPurple\or Lavender\fi}
\begin{pspicture}[showgrid=true](0,-1.2)(7,1.5)
  \psclip{\psframe[linestyle=none](0,-1.1)(7,1.1)}
  \multido{\in=0+1}{16}{%
     \psplot[linewidth=1pt,algebraic=true,linecolor=\getColor{\in}]{0}{7}
      {Derive(\in,1-x^2/2+x^4/24-x^6/720+x^8/40320-x^10/3628800+x^12/479001600-x^14/87178291200)}}
  \endpsclip
\end{pspicture}
\end{LTXexample}

\begin{LTXexample}[width=3.5cm]
\begin{pspicture}[shift=-2.5,showgrid=true,linewidth=1pt](0,-2)(3,3)
  \psplot[algebraic=true]{.001}{3}{x*ln(x)}  % f(x)
  \psplot[algebraic=true,linecolor=red]{.05}{3}{Derive(1,x*ln(x))} % f'(x)=1+ln(x)
\end{pspicture}
\end{LTXexample}


\clearpage
\section{Variable step for plotting a curve}
\subsection{Theory}

As you know with the \Lcs{psplot} macro, the curve is plotted
using a piece-wise linear curve. The step is given by the
parameter \Lkeyword{plotpoints}. For each step between $x_i$ and
$x_{i+1}$, the area defined between the curve and its
approximation (a segment) is majored by this formula :

\begin{minipage}[m]{.5\linewidth}
\[|\varepsilon|\le\frac{M_2(f)(x_{i+1}-x_i)^3}{12}\]

$M_2(f)$ is a majorant of the second derivative of $f$ in the interval $[x_i;x_{i+1}]$.
\end{minipage}
{\psset{unit=1cm, showpoints=false}
\begin{pspicture}[shift=-2,showgrid=true](0,-1)(6,3)
  \pscurve(0,0)(1,1)(3,2.2)(5,2)(6,1)\psline(1,1)(5,2)
  \psline(.5,0)(5.5,0)\psline(1,0)(1,1)\psline(5,0)(5,2)
  \rput[t](1,-.1){$x_n$}\rput[t](5,-.1){$x_{n+1}$}
  \psclip{\pscustom{\psecurve(0,0)(1,1)(3,2.2)(5,2)(6,1)\psline(5,2)}}
    \psframe[fillstyle=solid, fillcolor=gray](0,0)(5,5)
  \endpsclip
  \rput*(3,1.8){$\varepsilon$}
\end{pspicture}}



The parameter \Lkeyword{VarStep} (\false\ by default) activates
the variable step algorithm. It is set to a tolerance defined by
the parameter \Lkeyword{VarStepEpsilon} (\Lkeyval{default} by default,
accept real value). If this parameter is not set by the user, then
it is automatically computed using the default first step given by
the parameter \Lkeyword{plotpoints}. Then, for each step, $f''(x_n)$
and $f''(x_{n+1})$ are computed and the smaller is used as
$M_2(f)$, and then the step is approximated. This means that the
step is constant for second order polynomials.

\subsection{The cosine}

Different value for the tolerance from $0.01$ to $0.000\,1$, a factor $10$ between
each of them. In black, there is the classic \Lcs{psplot} behavior, and in
magenta the default variable step behavior.

\begin{center}
\bgroup
\psset{algebraic=true, VarStep=true, unit=2, showpoints=true, linecolor=red}
\begin{pspicture}(-0,-1)(3.14,2)\psgrid
  \psplot[VarStepEpsilon=.01]{0}{3.14}{cos(x)}
  \psplot[VarStepEpsilon=.001]{0}{3.14}{cos(x)+.15}
  \psplot[VarStepEpsilon=.0001]{0}{3.14}{cos(x)+.3}
  \psplot[linecolor=magenta]{0}{3.14}{cos(x)+.45}
  \psplot[VarStep=false, linewidth=2\pslinewidth, linecolor=black]{-0}{3.14}{cos(x)+.6}
\end{pspicture}
\egroup
\end{center}

\begin{lstlisting}
\psset{algebraic=true, VarStep=true, unit=2, showpoints=true, linecolor=red}
\begin{pspicture}[showgrid=true](-0,-1)(3.14,2)
  \psplot[VarStepEpsilon=.01]{0}{3.14}{cos(x)}
  \psplot[VarStepEpsilon=.001]{0}{3.14}{cos(x)+.15}
  \psplot[VarStepEpsilon=.0001]{0}{3.14}{cos(x)+.3}
  \psplot[linecolor=magenta]{0}{3.14}{cos(x)+.45}
  \psplot[VarStep=false,linewidth=1pt,linecolor=black]{-0}{3.14}{cos(x)+.6}
\end{pspicture}
\end{lstlisting}


\subsection{The Napierian Logarithm}

A really classic example which gives a bad beginning, the tolerance is set to $0.001$.

\begin{center}
\bgroup
\psset{algebraic=true, VarStep=true, linecolor=red, showpoints=true}
\begin{pspicture}[showgrid=true](0,-5)(16,4)
  \psplot[VarStep=false, linecolor=black]{.01}{16}{ln(x)+1}
  \psplot[linecolor=magenta]{.51}{16}{ln(x-1/2)+1/2}
  \psplot[VarStepEpsilon=.001]{1.01}{16}{ln(x-1)}
  \psplot[VarStepEpsilon=.01]{1.51}{16}{ln(x-1.5)-100/200}
\end{pspicture}
\egroup
\end{center}

\begin{lstlisting}
\psset{algebraic=true, VarStep=true, linecolor=red, showpoints=true}
\begin{pspicture}[showgrid=true](0,-5)(16,4)
  \psplot[VarStep=false, linecolor=black]{.01}{16}{ln(x)+1}
  \psplot[linecolor=magenta]{.51}{16}{ln(x-1/2)+1/2}
  \psplot[VarStepEpsilon=.001]{1.01}{16}{ln(x-1)}
  \psplot[VarStepEpsilon=.01]{1.51}{16}{ln(x-1.5)-100/200}
\end{pspicture}
\end{lstlisting}


\clearpage
\subsection{Sine of the inverse of $x$}
Impossible to draw, but let's try!

\begin{center}
\bgroup
\psset{xunit=64,algebraic=true,VarStep,linecolor=red,showpoints=true,linewidth=1pt}
\begin{pspicture}[showgrid=true](0,-1)(.5,1)
  \psplot[VarStepEpsilon=.0001]{.01}{.25}{sin(1/x)}
\end{pspicture}\\
\begin{pspicture}[showgrid=true](0,-1)(.5,1)
  \psplot[VarStepEpsilon=.00001]{.01}{.25}{sin(1/x)}
\end{pspicture}\\
\begin{pspicture}[showgrid=true](0,-1)(.5,1)
  \psplot[VarStepEpsilon=.000001]{.01}{.25}{sin(1/x)}
\end{pspicture}\\
\begin{pspicture}[showgrid=true](0,-1)(.5,1)
  \psplot[VarStep=false, linecolor=black]{.01}{.25}{sin(1/x)}
\end{pspicture}
\egroup
\end{center}

\begin{lstlisting}
\psset{xunit=64,algebraic=true,VarStep,linecolor=red,showpoints=true,linewidth=1pt}
\begin{pspicture}[showgrid=true](0,-1)(.5,1)
  \psplot[VarStepEpsilon=.0001]{.01}{.25}{sin(1/x)}
\end{pspicture}\\
\begin{pspicture}[showgrid=true](0,-1)(.5,1)
  \psplot[VarStepEpsilon=.00001]{.01}{.25}{sin(1/x)}
\end{pspicture}\\
\begin{pspicture}[showgrid=true](0,-1)(.5,1)
  \psplot[VarStepEpsilon=.000001]{.01}{.25}{sin(1/x)}
\end{pspicture}\\
\begin{pspicture}[showgrid=true](0,-1)(.5,1)
  \psplot[VarStep=false, linecolor=black]{.01}{.25}{sin(1/x)}
\end{pspicture}
\end{lstlisting}





\clearpage
\subsection{A really complicated function}

Just appreciate the difference between the normal behavior and the plotting with the
\Lkeyword{varStep} option. The function is:

\[f(x)=x-\frac{x^2}{10}+\ln(x)+\cos(2x)+\sin(x^2)-1\]

\begin{center}
\bgroup
\psset{xunit=3, algebraic=true, VarStep, showpoints=true}
\begin{pspicture}[showgrid=true](0,-2)(5,6)
  \psplot[VarStepEpsilon=.0005, linecolor=red]{.1}{5}{x-x^2/10+ln(x)+cos(2*x)+sin(x^2)}
  \psplot[linecolor=magenta]{.1}{5}{x-x^2/10+ln(x)+cos(2*x)+sin(x^2)+.5}
  \psplot[VarStep=false]{.1}{5}{x-x^2/10+ln(x)+cos(2*x)+sin(x^2)-1}
\end{pspicture}
\egroup
\end{center}

\begin{lstlisting}
\psset{xunit=3, algebraic=true, VarStep, showpoints=true}
\begin{pspicture}[showgrid=true](0,-2)(5,6)
  \psplot[VarStepEpsilon=.0005, linecolor=red]{.1}{5}{x-x^2/10+ln(x)+cos(2*x)+sin(x^2)}
  \psplot[linecolor=magenta]{.1}{5}{x-x^2/10+ln(x)+cos(2*x)+sin(x^2)+.5}
  \psplot[VarStep=false]{.1}{5}{x-x^2/10+ln(x)+cos(2*x)+sin(x^2)-1}
\end{pspicture}
\end{lstlisting}


\clearpage
\subsection{A hyperbola}

\begin{center}
\bgroup
\psset{algebraic=true, showpoints=true, unit=0.75}
\begin{pspicture}(-5,-4)(9,6)
  \psplot[linecolor=black]{-5}{1.8}{(x-1)/(x-2)}
  \psplot[VarStep=true, VarStepEpsilon=.001, linecolor=red]{2.2}{9}{(x-1)/(x-2)}
  \psaxes{->}(0,0)(-5,-4)(9,6)
\end{pspicture}
\egroup
\end{center}

\begin{lstlisting}
\psset{algebraic=true, showpoints=true, unit=0.75}
\begin{pspicture}(-5,-4)(9,6)
  \psplot[linecolor=black]{-5}{1.8}{(x-1)/(x-2)}
  \psplot[VarStep=true, VarStepEpsilon=.001, linecolor=red]{2.2}{9}{(x-1)/(x-2)}
  \psaxes{->}(0,0)(-5,-4)(9,6)
\end{pspicture}
\end{lstlisting}



\clearpage
\subsection{Using \nxLcs{psparametricplot}}

\begin{BDef}
\Lcs{parametricplot}\OptArgs\Largb{t0}\Largb{t1}\OptArg{PS commands}\Largb{x(t) y(t)}
\end{BDef}

\begin{center}
\bgroup
\psset{unit=2.5}
\begin{pspicture}[showgrid=true](-1,-1)(1,1)
\parametricplot[algebraic=true,linecolor=red,VarStep=true, showpoints=true,
                VarStepEpsilon=.0001]
                {-3.14}{3.14}{cos(3*t)|sin(2*t)}
\end{pspicture}
\begin{pspicture}[showgrid=true](-1,-1)(1,1)
\parametricplot[algebraic=true,linecolor=blue,VarStep=true, showpoints=false,
                VarStepEpsilon=.0001]
                {-3.14}{3.14}{cos(3*t)|sin(2*t)}
\end{pspicture}
\egroup
\end{center}

\begin{lstlisting}
\psset{unit=3}
\begin{pspicture}[showgrid=true](-1,-1)(1,1)
\parametricplot[algebraic=true,linecolor=red,VarStep=true, showpoints=true,
                VarStepEpsilon=.0001]
                {-3.14}{3.14}{cos(3*t)|sin(2*t)}
\end{pspicture}
\begin{pspicture}[showgrid=true](-1,-1)(1,1)
\parametricplot[algebraic=true,linecolor=blue,VarStep=true, showpoints=false,
                VarStepEpsilon=.0001]
                {-3.14}{3.14}{cos(3*t)|sin(2*t)}
\end{pspicture}
\end{lstlisting}


\begin{center}
\bgroup
\psset{unit=2.5}
\begin{pspicture}[showgrid=true](-1,-1)(1,1)
\parametricplot[algebraic=true,linecolor=red,VarStep=true, showpoints=true,
                VarStepEpsilon=.0001]
                {0}{47.115}{cos(5*t)|sin(3*t)}
\end{pspicture}
\begin{pspicture}[showgrid=true](-1,-1)(1,1)
\parametricplot[algebraic=true,linecolor=blue,VarStep=true, showpoints=false,
                VarStepEpsilon=.0001]
                {0}{47.115}{cos(5*t)|sin(3*t)}
\end{pspicture}
\egroup
\end{center}

\begin{lstlisting}
\psset{unit=2.5}
\begin{pspicture}[showgrid=true](-1,-1)(1,1)
\parametricplot[algebraic=true,linecolor=red,VarStep=true, showpoints=true,
                VarStepEpsilon=.0001]
                {0}{47.115}{cos(5*t)|sin(3*t)}
\end{pspicture}
\begin{pspicture}[showgrid=true](-1,-1)(1,1)
\parametricplot[algebraic=true,linecolor=blue,VarStep=true, showpoints=false,
                VarStepEpsilon=.0001]
                {0}{47.115}{cos(5*t)|sin(3*t)}
\end{pspicture}
\end{lstlisting}


\begin{center}
\bgroup
\psset{xunit=.5}
\begin{pspicture}[showgrid=true](0,0)(12.566,2)
\parametricplot[algebraic=true,linecolor=red,VarStep, showpoints=true,
        VarStepEpsilon=.01]{0}{12.566}{t+cos(-t-Pi/2)|1+sin(-t-Pi/2)}
\end{pspicture}
%
\begin{pspicture}[showgrid=true](0,0)(12.566,2)
\parametricplot[algebraic=true,linecolor=blue,VarStep, showpoints=false,
        VarStepEpsilon=.001]{0}{12.566}{t+cos(-t-Pi/2)|1+sin(-t-Pi/2)}
\end{pspicture}
\egroup
\end{center}

\begin{lstlisting}
\psset{xunit=.5}
\begin{pspicture}[showgrid=true](0,0)(12.566,2)
\parametricplot[algebraic=true,linecolor=red,VarStep, showpoints=true,
        VarStepEpsilon=.01]{0}{12.566}{t+cos(-t-Pi/2)|1+sin(-t-Pi/2)}
\end{pspicture}
%
\begin{pspicture}[showgrid=true](0,0)(12.566,2)
\parametricplot[algebraic=true,linecolor=blue,VarStep, showpoints=false,
        VarStepEpsilon=.001]{0}{12.566}{t+cos(-t-Pi/2)|1+sin(-t-Pi/2)}
\end{pspicture}
\end{lstlisting}


\section{New math functions and their derivatives}

\subsection{The inverse sine and its derivative}

\begin{center}
\bgroup
\psset{unit=1.5}
\begin{pspicture}[showgrid=true](-1,-2)(1,2)
  \psplot[linecolor=blue,algebraic=true]{-1}{1}{asin(x)}
\end{pspicture}
\hspace{1em}
\psset{algebraic, VarStep, VarStepEpsilon=.001, showpoints=true}
\begin{pspicture}[showgrid=true](-1,-2)(1,2)
  \psplot[linecolor=blue]{-.999}{.999}{asin(x)}
\end{pspicture}
\hspace{1em}
\begin{pspicture}[showgrid=true](-1,0)(1,4)
  \psplot[linecolor=blue]{-.97}{.97}{Derive(1,asin(x))}
\end{pspicture}
\hspace{1em}
\psset{algebraic=true, VarStep, VarStepEpsilon=.0001, showpoints=true}
\begin{pspicture}[showgrid=true](-1,0)(1,4)
  \psplot[linecolor=blue]{-.97}{.97}{Derive(1,asin(x))}
\end{pspicture}
\egroup
\end{center}

\begin{lstlisting}
\psset{unit=1.5}
\begin{pspicture}[showgrid=true](-1,-2)(1,2)
  \psplot[linecolor=blue,algebraic=true]{-1}{1}{asin(x)}
\end{pspicture}
\hspace{1em}
\psset{algebraic=true, VarStep, VarStepEpsilon=.001, showpoints=true}
\begin{pspicture}[showgrid=true](-1,-2)(1,2)
  \psplot[linecolor=blue]{-.999}{.999}{asin(x)}
\end{pspicture}
\hspace{1em}
\begin{pspicture}[showgrid=true](-1,0)(1,4)
  \psplot[linecolor=red]{-.97}{.97}{Derive(1,asin(x))}
\end{pspicture}
\hspace{1em}
\psset{algebraic=true, VarStep, VarStepEpsilon=.0001, showpoints=true}
\begin{pspicture}[showgrid=true](-1,0)(1,4)
  \psplot[linecolor=red]{-.97}{.97}{Derive(1,asin(x))}
\end{pspicture}
\end{lstlisting}


\subsection{The inverse cosine and its derivative}

\begin{center}
\bgroup
\psset{unit=1.5}
\begin{pspicture}[showgrid=true](-1,0)(1,3)
  \psplot[linecolor=blue,algebraic=true]{-1}{1}{acos(x)}
\end{pspicture}
\hspace{1em}
\psset{algebraic=true, VarStep, VarStepEpsilon=.001, showpoints=true}
\begin{pspicture}[showgrid=true](-1,0)(1,3)
  \psplot[linecolor=blue]{-.999}{.999}{acos(x)}
\end{pspicture}
\hspace{1em}
\begin{pspicture}[showgrid=true](-1,-4)(1,-1)
  \psplot[linecolor=blue]{-.97}{.97}{Derive(1,acos(x))}
\end{pspicture}
\hspace{1em}
\psset{algebraic=true, VarStep, VarStepEpsilon=.0001, showpoints=true}
\begin{pspicture}[showgrid=true](-1,-4)(1,-1)
  \psplot[linecolor=blue]{-.97}{.97}{Derive(1,acos(x))}
\end{pspicture}
\egroup
\end{center}

\begin{lstlisting}
\psset{unit=1.5}
\begin{pspicture}[showgrid=true](-1,0)(1,3)
  \psplot[linecolor=blue,algebraic=true]{-1}{1}{acos(x)}
\end{pspicture}
\hspace{1em}
\psset{algebraic=true, VarStep, VarStepEpsilon=.001, showpoints=true}
\begin{pspicture}[showgrid=true](-1,0)(1,3)
  \psplot[linecolor=blue]{-.999}{.999}{acos(x)}
\end{pspicture}
\hspace{1em}
\begin{pspicture}[showgrid=true](-1,-4)(1,-1)
  \psplot[linecolor=red]{-.97}{.97}{Derive(1,acos(x))}
\end{pspicture}
\hspace{1em}
\psset{algebraic=true, VarStep, VarStepEpsilon=.0001, showpoints=true}
\begin{pspicture}[showgrid=true](-1,-4)(1,-1)
  \psplot[linecolor=red]{-.97}{.97}{Derive(1,acos(x))}
\end{pspicture}
\end{lstlisting}



\subsection{The inverse tangent and its derivative}

\begin{center}
\bgroup
\begin{pspicture}[showgrid=true](-4,-2)(4,2)
\psset{algebraic=true}
  \psplot[linecolor=blue,linewidth=1pt]{-4}{4}{atg(x)}
  \psplot[linecolor=red,VarStep, VarStepEpsilon=.0001, showpoints=true]{-4}{4}{Derive(1,atg(x))}
\end{pspicture}
\hspace{1em}
\begin{pspicture}[showgrid=true](-4,-2)(4,2)
\psset{algebraic=true, VarStep, VarStepEpsilon=.001, showpoints=true}
  \psplot[linecolor=blue]{-4}{4}{atg(x)}
  \psplot[linecolor=red]{-4}{4}{Derive(1,atg(x))}
\end{pspicture}
\egroup
\end{center}

\begin{lstlisting}
\begin{pspicture}[showgrid=true](-4,-2)(4,2)
\psset{algebraic=true}
  \psplot[linecolor=blue,linewidth=1pt]{-4}{4}{atg(x)}
  \psplot[linecolor=red,VarStep, VarStepEpsilon=.0001, showpoints=true]{-4}{4}{Derive(1,atg(x))}
\end{pspicture}
\hspace{1em}
\begin{pspicture}[showgrid=true](-4,-2)(4,2)
\psset{algebraic=true, VarStep, VarStepEpsilon=.001, showpoints=true}
  \psplot[linecolor=blue]{-4}{4}{atg(x)}
  \psplot[linecolor=red]{-4}{4}{Derive(1,atg(x))}
\end{pspicture}
\end{lstlisting}

\subsection{Hyperbolic functions}

\begin{center}
\bgroup
\begin{pspicture}(-3,-4)(3,4)
\psset{algebraic=true}
  \psplot[linecolor=red,linewidth=1pt]{-2}{2}{sh(x)}
  \psplot[linecolor=blue,linewidth=1pt]{-2}{2}{ch(x)}
  \psplot[linecolor=green,linewidth=1pt]{-3}{3}{th(x)}
  \psaxes{->}(0,0)(-3,-4)(3,4)
\end{pspicture}
\hspace{1em}
\begin{pspicture}(-3,-4)(3,4)
\psset{algebraic=true, VarStep=true, VarStepEpsilon=.001, showpoints=true}
  \psplot[linecolor=red,linewidth=1pt]{-2}{2}{sh(x)}
  \psplot[linecolor=blue,linewidth=1pt]{-2}{2}{ch(x)}
  \psplot[linecolor=green,linewidth=1pt]{-3}{3}{th(x)}
  \psaxes{->}(0,0)(-3,-4)(3,4)
\end{pspicture}
\egroup
\end{center}

\begin{lstlisting}
\begin{pspicture}(-3,-4)(3,4)
\psset{algebraic=true}
  \psplot[linecolor=red,linewidth=1pt]{-2}{2}{sh(x)}
  \psplot[linecolor=blue,linewidth=1pt]{-2}{2}{ch(x)}
  \psplot[linecolor=green,linewidth=1pt]{-3}{3}{th(x)}
  \psaxes{->}(0,0)(-3,-4)(3,4)
\end{pspicture}
\hspace{1em}
\begin{pspicture}(-3,-4)(3,4)
\psset{algebraic=true, VarStep=true, VarStepEpsilon=.001, showpoints=true}
  \psplot[linecolor=red,linewidth=1pt]{-2}{2}{sh(x)}
  \psplot[linecolor=blue,linewidth=1pt]{-2}{2}{ch(x)}
  \psplot[linecolor=green,linewidth=1pt]{-3}{3}{th(x)}
  \psaxes{->}(0,0)(-3,-4)(3,4)
\end{pspicture}
\end{lstlisting}



\begin{center}
\bgroup
\begin{pspicture}(-3,-4)(3,4)
\psset{algebraic=true}
  \psplot[linecolor=red,linewidth=1pt]{-2}{2}{Derive(1,sh(x))}
  \psplot[linecolor=blue,linewidth=1pt]{-2}{2}{Derive(1,ch(x))}
  \psplot[linecolor=green,linewidth=1pt]{-3}{3}{Derive(1,th(x))}
  \psaxes{->}(0,0)(-3,-4)(3,4)
\end{pspicture}
\hspace{1em}
\begin{pspicture}(-3,-4)(3,4)
\psset{algebraic=true, VarStep=true, VarStepEpsilon=.001, showpoints=true}
  \psplot[linecolor=red,linewidth=1pt]{-2}{2}{Derive(1,sh(x))}
  \psplot[linecolor=blue,linewidth=1pt]{-2}{2}{Derive(1,ch(x))}
  \psplot[linecolor=green,linewidth=1pt]{-3}{3}{Derive(1,th(x))}
  \psaxes{->}(0,0)(-3,-4)(3,4)
\end{pspicture}
\egroup
\end{center}

\begin{lstlisting}
\begin{pspicture}(-3,-4)(3,4)
\psset{algebraic=true,linewidth=1pt}
  \psplot[linecolor=red,linewidth=1pt]{-2}{2}{Derive(1,sh(x))}
  \psplot[linecolor=blue,linewidth=1pt]{-2}{2}{Derive(1,ch(x))}
  \psplot[linecolor=green,linewidth=1pt]{-3}{3}{Derive(1,th(x))}
  \psaxes{->}(0,0)(-3,-4)(3,4)
\end{pspicture}
\hspace{1em}
\begin{pspicture}(-3,-4)(3,4)
\psset{algebraic=true, VarStep=true, VarStepEpsilon=.001, showpoints=true}
  \psplot[linecolor=red,linewidth=1pt]{-2}{2}{Derive(1,sh(x))}
  \psplot[linecolor=blue,linewidth=1pt]{-2}{2}{Derive(1,ch(x))}
  \psplot[linecolor=green,linewidth=1pt]{-3}{3}{Derive(1,th(x))}
  \psaxes{->}(0,0)(-3,-4)(3,4)
\end{pspicture}
\end{lstlisting}



\begin{center}
\bgroup
\begin{pspicture}(-7,-3)(7,3)
\psset{algebraic=true}
  \psplot[linecolor=red,linewidth=1pt]{-7}{7}{Argsh(x)}
  \psplot[linecolor=blue,linewidth=1pt]{1}{7}{Argch(x)}
  \psplot[linecolor=green,linewidth=1pt]{-.99}{.99}{Argth(x)}
  \psaxes{->}(0,0)(-7,-3)(7,3)
\end{pspicture}\\[\baselineskip]
\begin{pspicture}(-7,-3)(7,3)
  \psset{algebraic=true, VarStep, VarStepEpsilon=.001, showpoints=true}
  \psplot[linecolor=red,linewidth=1pt]{-7}{7}{Argsh(x)}
  \psplot[linecolor=blue,linewidth=1pt]{1.001}{7}{Argch(x)}
  \psplot[linecolor=green,linewidth=1pt]{-.99}{.99}{Argth(x)}
  \psaxes{->}(0,0)(-7,-3)(7,3)
\end{pspicture}
\egroup
\end{center}

\begin{lstlisting}
\begin{pspicture}(-7,-3)(7,3)
\psset{algebraic=true}
  \psplot[linecolor=red,linewidth=1pt]{-7}{7}{Argsh(x)}
  \psplot[linecolor=blue,linewidth=1pt]{1}{7}{Argch(x)}
  \psplot[linecolor=green,linewidth=1pt]{-.99}{.99}{Argth(x)}
  \psaxes{->}(0,0)(-7,-3)(7,3)
\end{pspicture}\\[\baselineskip]
\begin{pspicture}(-7,-3)(7,3)
  \psset{algebraic=true, VarStep, VarStepEpsilon=.001, showpoints=true}
  \psplot[linecolor=red,linewidth=1pt]{-7}{7}{Argsh(x)}
  \psplot[linecolor=blue,linewidth=1pt]{1.001}{7}{Argch(x)}
  \psplot[linecolor=green,linewidth=1pt]{-.99}{.99}{Argth(x)}
  \psaxes{->}(0,0)(-7,-3)(7,3)
\end{pspicture}
\end{lstlisting}



\begin{center}
\bgroup
\begin{pspicture}(-7,-0.5)(7,6)
\psset{algebraic=true}
  \psplot[linecolor=red,linewidth=1pt]{-7}{7}{Derive(1,Argsh(x))}
  \psplot[linecolor=blue,linewidth=1pt]{1.014}{7}{Derive(1,Argch(x))}
  \psplot[linecolor=green,linewidth=1pt]{-.9}{.9}{Derive(1,Argth(x))}
  \psaxes{->}(0,0)(-7,0)(7,6)
\end{pspicture}\\[\baselineskip]
\begin{pspicture}(-7,-0.5)(7,6)
\psset{algebraic=true}
  \psset{algebraic=true, VarStep=true, VarStepEpsilon=.001, showpoints=true}
  \psplot[linecolor=red,linewidth=1pt]{-7}{7}{Derive(1,Argsh(x))}
  \psplot[linecolor=blue,linewidth=1pt]{1.014}{7}{Derive(1,Argch(x))}
  \psplot[linecolor=green,linewidth=1pt]{-.9}{.9}{Derive(1,Argth(x))}
  \psaxes{->}(0,0)(-7,0)(7,6)
\end{pspicture}
\egroup
\end{center}

\begin{lstlisting}
\begin{pspicture}(-7,-0.5)(7,6)
\psset{algebraic=true}
  \psplot[linecolor=red,linewidth=1pt]{-7}{7}{Derive(1,Argsh(x))}
  \psplot[linecolor=blue,linewidth=1pt]{1.014}{7}{Derive(1,Argch(x))}
  \psplot[linecolor=green,linewidth=1pt]{-.9}{.9}{Derive(1,Argth(x))}
  \psaxes{->}(0,0)(-7,0)(7,6)
\end{pspicture}\\[\baselineskip]
\begin{pspicture}(-7,-0.5)(7,6)
\psset{algebraic=true}
  \psset{algebraic=true, VarStep=true, VarStepEpsilon=.001, showpoints=true}
  \psplot[linecolor=red,linewidth=1pt]{-7}{7}{Derive(1,Argsh(x))}
  \psplot[linecolor=blue,linewidth=1pt]{1.014}{7}{Derive(1,Argch(x))}
  \psplot[linecolor=green,linewidth=1pt]{-.9}{.9}{Derive(1,Argth(x))}
  \psaxes{->}(0,0)(-7,0)(7,6)
\end{pspicture}
\end{lstlisting}


\clearpage
%--------------------------------------------------------------------------------------
\section[\nxLcs{psplotDiffEqn} -- solving diffential equations]%
  {\nxLcs{psplotDiffEqn} -- solving diffential equations}
%--------------------------------------------------------------------------------------


 A differential equation of first order is like

\begin{align} y^\prime=f(x,y,y^\prime) \end{align}


where $y$ is a function of $x$. We define some vectors $Y=[y, y',
\cdots , y^{(n-1)}]$ and $Y^\prime=[y^\prime, y^{\prime\prime},
\cdots , y^{n}]$, depending on the order $n$. The syntax of the
macro is

\begin{BDef}
\Lcs{psplotDiffEqn}\OptArgs\Largb{x0}\Largb{x1}\Largb{y0}\Largb{f(x,y,y',...)}
\end{BDef}

\begin{itemize}\setlength\itemsep{0pt}\setlength\parsep{0pt}\setlength\parskip{0pt}
\item \verb+options+: the \verb+\psplotDiffEqn+ specific options and all other of PSTricks, which
make sense;
\item $x_0$: the start value;
\item $x_1$: the end value of the definition interval;
\item $y_0$: the initial values for $y(x_0)\ y'(x_0)\ \ldots$;
\item $f(x,y,y',...)$: the differential equation, depending to the number of initial values, e.g.:
    \verb+{0 1}+ for $y_0$ are two initial values, so that we have a differential equation of
    second order $f(x,y,y')$ and the macro leaves $y\ y'$ on the stack.
\end{itemize}

The new options are:


\begin{itemize}\setlength\itemsep{0pt}\setlength\parsep{0pt}\setlength\parskip{0pt}
\item \Lkeyword{method}: integration method (\verb+euler+ for order 1 euler method, \verb+rk4+ for
  4\textsuperscript{th} order Runge-Kutta method);
\item \Lkeyword{whichabs}: select the abscissa for plotting the graph, by default it is
  $x$, but you can specify a number which represent a position in the vector $y$;
\item \Lkeyword{whichord}: same as precedent for the ordinate, by default $y(0)$;
\item \Lkeyword{plotfuncx}: describe a ps function for the abscissa, parameter
  \Lkeyword{whichabs} becomes useless;
\item \Lkeyword{plotfuncy}: idem for the ordinate;
\item \Lkeyword{buildvector}: boolean parameter for specifying the input-output of the
  $f$ description:
  \begin{description}
  \item[\texttt{true}] (default): $y$ is put on the stack element by element, $y'$
    must be given in the same way;
  \item[\texttt{false}]: $y$ is put on the stack as a vector, $y'$ must be returned
  in the same way;
  \end{description}

\item \Lkeyword{algebraic=true}: algebraic=true description for $f$, \Lkeyword{buildvector}
  parameter is useless when activating this option.
\end{itemize}



\clearpage
\subsection{Variable step for differential equations}

A new algorithm has been added for adjusting the step according to the variations of
the curve. The parameter \Lkeyword{method} has a new possible value : \Lkeyword{varrkiv} to
activate the \Index{Runge-Kutta} method with variable step, then the parameter
\Lkeyword{varsteptol} (real value; \verb+.01+ by default) can control the tolerance of
the algortihm.

\begin{center}
\bgroup
\def\Funct{neg}\def\FunctAlg{-y[0]}
\psset{xunit=1.5, yunit=8, showpoints=true}
\begin{pspicture}[showgrid=true](0,0)(10,1.2)
  \psplot[linewidth=6\pslinewidth, linecolor=green, showpoints=false]{0}{10}{Euler x neg exp}
  \psplotDiffEqn[linecolor=magenta, method=varrkiv, varsteptol=.1, plotpoints=2]{0}{10}{1}{\Funct}
  \rput(0,.0){\psplotDiffEqn[linecolor=blue, method=varrkiv, varsteptol=.01, plotpoints=2]{0}{10}{1}{\Funct}}
  \rput(0,.1){\psplotDiffEqn[linecolor=Orange, method=varrkiv, varsteptol=.001, plotpoints=2]{0}{10}{1}{\Funct}}
  \rput(0,.2){\psplotDiffEqn[linecolor=red, method=varrkiv, varsteptol=.0001, plotpoints=2]{0}{10}{1}{\Funct}}
  \psset{linewidth=4\pslinewidth,showpoints=false}
  \rput*(3.3,.9){\psline[linecolor=magenta](-.75cm,0)}
  \rput*[l](3.3,.9){\small RK ordre 4 : $\varepsilon<10^{-1}$}
  \rput*(3.3,.8){\psline[linecolor=blue](-.75cm,0)}
  \rput*[l](3.3,.8){\small RK ordre 4 : $\varepsilon<10^{-2}$}
  \rput*(3.3,.7){\psline[linecolor=Orange](-.75cm,0)}
  \rput*[l](3.3,.7){\small RK ordre 4 : $\varepsilon<10^{-3}$}
  \rput*(3.3,.6){\psline[linecolor=red](-.75cm,0)}
  \rput*[l](3.3,.6){\small RK ordre 4 : $\varepsilon<10^{-4}$}
  \rput*(3.3,.5){\psline[linecolor=green](-.75cm,0)}
  \rput*[l](3.3,.5){\small solution exacte}
\end{pspicture}
{\captionof{figure}{Equation $y'=-y$ with $y_0=1$.}\label{fig:minusexpvarstep}}
\egroup
\end{center}


\begin{lstlisting}[wide=true]
\def\Funct{neg}\def\FunctAlg{-y[0]}
\psset{xunit=1.5, yunit=8, showpoints=true}
\begin{pspicture}[showgrid=true](0,0)(10,1.2)
  \psplot[linewidth=6\pslinewidth, linecolor=green, showpoints=false]{0}{10}{Euler x neg exp}
  \psplotDiffEqn[linecolor=magenta, method=varrkiv, varsteptol=.1, plotpoints=2]{0}{10}{1}{\Funct}
  \rput(0,.0){\psplotDiffEqn[linecolor=blue, method=varrkiv, varsteptol=.01, plotpoints=2]{0}{10}{1}{\Funct}}
  \rput(0,.1){\psplotDiffEqn[linecolor=Orange, method=varrkiv, varsteptol=.001, plotpoints=2]{0}{10}{1}{\Funct}}
  \rput(0,.2){\psplotDiffEqn[linecolor=red, method=varrkiv, varsteptol=.0001, plotpoints=2]{0}{10}{1}{\Funct}}
  \psset{linewidth=4\pslinewidth,showpoints=false}
  \rput*(3.3,.9){\psline[linecolor=magenta](-.75cm,0)}
  \rput*[l](3.3,.9){\small RK ordre 4 : $\varepsilon<10^{-1}$}
  \rput*(3.3,.8){\psline[linecolor=blue](-.75cm,0)}
  \rput*[l](3.3,.8){\small RK ordre 4 : $\varepsilon<10^{-2}$}
  \rput*(3.3,.7){\psline[linecolor=Orange](-.75cm,0)}
  \rput*[l](3.3,.7){\small RK ordre 4 : $\varepsilon<10^{-3}$}
  \rput*(3.3,.6){\psline[linecolor=red](-.75cm,0)}
  \rput*[l](3.3,.6){\small RK ordre 4 : $\varepsilon<10^{-4}$}
  \rput*(3.3,.5){\psline[linecolor=green](-.75cm,0)}
  \rput*[l](3.3,.5){\small solution exacte}
\end{pspicture}
\end{lstlisting}



\begin{center}
\bgroup
\def\Funct{exch neg}
\psset{xunit=1.5, yunit=5, method=varrkiv, showpoints=true}%%
\def\quatrepi{12.5663706144}
\begin{pspicture}(0,-1)(10,1.3)
  \psaxes{->}(0,0)(0,-1)(10,1.3)
  \psplot[linewidth=4\pslinewidth, linecolor=green, algebraic=true]{0}{10}{cos(x)}
  \rput(0,.0){\psplotDiffEqn[linecolor=magenta, plotpoints=7, varsteptol=.1]{0}{10}{1 0}{\Funct}}
  \rput(0,.0){\psplotDiffEqn[linecolor=blue, plotpoints=201, varsteptol=.01]{0}{10}{1 0}{\Funct}}
  \rput(0,.1){\psplotDiffEqn[linewidth=2\pslinewidth, linecolor=red, varsteptol=.001]{0}{10}{1 0}{\Funct}}
  \rput(0,.2){\psplotDiffEqn[linecolor=black, varsteptol=.0001]{0}{10}{1 0}{\Funct}}
  \rput(0,.3){\psplotDiffEqn[linecolor=Orange, varsteptol=.00001]{0}{10}{1 0}{\Funct}}
  \psset{linewidth=4\pslinewidth,showpoints=false}
  \rput*(2.3,.9){\psline[linecolor=magenta](-.75cm,0)}
  \rput*[l](2.3,.9){\small $\varepsilon<10^{-1}$}
  \rput*(2.3,.8){\psline[linecolor=blue](-.75cm,0)}
  \rput*[l](2.3,.8){\small $\varepsilon<10^{-2}$}
  \rput*(2.3,.7){\psline[linecolor=red](-.75cm,0)}
  \rput*[l](2.3,.7){\small $\varepsilon<10^{-3}$}
  \rput*(2.3,.6){\psline[linecolor=black](-.75cm,0)}
  \rput*[l](2.3,.6){\small $\varepsilon<10^{-4}$}
  \rput*(2.3,.5){\psline[linecolor=Orange](-.75cm,0)}
  \rput*[l](2.3,.5){\small $\varepsilon<10^{-5}$}
  \rput*(2.3,.4){\psline[linecolor=green](-.75cm,0)}
  \rput*[l](2.3,.4){\small solution exacte}
\end{pspicture}
{\captionof{figure}{Equation $y''=-y$}\label{fig:trigfunc}}
\egroup
\end{center}

\begin{lstlisting}[wide=true]
\def\Funct{exch neg}
\psset{xunit=1.5, yunit=5, method=varrkiv, showpoints=true}%%
\def\quatrepi{12.5663706144}
\begin{pspicture}(0,-1)(10,1.3)
  \psaxes{->}(0,0)(0,-1)(10,1.3)
  \psplot[linewidth=4\pslinewidth, linecolor=green, algebraic=true]{0}{10}{cos(x)}
  \rput(0,.0){\psplotDiffEqn[linecolor=magenta, plotpoints=7, varsteptol=.1]{0}{10}{1 0}{\Funct}}
  \rput(0,.0){\psplotDiffEqn[linecolor=blue, plotpoints=201, varsteptol=.01]{0}{10}{1 0}{\Funct}}
  \rput(0,.1){\psplotDiffEqn[linewidth=2\pslinewidth, linecolor=red, varsteptol=.001]{0}{10}{1 0}{\Funct}}
  \rput(0,.2){\psplotDiffEqn[linecolor=black, varsteptol=.0001]{0}{10}{1 0}{\Funct}}
  \rput(0,.3){\psplotDiffEqn[linecolor=Orange, varsteptol=.00001]{0}{10}{1 0}{\Funct}}
  \psset{linewidth=4\pslinewidth,showpoints=false}
  \rput*(2.3,.9){\psline[linecolor=magenta](-.75cm,0)}
  \rput*[l](2.3,.9){\small $\varepsilon<10^{-1}$}
  \rput*(2.3,.8){\psline[linecolor=blue](-.75cm,0)}
  \rput*[l](2.3,.8){\small $\varepsilon<10^{-2}$}
  \rput*(2.3,.7){\psline[linecolor=red](-.75cm,0)}
  \rput*[l](2.3,.7){\small $\varepsilon<10^{-3}$}
  \rput*(2.3,.6){\psline[linecolor=black](-.75cm,0)}
  \rput*[l](2.3,.6){\small $\varepsilon<10^{-4}$}
  \rput*(2.3,.5){\psline[linecolor=Orange](-.75cm,0)}
  \rput*[l](2.3,.5){\small $\varepsilon<10^{-5}$}
  \rput*(2.3,.4){\psline[linecolor=green](-.75cm,0)}
  \rput*[l](2.3,.4){\small solution exacte}
\end{pspicture}
\end{lstlisting}




\begin{center}
\bgroup
\def\Funct{exch}
\psset{xunit=4, yunit=1, method=varrkiv, showpoints=true}%%
\def\quatrepi{12.5663706144}
\begin{pspicture}(0,-0.5)(3,11)
  \psaxes{->}(0,0)(3,11)
  \psplot[linewidth=4\pslinewidth, linecolor=green, algebraic=true]{0}{3}{ch(x)}
  \rput(0,.0){\psplotDiffEqn[linecolor=magenta, varsteptol=.1]{0}{3}{1 0}{\Funct}}
  \rput(0,.3){\psplotDiffEqn[linecolor=blue, varsteptol=.01]{0}{3}{1 0}{\Funct}}
  \rput(0,.6){\psplotDiffEqn[linecolor=red, varsteptol=.001]{0}{3}{1 0}{\Funct}}
  \rput(0,.9){\psplotDiffEqn[linecolor=black, varsteptol=.0001]{0}{3}{1 0}{\Funct}}
  \rput(0,1.2){\psplotDiffEqn[linecolor=Orange, varsteptol=.00001]{0}{3}{1 0}{\Funct}}
  \psset{linewidth=4\pslinewidth,showpoints=false}
  \rput*(2.3,.9){\psline[linecolor=magenta](-.75cm,0)}
  \rput*[l](2.3,.9){\small $\varepsilon<10^{-1}$}
  \rput*(2.3,.8){\psline[linecolor=blue](-.75cm,0)}
  \rput*[l](2.3,.8){\small $\varepsilon<10^{-2}$}
  \rput*(2.3,.7){\psline[linecolor=red](-.75cm,0)}
  \rput*[l](2.3,.7){\small $\varepsilon<10^{-3}$}
  \rput*(2.3,.6){\psline[linecolor=black](-.75cm,0)}
  \rput*[l](2.3,.6){\small $\varepsilon<10^{-4}$}
  \rput*(2.3,.5){\psline[linecolor=Orange](-.75cm,0)}
  \rput*[l](2.3,.5){\small $\varepsilon<10^{-5}$}
  \rput*(2.3,.4){\psline[linecolor=green](-.75cm,0)}
  \rput*[l](2.3,.4){\small solution exacte}
\end{pspicture}
\captionof{figure}{Equation $y''=y$}
\egroup
\end{center}

\begin{lstlisting}[wide=true]
\def\Funct{exch}
\psset{xunit=4, yunit=1, method=varrkiv, showpoints=true}%%
\def\quatrepi{12.5663706144}
\begin{pspicture}(0,-0.5)(3,11)
  \psaxes{->}(0,0)(3,11)
  \psplot[linewidth=4\pslinewidth, linecolor=green, algebraic=true]{0}{3}{ch(x)}
  \rput(0,.0){\psplotDiffEqn[linecolor=magenta, varsteptol=.1]{0}{3}{1 0}{\Funct}}
  \rput(0,.3){\psplotDiffEqn[linecolor=blue, varsteptol=.01]{0}{3}{1 0}{\Funct}}
  \rput(0,.6){\psplotDiffEqn[linecolor=red, varsteptol=.001]{0}{3}{1 0}{\Funct}}
  \rput(0,.9){\psplotDiffEqn[linecolor=black, varsteptol=.0001]{0}{3}{1 0}{\Funct}}
  \rput(0,1.2){\psplotDiffEqn[linecolor=Orange, varsteptol=.00001]{0}{3}{1 0}{\Funct}}
  \psset{linewidth=4\pslinewidth,showpoints=false}
  \rput*(2.3,.9){\psline[linecolor=magenta](-.75cm,0)}
  \rput*[l](2.3,.9){\small $\varepsilon<10^{-1}$}
  \rput*(2.3,.8){\psline[linecolor=blue](-.75cm,0)}
  \rput*[l](2.3,.8){\small $\varepsilon<10^{-2}$}
  \rput*(2.3,.7){\psline[linecolor=red](-.75cm,0)}
  \rput*[l](2.3,.7){\small $\varepsilon<10^{-3}$}
  \rput*(2.3,.6){\psline[linecolor=black](-.75cm,0)}
  \rput*[l](2.3,.6){\small $\varepsilon<10^{-4}$}
  \rput*(2.3,.5){\psline[linecolor=Orange](-.75cm,0)}
  \rput*[l](2.3,.5){\small $\varepsilon<10^{-5}$}
  \rput*(2.3,.4){\psline[linecolor=green](-.75cm,0)}
  \rput*[l](2.3,.4){\small solution exacte}
\end{pspicture}
\end{lstlisting}




\clearpage
\subsection{Equation of second order}

Here is the traditional simulation of two stars attracting each
other according to the classical gravitation law in
$\displaystyle\frac{1}{r^2}$. In 2-Dimensions, the system to be
solved is composed of four second order differential equations. In
order to be described, each of them gives two first order
equations, then we obtain a 8 sized vectorial equation. In the
following example the masses of the stars are 1 and 20.

\[
\left\{
\begin{array}[m]{l}
  x''_1=\displaystyle\frac{M_2}{r^2}\cos(\theta)\\
  y''_1=\displaystyle\frac{M_2}{r^2}\sin(\theta)\\
  x''_2=\displaystyle\frac{M_1}{r^2}\cos(\theta)\\
  y''_2=\displaystyle\frac{M_1}{r^2}\sin(\theta)\\
\end{array}
\right.
\mbox{ avec }
\left\{
\begin{array}[m]{l}
  r^2=(x_1-x_2)^2+(y_1-y_2)^2\\
  \cos(\theta)=\displaystyle\frac{(x_1-x_2)}{r}\\
  \sin(\theta)=\displaystyle\frac{(y_1-y_2)}{r}\\
\end{array}
\right.
\mbox{%
\begin{pspicture}[shift=-2](5,4)\psset{arrowscale=2}
  \psframe[linewidth=.75\pslinewidth](5,4)
  \pstGeonode[PosAngle={-90,90}](1,1){M_1}(4,3){M_2}
  \pstHomO[HomCoef=.33, PointSymbol=none]{M_1}{M_2}[F_1]
  \psline[arrows=->](M_1)(F_1)
  \pstHomO[HomCoef=.33, PointSymbol=none]{M_2}{M_1}[F_2]
  \psline[arrows=->, arrowscale=2](M_2)(F_2)
  \pstGeonode[PointSymbol=none, PointName=none](M_2|M_1){A}
  \psline[linewidth=.5\pslinewidth](M_1)(A)
  \pstMarkAngle{A}{M_1}{M_2}{$\theta$}
  \ncline[linewidth=.5\pslinewidth, offset=.5, arrows=<->]{M_1}{M_2}
  \ncput*{$r$}
\end{pspicture}}
\]

\begin{table}[!htbp]
  \centering\small
    \begin{tabular}{|l@{}>{\ttfamily}l@{}>{ \ttfamily \%\% }l|}
      \hline
      && x1 y1 x'1 y'1 x2 y2 x'2 y'2\\
      &/yp2 exch def /xp2 exch def /ay2 exch def /ax2 exch def&mise en variables\\
      &/yp1 exch def /xp1 exch def /ay1 exch def /ax1 exch def&mise en variables\\
      &/ro2 ax2 ax1 sub dup mul ay2 ay1 sub dup mul add def&calcul de r*r\\
      &xp1 yp1&\\
      &ax2 ax1 sub ro2 sqrt div ro2 div&calcul de x''1\\
      &ay2 ay1 sub ro2 sqrt div ro2 div&calcul de y''1\\
      &xp2 yp2&\\
      &3 index -20 mul&calcul de x''2=-20x''1\\
      &3 index -20 mul&calcul de y''2=-20y''1\\
      \hline
    \end{tabular}
    \caption{\PS source code for the gravitational interaction}\label{intgravcode}
\end{table}

\begin{table}[!htbp]
  \centering
    \small\newcommand{\POW}{\symbol{'136}}
    \begin{tabular}{|l@{}>{\ttfamily}l@{}>{ \ttfamily \%\% }l|}
      \hline
      &y[2]|&y'[0]\\
      &y[3]|&y'[1]\\
      &(y[4]-y[0])/((y[4]-y[0])\POW 2+(y[5]-y[1])\POW 2)\POW 1.5|&y'[2]=y''[0]\\
      &(y[5]-y[1])/((y[4]-y[0])\POW 2+(y[5]-y[1])\POW 2)\POW 1.5|&y'[3]=y''[1]\\
      &y[6]|&y'[4]\\
      &y[7]|&y'[5]\\
      &20*(y[0]-y[4])/((y[4]-y[0])\POW 2+(y[5]-y[1])\POW 2)\POW 1.5|&y'[6]=y''[4]\\
      &20*(y[1]-y[5])/((y[4]-y[0])\POW 2+(y[5]-y[1])\POW 2)\POW 1.5&y'[7]=y''[5]\\
      \hline
    \end{tabular}
    \caption{Algebraic description for the gravitational interaction}\label{intgravalgcode}
\end{table}

\newcommand\Grav{%
  /yp2 exch def /xp2 exch def /ay2 exch def /ax2 exch def
  /yp1 exch def /xp1 exch def /ay1 exch def /ax1 exch def
  /ro2 ax2 ax1 sub dup mul ay2 ay1 sub dup mul add def
  xp1 yp1
  ax2 ax1 sub ro2 sqrt div ro2 div
  ay2 ay1 sub ro2 sqrt div ro2 div
  xp2 yp2
  3 index -20 mul
  3 index -20 mul}
\newcommand\GravAlg{%
  y[2]|y[3]|%
  (y[4]-y[0])/((y[4]-y[0])^2+(y[5]-y[1])^2)^1.5|%
  (y[5]-y[1])/((y[4]-y[0])^2+(y[5]-y[1])^2)^1.5|%
  y[6]|y[7]|%
  20*(y[0]-y[4])/((y[4]-y[0])^2+(y[5]-y[1])^2)^1.5|%
  20*(y[1]-y[5])/((y[4]-y[0])^2+(y[5]-y[1])^2)^1.5}
%%  0  1   2   3  4  5   6   7
%% x1 y1 x'1 y'1 x2 y2 x'2 y'2


\begin{LTXexample}[width=5cm,wide]
\def\InitCond{ 1  1  .1  0 -1 -1  -2   0}
\begin{pspicture}[shift=-2,showgrid=true](-3,-1.75)(2,1.5)
  \psplotDiffEqn[whichabs=0, whichord=1, linecolor=blue, method=rk4, plotpoints=100]{0}{3.95}{\InitCond}{\Grav}
  \psset{showpoints=true,whichabs=4, whichord=5}
  \psplotDiffEqn[linecolor=black, method=varrkiv, varsteptol=.0001, plotpoints=200]{0}{3.9}{\InitCond}{\Grav}
\end{pspicture}
\end{LTXexample}
\vspace{-2ex}
{\captionof{figure}{Gravitational interaction: fixed landmark, trajectory of the stars}\label{fig:InterGravRepFix}}



\bigskip
\begin{LTXexample}[width=5cm,wide]
\def\InitCond{ 1  1  .1  0 -1 -1  -2   0}
\begin{pspicture}[shift=-1.5,showgrid=true](-4,-1.75)(1,1)
  \psplotDiffEqn[linecolor=red, plotpoints=200,method=varrkiv, varsteptol=.0001, showpoints=true,
      plotfuncx=y dup 4 get exch 0 get sub,
      plotfuncy=dup 5 get exch 1 get sub ]{0}{3.9}{\InitCond}{\Grav}
\end{pspicture}
\end{LTXexample}
\vspace{-2ex}
{\captionof{figure}{Gravitational interaction : landmark defined by one star}\label{fig:IGnewrep}}


\begin{center}
\bgroup
\def\InitCond{ 1  1  .1   0 -1 -1  -2   0}
\psset{xunit=2}
\begin{pspicture}[showgrid=true](0,0)(8,9)
  \psset{showpoints=true}
  \psplotDiffEqn[linecolor=red, method=varrkiv, plotpoints=2, varsteptol=.0001,
      plotfuncy=dup 6 get dup mul exch 7 get dup mul add sqrt]{0}{8}{\InitCond}{\Grav}
  \psplotDiffEqn[linecolor=blue, method=varrkiv, plotpoints=2, varsteptol=.0001,
      plotfuncy=dup 2 get dup mul exch 3 get dup mul add sqrt]{0}{8}{\InitCond}{\Grav}
\end{pspicture}
\captionof{figure}{Gravitational interaction : speeds of the
stars} \egroup
\end{center}

\begin{lstlisting}
\psset{xunit=2}
\begin{pspicture}[showgrid=true](0,0)(8,9)
  \psset{showpoints=true}
  \psplotDiffEqn[linecolor=red, method=varrkiv, plotpoints=2, varsteptol=.0001,
      plotfuncy=dup 6 get dup mul exch 7 get dup mul add sqrt]{0}{8}{\InitCond}{\Grav}
  \psplotDiffEqn[linecolor=blue, method=varrkiv, plotpoints=2, varsteptol=.0001,
      plotfuncy=dup 2 get dup mul exch 3 get dup mul add sqrt]{0}{8}{\InitCond}{\Grav}
\end{pspicture}
\end{lstlisting}

%--------------------------------------------------------------------------------------
\clearpage
\subsubsection{Simple equation of first order $y'=y$}
%--------------------------------------------------------------------------------------

For the initial value $y(0)=1$ we have the solution $y(x)=e^x$. $y$ is always
on the stack, so we have to do nothing. Using the \Lkeyword{algebraic=true} option, we write it
as \verb$y[0]$. The following example shows different solutions depending to the number of plotpoints
with $y_0=1$:


\begin{center}
\bgroup
\psset{xunit=4, yunit=.4}
\begin{pspicture}(3,19)\psgrid[subgriddiv=1]
  \psplot[linewidth=6\pslinewidth, linecolor=green]{0}{3}{Euler x exp}
  \psplotDiffEqn[linecolor=magenta,plotpoints=16,algebraic=true]{0}{3}{1}{y[0]}
  \psplotDiffEqn[linecolor=blue,plotpoints=151]{0}{3}{1}{}
  \psplotDiffEqn[linecolor=red,method=rk4,plotpoints=15]{0}{3}{1}{}
  \psplotDiffEqn[linecolor=Orange,method=rk4,plotpoints=4]{0}{3}{1}{}
  \psset{linewidth=4\pslinewidth}
  \rput*(0.35,19){\psline[linecolor=magenta](-.75cm,0)}
  \rput*[l](0.35,19){\small Euler order 1 $h=0{,}2$}
  \rput*(0.35,17){\psline[linecolor=blue](-.75cm,0)}
  \rput*[l](0.35,17){\small Euler order 1 $h=0{,}02$}
  \rput*(0.35,15){\psline[linecolor=Orange](-.75cm,0)}
  \rput*[l](0.35,15){\small RK ordre 4 $h=1$}
  \rput*(0.35,13){\psline[linecolor=red](-.75cm,0)}
  \rput*[l](0.35,13){\small RK ordre 4 $h=0{,}2$}
  \rput*(0.35,11){\psline[linecolor=green](-.75cm,0)}
  \rput*[l](0.35,11){\small solution exacte}
\end{pspicture}
\egroup
\end{center}

\begin{lstlisting}
\psset{xunit=4, yunit=.4}
\begin{pspicture}(3,19)\psgrid[subgriddiv=1]
  \psplot[linewidth=6\pslinewidth, linecolor=green]{0}{3}{Euler x exp}
  \psplotDiffEqn[linecolor=magenta,plotpoints=16,algebraic=true]{0}{3}{1}{y[0]}
  \psplotDiffEqn[linecolor=blue,plotpoints=151]{0}{3}{1}{}
  \psplotDiffEqn[linecolor=red,method=rk4,plotpoints=15]{0}{3}{1}{}
  \psplotDiffEqn[linecolor=Orange,method=rk4,plotpoints=4]{0}{3}{1}{}
  \psset{linewidth=4\pslinewidth}
  \rput*(0.35,19){\psline[linecolor=magenta](-.75cm,0)}
  \rput*[l](0.35,19){\small Euler order 1 $h=0{,}2$}
  \rput*(0.35,17){\psline[linecolor=blue](-.75cm,0)}
  \rput*[l](0.35,17){\small Euler order 1 $h=0{,}02$}
  \rput*(0.35,15){\psline[linecolor=Orange](-.75cm,0)}
  \rput*[l](0.35,15){\small RK ordre 4 $h=1$}
  \rput*(0.35,13){\psline[linecolor=red](-.75cm,0)}
  \rput*[l](0.35,13){\small RK ordre 4 $h=0{,}2$}
  \rput*(0.35,11){\psline[linecolor=green](-.75cm,0)}
  \rput*[l](0.35,11){\small solution exacte}
\end{pspicture}
\end{lstlisting}

%--------------------------------------------------------------------------------------
\clearpage
\subsubsection{$y'=\displaystyle\frac{2-ty}{4-t^2}$}% $
%--------------------------------------------------------------------------------------

For the initial value $y(0)=1$ the exact solution is
$y(x)=\displaystyle\frac{t+\sqrt{4-t^2}}{2}$. The function $f$
described in PostScript code is like (y is still on the stack):
\begin{lstlisting}[style=syntax]
x              %% y x
mul            %% x*y
2 exch sub     %% 2-x*y
4 x dup mul    %% 2-x*y 4 x^2
sub            %% 2-x*y 4-x^2
div            %% (2-x*y)/(4-x^2)
\end{lstlisting}
\noindent
The following example uses $y_0=1$.

\begin{lstlisting}[style=syntax]
\newcommand{\InitCond}{1}
\newcommand{\Func}{x mul 2 exch sub 4 x dup mul sub div}
\newcommand{\FuncAlg}{(2-x*y[0])/(4-x^2)}
\end{lstlisting}

\begin{center}
\bgroup
\psset{xunit=6.4, yunit=9.6, showpoints=false}
\begin{pspicture}(0,1)(2,1.5)  \psgrid[griddots=10](0,1)(2,1.5)
  { \psset{linewidth=4\pslinewidth,linecolor=lightgray}
  \psplot{0}{1.8}{x dup dup mul 4 exch sub sqrt add 2 div}
  \psplot{1.8}{2}{x dup dup mul 4 exch sub sqrt add 2 div} }
  \def\InitCond{1}
  \def\Func{x mul 2 exch sub 4 x dup mul sub div}
  \psplotDiffEqn[linecolor=magenta, plotpoints=20]{0}{1.9}{\InitCond}{\Func}
  \psplotDiffEqn[linecolor=blue, plotpoints=191]{0}{1.9}{\InitCond}{\Func}
  \psplotDiffEqn[linecolor=red, method=rk4, plotpoints=11,%
     algebraic=true]{0}{1.9}{\InitCond}{(2-x*y[0])/(4-x^2)}
  \psplotDiffEqn[linecolor=Orange, method=rk4, plotpoints=21,%
     algebraic=true]{0}{1.9}{\InitCond}{(2-x*y[0])/(4-x^2)}
  \psset{linewidth=4\pslinewidth}\small
  \rput*(0,1.4){\psline[linecolor=magenta](-.75cm,0)}\rput*[l](0,1.4){Euler order 1 $h=0{,}1$}
  \rput*(0,1.35){\psline[linecolor=blue](-.75cm,0)}\rput*[l](0,1.35){Euler order 1 $h=0{,}01$}
  \rput*(0,1.3){\psline[linecolor=Orange](-.75cm,0)}\rput*[l](0,1.3){RK order 4 $h=0{,}19$}
  \rput*(0,1.25){\psline[linecolor=red](-.75cm,0)}\rput*[l](0,1.25){RK order 4 $h=0{,}095$}
  \rput*(0,1.2){\psline[linecolor=lightgray](-.75cm,0)}\rput*[l](0,1.2){exactly}
\end{pspicture}
\egroup
\end{center}

\begin{lstlisting}[xrightmargin=-1cm,xleftmargin=-1cm]
\psset{xunit=6.4, yunit=9.6, showpoints=false}
\begin{pspicture}(0,1)(2,1.7)  \psgrid[subgriddiv=5]
  { \psset{linewidth=4\pslinewidth,linecolor=lightgray}
  \psplot{0}{1.8}{x dup dup mul 4 exch sub sqrt add 2 div}
  \psplot{1.8}{2}{x dup dup mul 4 exch sub sqrt add 2 div} }
  \def\InitCond{1}
  \def\Func{x mul 2 exch sub 4 x dup mul sub div}
  \psplotDiffEqn[linecolor=magenta, plotpoints=20]{0}{1.9}{\InitCond}{\Func}
  \psplotDiffEqn[linecolor=blue, plotpoints=191]{0}{1.9}{\InitCond}{\Func}
  \psplotDiffEqn[linecolor=red, method=rk4, plotpoints=11,%
     algebraic=true]{0}{1.9}{\InitCond}{(2-x*y[0])/(4-x^2)}
  \psplotDiffEqn[linecolor=Orange, method=rk4, plotpoints=21,%
     algebraic=true]{0}{1.9}{\InitCond}{(2-x*y[0])/(4-x^2)}
  \psset{linewidth=4\pslinewidth}
  \rput*(0.3,1.6){\psline[linecolor=magenta](-.75cm,0)}\rput*[l](0.3,1.6){\small Euler order 1 $h=0{,}1$}
  \rput*(0.3,1.55){\psline[linecolor=blue](-.75cm,0)}\rput*[l](0.3,1.55){\small Euler order 1 $h=0{,}01$}
  \rput*(0.3,1.5){\psline[linecolor=Orange](-.75cm,0)}\rput*[l](0.3,1.5){\small RK order 4 $h=0{,}19$}
  \rput*(0.3,1.45){\psline[linecolor=red](-.75cm,0)}\rput*[l](0.3,1.45){\small RK order 4 $h=0{,}095$}
  \rput*(0.3,1.4){\psline[linecolor=lightgray](-.75cm,0)}\rput*[l](0.3,1.4){\small exactly}
\end{pspicture}
\end{lstlisting}


%--------------------------------------------------------------------------------------
\clearpage
\subsubsection{$y'=-2xy$}
%--------------------------------------------------------------------------------------

For $y(-1)=\frac{1}{e}$ we get $y(x)=e^{-x^2}$.

\begin{center}
\bgroup
\psset{unit=4}
\begin{pspicture}(-1,0)(3,1.1)\psgrid
  \psplot[linewidth=4\pslinewidth,linecolor=gray]{-1}{3}{Euler x dup mul neg exp}
  \psset{plotpoints=9}
  \psplotDiffEqn[linecolor=cyan]{-1}{3}{1 Euler div}{x -2 mul mul}
  \psplotDiffEqn[linecolor=yellow, method=rk4]{-1}{3}{1 Euler div}{x -2 mul mul}
  \psset{plotpoints=21}
  \psplotDiffEqn[linecolor=blue]{-1}{3}{1 Euler div}{x -2 mul mul}
  \psplotDiffEqn[linecolor=Orange, method=rk4]{-1}{3}{1 Euler div}{x -2 mul mul}
  \psset{linewidth=2\pslinewidth}
  \rput*(2,1){\psline[linecolor=Orange](-0.25,0)}
  \rput*[l](2,1){RK}
  \rput*(2,.9){\psline[linecolor=blue](-0.25,0)}
  \rput*[l](2,.9){\textsc{Euler}-1}
  \rput*(2,.8){\psline[linecolor=gray](-0.25,0)}
  \rput*[l](2,.8){solution}
\end{pspicture}
\egroup
\end{center}


\begin{lstlisting}
\psset{unit=4}
\begin{pspicture}(-1,0)(3,1.1)\psgrid
  \psplot[linewidth=4\pslinewidth,linecolor=gray]{-1}{3}{Euler x dup mul neg exp}
  \psset{plotpoints=9}
  \psplotDiffEqn[linecolor=cyan]{-1}{3}{1 Euler div}{x -2 mul mul}
  \psplotDiffEqn[linecolor=yellow, method=rk4]{-1}{3}{1 Euler div}{x -2 mul mul}
  \psset{plotpoints=21}
  \psplotDiffEqn[linecolor=blue]{-1}{3}{1 Euler div}{x -2 mul mul}
  \psplotDiffEqn[linecolor=Orange, method=rk4]{-1}{3}{1 Euler div}{x -2 mul mul}
  \psset{linewidth=2\pslinewidth}
  \rput*(2,1){\psline[linecolor=Orange](-0.25,0)}
  \rput*[l](2,1){RK}
  \rput*(2,.9){\psline[linecolor=blue](-0.25,0)}
  \rput*[l](2,.9){\textsc{Euler}-1}
  \rput*(2,.8){\psline[linecolor=gray](-0.25,0)}
  \rput*[l](2,.8){solution}
\end{pspicture}
\end{lstlisting}


%--------------------------------------------------------------------------------------
\clearpage
\subsubsection{Spiral of Cornu}
%--------------------------------------------------------------------------------------

The integrals of \Index{Fresnel}:
\begin{align} x & =\int^t_0\cos\frac{\pi t^2}{2}\mathrm{d}t \\
 y & =\int^t_0\sin\frac{\pi t^2}{2}\mathrm{d}t \\
\intertext{with}
 \dot{x} &= \cos\frac{\pi t^2}{2} \\
 \dot{y} & =\sin\frac{\pi t^2}{2}
 \end{align}

\begin{lstlisting}
\psset{unit=8}
\begin{pspicture}(1,1)\psgrid[subgriddiv=5]
  \psplotDiffEqn[whichabs=0,whichord=1,linecolor=red,method=rk4,algebraic=true,%
     plotpoints=500,showpoints=true]{0}{10}{0 0}{cos(Pi*x^2/2)|sin(Pi*x^2/2)}
\end{pspicture}
\end{lstlisting}


\begin{center}
\bgroup
\psset{unit=8}
\begin{pspicture}(1,1)\psgrid[subgriddiv=5]
  \psplotDiffEqn[whichabs=0,whichord=1,linecolor=red,method=rk4,algebraic=true,%
     plotpoints=500,showpoints=true]{0}{10}{0 0}{cos(Pi*x^2/2)|sin(Pi*x^2/2)}
\end{pspicture}
\egroup
\end{center}



%--------------------------------------------------------------------------------------
\clearpage
\subsubsection{Lotka-Volterra}
%--------------------------------------------------------------------------------------

The Lotka-Volterra model describes interactions between two species in an ecosystem, a 
predator and a prey. This represents our first multi-species model. Since we are considering 
two species, the model will involve two equations, one which describes how the prey 
population changes and the second which describes how the predator population changes.

For concreteness let us assume that the prey in our model are rabbits, and that the 
predators are foxes. If we let $R(t)$ and $F(t)$ represent the number of rabbits and 
foxes, respectively, that are alive at time t, then the Lotka-Volterra model is:
%
\begin{align}
\dot R &= a\cdot R - b\cdot R\cdot F\\
\dot F &= e\cdot b\cdot R\cdot F - c\cdot F
\end{align}
%
where the parameters are defined by:
\begin{description}
\item[a] is the natural growth rate of rabbits in the absence of predation,
\item[c] is the natural death rate of foxes in the absence of food (rabbits),
\item[b] is the death rate per encounter of rabbits due to predation,
\item[e] is the efficiency of turning predated rabbits into foxes.
\end{description}

The Stella model representing the \Index{Lotka-Volterra} model will be slightly more complex than the 
single species models we've dealt with before. The main difference is that our model will have 
two stocks (reservoirs), one for each species. Each species will have its own birth and death 
rates. In addition, the Lotka-Volterra model involves four parameters rather than two. All told, 
the Stella representation of the Lotka-Volterra model will use two stocks, four flows, four 
converters and many connectors.

\bgroup
\begin{center}
\def\InitCond{ 0 10 10}%% xa ya xl
\def\Faiglelapin{\Vaigle*(y[2]-y[0])/sqrt(y[1]^2+(y[2]-y[0])^2)|%
                 -\Vaigle*y[1]/sqrt(y[1]^2+(y[2]-y[0])^2)|%
                 -\Vlapin}
\def\Vlapin{1}  \def\Vaigle{1.6}
\psset{unit=.7,subgriddiv=0,gridcolor=lightgray,method=adams,algebraic=true,%
   plotpoints=20,showpoints=true}
\begin{pspicture}[showgrid=true](-3,-3)(10,10)
 \psplotDiffEqn[plotfuncy=pop 0,whichabs=2,linecolor=red]{0}{10}{\InitCond}{\Faiglelapin}
 \psplotDiffEqn[whichabs=0,whichord=1,linecolor=black,method=rk4]{0}{10}{\InitCond}{\Faiglelapin}
  \psplotDiffEqn[whichabs=0,whichord=1,linecolor=blue]{0}{10}{\InitCond}{\Faiglelapin}
\end{pspicture}
\end{center}

\begin{lstlisting}[label={fig:aiglelapin},xrightmargin=-1.5cm]
\def\InitCond{ 0 10 10}%% xa ya xl
\def\Faiglelapin{\Vaigle*(y[2]-y[0])/sqrt(y[1]^2+(y[2]-y[0])^2)|%
                 -\Vaigle*y[1]/sqrt(y[1]^2+(y[2]-y[0])^2)|%
                 -\Vlapin}
\def\Vlapin{1}  \def\Vaigle{1.6}
\psset{unit=.7,subgriddiv=0,gridcolor=lightgray,method=adams,algebraic=true,%
   plotpoints=20,showpoints=true}
\begin{pspicture}[showgrid=true](-3,-3)(10,10)
 \psplotDiffEqn[plotfuncy=pop 0,whichabs=2,linecolor=red]{0}{10}{\InitCond}{\Faiglelapin}
 \psplotDiffEqn[whichabs=0,whichord=1,linecolor=black,method=rk4]{0}{10}{\InitCond}{\Faiglelapin}
  \psplotDiffEqn[whichabs=0,whichord=1,linecolor=blue]{0}{10}{\InitCond}{\Faiglelapin}
\end{pspicture}
\end{lstlisting}


\begin{center}
\def\InitCond{ 0 10 10}%% xa ya xl
\def\Faiglelapin{\Vaigle*(y[2]-y[0])/sqrt(y[1]^2+(y[2]-y[0])^2)|%
                 -\Vaigle*y[1]/sqrt(y[1]^2+(y[2]-y[0])^2)|%
                 -\Vlapin}
\def\Vlapin{1}  \def\Vaigle{1.6}
\psset{unit=.7,subgriddiv=0,gridcolor=lightgray,method=adams,algebraic=true,%
   plotpoints=20,showpoints=true}
\begin{pspicture}[showgrid=true](0,-0.25)(10,14)
 \psplotDiffEqn[plotfuncy=dup 1 get dup mul exch dup 0 get exch 2 get sub dup
    mul add sqrt,linecolor=red,method=rk4]{0}{10}{\InitCond}{\Faiglelapin}
 \psplotDiffEqn[plotfuncy=dup 1 get dup mul exch dup 0 get exch 2 get sub dup
    mul add sqrt,linecolor=blue]{0}{10}{\InitCond}{\Faiglelapin}
 \psplotDiffEqn[plotfuncy=pop Func aload pop pop dup mul exch dup mul add sqrt,
    linecolor=yellow]{0}{10}{\InitCond}{\Faiglelapin}
\end{pspicture}
\end{center}
\egroup

\begin{lstlisting}[label={fig:aiglelapin},xrightmargin=-1.5cm]
\def\InitCond{ 0 10 10}%% xa ya xl
\def\Faiglelapin{\Vaigle*(y[2]-y[0])/sqrt(y[1]^2+(y[2]-y[0])^2)|%
                 -\Vaigle*y[1]/sqrt(y[1]^2+(y[2]-y[0])^2)|%
                 -\Vlapin}
\def\Vlapin{1}  \def\Vaigle{1.6}
\psset{unit=.7,subgriddiv=0,gridcolor=lightgray,method=adams,algebraic=true,%
   plotpoints=20,showpoints=true}
\begin{pspicture}[showgrid=true](10,12)
 \psplotDiffEqn[plotfuncy=dup 1 get dup mul exch dup 0 get exch 2 get sub dup
    mul add sqrt,linecolor=red,method=rk4]{0}{10}{\InitCond}{\Faiglelapin}
 \psplotDiffEqn[plotfuncy=dup 1 get dup mul exch dup 0 get exch 2 get sub dup
    mul add sqrt,linecolor=blue]{0}{10}{\InitCond}{\Faiglelapin}
 \psplotDiffEqn[plotfuncy=pop Func aload pop pop dup mul exch dup mul add sqrt,
    linecolor=yellow]{0}{10}{\InitCond}{\Faiglelapin}
\end{pspicture}
\end{lstlisting}


%--------------------------------------------------------------------------------------
\subsubsection{$y''=y$}
%--------------------------------------------------------------------------------------

Beginning with the initial equation $\displaystyle y(x)=Ae^x+Be^{-x}$ we get the hyperbolic
trigonometrical functions.

\begin{center}
\bgroup
\def\Funct{exch}   \psset{xunit=5cm, yunit=0.75cm}
\begin{pspicture}(0,-0.25)(2,7)\psgrid[subgriddiv=1,griddots=10]
 \psplot[linewidth=4\pslinewidth, linecolor=green]{0}{2}{Euler x exp}  %%e^x
 \psplotDiffEqn[linecolor=magenta, plotpoints=11]{0}{2}{1 1}{\Funct}
 \psplotDiffEqn[linecolor=blue, plotpoints=101]{0}{2}{1 1}{\Funct}
 \psplotDiffEqn[linecolor=red, method=rk4, plotpoints=11]{0}{2}{1 1}{\Funct}
 \psplot[linewidth=4\pslinewidth, linecolor=green]{0}{2}{Euler dup x exp  %%ch(x)
    exch x neg exp add 2 div}
 \psplotDiffEqn[linecolor=magenta, plotpoints=11]{0}{2}{1 0}{\Funct}
 \psplotDiffEqn[linecolor=blue, plotpoints=101]{0}{2}{1 0}{\Funct}
 \psplotDiffEqn[linecolor=red, method=rk4, plotpoints=11]{0}{2}{1 0}{\Funct}
 \psplot[linewidth=4\pslinewidth, linecolor=green]{0}{2}{Euler dup x exp
     exch x neg exp sub 2 div}  %%sh(x)
 \psplotDiffEqn[linecolor=magenta, plotpoints=11]{0}{2}{0 1}{\Funct}
 \psplotDiffEqn[linecolor=blue, plotpoints=101]{0}{2}{0 1}{\Funct}
 \psplotDiffEqn[linecolor=red, method=rk4, plotpoints=11]{0}{2}{0 1}{\Funct}
 \rput*(1.3,.9){\psline[linecolor=magenta](-.75cm,0)}\rput*[l](1.3,.9){\small\textsc{Euler} order 1 $h=1$}
 \rput*(1.3,.8){\psline[linecolor=blue](-.75cm,0)}\rput*[l](1.3,.8){\small\textsc{Euler} order 1 $h=0{,}1$}
 \rput*(1.3,.7){\psline[linecolor=red](-.75cm,0)}\rput*[l](1.3,.7){\small RK order 4 $h=1$}
 \rput*(1.3,.6){\psline[linecolor=green](-.75cm,0)}\rput*[l](1.3,.6){\small exact solution}
\end{pspicture}
\egroup
\end{center}

\begin{lstlisting}[label={fig:minusexp},xrightmargin=-1.5cm]
\def\Funct{exch}   \psset{xunit=5cm, yunit=0.75cm}
\begin{pspicture}(0,-0.25)(2,7)\psgrid[subgriddiv=1,griddots=10]
 \psplot[linewidth=4\pslinewidth, linecolor=green]{0}{2}{Euler x exp}  %%e^x
 \psplotDiffEqn[linecolor=magenta, plotpoints=11]{0}{2}{1 1}{\Funct}
 \psplotDiffEqn[linecolor=blue, plotpoints=101]{0}{2}{1 1}{\Funct}
 \psplotDiffEqn[linecolor=red, method=rk4, plotpoints=11]{0}{2}{1 1}{\Funct}
 \psplot[linewidth=4\pslinewidth, linecolor=green]{0}{2}{Euler dup x exp  %%ch(x)
    exch x neg exp add 2 div}
 \psplotDiffEqn[linecolor=magenta, plotpoints=11]{0}{2}{1 0}{\Funct}
 \psplotDiffEqn[linecolor=blue, plotpoints=101]{0}{2}{1 0}{\Funct}
 \psplotDiffEqn[linecolor=red, method=rk4, plotpoints=11]{0}{2}{1 0}{\Funct}
 \psplot[linewidth=4\pslinewidth, linecolor=green]{0}{2}{Euler dup x exp
     exch x neg exp sub 2 div}  %%sh(x)
 \psplotDiffEqn[linecolor=magenta, plotpoints=11]{0}{2}{0 1}{\Funct}
 \psplotDiffEqn[linecolor=blue, plotpoints=101]{0}{2}{0 1}{\Funct}
 \psplotDiffEqn[linecolor=red, method=rk4, plotpoints=11]{0}{2}{0 1}{\Funct}
 \rput*(1.3,.9){\psline[linecolor=magenta](-.75cm,0)}\rput*[l](1.3,.9){\small\textsc{Euler} order 1 $h=1$}
 \rput*(1.3,.8){\psline[linecolor=blue](-.75cm,0)}\rput*[l](1.3,.8){\small\textsc{Euler} order 1 $h=0{,}1$}
 \rput*(1.3,.7){\psline[linecolor=red](-.75cm,0)}\rput*[l](1.3,.7){\small RK order 4 $h=1$}
 \rput*(1.3,.6){\psline[linecolor=green](-.75cm,0)}\rput*[l](1.3,.6){\small exact solution}
\end{pspicture}
\end{lstlisting}

%--------------------------------------------------------------------------------------
\clearpage
\subsubsection{$y''=-y$}
%--------------------------------------------------------------------------------------
\begin{center}
\bgroup
\def\Funct{exch neg}
\psset{xunit=1, yunit=4}
\def\quatrepi{12.5663706144}%%4pi=12.5663706144
\begin{pspicture}(0,-1.25)(\quatrepi,1.25)\psgrid[subgriddiv=1,griddots=10]
 \psplot[linewidth=4\pslinewidth,linecolor=green]{0}{\quatrepi}{x RadtoDeg cos}%%cos(x)
 \psplotDiffEqn[linecolor=blue, plotpoints=201]{0}{3.1415926}{1 0}{\Funct}
 \psplotDiffEqn[linecolor=red, method=rk4, plotpoints=31]{0}{\quatrepi}{1 0}{\Funct}
 \psplot[linewidth=4\pslinewidth,linecolor=green]{0}{\quatrepi}{x RadtoDeg sin}  %%sin(x)
 \psplotDiffEqn[linecolor=blue,plotpoints=201]{0}{3.1415926}{0 1}{\Funct}
 \psplotDiffEqn[linecolor=red,method=rk4, plotpoints=31]{0}{\quatrepi}{0 1}{\Funct}
 \rput*(3.3,.9){\psline[linecolor=magenta](-.75cm,0)}\rput*[l](3.3,.9){\small Euler order 1 $h=1$}
 \rput*(3.3,.8){\psline[linecolor=blue](-.75cm,0)}\rput*[l](3.3,.8){\small Euler order 1 $h=0{,}1$}
 \rput*(3.3,.7){\psline[linecolor=red](-.75cm,0)}\rput*[l](3.3,.7){\small RK order 4 $h=1$}
 \rput*(3.3,.6){\psline[linecolor=green](-.75cm,0)}\rput*[l](3.3,.6){\small exact solution}
\end{pspicture}
\egroup
\end{center}

\begin{lstlisting}[label={fig:minusexp2}]
\def\Funct{exch neg}
\psset{xunit=1, yunit=4}
\def\quatrepi{12.5663706144}%%4pi=12.5663706144
\begin{pspicture}(0,-1.25)(\quatrepi,1.25)\psgrid[subgriddiv=1,griddots=10]
 \psplot[linewidth=4\pslinewidth,linecolor=green]{0}{\quatrepi}{x RadtoDeg cos}%%cos(x)
 \psplotDiffEqn[linecolor=blue, plotpoints=201]{0}{3.1415926}{1 0}{\Funct}
 \psplotDiffEqn[linecolor=red, method=rk4, plotpoints=31]{0}{\quatrepi}{1 0}{\Funct}
 \psplot[linewidth=4\pslinewidth,linecolor=green]{0}{\quatrepi}{x RadtoDeg sin}  %%sin(x)
 \psplotDiffEqn[linecolor=blue,plotpoints=201]{0}{3.1415926}{0 1}{\Funct}
 \psplotDiffEqn[linecolor=red,method=rk4, plotpoints=31]{0}{\quatrepi}{0 1}{\Funct}
 \rput*(3.3,.9){\psline[linecolor=magenta](-.75cm,0)}\rput*[l](3.3,.9){\small Euler order 1 $h=1$}
 \rput*(3.3,.8){\psline[linecolor=blue](-.75cm,0)}\rput*[l](3.3,.8){\small Euler order 1 $h=0{,}1$}
 \rput*(3.3,.7){\psline[linecolor=red](-.75cm,0)}\rput*[l](3.3,.7){\small RK order 4 $h=1$}
 \rput*(3.3,.6){\psline[linecolor=green](-.75cm,0)}\rput*[l](3.3,.6){\small exact solution}
\end{pspicture}
\end{lstlisting}

%--------------------------------------------------------------------------------------
\clearpage
\subsubsection{The mechanical pendulum: $y''=-\frac{g}{l}\sin(y)$}% $
%--------------------------------------------------------------------------------------

For small \Index{oscillation}s $\sin(y)\simeq y$:

\[ y(x)=y_0\cos\left(\sqrt{\frac{g}{l}}x\right) \]

The function $f$ is written in PostScript code:

\begin{lstlisting}[style=syntax]
exch RadtoDeg sin -9.8 mul %% y' -gsin(y)
\end{lstlisting}

\begin{center}
\bgroup
\def\Func{y[1]|-9.8*sin(y[0])}
\psset{yunit=2,xunit=4,algebraic=true,linewidth=1.5pt}
\begin{pspicture}(0,-2.25)(3,2.25)
  \psaxes{->}(0,0)(0,-2)(3,2)
  \psplot[linewidth=3\pslinewidth, linecolor=Orange]{0}{3}{.1*cos(sqrt(9.8)*x)}
  \psset{method=rk4,plotpoints=50,linecolor=blue}
  \psplotDiffEqn{0}{3}{.1 0}{\Func}
  \psplot[linewidth=3\pslinewidth,linecolor=Orange]{0}{3}{.25*cos(sqrt(9.8)*x)}
  \psplotDiffEqn{0}{3}{.25 0}{\Func}
  \psplotDiffEqn{0}{3}{.5 0}{\Func}
  \psplotDiffEqn{0}{3}{1 0}{\Func}
  \psplotDiffEqn[plotpoints=100]{0}{3}{Pi 2 div 0}{\Func}
\end{pspicture}
\egroup
\end{center}

\begin{lstlisting}[label=fig:second]
\def\Func{y[1]|-9.8*sin(y[0])}
\psset{yunit=2,xunit=4,algebraic=true,linewidth=1.5pt}
\begin{pspicture}(0,-2.25)(3,2.25)
  \psaxes{->}(0,0)(0,-2)(3,2)
  \psplot[linewidth=3\pslinewidth, linecolor=Orange]{0}{3}{.1*cos(sqrt(9.8)*x)}
  \psset{method=rk4,plotpoints=50,linecolor=blue}
  \psplotDiffEqn{0}{3}{.1 0}{\Func}
  \psplot[linewidth=3\pslinewidth,linecolor=Orange]{0}{3}{.25*cos(sqrt(9.8)*x)}
  \psplotDiffEqn{0}{3}{.25 0}{\Func}
  \psplotDiffEqn{0}{3}{.5 0}{\Func}
  \psplotDiffEqn{0}{3}{1 0}{\Func}
  \psplotDiffEqn[plotpoints=100]{0}{3}{Pi 2 div 0}{\Func}
\end{pspicture}
\end{lstlisting}

%--------------------------------------------------------------------------------------
\clearpage
\subsubsection{$y''=-\frac{y'}{4}-2y$}% $
%--------------------------------------------------------------------------------------

For $y_0=5$ and $y'_0=0$ the solution is:

\[
5e^{-\frac{x}{8}}\left(\cos\left(\omega x\right)+\frac{\sin(\omega x)}{8\omega}\right)
\mbox{ avec } \omega=\frac{\sqrt{127}}{8}
\]

\begin{center}
\bgroup
\psset{xunit=.6,yunit=0.8,plotpoints=500}
\begin{pspicture}(0,-4.25)(26,5.25)
  \psaxes{->}(0,0)(0,-4)(26,5)
  \psplot[plotpoints=200,linewidth=4\pslinewidth,linecolor=gray]{0}{26}{%
     Euler x -8 div exp x 127 sqrt 8 div mul RadtoDeg dup cos 5 mul exch sin 127 sqrt div 5 mul add mul}
  \psplotDiffEqn[linecolor=red,linewidth=5\pslinewidth]{0}{26}{5 0}
     {dup 3 1 roll -4 div exch 2 mul sub}
  \psplotDiffEqn[linecolor=black,algebraic=true]{0}{26}{5 0} {y[1]|-y[1]/4-2*y[0]}
  \psset{method=rk4, plotpoints=50}
  \psplotDiffEqn[linecolor=blue,linewidth=5\pslinewidth]{0}{26}{5 0}{%
      dup 3 1 roll -4 div exch 2 mul sub}
  \psplotDiffEqn[linecolor=black,algebraic=true]{0}{26}{5 0}{y[1]|-y[1]/4-2*y[0]}
\end{pspicture}
\egroup
\end{center}

\begin{lstlisting}
\psset{xunit=.6,yunit=0.8,plotpoints=500}
\begin{pspicture}(0,-4.25)(26,5.25)
  \psaxes{->}(0,0)(0,-4)(26,5)
  \psplot[plotpoints=200,linewidth=4\pslinewidth,linecolor=gray]{0}{26}{%
     Euler x -8 div exp x 127 sqrt 8 div mul RadtoDeg dup cos 5 mul exch sin 127 sqrt div 5 mul add mul}
  \psplotDiffEqn[linecolor=red,linewidth=5\pslinewidth]{0}{26}{5 0}
     {dup 3 1 roll -4 div exch 2 mul sub}
  \psplotDiffEqn[linecolor=black,algebraic=true]{0}{26}{5 0} {y[1]|-y[1]/4-2*y[0]}
  \psset{method=rk4, plotpoints=50}
  \psplotDiffEqn[linecolor=blue,linewidth=5\pslinewidth]{0}{26}{5 0}{%
      dup 3 1 roll -4 div exch 2 mul sub}
  \psplotDiffEqn[linecolor=black,algebraic=true]{0}{26}{5 0}{y[1]|-y[1]/4-2*y[0]}
\end{pspicture}
\end{lstlisting}


\clearpage
\subsection{Save final state of a equation}
With the macros \Lcs{BeginSaveFinalState} and \Lcs{EndSaveFinalState} the
end values of a differential equation
can be saved and then used with the optional argument \Lkeyword{GetFinalState}  
as starting values for another equation.

\begin{lstlisting}
\psset{unit=10cm,linewidth=2pt}
\begin{pspicture}(1,1)\psgrid[subgridcolor=black!20,subgriddiv=20]
\BeginSaveFinalState
 \psplotDiffEqn[
   whichabs=0,whichord=1,linecolor=red,method=rk4,
   plotpoints=10,showpoints=true]{0}{1}{0 0}{
   pop pop
   x dup mul 2 div 180 mul cos %% dx/dt
   x dup mul 2 div 180 mul sin %% dy/dt
 }
 \psplotDiffEqn[GetFinalState,
   whichabs=0,whichord=1,linecolor=blue,method=rk4,%SaveFinalState,
   plotpoints=10,showpoints=true]{1}{2}{0 0}{
   pop pop
   x dup mul 2 div 180 mul cos %% dx/dt
   x dup mul 2 div 180 mul sin %% dy/dt
 }
 \psplotDiffEqn[GetFinalState,
   whichabs=0,whichord=1,linecolor=cyan,method=rk4,%SaveFinalState,
   plotpoints=19,showpoints=true]{2}{3}{0 0 }{
   pop pop
   x dup mul 2 div 180 mul cos %% dx/dt
   x dup mul 2 div 180 mul sin %% dy/dt
 }
\EndSaveFinalState
\end{pspicture}
\end{lstlisting}


\bigskip
\begin{center}
\psset{unit=6cm,linewidth=2pt}
\begin{pspicture}(1,1)\psgrid[subgridcolor=black!20,subgriddiv=20]
\BeginSaveFinalState
 \psplotDiffEqn[
   whichabs=0,whichord=1,linecolor=red,method=rk4,
   plotpoints=10,showpoints=true]{0}{1}{0 0}{
   pop pop
   x dup mul 2 div 180 mul cos %% dx/dt
   x dup mul 2 div 180 mul sin %% dy/dt
 }
 \psplotDiffEqn[GetFinalState,
   whichabs=0,whichord=1,linecolor=blue,method=rk4,%SaveFinalState,
   plotpoints=10,showpoints=true]{1}{2}{0 0}{
   pop pop
   x dup mul 2 div 180 mul cos %% dx/dt
   x dup mul 2 div 180 mul sin %% dy/dt
 }
 \psplotDiffEqn[GetFinalState,
   whichabs=0,whichord=1,linecolor=cyan,method=rk4,%SaveFinalState,
   plotpoints=19,showpoints=true]{2}{3}{0 0 }{
   pop pop
   x dup mul 2 div 180 mul cos %% dx/dt
   x dup mul 2 div 180 mul sin %% dy/dt
 }
\EndSaveFinalState
\end{pspicture}
\end{center}

\psset{unit=1cm,linewidth=0.75pt}


%--------------------------------------------------------------------------------------
\clearpage
\section{\nxLcs{psMatrixPlot}}\label{sec:psMatrix}
%--------------------------------------------------------------------------------------
\begin{filecontents}{matrix.data}
/dotmatrix [ %
0  1  1  0  0  0  0  1  1  1
0  1  1  0  1  1  1  0  1  0
1  0  1  1  0  0  0  1  1  0
0  0  1  0  0  0  0  0  1  1
1  1  1  1  1  0  1  0  0  1
0  0  1  1  0  1  0  1  1  1
1  0  0  0  1  1  0  0  0  1
0  0  0  1  1  1  0  1  1  0
1  1  0  0  0  0  1  0  0  1
1  0  1  0  0  1  1  1  0  0
] def
\end{filecontents}


This macro allows you to visualize a matrix. The datafile must be
defined as a PostScript matrix named \Lps{dotmatrix}:
\begin{lstlisting}[style=syntax]
/dotmatrix [ %  <------------ important line
0  1  1  0  0  0  0  1  1  1
0  1  1  0  1  1  1  0  1  0
1  0  1  1  0  0  0  1  1  0
0  0  1  0  0  0  0  0  1  1
1  1  1  1  1  0  1  0  0  1
0  0  1  1  0  1  0  1  1  1
1  0  0  0  1  1  0  0  0  1
0  0  0  1  1  1  0  1  1  0
1  1  0  0  0  0  1  0  0  1
1  0  1  0  0  1  1  1  0  0
] def        %  <------------ important line
\end{lstlisting}

Only the value 0 is important, in which case nothing happens, and
for all other cases a dot is printed. The syntax of the macro is:

\begin{BDef}
\Lcs{psMatrixPlot}\OptArgs\Largb{rows}\Largb{columns}\Largb{data file}
\end{BDef}

The \Index{matrix} is scanned line by line from the the first one to the
last. In general it appears as a bottom-to-top version of the
above listed matrix, the first row $0\,1\,1\,0\,0\,0\,0\,1\,1\,1$
is the first plotted line ($y=1$). With the option
\Lkeyword{ChangeOrder}=\true\ it looks exactly like the above view.

\bgroup
\begin{center}
\psscalebox{0.6}{%
\begin{pspicture}(-0.5,-0.75)(11,11)
  \psaxes{->}(11,11)
  \psMatrixPlot[dotsize=1.1cm,dotstyle=square*,linecolor=magenta]%
    {10}{10}{matrix.data}
  \psMatrixPlot[dotsize=.5cm,dotstyle=o,ChangeOrder]{10}{10}{matrix.data}
\end{pspicture}}\quad
\psscalebox{0.6}{%
\begin{pspicture}(-0.5,-0.75)(11,11)
  \psaxes[ticksize=-5pt 0]{->}(11,11)
  \psMatrixPlot[dotsize=1.1cm,dotstyle=square*,linecolor=magenta,XYoffset=-0.5]%
    {10}{10}{matrix.data}
  \psMatrixPlot[dotsize=.5cm,dotstyle=o,ChangeOrder,XYoffset=-0.5]{10}{10}{matrix.data}
\end{pspicture}}
\end{center}

\begin{lstlisting}
\psscalebox{0.6}{%
\begin{pspicture}(-0.5,-0.75)(11,11)
  \psaxes[ticksize=-5pt 0]{->}(11,11)
  \psMatrixPlot[dotsize=1.1cm,dotstyle=square*,linecolor=magenta]%
    {10}{10}{matrix.data}
  \psMatrixPlot[dotsize=.5cm,dotstyle=o,ChangeOrder]{10}{10}{matrix.data}
\end{pspicture}}\quad
\psscalebox{0.6}{%
\begin{pspicture}(-0.5,-0.75)(11,11)
  \psaxes{->}(11,11)
  \psMatrixPlot[dotsize=1.1cm,dotstyle=square*,linecolor=magenta,XYoffset=-0.5]%
    {10}{10}{matrix.data}
  \psMatrixPlot[dotsize=.5cm,dotstyle=o,ChangeOrder,XYoffset=-0.5]{10}{10}{matrix.data}
\end{pspicture}}
\end{lstlisting}

\begin{LTXexample}[pos=t,preset=\centering]
\begin{pspicture}(-0.5,-0.75)(11,11)
  \psaxes[ticksize=-5pt 0]{->}(11,11)
  \psMatrixPlot[dotscale=3,dotstyle=*,linecolor=blue]{10}{8}{matrix.data}
\end{pspicture}
\end{LTXexample}

\clearpage
With the \Lkeyword{colorType}=1 the data is printed as continous color
in the range of the wavelength. The smallest value of the data array
is set to red and the biggest value is set to violett. All other values
are substituted by the corresponding color of the wavlength.
\Lkeyword{colorType}=2 ist the same, but vice versa
with the color, from violet to red. \Lkeyword{colorType}=3 is the grayscale
image and \Lkeyword{colorType}=4 the same invers.

The following examples use a 200$\times$200
matrix data, which is saved as /dotmatrix [...] in the file \LFile{pstricks-add-doc.dat}.

\begin{LTXexample}[pos=t,preset=\centering]
\begin{pspicture}(10,10)
  \psMatrixPlot[colorType=1,xStep=0.05,yStep=0.05]{200}{200}{dotmatrix.data}
\end{pspicture}
\end{LTXexample}

\begin{LTXexample}[pos=t,preset=\centering]
\begin{pspicture}(10,10)
  \psMatrixPlot[colorType=2,xStep=0.05,yStep=0.05]{200}{200}{dotmatrix.data}
\end{pspicture}
\end{LTXexample}

\begin{LTXexample}[pos=t,preset=\centering]
\begin{pspicture}(10,10)
  \psMatrixPlot[colorType=3,xStep=0.05,yStep=0.05]{200}{200}{dotmatrix.data}
\end{pspicture}
\end{LTXexample}

\begin{LTXexample}[pos=t,preset=\centering]
\begin{pspicture}(10,10)
  \psMatrixPlot[colorType=4,xStep=0.05,yStep=0.05]{200}{200}{dotmatrix.data}
\end{pspicture}
\end{LTXexample}
\egroup

\clearpage
With the \Lkeyword{colorType}=5 the color setting can be user defined by the
optional argument \Lkeyword{colorTypeDef}. On the stack is the current value
which can be used for the setting but must be left on the stack when everything
is finished. The following example prints the 0 as color white, the value 1 as
black and all other values depending to the corresponding gray value.

\begin{filecontents*}{matrix1.data}
/dotmatrix [ % <------------ important line
3 0 0 0 0 0 0 0 1 2
0 0 0 0 0 0 0 1 2 1
8 0 0 0 0 0 1 2 1 0
0 0 0 0 0 1 2 1 0 0
0 0 0 0 1 2 1 0 0 0
9 0 0 1 2 1 3 0 0 0
0 0 1 2 1 4 0 0 0 0
0 1 2 1 5 0 0 0 0 0
1 2 1 6 0 0 0 0 0 0
2 1 7 0 0 0 0 0 0 3
] def % <------------ important line
\end{filecontents*}

\begin{center}
\psscalebox{0.7}{%
\begin{pspicture}(-0.5,-0.75)(11,11)
\psaxes[ticksize=-5pt 0]{->}(11,11)
\psMatrixPlot[
  colorType=5,
  colorTypeDef={
    dup /value exch def % save value and leave one on the stack
    value Min sub dMaxMin div neg 1 add 300 mul 400 add \pswavelengthToGRAY 
    value 0 eq \pslbrace 1 \psrbrace if % 
    value 1 eq \pslbrace 0 \psrbrace if  
    setgray 
  },
  dotsize=1.1cm,xStep=1,yStep=1,dotstyle=square*]{10}{10}{matrix1.data}
\end{pspicture}}
\end{center}


\begin{lstlisting}
\begin{filecontents}{matrix1.data}
/dotmatrix [ % <------------ important line
3 0 0 0 0 0 0 0 1 2
0 0 0 0 0 0 0 1 2 1
8 0 0 0 0 0 1 2 1 0
0 0 0 0 0 1 2 1 0 0
0 0 0 0 1 2 1 0 0 0
9 0 0 1 2 1 3 0 0 0
0 0 1 2 1 4 0 0 0 0
0 1 2 1 5 0 0 0 0 0
1 2 1 6 0 0 0 0 0 0
2 1 7 0 0 0 0 0 0 3
] def % <------------ important line
\end{filecontents}
\psscalebox{0.7}{%
\begin{pspicture}(-0.5,-0.75)(11,11)
\psaxes[ticksize=-5pt 0]{->}(11,11)
\psMatrixPlot[
  colorType=5,
  colorTypeDef={
    dup /value exch def % save value and leave one on the stack
    value Min sub dMaxMin div neg 1 add 300 mul 400 add \pswavelengthToGRAY 
    value 0 eq \pslbrace 1 \psrbrace if % 
    value 1 eq \pslbrace 0 \psrbrace if  
    setgray 
  },
  dotsize=1.1cm,xStep=1,yStep=1,dotstyle=square*]{10}{10}{matrix1.data}
\end{pspicture}}
\end{lstlisting}


\Lps{if} statements in the color definition must be enclosed with \Lcs{pslbrace} and \Lcs{psrbrace}
when they are parentheses used in PostScript. In the above example the color definition should be
modified when the matrix is a real big one, in such a case a nested \Lps{ifelse} makes more sense:

\begin{lstlisting}
  colorTypeDef={
    dup /value exch def 
    value 0 eq 
      \pslbrace 1 setgray \psrbrace
      \pslbrace value 1 eq 
        \pslbrace 0 setgray \psrbrace
        \pslbrace Min sub dMaxMin div neg 1 add 300 mul 400 add
          \pswavelengthToGRAY setgray \psrbrace ifelse
      \psrbrace ifelse 
  },
\end{lstlisting}

Replace the \Lcs{pslbrace} and \Lcs{psrbrace} with \{ and \} if it maybe confusing to read:

\begin{lstlisting}
    dup /value exch def 
    value 0 eq 
      { 1 setgray }
      { value 1 eq 
        { 0 setgray }
        { Min sub dMaxMin div neg 1 add 300 mul 400 add
          \pswavelengthToGRAY setgray } ifelse
      } ifelse 
\end{lstlisting}

Another possibility is to define the color procedure onside the data file, where
it \emph{must} be named \Lps{colorTypeDef}. If such a definition exists, the one from
the optional argument \Lkeyword{colorTypeDef} will be ignored. There can be no
\TeX-specific code inside this definition because it is read on PostScript level,
the reason why \Lcs{pswavelengthToGRAY} cannot be used.

\begin{center}
\begin{filecontents}{matrix1.data}
/colorTypeDef {
  dup /value exch def 
  value 0 eq 
    { 1 setgray }
    { value 1 eq 
      { 0 setgray }
      { Min sub dMaxMin div neg 1 add 300 mul 400 add
%        \pswavelengthToGRAY not possible
         tx@addDict begin wavelengthToRGB Red Green Blue end 
        setrgbcolor
      } ifelse
    } ifelse 
} def
/dotmatrix [ % <------------ important line
3 0 0 0 0 0 0 0 1 2
0 0 0 0 0 0 0 1 2 1
8 0 0 0 0 0 1 2 1 0
0 0 0 0 0 1 2 1 0 0
0 0 0 0 1 2 1 0 0 0
9 0 0 1 2 1 3 0 0 0
0 0 1 2 1 4 0 0 0 0
0 1 2 1 5 0 0 0 0 0
1 2 1 6 0 0 0 0 0 0
2 1 7 0 0 0 0 0 0 3
] def % <------------ important line
\end{filecontents}
\psscalebox{0.7}{%
\begin{pspicture}(-0.5,-0.75)(11,11)
\psaxes[ticksize=-5pt 0]{->}(11,11)
\psMatrixPlot[
  colorType=5,dotsize=1.1cm,xStep=1,yStep=1,dotstyle=square*]{10}{10}{matrix1.data}
\end{pspicture}}
\end{center}

\begin{lstlisting}
\begin{filecontents}{matrix1.data}
/colorTypeDef {
  dup /value exch def 
  value 0 eq 
    { 1 setgray }
    { value 1 eq 
      { 0 setgray }
      { Min sub dMaxMin div neg 1 add 300 mul 400 add
%        \pswavelengthToRGB not possible
         tx@addDict begin wavelengthToRGB Red Green Blue end 
        setrgbcolor
      } ifelse
    } ifelse 
} def
/dotmatrix [ % <------------ important line
3 0 0 0 0 0 0 0 1 2
0 0 0 0 0 0 0 1 2 1
8 0 0 0 0 0 1 2 1 0
0 0 0 0 0 1 2 1 0 0
0 0 0 0 1 2 1 0 0 0
9 0 0 1 2 1 3 0 0 0
0 0 1 2 1 4 0 0 0 0
0 1 2 1 5 0 0 0 0 0
1 2 1 6 0 0 0 0 0 0
2 1 7 0 0 0 0 0 0 3
] def % <------------ important line
\end{filecontents}
\psscalebox{0.7}{%
\begin{pspicture}(-0.5,-0.75)(11,11)
\psaxes[ticksize=-5pt 0]{->}(11,11)
\psMatrixPlot[colorType=5,dotsize=1.1cm,xStep=1,yStep=1,
  dotstyle=square*]{10}{10}{matrix1.data}
\end{pspicture}}
\end{lstlisting}



%--------------------------------------------------------------------------------------
\section{Dashed Lines}
%--------------------------------------------------------------------------------------
Tobias Nähring has implemented an enhanced feature for dashed
lines. The number of arguments is no longer limited.

\begin{BDef}
\Lkeyword{dash}=value1\OptArg*{unit} value2\OptArg*{unit} \ldots
\end{BDef}

\begin{LTXexample}[width=0.4\linewidth]
\psset{linewidth=2.5pt,unit=0.6}
\begin{pspicture}(-5,-4)(5,4)
 \psgrid[subgriddiv=0,griddots=10,gridlabels=0pt]
  \psset{linestyle=dashed}
  \pscurve[dash=5mm 1mm 1mm 1mm,linewidth=0.1](-5,4)(-4,3)(-3,4)(-2,3)
  \psline[dash=5mm 1mm 1mm 1mm 1mm 1mm 1mm 1mm 1mm 1mm](-5,0.9)(5,0.9)
  \psccurve[linestyle=solid](0,0)(1,0)(1,1)(0,1)
  \psccurve[linestyle=dashed,dash=5mm 2mm 0.1 0.2,linetype=0](0,0)(-2.5,0)(-2.5,-2.5)(0,-2.5)
  \pscurve[dash=3mm 3mm 1mm 1mm,linecolor=red,linewidth=2pt](5,-4)(5,2)(4.5,3.5)(3,4)(-5,4)
\end{pspicture}
\end{LTXexample}



\clearpage
%--------------------------------------------------------------------------------------
\section{Arrows}
%--------------------------------------------------------------------------------------
\subsection{Definition}
%--------------------------------------------------------------------------------------
\LPack{pstricks-add} defines the following "`arrows"':

\begin{center}
  \bgroup
  \def\myline#1{\psline[linecolor=red,linewidth=0.5pt,arrowscale=1.5]{#1}(0,1ex)(1.3,1ex)}%
  \psset{arrowscale=1.5}
  \begin{tabular}{@{} c @{\qquad} p{3cm} l @{}}%
    Value & Example & Name \\[2pt]\hline
    \Lnotation{-}      & \myline{-}      & None\\
    \Lnotation{<->}    & \myline{<->}    & Arrowheads.\\
    \Lnotation{>-<}    & \myline{>-<}    & Reverse arrowheads.\\
    \Lnotation{<{<}-{>}>}  & \myline{<<->>}  & Double arrowheads.\\
    \Lnotation{{>}>-{<}<}  & \myline{>>-<<}  & Double reverse arrowheads.\\
    \Lnotation{{|}-{|}}    & \myline{|-|}    & T-bars, flush to endpoints.\\
    \Lnotation{{|}*-{|}*}  & \myline{|*-|*}  & T-bars, centered on endpoints.\\
    \Lnotation{[-]}    & \myline{[-]}    & Square brackets.\\
    \Lnotation{]-[}    & \myline{]-[}    & Reversed square brackets.\\
    \Lnotation{(-)}    & \myline{(-)}    & Rounded brackets.\\
    \Lnotation{)-(}    & \myline{)-(}    & Reversed rounded brackets.\\
    \Lnotation{o-o}    & \myline{o-o}    & Circles, centered on endpoints.\\
    \Lnotation{*-*}    & \myline{*-*}    & Disks, centered on endpoints.\\
    \Lnotation{oo-oo}  & \myline{oo-oo}  & Circles, flush to endpoints.\\
    \Lnotation{**-**}  & \myline{**-**}  & Disks, flush to endpoints.\\
    \Lnotation{{|}<->{|}}  & \myline{|<->|}  & T-bars and arrows.\\
    \Lnotation{{|}>-<{|}}  & \myline{|>-<|}  & T-bars and reverse arrows.\\
    \Lnotation{h-h{|}}   & \myline{h-h}    & left/right hook arrows.\\
    \Lnotation{H-H{|}}   & \myline{H-H}    & left/right hook arrows.\\
    \Lnotation{v-v|}   & \myline{v-v}    & left/right inside vee arrows.\\
    \Lnotation{V-V|}   & \myline{V-V}    & left/right outside vee arrows.\\
    \Lnotation{f-f|}   & \myline{f-f}    & left/right inside filled arrows.\\
    \Lnotation{F-F|}   & \myline{F-F}    & left/right outside filled arrows.\\
    \Lnotation{t-t|}   & \myline{t-t}    & left/right inside slash arrows.\\[5pt]
    \Lnotation{T-T|}   & \myline{T-T}    & left/right outside slash arrows.\\
  \end{tabular}
  \egroup
\end{center}



You can also mix and match, e.g., \Lnotation{->}, \Lnotation{*-)} and \Lnotation{[->} are all valid values
of the \Lkeyword{arrows} parameter. The parameter can be set with

\begin{BDef}
\Lcs{psset}\Largb{arrows=<type>}
\end{BDef}

\noindent or for some macros with a special option, like\\[5pt]
\noindent\verb|\psline[<general options>]{<arrow type>}(A)(B)|\\
\noindent\verb/\psline[linecolor=red,linewidth=2pt]{|->}(0,0)(0,2)/ \ \psline[linecolor=red,linewidth=2pt]{|->}(0,0)(0,2)

\subsection{Multiple arrows}
There are two new options which are only valid for the arrow type \verb+<<+ or \verb+>>+.
\verb+nArrow+ sets both, the \verb+nArrowA+ and the  \verb+nArrowB+ parameter. The meaning
is declared in the following tables. Without setting one of these parameters the behaviour
is like the one described in the old PSTricks manual.

\begin{center}
\begin{tabular}{@{}lc@{}}%
    Value & Meaning \\[2pt]\hline
    \Lnotation{-{>}>}   & \ -A \\
    \Lnotation{{<}<-{>}>} & A-A\\
    \Lnotation{{<}<-}   & A-\ \\
    \Lnotation{{>}>-}   & B-\ \\
    \Lnotation{-{<}<}   & \ -B\\
    \Lnotation{{>}>-{<}<} & B-B\\
    \Lnotation{{>}>-{>}>} & B-A\\
    \Lnotation{{<}<-{<}<} & A-B
  \end{tabular}
\end{center}




\begin{center}
  \bgroup
  \psset{linecolor=red,linewidth=1pt,arrowscale=2}%
  \begin{tabular}{lp{2.8cm}}%
    Value & Example \\[2pt]\hline
    \verb+\psline{->>}(0,1ex)(2.3,1ex)+  & \psline{->>}(0,1ex)(2.3,1ex) \\
    \verb+\psline[nArrowsA=3]{->>}(0,1ex)(2.3,1ex)+  & \psline[nArrowsA=3]{->>}(0,1ex)(2.3,1ex)\\
    \verb+\psline[nArrowsA=5]{->>}(0,1ex)(2.3,1ex)+  & \psline[nArrowsA=5]{->>}(0,1ex)(2.3,1ex)\\
    \verb+\psline{<<-}(0,1ex)(2.3,1ex)+  & \psline{<<-}(0,1ex)(2.3,1ex)\\
    \verb+\psline[nArrowsA=3]{<<-}(0,1ex)(2.3,1ex)+  & \psline[nArrowsA=3]{<<-}(0,1ex)(2.3,1ex)\\
    \verb+\psline[nArrowsA=5]{<<-}(0,1ex)(2.3,1ex)+  & \psline[nArrowsA=5]{<<-}(0,1ex)(2.3,1ex)\\
    \verb+\psline{<<->>}(0,1ex)(2.3,1ex)+  & \psline{<<->>}(0,1ex)(2.3,1ex)\\
    \verb+\psline[nArrowsA=3]{<<->>}(0,1ex)(2.3,1ex)+  & \psline[nArrowsA=3]{<<->>}(0,1ex)(2.3,1ex)\\
    \verb+\psline[nArrowsA=5]{<<->>}(0,1ex)(2.3,1ex)+  & \psline[nArrowsA=5]{<<->>}(0,1ex)(2.3,1ex)\\
    \verb+\psline{<<-|}(0,1ex)(2.3,1ex)+  & \psline{<<-|}(0,1ex)(2.3,1ex)\\
    \verb+\psline[nArrowsA=3]{<<-<<}(0,1ex)(2.3,1ex)+  & \psline[nArrowsA=3]{<<-<<}(0,1ex)(2.3,1ex)\\
    \verb+\psline[nArrowsA=5]{<<-o}(0,1ex)(2.3,1ex)+  & \psline[nArrowsA=5]{<<-o}(0,1ex)(2.3,1ex)\\
    \verb+\psline[nArrowsA=3,nArrowsB=4]{<<-<<}(0,1ex)(2.3,1ex)+  & \psline[nArrowsA=3,nArrowsB=4]{<<-<<}(0,1ex)(2.3,1ex)\\
    \verb+\psline[nArrowsA=3,nArrowsB=4]{>>->>}(0,1ex)(2.3,1ex)+  & \psline[nArrowsA=3,nArrowsB=4]{>>->>}(0,1ex)(2.3,1ex)\\
    \verb+\psline[nArrowsA=1,nArrowsB=4]{>>->>}(0,1ex)(2.3,1ex)+  & \psline[nArrowsA=1,nArrowsB=4]{>>->>}(0,1ex)(2.3,1ex)\\
  \end{tabular}
  \egroup
\end{center}



\subsection{\texttt{hookarrow}}
%\begin{LTXexample}
\bgroup
\psset{arrowsize=8pt,arrowlength=1,linewidth=1pt,nodesep=2pt,shortput=tablr}
\large
\begin{psmatrix}[colsep=12mm,rowsep=10mm]
        &   & $R_2$            \\
        &   &   0   &   & $R_3$\\
$e_b:S$ & 1 &       & 1 & 0    \\
        &   &   0              \\
        &   &   $R_1$          \\
\end{psmatrix}
\ncline{h-}{1,3}{2,3}<{$e_{r2}$}>{$f_{r2}$}
\ncline{-h}{2,3}{3,2}<{$e_1$}
\ncline{-h}{3,1}{3,2}^{$e_s$}_{$f_{s}$}
\ncline{-h}{3,2}{4,3}>{$e_3$}<{$f_3$}
\ncline{-h}{4,3}{3,4}>{$e_4$}<{$f_4$}
\ncline{-h}{3,4}{2,3}>{$e_2$}<{$f_2$}
\ncline{-h}{3,4}{3,5}^{$e_5$}
\ncline{-h}{3,5}{2,5}<{$e_{r3}$}>{$f_{r3}$}
\ncline{-h}{4,3}{5,3}<{$e_{r1}$}>{$f_{r1}$}
%\end{LTXexample}
\egroup

\begin{lstlisting}
\psset{arrowsize=8pt,arrowlength=1,linewidth=1pt,nodesep=2pt,shortput=tablr}
\large
\begin{psmatrix}[colsep=12mm,rowsep=10mm]
        &   & $R_2$            \\
        &   &   0   &   & $R_3$\\
$e_b:S$ & 1 &       & 1 & 0    \\
        &   &   0              \\
        &   &   $R_1$          \\
\end{psmatrix}
\ncline{h-}{1,3}{2,3}<{$e_{r2}$}>{$f_{r2}$}\ncline{-h}{2,3}{3,2}<{$e_1$}
\ncline{-h}{3,1}{3,2}^{$e_s$}_{$f_{s}$}    \ncline{-h}{3,2}{4,3}>{$e_3$}<{$f_3$}
\ncline{-h}{4,3}{3,4}>{$e_4$}<{$f_4$}      \ncline{-h}{3,4}{2,3}>{$e_2$}<{$f_2$}
\ncline{-h}{3,4}{3,5}^{$e_5$}              
\ncline{-h}{3,5}{2,5}<{$e_{r3}$}>{$f_{r3}$}
\ncline{-h}{4,3}{5,3}<{$e_{r1}$}>{$f_{r1}$}
\end{lstlisting}



\subsection{\texttt{hookrightarrow} and \texttt{hookleftarrow}}
This is another type of arrow and is abbreviated with \Lnotation{H}.
The length and width of the hook is set by the new options
\Lkeyword{hooklength} and \Lkeyword{hookwidth}, which are by default set
to
%
\begin{BDef}
\Lcs{psset}\Largb{hooklength=3mm,hookwidth=1mm}
\end{BDef}
%
If the line begins with a right hook then the line ends with a left hook and vice versa:

\begin{LTXexample}[width=3cm]
\begin{pspicture}(3,4)
\psline[linewidth=5pt,linecolor=blue,hooklength=5mm,hookwidth=-3mm]{H->}(0,3.5)(3,3.5)
\psline[linewidth=5pt,linecolor=red,hooklength=5mm,hookwidth=3mm]{H->}(0,2.5)(3,2.5)
\psline[linewidth=5pt,hooklength=5mm,hookwidth=3mm]{H-H}(0,1.5)(3,1.5)
\psline[linewidth=1pt]{H-H}(0,0.5)(3,0.5)
\end{pspicture}
\end{LTXexample}


\begin{LTXexample}[width=7.25cm]
$\begin{psmatrix}
E&W_i(X)&&Y\\
&&W_j(X)
\psset{arrows=->,nodesep=3pt,linewidth=2pt}
\everypsbox{\scriptstyle}
\ncline[linecolor=red,arrows=H->,%
  hooklength=4mm,hookwidth=2mm]{1,1}{1,2}
\ncline{1,2}{1,4}^{\tilde{t}}
\ncline{1,2}{2,3}<{W_{ij}}
\ncline{2,3}{1,4}>{\tilde{s}}
\end{psmatrix}$
\end{LTXexample}


%--------------------------------------------------------------------------------------
\subsection{\nxLkeyword{ArrowInside} Option}
%--------------------------------------------------------------------------------------

It is now possible to have arrows inside lines and not only at the
beginning or the end. The new defined options

\psset{arrowscale=2,linecolor=red,unit=1cm,linewidth=1.5pt}
\begin{longtable}{l|>{\RaggedRight}p{8.5cm}|p{2.2cm}}
Name & Example & Output\\\hline
\endfirsthead
Name & Example & Output\\\hline
\endhead
\Lkeyword{ArrowInside} &
  \texttt{\textbackslash psline[ArrowInside=->](0,0)(2,0)} &
  \psline[ArrowInside=->](0,0.1)(2,0.1) \\
\Lkeyword{ArrowInsidePos} & \texttt{\textbackslash psline[ArrowInside=->,\%}
  \hspace*{20pt}\texttt{ArrowInsidePos=0.25](0,0)(2,0)}
& \psline[ArrowInside=->, ArrowInsidePos=0.25](0,0.1)(2,0.1) \\
\Lkeyword{ArrowInsidePos} & \texttt{\textbackslash psline[ArrowInside=->,\%}
  \hspace*{20pt}\texttt{ArrowInsidePos=10](0,0)(2,0)}
& \psline[ArrowInside=->, ArrowInsidePos=10](0,0.1)(2,0.1) \\
\Lkeyword{ArrowInsideNo} & \texttt{\textbackslash psline[ArrowInside=->,\%}
  \hspace*{20pt}\texttt{ArrowInsideNo=2](0,0)(2,0)}
& \psline[ArrowInside=->, ArrowInsideNo=2](0,0.1)(2,0.1) \\
\Lkeyword{ArrowInsideOffset} & \texttt{\textbackslash psline[ArrowInside=->,\%}
  \hspace*{20pt}\texttt{ArrowInsideNo=2,\%}\newline
  \hspace*{20pt}\texttt{ArrowInsideOffset=0.1](0,0)(2,0)}
& \psline[ArrowInside=->, ArrowInsideNo=2,ArrowInsideOffset=0.1](0,0.1)(2,0.1) \\
%
\Lkeyword{ArrowInside} & \texttt{\textbackslash psline[ArrowInside=->]\{->\}(0,0)(2,0)} &
  \psline[ArrowInside=->]{->}(0,0)(2,0)\\
\Lkeyword{ArrowInsidePos} & \texttt{\textbackslash psline[ArrowInside=->,\%}
  \hspace*{20pt}\texttt{ArrowInsidePos=0.25]\{->\}(0,0)(2,0)}
  & \psline[ArrowInside=->, ArrowInsidePos=0.25]{->}(0,0)(2,0) \\
\Lkeyword{ArrowInsidePos} & \texttt{\textbackslash psline[ArrowInside=->,\%}
  \hspace*{20pt}\texttt{ArrowInsidePos=10]\{->\}(0,0)(2,0)}
  & \psline[ArrowInside=->, ArrowInsidePos=10]{->}(0,0)(2,0) \\
\Lkeyword{ArrowInsideNo} & \texttt{\textbackslash psline[ArrowInside=->,\%}
  \hspace*{20pt}\texttt{ArrowInsideNo=2]\{->\}(0,0)(2,0)}
  & \psline[ArrowInside=->, ArrowInsideNo=2]{->}(0,0)(2,0) \\
\Lkeyword{ArrowInsideOffset} & \texttt{\textbackslash psline[ArrowInside=->,\%}
  \hspace*{20pt}\texttt{ArrowInsideNo=2,\%}\newline
  \hspace*{20pt}\texttt{ArrowInsideOffset=0.1]\{->\}(0,0)(2,0)}
  & \psline[ArrowInside=->, ArrowInsideNo=2,ArrowInsideOffset=0.1]{->}(0,0)(2,0) \\
%
\Lkeyword{ArrowFill} & \texttt{\textbackslash psline[ArrowFill=false,\%}
  \hspace*{20pt}\texttt{arrowinset=0]\{->\}(0,0)(2,0)} &
  \psline[ArrowFill=false,arrowinset=0]{->}(0,0)(2,0)\\
\Lkeyword{ArrowFill} & \texttt{\textbackslash psline[ArrowFill=false,\%}
  \hspace*{20pt}\texttt{arrowinset=0]\{<<->>\}(0,0)(2,0)} &
  \psline[ArrowFill=false,arrowinset=0]{<<->>}(0,0)(2,0)\\
\Lkeyword{ArrowFill} & \texttt{\textbackslash psline[ArrowInside=->,\%}\newline
  \hspace*{20pt}\texttt{arrowinset=0,\%}\newline
  \hspace*{20pt}\texttt{ArrowFill=false,\%}\newline
  \hspace*{20pt}\texttt{ArrowInsideNo=2,\%}\newline
  \hspace*{20pt}\texttt{ArrowInsideOffset=0.1]\{->\}(0,0)(2,0)}
  & \psline[ArrowInside=->, ArrowFill=false,ArrowInsideNo=2,ArrowInsideOffset=0.1]{->}(0,0)(2,0) \\
\end{longtable}

\medskip
Without the default arrow definition there is only the one inside
the line, defined by the type and the position. The position is
relative to the length of the whole line. $0.25$ means at $25\%$
of the line length. The peak of the arrow gets the coordinates
which are calculated by the macro. If you want arrows with an
absolute position difference, then choose a value greater than
\verb|1|, e.\,g. \verb|10| which places an arrow every 10~pt. The
default unit \verb|pt| cannot be changed.

\medskip
\noindent
\begin{tabularx}{\linewidth}{@{\color{red}\vrule width 2pt}lX@{}}
& The \Lkeyword{ArrowInside} takes only arrow definitions like \Lnotation{->} into account.
Arrows from right to left (\Lnotation{<-}) are not possible and ignored. If you need
such arrows, change the order of the pairs of coordinates for the line or curve macro.
\end{tabularx}

%--------------------------------------------------------------------------------------
\subsection{\nxLkeyword{ArrowFill} Option}
%--------------------------------------------------------------------------------------

By default all arrows are filled polygons. With the option
\Lkeyset{ArrowFill=false} there are ''white`` arrows. Only for the
beginning/end arrows are they empty, the inside arrows are
overpainted by the line.


\psset{arrowscale=1}
\begin{LTXexample}[width=3.5cm]
\psset{arrowscale=2.5}
\psline[linecolor=red,arrowinset=0]{<->}(-1,0)(2,0)
\end{LTXexample}

\begin{LTXexample}[width=3.5cm]
\psset{arrowscale=2.5}
\psline[linecolor=red,arrowinset=0,ArrowFill=false]{<->}(-1,0)(2,0)
\end{LTXexample}

\begin{LTXexample}[width=3.5cm]
\psset{arrowscale=2.5}
\psline[linecolor=red,arrowinset=0,arrowsize=0.2,
  ArrowFill=false]{<->}(-1,0)(2,0)
\end{LTXexample}

\begin{LTXexample}[width=3.5cm]
\psline[linecolor=blue,arrowscale=4,
  ArrowFill]{>>->>}(-1,0)(2,0)
\end{LTXexample}

\begin{LTXexample}[width=3.5cm]
\psline[linecolor=blue,arrowscale=4,
  ArrowFill=false]{>>->>}(-1,0)(2,0)
\rule{3cm}{0pt}\\[30pt]
\end{LTXexample}

\begin{LTXexample}[width=3.5cm]
\psline[linecolor=blue,arrowscale=4,
  ArrowFill]{>|->|}(-1,0)(2,0)
\end{LTXexample}

\begin{LTXexample}[width=3.5cm]
\psline[linecolor=blue,arrowscale=4,
  ArrowFill=false]{>|->|}(-1,0)(2,0)%
\end{LTXexample}


%--------------------------------------------------------------------------------------
\subsection{Examples}
%--------------------------------------------------------------------------------------

All examples are printed with \verb|\psset{arrowscale=2,linecolor=red}|.
\subsubsection{\nxLcs{psline}}

\bigskip
\begin{LTXexample}[width=2.5cm]
\begin{pspicture}(2,2)
\psset{arrowscale=2,ArrowFill=true}
\psline[ArrowInside=->]{|<->|}(2,1)
\end{pspicture}
\end{LTXexample}

\begin{LTXexample}[width=2.5cm]
\begin{pspicture}(2,2)
\psset{arrowscale=2,ArrowFill=true}
\psline[ArrowInside=-|]{|-|}(2,1)
\end{pspicture}
\end{LTXexample}

\begin{LTXexample}[width=2.5cm]
\begin{pspicture}(2,2)
\psset{arrowscale=2,ArrowFill=true}
\psline[ArrowInside=->,ArrowInsideNo=2]{->}(2,1)
\end{pspicture}
\end{LTXexample}

\begin{LTXexample}[width=2.5cm]
\begin{pspicture}(2,2)
\psset{arrowscale=2,ArrowFill=true}
\psline[ArrowInside=->,ArrowInsideNo=2,ArrowInsideOffset=0.1]{->}(2,1)
\end{pspicture}
\end{LTXexample}

\begin{LTXexample}[width=6.5cm]
\begin{pspicture}(6,2)
\psset{arrowscale=2,ArrowFill=true}
\psline[ArrowInside=-*]{->}(0,0)(2,1)(3,0)(4,0)(6,2)
\end{pspicture}
\end{LTXexample}

\begin{LTXexample}[width=6.5cm]
\begin{pspicture}(6,2)
\psset{arrowscale=2,ArrowFill=true}
\psline[ArrowInside=-*,ArrowInsidePos=0.25]{->}(0,0)(2,1)(3,0)(4,0)(6,2)
\end{pspicture}
\end{LTXexample}

\begin{LTXexample}[width=6.5cm]
\begin{pspicture}(6,2)
\psset{arrowscale=2,ArrowFill=true}
\psline[ArrowInside=-*,ArrowInsidePos=0.25,ArrowInsideNo=2]{->}%
   (0,0)(2,1)(3,0)(4,0)(6,2)
\end{pspicture}
\end{LTXexample}

\begin{LTXexample}[width=6.5cm]
\begin{pspicture}(6,2)
\psset{arrowscale=2,ArrowFill=true}
\psline[ArrowInside=->, ArrowInsidePos=0.25]{->}%
        (0,0)(2,1)(3,0)(4,0)(6,2)
\end{pspicture}
\end{LTXexample}

\begin{LTXexample}[width=6.5cm]
\begin{pspicture}(6,2)
\psset{arrowscale=2,ArrowFill=true}
\psline[linestyle=none,ArrowInside=->,ArrowInsidePos=0.25]{->}%
        (0,0)(2,1)(3,0)(4,0)(6,2)
\end{pspicture}
\end{LTXexample}

\begin{LTXexample}[width=6.5cm]
\begin{pspicture}(6,2)
\psset{arrowscale=2,ArrowFill=true}
\psline[ArrowInside=-<, ArrowInsidePos=0.75]{->}%
     (0,0)(2,1)(3,0)(4,0)(6,2)
\end{pspicture}
\end{LTXexample}

\begin{LTXexample}[width=6.5cm]
\begin{pspicture}(6,2)
\psset{arrowscale=2,ArrowFill=true,ArrowInside=-*}
\psline(0,0)(2,1)(3,0)(4,0)(6,2)
\psset{linestyle=none}
\psline[ArrowInsidePos=0](0,0)(2,1)(3,0)(4,0)(6,2)
\psline[ArrowInsidePos=1](0,0)(2,1)(3,0)(4,0)(6,2)
\end{pspicture}
\end{LTXexample}

\begin{LTXexample}[width=6.5cm]
\begin{pspicture}(6,5)
\psset{arrowscale=2,ArrowFill=true}
\psline[ArrowInside=->,ArrowInsidePos=20](0,0)(3,0)%
       (3,3)(1,3)(1,5)(5,5)(5,0)(7,0)(6,3)
\end{pspicture}
\end{LTXexample}

\begin{LTXexample}[width=6.5cm]
\begin{pspicture}(6,2)
\psset{arrowscale=2,ArrowFill=true}
\psline[ArrowInside=-|]{<->}(0,2)(2,0)(3,2)(4,0)(6,2)
\end{pspicture}
\end{LTXexample}

%--------------------------------------------------------------------------------------
\subsubsection{\nxLcs{pspolygon}}
%--------------------------------------------------------------------------------------
% Polygons (\pspolygon macro)

\begin{LTXexample}[width=6.5cm]
\begin{pspicture}(6,3)
\psset{arrowscale=2}
\pspolygon[ArrowInside=-|](0,0)(3,3)(6,3)(6,1)
\end{pspicture}
\end{LTXexample}

\begin{LTXexample}[width=6.5cm]
\begin{pspicture}(6,3)
\psset{arrowscale=2}
\pspolygon[ArrowInside=->,ArrowInsidePos=0.25]%
     (0,0)(3,3)(6,3)(6,1)
\end{pspicture}
\end{LTXexample}

\begin{LTXexample}[width=6.5cm]
\begin{pspicture}(6,3)
\psset{arrowscale=2}
\pspolygon[ArrowInside=->,ArrowInsideNo=4]%
       (0,0)(3,3)(6,3)(6,1)
\end{pspicture}
\end{LTXexample}

\begin{LTXexample}[width=6.5cm]
\begin{pspicture}(6,3)
\psset{arrowscale=2}
\pspolygon[ArrowInside=->,ArrowInsideNo=4,%
   ArrowInsideOffset=0.1](0,0)(3,3)(6,3)(6,1)
\end{pspicture}
\end{LTXexample}

\begin{LTXexample}[width=6.5cm]
\begin{pspicture}(6,3)
\psset{arrowscale=2}
 \pspolygon[ArrowInside=-|](0,0)(3,3)(6,3)(6,1)
 \psset{linestyle=none,ArrowInside=-*}
 \pspolygon[ArrowInsidePos=0](0,0)(3,3)(6,3)(6,1)
 \pspolygon[ArrowInsidePos=1](0,0)(3,3)(6,3)(6,1)
 \psset{ArrowInside=-o}
 \pspolygon[ArrowInsidePos=0.25](0,0)(3,3)(6,3)(6,1)
 \pspolygon[ArrowInsidePos=0.75](0,0)(3,3)(6,3)(6,1)
\end{pspicture}
\end{LTXexample}

\psset{linestyle=solid}

\begin{LTXexample}[width=6.5cm]
\begin{pspicture}(6,5)
\psset{arrowscale=2}
  \pspolygon[ArrowInside=->,ArrowInsidePos=20]%
    (0,0)(3,0)(3,3)(1,3)(1,5)(5,5)(5,0)(7,0)(6,3)
\end{pspicture}
\end{LTXexample}


%--------------------------------------------------------------------------------------
\subsubsection{\nxLcs{psbezier}}
%--------------------------------------------------------------------------------------
% Bezier curves (\psbezier macro)


\begin{LTXexample}[width=3.5cm]
\begin{pspicture}(3,3)
\psset{arrowscale=2}
  \psbezier[ArrowInside=-|](0,1)(1,0)(2,1)(3,3)
  \psset{linestyle=none,ArrowInside=-o}
  \psbezier[ArrowInsidePos=0.25](0,1)(1,0)(2,1)(3,3)
  \psbezier[ArrowInsidePos=0.75](0,1)(1,0)(2,1)(3,3)
  \psset{linestyle=none,ArrowInside=-*}
  \psbezier[ArrowInsidePos=0](0,1)(1,0)(2,1)(3,3)
  \psbezier[ArrowInsidePos=1](0,1)(1,0)(2,1)(3,3)
\end{pspicture}
\end{LTXexample}



\resetOptions
\begin{LTXexample}[width=4.5cm]
\begin{pspicture}(4,3)
\psset{arrowscale=2}
\psbezier[ArrowInside=->,showpoints]%
  {*-*}(0,0)(2,3)(3,0)(4,2)
\end{pspicture}
\end{LTXexample}




\begin{LTXexample}[width=4.5cm]
\begin{pspicture}(4,3)
\psset{arrowscale=2}
  \psbezier[ArrowInside=->,showpoints=true,
      ArrowInsideNo=2](0,0)(2,3)(3,0)(4,2)
\end{pspicture}
\end{LTXexample}


\begin{LTXexample}[width=4.5cm]
\begin{pspicture}(4,3)
\psset{arrowscale=2}
  \psbezier[ArrowInside=->,showpoints=true,
     ArrowInsideNo=2,ArrowInsideOffset=-0.2]%
      {->}(0,0)(2,3)(3,0)(4,2)
\end{pspicture}
\end{LTXexample}


\begin{LTXexample}[width=5.5cm]
\begin{pspicture}(5,3)
\psset{arrowscale=2}
  \psbezier[ArrowInsideNo=9,ArrowInside=-|,%
    showpoints=true]{*-*}(0,0)(1,3)(3,0)(5,3)
\end{pspicture}
\end{LTXexample}

\begin{LTXexample}[width=4.5cm]
\begin{pspicture}(4,3)
\psset{arrowscale=2}
  \psset{ArrowInside=-|}
  \psbezier[ArrowInsidePos=0.25,showpoints=true]{*-*}(2,3)(3,0)(4,2)
  \psset{linestyle=none}
  \psbezier[ArrowInsidePos=0.75](0,0)(2,3)(3,0)(4,2)
\end{pspicture}
\end{LTXexample}

\begin{LTXexample}[width=5.5cm]
\begin{pspicture}(5,6)
\psset{arrowscale=2}
  \pnode(3,4){A}\pnode(5,6){B}\pnode(5,0){C}
  \psbezier[ArrowInside=->,%
     showpoints=true](A)(B)(C)
  \psset{linestyle=none,ArrowInside=-<}
  \psbezier[ArrowInsideNo=4](0,0)(A)(B)(C)
  \psset{ArrowInside=-o}
  \psbezier[ArrowInsidePos=0.1](0,0)(A)(B)(C)
  \psbezier[ArrowInsidePos=0.9](0,0)(A)(B)(C)
  \psset{ArrowInside=-*}
  \psbezier[ArrowInsidePos=0.3](0,0)(A)(B)(C)
  \psbezier[ArrowInsidePos=0.7](0,0)(A)(B)(C)
\end{pspicture}
\end{LTXexample}

\psset{linestyle=solid}

\begin{LTXexample}[pos=t]
\begin{pspicture}(-3,-5)(15,5)
  \psbezier[ArrowInsideNo=19,%
      ArrowInside=->,ArrowFill=false,%
      showpoints=true]{->}(-3,0)(5,-5)(8,5)(15,-5)
\end{pspicture}
\end{LTXexample}



%--------------------------------------------------------------------------------------
\subsubsection{\nxLcs{pcline}}
%--------------------------------------------------------------------------------------
These examples need the package \verb|pst-node|.

% Lines (\pcline macro)
\begin{LTXexample}[width=2.5cm]
\begin{pspicture}(2,1)
\psset{arrowscale=2}
\pcline[ArrowInside=->](0,0)(2,1)
\end{pspicture}
\end{LTXexample}


\begin{LTXexample}[width=2.5cm]
\begin{pspicture}(2,1)
\psset{arrowscale=2}
\pcline[ArrowInside=->]{<->}(0,0)(2,1)
\end{pspicture}
\end{LTXexample}


\begin{LTXexample}[width=2.5cm]
\begin{pspicture}(2,1)
\psset{arrowscale=2}
\pcline[ArrowInside=-|,ArrowInsidePos=0.75]{|-|}(0,0)(2,1)
\end{pspicture}
\end{LTXexample}


\begin{LTXexample}[width=2.5cm]
\psset{arrowscale=2}
\pcline[ArrowInside=->,ArrowInsidePos=0.65]{*-*}(0,0)(2,0)
\naput[labelsep=0.3]{\large$g$}
\end{LTXexample}


\begin{LTXexample}[width=2.5cm]
\psset{arrowscale=2}
\pcline[ArrowInside=->,ArrowInsidePos=10]{|-|}(0,0)(2,0)
\naput[labelsep=0.3]{\large$l$}
\end{LTXexample}



%--------------------------------------------------------------------------------------
\subsubsection{\nxLcs{pccurve}}
%--------------------------------------------------------------------------------------
These examples also need the package \verb|pst-node|.

\begin{LTXexample}[width=2.5cm]
\begin{pspicture}(2,2)
\psset{arrowscale=2}
\pccurve[ArrowInside=->,ArrowInsidePos=0.65,showpoints=true]{*-*}(0,0)(2,2)
\naput[labelsep=0.3]{\large$h$}
\end{pspicture}
\end{LTXexample}


\begin{LTXexample}[width=2.5cm]
\begin{pspicture}(2,2)
\psset{arrowscale=2}
\pccurve[ArrowInside=->,ArrowInsideNo=3,showpoints=true]{|->}(0,0)(2,2)
\naput[labelsep=0.3]{\large$i$}
\end{pspicture}
\end{LTXexample}


\begin{LTXexample}[width=4.5cm]
\begin{pspicture}(4,4)
\psset{arrowscale=2}
\pccurve[ArrowInside=->,ArrowInsidePos=20]{|-|}(0,0)(4,4)
\naput[labelsep=0.3]{\large$k$}
\end{pspicture}
\end{LTXexample}

\clearpage

\subsection{Special arrows \texttt{v--V},\texttt{t--T}, and \texttt{f--F}}

Possible optional arguments are

\psset{linecolor=black}

\begin{center}
\begin{tabular}{@{}l|l@{}}\toprule
\emph{name} & \emph{meaning}\\\hline
\Lkeyword{veearrowlength} & default is 3mm\\
\Lkeyword{veearrowangle} & default is 30\\
\Lkeyword{veearrowlinewidth} & default is 0.35mm\\
\Lkeyword{filledveearrowlength} & default is 3mm\\
\Lkeyword{filledveearrowangle} & default is 15\\
\Lkeyword{filledveearrowlinewidth} & default is 0.35mm\\
\Lkeyword{tickarrowlength} & default is 1.5mm\\
\Lkeyword{tickarrowlinewidth} & default is 0.35mm\\
\Lkeyword{arrowlinestyle}     & default is solid\\\bottomrule
\end{tabular}
\end{center}


\begin{LTXexample}[width=4cm]
\psset{unit=5mm}
\begin{pspicture}(4,6)
  \psset{dimen=middle,arrows=c-c,
    arrowscale=2,linewidth=.25mm}
  \psline[linecolor=red,linewidth=.05mm](0,0)(0,6)
  \psline[linecolor=red,linewidth=.05mm](4,0)(4,6)
  \psline{v-v}(0,6)(4,6)
  \psline{v-V}(0,4)(4,4)
  \psline{V-v}(0,2)(4,2)
  \psline{V-V}(0,0)(4,0)
\end{pspicture}
\end{LTXexample}


\begin{LTXexample}[width=4cm]
\psset{unit=5mm}
\begin{pspicture}(4,6)
  \psset{dimen=middle,arrows=c-c,
    arrowscale=2,linewidth=.25mm}
  \psline[linecolor=red,linewidth=.05mm](0,0)(0,6)
  \psline[linecolor=red,linewidth=.05mm](4,0)(4,6)
  \psline{f-f}(0,6)(4,6)
  \psline{f-F}(0,4)(4,4)
  \psline{F-f}(0,2)(4,2)
  \psline{F-F}(0,0)(4,0)
\end{pspicture}
\end{LTXexample}


\begin{LTXexample}[width=4cm]
\psset{unit=5mm}
\begin{pspicture}(4,6)
  \psset{dimen=middle,arrows=c-c,linewidth=.25mm}
  \psline[linecolor=red,linewidth=.05mm](0,0)(0,6)
  \psline[linecolor=red,linewidth=.05mm](4,0)(4,6)
  \psline{t-t}(0,6)(4,6)
  \psline{t-T}(0,4)(4,4)
  \psline{T-t}(0,2)(4,2)
  \psline{T-T}(0,0)(4,0)
\end{pspicture}
\end{LTXexample}

\begin{LTXexample}[pos=t,vsep=5mm]
\psset{unit=5mm}
 \begin{pspicture}(10,6)
 \psset{dimen=middle,arrows=c-c,arrowscale=2,linewidth=.25mm,
        arrowlinestyle=dashed,dash=1.5pt 1pt}
 \psline[linecolor=red,linewidth=.05mm](0,0)(0,6)
 \psline[linecolor=red,linewidth=.05mm](4,0)(4,6)
 \psline{v-v}(0,6)(4,6) \psline{v-V}(0,4)(4,4)
 \psline{V-v}(0,2)(4,2) \psline{V-V}(0,0)(4,0)
 \psline[linecolor=red,linewidth=.05mm](6,0)(6,6)
 \psline[linecolor=red,linewidth=.05mm](10,0)(10,6)
 \psset{arrowlinestyle=dotted,dotsep=0.8pt}
 \psline{v-v}(6,6)(10,6) \psline{v-V}(6,4)(10,4)
 \psline{V-v}(6,2)(10,2) \psline{V-V}(6,0)(10,0)
\end{pspicture}
\end{LTXexample}

\begin{LTXexample}[pos=t,vsep=5mm]
\psset{unit=5mm}
 \begin{pspicture}(10,7)
 \psset{dimen=middle,arrows=c-c,arrowscale=2,linewidth=.25mm,
        arrowlinestyle=dashed,dash=1.5pt 1pt}
 \psline[linecolor=red,linewidth=.05mm](0,0)(0,6)
 \psline[linecolor=red,linewidth=.05mm](4,0)(4,6)
 \psline{t-t}(0,6)(4,6) \psline{t-T}(0,4)(4,4)
 \psline{T-t}(0,2)(4,2) \psline{T-T}(0,0)(4,0)
 \psline[linecolor=red,linewidth=.05mm](6,0)(6,6)
 \psline[linecolor=red,linewidth=.05mm](10,0)(10,6)
 \psset{arrowlinestyle=dotted,dotsep=0.8pt}
 \psline{t-t}(6,6)(10,6) \psline{t-T}(6,4)(10,4)
 \psline{T-t}(6,2)(10,2) \psline{T-T}(6,0)(10,0)
\end{pspicture}
\end{LTXexample}




\subsection{Special arrow option \texttt{arrowLW}}

Only for the arrowtype \Lnotation{o} and \Lnotation{*} it is possible to
set the arrowlinewidth with the optional keyword \Lkeyword{arrowLW}.
When scaling an arrow by the keyword \Lkeyword{arrowscale} the width
of the borderline is also scaled. With the optional argument
\Lkeyword{arrowLW} the line width can be set separately and is not
taken into account by the scaling value.

\begin{LTXexample}[width=4cm]
\begin{pspicture}(4,6)
\psline[arrowscale=3,arrows=*-o](0,5)(4,5)
\psline[arrowscale=3,arrows=*-o,
  arrowLW=0.5pt](0,3)(4,3)
\psline[arrowscale=3,arrows=*-o,
  arrowLW=0.3333\pslinewidth](0,1)(4,1)
\end{pspicture}
\end{LTXexample}



\section{Ticks and other marks along a curve}
\subsection{Quick overview}

The macros described below allow you to place tick and other marks along an arbitrary 
parametric curve with placement rules similar to those used by \Lcs{psaxes} in 
the \LPack{pst-plot} package. You have to define a metric function along the curve to 
govern tick placement. That function can be a specified function of {\tt x,y} which 
should increase along the curve, or it can be an function whose increment is a specified 
positive function of {\tt x, y, dx, dy, ds} where the last term is the arc-length element 
that you could specify alternately as {\tt dx dup mul dy dup mul add sqrt}.
% start new material


In addition, a new command \Lcs{Put} is proposed, expanding as appropriate to \Lcs{rput} or \Lcs{uput}. Its syntax is

\begin{BDef}
\LcsStar{Put}\OptArgs\OptArg*{\Largb{<ref>}}\Largr{<position>}\Largb{<stuff>}
\end{BDef}

where the optional {\tt *} blanks the background, the optional \OptArgs\ may be used to specify a rotation 
using any form acceptable to \Lcs{SpecialCoor} (eg, \nxLkeyword{rot=45} or \Lkeyword{rot}\verb|={(1,1)}| 
or \Lkeyword{rot}\verb|=(P)|, and \Larg{ref} takes one of 
two forms: \verb=(a)= a refpt such as {\tt Bl}, in which case \Lcs{rput} is called; (b) a polar form of offset 
(eg, \verb=7pt;30=, or \verb=;(P)= --- in the latter case, \Ldim{pslabelsep} is substituted for the missing 
radius), in which case a modified form of \Lcs{uput} is called. The idea of \Lcs{Put} is to allow  {\tt position}, 
{\tt ref} and {\tt rot} to be specified in any of the forms acceptable to \Lcs{SpecialCoor} and to do so with 
the same output no matter what form is used. The cost of this consistency is that \Lcs{Put} can lead to results 
that differ from \Lcs{uput} in some special cases. 


\subsection{Details}
Suppose you have drawn a parametric curve using \Lcs{psparametricplot}, and you wish to 
indicate some points on the curve using tick-marks like those  on the axes. This is a 
two-step process, the first of which serves to define at the PostScript level a 
number of data arrays containing information about the curve. Those arrays are used 
in the second step to compute tick positions and draw the ticks. The first step is 
to run the macro \Lcs{pscurvepoints}. For example,

\begin{verbatim}
\pscurvepoints[plotpoints=20]{0}{6}{t t t mul 12 div}{Pt}%
\end{verbatim}
makes a virtual (ie, data only---nothing is rendered) polyline with 20 vertices approximating 
the curve $x(t)=t, y(t)=t^2/12$, $0\le t\le 6$. The last argument {\tt Pt} is the root name 
given to the data arrays.  PostScript arrays will be created with the following names: {\tt Pt.X, Pt.Y} 
for the coordinates of the vertices, {\tt PtDelta.X, PtDelta.Y} for the increments between the 
vertices (using, eg, {\tt PtDelta.X[2]=Pt.X[2]-Pt.X[1]}) and {\tt PtNormal.X, PtNormal.Y} for 
a vector normal to {\tt PtDelta.X, PtDelta.Y} in the visual, not mathematical, sense. 
(Both senses are the same if the scales on the axes are identical.) The {\tt Normal} is 
always constructed so as to point ``upward'' (ie, to your left) as you traverse the curve 
in the positive direction. The PostScript variable {\tt unitratio} provides the ratio of 
the unit on the y axis to that on x axis, and {\tt unitratiosq} is its square. All of 
these PostScript objects are stored in the main {\tt pstricks} dictionary \Lps{tx@Dict} 
which should be automatically made available when using many {\tt pstricks} macros. 
If {\tt gs} returns you an error message like
\begin{verbatim}
Error: /undefined in Pt.X
\end{verbatim}
then you may need to enclose the offending PostScript code within a block of the form
\begin{verbatim}
tx@Dict begin ... end
\end{verbatim}
so that the dictionary is made available.

With this preparation, the main tick-making macro may be run. For example,
\begin{verbatim}
\pspolylineticks{Pt}{ dx dy add 3 div }{1}{2}%
\end{verbatim}
looks for data arrays made using \Lcs{pscurvepoints} with the root name {\tt Pt}. The next argument, 
{\tt dx dy add 3 div}, specifies the (PostScript) function of increments that should be used to 
construct the metric. If the keyword \verb|metricInitValue| is defined, eg, with 
\Lcs{psset}\Largb{\Lkeyword{metricInitValue}=2.5}, it is used as the initial value of the metric, 
otherwise it is defined to be 0. In the previous example, the increment function is always 
positive, and care should be taken to guarantee this is so or the results will not be meaningful. 
(If we wanted to use arc-length, the function would have been {\tt ds}, assuming equal scales on 
the axes.) The last two arguments determine the index of the first tick and the number of ticks. 
Tick numbering begins with index 0, so the example says to drop the first tick and draw the 
next 2 ticks. In this example, where all keywords take their default values, ticks are 
potentially located at values on the curve where the metric takes a positive integer value. 
In the arc-length example, the tick with index 0 is at the beginning of the curve, and subsequent 
ticks are at unit distance, measured along the curve. At each index where a tick is drawn, a 
\Lcs{pnode} is created: In this example, you create nodes {\tt PtTick1, PtTick2} on the curve 
where the ticks are located. This is handy for placing labels using, eg, \Lcs{Put}. In 
addition, PostScript data arrays (in this example, {\tt PtTickN.X, PtTickN.Y} of the normals 
at these nodes are stored in the dictionary {\tt TDict}. More importantly, the tangent and 
normal vectors at {\tt PtTick0} etc are constructed as nodes with names {\tt PtTangent0, PtNormal0} 
etc. See the last example below for typical usage.

The shape of the ticks is governed by the keywords \Lkeyword{ticksize} (default value {\tt -4pt 4pt})
 and \Lkeyword{tickwidth} (default value \verb|.5\linewidth|.) With the default settings, ticks 
 are drawn perpendicular the the curve extending {\tt 4pt} to each side. The line
\begin{verbatim}
\pspolylineticks[ticksize=-6pt 6pt]{Pt}{ dx dy add 3 div }{1}{2}%
\end{verbatim}
would draw longer ticks than the default.

Placement of the ticks is governed by the keywords \Lkeyword{Ds} and \Lkeyword{Os}, whose meaning for the 
curve is similar to (but not the same as) the meanings of \Lkeyword{Dx} and \Lkeyword{Ox} with respect to the x axis. 
That is, if {\tt Ds=2} and {\tt Os=0}, ticks will be drawn where the metric takes 
values 0, 2, 4 and so on. More generally, ticks are placed where the metric takes 
value {\tt Os, Os+Ds, Os+2*Ds,...}, as long as those positions are on the curve. If \Lkeyword{Os} 
has an empty value as a result, say, of \verb|\psset{Os=}|, then \Lkeyword{Os} is set internally 
to the initial metric value. If \Lkeyword{Ds} has an empty value, it is set internally to the 
final metric value less the initial metric value, divided by 10. 

To draw major and minor ticks requires two passes---one to draw the minor ticks and then one to draw the major ticks.

Note that a ticks may be placed at arbitrary metric values on the curve by running the macro once for each point, like:
\begin{verbatim}
\pspolylineticks[ticksize=-6pt 6pt,Os=1.3]{Pt}{ dx dy add 3 div }{0}{1}%
\pspolylineticks[ticksize=-6pt 6pt,Os=2.4]{Pt}{ dx dy add 3 div }{0}{1}%
\end{verbatim}

You may also dispense entirely with the tick and use the macro to generate a node sequence 
that can be used to place other graphic objects. For example:
\begin{verbatim}
\pspolylineticks[ticksize=0pt 0pt]{Pt}{ dx dy add 3 div }{0}{3}%
%This defines nodes PtTick0..PtTick2
\multido{\iA=0+1}{3}{\psdot(PtTick\iA)}
\end{verbatim}


There is another way to define a metric function without using increments. If the keywork \Lkeyword{metricFunction} is set to \true, 
then the function you present as an argument to \Lcs{pspolylineticks} must be a function of 
$x$ and $y$ only, and must be designed to increase along the curve. It is useful only in 
those cases where, in essence, the increment function can be explicitly integrated. 
For example, in the elliptical motion of planets and comets around the sun, it is not hard 
to integrate the area function explicitly, and this provides a convenient metric, being proportional to time elapsed.

There is some useful information left in the log by these macros. 
They report the starting and ending values of the metric function, 
the the range of indices for the Tick related arrays.

\subsection{Examples}
The examples in this section make use of very recent (as of May, 2010) versions 
of \LPack{pstricks} and related packages. 
%If the {\tt pst-grapha} package is not available on CTAN,  download it from
%\begin{verbatim}
%http://math.ucsd.edu/~msharpe/pst-grapha.dmg
%\end{verbatim}

The first couple of examples are constructed entirely by hand, and have no interest 
other than to illustrate what is going on under the surface in the simplest case.

\begin{LTXexample}[pos=t]
\begin{pspicture}(-1,-1)(10,4)
\psline[showpoints=true](1,2)(4,0)(9,3)%
\uput[180](1,2){$s=0$}%
\uput[-90](4,0){$s=1$}%
\uput[0](9,3){$s=2$}%
\makeatletter% need to use macro names containing @
\pstVerb{tx@Dict begin %the pstricks dictionary
% declare arrays of length 3 (indices 0,1,2) to hold points, 
% differences and normals
/unitratiosq 1 def % yunit=xunit
/P.X [ 1 4 9 ] def %array of x coords
/P.Y [ 2 0 3 ] def %array of y coords
/PDelta.X [ 0 3 5 ] def % 3=4-1, 5=9-4, 0 never used
/PDelta.Y [ 0 -2 3 ] def % -2=0-2, 3=3-0, 0 never used
% normal to (3,-2) is (2,3), normal to (5,3) is (-3,5)
/PNormal.X [ 2 2 -3 ] def % index 0 =index 1
/PNormal.Y [ 3 3 5 ] def % index 0 = index 1
end }
\def\Ppointcount{2}
\makeatother
% make ticks using metric function with values 0,1,2
\pspolylineticks[Os=.5,Ds=1]{P}{1}{0}{2}
% ticks at s=0.5,1.5 (increment function =1)
\uput[-135](PTick0){$s=0.5$}%
\uput[-45](PTick1){$s=1.5$}%
\end{pspicture}
\end{LTXexample}

\clearpage
Now the same data, but with arc-length as metric. We change the last few lines:

\begin{LTXexample}[pos=t]
\begin{pspicture}(-1,-1)(10,4)
\psline[showpoints=true](1,2)(4,0)(9,3)%
%\uput[180](1,2){$s=0$}%
%\uput[-90](4,0){$s=1$}%
%\uput[0](9,3){$s=2$}%
\makeatletter% need to use macro names containing @
\pstVerb{tx@Dict begin %the pstricks dictionary
% declare arrays of length 3 (indices 0,1,2) to hold points, 
% differences and normals
/unitratiosq 1 def % yunit=xunit
/P.X [ 1 4 9 ] def %array of x coords
/P.Y [ 2 0 3 ] def %array of y coords
/PDelta.X [ 0 3 5 ] def % 3=4-1, 5=9-4, 0 never used
/PDelta.Y [ 0 -2 3 ] def % -2=0-2, 3=3-0, 0 never used
% normal to (3,-2) is (2,3), normal to (5,3) is (-3,5)
/PNormal.X [ 2 2 -3 ] def % index 0 =index 1
/PNormal.Y [ 3 3 5 ] def % index 0 = index 1
end }
\def\Ppointcount{2}
\makeatother
% make ticks using metric function arc-length
\pspolylineticks[Os=1,Ds=1]{P}{ ds }{0}{9}
% ticks at s=1,2... (increment function = distance)
\uput[-135](PTick0){$s=1$}%
\uput[-135](PTick1){$s=2$}%
\end{pspicture}
\end{LTXexample}

\clearpage
Once again the same data, but with metric equal to the x coordinate. Change the last few lines to:

\begin{LTXexample}[pos=t]
\begin{pspicture}(-1,-1)(10,4)
\psline[showpoints=true](1,2)(4,0)(9,3)%
%\uput[180](1,2){$s=0$}%
%\uput[-90](4,0){$s=1$}%
%\uput[0](9,3){$s=2$}%
\makeatletter% need to use macro names containing @
\pstVerb{tx@Dict begin %the pstricks dictionary
% declare arrays of length 3 (indices 0,1,2) to hold points, 
% differences and normals
/unitratiosq 1 def % yunit=xunit
/P.X [ 1 4 9 ] def %array of x coords
/P.Y [ 2 0 3 ] def %array of y coords
/PDelta.X [ 0 3 5 ] def % 3=4-1, 5=9-4, 0 never used
/PDelta.Y [ 0 -2 3 ] def % -2=0-2, 3=3-0, 0 never used
% normal to (3,-2) is (2,3), normal to (5,3) is (-3,5)
/PNormal.X [ 2 2 -3 ] def % index 0 =index 1
/PNormal.Y [ 3 3 5 ] def % index 0 = index 1
end }
\def\Ppointcount{2}
\makeatother
% make ticks using metric function arc-length
\pspolylineticks[metricFunction,Os=1,Ds=2]{P}{ x }{0}{5}
% ticks at x=1,3,... , start at tick index 0, draw 5 ticks
% the tick at s=1 has index 0
% ticks at s=1,2... (increment function = distance)
\uput[-135](PTick0){$s=1$}%
\uput[-135](PTick1){$s=3$}%
\end{pspicture}
\end{LTXexample}

\clearpage
The next example is a smooth path where subticks are drawn first, followed by major ticks. 
The metric is arc-length with initial value $s=1$.
\begin{LTXexample}[pos=t]
\begin{pspicture}(-1,-1)(10,4)
%\parametricplot[algebraic]{0}{9}{(t^2)/9 | 4*Ex(-t)*(1+t+(t^{2})/2+(t^{3})/6)}
\psparametricplot[algebraic]{0}{9}{t^2/9 | sin(t)+1}%
\pscurvepoints{0}{9}{(t^2)/9 | sin(t)+1}{P}%
% make ticks using  arc-length metric
\pspolylineticks[metricInitValue=1,ticksize=-2pt 2pt,Os=1,Ds=.2]{P}{ ds }{1}{56}%
\pspolylineticks[metricInitValue=1,Os=1,Ds=2]{P}{ ds }{0}{6}%
\multido{\iA=1+1,\iB=3+2}{5}{\Put{6pt;(PNormal\iA)}(PTick\iA){\tiny \iB}}
%\nodexn{(PTick\iA)+(10pt;{(PNormal\iA)})}{Q}\rput(Q){\tiny \iB}}%
%\multido{\iA=1+1,\iB=3+2}{5}{\uput{6pt}[{(PNormal\iA)}](PTick\iA){\iB}}%
% ticks at x=1,3,... , start at tick index 0, draw 5 ticks
% the tick at s=1 has index 0
% ticks at s=1,2... (increment function = distance)
\end{pspicture}
\end{LTXexample}

\clearpage
Suppose for the next example that we have an ellipse $x^2/a^2+y^2/b^2=1$ ($a>b$) with 
eccentricity $\epsilon=(1-b^2/a^2)^{1/2}$. With planetary motion in mind, a natural metric 
for the ellipse is the area swept out by the radial line from the focus $(\epsilon a,0)$ 
starting from $(a,0)$ around to an arbitrary location $(x,y)$, where $y>0$, as this quantity 
is proportional to the time elapsed since perihelion. A routine calculation gives the following formula:
\[A=\frac{ab}{2}\arccos\bigg(\frac{x}{a}\bigg)-\frac{\epsilon a y}{2}.\]
Remembering that PostScript's {\tt acos} gives  its result in degrees, not radians, we have the 
following, drawn for the case $a=4$, $b=3$.

\begin{LTXexample}[pos=t]
\begin{pspicture}(-4.5,-.5)(4.5,3.5)
\pstVerb{ /smajor 4 def /sminor 3 def % define semimajor, semiminor 
/ecc 1 sminor smajor div dup mul sub sqrt def % compute eccentricity
/ab smajor sminor mul 2 div def %first coeff
/ea smajor ecc mul 2 div def }% second coeff
\psparametricplot[algebraic]{0}{3.142}{smajor*cos(t) | sminor*sin(t)}%
\pscurvepoints{0}{3.142}{smajor*cos(t) | sminor*sin(t)}{P}%
\pspolylineticks[metricFunction,Ds=2,ticksize=-1.5pt 0]{P}{ ab x smajor div acos %
180 div PI mul mul  ea y mul sub }{1}{9}%
\pnode(! ecc smajor mul 0){S}% focus
\psline[linecolor=lightgray](S)(!smajor 0)%
\multido{\i=1+1}{9}{\psline[linecolor=lightgray](S)(PTick\i)}
\psdot(S)
\end{pspicture}
\end{LTXexample}

\clearpage
The next examples works without visible ticks, using the macros to construct nodes at which other objects will be placed.

\begin{LTXexample}[pos=t]
\begin{pspicture}(-1,-1)(10,4)
\psparametricplot[algebraic]{0}{9}{t| 3*Ex(-t)*(1+t+t^2/2+t^3/6)}
\pscurvepoints{0}{9}{t| 3*Ex(-t)*(1+t+t^2/2+t^3/6)}{P}%
\pspolylineticks[Os=1,Ds=1,ticksize=0 0]{P}{ ds }{0}{9}%
\multido{\i=0+1}{9}{\psdot[dotscale=1.5,dotstyle=o](PTick\i)}%
% ticks at s=1,2,... , start at tick index 0, set 9 ticks
% the tick at s=1 has index 0
% ticks at s=1,2... (increment function = distance)
\multido{\i=0+3}{3}{\Put[rot=(PTangent\i)]{7pt;(PNormal\i)}(PTick\i){PTick\i}}%
\uput[-135](PTick1){$s=2$}%
\end{pspicture}
\end{LTXexample}

This variant also has no visible ticks, but makes a color gradient along the curve based on arc-length from the start.

\begin{LTXexample}[pos=t]
\begin{pspicture}(-1,-1)(10,4)
\psparametricplot[plotpoints=200,linecolor=white]{0}{360}{ t cos 1 add 4 mul t 1 add 20 div ln 2 div 1 add }
\pscurvepoints[plotpoints=200]{0}{360}{ t cos 1 add 4 mul t 1 add 20 div ln 2 div 1 add }{P}%
\pspolylineticks[Os=0,Ds=.2,ticksize=0 0]{P}{ ds }{0}{90}%
\definecolorseries{ctest}{hsb}{last}{green}{violet}
\resetcolorseries[88]{ctest}%
\multido{\iA=0+1,\iB=1+1}{87}{\psline[linewidth=2pt,linecolor=ctest!![\iB](PTick\iA)(PTick\iB)}%
\end{pspicture}
\end{LTXexample}

\clearpage
Here is a another variant of this technique which allows arrows to be placed at locations 
on the curve where the metric takes particular values.

\begin{LTXexample}[pos=t]
\begin{pspicture}(-1,-1)(10,4.5)
\psparametricplot[plotpoints=100]{0}{360}{t cos 1 add 5 mul t sin 1 add 2 mul}
\pscurvepoints[plotpoints=100]{0}{360}{t cos 1 add 5 mul t sin 1 add 2 mul}{P}%
\pspolylineticks[Os=0,Ds=2.3,ticksize=0 0]{P}%
{ ds }{0}{10}% distance
\multido{\i=0+1}{10}{\psrline[arrows=->,arrowscale=1.5](PTick\i)(2pt;{(PTangent\i)})}%
\end{pspicture}
\end{LTXexample}

\section{Troubleshooting}
If you get PostScript errors when you process your file, the  most likely culprit is the 
function you specified to define the metric. There are some  things to look out for:
\begin{itemize}
\item If \Lkeyword{metricFunction}, the function you specify in PostScript code must 
involve only {\tt x} and {\tt y}, and must leave exactly one real value on the stack as a result of 
substituting specific values for {\tt x} and {\tt y}. The function must be strictly increasing on the curve.
\item If \Lkeyword{metricFunction}=\false (the default), the function you specify in PostScript 
code must involve only the variables {\tt x}, {\tt y}, {\tt dx}, {\tt dy}, {\tt ds} (where {\tt ds} 
is defined to be the arc-length element {\tt dx dup mul dy dup mul add sqrt}, and must leave exactly 
one strictly positive real value on the stack when specific values are substituted for those variables. 
The constant function {\tt 1} gives equal weight to each segment in the curve, so in effect it gives 
you the original parametrization, up to a constant factor.
\item If the function you specify in \Lcs{parametricplot} and \Lcs{pscurvepoints} is \Lkeyword{algebraic}, 
make sure you follow precisely the syntax it understands. In complex cases, PostScript may be the safer solution.
\item It is unwise to use a different resolution for \Lcs{psparametricplot} and \Lcs{pscurvepoints}. 
The default value of \Lkeyword{plotpoints}=50 is marginal except for modest curve segments, and 200 should 
suffice for most smooth curves.
\end{itemize}


%--------------------------------------------------------------------------------------
\section{Transparent colors}
%--------------------------------------------------------------------------------------

Transparency is now part of the main \LPack{pstricks} package.
But pay attention, the names and syntax have changed and you need
to run \Lprog{ps2pdf} with the option
\Loption{-dCompatibilityLevel}=1.4.


%--------------------------------------------------------------------------------------
\section{,,Manipulating transparent colors''}
%--------------------------------------------------------------------------------------

\LPack{pstricks-add} supports real transparency and a simulated one with hatch lines:
\begin{lstlisting}
\def\defineTColor{\@ifnextchar[{\defineTColor@i}{\defineTColor@i[]}}
\def\defineTColor@i[#1]#2#3{%     transparency "Colors"
  \newpsstyle{#2}{%
     fillstyle=vlines,hatchwidth=0.1\pslinewidth,
     hatchsep=1\pslinewidth,hatchcolor=#3,#1%
  }%
}
\defineTColor{TRed}{red}
\defineTColor{TGreen}{green}
\defineTColor{TBlue}{blue}
\end{lstlisting}

There are three predefined "'transparent"` colors \verb+TRed+,
\verb+TGreen+, \verb+TBlue+. They are used as \PST{} styles and
not as colors:

\bgroup
\begin{LTXexample}[pos=t,preset=\centering]
\begin{pspicture}(-3,-5)(5,5)
\psframe(-1,-3)(5,5) % objet de base
\psrotate(2,-2){15}{%
  \psframe[style=TRed](-1,-3)(5,5)}
\psrotate(2,-2){30}{%
  \psframe[style=TGreen](-1,-3)(5,5)}
\psrotate(2,-2){45}{%
  \psframe[style=TBlue](-1,-3)(5,5)}
\psframe[linewidth=3pt](-1,-3)(5,5)
\psdots[dotstyle=+,dotangle=45,dotscale=3](2,-2) % centre de la rotation
\end{pspicture}
\end{LTXexample}
\egroup

%--------------------------------------------------------------------------------------
\section{Calculated colors}
%--------------------------------------------------------------------------------------
The \verb+xcolor+ package (version 2.6) has a new feature for defining colors:
\begin{lstlisting}[style=syntax]
  \definecolor[ps]{<name>}{<model>}{< PS code >}
\end{lstlisting}

\verb+model+ can be one of the color models, which \PS will
understand, e.g. \verb+rgb+. With this definition the color is
calculated on the \PS side.
\begin{LTXexample}[pos=t,preset=\centering]
\definecolor[ps]{bl}{rgb}{tx@addDict begin  Red Green Blue end}%
\psset{unit=1bp}
\begin{pspicture}(0,-30)(400,100)
\multido{\iLAMBDA=0+1}{400}{%
  \pstVerb{
    \iLAMBDA\space 379 add dup /lambda exch def
    tx@addDict begin  wavelengthToRGB end
  }%
  \psline[linecolor=bl](\iLAMBDA,0)(\iLAMBDA,100)%
}
\psaxes[yAxis=false,Ox=350,dx=50bp,Dx=50]{->}(-29,-10)(420,100)
\uput[-90](420,-10){$\lambda$[\textsf{nm}]}
\end{pspicture}
\end{LTXexample}


\begin{center}
\newcommand{\Touch}{%
\psframe[linestyle=none,fillstyle=solid,fillcolor=bl,dimen=middle](0.1,0.75)}
\definecolor[ps]{bl}{rgb}{tx@addDict begin Red Green Blue end}%
% Echelle 1cm <-> 40 nm
%         1 nm <-> 0.025 cm
\psframebox[fillstyle=solid,fillcolor=black]{%
\begin{pspicture}(-1,-0.5)(12,1.5)
\multido{\iLAMBDA=380+2}{200}{%
  \pstVerb{
    /lambda \iLAMBDA\space def
    lambda
    tx@addDict begin  wavelengthToRGB end
  }%
 \rput(! lambda 0.025 mul 9.5 sub 0){\Touch}
}
\multido{\n=0+1,\iDiv=380+40}{11}{%
    \psline[linecolor=white](\n,0.1)(\n,-0.1)
    \uput[270](\n,0){\textbf{\white\iDiv}}}
    \psline[linecolor=white]{->}(11,0)
    \uput[270](11,0){\textbf{\white$\lambda$(nm)}}
\end{pspicture}}

\psframebox[fillstyle=solid,fillcolor=black]{%
\begin{pspicture}(-1,-0.5)(12,1)
  \pstVerb{
    /lambda 656 def
    lambda
    tx@addDict begin  wavelengthToRGB end
  }%
 \rput(! 656 0.025 mul 9.5 sub 0){\Touch}
  \pstVerb{
    /lambda 486 def
    lambda
    tx@addDict begin  wavelengthToRGB end
  }%
 \rput(! 486 0.025 mul 9.5 sub 0){\Touch}
   \pstVerb{
    /lambda 434 def
    lambda
    tx@addDict begin  wavelengthToRGB end
  }%
 \rput(! 434 0.025 mul 9.5 sub 0){\Touch}
  \pstVerb{
    /lambda 410 def
    lambda
    tx@addDict begin  wavelengthToRGB end
  }%
 \rput(! 410 0.025 mul 9.5 sub 0){\Touch}
\multido{\n=0+1,\iDiv=380+40}{11}{%
    \psline[linecolor=white](\n,0.1)(\n,-0.1)
    \uput[270](\n,0){\textbf{\white\iDiv}}}
    \psline[linecolor=white]{->}(11,0)
    \uput[270](11,0){\textbf{\white$\lambda$(nm)}}
\end{pspicture}}

\Index{Spectrum} of \Index{hydrogen} emission (Manuel Luque)
\end{center}

\begin{lstlisting}
\newcommand\Touch{%
\psframe[linestyle=none,fillstyle=solid,fillcolor=bl,dimen=middle](0.1,0.75)}
\definecolor[ps]{bl}{rgb}{tx@addDict begin Red Green Blue end}%
% Echelle 1cm <-> 40 nm
%         1 nm <-> 0.025 cm
\psframebox[fillstyle=solid,fillcolor=black]{%
\begin{pspicture}(-1,-0.5)(12,1.5)
\multido{\iLAMBDA=380+2}{200}{%
  \pstVerb{
    /lambda \iLAMBDA\space def
    lambda
    tx@addDict begin  wavelengthToRGB end
  }%
 \rput(! lambda 0.025 mul 9.5 sub 0){\Touch}
}
\multido{\n=0+1,\iDiv=380+40}{11}{%
    \psline[linecolor=white](\n,0.1)(\n,-0.1)
    \uput[270](\n,0){\textbf{\white\iDiv}}}
    \psline[linecolor=white]{->}(11,0)
    \uput[270](11,0){\textbf{\white$\lambda$(nm)}}
\end{pspicture}}

\psframebox[fillstyle=solid,fillcolor=black]{%
\begin{pspicture}(-1,-0.5)(12,1)
  \pstVerb{
    /lambda 656 def
    lambda
    tx@addDict begin  wavelengthToRGB end
  }%
 \rput(! 656 0.025 mul 9.5 sub 0){\Touch}
  \pstVerb{
    /lambda 486 def
    lambda
    tx@addDict begin  wavelengthToRGB end
  }%
 \rput(! 486 0.025 mul 9.5 sub 0){\Touch}
   \pstVerb{
    /lambda 434 def
    lambda
    tx@addDict begin  wavelengthToRGB end
  }%
 \rput(! 434 0.025 mul 9.5 sub 0){\Touch}
  \pstVerb{
    /lambda 410 def
    lambda
    tx@addDict begin  wavelengthToRGB end
  }%
 \rput(! 410 0.025 mul 9.5 sub 0){\Touch}
\multido{\n=0+1,\iDiv=380+40}{11}{%
    \psline[linecolor=white](\n,0.1)(\n,-0.1)
    \uput[270](\n,0){\textbf{\white\iDiv}}}
    \psline[linecolor=white]{->}(11,0)
    \uput[270](11,0){\textbf{\white$\lambda$(nm)}}
\end{pspicture}}

Spectrum of hydrogen emission (Manuel Luque)
\end{lstlisting}



%--------------------------------------------------------------------------------------
\section{Gouraud shading}
%--------------------------------------------------------------------------------------
\begin{quotation}
\Index{Gouraud} shading is a method used in computer graphics to simulate the differing effects of
light and colour across the surface of an object. In practice, Gouraud shading is used to
achieve smooth lighting on low-polygon surfaces without the heavy computational requirements
of calculating lighting for each pixel. The technique was first presented by Henri Gouraud in 1971.\\
~\hfill{\small \url{http://www.wikipedia.org}}
\end{quotation}

PostScript level 3 supports this kind of shading and it can only
be seen with Acroread 7 or later. The syntax is easy:

\begin{lstlisting}[style=syntax]
  \psGTriangle(x1,y1)(x2,y2)(x3,y3){color1}{color2}{color3}
\end{lstlisting}

\psset{unit=0.75cm}

\begin{LTXexample}[pos=t,preset=\centering]
\begin{pspicture}(0,-.25)(10,10)
  \psGTriangle(0,0)(5,10)(10,0){red}{green}{blue}
\end{pspicture}
\end{LTXexample}

\begin{LTXexample}[pos=t,preset=\centering]
\begin{pspicture}(0,-.25)(10,10)
  \psGTriangle*(0,0)(9,10)(10,3){black}{white!50}{red!50!green!95}
\end{pspicture}
\end{LTXexample}

\begin{LTXexample}[pos=t,preset=\centering]
\begin{pspicture}(0,-.25)(10,10)
  \psGTriangle*(0,0)(5,10)(10,0){-red!100!green!84!blue!86}
                               {-red!80!green!100!blue!40}
                               {-red!60!green!30!blue!100}
\end{pspicture}
\end{LTXexample}

\begin{LTXexample}[pos=t,preset=\centering]
\definecolor{rose}{rgb}{1.00, 0.84, 0.88}
\definecolor{vertpommepasmure}{rgb}{0.80, 1.0, 0.40}
\definecolor{fushia}{rgb}{0.60, 0.30, 1.0}
\begin{pspicture}(0,-.25)(10,10)
  \psGTriangle(0,0)(5,10)(10,0){rose}{vertpommepasmure}{fushia}
\end{pspicture}
\end{LTXexample}

\section{Internal color macros}
The internal macros \Lcs{pswavelengthToRGB} and \Lcs{pswavelengthToRGB} can be used for own purposed.
They are defines as follows:

\begin{lstlisting}
\def\pswavelengthToGRAY{ tx@addDict begin wavelengthToGRAY end }
\def\pswavelengthToRGB{ tx@addDict begin wavelengthToRGB Red Green Blue end }
\end{lstlisting}

both macros leave the value(s) on the stack which then can be used for further
manipulating or setting the color with \Lps{setgray} or \Lps{setrgbcolor}. 
For an example see Section~\ref{sec:psMatrix}.

\appendix


%--------------------------------------------------------------------------------------
\clearpage
\section{\nxLcs{resetOptions}}
%--------------------------------------------------------------------------------------

Sometimes it is difficult to know what options, which are changed
inside a long document, are different to the default ones. With
this macro all options belonging to \LPack{pst-plot} can be reset.
This refers to all options of the packages \LPack{pstricks},
\LPack{pst-plot} and \LPack{pst-node}.



%--------------------------------------------------------------------------------------
\section{PostScript}
%--------------------------------------------------------------------------------------

\Index{PostScript} uses the stack system and the LIFO system, "'Last In, First Out"`.

\newlength{\Li}\settowidth{\Li}{Function}
\begin{table}[htbp]
\caption{Some primitive PostScript macros}\label{tab:primpost}
\centering
\ttfamily
    \begin{tabular}{@{} l | r@{ $\rightarrow$ } l @{}}\hline
    \multirow{2}{\Li}{\normalfont\emph{Function}} & \multicolumn{2}{ c }{\normalfont\emph{Meaning}}\\
    &\normalfont\emph{on stack before} & \normalfont\emph{after}\\\hline
    \Lps{add} & $x\quad y$&$x+y$\\ 
    \Lps{sub} & $x\quad y$&$x-y$\\ 
    \Lps{mul} & $x\quad y$&$x\times y$\\ 
    \Lps{div} & $x\quad y$&$x\div y$\\ 
    \Lps{sqrt} & $x$&$\sqrt{x}$\\ 
    \Lps{abs} & $x$&$|x|$\\ 
    \Lps{neg} & $x$&$-x$\\ 
    \Lps{cos} & $x$&$\cos(x)$ ($x$ in degrees)\\ 
    \Lps{sin} & $x$&$\sin(x)$ ($x$  in degrees)\\ 
    \Lps{tan} & $x$&$\tan(x)$ ($x$  in degrees)\\ 
    \Lps{atan} & $y\quad x$&$\angle{(\vec{Ox};\vec{OM})}$ (in degrees of $M(x,y)$)\\ 
    \Lps{ln} & $x$&$\ln(x)$\\ 
    \Lps{log} & $x$&$\log(x)$\\ 
    \Lps{array} & $n$&\normalfont$v$ (of dimension $n$)\\ 
    \Lps{aload} & $v$&$x_1\quad x_2\quad \cdots\quad x_n\quad v$\\ 
    \Lps{astore} & $x_1\quad x_2\quad \cdots\quad x_n\quad v$ & $[v]$\\ 
    \Lps{pop} & $x$ & --\\ 
    \Lps{dup} & $x$ & $x\quad x$ \\\hline
%    \Lps{roll} & $x_1\quad x_2\quad \cdots\quad x_n\quad n p$ &\\\hline
  \end{tabular}
\end{table}


\clearpage
\section{List of all optional arguments for \texttt{pstricks-add}}

\xkvview{family=pstricks-add,columns={key,type,default}}





\nocite{*}
\bgroup
\RaggedRight
\bibliographystyle{plain}
\bibliography{pstricks-add-doc}
\egroup

\printindex




\end{document}


\usepackage[utf8]{inputenc}
\usepackage{pstricks-add}
\let\pstricksaddFV\fileversion
\usepackage{pst-eucl,pst-fun,pst-func,multirow}
\usepackage{pifont}
\let\belowcaptionskip\abovecaptionskip
%
\def\textat{\char064}%
\newdimen\fullWidth
\makeatletter
\renewcommand*\l@section{\@dottedtocline{1}{2em}{2.3em}}
\renewcommand*\l@subsection{\@dottedtocline{2}{3.8em}{3.2em}}
\renewcommand*\l@subsubsection{\@dottedtocline{3}{7.0em}{4.1em}}
\renewcommand*\l@paragraph{\@dottedtocline{4}{10em}{5em}}
\makeatother
\lstset{explpreset={pos=l,width=-99pt,overhang=0pt,hsep=\columnsep,vsep=\bigskipamount,rframe={}},
    escapechar=§}

\def\bgImage{\psset{unit=1.5}
\begin{pspicture}(-3,-3)(3,3)
\psChart[userColor={red!30,green!30,blue!40,gray,cyan!50,
    magenta!60,cyan},chartSep=30pt,shadow=true,shadowsize=5pt]{34.5,17.2,20.7,15.5,5.2,6.9}{6}{2}
\psset{nodesepA=5pt,nodesepB=-10pt}
\ncline{psChartO1}{psChart1}\nput{0}{psChartO1}{1000 (34.5\%)}
\ncline{psChartO2}{psChart2}\nput{150}{psChartO2}{500 (17.2\%)}
\ncline{psChartO3}{psChart3}\nput{-90}{psChartO3}{600 (20.7\%)}
\ncline{psChartO4}{psChart4}\nput{0}{psChartO4}{450 (15.5\%)}
\ncline{psChartO5}{psChart5}\nput{0}{psChartO5}{150 (5.2\%)}
\ncline{psChartO6}{psChart6}\nput{0}{psChartO6}{200 (6.9\%)}
\bfseries%
\rput(psChartI1){Taxes}\rput(psChartI2){Rent}\rput(psChartI3){Bills}
\rput(psChartI4){Car}\rput(psChartI5){Gas}\rput(psChartI6){Food}
\end{pspicture}}

\begin{document}
\title{\texttt{pstricks-add}\\additionals Macros for \texttt{pstricks}\\
    \small v.\pstricksaddFV}
%\docauthor{Herbert Vo\ss}
\author{Dominique Rodriguez\\Michael Sharpe\\Herbert Vo\ss}
\date{\today}

\maketitle

\fullWidth=\linewidth
\advance\fullWidth by \marginparsep
\advance\fullWidth by \marginparwidth


\begin{abstract}
This version of \verb+pstricks-add+ needs \verb+pstricks.tex+
version >1.04 from June 2004, otherwise the additional macros may
not work as expected. The ellipsis material and the option
\verb+asolid+ (renamed to \verb+eofill+) are
\index{fillstyle!eofill@\texttt{eofill}} now part of the new
\verb+pstricks.tex+ package, available on CTAN. \LPack{pstricks-add} will for ever be
an experimental and dynamical package, try it at your own risk.

\begin{itemize}
\item It is important to load \LPack{pstricks-add} as the \textbf{last} PSTricks related package, otherwise
a lot of the macros won't work in the expected way.
\item \LPack{pstricks-add} uses the extended version of the keyval package. So be sure that
you have installed \LPack{pst-xkey} which is part of the
\LPack{xkeyval}-package, and that all packages that use the old
keyval interface are loaded \textbf{before} the
\LPack{xkeyval}.\cite{xkeyval}
\item the option \Lkeyword{tickstyle} from \LPack{pst-plot} is no longer supported; use \Lkeyword{ticksize} instead.
\item the option \Lkeyword{xyLabel} is no longer supported; use the option \Lkeyword{labelFontSize} instead.
\item if \LPack{pstricks-add} is loaded together with the package  \LPack{pst-func} then  \Lkeyword{InsideArrow}
    of the \Lcs{psbezier} macro doesn't work!
\end{itemize}

\vfill
\noindent
Thanks to:  
Hendri Adriaens;
Stefano Baroni;
Martin Chicoine;
Gerry Coombes;
Ulrich Dirr;
Christophe Fourey;
Hubert G\"a\ss lein;
J\"urgen Gilg;
Denis Girou;
Pablo Gonzáles;
Peter Hutnick;
Christophe Jorssen;
Uwe Kern;
Manuel Luque;
Jens-Uwe Morawski;
Tobias N\"ahring;
Rolf Niepraschk;
Alan Ristow;
Christine R\"omer;
Arnaud Schmittbuhl;
John Smith;
Timothy Van Zandt
\end{abstract}

\clearpage
\tableofcontents


\clearpage

\section{\nxLcs{psGetSlope} and \nxLcs{psGetDistance}}
%--------------------------------------------------------------------------------------

\begin{BDef}
\Lcs{psGetSlope}\coord1\coord2\Lcs{\Larga{macro}}\\
\Lcs{psGetDistance}\coord1\coord2\Lcs{\Larga{macro}}
\end{BDef}

\begin{LTXexample}[width=4cm]
\psGetSlope(-2,1)(3,1)\SlopeVal \SlopeVal \quad
\psGetDistance(-2,1)(3,1)\DVal \DVal\\
\psGetSlope(-2,1)(-3,-1)\SlopeVal \SlopeVal\quad
\psGetDistance(-2,1)(-3,-1)\DVal \DVal\\
\psGetSlope(-2,0)(3,-1)\SlopeVal \SlopeVal\quad
\psGetDistance(-2,0)(3,-1)\DVal \DVal\\
\psGetSlope(-2111,-12)(3,1)\SlopeVal \SlopeVal\quad
%\psGetDistance(-2111,-12)(3,1)\DVal ==> Overflow!
\end{LTXexample}

\clearpage

%--------------------------------------------------------------------------------------
\section{"`Handmade"' lines :-)}
%--------------------------------------------------------------------------------------

\begin{BDef}
\Lcs{pslineByHand}\OptArgs\coord1\coord2\coord3 \ldots
\end{BDef}

\begin{LTXexample}[width=0.4\linewidth]
\begin{pspicture}(4,6)
\psset{unit=2cm}
  \pslineByHand[linecolor=red](0,0)(0,2)(2,2)(2,0)(0,0)(2,2)(1,3)(0,2)(2,0)
\end{pspicture}
\end{LTXexample}

\iffalse
  \pslineByHand( 1.20, 1.50)( 1.20, 1.51)( 1.20, 1.53)( 1.20, 1.54)( 1.19, 1.55)( 1.19, 1.56)
    ( 1.19, 1.57)( 1.18, 1.59)( 1.18, 1.60)( 1.17, 1.61)( 1.16, 1.62)( 1.15, 1.63)( 1.15, 1.64)
    ( 1.14, 1.65)( 1.13, 1.65)( 1.12, 1.66)( 1.11, 1.67)( 1.10, 1.68)( 1.09, 1.68)( 1.07, 1.69)
    ( 1.06, 1.69)( 1.05, 1.69)( 1.04, 1.70)( 1.03, 1.70)( 1.01, 1.70)( 1.00, 1.70)( 0.99, 1.70)
    ( 0.97, 1.70)( 0.96, 1.70)( 0.95, 1.69)( 0.94, 1.69)( 0.93, 1.69)( 0.91, 1.68)( 0.90, 1.68)
    ( 0.89, 1.67)( 0.88, 1.66)( 0.87, 1.65)( 0.86, 1.65)( 0.85, 1.64)( 0.85, 1.63)( 0.84, 1.62)
    ( 0.83, 1.61)( 0.82, 1.60)( 0.82, 1.59)( 0.81, 1.57)( 0.81, 1.56)( 0.81, 1.55)( 0.80, 1.54)
    ( 0.80, 1.53)( 0.80, 1.51)( 0.80, 1.50)( 0.80, 1.49)( 0.80, 1.47)( 0.80, 1.46)( 0.81, 1.45)
    ( 0.81, 1.44)( 0.81, 1.43)( 0.82, 1.41)( 0.82, 1.40)( 0.83, 1.39)( 0.84, 1.38)( 0.85, 1.37)
    ( 0.85, 1.36)( 0.86, 1.35)( 0.87, 1.35)( 0.88, 1.34)( 0.89, 1.33)( 0.90, 1.32)( 0.91, 1.32)
    ( 0.93, 1.31)( 0.94, 1.31)( 0.95, 1.31)( 0.96, 1.30)( 0.97, 1.30)( 0.99, 1.30)( 1.00, 1.30)
    ( 1.01, 1.30)( 1.03, 1.30)( 1.04, 1.30)( 1.05, 1.31)( 1.06, 1.31)( 1.07, 1.31)( 1.09, 1.32)
    ( 1.10, 1.32)( 1.11, 1.33)( 1.12, 1.34)( 1.13, 1.35)( 1.14, 1.35)( 1.15, 1.36)( 1.15, 1.37)
    ( 1.16, 1.38)( 1.17, 1.39)( 1.18, 1.40)( 1.18, 1.41)( 1.19, 1.43)( 1.19, 1.44)( 1.19, 1.45)
    ( 1.20, 1.46)( 1.20, 1.47)( 1.20, 1.49)( 1.20, 1.50)
\fi

\begin{LTXexample}[pos=t]
\begin{pspicture}(\linewidth,3)
\multido{\rA=0.00+0.25}{12}{\pslineByHand[linecolor=blue](0,\rA)(\linewidth,\rA)}
\end{pspicture}
\end{LTXexample}

The amplitude and the width can be changed by the optional arguments \Lkeyword{varsteptol} and
\Lkeyword{VarStepEpsilon}. Both are preset to \verb+VarStepEpsilon=2,varsteptol=0.8+.


\begin{LTXexample}[pos=t]
\begin{pspicture}(\linewidth,3)
\multido{\rA=0.00+0.25}{12}{%
  \pslineByHand[linecolor=blue,VarStepEpsilon=4,varsteptol=2](0,\rA)(\linewidth,\rA)}
\end{pspicture}
\end{LTXexample}

\clearpage

%--------------------------------------------------------------------------------------
\section{\nxLcs{rmultiput}: a multiple \nxLcs{rput}}
%--------------------------------------------------------------------------------------
\verb+PSTricks+ already has a \Lcs{multirput}, which puts a box n
times with a difference of $dx$ and $dy$ relative to each other.
It is not possible to put it with a different distance from one
point to the next. This is possible with \Lcs{rmultiput}:

\begin{BDef}
\LcsStar{rmultiput}\OptArgs\Largb{any material}\coord1\coord2\ldots\Largr{\coord{n}}
\end{BDef}

\begin{LTXexample}[width=6.2cm]
\psset{unit=0.75}
\begin{pspicture}(-4,-4)(4,4)
\rmultiput[rot=45]{\red\psscalebox{3}{\ding{250}}}%
    (-2,-4)(-2,-3)(-3,-3)(-2,-1)(0,0)(1,2)(1.5,3)(3,3)
\rmultiput[rot=90,ref=lC]{\blue\psscalebox{2}{\ding{253}}}%
    (-2,2.5)(-2,2.5)(-3,2.5)(-2,1)(1,-2)(1.5,-3)(3,-3)
\psgrid[subgriddiv=0,gridcolor=lightgray]
\end{pspicture}
\end{LTXexample}

\clearpage


%--------------------------------------------------------------------------------------
\section{\nxLcs{psVector}: Drawing relative vector lines}
%--------------------------------------------------------------------------------------

The new macros \Lcs{psStartPoint} and \Lcs{psVector} allow to draw a series of
vectors which start point refers to the endpoint of the last drawn vector. The 
coordinates of the endpoint are \emph{always} interpreted relative to the last
the vector. The first vector refers to the coordinates set by \Lcs{psStartPoint}.
With the boolean argument one can draw the horizontal angle of the vector.

The style of the angle arc is saved in \Lkeyval{psMarkAngleStyle} and the style
for the horizontal line in \Lkeyval{psMarkAngleLineStyle} and preset to

\begin{lstlisting}
\newpsstyle{psMarkAngleStyle}{arrows=->,arrowsize=4pt}
\newpsstyle{psMarkAngleLineStyle}{linestyle=dotted}
\end{lstlisting}


\begin{pspicture}[showgrid](10,10)
 \psStartPoint(1,1)
 \psVector(3;30)\psVector(4;60)\psVector[linecolor=red](3;10)
 \psVector[linestyle=dashed](4;110)
 \psStartPoint(1,1)\psset{markAngle}
 \psVector[linestyle=dashed](4;110)\psVector[linecolor=red](3;10)
 \psVector(4;60)\psVector(3;30)
\end{pspicture}

\begin{lstlisting}
\begin{pspicture}[showgrid](10,10)
 \psStartPoint(1,1)
 \psVector(3;30)\psVector(4;60)\psVector[linecolor=red](3;10)
 \psVector[linestyle=dashed](4;110)
 \psStartPoint(1,1)\psset{markAngle}
 \psVector[linestyle=dashed](4;110)\psVector[linecolor=red](3;10)
 \psVector(4;60)\psVector(3;30)
\end{pspicture}
\end{lstlisting}

All end points of the vectors are saved in node names with the preset name \verb=Vector#=,
where \# is the consecutive  number of the nodes. \verb=Vector0= ist the starting point of
the first \Lcs{psVector}. With the macro \Lcs{psStartPoint} one can set the starting point and
with optional argument the name of the nodes. \verb=Vector3= is the default node name of
the endpoint of the third vector or the name of the starting point of the forth vector.

\begin{BDef}
\Lcs{psStartPoint}\OptArg{node basename}\Largr{$x$,$y$}
\end{BDef}

\begin{pspicture}[showgrid,linewidth=1pt](10,10.4)
 \psStartPoint[A](1,1)% nodes have the base name A
 \psVector(3;30)\psVector(4;60)\psVector[linecolor=red](3;10)
 \psVector[linestyle=dashed](4;110)
 \psline{->}(A0)(A4)
 \psStartPoint[B](1,1)\psset{markAngle}% nodes have the base name B
 \psVector[linestyle=dashed](4;110)
 \psVector[linecolor=red](3;10)
 \psVector(4;60)\psVector(3;30)
 \psline[arrows=-D>,arrowscale=2,linewidth=1.5pt,linecolor=red](B2)(A2)
 \psline[arrows=-D>,arrowscale=2,linewidth=1.5pt,linecolor=blue](A3)(B3)
 \multido{\iA=0+1}{5}{\uput[0](A\iA){A\iA}\uput[180](B\iA){B\iA}}
\end{pspicture}

\begin{lstlisting}
\begin{pspicture}[showgrid,linewidth=1pt](10,10.4)
 \psStartPoint[A](1,1)% nodes have the base name A
 \psVector(3;30)\psVector(4;60)\psVector[linecolor=red](3;10)
 \psVector[linestyle=dashed](4;110)
 \psline{->}(A0)(A4)
 \psStartPoint[B](1,1)\psset{markAngle}% nodes have the base name B
 \psVector[linestyle=dashed](4;110)
 \psVector[linecolor=red](3;10)
 \psVector(4;60)\psVector(3;30)
 \psline[arrows=-D>,arrowscale=2,linewidth=1.5pt,linecolor=red](B2)(A2)
 \psline[arrows=-D>,arrowscale=2,linewidth=1.5pt,linecolor=blue](A3)(B3)
 \multido{\iA=0+1}{5}{\uput[0](A\iA){A\iA}\uput[180](B\iA){B\iA}
 \end{pspicture}
\end{lstlisting}

\clearpage


%--------------------------------------------------------------------------------------
\section{\nxLcs{psCircleTangents}: Calculating tangent lines of circles}
%--------------------------------------------------------------------------------------

The macro calculates the points on a circle where tangent lines from another
point or another circle are drawn.

\begin{BDef}
\Lcs{psCircleTangents}\Largr{$x1,y1$}\Largr{$x2,y2$}\Largb{Radius}\\
\Lcs{psCircleTangents}\Largr{$x1,y1$}\Largb{Radius}\Largr{$x2,y2$}\Largb{Radius}
\end{BDef}

In the first case the coordinates of a point and the center and the radius
of a circle must be given. The names of the calculates node names are \verb=CircleT1=
and \verb=CircleT2=.

\bigskip
\begin{pspicture}[showgrid](0,3)(10,10)
\psdot(2,4)\pscircle(7,7){2}
\psCircleTangents(2,4)(7,7){2}
\pcline[nodesep=-1cm,linecolor=blue](2,4)(CircleT1)
\pcline[nodesep=-1cm,linecolor=blue](2,4)(CircleT2)
\psdots(CircleT1)(CircleT2)
\uput[-80](CircleT1){T1}\uput[115](CircleT2){T2}
\end{pspicture}


\begin{lstlisting}
\begin{pspicture}[showgrid](0,3)(10,10)
\psdot(2,4)\pscircle(7,7){2}
\psCircleTangents(2,4)(7,7){2}
\pcline[nodesep=-1cm,linecolor=blue](2,4)(CircleT1)
\pcline[nodesep=-1cm,linecolor=blue](2,4)(CircleT2)
\psdots(CircleT1)(CircleT2)
\uput[-80](CircleT1){T1}\uput[115](CircleT2){T2}
\end{pspicture}
\end{lstlisting}

\bigskip
When using the other variant of the macro two circles must be given. The macro then defines
ten nodes, named \verb=CircleTC1= and \verb=CircleTC2= for the two intersection points,
 \verb=CircleTO1=, \verb=CircleTO2=, \verb=CircleTO3=, and \verb=CircleTO4= for the four
 nodes of the outer tangent lines and 
  \verb=CircleTI1=, \verb=CircleTI2=, \verb=CircleTI3=, and \verb=CircleTI4= for the
  four nodes of the inner tangent lines.

\bigskip
\begin{pspicture}[showgrid](-2,-2)(10,10)
\pscircle(1,1){1}\pscircle(7,7){3}
\psCircleTangents(1,1){1}(7,7){3}
\pcline[nodesep=-1cm,linecolor=blue](CircleTO1)(CircleTO2)
\pcline[nodesep=-1cm,linecolor=blue](CircleTO3)(CircleTO4)
\pcline[nodesep=-1cm,linecolor=red](CircleTI1)(CircleTI2)
\pcline[nodesep=-1cm,linecolor=red](CircleTI3)(CircleTI4)
\psdots(CircleTC1)(CircleTC2)%
  (CircleTO1)(CircleTO2)(CircleTO3)(CircleTO4)%
  (CircleTI1)(CircleTI2)(CircleTI3)(CircleTI4)%
\uput[0](CircleTC1){TC1}\uput[0](CircleTC2){TC2}
\uput[-80](CircleTI1){TI1}\uput[115](CircleTI2){TI2}
\uput[150](CircleTI3){TI3}\uput[-45](CircleTI4){TI4}
\uput[-80](CircleTO1){TO1}\uput[150](CircleTO2){TO2}
\uput[150](CircleTO3){TO3}\uput[-45](CircleTO4){TO4}
\end{pspicture}

\bigskip
\begin{lstlisting}
\begin{pspicture}[showgrid](-2,-2)(10,10)
\pscircle(1,1){1}\pscircle(7,7){3}
\psCircleTangents(1,1){1}(7,7){3}
\pcline[nodesep=-1cm,linecolor=blue](CircleTO1)(CircleTO2)
\pcline[nodesep=-1cm,linecolor=blue](CircleTO3)(CircleTO4)
\pcline[nodesep=-1cm,linecolor=red](CircleTI1)(CircleTI2)
\pcline[nodesep=-1cm,linecolor=red](CircleTI3)(CircleTI4)
\psdots(CircleTC1)\psdots(CircleTC2)%
  (CircleTO1)(CircleTO2)(CircleTO3)(CircleTO4)%
  (CircleTI1)(CircleTI2)(CircleTI3)(CircleTI4)%
\uput[0](CircleTC1){TC1}\uput[0](CircleTC2){TC2}
\uput[-80](CircleTI1){TI1}\uput[115](CircleTI2){TI2}
\uput[150](CircleTI3){TI3}\uput[-45](CircleTI4){TI4}
\uput[-80](CircleTO1){TO1}\uput[150](CircleTO2){TO2}
\uput[150](CircleTO3){TO3}\uput[-45](CircleTO4){TO4}
\end{pspicture}
\end{lstlisting}


\clearpage

%--------------------------------------------------------------------------------------
\section{\nxLcs{psEllipseTangents}: Calculating tangent lines of an ellipse}
%--------------------------------------------------------------------------------------

The macro calculates the two points on an ellipse where tangent lines from an outside  point
 are drawn.

\begin{BDef}
\Lcs{psEllipseTangents}\Largr{$x_0,y_0$}\Largr{$a,b$}\Largr{$x_p,y_p$}\\
\end{BDef}

The first two pairs of coordinates are the same as the ones for the default ellipse.
The names of the calculates node names are \verb=EllipseT1=
and \verb=EllipseT2=.

\bigskip
\begin{pspicture}[showgrid](0,3)(10,10)
\psdot(2,4)\psellipse(7,7)(3,1.5)
\psEllipseTangents(7,7)(3,1.5)(2,4)
\pcline[nodesep=-1cm,linecolor=blue](2,4)(EllipseT1)
\pcline[nodesep=-1cm,linecolor=blue](2,4)(EllipseT2)
\psdots(EllipseT1)(EllipseT2)
\uput[-80](EllipseT1){T1}\uput[115](EllipseT2){T2}
\end{pspicture}


\begin{lstlisting}
\begin{pspicture}[showgrid](0,3)(10,10)
\psdot(2,4)\psellipse(7,7)(3,1.5)
\psEllipseTangents(7,7)(3,1.5)(2,4)
\pcline[nodesep=-1cm,linecolor=blue](2,4)(EllipseT1)
\pcline[nodesep=-1cm,linecolor=blue](2,4)(EllipseT2)
\psdots(EllipseT1)(EllipseT2)
\uput[-80](EllipseT1){T1}\uput[115](EllipseT2){T2}
\end{pspicture}
\end{lstlisting}


\clearpage

%--------------------------------------------------------------------------------------
\section{\nxLcs{psrotate}: Rotating objects}
%--------------------------------------------------------------------------------------
\Lcs{rput} also has an optional argument for rotating objects, but
it always depends on the \Lcs{rput} coordinates. With
\Lcs{psrotate} the rotating center can be placed anywhere. The
rotation is done with \verb+\pscustom+, all optional arguments are
only valid if they are part of the \verb+\pscustom+ macro.

\begin{BDef}
\Lcs{psrotate}\OptArgs\Largr{$x,y$}\Largb{rot angle}\Largb{object}
\end{BDef}

\begin{LTXexample}[width=0.4\linewidth]
\psset{unit=0.75}
\begin{pspicture}(-0.5,-3.5)(8.5,4.5)
  \psaxes{->}(0,0)(-0.5,-3)(8.5,4.5)
  \psdots[linecolor=red,dotscale=1.5](2,1)
  \psarc[linecolor=red,linewidth=0.4pt,showpoints=true]
        {->}(2,1){3}{0}{60}
  \pspolygon[linecolor=green,linewidth=1pt](2,1)(5,1.1)(6,-1)(2,-2)
  \psrotate(2,1){60}{%
    \pspolygon[linecolor=blue,linewidth=1pt](2,1)(5,1.1)(6,-1)(2,-2)}
\end{pspicture}
\end{LTXexample}


\begin{LTXexample}[width=6cm]
\begin{pspicture}(-1,-1)(3,6)
\def\canne{%  Idea by Manuel Luque
  \psgrid[subgriddiv=0](-1,0)(1,5)
  \pscustom[linewidth=2mm]{\psline(0,4)\psarcn(0.3,4){0.3}{180}{360}}%
  \pscircle*(0.6,4){0.1}\pstriangle*(0,0)(0.2,-0.3)}
\def\Object{}
  \canne
  \psrotate(0.3,4){45}{\psset{linecolor=red!50}\canne}
  \psrotate(0.3,4){90}{\psset{linecolor=blue!50}\canne}
  \psrotate(0.3,4){360}{\psset{linecolor=cyan!50}\canne}
  \psdot[linecolor=red](0.3,4)
\end{pspicture}
\end{LTXexample}


\begin{LTXexample}[pos=t]
\begin{pspicture}(0,-6)(15,5)
\def\majorette{\psline[linewidth=0.5mm](0,2)%  Idea by Manuel Luque
               \pscircle[fillstyle=solid]{0.1}
               \pscircle[fillstyle=solid](0,2){0.1}}
  \psaxes[linewidth=0.5pt]{->}(0,0)(0,-5)(15,5)
  \pstVerb{/V0 10 def /Alpha 45 def}% vitesse initiale, angle de lancement
  \multido{\nT=0.0+0.05,\iA=0+40}{41}{%
    \pstVerb{/nT \nT\space def}%
    \rput(!V0 Alpha cos mul nT mul -9.81 2 div nT dup mul mul V0 Alpha sin mul nT mul add){%
       \psrotate(0,1){\iA}{\majorette\psdot[linecolor=red](0,1)\psdot[linecolor=green](0,2)}}}
  \parametricplot[linecolor=red]{0}{2}{% trajectoire du milieu
     V0 Alpha cos mul t mul -9.81 2 div t dup mul mul V0 Alpha sin mul t mul add 1 add}
  \parametricplot[linecolor=green,plotpoints=360]{0}{2}{% d'une extremite
     V0 Alpha cos mul t mul 800 t mul sin sub % x(t)
     -9.81 2 div t dup mul mul V0 Alpha sin mul t mul add 1 add 800 t mul cos add }%y(t)
\end{pspicture}
\end{LTXexample}


\clearpage

%--------------------------------------------------------------------------------------
\section{\nxLcs{psComment}: comments to a graphic}
%--------------------------------------------------------------------------------------

\begin{BDef}
\LcsStar{psComment}\OptArgs\OptArg*{\Largb{arrows}}\coord0\coord1\Largb{Text}\OptArg{line macro}\OptArg{put macro}
\end{BDef}

By default the macro uses the \Lcs{ncline} macro to draw a line from the first to the
second point, it can be changed with the first additional optional argument. The label is
put by default with \Lcs{rput}, which can be changed with the last optional argument.
If this is used, then the line macro has also be defined, eg \verb+\psComment(A)(B){text}[\ncarc][\ncput}+
At least, leave the argument empty.


\begin{LTXexample}[pos=t,wide]
\SpecialCoor\newpsstyle{weiss}{fillstyle=solid,fillcolor=white}
\footnotesize\psset{unit=0.5cm,dimen=middle}
\begin{pspicture}(-12,-4)(6,10)
\psframe*[linecolor=black!20](-5,-3)(5,7) \psframe*[linecolor=black!40](-5,3)(5,6)
\pscircle(-8.19,5.51){0.2}
\psframe[fillcolor=white,fillstyle=solid](-5.8,3.6)(4.3,5.8)
\psframe(-8.98,3.14)(-5.8,6.32)
\multido{\rA=-4.1+1.3}{5}{\rput(\rA,-2.4){\psframe[style=weiss](1.1,6)
  \psline(0,0)(1.1,0.5)(0,1)(1.1,1.6)(0,2.2)(1.1,2.7)(0,3.2)(1.1,3.2)}}
\pspolygon*(-4.1,3.7)(-4.1,3)(-3,3)(-3.01,3.7)(-3.54,4.19)
\pspolygon*(1.09,3.7)(1.1,3)(2.2,3)(2.18,3.7)(1.65,4.24)
\pspolygon*(-2.78,3.7)(-2.8,3)(-1.7,3)(-1.71,3.7)(-2.27,4.04)
\pspolygon*(-1.51,3.7)(-1.5,3)(-0.4,3)(-0.41,3.7)(-1.02,4.17)
\pspolygon*(-0.21,3.7)(-0.2,3)(0.9,3)(0.89,3.7)(0.3,4.04)
\psline(-5,3.83)(-4.15,3.86)(-3.5,4.3)(-2.85,3.81)(-2.22,4.21)(-1.6,3.86)(-0.99,4.33)
       (-0.28,3.83)(0.35,4.19)(0.97,3.83)(1.65,4.39)(2.2,4.01)(3.57,4.89)(2.41,5.8)
  \psline(-5,5.8)(-5.78,5.8)  \psline(-5.78,5.47)(2.85,5.47)
  \psline(-5.8,3.52)(-5,3.5)  \psline(3.57,4.89)(-5.8,4.89)
  \psComment*[ref=r]{->}(-8.14,1.19)(-4.31,3.27){Mantelstift}
  \psComment*[ref=r]{->}(-8.17,-0.56)(-4.37,1.59){Kernstift}[\ncarc]
  \psComment*[ref=r]{->}(-7.91,-2.24)(-4.44,-0.23){Feder}[\ncarc]
  \psComment[npos=-0.1]{->}(-3.48,8.72)(-1.33,5.46){Nur f\"ur Profil}
\end{pspicture}
\end{LTXexample}

\clearpage
%--------------------------------------------------------------------------------------
\section{\nxLcs{psChart}: a pie chart}
%--------------------------------------------------------------------------------------

\begin{BDef}
\Lcs{psChart}\OptArgs\Largb{comma separated value list}\Largb{comma separated value list}\Largb{radius}
\end{BDef}

The special optional arguments for the \Lcs{psChart} macro are as follows:

\noindent
\begin{tabularx}{\linewidth}{@{}>{\ttfamily}lX>{\ttfamily}l@{}}
\textrm{\emph{name}} & \textrm{\emph{description}} & \textrm{\emph{default}}\\\hline
\Lkeyword{chartSep}  & distance from the pie chart center to an outraged pie piece & 10pt\\
\Lkeyword{chartColor} & gray or colored pie (values are: \texttt{gray} or \texttt{color})& gray\\
\Lkeyword{userColor} & a comma separated list of user defined colors for the pie & \{\}\\
\Lkeyword{chartNodeI}& the position of the inner node, relative to the radius & 0.75\\
\Lkeyword{chartNodeO}& the position of the outer node, relative to the radius & 1.5
\end{tabularx}

\bigskip
The first mandatory argument is the list of the values and may not be empty. The second
one is a list of outraged pieces, numbered consecutively from 1 to up the total number
of values. The list of user defined colors must be enclosed in braces!

The macro \Lcs{psChart} defines for every value three nodes at the half angle and
in distances from 0.75, 1, and 1.25 times of the radius from the origin. The nodes
are named as \verb+psChartI?+, \verb+psChart?+, and \verb+psChartO?+, where ? is the number of
the pie. The letter I leads to the inner node and the letter O to the outer node.
The distance can be changed with the optional arguments \Lkeyword{chartNodeI} and
\Lkeyword{chartNodeO} in the usual way with \verb+\psset{chartNodeI=...,chartNodeO=...}+.

The other one is the node on the circle line.
The origin is by default \texttt{(0,0)}. Moving the pie to another position can be done as
usual with the \Lcs{rput}-macro. The used colors are named internally as \Lkeyword{chartFillColor?}
and can be used by the user for coloring lines or text.

\begin{LTXexample}[width=6cm]
\begin{pspicture}(-3,-3)(3,3)
\psChart{ 23, 29, 3, 26, 28, 14 }{}{2}
\multido{\iA=1+1}{6}{%
  \psdot(psChart\iA)\psdot(psChartI\iA)\psdot(psChartO\iA)%
  \psline[linestyle=dashed,linecolor=white](psChart\iA)
  \psline[linestyle=dashed](psChart\iA)(psChartO\iA)}
\end{pspicture}
\end{LTXexample}

\begin{LTXexample}[width=6cm]
\begin{pspicture}(-3,-3)(3,3)
\psChart[chartColor=color]{45,90}{1}{2}
\ncline[linecolor=-chartFillColor1,
  nodesepB=-20pt]{psChartO1}{psChart1}
\rput[l](psChartO1){%
  \textcolor{chartFillColor1}{pie no 1}}
\ncline[linecolor=-chartFillColor2,
  nodesepB=-20pt]{psChartO2}{psChart2}
\rput[lt](psChartO2){%
  \textcolor{chartFillColor2}{pie no 2}}
\end{pspicture}
\end{LTXexample}

\begin{LTXexample}[width=7.5cm]
\psframebox[fillcolor=black!20,
  fillstyle=solid]{%
\begin{pspicture}(-3.5,-3.5)(4.25,3.5)
\psChart[chartColor=color]%
  {23, 29, 3, 26, 28, 14, 17, 4, 9}{}{2}
\multido{\iA=1+1}{9}{%
  \ncline[linecolor=-chartFillColor\iA,
    nodesepB=-10pt]{psChartO\iA}{psChart\iA}
  \rput[l](psChartO\iA){%
    \textcolor{chartFillColor\iA}{pie no \iA}}}
\end{pspicture}}
\end{LTXexample}

\begin{LTXexample}[width=6cm]
\begin{pspicture}(-3,-3)(3,3)
\psChart[userColor={red!30,green!30,
    blue!40,gray,magenta!60,cyan}]%
      { 23, 29, 3, 26, 28, 14 }{1,4}{2}
\end{pspicture}
\end{LTXexample}

\begin{LTXexample}[width=6cm]
\begin{pspicture}(-3,-2.5)(3,2.5)
\psChart{ 23, 29, 3, 26, 28, 14 }{}{2}
\multido{\iA=1+1}{6}{\rput*(psChartI\iA){\iA}}
\end{pspicture}
\end{LTXexample}


%\begin{LTXexample}[pos=t]
\psset{unit=1.5}
\begin{pspicture}(-3,-3)(3,3)
\psChart[userColor={red!30,green!30,blue!40,gray,cyan!50,
    magenta!60,cyan},chartSep=30pt,shadow=true,shadowsize=5pt]{34.5,17.2,20.7,15.5,5.2,6.9}{6}{2}
\psset{nodesepA=5pt,nodesepB=-10pt}
\ncline{psChartO1}{psChart1}\nput{0}{psChartO1}{1000 (34.5\%)}
\ncline{psChartO2}{psChart2}\nput{150}{psChartO2}{500 (17.2\%)}
\ncline{psChartO3}{psChart3}\nput{-90}{psChartO3}{600 (20.7\%)}
\ncline{psChartO4}{psChart4}\nput{0}{psChartO4}{450 (15.5\%)}
\ncline{psChartO5}{psChart5}\nput{0}{psChartO5}{150 (5.2\%)}
\ncline{psChartO6}{psChart6}\nput{0}{psChartO6}{200 (6.9\%)}
\bfseries%
\rput(psChartI1){Taxes}\rput(psChartI2){Rent}\rput(psChartI3){Bills}
\rput(psChartI4){Car}\rput(psChartI5){Gas}\rput(psChartI6){Food}
\end{pspicture}
%\end{LTXexample}
\psset{unit=1cm}

\begin{lstlisting}
\psset{unit=1.5}
\begin{pspicture}(-3,-3)(3,3)
\psChart[userColor={red!30,green!30,blue!40,gray,cyan!50,
    magenta!60,cyan},chartSep=30pt,shadow=true,shadowsize=5pt]{34.5,17.2,20.7,15.5,5.2,6.9}{6}{2}
\psset{nodesepA=5pt,nodesepB=-10pt}
\ncline{psChartO1}{psChart1}\nput{0}{psChartO1}{1000 (34.5\%)}
\ncline{psChartO2}{psChart2}\nput{150}{psChartO2}{500 (17.2\%)}
\ncline{psChartO3}{psChart3}\nput{-90}{psChartO3}{600 (20.7\%)}
\ncline{psChartO4}{psChart4}\nput{0}{psChartO4}{450 (15.5\%)}
\ncline{psChartO5}{psChart5}\nput{0}{psChartO5}{150 (5.2\%)}
\ncline{psChartO6}{psChart6}\nput{0}{psChartO6}{200 (6.9\%)}
\bfseries%
\rput(psChartI1){Taxes}\rput(psChartI2){Rent}\rput(psChartI3){Bills}
\rput(psChartI4){Car}\rput(psChartI5){Gas}\rput(psChartI6){Food}
\end{pspicture}
\end{lstlisting}



\clearpage
%--------------------------------------------------------------------------------------
\section{\nxLcs{psHomothetie}: central dilatation}
%--------------------------------------------------------------------------------------

\begin{BDef}
\Lcs{psHomothetie}\OptArgs\Largr{center}\Largb{factor}\Largb{object}
\end{BDef}

\begin{LTXexample}[width=9cm]
\begin{pspicture}[showgrid=true](-5,-4)(4,8)
\psBill% needs package pst-fun
\psHomothetie[linecolor=blue](4,-3){2}{\psBill}
\psdots[dotsize=3pt,linecolor=red](4,-3)
\psplot[linestyle=dashed,linecolor=red]{-5}{4}%
  [ /m -3 -0.85 sub 4 0.6 sub div def ]
  { m x mul m 4 mul sub 3 sub }%
\psHomothetie[linecolor=green](4,-3){-0.2}{\psBill}
\end{pspicture}
\end{LTXexample}

%\pstVerb{ /m -3 -0.85 sub 4 0.6 sub div def }


\clearpage

%--------------------------------------------------------------------------------------
\section{\nxLcs{psbrace}}
%--------------------------------------------------------------------------------------
\begin{BDef}
\LcsStar{psbrace}\OptArgs\Largr{A}\Largr{B}\Largb{text}
\end{BDef}

Additional to all other available options from \LPack{pstricks} or the other
related packages,  there are two new option, named  \Lkeyword{braceWidth} and
\Lkeyword{bracePos}. All important ones are shown in the following graphics
and table.

\begin{center}
\begin{pspicture}[showgrid=true](10,5)
  \psbrace[braceWidth=1cm,braceWidthInner=1cm,
    braceWidthOuter=1cm,bracePos=0.6,fillcolor=white,
    nodesepA=10mm,nodesepB=10mm](0,5)(10,5){\fbox{Label}}
\pcline{<->}(3,3)(3,4)\ncput*{\footnotesize\ttfamily braceWidth}
\pcline{<->}(3,4)(3,5)\ncput*{\footnotesize\ttfamily braceWidthInner}
\pcline{<->}(3,2)(3,3)\ncput*{\footnotesize\ttfamily braceWidthOuter}
\pcline{<->}(6,1)(6,2)\ncput{\footnotesize\ttfamily nodesepB}
\pcline{<->}(6,1)(7,1)\ncput*{\footnotesize\ttfamily A}
\pcline{<->}(0,0.5)(6,0.5)\ncput*{\footnotesize\ttfamily bracePos}
\psdot[dotscale=2](0,5)\uput[0](0,5){\textbf{A}}
\psdot[dotscale=2](10,5)\uput[180](10,5){\textbf{B}}
\end{pspicture}
\end{center}

A positive value for \Lkeyword{nodesepA} and \Lkeyword{nodesepB} shifts the label to the upper right
and a negative value to the lower left. This does not depends on
the value for the rotating of the label!

\begin{center}
\begin{tabular}{@{}l|l@{}}
name & meaning\\\hline
\Lkeyword{braceWidth} & default is \Lcs{pslinewidth}\\
\Lkeyword{braceWidthInner} & default is \verb+10\pslinewidth+\\
\Lkeyword{braceWidthOuter} & default is \verb+10\pslinewidth+\\
\Lkeyword{bracePos} & relative position (default is $0.5$)\\
\Lkeyword{nodesepA} & x-separation (default is $0pt$)\\
\Lkeyword{nodesepB} & y-separation (default is $0pt$)\\
\Lkeyword{rot} & additional rotating for the text (default is $0$)\\
\Lkeyword{ref} & reference point for the text (default is c)\\
\Lkeyword{fillcolor} & default is black
\end{tabular}
\end{center}

By default the text is written perpendicular to the brace line and
can be changed with the \LPack{pstricks} option \Lkeyword{rot}=\ldots\ The
text parameter can take any object and may also be empty. The
reference point can be any value of the combination of \Lkeyval{l}
(left) or \Lkeyval{r} (right) and \Lkeyval{b} (bottom) or \Lkeyval{B}
(Baseline) or \Lkeyval{C} (center) or \Lkeyval{t} (top), where the
default is \Lkeyval{c}, the center of the object.



\begin{LTXexample}[width=4.5cm]
\begin{pspicture}(4,4)
\psgrid[subgriddiv=0,griddots=10]
\pnode(0,0){A}
\pnode(4,4){B}
\psbrace[linecolor=red,ref=lC](A)(B){Text I}
\psbrace*[linecolor=blue,ref=lC](3,4)(0,1){Text II}
\psbrace[fillcolor=white](3,0)(3,4){III}
\end{pspicture}
\end{LTXexample}

\bigskip
The option \Lcs{specialCoor} is enabled, so that all types of coordinates
are possible, (nodename), ($x,y$), ($nodeA|nodeB$), \ldots
The star version fills the inner of the \Index{brace} with the current linecolor.
With the fillcolor \verb+white+ or any other background color the brace can
be "`unfilled"'.

\begin{LTXexample}
\begin{pspicture}(8,2.5)
\psbrace(0,0)(0,2){\fbox{Text}}%
\psbrace[nodesepA=10pt](2,0)(2,2){\fbox{Text}}
\psbrace[ref=lC](4,0)(4,2){\fbox{Text}}
\psbrace[ref=lt,rot=90,nodesepB=-15pt](6,0)(6,2){\fbox{Text}}
\psbrace[ref=lt,rot=90,nodesepA=-5pt,nodesepB=15pt](8,2)(8,0){\fbox{Text}}
\end{pspicture}
\end{LTXexample}


\begin{LTXexample}
\def\someMath{$\int\limits_1^{\infty}\frac{1}{x^2}\,dx=1$}
\begin{pspicture}(8,2.5)
\psbrace[ref=lC](0,0)(0,2){\someMath}%
\psbrace[rot=90](2,0)(2,2){\someMath}
\psbrace[ref=lC](4,0)(4,2){\someMath}
\psbrace[ref=lt,rot=90,nodesepB=-30pt](6,0)(6,2){\someMath}
\psbrace[ref=lt,rot=90,nodesepB=30pt](8,2)(8,0){\someMath}
\end{pspicture}
\end{LTXexample}

%$

\begin{LTXexample}
\begin{pspicture}(\linewidth,5)
\psbrace(0,0.5)(\linewidth,0.5){\fbox{Text}}%
\psbrace[bracePos=0.25,nodesepB=10pt,rot=90](0,2)(\linewidth,2){\fbox{Text}}
\psbrace[ref=lC,nodesepA=-3.5cm,nodesepB=15pt,rot=90](0,4)(\linewidth,4){%
   \fbox{some very, very long wonderful Text}}
\end{pspicture}
\end{LTXexample}


\begin{LTXexample}[width=8cm]
\psset{unit=0.8}
\begin{pspicture}(10,11)
\psgrid[subgriddiv=0,griddots=10]
\pnode(0,0){A}
\pnode(4,6){B}
\psbrace[ref=lC](A)(B){One}
\psbrace[rot=180,nodesepA=-5pt,ref=rb](B)(A){Two}
\psbrace[linecolor=blue,bracePos=0.25,ref=lB](8,1)(1,7){Three}
\psbrace[braceWidth=-1mm,rot=180,ref=rB](8,1)(1,7){Four}
\psbrace*[linearc=0.5,fillstyle=none,linewidth=1pt,braceWidth=1.5pt,
  bracePos=0.25,ref=lC](8,1)(8,9){A}
\psbrace(4,9)(6,9){}
\psbrace(6,9)(6,7){}
\psbrace(6,7)(4,7){}
\psbrace(4,7)(4,9){}
\psset{linecolor=red}
\psbrace*[ref=lb](7,10)(3,10){I}
\psbrace*[ref=lb,bracePos=0.75](3,10)(3,6){II}
\psbrace*[ref=lb](3,6)(7,6){III}
\psbrace*[ref=lb](7,6)(7,10){IV}
\end{pspicture}
\end{LTXexample}

%$

\begin{LTXexample}[width=5cm]
\[
\begin{pmatrix}
    \Rnode[vref=2ex]{A}{~1} \\
    & \ddots \\
    && \Rnode[href=2]{B}{1} \\
    &&& \Rnode[vref=2ex]{C}{0} \\
    &&&& \ddots \\
    &&&&& \Rnode[href=2]{D}{0}~ \\
\end{pmatrix}
\]
\psbrace[rot=-90,nodesepB=-0.5,nodesepA=-0.2](B)(A){\small n times}
\psbrace[rot=-90,nodesepB=-0.5,nodesepA=-0.2](D)(C){\small n times}
\end{LTXexample}


\clearpage
It is also possible to put a vertical brace around a
default paragraph. This works by setting two invisible nodes at
the beginning and the end of the paragraph. Indentation is
possible with a minipage.

\small
Some nonsense text, which is nothing more than nonsense.
Some nonsense text, which is nothing more than nonsense.

\noindent\rnode{A}{}

\vspace*{-1ex}
Some nonsense text, which is nothing more than nonsense.
Some nonsense text, which is nothing more than nonsense.
Some nonsense text, which is nothing more than nonsense.
Some nonsense text, which is nothing more than nonsense.
Some nonsense text, which is nothing more than nonsense.
Some nonsense text, which is nothing more than nonsense.
Some nonsense text, which is nothing more than nonsense.
Some nonsense text, which is nothing more than nonsense.

\vspace*{-2ex}\noindent\rnode{B}{}\psbrace*[linecolor=red](A)(B){}

Some nonsense text, which is nothing more than nonsense.
Some nonsense text, which is nothing more than nonsense.

\medskip\hfill\begin{minipage}{0.95\linewidth}
\noindent\rnode{A}{}

\vspace*{-1ex}
Some nonsense text, which is nothing more than nonsense.
Some nonsense text, which is nothing more than nonsense.
Some nonsense text, which is nothing more than nonsense.
Some nonsense text, which is nothing more than nonsense.
Some nonsense text, which is nothing more than nonsense.
Some nonsense text, which is nothing more than nonsense.
Some nonsense text, which is nothing more than nonsense.
Some nonsense text, which is nothing more than nonsense.

\vspace*{-2ex}
\noindent\rnode{B}{}\psbrace[linecolor=red](A)(B){}
\end{minipage}

\normalsize

\begin{lstlisting}
Some nonsense text, which is nothing more than nonsense.
Some nonsense text, which is nothing more than nonsense.

\noindent\rnode{A}{}

\vspace*{-1ex}
Some nonsense text, which is nothing more than nonsense.
Some nonsense text, which is nothing more than nonsense.
Some nonsense text, which is nothing more than nonsense.
Some nonsense text, which is nothing more than nonsense.
Some nonsense text, which is nothing more than nonsense.
Some nonsense text, which is nothing more than nonsense.
Some nonsense text, which is nothing more than nonsense.
Some nonsense text, which is nothing more than nonsense.

\vspace*{-2ex}\noindent\rnode{B}{}\psbrace[linecolor=red](A)(B){}

Some nonsense text, which is nothing more than nonsense.
Some nonsense text, which is nothing more than nonsense.

\medskip\hfill\begin{minipage}{0.95\linewidth}
\noindent\rnode{A}{}

\vspace*{-1ex}
Some nonsense text, which is nothing more than nonsense.
Some nonsense text, which is nothing more than nonsense.
Some nonsense text, which is nothing more than nonsense.
Some nonsense text, which is nothing more than nonsense.
Some nonsense text, which is nothing more than nonsense.
Some nonsense text, which is nothing more than nonsense.
Some nonsense text, which is nothing more than nonsense.
Some nonsense text, which is nothing more than nonsense.

\vspace*{-2ex}\noindent\rnode{B}{}\psbrace[linecolor=red](A)(B){}
\end{minipage}
\end{lstlisting}

\clearpage


%--------------------------------------------------------------------------------------
\section{Random dots}
%--------------------------------------------------------------------------------------
The syntax of the new macro \Lcs{psRandom} is:

\begin{BDef}
\Lcs{psRandom}\OptArgs\Largb{}\\
\Lcs{psRandom}\OptArgs\OptArg*{\Largr{$x_{Min},y_{Min}$}}\OptArg*{\Largr{$x_{Max},y_{Max}$}}\Largb{clip path} %$
%\psRandom[<option>](<xMax,yMax>){<clip path>}
%\psRandom[<option>](<xMin,yMin>)(<xMax,yMax>){<clip path>}
\end{BDef}

If there is no area for the dots defined, then \verb+(0,0)(1,1)+ in the current
scale setting is used for placing the dots. If there is only one \Largr{$x_{Max},y_{Max}$} %$
defined, then \verb+(0,0)+ is used for the other point.
This area should be greater than the clipping
path to be sure that the dots are placed over the full area. The clipping path can
be everything. If no clipping path is given, then the frame \verb+(0,0)(1,1)+
in user coordinates is used.  The new options are:

\begin{center}
\begin{tabular}{@{}l|l|l@{}}
name & default\\\hline
\Lkeyword{randomPoints} &   \verb|1000| & number of random dots\tabularnewline
\Lkeyword{color} & \false & random color\tabularnewline
\end{tabular}
\end{center}


\begin{LTXexample}[width=0.3\linewidth]
\psset{unit=5cm}
\begin{pspicture}(1,1)
  \psRandom[dotsize=1pt,fillstyle=solid](1,1){\pscircle(0.5,0.5){0.5}}
\end{pspicture}
\begin{pspicture}(1,1)
  \psRandom[dotsize=2pt,randomPoints=5000,color,%
      fillstyle=solid](1,1){\pscircle(0.5,0.5){0.5}}
\end{pspicture}
\end{LTXexample}

\begin{LTXexample}[width=0.4\linewidth]
\psset{unit=5cm}
\begin{pspicture}(1,1)
  \psRandom[randomPoints=200,dotsize=8pt,dotstyle=+]{}
\end{pspicture}
\begin{pspicture}(1.5,1)
  \psRandom[dotsize=5pt,color](0,0)(1.5,0.8){\psellipse(0.75,0.4)(0.75,0.4)}
\end{pspicture}
\end{LTXexample}

\begin{LTXexample}
\psset{unit=2.5cm}
\begin{pspicture}(0,-1)(3,1)
  \psRandom[dotsize=4pt,dotstyle=o,linecolor=blue,fillcolor=red,%
     fillstyle=solid,randomPoints=1000]%
      (0,-1)(3,1){\psplot{0}{3.14}{ x 114 mul sin }}
\end{pspicture}
\end{LTXexample}

\psset{unit=1cm}


\clearpage
 %--------------------------------------------------------------------------------------
\section{\nxLcs{psDice}}
 %--------------------------------------------------------------------------------------
\Lcs{psdice} creates the view of a dice. The number on the dice is the only parameter.
The optional parameters, like the color can be used as usual. The macro is a box of
dimension zero and is placed
at the current point. Use  the \Lcs{rput} macro to place it anywhere. The optional
argument \Lkeyword{unit} can be used to scale the dice. the default size of
the dice $1\mathrm{cm}\times1\mathrm{cm}$.

\begin{center}
\begin{pspicture}(-1,-1)(8,9)
\multido{\iA=1+1}{6}{%
  \rput(\iA,7.5){\Huge\psdice[unit=0.75,linecolor=red!80]{\iA}}
  \rput(! -0.5 7 \iA\space sub){\Huge\psdice[unit=0.75,linecolor=blue!70]{\iA}}%
  \multido{\iB=1+1}{6}{%
    \rput(! \iA\space 7 \iB\space sub){%
      \rnode[c]{p\iA\iB}{\makebox[1em][l]{\strut\psPrintValue[fontscale=12]{\iA\space \iB\space add}}}%
}}}
\ncbox[linearc=0.35,nodesep=0.2,linestyle=dotted]{p11}{p66}
\ncbox[linearc=0.35,nodesep=0.2,linestyle=dashed]{p15}{p51}
\rput{90}(-1.5,3.5){1. dice}
\rput{0}(3.5,8.5){2. dice}
\psline[linewidth=1.5pt](0.25,0.5)(0.25,8)
\psline[linewidth=1.5pt](-1,6.75)(6.5,6.75)
\end{pspicture}
\end{center}

\begin{lstlisting}
\begin{pspicture}(-1,-1)(8,8)
\multido{\iA=1+1}{6}{%
  \rput(\iA,7.5){\Huge\psdice[unit=0.75,linecolor=red!80]{\iA}}
  \rput(! -0.5 7 \iA\space sub){\Huge\psdice[unit=0.75,linecolor=blue!70]{\iA}}%
  \multido{\iB=1+1}{6}{%
    \rput(! \iA\space 7 \iB\space sub){%
      \rnode[c]{p\iA\iB}{\makebox[1em][l]{\strut\psPrintValue[fontscale=12]{\iA\space \iB\space add}}}%
}}}
\ncbox[linearc=0.35,nodesep=0.2,linestyle=dotted]{p11}{p66}
\ncbox[linearc=0.35,nodesep=0.2,linestyle=dashed]{p15}{p51}
\rput{90}(-1.5,3.5){1. dice}
\rput{0}(3.5,8.5){2. dice}
\psline[linewidth=1.5pt](0.25,0.5)(0.25,8)
\psline[linewidth=1.5pt](-1,6.75)(6.5,6.75)
\end{pspicture}
\end{lstlisting}

\clearpage
%--------------------------------------------------------------------------------------
\section{\nxLcs{psFormatInt}}
%--------------------------------------------------------------------------------------
There exist some packages and a lot of code to format an integer like $1\,000\,000$
or $1,234,567$ (in Europe $1.234.567$). But all packages expect a real number as
argument and cannot handle macros as an argument. For this case \LPack{pstricks-add}
has a macro \Lcs{psFormatInt} which can handle both:

\begin{LTXexample}[width=3cm]
\psFormatInt{1234567}\\
\psFormatInt[intSeparator={,}]{1234567}\\
\psFormatInt[intSeparator=.]{1234567}\\
\psFormatInt[intSeparator=$\cdot$]{1234567}\\
\def\temp{965432}
\psFormatInt{\temp}
\end{LTXexample}

With the option \Lkeyword{intSeparator} the symbol can be changed to any any non-number character.


\clearpage

%--------------------------------------------------------------------------------------
\section{\nxLcs{psRelNode} and \nxLcs{psDefPSPNodes}}
%--------------------------------------------------------------------------------------
With these macros it is possible to put a node relative to a given line or given
\Lenv{pspicture}-environment. In the frist case the parameters are
the angle and the length factor:

\begin{BDef}
\Lcs{psRelNode}\Largs{P0}\Largs{P1}\Largb{length factor}\Largb{end node name}\\
\Lcs{psDefPSPNodes}
\end{BDef}

The length factor relates to the distance $\overline{P_0P_1}$ and
the end node name must be a valid nodename and shouldn't contain
any of the special PostScript characters. There are two valid
options:

\begin{tabularx}{\linewidth}{@{} l|l| X @{} }
name & default & meaning\\\hline 
\Lkeyword{angle} & $0$ & angle between the given line $\overline{P_0P_1}$ and the new one
	$\overline{P_0P_{endNode}}$\tabularnewline 
\Lkeyword{trueAngle} & \false & defines whether the angle refers to the seen line or to
the mathematical one, which respect the scaling factors
\Lkeyword{xunit} and \Lkeyword{yunit}.
\end{tabularx}

\begin{LTXexample}[width=7cm]
\begin{pspicture}[showgrid](7,6)
  \pnode(3,3){A}\pnode(4,2){B}
  \psline[nodesep=-3,linewidth=0.5pt](A)(B)
  \multido{\iA=0+30}{12}{%
    \psRelNode[angle=\iA](A)(B){2}{C}%
    \qdisk(C){2pt}
    \uput[0](C){\iA}}
\end{pspicture}
\end{LTXexample}

In the second case the new macro \Lcs{psDefPSPNodes} defines nine nodes that corresponds to
nine particular points (namely bottom left, bottom center,
bottom right, center left, center center, center right, top left,
top center, top right) of the \Lenv{pspicture} box.

\begin{LTXexample}[width=6cm,wide=false]
\begin{pspicture}[showgrid=true](-1,-1)(4,4)
  \psDefPSPNodes
  \psdots(PSPbl)(PSPbc)(PSPbr)
      (PSPcl)(PSPcc)(PSPcr)(PSPtl)(PSPtc)(PSPtr)
  \uput[90](PSPbl){PSPbl} \uput[90](PSPbc){PSPbc}
  \uput[90](PSPbr){PSPbr} \uput[90](PSPcl){PSPcl}
  \uput[90](PSPcc){PSPcc} \uput[90](PSPcr){PSPcr}
  \uput[90](PSPtl){PSPtl} \uput[90](PSPtc){PSPtc}
  \uput[90](PSPtr){PSPtr}
\end{pspicture}
\end{LTXexample}

The name of the nodes are predefined as:

\begin{lstlisting}[style=syntax]
\psset[pst-PSPNodes]{blName=PSPbl,bcName=PSPbc,brName=PSPbr,
  clName=PSPcl,ccName=PSPcc,crName=PSPcr,tlName=PSPtl,tcName=PSPtc,trName=PSPtr}
\end{lstlisting}

and can be modified in the same way.
%I guess you modified the family to have the pstricks-add one so the
%\xkvview would have to be adapted.

%--------------------------------------------------------------------------------------
\section{\nxLcs{psRelLine}}
%--------------------------------------------------------------------------------------
With this macro it is possible to plot lines relative to a given one. Parameter are
the angle and the length factor:

\begin{BDef}
\Lcs{psRelLine}\Largr{P0}\Largr{P1}\Largb{length factor}\Largb{<end node name>}\\
\Lcs{psRelLine}\OptArg{\Largb{arrows}}\Largr{P0}\Largr{P1}\Largb{length factor}\Largb{end node name}\\
\Lcs{psRelLine}\OptArgs\Largr{P0}\Largr{P1}\Largb{length factor}\Largb{end node name}\\
\Lcs{psRelLine}\OptArgs\OptArg{\Largb{arrows}}\Largr{P0}\Largr{P1}\Largb{length factor}\Largb{end node name}
\end{BDef}

The length factor relates to the distance $\overline{P_0P_1}$ and
the end node name must be a valid nodename and shouldn't contain
any of the special PostScript characters. There are two valid
options which are described in the foregoing section for
\Lcs{psRelNode}.

The following two figures show the same, the first one with a scaling different to $1:1$,
this is the reason why the end points are on an ellipse and not on a circle like in the
second figure.

\begin{LTXexample}[width=5cm]
\psset{yunit=2,xunit=1}
\begin{pspicture}(-2,-2)(3,2)
\psgrid[subgriddiv=2,subgriddots=10,gridcolor=lightgray]
\pnode(-1,0){A}\pnode(3,2){B}
\psline[linecolor=red](A)(B)
\psRelLine[linecolor=blue,angle=30](-1,0)(B){0.5}{EndNode}
\qdisk(EndNode){2pt}
\psRelLine[linecolor=blue,angle=-30](A)(B){0.5}{EndNode}
\qdisk(EndNode){2pt}
\psRelLine[linecolor=magenta,angle=90](-1,0)(3,2){0.5}{EndNode}
\qdisk(EndNode){2pt}
\psRelLine[linecolor=magenta,angle=-90](A)(B){0.5}{EndNode}
\qdisk(EndNode){2pt}
\end{pspicture}
\end{LTXexample}

\begin{LTXexample}[width=5cm]
\begin{pspicture}(-2,-2)(3,2)
\psgrid[subgriddiv=2,subgriddots=10,gridcolor=lightgray]
\pnode(-1,0){A}\pnode(3,2){B}
\psline[linecolor=red](A)(B)
\psarc[linestyle=dashed](A){2.23}{-90}{135}
\psRelLine[linecolor=blue,angle=30](-1,0)(B){0.5}{EndNode}
\qdisk(EndNode){2pt}
\psRelLine[linecolor=blue,angle=-30](A)(B){0.5}{EndNode}
\qdisk(EndNode){2pt}
\psRelLine[linecolor=magenta,angle=90](-1,0)(3,2){0.5}{EndNode}
\qdisk(EndNode){2pt}
\psRelLine[linecolor=magenta,angle=-90](A)(B){0.5}{EndNode}
\qdisk(EndNode){2pt}
\end{pspicture}
\end{LTXexample}

\medskip
The following figure has also a different scaling, but has set the
option \Lkeyword{trueAngle}, all angles refer to "what you see".

\begin{LTXexample}[width=6.5cm]
\psset{yunit=2,xunit=1}
\begin{pspicture}(-3,-1)(3,2)\psgrid[subgridcolor=lightgray]
\pnode(-1,0){A}\pnode(3,2){B}
\psline[linecolor=red](A)(B)
\psarc(A){2.83}{-45}{135}
\psRelLine[linecolor=blue,angle=30,trueAngle](A)(B){0.5}{EndNode}
\qdisk(EndNode){2pt}
\psRelLine[linecolor=blue,angle=-30,trueAngle](A)(B){0.5}{EndNode}
\qdisk(EndNode){2pt}
\psRelLine[linecolor=magenta,angle=90,trueAngle](A)(B){0.5}{EndNode}
\qdisk(EndNode){2pt}
\psRelLine[linecolor=magenta,angle=-90,trueAngle](A)(B){0.5}{EndNode}
\qdisk(EndNode){2pt}
\end{pspicture}
\end{LTXexample}

\medskip
Two examples using \verb+\multido+ to show the behaviour of the
options \verb+trueAngle+ and \verb+angle+.

\medskip
\begin{LTXexample}[width=8cm]
\psset{yunit=4,xunit=2}
\begin{pspicture}(-1,0)(3,2)\psgrid[subgridcolor=lightgray]
\pnode(-1,0){A}\pnode(1,1){B}
\psline[linecolor=red](A)(3,2)
\multido{\iA=0+10}{36}{%
  \psRelLine[linecolor=blue,angle=\iA](B)(A){-0.5}{EndNode}
  \qdisk(EndNode){2pt}
}
\end{pspicture}
\end{LTXexample}

\begin{LTXexample}[width=8cm]
\psset{yunit=4,xunit=2}
\begin{pspicture}(-1,0)(3,2)\psgrid[subgridcolor=lightgray]
\pnode(-1,0){A}\pnode(1,1){B}
\psline[linecolor=red](A)(3,2)
\multido{\iA=0+10}{36}{%
  \psRelLine[linecolor=magenta,angle=\iA,trueAngle]{->}(B)(A){-0.5}{EndNode}
}
\end{pspicture}
\end{LTXexample}

\begin{center}
\bgroup
\psset{xunit=0.75\linewidth,yunit=0.75\linewidth,trueAngle}%
\begin{pspicture}(1,0.6)%\psgrid
  \pnode(.3,.35){Vk} \pnode(.375,.35){D} \pnode(0,.4){DST1} \pnode(1,.18){DST2}
  \pnode(0,.1){A1}   \pnode(1,.31){A1}
  { \psset{linewidth=.02,linestyle=dashed,linecolor=gray}%
    \pcline(DST1)(DST2) % <- Druckseitentangente
    \pcline(A2)(A1) % <- Anstr\"omrichtung
    \lput*{:U}{\small Anstr\"omrichtung $v_{\infty}$} }%
  \psIntersectionPoint(A1)(A2)(DST1)(DST2){Hk}
  \pscurve(Hk)(.4,.38)(Vk)(.36,.33)(.5,.32)(Hk)
  \psParallelLine[linecolor=red!75!green,arrows=->,arrowscale=2](Vk)(Hk)(D){.1}{FtE}
  \psRelLine[linecolor=red!75!green,arrows=->,arrowscale=2,angle=90](D)(FtE){4}{Fn}% why "4"?
  \psParallelLine[linestyle=dashed](D)(FtE)(Fn){.1}{Fnr1}
  \psRelLine[linestyle=dashed,angle=90](FtE)(D){-4}{Fnr2} % why "-4"?
  \psline[linewidth=1.5pt,arrows=->,arrowscale=2](D)(Fnr2)
  \psIntersectionPoint(D)([nodesep=2]D)(Fnr1)([offset=-4]Fnr1){Fh}
  \psIntersectionPoint(D)([offset=2]D)(Fnr1)([nodesep=4]Fnr1){Fv}
  \psline[linecolor=blue,arrows=->,arrowscale=2](D)(Fh)
  \psline[linecolor=blue,arrows=->,arrowscale=2](D)(Fv)
  \psline[linestyle=dotted](Fh)(Fnr1)  \psline[linestyle=dotted](Fv)(Fnr1)
  \uput{.1}[0](Fh){\blue $F_{H}$}   \uput{.1}[180](Fv){\blue $F_{V}$}
  \uput{.1}[-45](Fnr1){$F_{R}$}     \uput{.1}[90](Fn){\color{red!75!green}$F_{N}$}
  \uput{.25}[-90](FtE){\color{red!75!green}$F_{T}$}
\end{pspicture}
\egroup
\end{center}
\begin{lstlisting}
\psset{xunit=0.75\linewidth,yunit=0.75\linewidth,trueAngle}%
\end{center}
\begin{pspicture}(1,0.6)%\psgrid
  \pnode(.3,.35){Vk} \pnode(.375,.35){D} \pnode(0,.4){DST1} \pnode(1,.18){DST2}
  \pnode(0,.1){A1}   \pnode(1,.31){A1}
  { \psset{linewidth=.02,linestyle=dashed,linecolor=gray}%
    \pcline(DST1)(DST2) % <- Druckseitentangente
    \pcline(A2)(A1) % <- Anstr"omrichtung
    \lput*{:U}{\small Anstr"omrichtung $v_{\infty}$} }%
  \psIntersectionPoint(A1)(A2)(DST1)(DST2){Hk}
  \pscurve(Hk)(.4,.38)(Vk)(.36,.33)(.5,.32)(Hk)
  \psParallelLine[linecolor=red!75!green,arrows=->,arrowscale=2](Vk)(Hk)(D){.1}{FtE}
  \psRelLine[linecolor=red!75!green,arrows=->,arrowscale=2,angle=90](D)(FtE){4}{Fn}% why "4"?
  \psParallelLine[linestyle=dashed](D)(FtE)(Fn){.1}{Fnr1}
  \psRelLine[linestyle=dashed,angle=90](FtE)(D){-4}{Fnr2} % why "-4"?
  \psline[linewidth=1.5pt,arrows=->,arrowscale=2](D)(Fnr2)
  \psIntersectionPoint(D)([nodesep=2]D)(Fnr1)([offset=-4]Fnr1){Fh}
  \psIntersectionPoint(D)([offset=2]D)(Fnr1)([nodesep=4]Fnr1){Fv}
  \psline[linecolor=blue,arrows=->,arrowscale=2](D)(Fh)
  \psline[linecolor=blue,arrows=->,arrowscale=2](D)(Fv)
  \psline[linestyle=dotted](Fh)(Fnr1)  \psline[linestyle=dotted](Fv)(Fnr1)
  \uput{.1}[0](Fh){\blue $F_{H}$}   \uput{.1}[180](Fv){\blue $F_{V}$}
  \uput{.1}[-45](Fnr1){$F_{R}$}     \uput{.1}[90](Fn){\color{red!75!green}$F_{N}$}
  \uput{.25}[-90](FtE){\color{red!75!green}$F_{T}$}
\end{pspicture}
\end{lstlisting}


%--------------------------------------------------------------------------------------
\section{\nxLcs{psParallelLine}}
%--------------------------------------------------------------------------------------
With this macro it is possible to plot lines relative to a given one, which is parallel.
There is no special parameter here.

\begin{lstlisting}[style=syntax]
\psParallelLine(<P0>)(<P1>)(<P2>){<length>}{<end node name>}
\psParallelLine{<arrows>}(<P0>)(<P1>)(<P2>){<length>}{<end node name>}
\psParallelLine[<options>](<P0>)(<P1>)(<P2>){<length>}{<end node name>}
\psParallelLine[<options>]{<arrows>}(<P0>)(<P1>)(<P2>){<length>}{<end node name>}
\end{lstlisting}

The line starts at $P_2$, is parallel to $\overline{P_0P_1}$ and
the length of this parallel line depends on the length factor. The
end node name must be a valid nodename and shouldn't contain any
of the special PostScript characters.

\begin{LTXexample}
\begin{pspicture*}(-5,-4)(5,3.5)
  \psgrid[subgriddiv=0,griddots=5]
  \pnode(2,-2){FF}\qdisk(FF){1.5pt}
  \pnode(-5,5){A}\pnode(0,0){O}
  \multido{\nCountA=-2.4+0.4}{9}{%
    \psParallelLine[linecolor=red](O)(A)(0,\nCountA){9}{P1}
    \psline[linecolor=red](0,\nCountA)(FF)
    \psRelLine[linecolor=red](0,\nCountA)(FF){9}{P2}
  }
  \psline[linecolor=blue](A)(FF)
  \psRelLine[linecolor=blue](A)(FF){5}{END1}
  \psline[linewidth=2pt,arrows=->](2,0)(FF)
\end{pspicture*}
\end{LTXexample}


%--------------------------------------------------------------------------------------
\section{\nxLcs{psIntersectionPoint}}
%--------------------------------------------------------------------------------------
This macro calculates the intersection point of two lines, given by the four coordinates.
There is no special parameter here.
\begin{lstlisting}[style=syntax]
\psIntersectionPoint(<P0>)(<P1>)(<P2>)(<P3>){<node name>}
\end{lstlisting}

\begin{LTXexample}[width=5.5cm]
\psset{unit=0.5cm}
\begin{pspicture}(-5,-4)(5,5)
  \psaxes[labelFontSize=\scriptstyle,
    dx=2,Dx=2,dy=2,Dy=2]{->}(0,0)(-5,-4)(5,5)
  \psline[linecolor=red,linewidth=2pt](-5,-1)(5,5)
  \psline[linecolor=blue,linewidth=2pt](-5,3)(5,-4)
  \qdisk(-5,-1){2pt}\uput[-90](-5,-1){A}
  \qdisk(5,5){2pt}\uput[-90](5,5){B}
  \qdisk(-5,3){2pt}\uput[-90](-5,3){C}
  \qdisk(5,-4){2pt}\uput[-90](5,-4){D}
  \psIntersectionPoint(-5,-1)(5,5)(-5,3)(5,-4){IP}
  \qdisk(IP){3pt}\uput{0.3}[90](IP){IP}
  \psline[linestyle=dashed](IP|0,0)(IP)(0,0|IP)
\end{pspicture}
\end{LTXexample}

\clearpage

%--------------------------------------------------------------------------------------
\section[\nxLcs{psCancel}]{\nxLcs{psCancel}\footnotemark}
%--------------------------------------------------------------------------------------
\footnotetext{Thanks to by Stefano Baroni} This macro works like
the \Lcs{cancel} macro from the package of the same name but it
allows as argument any contents, not only letters but also a
complex graphic.

\begin{BDef}
\LcsStar{psCancel}\OptArgs\Largb{contents}%
\end{BDef}

All optional arguments for lines and boxes are valid and can be
used in the usual way. The star option fills the underlying box
rectangle with the linecolor. This can be transparent if
\Lkeyword{opacity} is set to a value less than 1. This can be used
in presentation to strike out words, equations, and graphic
objects. Lines can also be transparent when the option
\Lkeyword{strokeopacity} is used.

\begingroup
\psCancel{A} \psCancel[linecolor=red]{Tikz :-)} \quad
\psCancel[linecolor=blue,doubleline=true]{%
  \readdata{\data}{demo1.data}
  \psset{shift=*,xAxisLabel=x-Axis,yAxisLabel=y-Axis,llx=-13mm,lly=-7mm,
      xAxisLabelPos={c,-1},yAxisLabelPos={-7,c}}
  \pstScalePoints(1,0.00000001){}{}
  \begin{psgraph}[axesstyle=frame,xticksize=0 7.5,yticksize=0 25,subticksize=1,
     ylabelFactor=\cdot 10^8,Dx=5,Dy=1,xsubticks=2](0,0)(25,7.5){5.5cm}{5cm}
  \listplot[linecolor=red, linewidth=2pt, showpoints=true]{\data}
  \end{psgraph}} \qquad% end of Cancel
\psCancel[linewidth=3pt,linecolor=red,
    strokeopacity=0.5]{\tabular[b]{c}first line\\second line\endtabular}\quad
\psCancel*[linecolor=red!50,opacity=0.5]{\tabular[b]{c}first line\\second line\endtabular}
\quad
\psCancel*[linecolor=blue!30,opacity=0.5]{%
  \readdata{\data}{demo1.data}
  \psset{shift=*,xAxisLabel=x-Axis,yAxisLabel=y-Axis,llx=-15mm,lly=-7mm,urx=1mm,
      xAxisLabelPos={c,-1},yAxisLabelPos={-7,c}}
  \pstScalePoints(1,0.00000001){}{}
  \begin{psgraph}[axesstyle=frame,xticksize=0 7.5,yticksize=0 25,subticksize=1,
     ylabelFactor=\cdot 10^8,Dx=5,Dy=1,xsubticks=2](0,0)(25,7.5){5.5cm}{5cm}
  \listplot[linecolor=red, linewidth=2pt, showpoints=true]{\data}
  \end{psgraph}} \quad% end of Cancel
\psCancel[linewidth=4pt,strokeopacity=0.5]{\parbox{8cm}{\[
  \binom{x_R}{y_R} = \underbrace{r\vphantom{\binom{A}{B}}}_{\text{Scaling}}\cdot
    \underbrace{\begin{pmatrix}
        \sin\gamma & -\cos\gamma \\
      \cos \gamma & \sin \gamma \\
      \end{pmatrix}}_{\text{Rotation}} \binom{x_K}{y_K} +
  \underbrace{\binom{t_x}{t_y}}_{\text{Translation}} \]} }% end of psCancel
\endgroup

\bigskip
\begin{lstlisting}
\psCancel{A} \psCancel[linecolor=red]{Tikz :-)} \quad
\psCancel[linecolor=blue,doubleline=true]{%
  \readdata{\data}{demo1.data}
  \psset{shift=*,xAxisLabel=x-Axis,yAxisLabel=y-Axis,llx=-13mm,lly=-7mm,
      xAxisLabelPos={c,-1},yAxisLabelPos={-7,c}}
  \pstScalePoints(1,0.00000001){}{}
  \begin{psgraph}[axesstyle=frame,xticksize=0 7.5,yticksize=0 25,subticksize=1,
     ylabelFactor=\cdot 10^8,Dx=5,Dy=1,xsubticks=2](0,0)(25,7.5){5.5cm}{5cm}
  \listplot[linecolor=red, linewidth=2pt, showpoints=true]{\data}
  \end{psgraph}} \qquad% end of Cancel
\psCancel[linewidth=3pt,linecolor=red,
    strokeopacity=0.5]{\tabular[b]{c}first line\\second line\endtabular}\quad
\psCancel*[linecolor=red!50,opacity=0.5]{\tabular[b]{c}first line\\second line\endtabular}
\quad
\psCancel*[linecolor=blue!30,opacity=0.5]{%
  \readdata{\data}{demo1.data}
  \psset{shift=*,xAxisLabel=x-Axis,yAxisLabel=y-Axis,llx=-15mm,lly=-7mm,urx=1mm,
      xAxisLabelPos={c,-1},yAxisLabelPos={-7,c}}
  \pstScalePoints(1,0.00000001){}{}
  \begin{psgraph}[axesstyle=frame,xticksize=0 7.5,yticksize=0 25,subticksize=1,
     ylabelFactor=\cdot 10^8,Dx=5,Dy=1,xsubticks=2](0,0)(25,7.5){5.5cm}{5cm}
  \listplot[linecolor=red, linewidth=2pt, showpoints=true]{\data}
  \end{psgraph}} \quad% end of Cancel
\psCancel[linewidth=4pt,strokeopacity=0.5]{\parbox{8cm}{\[
  \binom{x_R}{y_R} = \underbrace{r\vphantom{\binom{A}{B}}}_{\text{Scaling}}\cdot
    \underbrace{\begin{pmatrix}
        \sin\gamma & -\cos\gamma \\
      \cos \gamma & \sin \gamma \\
      \end{pmatrix}}_{\text{Rotation}} \binom{x_K}{y_K} +
  \underbrace{\binom{t_x}{t_y}}_{\text{Translation}} \]} }% end of psCancel
\end{lstlisting}

The optional argument \Lkeyword{cancelType} allows to define the lines for the non star version.
Possible values are \Lkeyval{x} for a cross, \Lkeyval{s} for a slash, and \Lkeyval{b}
for a backslash. It is also possible to use the long words for the \Lkeyval{slash} and the \Lkeyval{backslash}.
An empty value is always assumed as a \Lkeyval{x}.

\begin{LTXexample}[pos=t,wide]
\psset{linewidth=3pt,strokeopacity=0.4}
\psCancel{\tabular[b]{c}first line\\second line\endtabular}   \quad
\psCancel[cancelType=x]{\tabular[b]{c}first line\\second line\endtabular}\quad
\psCancel[cancelType=s]{\tabular[b]{c}first line\\second line\endtabular}\quad
\psCancel[cancelType=b]{\tabular[b]{c}first line\\second line\endtabular}
\end{LTXexample}

\clearpage
%--------------------------------------------------------------------------------------
\section{\nxLcs{psStep}}
%--------------------------------------------------------------------------------------
\Lcs{psStep} calculates a step function for the upper or lower
sum or the max/min of the \Index{Riemann} integral definition of a given
function. The available option is

\Lkeyset{StepType=lower}|\Lkeyval{upper}|\Lkeyval{Riemann}|\Lkeyval{infimum}|\Lkeyval{supremum} or alternative
\Lkeyset{StepType=l}|\Lkeyval{u}|\Lkeyval{R}|\Lkeyval{i}|\Lkeyval{s}

with \Lkeyword{lower} as the default setting. The syntax of the function is

\begin{BDef}
\Lcs{psStep}\OptArgs\Largr(x1,x2)\Largb{n}\Largb{function}
\end{BDef}


(x1,x2) is the given interval for the step wise calculated
function, n is the number of the rectangles and \Larg{function} is
the mathematical function in postfix or algebraic=true notation (with
\Lkeyset{algebraic=true}).

\begin{LTXexample}[pos=t,preset=\centering]
\begin{pspicture}(-0.5,-0.5)(10,3)
 \psaxes[labelFontSize=\scriptstyle]{->}(10,3)
 \psplot[plotpoints=100,linewidth=1.5pt,algebraic=true]{0}{10}{sqrt(x)}
 \psStep[linecolor=magenta,StepType=upper,fillstyle=hlines](0,9){9}{x sqrt}
 \psStep[linecolor=blue,fillstyle=vlines](0,9){9}{x sqrt }
\end{pspicture}
\end{LTXexample}

\begin{LTXexample}[pos=t,preset=\centering]
\psset{plotpoints=200}
\begin{pspicture}(-0.5,-2.25)(10,3)
  \psaxes[labelFontSize=\scriptstyle]{->}(0,0)(0,-2.25)(10,3)
 \psplot[linewidth=1.5pt,algebraic=true]{0}{10}{sqrt(x)*sin(x)}
 \psStep[algebraic=true,linecolor=magenta,StepType=upper](0,9){20}{sqrt(x)*sin(x)}
 \psStep[linecolor=blue,linestyle=dashed](0,9){20}{x sqrt x RadtoDeg sin mul}
\end{pspicture}
\end{LTXexample}

\begin{LTXexample}[pos=t,preset=\centering]
\psset{yunit=1.25cm,plotpoints=200}
\begin{pspicture}(-0.5,-1.5)(10,1.5)
 \psaxes[labelFontSize=\scriptstyle]{->}(0,0)(0,-1.5)(10,1.5)
 \psStep[algebraic=true,StepType=Riemann,fillstyle=solid,fillcolor=black!10](0,10){50}%
    {sqrt(x)*cos(x)*sin(x)}
 \psplot[linewidth=1.5pt,algebraic=true]{0}{10}{sqrt(x)*cos(x)*sin(x)}
\end{pspicture}
\end{LTXexample}


\begin{LTXexample}[pos=t,preset=\centering]
\psset{yunit=1.25cm,plotpoints=200}
\begin{pspicture}(-0.5,-1.5)(10,1.5)
 \psaxes[labelFontSize=\scriptstyle]{->}(0,0)(0,-1.5)(10,1.5)
 \psStep[algebraic=true,StepType=infimum,fillstyle=solid,fillcolor=black!10](0,10){50}%
    {sqrt(x)*cos(x)*sin(x)}
 \psplot[linewidth=1.5pt,algebraic=true]{0}{10}{sqrt(x)*cos(x)*sin(x)}
\end{pspicture}
\end{LTXexample}

\begin{LTXexample}[pos=t,preset=\centering]
\psset{yunit=1.25cm,plotpoints=200}
\begin{pspicture}(-0.5,-1.5)(10,1.5)
 \psaxes[labelFontSize=\scriptstyle]{->}(0,0)(0,-1.5)(10,1.5)
 \psStep[algebraic=true,StepType=supremum,fillstyle=solid,fillcolor=black!10](0,10){50}%
    {sqrt(x)*cos(x)*sin(x)}
 \psplot[linewidth=1.5pt,algebraic=true]{0}{10}{sqrt(x)*cos(x)*sin(x)}
\end{pspicture}
\end{LTXexample}

\begin{LTXexample}[pos=t,preset=\centering]
\psset{unit=1.5cm,plotpoints=200}
\begin{pspicture}[plotpoints=200](-0.5,-3)(10,2.5)
  \psStep[algebraic=true,fillstyle=solid,fillcolor=yellow](0.001,9.5){40}{2*sqrt(x)*cos(ln(x))*sin(x)}
  \psStep[algebraic=true,StepType=Riemann,fillstyle=solid,fillcolor=blue](0.001,9.5){40}{2*sqrt(x)*cos(ln(x))*sin(x)}
  \psaxes[labelFontSize=\scriptstyle]{->}(0,0)(0,-2.75)(10,2.5)
  \psplot[algebraic=true,linecolor=white]{0.001}{9.75}{2*sqrt(x)*cos(ln(x))*sin(x)}
  \uput[90](6,1.2){$f(x)=2\cdot\sqrt{x}\cdot\cos{(\ln{x})}\cdot\sin{x}$}
\end{pspicture}
\end{LTXexample}

\clearpage
%--------------------------------------------------------------------------------------

\section{Tangent lines}
There are two macros for plotting a tangent line or the tangent normal line.
The first one is \Lcs{psTangentLine} which expects three pairs of coordinates,
a $x$ and a $dx$ value. The second one is \Lcs{psplotTangent} which expects 
a function for the curve. \xLkeyword{Tnormal}

\subsection{\nxLcs{psTangentLine} and option \nxLkeyword{Tnormal}}

\begin{BDef}
\Lcs{psTangentLine}\OptArgs\coord1\coord2\coord3\Largb{x}\Largb{dx}
\end{BDef}

\begin{LTXexample}[width=0.45\linewidth,wide]
\psset{unit=2}
\begin{pspicture}[showgrid=true](1,-1)(4,1)
  \pscurve[showpoints=true]
    (2.1,-0.2)(2.5,0.2)(3.2,0.235)(3.8,-0.2)
  \psTangentLine[Tnormal,arrows=->,
    linecolor=red](2.5,0.2)(3.2,0.235)%
      (3.8,-0.2){3}{0.1}
  \psTangentLine[arrows=<->,
    linecolor=blue](2.5,0.2)(3.2,0.235)%
      (3.8,-0.2){3}{0.5}
\end{pspicture}
\end{LTXexample}

In special cases one has to use \Lkeyword{curvature}\verb+=1 1 1+ for the macro \Lcs{pscurve}
to get the same equation for the curve as \Lcs{psplotTangentLine} does.

\begin{LTXexample}[pos=t,preset=\centering,wide]
\psset{unit=2}
\begin{pspicture}[showgrid=true](2,-1)(6,2)
\pscurve[showpoints=true,
  curvature=1 1 1](2.1,-0.2)(2.5,0.2)(3.2,0.235)(5.8,2)
\pscurve[showpoints=true,linecolor=green,
  curvature=1 1 1](2.5,0.2)(3.2,0.235)(5.8,2)
\psTangentLine[Tnormal,arrows=->,linecolor=red](2.5,0.2)(3.2,0.235)(5.8,2){4.6}{0.6}
\psTangentLine[arrows=<->,linecolor=blue](2.5,0.2)(3.2,0.235)(5.8,2){4.5}{0.6}
\end{pspicture}
\end{LTXexample}


The end points are saved as nodes \verb=OCurve=, \verb=ETangent=, and \verb=ENormal=. They can
be used in the default ways for nodes:

\begin{LTXexample}[pos=t,preset=\centering,wide]
\psset{unit=4,arrowscale=2}
\begin{pspicture}(0.1,-0.1)(4,1)
\pscurve[showpoints=true](2.1,-0.2)(2.5,0.2)(3.2,0.4)(3.8,-0.2)
\psTangentLine[Tnormal,arrows=->,linecolor=red](2.5,0.2)(3.2,0.4)(3.8,-0.2){3.5}{0.5}
\psTangentLine[arrows=->,linecolor=blue](2.5,0.2)(3.2,0.4)(3.8,-0.2){3.5}{0.5}
\pcline[linestyle=dashed]{->}(OCurve)(ETangent|OCurve)\naput{$v_x$}
\pcline[linestyle=dashed]{->}(ETangent|OCurve)(ETangent)\naput{$v_y$}% double coordinate (x,y|x,y)
\end{pspicture}
\end{LTXexample}


\subsection{\nxLcs{psplotTangent} and option \nxLkeyword{Tnormal}}
%--------------------------------------------------------------------------------------
There is an additional option, named \Lkeyword{Derive} for an
alternative function (see following example) to calculate the
slope of the tangent. This will be in general the first
derivative, but can also be any other function. If this option is
different to to the default value \Lkeyset{Derive=default}, then this
function is taken to calculate the slope. For the other cases,
\LPack{pstricks-add} builds a secant with -0.00005<x<0.00005,
calculates the slope and takes this for the tangent. This may be
problematic in some cases of special functions or $x$ values, then
it may be appropriate to use the Derive option.

\begin{BDef}
\LcsStar{psplotTangent}\OptArgs\Largb{x}\Largb{dx}\Largb{function}
\end{BDef}



The macro expects three parameters:

\begin{description}
\item[$x$]: the $x$ value of the function for which the tangent should be calculated
\item[$dx$]: the $dx$ to both sides of the $x$ value
\item[$f(x)$]: the function in infix (with option \Lkeyword{algebraic}) or the default
postfix (PostScript) notation
\end{description}

The following examples show the use of the algebraic=true option together with the Derive option.
Remember that using the \Lkeyword{algebraic} option implies that the angles have to be in the
radian unit!

\begin{center}
\bgroup
\def\F{x RadtoDeg dup dup cos exch 2 mul cos add exch 3 mul cos add}
\def\Fp{x RadtoDeg dup dup sin exch 2 mul sin 2 mul add exch 3 mul sin 3 mul add neg}
\psset{plotpoints=1001}
\begin{pspicture}(-7.5,-2.5)(7.5,4)%X\psgrid
  \psaxes{->}(0,0)(-7.5,-2)(7.5,3.5)
  \psplot[linewidth=3\pslinewidth]{-7}{7}{\F}
  \psset{linecolor=red, arrows=<->, arrowscale=2}
  \multido{\n=-7+1}{8}{\psplotTangent{\n}{1}{\F}}
  \psset{linecolor=magenta, arrows=<->, arrowscale=2}%
  \multido{\n=0+1}{8}{\psplotTangent[linecolor=blue, Derive=\Fp]{\n}{1}{\F}}
\end{pspicture}
\egroup
\end{center}

\begin{lstlisting}
\def\F{x RadtoDeg dup dup cos exch 2 mul cos add exch 3 mul cos add}
\def\Fp{x RadtoDeg dup dup sin exch 2 mul sin 2 mul add exch 3 mul sin 3 mul add neg}
\psset{plotpoints=1001}
\begin{pspicture}(-7.5,-2.5)(7.5,4)%X\psgrid
  \psaxes{->}(0,0)(-7.5,-2)(7.5,3.5)
  \psplot[linewidth=3\pslinewidth]{-7}{7}{\F}
  \psset{linecolor=red, arrows=<->, arrowscale=2}
  \multido{\n=-7+1}{8}{\psplotTangent{\n}{1}{\F}}
  \psset{linecolor=magenta, arrows=<->, arrowscale=2}%
  \multido{\n=0+1}{8}{\psplotTangent[linecolor=blue, §\ON§Derive=\Fp§\OFF§]{\n}{1}{\F}}
\end{pspicture}
\end{lstlisting}

The star version plots only the tangent line in the positive $x$-direction:

\begin{center}
\bgroup
\def\Falg{cos(x)+cos(2*x)+cos(3*x)}   \def\Fpalg{-sin(x)-2*sin(2*x)-3*sin(3*x)}
\begin{pspicture}(-7.5,-2.5)(7.5,4)%\psgrid
  \psaxes{->}(0,0)(-7.5,-2)(7.5,3.5)
  \psplot[linewidth=1.5pt,algebraic=true,plotpoints=500]{-7.5}{7.5}{\Falg}
  \multido{\n=-7+1}{8}{\psplotTangent*[linecolor=red,arrows=->,arrowscale=2,algebraic=true]{\n}{1}{\Falg}}
  \multido{\n=0+1}{8}{\psplotTangent*[linecolor=magenta,%
     arrows=->,arrowscale=2,algebraic=true,Derive={\Fpalg}]{\n}{1}{\Falg}}
\end{pspicture}
\egroup
\end{center}

\begin{lstlisting}
\def\Falg{cos(x)+cos(2*x)+cos(3*x)}   \def\Fpalg{-sin(x)-2*sin(2*x)-3*sin(3*x)}
\begin{pspicture}(-7.5,-2.5)(7.5,4)%\psgrid
  \psaxes{->}(0,0)(-7.5,-2)(7.5,3.5)
  \psplot[linewidth=1.5pt,algebraic=true,plotpoints=500]{-7.5}{7.5}{\Falg}
  \multido{\n=-7+1}{8}{\psplotTangent*[linecolor=red,arrows=->,arrowscale=2,algebraic=true]{\n}{1}{\Falg}}
  \multido{\n=0+1}{8}{\psplotTangent*[linecolor=magenta,%
     arrows=->,arrowscale=2,algebraic=true,Derive={\Fpalg}]{\n}{1}{\Falg}}
\end{pspicture}
\end{lstlisting}

The next example shows the use of the \Lkeyword{Derive} option to draw
the perpendicular line to the tangent.

\begin{LTXexample}[width=8cm,wide]
\begin{pspicture}(-0.5,-0.5)(7.25,7.25)
  \def\Func{10 x div}
  \psaxes[arrowscale=1.5]{->}(7,7)
  \psplot[linewidth=2pt,algebraic=true]{1.5}{5}{10/x}
  \psplotTangent[linewidth=.5\pslinewidth,linecolor=red,algebraic=true]{3}{2}{10/x}
  \psplotTangent[linewidth=.5\pslinewidth,linecolor=blue,algebraic=true,Derive=(x*x)/10]{3}{2}{10/x}
  \psline[linestyle=dashed](!0 /x 3 def \Func)(!3 /x 3 def \Func)(3,0)
\end{pspicture}
\end{LTXexample}

By setting the optional argument \Lkeyword{Tnormal} one can plot the
normal of the tangent line. It always starts at the given point.

\begin{LTXexample}[width=8cm,wide]
\begin{pspicture}(-0.5,-0.5)(7.25,7.25)
  \def\Func{10 x div}
  \psaxes[arrowscale=1.5]{->}(7,7)
  \psplot[linewidth=2pt]{1.5}{5}{\Func}
  \psplotTangent[linewidth=1.5\pslinewidth,linecolor=red]{3}{2}{\Func}
  \psplotTangent[linewidth=1.5\pslinewidth,linecolor=blue,Tnormal]{3}{2}{\Func}
  \psline[linestyle=dashed](!0 /x 3 def \Func)(!3 /x 3 def \Func)(3,0)
\end{pspicture}
\end{LTXexample}


Let's work with the classical \Index{cardioid}: $r=2(1+\cos(\theta))$ and
$\displaystyle \frac{d r}{d\theta}=-2\sin(\theta)$. The \Lkeyword{Derive}
option always expects the $\frac{d r}{d\theta}$ value and uses
internally the equation for the derivative of implicitly defined
functions:

\[
\frac{dy}{dx}=\frac{r^\prime\cdot\sin\theta + x}{r^\prime\cdot\cos\theta - y}
\]
where $x=r\cdot\cos\theta$ and $y=r\cdot\sin\theta$


\begin{LTXexample}[width=6cm,wide]
\begin{pspicture}(-1,-3)(5,3)%\psgrid[subgridcolor=lightgray]
  \psaxes{->}(0,0)(-1,-3)(5,3)
  \psplot[polarplot,linewidth=3\pslinewidth,linecolor=blue,%
     plotpoints=500]{0}{360}{1 x cos add 2 mul}
\end{pspicture}
\end{LTXexample}

\psset{algebraic=false}
\begin{LTXexample}[width=6cm,wide]
\begin{pspicture}(-1,-3)(5,3)%\psgrid[subgridcolor=lightgray]
  \psaxes{->}(0,0)(-1,-3)(5,3)
  \psplot[polarplot,linewidth=3\pslinewidth,linecolor=blue,plotpoints=500]{0}{360}{1 x cos add 2 mul}
  \multido{\n=0+36}{10}{%
     \psplotTangent[polarplot,linecolor=red,arrows=<->]{\n}{1.5}{1 x cos add 2 mul} }
\end{pspicture}
\end{LTXexample}

\begin{LTXexample}[width=6cm,wide]
\begin{pspicture}(-1,-3)(5,3)%\psgrid[subgridcolor=lightgray]
  \psaxes{->}(0,0)(-1,-3)(5,3)
  \psplot[polarplot,linewidth=3\pslinewidth,linecolor=blue,algebraic=true,plotpoints=500]{0}{6.289}{2*(1+cos(x))}
  \multido{\r=0.000+0.314}{21}{%
     \psplotTangent[polarplot,Derive=-2*sin(x),algebraic=true,linecolor=red,arrows=<->]{\r}{1.5}{2*(1+cos(x))} }
\end{pspicture}
\end{LTXexample}


Let's work with a \Index{Lissajou curve}:
 $\displaystyle\left\{\begin{array}{l}x=3.5\cos(2t)\\y=3.5\sin(6t)\end{array}\right.$
whose derivative is :
 $\displaystyle\left\{\begin{array}{l}x=-7\sin(2t)\\y=21\cos(6t)\end{array}\right.$

The parameter must be the letter $t$ instead of $x$ and when using
the \Lkeyword{algebraic=true} option you must separate the two equations by
a \Lnotation{|} (see example).

\begin{LTXexample}[pos=t,wide]
\def\Lissa{t dup 2 RadtoDeg mul cos 3.5 mul exch 6 mul RadtoDeg sin 3.5 mul}%
\psset{yunit=0.6}
\begin{pspicture}(-4,-4)(4,6)
  \parametricplot[plotpoints=500,linewidth=3\pslinewidth]{0}{3.141592}{\Lissa}
  \multido{\r=0.000+0.314}{11}{%
    \psplotTangent[linecolor=red,arrows=<->]{\r}{1.5}{\Lissa} }
  \multido{\r=0.157+0.314}{11}{%
    \psplotTangent[linecolor=blue,arrows=<->]{\r}{1.5}{\Lissa} }
\end{pspicture}\hfill%
\def\LissaAlg{3.5*cos(2*t)|3.5*sin(6*t)}  \def\LissaAlgDer{-7*sin(2*t)|21*cos(6*t)}%
\begin{pspicture}(-4,-4)(4,6)
  \parametricplot[algebraic=true,plotpoints=500,linewidth=3\pslinewidth]{0}{3.141592}{\LissaAlg}
  \multido{\r=0.000+0.314}{11}{%
    \psplotTangent[algebraic=true,linecolor=red,arrows=<->]{\r}{1.5}{\LissaAlg} }
  \multido{\r=0.157+0.314}{11}{%
    \psplotTangent[algebraic=true,linecolor=blue,arrows=<->,%
       Derive=\LissaAlgDer]{\r}{1.5}{\LissaAlg} }
\end{pspicture}
\end{LTXexample}


\clearpage
\section{Successive derivatives of a function}

The new PostScript function \Lps{Derive} has been added for
plotting successive derivatives of a function. It must be used
with the \Lkeyword{algebraic=true} option. This function has two arguments:

\begin{enumerate}
\item a positive integer which defines the order of the derivative; obviously $0$ means the
  function itself!
\item a function of variable $x$ which can be any function using common operators,
\end{enumerate}

Do not think that the derivative is approximated, the internal PostScript engine will
compute the real derivative using a formal derivative engine.

The following diagram contains the plot of the polynomial:

\[ f(x)=\sum_{i=0}^{14}\frac{(-1)^{i}x^{2i}}{i!}=1-\frac{x^2}{2}+\frac{x^4}{4!}-\frac{x^6}{6!}+\frac{x^8}{8!}-
          \frac{x^{10}}{10!}+\frac{x^{12}}{12!}-\frac{x^{14}}{14!}\]

and of its first 15 derivatives. It is the sequence definition of
the cosine.


\begin{LTXexample}[pos=t,wide,preset=\centering]
\psset{unit=2}
\def\getColor#1{\ifcase#1 Tan\or RedOrange\or magenta\or yellow\or green\or Orange\or blue\or
  DarkOrchid\or BrickRed\or Rhodamine\or OliveGreen\or Goldenrod\or Mahogany\or
  OrangeRed\or CarnationPink\or RoyalPurple\or Lavender\fi}
\begin{pspicture}[showgrid=true](0,-1.2)(7,1.5)
  \psclip{\psframe[linestyle=none](0,-1.1)(7,1.1)}
  \multido{\in=0+1}{16}{%
     \psplot[linewidth=1pt,algebraic=true,linecolor=\getColor{\in}]{0}{7}
      {Derive(\in,1-x^2/2+x^4/24-x^6/720+x^8/40320-x^10/3628800+x^12/479001600-x^14/87178291200)}}
  \endpsclip
\end{pspicture}
\end{LTXexample}

\begin{LTXexample}[width=3.5cm]
\begin{pspicture}[shift=-2.5,showgrid=true,linewidth=1pt](0,-2)(3,3)
  \psplot[algebraic=true]{.001}{3}{x*ln(x)}  % f(x)
  \psplot[algebraic=true,linecolor=red]{.05}{3}{Derive(1,x*ln(x))} % f'(x)=1+ln(x)
\end{pspicture}
\end{LTXexample}


\clearpage
\section{Variable step for plotting a curve}
\subsection{Theory}

As you know with the \Lcs{psplot} macro, the curve is plotted
using a piece-wise linear curve. The step is given by the
parameter \Lkeyword{plotpoints}. For each step between $x_i$ and
$x_{i+1}$, the area defined between the curve and its
approximation (a segment) is majored by this formula :

\begin{minipage}[m]{.5\linewidth}
\[|\varepsilon|\le\frac{M_2(f)(x_{i+1}-x_i)^3}{12}\]

$M_2(f)$ is a majorant of the second derivative of $f$ in the interval $[x_i;x_{i+1}]$.
\end{minipage}
{\psset{unit=1cm, showpoints=false}
\begin{pspicture}[shift=-2,showgrid=true](0,-1)(6,3)
  \pscurve(0,0)(1,1)(3,2.2)(5,2)(6,1)\psline(1,1)(5,2)
  \psline(.5,0)(5.5,0)\psline(1,0)(1,1)\psline(5,0)(5,2)
  \rput[t](1,-.1){$x_n$}\rput[t](5,-.1){$x_{n+1}$}
  \psclip{\pscustom{\psecurve(0,0)(1,1)(3,2.2)(5,2)(6,1)\psline(5,2)}}
    \psframe[fillstyle=solid, fillcolor=gray](0,0)(5,5)
  \endpsclip
  \rput*(3,1.8){$\varepsilon$}
\end{pspicture}}



The parameter \Lkeyword{VarStep} (\false\ by default) activates
the variable step algorithm. It is set to a tolerance defined by
the parameter \Lkeyword{VarStepEpsilon} (\Lkeyval{default} by default,
accept real value). If this parameter is not set by the user, then
it is automatically computed using the default first step given by
the parameter \Lkeyword{plotpoints}. Then, for each step, $f''(x_n)$
and $f''(x_{n+1})$ are computed and the smaller is used as
$M_2(f)$, and then the step is approximated. This means that the
step is constant for second order polynomials.

\subsection{The cosine}

Different value for the tolerance from $0.01$ to $0.000\,1$, a factor $10$ between
each of them. In black, there is the classic \Lcs{psplot} behavior, and in
magenta the default variable step behavior.

\begin{center}
\bgroup
\psset{algebraic=true, VarStep=true, unit=2, showpoints=true, linecolor=red}
\begin{pspicture}(-0,-1)(3.14,2)\psgrid
  \psplot[VarStepEpsilon=.01]{0}{3.14}{cos(x)}
  \psplot[VarStepEpsilon=.001]{0}{3.14}{cos(x)+.15}
  \psplot[VarStepEpsilon=.0001]{0}{3.14}{cos(x)+.3}
  \psplot[linecolor=magenta]{0}{3.14}{cos(x)+.45}
  \psplot[VarStep=false, linewidth=2\pslinewidth, linecolor=black]{-0}{3.14}{cos(x)+.6}
\end{pspicture}
\egroup
\end{center}

\begin{lstlisting}
\psset{algebraic=true, VarStep=true, unit=2, showpoints=true, linecolor=red}
\begin{pspicture}[showgrid=true](-0,-1)(3.14,2)
  \psplot[VarStepEpsilon=.01]{0}{3.14}{cos(x)}
  \psplot[VarStepEpsilon=.001]{0}{3.14}{cos(x)+.15}
  \psplot[VarStepEpsilon=.0001]{0}{3.14}{cos(x)+.3}
  \psplot[linecolor=magenta]{0}{3.14}{cos(x)+.45}
  \psplot[VarStep=false,linewidth=1pt,linecolor=black]{-0}{3.14}{cos(x)+.6}
\end{pspicture}
\end{lstlisting}


\subsection{The Napierian Logarithm}

A really classic example which gives a bad beginning, the tolerance is set to $0.001$.

\begin{center}
\bgroup
\psset{algebraic=true, VarStep=true, linecolor=red, showpoints=true}
\begin{pspicture}[showgrid=true](0,-5)(16,4)
  \psplot[VarStep=false, linecolor=black]{.01}{16}{ln(x)+1}
  \psplot[linecolor=magenta]{.51}{16}{ln(x-1/2)+1/2}
  \psplot[VarStepEpsilon=.001]{1.01}{16}{ln(x-1)}
  \psplot[VarStepEpsilon=.01]{1.51}{16}{ln(x-1.5)-100/200}
\end{pspicture}
\egroup
\end{center}

\begin{lstlisting}
\psset{algebraic=true, VarStep=true, linecolor=red, showpoints=true}
\begin{pspicture}[showgrid=true](0,-5)(16,4)
  \psplot[VarStep=false, linecolor=black]{.01}{16}{ln(x)+1}
  \psplot[linecolor=magenta]{.51}{16}{ln(x-1/2)+1/2}
  \psplot[VarStepEpsilon=.001]{1.01}{16}{ln(x-1)}
  \psplot[VarStepEpsilon=.01]{1.51}{16}{ln(x-1.5)-100/200}
\end{pspicture}
\end{lstlisting}


\clearpage
\subsection{Sine of the inverse of $x$}
Impossible to draw, but let's try!

\begin{center}
\bgroup
\psset{xunit=64,algebraic=true,VarStep,linecolor=red,showpoints=true,linewidth=1pt}
\begin{pspicture}[showgrid=true](0,-1)(.5,1)
  \psplot[VarStepEpsilon=.0001]{.01}{.25}{sin(1/x)}
\end{pspicture}\\
\begin{pspicture}[showgrid=true](0,-1)(.5,1)
  \psplot[VarStepEpsilon=.00001]{.01}{.25}{sin(1/x)}
\end{pspicture}\\
\begin{pspicture}[showgrid=true](0,-1)(.5,1)
  \psplot[VarStepEpsilon=.000001]{.01}{.25}{sin(1/x)}
\end{pspicture}\\
\begin{pspicture}[showgrid=true](0,-1)(.5,1)
  \psplot[VarStep=false, linecolor=black]{.01}{.25}{sin(1/x)}
\end{pspicture}
\egroup
\end{center}

\begin{lstlisting}
\psset{xunit=64,algebraic=true,VarStep,linecolor=red,showpoints=true,linewidth=1pt}
\begin{pspicture}[showgrid=true](0,-1)(.5,1)
  \psplot[VarStepEpsilon=.0001]{.01}{.25}{sin(1/x)}
\end{pspicture}\\
\begin{pspicture}[showgrid=true](0,-1)(.5,1)
  \psplot[VarStepEpsilon=.00001]{.01}{.25}{sin(1/x)}
\end{pspicture}\\
\begin{pspicture}[showgrid=true](0,-1)(.5,1)
  \psplot[VarStepEpsilon=.000001]{.01}{.25}{sin(1/x)}
\end{pspicture}\\
\begin{pspicture}[showgrid=true](0,-1)(.5,1)
  \psplot[VarStep=false, linecolor=black]{.01}{.25}{sin(1/x)}
\end{pspicture}
\end{lstlisting}





\clearpage
\subsection{A really complicated function}

Just appreciate the difference between the normal behavior and the plotting with the
\Lkeyword{varStep} option. The function is:

\[f(x)=x-\frac{x^2}{10}+\ln(x)+\cos(2x)+\sin(x^2)-1\]

\begin{center}
\bgroup
\psset{xunit=3, algebraic=true, VarStep, showpoints=true}
\begin{pspicture}[showgrid=true](0,-2)(5,6)
  \psplot[VarStepEpsilon=.0005, linecolor=red]{.1}{5}{x-x^2/10+ln(x)+cos(2*x)+sin(x^2)}
  \psplot[linecolor=magenta]{.1}{5}{x-x^2/10+ln(x)+cos(2*x)+sin(x^2)+.5}
  \psplot[VarStep=false]{.1}{5}{x-x^2/10+ln(x)+cos(2*x)+sin(x^2)-1}
\end{pspicture}
\egroup
\end{center}

\begin{lstlisting}
\psset{xunit=3, algebraic=true, VarStep, showpoints=true}
\begin{pspicture}[showgrid=true](0,-2)(5,6)
  \psplot[VarStepEpsilon=.0005, linecolor=red]{.1}{5}{x-x^2/10+ln(x)+cos(2*x)+sin(x^2)}
  \psplot[linecolor=magenta]{.1}{5}{x-x^2/10+ln(x)+cos(2*x)+sin(x^2)+.5}
  \psplot[VarStep=false]{.1}{5}{x-x^2/10+ln(x)+cos(2*x)+sin(x^2)-1}
\end{pspicture}
\end{lstlisting}


\clearpage
\subsection{A hyperbola}

\begin{center}
\bgroup
\psset{algebraic=true, showpoints=true, unit=0.75}
\begin{pspicture}(-5,-4)(9,6)
  \psplot[linecolor=black]{-5}{1.8}{(x-1)/(x-2)}
  \psplot[VarStep=true, VarStepEpsilon=.001, linecolor=red]{2.2}{9}{(x-1)/(x-2)}
  \psaxes{->}(0,0)(-5,-4)(9,6)
\end{pspicture}
\egroup
\end{center}

\begin{lstlisting}
\psset{algebraic=true, showpoints=true, unit=0.75}
\begin{pspicture}(-5,-4)(9,6)
  \psplot[linecolor=black]{-5}{1.8}{(x-1)/(x-2)}
  \psplot[VarStep=true, VarStepEpsilon=.001, linecolor=red]{2.2}{9}{(x-1)/(x-2)}
  \psaxes{->}(0,0)(-5,-4)(9,6)
\end{pspicture}
\end{lstlisting}



\clearpage
\subsection{Using \nxLcs{psparametricplot}}

\begin{BDef}
\Lcs{parametricplot}\OptArgs\Largb{t0}\Largb{t1}\OptArg{PS commands}\Largb{x(t) y(t)}
\end{BDef}

\begin{center}
\bgroup
\psset{unit=2.5}
\begin{pspicture}[showgrid=true](-1,-1)(1,1)
\parametricplot[algebraic=true,linecolor=red,VarStep=true, showpoints=true,
                VarStepEpsilon=.0001]
                {-3.14}{3.14}{cos(3*t)|sin(2*t)}
\end{pspicture}
\begin{pspicture}[showgrid=true](-1,-1)(1,1)
\parametricplot[algebraic=true,linecolor=blue,VarStep=true, showpoints=false,
                VarStepEpsilon=.0001]
                {-3.14}{3.14}{cos(3*t)|sin(2*t)}
\end{pspicture}
\egroup
\end{center}

\begin{lstlisting}
\psset{unit=3}
\begin{pspicture}[showgrid=true](-1,-1)(1,1)
\parametricplot[algebraic=true,linecolor=red,VarStep=true, showpoints=true,
                VarStepEpsilon=.0001]
                {-3.14}{3.14}{cos(3*t)|sin(2*t)}
\end{pspicture}
\begin{pspicture}[showgrid=true](-1,-1)(1,1)
\parametricplot[algebraic=true,linecolor=blue,VarStep=true, showpoints=false,
                VarStepEpsilon=.0001]
                {-3.14}{3.14}{cos(3*t)|sin(2*t)}
\end{pspicture}
\end{lstlisting}


\begin{center}
\bgroup
\psset{unit=2.5}
\begin{pspicture}[showgrid=true](-1,-1)(1,1)
\parametricplot[algebraic=true,linecolor=red,VarStep=true, showpoints=true,
                VarStepEpsilon=.0001]
                {0}{47.115}{cos(5*t)|sin(3*t)}
\end{pspicture}
\begin{pspicture}[showgrid=true](-1,-1)(1,1)
\parametricplot[algebraic=true,linecolor=blue,VarStep=true, showpoints=false,
                VarStepEpsilon=.0001]
                {0}{47.115}{cos(5*t)|sin(3*t)}
\end{pspicture}
\egroup
\end{center}

\begin{lstlisting}
\psset{unit=2.5}
\begin{pspicture}[showgrid=true](-1,-1)(1,1)
\parametricplot[algebraic=true,linecolor=red,VarStep=true, showpoints=true,
                VarStepEpsilon=.0001]
                {0}{47.115}{cos(5*t)|sin(3*t)}
\end{pspicture}
\begin{pspicture}[showgrid=true](-1,-1)(1,1)
\parametricplot[algebraic=true,linecolor=blue,VarStep=true, showpoints=false,
                VarStepEpsilon=.0001]
                {0}{47.115}{cos(5*t)|sin(3*t)}
\end{pspicture}
\end{lstlisting}


\begin{center}
\bgroup
\psset{xunit=.5}
\begin{pspicture}[showgrid=true](0,0)(12.566,2)
\parametricplot[algebraic=true,linecolor=red,VarStep, showpoints=true,
        VarStepEpsilon=.01]{0}{12.566}{t+cos(-t-Pi/2)|1+sin(-t-Pi/2)}
\end{pspicture}
%
\begin{pspicture}[showgrid=true](0,0)(12.566,2)
\parametricplot[algebraic=true,linecolor=blue,VarStep, showpoints=false,
        VarStepEpsilon=.001]{0}{12.566}{t+cos(-t-Pi/2)|1+sin(-t-Pi/2)}
\end{pspicture}
\egroup
\end{center}

\begin{lstlisting}
\psset{xunit=.5}
\begin{pspicture}[showgrid=true](0,0)(12.566,2)
\parametricplot[algebraic=true,linecolor=red,VarStep, showpoints=true,
        VarStepEpsilon=.01]{0}{12.566}{t+cos(-t-Pi/2)|1+sin(-t-Pi/2)}
\end{pspicture}
%
\begin{pspicture}[showgrid=true](0,0)(12.566,2)
\parametricplot[algebraic=true,linecolor=blue,VarStep, showpoints=false,
        VarStepEpsilon=.001]{0}{12.566}{t+cos(-t-Pi/2)|1+sin(-t-Pi/2)}
\end{pspicture}
\end{lstlisting}


\section{New math functions and their derivatives}

\subsection{The inverse sine and its derivative}

\begin{center}
\bgroup
\psset{unit=1.5}
\begin{pspicture}[showgrid=true](-1,-2)(1,2)
  \psplot[linecolor=blue,algebraic=true]{-1}{1}{asin(x)}
\end{pspicture}
\hspace{1em}
\psset{algebraic, VarStep, VarStepEpsilon=.001, showpoints=true}
\begin{pspicture}[showgrid=true](-1,-2)(1,2)
  \psplot[linecolor=blue]{-.999}{.999}{asin(x)}
\end{pspicture}
\hspace{1em}
\begin{pspicture}[showgrid=true](-1,0)(1,4)
  \psplot[linecolor=blue]{-.97}{.97}{Derive(1,asin(x))}
\end{pspicture}
\hspace{1em}
\psset{algebraic=true, VarStep, VarStepEpsilon=.0001, showpoints=true}
\begin{pspicture}[showgrid=true](-1,0)(1,4)
  \psplot[linecolor=blue]{-.97}{.97}{Derive(1,asin(x))}
\end{pspicture}
\egroup
\end{center}

\begin{lstlisting}
\psset{unit=1.5}
\begin{pspicture}[showgrid=true](-1,-2)(1,2)
  \psplot[linecolor=blue,algebraic=true]{-1}{1}{asin(x)}
\end{pspicture}
\hspace{1em}
\psset{algebraic=true, VarStep, VarStepEpsilon=.001, showpoints=true}
\begin{pspicture}[showgrid=true](-1,-2)(1,2)
  \psplot[linecolor=blue]{-.999}{.999}{asin(x)}
\end{pspicture}
\hspace{1em}
\begin{pspicture}[showgrid=true](-1,0)(1,4)
  \psplot[linecolor=red]{-.97}{.97}{Derive(1,asin(x))}
\end{pspicture}
\hspace{1em}
\psset{algebraic=true, VarStep, VarStepEpsilon=.0001, showpoints=true}
\begin{pspicture}[showgrid=true](-1,0)(1,4)
  \psplot[linecolor=red]{-.97}{.97}{Derive(1,asin(x))}
\end{pspicture}
\end{lstlisting}


\subsection{The inverse cosine and its derivative}

\begin{center}
\bgroup
\psset{unit=1.5}
\begin{pspicture}[showgrid=true](-1,0)(1,3)
  \psplot[linecolor=blue,algebraic=true]{-1}{1}{acos(x)}
\end{pspicture}
\hspace{1em}
\psset{algebraic=true, VarStep, VarStepEpsilon=.001, showpoints=true}
\begin{pspicture}[showgrid=true](-1,0)(1,3)
  \psplot[linecolor=blue]{-.999}{.999}{acos(x)}
\end{pspicture}
\hspace{1em}
\begin{pspicture}[showgrid=true](-1,-4)(1,-1)
  \psplot[linecolor=blue]{-.97}{.97}{Derive(1,acos(x))}
\end{pspicture}
\hspace{1em}
\psset{algebraic=true, VarStep, VarStepEpsilon=.0001, showpoints=true}
\begin{pspicture}[showgrid=true](-1,-4)(1,-1)
  \psplot[linecolor=blue]{-.97}{.97}{Derive(1,acos(x))}
\end{pspicture}
\egroup
\end{center}

\begin{lstlisting}
\psset{unit=1.5}
\begin{pspicture}[showgrid=true](-1,0)(1,3)
  \psplot[linecolor=blue,algebraic=true]{-1}{1}{acos(x)}
\end{pspicture}
\hspace{1em}
\psset{algebraic=true, VarStep, VarStepEpsilon=.001, showpoints=true}
\begin{pspicture}[showgrid=true](-1,0)(1,3)
  \psplot[linecolor=blue]{-.999}{.999}{acos(x)}
\end{pspicture}
\hspace{1em}
\begin{pspicture}[showgrid=true](-1,-4)(1,-1)
  \psplot[linecolor=red]{-.97}{.97}{Derive(1,acos(x))}
\end{pspicture}
\hspace{1em}
\psset{algebraic=true, VarStep, VarStepEpsilon=.0001, showpoints=true}
\begin{pspicture}[showgrid=true](-1,-4)(1,-1)
  \psplot[linecolor=red]{-.97}{.97}{Derive(1,acos(x))}
\end{pspicture}
\end{lstlisting}



\subsection{The inverse tangent and its derivative}

\begin{center}
\bgroup
\begin{pspicture}[showgrid=true](-4,-2)(4,2)
\psset{algebraic=true}
  \psplot[linecolor=blue,linewidth=1pt]{-4}{4}{atg(x)}
  \psplot[linecolor=red,VarStep, VarStepEpsilon=.0001, showpoints=true]{-4}{4}{Derive(1,atg(x))}
\end{pspicture}
\hspace{1em}
\begin{pspicture}[showgrid=true](-4,-2)(4,2)
\psset{algebraic=true, VarStep, VarStepEpsilon=.001, showpoints=true}
  \psplot[linecolor=blue]{-4}{4}{atg(x)}
  \psplot[linecolor=red]{-4}{4}{Derive(1,atg(x))}
\end{pspicture}
\egroup
\end{center}

\begin{lstlisting}
\begin{pspicture}[showgrid=true](-4,-2)(4,2)
\psset{algebraic=true}
  \psplot[linecolor=blue,linewidth=1pt]{-4}{4}{atg(x)}
  \psplot[linecolor=red,VarStep, VarStepEpsilon=.0001, showpoints=true]{-4}{4}{Derive(1,atg(x))}
\end{pspicture}
\hspace{1em}
\begin{pspicture}[showgrid=true](-4,-2)(4,2)
\psset{algebraic=true, VarStep, VarStepEpsilon=.001, showpoints=true}
  \psplot[linecolor=blue]{-4}{4}{atg(x)}
  \psplot[linecolor=red]{-4}{4}{Derive(1,atg(x))}
\end{pspicture}
\end{lstlisting}

\subsection{Hyperbolic functions}

\begin{center}
\bgroup
\begin{pspicture}(-3,-4)(3,4)
\psset{algebraic=true}
  \psplot[linecolor=red,linewidth=1pt]{-2}{2}{sh(x)}
  \psplot[linecolor=blue,linewidth=1pt]{-2}{2}{ch(x)}
  \psplot[linecolor=green,linewidth=1pt]{-3}{3}{th(x)}
  \psaxes{->}(0,0)(-3,-4)(3,4)
\end{pspicture}
\hspace{1em}
\begin{pspicture}(-3,-4)(3,4)
\psset{algebraic=true, VarStep=true, VarStepEpsilon=.001, showpoints=true}
  \psplot[linecolor=red,linewidth=1pt]{-2}{2}{sh(x)}
  \psplot[linecolor=blue,linewidth=1pt]{-2}{2}{ch(x)}
  \psplot[linecolor=green,linewidth=1pt]{-3}{3}{th(x)}
  \psaxes{->}(0,0)(-3,-4)(3,4)
\end{pspicture}
\egroup
\end{center}

\begin{lstlisting}
\begin{pspicture}(-3,-4)(3,4)
\psset{algebraic=true}
  \psplot[linecolor=red,linewidth=1pt]{-2}{2}{sh(x)}
  \psplot[linecolor=blue,linewidth=1pt]{-2}{2}{ch(x)}
  \psplot[linecolor=green,linewidth=1pt]{-3}{3}{th(x)}
  \psaxes{->}(0,0)(-3,-4)(3,4)
\end{pspicture}
\hspace{1em}
\begin{pspicture}(-3,-4)(3,4)
\psset{algebraic=true, VarStep=true, VarStepEpsilon=.001, showpoints=true}
  \psplot[linecolor=red,linewidth=1pt]{-2}{2}{sh(x)}
  \psplot[linecolor=blue,linewidth=1pt]{-2}{2}{ch(x)}
  \psplot[linecolor=green,linewidth=1pt]{-3}{3}{th(x)}
  \psaxes{->}(0,0)(-3,-4)(3,4)
\end{pspicture}
\end{lstlisting}



\begin{center}
\bgroup
\begin{pspicture}(-3,-4)(3,4)
\psset{algebraic=true}
  \psplot[linecolor=red,linewidth=1pt]{-2}{2}{Derive(1,sh(x))}
  \psplot[linecolor=blue,linewidth=1pt]{-2}{2}{Derive(1,ch(x))}
  \psplot[linecolor=green,linewidth=1pt]{-3}{3}{Derive(1,th(x))}
  \psaxes{->}(0,0)(-3,-4)(3,4)
\end{pspicture}
\hspace{1em}
\begin{pspicture}(-3,-4)(3,4)
\psset{algebraic=true, VarStep=true, VarStepEpsilon=.001, showpoints=true}
  \psplot[linecolor=red,linewidth=1pt]{-2}{2}{Derive(1,sh(x))}
  \psplot[linecolor=blue,linewidth=1pt]{-2}{2}{Derive(1,ch(x))}
  \psplot[linecolor=green,linewidth=1pt]{-3}{3}{Derive(1,th(x))}
  \psaxes{->}(0,0)(-3,-4)(3,4)
\end{pspicture}
\egroup
\end{center}

\begin{lstlisting}
\begin{pspicture}(-3,-4)(3,4)
\psset{algebraic=true,linewidth=1pt}
  \psplot[linecolor=red,linewidth=1pt]{-2}{2}{Derive(1,sh(x))}
  \psplot[linecolor=blue,linewidth=1pt]{-2}{2}{Derive(1,ch(x))}
  \psplot[linecolor=green,linewidth=1pt]{-3}{3}{Derive(1,th(x))}
  \psaxes{->}(0,0)(-3,-4)(3,4)
\end{pspicture}
\hspace{1em}
\begin{pspicture}(-3,-4)(3,4)
\psset{algebraic=true, VarStep=true, VarStepEpsilon=.001, showpoints=true}
  \psplot[linecolor=red,linewidth=1pt]{-2}{2}{Derive(1,sh(x))}
  \psplot[linecolor=blue,linewidth=1pt]{-2}{2}{Derive(1,ch(x))}
  \psplot[linecolor=green,linewidth=1pt]{-3}{3}{Derive(1,th(x))}
  \psaxes{->}(0,0)(-3,-4)(3,4)
\end{pspicture}
\end{lstlisting}



\begin{center}
\bgroup
\begin{pspicture}(-7,-3)(7,3)
\psset{algebraic=true}
  \psplot[linecolor=red,linewidth=1pt]{-7}{7}{Argsh(x)}
  \psplot[linecolor=blue,linewidth=1pt]{1}{7}{Argch(x)}
  \psplot[linecolor=green,linewidth=1pt]{-.99}{.99}{Argth(x)}
  \psaxes{->}(0,0)(-7,-3)(7,3)
\end{pspicture}\\[\baselineskip]
\begin{pspicture}(-7,-3)(7,3)
  \psset{algebraic=true, VarStep, VarStepEpsilon=.001, showpoints=true}
  \psplot[linecolor=red,linewidth=1pt]{-7}{7}{Argsh(x)}
  \psplot[linecolor=blue,linewidth=1pt]{1.001}{7}{Argch(x)}
  \psplot[linecolor=green,linewidth=1pt]{-.99}{.99}{Argth(x)}
  \psaxes{->}(0,0)(-7,-3)(7,3)
\end{pspicture}
\egroup
\end{center}

\begin{lstlisting}
\begin{pspicture}(-7,-3)(7,3)
\psset{algebraic=true}
  \psplot[linecolor=red,linewidth=1pt]{-7}{7}{Argsh(x)}
  \psplot[linecolor=blue,linewidth=1pt]{1}{7}{Argch(x)}
  \psplot[linecolor=green,linewidth=1pt]{-.99}{.99}{Argth(x)}
  \psaxes{->}(0,0)(-7,-3)(7,3)
\end{pspicture}\\[\baselineskip]
\begin{pspicture}(-7,-3)(7,3)
  \psset{algebraic=true, VarStep, VarStepEpsilon=.001, showpoints=true}
  \psplot[linecolor=red,linewidth=1pt]{-7}{7}{Argsh(x)}
  \psplot[linecolor=blue,linewidth=1pt]{1.001}{7}{Argch(x)}
  \psplot[linecolor=green,linewidth=1pt]{-.99}{.99}{Argth(x)}
  \psaxes{->}(0,0)(-7,-3)(7,3)
\end{pspicture}
\end{lstlisting}



\begin{center}
\bgroup
\begin{pspicture}(-7,-0.5)(7,6)
\psset{algebraic=true}
  \psplot[linecolor=red,linewidth=1pt]{-7}{7}{Derive(1,Argsh(x))}
  \psplot[linecolor=blue,linewidth=1pt]{1.014}{7}{Derive(1,Argch(x))}
  \psplot[linecolor=green,linewidth=1pt]{-.9}{.9}{Derive(1,Argth(x))}
  \psaxes{->}(0,0)(-7,0)(7,6)
\end{pspicture}\\[\baselineskip]
\begin{pspicture}(-7,-0.5)(7,6)
\psset{algebraic=true}
  \psset{algebraic=true, VarStep=true, VarStepEpsilon=.001, showpoints=true}
  \psplot[linecolor=red,linewidth=1pt]{-7}{7}{Derive(1,Argsh(x))}
  \psplot[linecolor=blue,linewidth=1pt]{1.014}{7}{Derive(1,Argch(x))}
  \psplot[linecolor=green,linewidth=1pt]{-.9}{.9}{Derive(1,Argth(x))}
  \psaxes{->}(0,0)(-7,0)(7,6)
\end{pspicture}
\egroup
\end{center}

\begin{lstlisting}
\begin{pspicture}(-7,-0.5)(7,6)
\psset{algebraic=true}
  \psplot[linecolor=red,linewidth=1pt]{-7}{7}{Derive(1,Argsh(x))}
  \psplot[linecolor=blue,linewidth=1pt]{1.014}{7}{Derive(1,Argch(x))}
  \psplot[linecolor=green,linewidth=1pt]{-.9}{.9}{Derive(1,Argth(x))}
  \psaxes{->}(0,0)(-7,0)(7,6)
\end{pspicture}\\[\baselineskip]
\begin{pspicture}(-7,-0.5)(7,6)
\psset{algebraic=true}
  \psset{algebraic=true, VarStep=true, VarStepEpsilon=.001, showpoints=true}
  \psplot[linecolor=red,linewidth=1pt]{-7}{7}{Derive(1,Argsh(x))}
  \psplot[linecolor=blue,linewidth=1pt]{1.014}{7}{Derive(1,Argch(x))}
  \psplot[linecolor=green,linewidth=1pt]{-.9}{.9}{Derive(1,Argth(x))}
  \psaxes{->}(0,0)(-7,0)(7,6)
\end{pspicture}
\end{lstlisting}


\clearpage
%--------------------------------------------------------------------------------------
\section[\nxLcs{psplotDiffEqn} -- solving diffential equations]%
  {\nxLcs{psplotDiffEqn} -- solving diffential equations}
%--------------------------------------------------------------------------------------


 A differential equation of first order is like

\begin{align} y^\prime=f(x,y,y^\prime) \end{align}


where $y$ is a function of $x$. We define some vectors $Y=[y, y',
\cdots , y^{(n-1)}]$ and $Y^\prime=[y^\prime, y^{\prime\prime},
\cdots , y^{n}]$, depending on the order $n$. The syntax of the
macro is

\begin{BDef}
\Lcs{psplotDiffEqn}\OptArgs\Largb{x0}\Largb{x1}\Largb{y0}\Largb{f(x,y,y',...)}
\end{BDef}

\begin{itemize}\setlength\itemsep{0pt}\setlength\parsep{0pt}\setlength\parskip{0pt}
\item \verb+options+: the \verb+\psplotDiffEqn+ specific options and all other of PSTricks, which
make sense;
\item $x_0$: the start value;
\item $x_1$: the end value of the definition interval;
\item $y_0$: the initial values for $y(x_0)\ y'(x_0)\ \ldots$;
\item $f(x,y,y',...)$: the differential equation, depending to the number of initial values, e.g.:
    \verb+{0 1}+ for $y_0$ are two initial values, so that we have a differential equation of
    second order $f(x,y,y')$ and the macro leaves $y\ y'$ on the stack.
\end{itemize}

The new options are:


\begin{itemize}\setlength\itemsep{0pt}\setlength\parsep{0pt}\setlength\parskip{0pt}
\item \Lkeyword{method}: integration method (\verb+euler+ for order 1 euler method, \verb+rk4+ for
  4\textsuperscript{th} order Runge-Kutta method);
\item \Lkeyword{whichabs}: select the abscissa for plotting the graph, by default it is
  $x$, but you can specify a number which represent a position in the vector $y$;
\item \Lkeyword{whichord}: same as precedent for the ordinate, by default $y(0)$;
\item \Lkeyword{plotfuncx}: describe a ps function for the abscissa, parameter
  \Lkeyword{whichabs} becomes useless;
\item \Lkeyword{plotfuncy}: idem for the ordinate;
\item \Lkeyword{buildvector}: boolean parameter for specifying the input-output of the
  $f$ description:
  \begin{description}
  \item[\texttt{true}] (default): $y$ is put on the stack element by element, $y'$
    must be given in the same way;
  \item[\texttt{false}]: $y$ is put on the stack as a vector, $y'$ must be returned
  in the same way;
  \end{description}

\item \Lkeyword{algebraic=true}: algebraic=true description for $f$, \Lkeyword{buildvector}
  parameter is useless when activating this option.
\end{itemize}



\clearpage
\subsection{Variable step for differential equations}

A new algorithm has been added for adjusting the step according to the variations of
the curve. The parameter \Lkeyword{method} has a new possible value : \Lkeyword{varrkiv} to
activate the \Index{Runge-Kutta} method with variable step, then the parameter
\Lkeyword{varsteptol} (real value; \verb+.01+ by default) can control the tolerance of
the algortihm.

\begin{center}
\bgroup
\def\Funct{neg}\def\FunctAlg{-y[0]}
\psset{xunit=1.5, yunit=8, showpoints=true}
\begin{pspicture}[showgrid=true](0,0)(10,1.2)
  \psplot[linewidth=6\pslinewidth, linecolor=green, showpoints=false]{0}{10}{Euler x neg exp}
  \psplotDiffEqn[linecolor=magenta, method=varrkiv, varsteptol=.1, plotpoints=2]{0}{10}{1}{\Funct}
  \rput(0,.0){\psplotDiffEqn[linecolor=blue, method=varrkiv, varsteptol=.01, plotpoints=2]{0}{10}{1}{\Funct}}
  \rput(0,.1){\psplotDiffEqn[linecolor=Orange, method=varrkiv, varsteptol=.001, plotpoints=2]{0}{10}{1}{\Funct}}
  \rput(0,.2){\psplotDiffEqn[linecolor=red, method=varrkiv, varsteptol=.0001, plotpoints=2]{0}{10}{1}{\Funct}}
  \psset{linewidth=4\pslinewidth,showpoints=false}
  \rput*(3.3,.9){\psline[linecolor=magenta](-.75cm,0)}
  \rput*[l](3.3,.9){\small RK ordre 4 : $\varepsilon<10^{-1}$}
  \rput*(3.3,.8){\psline[linecolor=blue](-.75cm,0)}
  \rput*[l](3.3,.8){\small RK ordre 4 : $\varepsilon<10^{-2}$}
  \rput*(3.3,.7){\psline[linecolor=Orange](-.75cm,0)}
  \rput*[l](3.3,.7){\small RK ordre 4 : $\varepsilon<10^{-3}$}
  \rput*(3.3,.6){\psline[linecolor=red](-.75cm,0)}
  \rput*[l](3.3,.6){\small RK ordre 4 : $\varepsilon<10^{-4}$}
  \rput*(3.3,.5){\psline[linecolor=green](-.75cm,0)}
  \rput*[l](3.3,.5){\small solution exacte}
\end{pspicture}
{\captionof{figure}{Equation $y'=-y$ with $y_0=1$.}\label{fig:minusexpvarstep}}
\egroup
\end{center}


\begin{lstlisting}[wide=true]
\def\Funct{neg}\def\FunctAlg{-y[0]}
\psset{xunit=1.5, yunit=8, showpoints=true}
\begin{pspicture}[showgrid=true](0,0)(10,1.2)
  \psplot[linewidth=6\pslinewidth, linecolor=green, showpoints=false]{0}{10}{Euler x neg exp}
  \psplotDiffEqn[linecolor=magenta, method=varrkiv, varsteptol=.1, plotpoints=2]{0}{10}{1}{\Funct}
  \rput(0,.0){\psplotDiffEqn[linecolor=blue, method=varrkiv, varsteptol=.01, plotpoints=2]{0}{10}{1}{\Funct}}
  \rput(0,.1){\psplotDiffEqn[linecolor=Orange, method=varrkiv, varsteptol=.001, plotpoints=2]{0}{10}{1}{\Funct}}
  \rput(0,.2){\psplotDiffEqn[linecolor=red, method=varrkiv, varsteptol=.0001, plotpoints=2]{0}{10}{1}{\Funct}}
  \psset{linewidth=4\pslinewidth,showpoints=false}
  \rput*(3.3,.9){\psline[linecolor=magenta](-.75cm,0)}
  \rput*[l](3.3,.9){\small RK ordre 4 : $\varepsilon<10^{-1}$}
  \rput*(3.3,.8){\psline[linecolor=blue](-.75cm,0)}
  \rput*[l](3.3,.8){\small RK ordre 4 : $\varepsilon<10^{-2}$}
  \rput*(3.3,.7){\psline[linecolor=Orange](-.75cm,0)}
  \rput*[l](3.3,.7){\small RK ordre 4 : $\varepsilon<10^{-3}$}
  \rput*(3.3,.6){\psline[linecolor=red](-.75cm,0)}
  \rput*[l](3.3,.6){\small RK ordre 4 : $\varepsilon<10^{-4}$}
  \rput*(3.3,.5){\psline[linecolor=green](-.75cm,0)}
  \rput*[l](3.3,.5){\small solution exacte}
\end{pspicture}
\end{lstlisting}



\begin{center}
\bgroup
\def\Funct{exch neg}
\psset{xunit=1.5, yunit=5, method=varrkiv, showpoints=true}%%
\def\quatrepi{12.5663706144}
\begin{pspicture}(0,-1)(10,1.3)
  \psaxes{->}(0,0)(0,-1)(10,1.3)
  \psplot[linewidth=4\pslinewidth, linecolor=green, algebraic=true]{0}{10}{cos(x)}
  \rput(0,.0){\psplotDiffEqn[linecolor=magenta, plotpoints=7, varsteptol=.1]{0}{10}{1 0}{\Funct}}
  \rput(0,.0){\psplotDiffEqn[linecolor=blue, plotpoints=201, varsteptol=.01]{0}{10}{1 0}{\Funct}}
  \rput(0,.1){\psplotDiffEqn[linewidth=2\pslinewidth, linecolor=red, varsteptol=.001]{0}{10}{1 0}{\Funct}}
  \rput(0,.2){\psplotDiffEqn[linecolor=black, varsteptol=.0001]{0}{10}{1 0}{\Funct}}
  \rput(0,.3){\psplotDiffEqn[linecolor=Orange, varsteptol=.00001]{0}{10}{1 0}{\Funct}}
  \psset{linewidth=4\pslinewidth,showpoints=false}
  \rput*(2.3,.9){\psline[linecolor=magenta](-.75cm,0)}
  \rput*[l](2.3,.9){\small $\varepsilon<10^{-1}$}
  \rput*(2.3,.8){\psline[linecolor=blue](-.75cm,0)}
  \rput*[l](2.3,.8){\small $\varepsilon<10^{-2}$}
  \rput*(2.3,.7){\psline[linecolor=red](-.75cm,0)}
  \rput*[l](2.3,.7){\small $\varepsilon<10^{-3}$}
  \rput*(2.3,.6){\psline[linecolor=black](-.75cm,0)}
  \rput*[l](2.3,.6){\small $\varepsilon<10^{-4}$}
  \rput*(2.3,.5){\psline[linecolor=Orange](-.75cm,0)}
  \rput*[l](2.3,.5){\small $\varepsilon<10^{-5}$}
  \rput*(2.3,.4){\psline[linecolor=green](-.75cm,0)}
  \rput*[l](2.3,.4){\small solution exacte}
\end{pspicture}
{\captionof{figure}{Equation $y''=-y$}\label{fig:trigfunc}}
\egroup
\end{center}

\begin{lstlisting}[wide=true]
\def\Funct{exch neg}
\psset{xunit=1.5, yunit=5, method=varrkiv, showpoints=true}%%
\def\quatrepi{12.5663706144}
\begin{pspicture}(0,-1)(10,1.3)
  \psaxes{->}(0,0)(0,-1)(10,1.3)
  \psplot[linewidth=4\pslinewidth, linecolor=green, algebraic=true]{0}{10}{cos(x)}
  \rput(0,.0){\psplotDiffEqn[linecolor=magenta, plotpoints=7, varsteptol=.1]{0}{10}{1 0}{\Funct}}
  \rput(0,.0){\psplotDiffEqn[linecolor=blue, plotpoints=201, varsteptol=.01]{0}{10}{1 0}{\Funct}}
  \rput(0,.1){\psplotDiffEqn[linewidth=2\pslinewidth, linecolor=red, varsteptol=.001]{0}{10}{1 0}{\Funct}}
  \rput(0,.2){\psplotDiffEqn[linecolor=black, varsteptol=.0001]{0}{10}{1 0}{\Funct}}
  \rput(0,.3){\psplotDiffEqn[linecolor=Orange, varsteptol=.00001]{0}{10}{1 0}{\Funct}}
  \psset{linewidth=4\pslinewidth,showpoints=false}
  \rput*(2.3,.9){\psline[linecolor=magenta](-.75cm,0)}
  \rput*[l](2.3,.9){\small $\varepsilon<10^{-1}$}
  \rput*(2.3,.8){\psline[linecolor=blue](-.75cm,0)}
  \rput*[l](2.3,.8){\small $\varepsilon<10^{-2}$}
  \rput*(2.3,.7){\psline[linecolor=red](-.75cm,0)}
  \rput*[l](2.3,.7){\small $\varepsilon<10^{-3}$}
  \rput*(2.3,.6){\psline[linecolor=black](-.75cm,0)}
  \rput*[l](2.3,.6){\small $\varepsilon<10^{-4}$}
  \rput*(2.3,.5){\psline[linecolor=Orange](-.75cm,0)}
  \rput*[l](2.3,.5){\small $\varepsilon<10^{-5}$}
  \rput*(2.3,.4){\psline[linecolor=green](-.75cm,0)}
  \rput*[l](2.3,.4){\small solution exacte}
\end{pspicture}
\end{lstlisting}




\begin{center}
\bgroup
\def\Funct{exch}
\psset{xunit=4, yunit=1, method=varrkiv, showpoints=true}%%
\def\quatrepi{12.5663706144}
\begin{pspicture}(0,-0.5)(3,11)
  \psaxes{->}(0,0)(3,11)
  \psplot[linewidth=4\pslinewidth, linecolor=green, algebraic=true]{0}{3}{ch(x)}
  \rput(0,.0){\psplotDiffEqn[linecolor=magenta, varsteptol=.1]{0}{3}{1 0}{\Funct}}
  \rput(0,.3){\psplotDiffEqn[linecolor=blue, varsteptol=.01]{0}{3}{1 0}{\Funct}}
  \rput(0,.6){\psplotDiffEqn[linecolor=red, varsteptol=.001]{0}{3}{1 0}{\Funct}}
  \rput(0,.9){\psplotDiffEqn[linecolor=black, varsteptol=.0001]{0}{3}{1 0}{\Funct}}
  \rput(0,1.2){\psplotDiffEqn[linecolor=Orange, varsteptol=.00001]{0}{3}{1 0}{\Funct}}
  \psset{linewidth=4\pslinewidth,showpoints=false}
  \rput*(2.3,.9){\psline[linecolor=magenta](-.75cm,0)}
  \rput*[l](2.3,.9){\small $\varepsilon<10^{-1}$}
  \rput*(2.3,.8){\psline[linecolor=blue](-.75cm,0)}
  \rput*[l](2.3,.8){\small $\varepsilon<10^{-2}$}
  \rput*(2.3,.7){\psline[linecolor=red](-.75cm,0)}
  \rput*[l](2.3,.7){\small $\varepsilon<10^{-3}$}
  \rput*(2.3,.6){\psline[linecolor=black](-.75cm,0)}
  \rput*[l](2.3,.6){\small $\varepsilon<10^{-4}$}
  \rput*(2.3,.5){\psline[linecolor=Orange](-.75cm,0)}
  \rput*[l](2.3,.5){\small $\varepsilon<10^{-5}$}
  \rput*(2.3,.4){\psline[linecolor=green](-.75cm,0)}
  \rput*[l](2.3,.4){\small solution exacte}
\end{pspicture}
\captionof{figure}{Equation $y''=y$}
\egroup
\end{center}

\begin{lstlisting}[wide=true]
\def\Funct{exch}
\psset{xunit=4, yunit=1, method=varrkiv, showpoints=true}%%
\def\quatrepi{12.5663706144}
\begin{pspicture}(0,-0.5)(3,11)
  \psaxes{->}(0,0)(3,11)
  \psplot[linewidth=4\pslinewidth, linecolor=green, algebraic=true]{0}{3}{ch(x)}
  \rput(0,.0){\psplotDiffEqn[linecolor=magenta, varsteptol=.1]{0}{3}{1 0}{\Funct}}
  \rput(0,.3){\psplotDiffEqn[linecolor=blue, varsteptol=.01]{0}{3}{1 0}{\Funct}}
  \rput(0,.6){\psplotDiffEqn[linecolor=red, varsteptol=.001]{0}{3}{1 0}{\Funct}}
  \rput(0,.9){\psplotDiffEqn[linecolor=black, varsteptol=.0001]{0}{3}{1 0}{\Funct}}
  \rput(0,1.2){\psplotDiffEqn[linecolor=Orange, varsteptol=.00001]{0}{3}{1 0}{\Funct}}
  \psset{linewidth=4\pslinewidth,showpoints=false}
  \rput*(2.3,.9){\psline[linecolor=magenta](-.75cm,0)}
  \rput*[l](2.3,.9){\small $\varepsilon<10^{-1}$}
  \rput*(2.3,.8){\psline[linecolor=blue](-.75cm,0)}
  \rput*[l](2.3,.8){\small $\varepsilon<10^{-2}$}
  \rput*(2.3,.7){\psline[linecolor=red](-.75cm,0)}
  \rput*[l](2.3,.7){\small $\varepsilon<10^{-3}$}
  \rput*(2.3,.6){\psline[linecolor=black](-.75cm,0)}
  \rput*[l](2.3,.6){\small $\varepsilon<10^{-4}$}
  \rput*(2.3,.5){\psline[linecolor=Orange](-.75cm,0)}
  \rput*[l](2.3,.5){\small $\varepsilon<10^{-5}$}
  \rput*(2.3,.4){\psline[linecolor=green](-.75cm,0)}
  \rput*[l](2.3,.4){\small solution exacte}
\end{pspicture}
\end{lstlisting}




\clearpage
\subsection{Equation of second order}

Here is the traditional simulation of two stars attracting each
other according to the classical gravitation law in
$\displaystyle\frac{1}{r^2}$. In 2-Dimensions, the system to be
solved is composed of four second order differential equations. In
order to be described, each of them gives two first order
equations, then we obtain a 8 sized vectorial equation. In the
following example the masses of the stars are 1 and 20.

\[
\left\{
\begin{array}[m]{l}
  x''_1=\displaystyle\frac{M_2}{r^2}\cos(\theta)\\
  y''_1=\displaystyle\frac{M_2}{r^2}\sin(\theta)\\
  x''_2=\displaystyle\frac{M_1}{r^2}\cos(\theta)\\
  y''_2=\displaystyle\frac{M_1}{r^2}\sin(\theta)\\
\end{array}
\right.
\mbox{ avec }
\left\{
\begin{array}[m]{l}
  r^2=(x_1-x_2)^2+(y_1-y_2)^2\\
  \cos(\theta)=\displaystyle\frac{(x_1-x_2)}{r}\\
  \sin(\theta)=\displaystyle\frac{(y_1-y_2)}{r}\\
\end{array}
\right.
\mbox{%
\begin{pspicture}[shift=-2](5,4)\psset{arrowscale=2}
  \psframe[linewidth=.75\pslinewidth](5,4)
  \pstGeonode[PosAngle={-90,90}](1,1){M_1}(4,3){M_2}
  \pstHomO[HomCoef=.33, PointSymbol=none]{M_1}{M_2}[F_1]
  \psline[arrows=->](M_1)(F_1)
  \pstHomO[HomCoef=.33, PointSymbol=none]{M_2}{M_1}[F_2]
  \psline[arrows=->, arrowscale=2](M_2)(F_2)
  \pstGeonode[PointSymbol=none, PointName=none](M_2|M_1){A}
  \psline[linewidth=.5\pslinewidth](M_1)(A)
  \pstMarkAngle{A}{M_1}{M_2}{$\theta$}
  \ncline[linewidth=.5\pslinewidth, offset=.5, arrows=<->]{M_1}{M_2}
  \ncput*{$r$}
\end{pspicture}}
\]

\begin{table}[!htbp]
  \centering\small
    \begin{tabular}{|l@{}>{\ttfamily}l@{}>{ \ttfamily \%\% }l|}
      \hline
      && x1 y1 x'1 y'1 x2 y2 x'2 y'2\\
      &/yp2 exch def /xp2 exch def /ay2 exch def /ax2 exch def&mise en variables\\
      &/yp1 exch def /xp1 exch def /ay1 exch def /ax1 exch def&mise en variables\\
      &/ro2 ax2 ax1 sub dup mul ay2 ay1 sub dup mul add def&calcul de r*r\\
      &xp1 yp1&\\
      &ax2 ax1 sub ro2 sqrt div ro2 div&calcul de x''1\\
      &ay2 ay1 sub ro2 sqrt div ro2 div&calcul de y''1\\
      &xp2 yp2&\\
      &3 index -20 mul&calcul de x''2=-20x''1\\
      &3 index -20 mul&calcul de y''2=-20y''1\\
      \hline
    \end{tabular}
    \caption{\PS source code for the gravitational interaction}\label{intgravcode}
\end{table}

\begin{table}[!htbp]
  \centering
    \small\newcommand{\POW}{\symbol{'136}}
    \begin{tabular}{|l@{}>{\ttfamily}l@{}>{ \ttfamily \%\% }l|}
      \hline
      &y[2]|&y'[0]\\
      &y[3]|&y'[1]\\
      &(y[4]-y[0])/((y[4]-y[0])\POW 2+(y[5]-y[1])\POW 2)\POW 1.5|&y'[2]=y''[0]\\
      &(y[5]-y[1])/((y[4]-y[0])\POW 2+(y[5]-y[1])\POW 2)\POW 1.5|&y'[3]=y''[1]\\
      &y[6]|&y'[4]\\
      &y[7]|&y'[5]\\
      &20*(y[0]-y[4])/((y[4]-y[0])\POW 2+(y[5]-y[1])\POW 2)\POW 1.5|&y'[6]=y''[4]\\
      &20*(y[1]-y[5])/((y[4]-y[0])\POW 2+(y[5]-y[1])\POW 2)\POW 1.5&y'[7]=y''[5]\\
      \hline
    \end{tabular}
    \caption{Algebraic description for the gravitational interaction}\label{intgravalgcode}
\end{table}

\newcommand\Grav{%
  /yp2 exch def /xp2 exch def /ay2 exch def /ax2 exch def
  /yp1 exch def /xp1 exch def /ay1 exch def /ax1 exch def
  /ro2 ax2 ax1 sub dup mul ay2 ay1 sub dup mul add def
  xp1 yp1
  ax2 ax1 sub ro2 sqrt div ro2 div
  ay2 ay1 sub ro2 sqrt div ro2 div
  xp2 yp2
  3 index -20 mul
  3 index -20 mul}
\newcommand\GravAlg{%
  y[2]|y[3]|%
  (y[4]-y[0])/((y[4]-y[0])^2+(y[5]-y[1])^2)^1.5|%
  (y[5]-y[1])/((y[4]-y[0])^2+(y[5]-y[1])^2)^1.5|%
  y[6]|y[7]|%
  20*(y[0]-y[4])/((y[4]-y[0])^2+(y[5]-y[1])^2)^1.5|%
  20*(y[1]-y[5])/((y[4]-y[0])^2+(y[5]-y[1])^2)^1.5}
%%  0  1   2   3  4  5   6   7
%% x1 y1 x'1 y'1 x2 y2 x'2 y'2


\begin{LTXexample}[width=5cm,wide]
\def\InitCond{ 1  1  .1  0 -1 -1  -2   0}
\begin{pspicture}[shift=-2,showgrid=true](-3,-1.75)(2,1.5)
  \psplotDiffEqn[whichabs=0, whichord=1, linecolor=blue, method=rk4, plotpoints=100]{0}{3.95}{\InitCond}{\Grav}
  \psset{showpoints=true,whichabs=4, whichord=5}
  \psplotDiffEqn[linecolor=black, method=varrkiv, varsteptol=.0001, plotpoints=200]{0}{3.9}{\InitCond}{\Grav}
\end{pspicture}
\end{LTXexample}
\vspace{-2ex}
{\captionof{figure}{Gravitational interaction: fixed landmark, trajectory of the stars}\label{fig:InterGravRepFix}}



\bigskip
\begin{LTXexample}[width=5cm,wide]
\def\InitCond{ 1  1  .1  0 -1 -1  -2   0}
\begin{pspicture}[shift=-1.5,showgrid=true](-4,-1.75)(1,1)
  \psplotDiffEqn[linecolor=red, plotpoints=200,method=varrkiv, varsteptol=.0001, showpoints=true,
      plotfuncx=y dup 4 get exch 0 get sub,
      plotfuncy=dup 5 get exch 1 get sub ]{0}{3.9}{\InitCond}{\Grav}
\end{pspicture}
\end{LTXexample}
\vspace{-2ex}
{\captionof{figure}{Gravitational interaction : landmark defined by one star}\label{fig:IGnewrep}}


\begin{center}
\bgroup
\def\InitCond{ 1  1  .1   0 -1 -1  -2   0}
\psset{xunit=2}
\begin{pspicture}[showgrid=true](0,0)(8,9)
  \psset{showpoints=true}
  \psplotDiffEqn[linecolor=red, method=varrkiv, plotpoints=2, varsteptol=.0001,
      plotfuncy=dup 6 get dup mul exch 7 get dup mul add sqrt]{0}{8}{\InitCond}{\Grav}
  \psplotDiffEqn[linecolor=blue, method=varrkiv, plotpoints=2, varsteptol=.0001,
      plotfuncy=dup 2 get dup mul exch 3 get dup mul add sqrt]{0}{8}{\InitCond}{\Grav}
\end{pspicture}
\captionof{figure}{Gravitational interaction : speeds of the
stars} \egroup
\end{center}

\begin{lstlisting}
\psset{xunit=2}
\begin{pspicture}[showgrid=true](0,0)(8,9)
  \psset{showpoints=true}
  \psplotDiffEqn[linecolor=red, method=varrkiv, plotpoints=2, varsteptol=.0001,
      plotfuncy=dup 6 get dup mul exch 7 get dup mul add sqrt]{0}{8}{\InitCond}{\Grav}
  \psplotDiffEqn[linecolor=blue, method=varrkiv, plotpoints=2, varsteptol=.0001,
      plotfuncy=dup 2 get dup mul exch 3 get dup mul add sqrt]{0}{8}{\InitCond}{\Grav}
\end{pspicture}
\end{lstlisting}

%--------------------------------------------------------------------------------------
\clearpage
\subsubsection{Simple equation of first order $y'=y$}
%--------------------------------------------------------------------------------------

For the initial value $y(0)=1$ we have the solution $y(x)=e^x$. $y$ is always
on the stack, so we have to do nothing. Using the \Lkeyword{algebraic=true} option, we write it
as \verb$y[0]$. The following example shows different solutions depending to the number of plotpoints
with $y_0=1$:


\begin{center}
\bgroup
\psset{xunit=4, yunit=.4}
\begin{pspicture}(3,19)\psgrid[subgriddiv=1]
  \psplot[linewidth=6\pslinewidth, linecolor=green]{0}{3}{Euler x exp}
  \psplotDiffEqn[linecolor=magenta,plotpoints=16,algebraic=true]{0}{3}{1}{y[0]}
  \psplotDiffEqn[linecolor=blue,plotpoints=151]{0}{3}{1}{}
  \psplotDiffEqn[linecolor=red,method=rk4,plotpoints=15]{0}{3}{1}{}
  \psplotDiffEqn[linecolor=Orange,method=rk4,plotpoints=4]{0}{3}{1}{}
  \psset{linewidth=4\pslinewidth}
  \rput*(0.35,19){\psline[linecolor=magenta](-.75cm,0)}
  \rput*[l](0.35,19){\small Euler order 1 $h=0{,}2$}
  \rput*(0.35,17){\psline[linecolor=blue](-.75cm,0)}
  \rput*[l](0.35,17){\small Euler order 1 $h=0{,}02$}
  \rput*(0.35,15){\psline[linecolor=Orange](-.75cm,0)}
  \rput*[l](0.35,15){\small RK ordre 4 $h=1$}
  \rput*(0.35,13){\psline[linecolor=red](-.75cm,0)}
  \rput*[l](0.35,13){\small RK ordre 4 $h=0{,}2$}
  \rput*(0.35,11){\psline[linecolor=green](-.75cm,0)}
  \rput*[l](0.35,11){\small solution exacte}
\end{pspicture}
\egroup
\end{center}

\begin{lstlisting}
\psset{xunit=4, yunit=.4}
\begin{pspicture}(3,19)\psgrid[subgriddiv=1]
  \psplot[linewidth=6\pslinewidth, linecolor=green]{0}{3}{Euler x exp}
  \psplotDiffEqn[linecolor=magenta,plotpoints=16,algebraic=true]{0}{3}{1}{y[0]}
  \psplotDiffEqn[linecolor=blue,plotpoints=151]{0}{3}{1}{}
  \psplotDiffEqn[linecolor=red,method=rk4,plotpoints=15]{0}{3}{1}{}
  \psplotDiffEqn[linecolor=Orange,method=rk4,plotpoints=4]{0}{3}{1}{}
  \psset{linewidth=4\pslinewidth}
  \rput*(0.35,19){\psline[linecolor=magenta](-.75cm,0)}
  \rput*[l](0.35,19){\small Euler order 1 $h=0{,}2$}
  \rput*(0.35,17){\psline[linecolor=blue](-.75cm,0)}
  \rput*[l](0.35,17){\small Euler order 1 $h=0{,}02$}
  \rput*(0.35,15){\psline[linecolor=Orange](-.75cm,0)}
  \rput*[l](0.35,15){\small RK ordre 4 $h=1$}
  \rput*(0.35,13){\psline[linecolor=red](-.75cm,0)}
  \rput*[l](0.35,13){\small RK ordre 4 $h=0{,}2$}
  \rput*(0.35,11){\psline[linecolor=green](-.75cm,0)}
  \rput*[l](0.35,11){\small solution exacte}
\end{pspicture}
\end{lstlisting}

%--------------------------------------------------------------------------------------
\clearpage
\subsubsection{$y'=\displaystyle\frac{2-ty}{4-t^2}$}% $
%--------------------------------------------------------------------------------------

For the initial value $y(0)=1$ the exact solution is
$y(x)=\displaystyle\frac{t+\sqrt{4-t^2}}{2}$. The function $f$
described in PostScript code is like (y is still on the stack):
\begin{lstlisting}[style=syntax]
x              %% y x
mul            %% x*y
2 exch sub     %% 2-x*y
4 x dup mul    %% 2-x*y 4 x^2
sub            %% 2-x*y 4-x^2
div            %% (2-x*y)/(4-x^2)
\end{lstlisting}
\noindent
The following example uses $y_0=1$.

\begin{lstlisting}[style=syntax]
\newcommand{\InitCond}{1}
\newcommand{\Func}{x mul 2 exch sub 4 x dup mul sub div}
\newcommand{\FuncAlg}{(2-x*y[0])/(4-x^2)}
\end{lstlisting}

\begin{center}
\bgroup
\psset{xunit=6.4, yunit=9.6, showpoints=false}
\begin{pspicture}(0,1)(2,1.5)  \psgrid[griddots=10](0,1)(2,1.5)
  { \psset{linewidth=4\pslinewidth,linecolor=lightgray}
  \psplot{0}{1.8}{x dup dup mul 4 exch sub sqrt add 2 div}
  \psplot{1.8}{2}{x dup dup mul 4 exch sub sqrt add 2 div} }
  \def\InitCond{1}
  \def\Func{x mul 2 exch sub 4 x dup mul sub div}
  \psplotDiffEqn[linecolor=magenta, plotpoints=20]{0}{1.9}{\InitCond}{\Func}
  \psplotDiffEqn[linecolor=blue, plotpoints=191]{0}{1.9}{\InitCond}{\Func}
  \psplotDiffEqn[linecolor=red, method=rk4, plotpoints=11,%
     algebraic=true]{0}{1.9}{\InitCond}{(2-x*y[0])/(4-x^2)}
  \psplotDiffEqn[linecolor=Orange, method=rk4, plotpoints=21,%
     algebraic=true]{0}{1.9}{\InitCond}{(2-x*y[0])/(4-x^2)}
  \psset{linewidth=4\pslinewidth}\small
  \rput*(0,1.4){\psline[linecolor=magenta](-.75cm,0)}\rput*[l](0,1.4){Euler order 1 $h=0{,}1$}
  \rput*(0,1.35){\psline[linecolor=blue](-.75cm,0)}\rput*[l](0,1.35){Euler order 1 $h=0{,}01$}
  \rput*(0,1.3){\psline[linecolor=Orange](-.75cm,0)}\rput*[l](0,1.3){RK order 4 $h=0{,}19$}
  \rput*(0,1.25){\psline[linecolor=red](-.75cm,0)}\rput*[l](0,1.25){RK order 4 $h=0{,}095$}
  \rput*(0,1.2){\psline[linecolor=lightgray](-.75cm,0)}\rput*[l](0,1.2){exactly}
\end{pspicture}
\egroup
\end{center}

\begin{lstlisting}[xrightmargin=-1cm,xleftmargin=-1cm]
\psset{xunit=6.4, yunit=9.6, showpoints=false}
\begin{pspicture}(0,1)(2,1.7)  \psgrid[subgriddiv=5]
  { \psset{linewidth=4\pslinewidth,linecolor=lightgray}
  \psplot{0}{1.8}{x dup dup mul 4 exch sub sqrt add 2 div}
  \psplot{1.8}{2}{x dup dup mul 4 exch sub sqrt add 2 div} }
  \def\InitCond{1}
  \def\Func{x mul 2 exch sub 4 x dup mul sub div}
  \psplotDiffEqn[linecolor=magenta, plotpoints=20]{0}{1.9}{\InitCond}{\Func}
  \psplotDiffEqn[linecolor=blue, plotpoints=191]{0}{1.9}{\InitCond}{\Func}
  \psplotDiffEqn[linecolor=red, method=rk4, plotpoints=11,%
     algebraic=true]{0}{1.9}{\InitCond}{(2-x*y[0])/(4-x^2)}
  \psplotDiffEqn[linecolor=Orange, method=rk4, plotpoints=21,%
     algebraic=true]{0}{1.9}{\InitCond}{(2-x*y[0])/(4-x^2)}
  \psset{linewidth=4\pslinewidth}
  \rput*(0.3,1.6){\psline[linecolor=magenta](-.75cm,0)}\rput*[l](0.3,1.6){\small Euler order 1 $h=0{,}1$}
  \rput*(0.3,1.55){\psline[linecolor=blue](-.75cm,0)}\rput*[l](0.3,1.55){\small Euler order 1 $h=0{,}01$}
  \rput*(0.3,1.5){\psline[linecolor=Orange](-.75cm,0)}\rput*[l](0.3,1.5){\small RK order 4 $h=0{,}19$}
  \rput*(0.3,1.45){\psline[linecolor=red](-.75cm,0)}\rput*[l](0.3,1.45){\small RK order 4 $h=0{,}095$}
  \rput*(0.3,1.4){\psline[linecolor=lightgray](-.75cm,0)}\rput*[l](0.3,1.4){\small exactly}
\end{pspicture}
\end{lstlisting}


%--------------------------------------------------------------------------------------
\clearpage
\subsubsection{$y'=-2xy$}
%--------------------------------------------------------------------------------------

For $y(-1)=\frac{1}{e}$ we get $y(x)=e^{-x^2}$.

\begin{center}
\bgroup
\psset{unit=4}
\begin{pspicture}(-1,0)(3,1.1)\psgrid
  \psplot[linewidth=4\pslinewidth,linecolor=gray]{-1}{3}{Euler x dup mul neg exp}
  \psset{plotpoints=9}
  \psplotDiffEqn[linecolor=cyan]{-1}{3}{1 Euler div}{x -2 mul mul}
  \psplotDiffEqn[linecolor=yellow, method=rk4]{-1}{3}{1 Euler div}{x -2 mul mul}
  \psset{plotpoints=21}
  \psplotDiffEqn[linecolor=blue]{-1}{3}{1 Euler div}{x -2 mul mul}
  \psplotDiffEqn[linecolor=Orange, method=rk4]{-1}{3}{1 Euler div}{x -2 mul mul}
  \psset{linewidth=2\pslinewidth}
  \rput*(2,1){\psline[linecolor=Orange](-0.25,0)}
  \rput*[l](2,1){RK}
  \rput*(2,.9){\psline[linecolor=blue](-0.25,0)}
  \rput*[l](2,.9){\textsc{Euler}-1}
  \rput*(2,.8){\psline[linecolor=gray](-0.25,0)}
  \rput*[l](2,.8){solution}
\end{pspicture}
\egroup
\end{center}


\begin{lstlisting}
\psset{unit=4}
\begin{pspicture}(-1,0)(3,1.1)\psgrid
  \psplot[linewidth=4\pslinewidth,linecolor=gray]{-1}{3}{Euler x dup mul neg exp}
  \psset{plotpoints=9}
  \psplotDiffEqn[linecolor=cyan]{-1}{3}{1 Euler div}{x -2 mul mul}
  \psplotDiffEqn[linecolor=yellow, method=rk4]{-1}{3}{1 Euler div}{x -2 mul mul}
  \psset{plotpoints=21}
  \psplotDiffEqn[linecolor=blue]{-1}{3}{1 Euler div}{x -2 mul mul}
  \psplotDiffEqn[linecolor=Orange, method=rk4]{-1}{3}{1 Euler div}{x -2 mul mul}
  \psset{linewidth=2\pslinewidth}
  \rput*(2,1){\psline[linecolor=Orange](-0.25,0)}
  \rput*[l](2,1){RK}
  \rput*(2,.9){\psline[linecolor=blue](-0.25,0)}
  \rput*[l](2,.9){\textsc{Euler}-1}
  \rput*(2,.8){\psline[linecolor=gray](-0.25,0)}
  \rput*[l](2,.8){solution}
\end{pspicture}
\end{lstlisting}


%--------------------------------------------------------------------------------------
\clearpage
\subsubsection{Spiral of Cornu}
%--------------------------------------------------------------------------------------

The integrals of \Index{Fresnel}:
\begin{align} x & =\int^t_0\cos\frac{\pi t^2}{2}\mathrm{d}t \\
 y & =\int^t_0\sin\frac{\pi t^2}{2}\mathrm{d}t \\
\intertext{with}
 \dot{x} &= \cos\frac{\pi t^2}{2} \\
 \dot{y} & =\sin\frac{\pi t^2}{2}
 \end{align}

\begin{lstlisting}
\psset{unit=8}
\begin{pspicture}(1,1)\psgrid[subgriddiv=5]
  \psplotDiffEqn[whichabs=0,whichord=1,linecolor=red,method=rk4,algebraic=true,%
     plotpoints=500,showpoints=true]{0}{10}{0 0}{cos(Pi*x^2/2)|sin(Pi*x^2/2)}
\end{pspicture}
\end{lstlisting}


\begin{center}
\bgroup
\psset{unit=8}
\begin{pspicture}(1,1)\psgrid[subgriddiv=5]
  \psplotDiffEqn[whichabs=0,whichord=1,linecolor=red,method=rk4,algebraic=true,%
     plotpoints=500,showpoints=true]{0}{10}{0 0}{cos(Pi*x^2/2)|sin(Pi*x^2/2)}
\end{pspicture}
\egroup
\end{center}



%--------------------------------------------------------------------------------------
\clearpage
\subsubsection{Lotka-Volterra}
%--------------------------------------------------------------------------------------

The Lotka-Volterra model describes interactions between two species in an ecosystem, a 
predator and a prey. This represents our first multi-species model. Since we are considering 
two species, the model will involve two equations, one which describes how the prey 
population changes and the second which describes how the predator population changes.

For concreteness let us assume that the prey in our model are rabbits, and that the 
predators are foxes. If we let $R(t)$ and $F(t)$ represent the number of rabbits and 
foxes, respectively, that are alive at time t, then the Lotka-Volterra model is:
%
\begin{align}
\dot R &= a\cdot R - b\cdot R\cdot F\\
\dot F &= e\cdot b\cdot R\cdot F - c\cdot F
\end{align}
%
where the parameters are defined by:
\begin{description}
\item[a] is the natural growth rate of rabbits in the absence of predation,
\item[c] is the natural death rate of foxes in the absence of food (rabbits),
\item[b] is the death rate per encounter of rabbits due to predation,
\item[e] is the efficiency of turning predated rabbits into foxes.
\end{description}

The Stella model representing the \Index{Lotka-Volterra} model will be slightly more complex than the 
single species models we've dealt with before. The main difference is that our model will have 
two stocks (reservoirs), one for each species. Each species will have its own birth and death 
rates. In addition, the Lotka-Volterra model involves four parameters rather than two. All told, 
the Stella representation of the Lotka-Volterra model will use two stocks, four flows, four 
converters and many connectors.

\bgroup
\begin{center}
\def\InitCond{ 0 10 10}%% xa ya xl
\def\Faiglelapin{\Vaigle*(y[2]-y[0])/sqrt(y[1]^2+(y[2]-y[0])^2)|%
                 -\Vaigle*y[1]/sqrt(y[1]^2+(y[2]-y[0])^2)|%
                 -\Vlapin}
\def\Vlapin{1}  \def\Vaigle{1.6}
\psset{unit=.7,subgriddiv=0,gridcolor=lightgray,method=adams,algebraic=true,%
   plotpoints=20,showpoints=true}
\begin{pspicture}[showgrid=true](-3,-3)(10,10)
 \psplotDiffEqn[plotfuncy=pop 0,whichabs=2,linecolor=red]{0}{10}{\InitCond}{\Faiglelapin}
 \psplotDiffEqn[whichabs=0,whichord=1,linecolor=black,method=rk4]{0}{10}{\InitCond}{\Faiglelapin}
  \psplotDiffEqn[whichabs=0,whichord=1,linecolor=blue]{0}{10}{\InitCond}{\Faiglelapin}
\end{pspicture}
\end{center}

\begin{lstlisting}[label={fig:aiglelapin},xrightmargin=-1.5cm]
\def\InitCond{ 0 10 10}%% xa ya xl
\def\Faiglelapin{\Vaigle*(y[2]-y[0])/sqrt(y[1]^2+(y[2]-y[0])^2)|%
                 -\Vaigle*y[1]/sqrt(y[1]^2+(y[2]-y[0])^2)|%
                 -\Vlapin}
\def\Vlapin{1}  \def\Vaigle{1.6}
\psset{unit=.7,subgriddiv=0,gridcolor=lightgray,method=adams,algebraic=true,%
   plotpoints=20,showpoints=true}
\begin{pspicture}[showgrid=true](-3,-3)(10,10)
 \psplotDiffEqn[plotfuncy=pop 0,whichabs=2,linecolor=red]{0}{10}{\InitCond}{\Faiglelapin}
 \psplotDiffEqn[whichabs=0,whichord=1,linecolor=black,method=rk4]{0}{10}{\InitCond}{\Faiglelapin}
  \psplotDiffEqn[whichabs=0,whichord=1,linecolor=blue]{0}{10}{\InitCond}{\Faiglelapin}
\end{pspicture}
\end{lstlisting}


\begin{center}
\def\InitCond{ 0 10 10}%% xa ya xl
\def\Faiglelapin{\Vaigle*(y[2]-y[0])/sqrt(y[1]^2+(y[2]-y[0])^2)|%
                 -\Vaigle*y[1]/sqrt(y[1]^2+(y[2]-y[0])^2)|%
                 -\Vlapin}
\def\Vlapin{1}  \def\Vaigle{1.6}
\psset{unit=.7,subgriddiv=0,gridcolor=lightgray,method=adams,algebraic=true,%
   plotpoints=20,showpoints=true}
\begin{pspicture}[showgrid=true](0,-0.25)(10,14)
 \psplotDiffEqn[plotfuncy=dup 1 get dup mul exch dup 0 get exch 2 get sub dup
    mul add sqrt,linecolor=red,method=rk4]{0}{10}{\InitCond}{\Faiglelapin}
 \psplotDiffEqn[plotfuncy=dup 1 get dup mul exch dup 0 get exch 2 get sub dup
    mul add sqrt,linecolor=blue]{0}{10}{\InitCond}{\Faiglelapin}
 \psplotDiffEqn[plotfuncy=pop Func aload pop pop dup mul exch dup mul add sqrt,
    linecolor=yellow]{0}{10}{\InitCond}{\Faiglelapin}
\end{pspicture}
\end{center}
\egroup

\begin{lstlisting}[label={fig:aiglelapin},xrightmargin=-1.5cm]
\def\InitCond{ 0 10 10}%% xa ya xl
\def\Faiglelapin{\Vaigle*(y[2]-y[0])/sqrt(y[1]^2+(y[2]-y[0])^2)|%
                 -\Vaigle*y[1]/sqrt(y[1]^2+(y[2]-y[0])^2)|%
                 -\Vlapin}
\def\Vlapin{1}  \def\Vaigle{1.6}
\psset{unit=.7,subgriddiv=0,gridcolor=lightgray,method=adams,algebraic=true,%
   plotpoints=20,showpoints=true}
\begin{pspicture}[showgrid=true](10,12)
 \psplotDiffEqn[plotfuncy=dup 1 get dup mul exch dup 0 get exch 2 get sub dup
    mul add sqrt,linecolor=red,method=rk4]{0}{10}{\InitCond}{\Faiglelapin}
 \psplotDiffEqn[plotfuncy=dup 1 get dup mul exch dup 0 get exch 2 get sub dup
    mul add sqrt,linecolor=blue]{0}{10}{\InitCond}{\Faiglelapin}
 \psplotDiffEqn[plotfuncy=pop Func aload pop pop dup mul exch dup mul add sqrt,
    linecolor=yellow]{0}{10}{\InitCond}{\Faiglelapin}
\end{pspicture}
\end{lstlisting}


%--------------------------------------------------------------------------------------
\subsubsection{$y''=y$}
%--------------------------------------------------------------------------------------

Beginning with the initial equation $\displaystyle y(x)=Ae^x+Be^{-x}$ we get the hyperbolic
trigonometrical functions.

\begin{center}
\bgroup
\def\Funct{exch}   \psset{xunit=5cm, yunit=0.75cm}
\begin{pspicture}(0,-0.25)(2,7)\psgrid[subgriddiv=1,griddots=10]
 \psplot[linewidth=4\pslinewidth, linecolor=green]{0}{2}{Euler x exp}  %%e^x
 \psplotDiffEqn[linecolor=magenta, plotpoints=11]{0}{2}{1 1}{\Funct}
 \psplotDiffEqn[linecolor=blue, plotpoints=101]{0}{2}{1 1}{\Funct}
 \psplotDiffEqn[linecolor=red, method=rk4, plotpoints=11]{0}{2}{1 1}{\Funct}
 \psplot[linewidth=4\pslinewidth, linecolor=green]{0}{2}{Euler dup x exp  %%ch(x)
    exch x neg exp add 2 div}
 \psplotDiffEqn[linecolor=magenta, plotpoints=11]{0}{2}{1 0}{\Funct}
 \psplotDiffEqn[linecolor=blue, plotpoints=101]{0}{2}{1 0}{\Funct}
 \psplotDiffEqn[linecolor=red, method=rk4, plotpoints=11]{0}{2}{1 0}{\Funct}
 \psplot[linewidth=4\pslinewidth, linecolor=green]{0}{2}{Euler dup x exp
     exch x neg exp sub 2 div}  %%sh(x)
 \psplotDiffEqn[linecolor=magenta, plotpoints=11]{0}{2}{0 1}{\Funct}
 \psplotDiffEqn[linecolor=blue, plotpoints=101]{0}{2}{0 1}{\Funct}
 \psplotDiffEqn[linecolor=red, method=rk4, plotpoints=11]{0}{2}{0 1}{\Funct}
 \rput*(1.3,.9){\psline[linecolor=magenta](-.75cm,0)}\rput*[l](1.3,.9){\small\textsc{Euler} order 1 $h=1$}
 \rput*(1.3,.8){\psline[linecolor=blue](-.75cm,0)}\rput*[l](1.3,.8){\small\textsc{Euler} order 1 $h=0{,}1$}
 \rput*(1.3,.7){\psline[linecolor=red](-.75cm,0)}\rput*[l](1.3,.7){\small RK order 4 $h=1$}
 \rput*(1.3,.6){\psline[linecolor=green](-.75cm,0)}\rput*[l](1.3,.6){\small exact solution}
\end{pspicture}
\egroup
\end{center}

\begin{lstlisting}[label={fig:minusexp},xrightmargin=-1.5cm]
\def\Funct{exch}   \psset{xunit=5cm, yunit=0.75cm}
\begin{pspicture}(0,-0.25)(2,7)\psgrid[subgriddiv=1,griddots=10]
 \psplot[linewidth=4\pslinewidth, linecolor=green]{0}{2}{Euler x exp}  %%e^x
 \psplotDiffEqn[linecolor=magenta, plotpoints=11]{0}{2}{1 1}{\Funct}
 \psplotDiffEqn[linecolor=blue, plotpoints=101]{0}{2}{1 1}{\Funct}
 \psplotDiffEqn[linecolor=red, method=rk4, plotpoints=11]{0}{2}{1 1}{\Funct}
 \psplot[linewidth=4\pslinewidth, linecolor=green]{0}{2}{Euler dup x exp  %%ch(x)
    exch x neg exp add 2 div}
 \psplotDiffEqn[linecolor=magenta, plotpoints=11]{0}{2}{1 0}{\Funct}
 \psplotDiffEqn[linecolor=blue, plotpoints=101]{0}{2}{1 0}{\Funct}
 \psplotDiffEqn[linecolor=red, method=rk4, plotpoints=11]{0}{2}{1 0}{\Funct}
 \psplot[linewidth=4\pslinewidth, linecolor=green]{0}{2}{Euler dup x exp
     exch x neg exp sub 2 div}  %%sh(x)
 \psplotDiffEqn[linecolor=magenta, plotpoints=11]{0}{2}{0 1}{\Funct}
 \psplotDiffEqn[linecolor=blue, plotpoints=101]{0}{2}{0 1}{\Funct}
 \psplotDiffEqn[linecolor=red, method=rk4, plotpoints=11]{0}{2}{0 1}{\Funct}
 \rput*(1.3,.9){\psline[linecolor=magenta](-.75cm,0)}\rput*[l](1.3,.9){\small\textsc{Euler} order 1 $h=1$}
 \rput*(1.3,.8){\psline[linecolor=blue](-.75cm,0)}\rput*[l](1.3,.8){\small\textsc{Euler} order 1 $h=0{,}1$}
 \rput*(1.3,.7){\psline[linecolor=red](-.75cm,0)}\rput*[l](1.3,.7){\small RK order 4 $h=1$}
 \rput*(1.3,.6){\psline[linecolor=green](-.75cm,0)}\rput*[l](1.3,.6){\small exact solution}
\end{pspicture}
\end{lstlisting}

%--------------------------------------------------------------------------------------
\clearpage
\subsubsection{$y''=-y$}
%--------------------------------------------------------------------------------------
\begin{center}
\bgroup
\def\Funct{exch neg}
\psset{xunit=1, yunit=4}
\def\quatrepi{12.5663706144}%%4pi=12.5663706144
\begin{pspicture}(0,-1.25)(\quatrepi,1.25)\psgrid[subgriddiv=1,griddots=10]
 \psplot[linewidth=4\pslinewidth,linecolor=green]{0}{\quatrepi}{x RadtoDeg cos}%%cos(x)
 \psplotDiffEqn[linecolor=blue, plotpoints=201]{0}{3.1415926}{1 0}{\Funct}
 \psplotDiffEqn[linecolor=red, method=rk4, plotpoints=31]{0}{\quatrepi}{1 0}{\Funct}
 \psplot[linewidth=4\pslinewidth,linecolor=green]{0}{\quatrepi}{x RadtoDeg sin}  %%sin(x)
 \psplotDiffEqn[linecolor=blue,plotpoints=201]{0}{3.1415926}{0 1}{\Funct}
 \psplotDiffEqn[linecolor=red,method=rk4, plotpoints=31]{0}{\quatrepi}{0 1}{\Funct}
 \rput*(3.3,.9){\psline[linecolor=magenta](-.75cm,0)}\rput*[l](3.3,.9){\small Euler order 1 $h=1$}
 \rput*(3.3,.8){\psline[linecolor=blue](-.75cm,0)}\rput*[l](3.3,.8){\small Euler order 1 $h=0{,}1$}
 \rput*(3.3,.7){\psline[linecolor=red](-.75cm,0)}\rput*[l](3.3,.7){\small RK order 4 $h=1$}
 \rput*(3.3,.6){\psline[linecolor=green](-.75cm,0)}\rput*[l](3.3,.6){\small exact solution}
\end{pspicture}
\egroup
\end{center}

\begin{lstlisting}[label={fig:minusexp2}]
\def\Funct{exch neg}
\psset{xunit=1, yunit=4}
\def\quatrepi{12.5663706144}%%4pi=12.5663706144
\begin{pspicture}(0,-1.25)(\quatrepi,1.25)\psgrid[subgriddiv=1,griddots=10]
 \psplot[linewidth=4\pslinewidth,linecolor=green]{0}{\quatrepi}{x RadtoDeg cos}%%cos(x)
 \psplotDiffEqn[linecolor=blue, plotpoints=201]{0}{3.1415926}{1 0}{\Funct}
 \psplotDiffEqn[linecolor=red, method=rk4, plotpoints=31]{0}{\quatrepi}{1 0}{\Funct}
 \psplot[linewidth=4\pslinewidth,linecolor=green]{0}{\quatrepi}{x RadtoDeg sin}  %%sin(x)
 \psplotDiffEqn[linecolor=blue,plotpoints=201]{0}{3.1415926}{0 1}{\Funct}
 \psplotDiffEqn[linecolor=red,method=rk4, plotpoints=31]{0}{\quatrepi}{0 1}{\Funct}
 \rput*(3.3,.9){\psline[linecolor=magenta](-.75cm,0)}\rput*[l](3.3,.9){\small Euler order 1 $h=1$}
 \rput*(3.3,.8){\psline[linecolor=blue](-.75cm,0)}\rput*[l](3.3,.8){\small Euler order 1 $h=0{,}1$}
 \rput*(3.3,.7){\psline[linecolor=red](-.75cm,0)}\rput*[l](3.3,.7){\small RK order 4 $h=1$}
 \rput*(3.3,.6){\psline[linecolor=green](-.75cm,0)}\rput*[l](3.3,.6){\small exact solution}
\end{pspicture}
\end{lstlisting}

%--------------------------------------------------------------------------------------
\clearpage
\subsubsection{The mechanical pendulum: $y''=-\frac{g}{l}\sin(y)$}% $
%--------------------------------------------------------------------------------------

For small \Index{oscillation}s $\sin(y)\simeq y$:

\[ y(x)=y_0\cos\left(\sqrt{\frac{g}{l}}x\right) \]

The function $f$ is written in PostScript code:

\begin{lstlisting}[style=syntax]
exch RadtoDeg sin -9.8 mul %% y' -gsin(y)
\end{lstlisting}

\begin{center}
\bgroup
\def\Func{y[1]|-9.8*sin(y[0])}
\psset{yunit=2,xunit=4,algebraic=true,linewidth=1.5pt}
\begin{pspicture}(0,-2.25)(3,2.25)
  \psaxes{->}(0,0)(0,-2)(3,2)
  \psplot[linewidth=3\pslinewidth, linecolor=Orange]{0}{3}{.1*cos(sqrt(9.8)*x)}
  \psset{method=rk4,plotpoints=50,linecolor=blue}
  \psplotDiffEqn{0}{3}{.1 0}{\Func}
  \psplot[linewidth=3\pslinewidth,linecolor=Orange]{0}{3}{.25*cos(sqrt(9.8)*x)}
  \psplotDiffEqn{0}{3}{.25 0}{\Func}
  \psplotDiffEqn{0}{3}{.5 0}{\Func}
  \psplotDiffEqn{0}{3}{1 0}{\Func}
  \psplotDiffEqn[plotpoints=100]{0}{3}{Pi 2 div 0}{\Func}
\end{pspicture}
\egroup
\end{center}

\begin{lstlisting}[label=fig:second]
\def\Func{y[1]|-9.8*sin(y[0])}
\psset{yunit=2,xunit=4,algebraic=true,linewidth=1.5pt}
\begin{pspicture}(0,-2.25)(3,2.25)
  \psaxes{->}(0,0)(0,-2)(3,2)
  \psplot[linewidth=3\pslinewidth, linecolor=Orange]{0}{3}{.1*cos(sqrt(9.8)*x)}
  \psset{method=rk4,plotpoints=50,linecolor=blue}
  \psplotDiffEqn{0}{3}{.1 0}{\Func}
  \psplot[linewidth=3\pslinewidth,linecolor=Orange]{0}{3}{.25*cos(sqrt(9.8)*x)}
  \psplotDiffEqn{0}{3}{.25 0}{\Func}
  \psplotDiffEqn{0}{3}{.5 0}{\Func}
  \psplotDiffEqn{0}{3}{1 0}{\Func}
  \psplotDiffEqn[plotpoints=100]{0}{3}{Pi 2 div 0}{\Func}
\end{pspicture}
\end{lstlisting}

%--------------------------------------------------------------------------------------
\clearpage
\subsubsection{$y''=-\frac{y'}{4}-2y$}% $
%--------------------------------------------------------------------------------------

For $y_0=5$ and $y'_0=0$ the solution is:

\[
5e^{-\frac{x}{8}}\left(\cos\left(\omega x\right)+\frac{\sin(\omega x)}{8\omega}\right)
\mbox{ avec } \omega=\frac{\sqrt{127}}{8}
\]

\begin{center}
\bgroup
\psset{xunit=.6,yunit=0.8,plotpoints=500}
\begin{pspicture}(0,-4.25)(26,5.25)
  \psaxes{->}(0,0)(0,-4)(26,5)
  \psplot[plotpoints=200,linewidth=4\pslinewidth,linecolor=gray]{0}{26}{%
     Euler x -8 div exp x 127 sqrt 8 div mul RadtoDeg dup cos 5 mul exch sin 127 sqrt div 5 mul add mul}
  \psplotDiffEqn[linecolor=red,linewidth=5\pslinewidth]{0}{26}{5 0}
     {dup 3 1 roll -4 div exch 2 mul sub}
  \psplotDiffEqn[linecolor=black,algebraic=true]{0}{26}{5 0} {y[1]|-y[1]/4-2*y[0]}
  \psset{method=rk4, plotpoints=50}
  \psplotDiffEqn[linecolor=blue,linewidth=5\pslinewidth]{0}{26}{5 0}{%
      dup 3 1 roll -4 div exch 2 mul sub}
  \psplotDiffEqn[linecolor=black,algebraic=true]{0}{26}{5 0}{y[1]|-y[1]/4-2*y[0]}
\end{pspicture}
\egroup
\end{center}

\begin{lstlisting}
\psset{xunit=.6,yunit=0.8,plotpoints=500}
\begin{pspicture}(0,-4.25)(26,5.25)
  \psaxes{->}(0,0)(0,-4)(26,5)
  \psplot[plotpoints=200,linewidth=4\pslinewidth,linecolor=gray]{0}{26}{%
     Euler x -8 div exp x 127 sqrt 8 div mul RadtoDeg dup cos 5 mul exch sin 127 sqrt div 5 mul add mul}
  \psplotDiffEqn[linecolor=red,linewidth=5\pslinewidth]{0}{26}{5 0}
     {dup 3 1 roll -4 div exch 2 mul sub}
  \psplotDiffEqn[linecolor=black,algebraic=true]{0}{26}{5 0} {y[1]|-y[1]/4-2*y[0]}
  \psset{method=rk4, plotpoints=50}
  \psplotDiffEqn[linecolor=blue,linewidth=5\pslinewidth]{0}{26}{5 0}{%
      dup 3 1 roll -4 div exch 2 mul sub}
  \psplotDiffEqn[linecolor=black,algebraic=true]{0}{26}{5 0}{y[1]|-y[1]/4-2*y[0]}
\end{pspicture}
\end{lstlisting}


\clearpage
\subsection{Save final state of a equation}
With the macros \Lcs{BeginSaveFinalState} and \Lcs{EndSaveFinalState} the
end values of a differential equation
can be saved and then used with the optional argument \Lkeyword{GetFinalState}  
as starting values for another equation.

\begin{lstlisting}
\psset{unit=10cm,linewidth=2pt}
\begin{pspicture}(1,1)\psgrid[subgridcolor=black!20,subgriddiv=20]
\BeginSaveFinalState
 \psplotDiffEqn[
   whichabs=0,whichord=1,linecolor=red,method=rk4,
   plotpoints=10,showpoints=true]{0}{1}{0 0}{
   pop pop
   x dup mul 2 div 180 mul cos %% dx/dt
   x dup mul 2 div 180 mul sin %% dy/dt
 }
 \psplotDiffEqn[GetFinalState,
   whichabs=0,whichord=1,linecolor=blue,method=rk4,%SaveFinalState,
   plotpoints=10,showpoints=true]{1}{2}{0 0}{
   pop pop
   x dup mul 2 div 180 mul cos %% dx/dt
   x dup mul 2 div 180 mul sin %% dy/dt
 }
 \psplotDiffEqn[GetFinalState,
   whichabs=0,whichord=1,linecolor=cyan,method=rk4,%SaveFinalState,
   plotpoints=19,showpoints=true]{2}{3}{0 0 }{
   pop pop
   x dup mul 2 div 180 mul cos %% dx/dt
   x dup mul 2 div 180 mul sin %% dy/dt
 }
\EndSaveFinalState
\end{pspicture}
\end{lstlisting}


\bigskip
\begin{center}
\psset{unit=6cm,linewidth=2pt}
\begin{pspicture}(1,1)\psgrid[subgridcolor=black!20,subgriddiv=20]
\BeginSaveFinalState
 \psplotDiffEqn[
   whichabs=0,whichord=1,linecolor=red,method=rk4,
   plotpoints=10,showpoints=true]{0}{1}{0 0}{
   pop pop
   x dup mul 2 div 180 mul cos %% dx/dt
   x dup mul 2 div 180 mul sin %% dy/dt
 }
 \psplotDiffEqn[GetFinalState,
   whichabs=0,whichord=1,linecolor=blue,method=rk4,%SaveFinalState,
   plotpoints=10,showpoints=true]{1}{2}{0 0}{
   pop pop
   x dup mul 2 div 180 mul cos %% dx/dt
   x dup mul 2 div 180 mul sin %% dy/dt
 }
 \psplotDiffEqn[GetFinalState,
   whichabs=0,whichord=1,linecolor=cyan,method=rk4,%SaveFinalState,
   plotpoints=19,showpoints=true]{2}{3}{0 0 }{
   pop pop
   x dup mul 2 div 180 mul cos %% dx/dt
   x dup mul 2 div 180 mul sin %% dy/dt
 }
\EndSaveFinalState
\end{pspicture}
\end{center}

\psset{unit=1cm,linewidth=0.75pt}


%--------------------------------------------------------------------------------------
\clearpage
\section{\nxLcs{psMatrixPlot}}\label{sec:psMatrix}
%--------------------------------------------------------------------------------------
\begin{filecontents}{matrix.data}
/dotmatrix [ %
0  1  1  0  0  0  0  1  1  1
0  1  1  0  1  1  1  0  1  0
1  0  1  1  0  0  0  1  1  0
0  0  1  0  0  0  0  0  1  1
1  1  1  1  1  0  1  0  0  1
0  0  1  1  0  1  0  1  1  1
1  0  0  0  1  1  0  0  0  1
0  0  0  1  1  1  0  1  1  0
1  1  0  0  0  0  1  0  0  1
1  0  1  0  0  1  1  1  0  0
] def
\end{filecontents}


This macro allows you to visualize a matrix. The datafile must be
defined as a PostScript matrix named \Lps{dotmatrix}:
\begin{lstlisting}[style=syntax]
/dotmatrix [ %  <------------ important line
0  1  1  0  0  0  0  1  1  1
0  1  1  0  1  1  1  0  1  0
1  0  1  1  0  0  0  1  1  0
0  0  1  0  0  0  0  0  1  1
1  1  1  1  1  0  1  0  0  1
0  0  1  1  0  1  0  1  1  1
1  0  0  0  1  1  0  0  0  1
0  0  0  1  1  1  0  1  1  0
1  1  0  0  0  0  1  0  0  1
1  0  1  0  0  1  1  1  0  0
] def        %  <------------ important line
\end{lstlisting}

Only the value 0 is important, in which case nothing happens, and
for all other cases a dot is printed. The syntax of the macro is:

\begin{BDef}
\Lcs{psMatrixPlot}\OptArgs\Largb{rows}\Largb{columns}\Largb{data file}
\end{BDef}

The \Index{matrix} is scanned line by line from the the first one to the
last. In general it appears as a bottom-to-top version of the
above listed matrix, the first row $0\,1\,1\,0\,0\,0\,0\,1\,1\,1$
is the first plotted line ($y=1$). With the option
\Lkeyword{ChangeOrder}=\true\ it looks exactly like the above view.

\bgroup
\begin{center}
\psscalebox{0.6}{%
\begin{pspicture}(-0.5,-0.75)(11,11)
  \psaxes{->}(11,11)
  \psMatrixPlot[dotsize=1.1cm,dotstyle=square*,linecolor=magenta]%
    {10}{10}{matrix.data}
  \psMatrixPlot[dotsize=.5cm,dotstyle=o,ChangeOrder]{10}{10}{matrix.data}
\end{pspicture}}\quad
\psscalebox{0.6}{%
\begin{pspicture}(-0.5,-0.75)(11,11)
  \psaxes[ticksize=-5pt 0]{->}(11,11)
  \psMatrixPlot[dotsize=1.1cm,dotstyle=square*,linecolor=magenta,XYoffset=-0.5]%
    {10}{10}{matrix.data}
  \psMatrixPlot[dotsize=.5cm,dotstyle=o,ChangeOrder,XYoffset=-0.5]{10}{10}{matrix.data}
\end{pspicture}}
\end{center}

\begin{lstlisting}
\psscalebox{0.6}{%
\begin{pspicture}(-0.5,-0.75)(11,11)
  \psaxes[ticksize=-5pt 0]{->}(11,11)
  \psMatrixPlot[dotsize=1.1cm,dotstyle=square*,linecolor=magenta]%
    {10}{10}{matrix.data}
  \psMatrixPlot[dotsize=.5cm,dotstyle=o,ChangeOrder]{10}{10}{matrix.data}
\end{pspicture}}\quad
\psscalebox{0.6}{%
\begin{pspicture}(-0.5,-0.75)(11,11)
  \psaxes{->}(11,11)
  \psMatrixPlot[dotsize=1.1cm,dotstyle=square*,linecolor=magenta,XYoffset=-0.5]%
    {10}{10}{matrix.data}
  \psMatrixPlot[dotsize=.5cm,dotstyle=o,ChangeOrder,XYoffset=-0.5]{10}{10}{matrix.data}
\end{pspicture}}
\end{lstlisting}

\begin{LTXexample}[pos=t,preset=\centering]
\begin{pspicture}(-0.5,-0.75)(11,11)
  \psaxes[ticksize=-5pt 0]{->}(11,11)
  \psMatrixPlot[dotscale=3,dotstyle=*,linecolor=blue]{10}{8}{matrix.data}
\end{pspicture}
\end{LTXexample}

\clearpage
With the \Lkeyword{colorType}=1 the data is printed as continous color
in the range of the wavelength. The smallest value of the data array
is set to red and the biggest value is set to violett. All other values
are substituted by the corresponding color of the wavlength.
\Lkeyword{colorType}=2 ist the same, but vice versa
with the color, from violet to red. \Lkeyword{colorType}=3 is the grayscale
image and \Lkeyword{colorType}=4 the same invers.

The following examples use a 200$\times$200
matrix data, which is saved as /dotmatrix [...] in the file \LFile{pstricks-add-doc.dat}.

\begin{LTXexample}[pos=t,preset=\centering]
\begin{pspicture}(10,10)
  \psMatrixPlot[colorType=1,xStep=0.05,yStep=0.05]{200}{200}{dotmatrix.data}
\end{pspicture}
\end{LTXexample}

\begin{LTXexample}[pos=t,preset=\centering]
\begin{pspicture}(10,10)
  \psMatrixPlot[colorType=2,xStep=0.05,yStep=0.05]{200}{200}{dotmatrix.data}
\end{pspicture}
\end{LTXexample}

\begin{LTXexample}[pos=t,preset=\centering]
\begin{pspicture}(10,10)
  \psMatrixPlot[colorType=3,xStep=0.05,yStep=0.05]{200}{200}{dotmatrix.data}
\end{pspicture}
\end{LTXexample}

\begin{LTXexample}[pos=t,preset=\centering]
\begin{pspicture}(10,10)
  \psMatrixPlot[colorType=4,xStep=0.05,yStep=0.05]{200}{200}{dotmatrix.data}
\end{pspicture}
\end{LTXexample}
\egroup

\clearpage
With the \Lkeyword{colorType}=5 the color setting can be user defined by the
optional argument \Lkeyword{colorTypeDef}. On the stack is the current value
which can be used for the setting but must be left on the stack when everything
is finished. The following example prints the 0 as color white, the value 1 as
black and all other values depending to the corresponding gray value.

\begin{filecontents*}{matrix1.data}
/dotmatrix [ % <------------ important line
3 0 0 0 0 0 0 0 1 2
0 0 0 0 0 0 0 1 2 1
8 0 0 0 0 0 1 2 1 0
0 0 0 0 0 1 2 1 0 0
0 0 0 0 1 2 1 0 0 0
9 0 0 1 2 1 3 0 0 0
0 0 1 2 1 4 0 0 0 0
0 1 2 1 5 0 0 0 0 0
1 2 1 6 0 0 0 0 0 0
2 1 7 0 0 0 0 0 0 3
] def % <------------ important line
\end{filecontents*}

\begin{center}
\psscalebox{0.7}{%
\begin{pspicture}(-0.5,-0.75)(11,11)
\psaxes[ticksize=-5pt 0]{->}(11,11)
\psMatrixPlot[
  colorType=5,
  colorTypeDef={
    dup /value exch def % save value and leave one on the stack
    value Min sub dMaxMin div neg 1 add 300 mul 400 add \pswavelengthToGRAY 
    value 0 eq \pslbrace 1 \psrbrace if % 
    value 1 eq \pslbrace 0 \psrbrace if  
    setgray 
  },
  dotsize=1.1cm,xStep=1,yStep=1,dotstyle=square*]{10}{10}{matrix1.data}
\end{pspicture}}
\end{center}


\begin{lstlisting}
\begin{filecontents}{matrix1.data}
/dotmatrix [ % <------------ important line
3 0 0 0 0 0 0 0 1 2
0 0 0 0 0 0 0 1 2 1
8 0 0 0 0 0 1 2 1 0
0 0 0 0 0 1 2 1 0 0
0 0 0 0 1 2 1 0 0 0
9 0 0 1 2 1 3 0 0 0
0 0 1 2 1 4 0 0 0 0
0 1 2 1 5 0 0 0 0 0
1 2 1 6 0 0 0 0 0 0
2 1 7 0 0 0 0 0 0 3
] def % <------------ important line
\end{filecontents}
\psscalebox{0.7}{%
\begin{pspicture}(-0.5,-0.75)(11,11)
\psaxes[ticksize=-5pt 0]{->}(11,11)
\psMatrixPlot[
  colorType=5,
  colorTypeDef={
    dup /value exch def % save value and leave one on the stack
    value Min sub dMaxMin div neg 1 add 300 mul 400 add \pswavelengthToGRAY 
    value 0 eq \pslbrace 1 \psrbrace if % 
    value 1 eq \pslbrace 0 \psrbrace if  
    setgray 
  },
  dotsize=1.1cm,xStep=1,yStep=1,dotstyle=square*]{10}{10}{matrix1.data}
\end{pspicture}}
\end{lstlisting}


\Lps{if} statements in the color definition must be enclosed with \Lcs{pslbrace} and \Lcs{psrbrace}
when they are parentheses used in PostScript. In the above example the color definition should be
modified when the matrix is a real big one, in such a case a nested \Lps{ifelse} makes more sense:

\begin{lstlisting}
  colorTypeDef={
    dup /value exch def 
    value 0 eq 
      \pslbrace 1 setgray \psrbrace
      \pslbrace value 1 eq 
        \pslbrace 0 setgray \psrbrace
        \pslbrace Min sub dMaxMin div neg 1 add 300 mul 400 add
          \pswavelengthToGRAY setgray \psrbrace ifelse
      \psrbrace ifelse 
  },
\end{lstlisting}

Replace the \Lcs{pslbrace} and \Lcs{psrbrace} with \{ and \} if it maybe confusing to read:

\begin{lstlisting}
    dup /value exch def 
    value 0 eq 
      { 1 setgray }
      { value 1 eq 
        { 0 setgray }
        { Min sub dMaxMin div neg 1 add 300 mul 400 add
          \pswavelengthToGRAY setgray } ifelse
      } ifelse 
\end{lstlisting}

Another possibility is to define the color procedure onside the data file, where
it \emph{must} be named \Lps{colorTypeDef}. If such a definition exists, the one from
the optional argument \Lkeyword{colorTypeDef} will be ignored. There can be no
\TeX-specific code inside this definition because it is read on PostScript level,
the reason why \Lcs{pswavelengthToGRAY} cannot be used.

\begin{center}
\begin{filecontents}{matrix1.data}
/colorTypeDef {
  dup /value exch def 
  value 0 eq 
    { 1 setgray }
    { value 1 eq 
      { 0 setgray }
      { Min sub dMaxMin div neg 1 add 300 mul 400 add
%        \pswavelengthToGRAY not possible
         tx@addDict begin wavelengthToRGB Red Green Blue end 
        setrgbcolor
      } ifelse
    } ifelse 
} def
/dotmatrix [ % <------------ important line
3 0 0 0 0 0 0 0 1 2
0 0 0 0 0 0 0 1 2 1
8 0 0 0 0 0 1 2 1 0
0 0 0 0 0 1 2 1 0 0
0 0 0 0 1 2 1 0 0 0
9 0 0 1 2 1 3 0 0 0
0 0 1 2 1 4 0 0 0 0
0 1 2 1 5 0 0 0 0 0
1 2 1 6 0 0 0 0 0 0
2 1 7 0 0 0 0 0 0 3
] def % <------------ important line
\end{filecontents}
\psscalebox{0.7}{%
\begin{pspicture}(-0.5,-0.75)(11,11)
\psaxes[ticksize=-5pt 0]{->}(11,11)
\psMatrixPlot[
  colorType=5,dotsize=1.1cm,xStep=1,yStep=1,dotstyle=square*]{10}{10}{matrix1.data}
\end{pspicture}}
\end{center}

\begin{lstlisting}
\begin{filecontents}{matrix1.data}
/colorTypeDef {
  dup /value exch def 
  value 0 eq 
    { 1 setgray }
    { value 1 eq 
      { 0 setgray }
      { Min sub dMaxMin div neg 1 add 300 mul 400 add
%        \pswavelengthToRGB not possible
         tx@addDict begin wavelengthToRGB Red Green Blue end 
        setrgbcolor
      } ifelse
    } ifelse 
} def
/dotmatrix [ % <------------ important line
3 0 0 0 0 0 0 0 1 2
0 0 0 0 0 0 0 1 2 1
8 0 0 0 0 0 1 2 1 0
0 0 0 0 0 1 2 1 0 0
0 0 0 0 1 2 1 0 0 0
9 0 0 1 2 1 3 0 0 0
0 0 1 2 1 4 0 0 0 0
0 1 2 1 5 0 0 0 0 0
1 2 1 6 0 0 0 0 0 0
2 1 7 0 0 0 0 0 0 3
] def % <------------ important line
\end{filecontents}
\psscalebox{0.7}{%
\begin{pspicture}(-0.5,-0.75)(11,11)
\psaxes[ticksize=-5pt 0]{->}(11,11)
\psMatrixPlot[colorType=5,dotsize=1.1cm,xStep=1,yStep=1,
  dotstyle=square*]{10}{10}{matrix1.data}
\end{pspicture}}
\end{lstlisting}



%--------------------------------------------------------------------------------------
\section{Dashed Lines}
%--------------------------------------------------------------------------------------
Tobias Nähring has implemented an enhanced feature for dashed
lines. The number of arguments is no longer limited.

\begin{BDef}
\Lkeyword{dash}=value1\OptArg*{unit} value2\OptArg*{unit} \ldots
\end{BDef}

\begin{LTXexample}[width=0.4\linewidth]
\psset{linewidth=2.5pt,unit=0.6}
\begin{pspicture}(-5,-4)(5,4)
 \psgrid[subgriddiv=0,griddots=10,gridlabels=0pt]
  \psset{linestyle=dashed}
  \pscurve[dash=5mm 1mm 1mm 1mm,linewidth=0.1](-5,4)(-4,3)(-3,4)(-2,3)
  \psline[dash=5mm 1mm 1mm 1mm 1mm 1mm 1mm 1mm 1mm 1mm](-5,0.9)(5,0.9)
  \psccurve[linestyle=solid](0,0)(1,0)(1,1)(0,1)
  \psccurve[linestyle=dashed,dash=5mm 2mm 0.1 0.2,linetype=0](0,0)(-2.5,0)(-2.5,-2.5)(0,-2.5)
  \pscurve[dash=3mm 3mm 1mm 1mm,linecolor=red,linewidth=2pt](5,-4)(5,2)(4.5,3.5)(3,4)(-5,4)
\end{pspicture}
\end{LTXexample}



\clearpage
%--------------------------------------------------------------------------------------
\section{Arrows}
%--------------------------------------------------------------------------------------
\subsection{Definition}
%--------------------------------------------------------------------------------------
\LPack{pstricks-add} defines the following "`arrows"':

\begin{center}
  \bgroup
  \def\myline#1{\psline[linecolor=red,linewidth=0.5pt,arrowscale=1.5]{#1}(0,1ex)(1.3,1ex)}%
  \psset{arrowscale=1.5}
  \begin{tabular}{@{} c @{\qquad} p{3cm} l @{}}%
    Value & Example & Name \\[2pt]\hline
    \Lnotation{-}      & \myline{-}      & None\\
    \Lnotation{<->}    & \myline{<->}    & Arrowheads.\\
    \Lnotation{>-<}    & \myline{>-<}    & Reverse arrowheads.\\
    \Lnotation{<{<}-{>}>}  & \myline{<<->>}  & Double arrowheads.\\
    \Lnotation{{>}>-{<}<}  & \myline{>>-<<}  & Double reverse arrowheads.\\
    \Lnotation{{|}-{|}}    & \myline{|-|}    & T-bars, flush to endpoints.\\
    \Lnotation{{|}*-{|}*}  & \myline{|*-|*}  & T-bars, centered on endpoints.\\
    \Lnotation{[-]}    & \myline{[-]}    & Square brackets.\\
    \Lnotation{]-[}    & \myline{]-[}    & Reversed square brackets.\\
    \Lnotation{(-)}    & \myline{(-)}    & Rounded brackets.\\
    \Lnotation{)-(}    & \myline{)-(}    & Reversed rounded brackets.\\
    \Lnotation{o-o}    & \myline{o-o}    & Circles, centered on endpoints.\\
    \Lnotation{*-*}    & \myline{*-*}    & Disks, centered on endpoints.\\
    \Lnotation{oo-oo}  & \myline{oo-oo}  & Circles, flush to endpoints.\\
    \Lnotation{**-**}  & \myline{**-**}  & Disks, flush to endpoints.\\
    \Lnotation{{|}<->{|}}  & \myline{|<->|}  & T-bars and arrows.\\
    \Lnotation{{|}>-<{|}}  & \myline{|>-<|}  & T-bars and reverse arrows.\\
    \Lnotation{h-h{|}}   & \myline{h-h}    & left/right hook arrows.\\
    \Lnotation{H-H{|}}   & \myline{H-H}    & left/right hook arrows.\\
    \Lnotation{v-v|}   & \myline{v-v}    & left/right inside vee arrows.\\
    \Lnotation{V-V|}   & \myline{V-V}    & left/right outside vee arrows.\\
    \Lnotation{f-f|}   & \myline{f-f}    & left/right inside filled arrows.\\
    \Lnotation{F-F|}   & \myline{F-F}    & left/right outside filled arrows.\\
    \Lnotation{t-t|}   & \myline{t-t}    & left/right inside slash arrows.\\[5pt]
    \Lnotation{T-T|}   & \myline{T-T}    & left/right outside slash arrows.\\
  \end{tabular}
  \egroup
\end{center}



You can also mix and match, e.g., \Lnotation{->}, \Lnotation{*-)} and \Lnotation{[->} are all valid values
of the \Lkeyword{arrows} parameter. The parameter can be set with

\begin{BDef}
\Lcs{psset}\Largb{arrows=<type>}
\end{BDef}

\noindent or for some macros with a special option, like\\[5pt]
\noindent\verb|\psline[<general options>]{<arrow type>}(A)(B)|\\
\noindent\verb/\psline[linecolor=red,linewidth=2pt]{|->}(0,0)(0,2)/ \ \psline[linecolor=red,linewidth=2pt]{|->}(0,0)(0,2)

\subsection{Multiple arrows}
There are two new options which are only valid for the arrow type \verb+<<+ or \verb+>>+.
\verb+nArrow+ sets both, the \verb+nArrowA+ and the  \verb+nArrowB+ parameter. The meaning
is declared in the following tables. Without setting one of these parameters the behaviour
is like the one described in the old PSTricks manual.

\begin{center}
\begin{tabular}{@{}lc@{}}%
    Value & Meaning \\[2pt]\hline
    \Lnotation{-{>}>}   & \ -A \\
    \Lnotation{{<}<-{>}>} & A-A\\
    \Lnotation{{<}<-}   & A-\ \\
    \Lnotation{{>}>-}   & B-\ \\
    \Lnotation{-{<}<}   & \ -B\\
    \Lnotation{{>}>-{<}<} & B-B\\
    \Lnotation{{>}>-{>}>} & B-A\\
    \Lnotation{{<}<-{<}<} & A-B
  \end{tabular}
\end{center}




\begin{center}
  \bgroup
  \psset{linecolor=red,linewidth=1pt,arrowscale=2}%
  \begin{tabular}{lp{2.8cm}}%
    Value & Example \\[2pt]\hline
    \verb+\psline{->>}(0,1ex)(2.3,1ex)+  & \psline{->>}(0,1ex)(2.3,1ex) \\
    \verb+\psline[nArrowsA=3]{->>}(0,1ex)(2.3,1ex)+  & \psline[nArrowsA=3]{->>}(0,1ex)(2.3,1ex)\\
    \verb+\psline[nArrowsA=5]{->>}(0,1ex)(2.3,1ex)+  & \psline[nArrowsA=5]{->>}(0,1ex)(2.3,1ex)\\
    \verb+\psline{<<-}(0,1ex)(2.3,1ex)+  & \psline{<<-}(0,1ex)(2.3,1ex)\\
    \verb+\psline[nArrowsA=3]{<<-}(0,1ex)(2.3,1ex)+  & \psline[nArrowsA=3]{<<-}(0,1ex)(2.3,1ex)\\
    \verb+\psline[nArrowsA=5]{<<-}(0,1ex)(2.3,1ex)+  & \psline[nArrowsA=5]{<<-}(0,1ex)(2.3,1ex)\\
    \verb+\psline{<<->>}(0,1ex)(2.3,1ex)+  & \psline{<<->>}(0,1ex)(2.3,1ex)\\
    \verb+\psline[nArrowsA=3]{<<->>}(0,1ex)(2.3,1ex)+  & \psline[nArrowsA=3]{<<->>}(0,1ex)(2.3,1ex)\\
    \verb+\psline[nArrowsA=5]{<<->>}(0,1ex)(2.3,1ex)+  & \psline[nArrowsA=5]{<<->>}(0,1ex)(2.3,1ex)\\
    \verb+\psline{<<-|}(0,1ex)(2.3,1ex)+  & \psline{<<-|}(0,1ex)(2.3,1ex)\\
    \verb+\psline[nArrowsA=3]{<<-<<}(0,1ex)(2.3,1ex)+  & \psline[nArrowsA=3]{<<-<<}(0,1ex)(2.3,1ex)\\
    \verb+\psline[nArrowsA=5]{<<-o}(0,1ex)(2.3,1ex)+  & \psline[nArrowsA=5]{<<-o}(0,1ex)(2.3,1ex)\\
    \verb+\psline[nArrowsA=3,nArrowsB=4]{<<-<<}(0,1ex)(2.3,1ex)+  & \psline[nArrowsA=3,nArrowsB=4]{<<-<<}(0,1ex)(2.3,1ex)\\
    \verb+\psline[nArrowsA=3,nArrowsB=4]{>>->>}(0,1ex)(2.3,1ex)+  & \psline[nArrowsA=3,nArrowsB=4]{>>->>}(0,1ex)(2.3,1ex)\\
    \verb+\psline[nArrowsA=1,nArrowsB=4]{>>->>}(0,1ex)(2.3,1ex)+  & \psline[nArrowsA=1,nArrowsB=4]{>>->>}(0,1ex)(2.3,1ex)\\
  \end{tabular}
  \egroup
\end{center}



\subsection{\texttt{hookarrow}}
%\begin{LTXexample}
\bgroup
\psset{arrowsize=8pt,arrowlength=1,linewidth=1pt,nodesep=2pt,shortput=tablr}
\large
\begin{psmatrix}[colsep=12mm,rowsep=10mm]
        &   & $R_2$            \\
        &   &   0   &   & $R_3$\\
$e_b:S$ & 1 &       & 1 & 0    \\
        &   &   0              \\
        &   &   $R_1$          \\
\end{psmatrix}
\ncline{h-}{1,3}{2,3}<{$e_{r2}$}>{$f_{r2}$}
\ncline{-h}{2,3}{3,2}<{$e_1$}
\ncline{-h}{3,1}{3,2}^{$e_s$}_{$f_{s}$}
\ncline{-h}{3,2}{4,3}>{$e_3$}<{$f_3$}
\ncline{-h}{4,3}{3,4}>{$e_4$}<{$f_4$}
\ncline{-h}{3,4}{2,3}>{$e_2$}<{$f_2$}
\ncline{-h}{3,4}{3,5}^{$e_5$}
\ncline{-h}{3,5}{2,5}<{$e_{r3}$}>{$f_{r3}$}
\ncline{-h}{4,3}{5,3}<{$e_{r1}$}>{$f_{r1}$}
%\end{LTXexample}
\egroup

\begin{lstlisting}
\psset{arrowsize=8pt,arrowlength=1,linewidth=1pt,nodesep=2pt,shortput=tablr}
\large
\begin{psmatrix}[colsep=12mm,rowsep=10mm]
        &   & $R_2$            \\
        &   &   0   &   & $R_3$\\
$e_b:S$ & 1 &       & 1 & 0    \\
        &   &   0              \\
        &   &   $R_1$          \\
\end{psmatrix}
\ncline{h-}{1,3}{2,3}<{$e_{r2}$}>{$f_{r2}$}\ncline{-h}{2,3}{3,2}<{$e_1$}
\ncline{-h}{3,1}{3,2}^{$e_s$}_{$f_{s}$}    \ncline{-h}{3,2}{4,3}>{$e_3$}<{$f_3$}
\ncline{-h}{4,3}{3,4}>{$e_4$}<{$f_4$}      \ncline{-h}{3,4}{2,3}>{$e_2$}<{$f_2$}
\ncline{-h}{3,4}{3,5}^{$e_5$}              
\ncline{-h}{3,5}{2,5}<{$e_{r3}$}>{$f_{r3}$}
\ncline{-h}{4,3}{5,3}<{$e_{r1}$}>{$f_{r1}$}
\end{lstlisting}



\subsection{\texttt{hookrightarrow} and \texttt{hookleftarrow}}
This is another type of arrow and is abbreviated with \Lnotation{H}.
The length and width of the hook is set by the new options
\Lkeyword{hooklength} and \Lkeyword{hookwidth}, which are by default set
to
%
\begin{BDef}
\Lcs{psset}\Largb{hooklength=3mm,hookwidth=1mm}
\end{BDef}
%
If the line begins with a right hook then the line ends with a left hook and vice versa:

\begin{LTXexample}[width=3cm]
\begin{pspicture}(3,4)
\psline[linewidth=5pt,linecolor=blue,hooklength=5mm,hookwidth=-3mm]{H->}(0,3.5)(3,3.5)
\psline[linewidth=5pt,linecolor=red,hooklength=5mm,hookwidth=3mm]{H->}(0,2.5)(3,2.5)
\psline[linewidth=5pt,hooklength=5mm,hookwidth=3mm]{H-H}(0,1.5)(3,1.5)
\psline[linewidth=1pt]{H-H}(0,0.5)(3,0.5)
\end{pspicture}
\end{LTXexample}


\begin{LTXexample}[width=7.25cm]
$\begin{psmatrix}
E&W_i(X)&&Y\\
&&W_j(X)
\psset{arrows=->,nodesep=3pt,linewidth=2pt}
\everypsbox{\scriptstyle}
\ncline[linecolor=red,arrows=H->,%
  hooklength=4mm,hookwidth=2mm]{1,1}{1,2}
\ncline{1,2}{1,4}^{\tilde{t}}
\ncline{1,2}{2,3}<{W_{ij}}
\ncline{2,3}{1,4}>{\tilde{s}}
\end{psmatrix}$
\end{LTXexample}


%--------------------------------------------------------------------------------------
\subsection{\nxLkeyword{ArrowInside} Option}
%--------------------------------------------------------------------------------------

It is now possible to have arrows inside lines and not only at the
beginning or the end. The new defined options

\psset{arrowscale=2,linecolor=red,unit=1cm,linewidth=1.5pt}
\begin{longtable}{l|>{\RaggedRight}p{8.5cm}|p{2.2cm}}
Name & Example & Output\\\hline
\endfirsthead
Name & Example & Output\\\hline
\endhead
\Lkeyword{ArrowInside} &
  \texttt{\textbackslash psline[ArrowInside=->](0,0)(2,0)} &
  \psline[ArrowInside=->](0,0.1)(2,0.1) \\
\Lkeyword{ArrowInsidePos} & \texttt{\textbackslash psline[ArrowInside=->,\%}
  \hspace*{20pt}\texttt{ArrowInsidePos=0.25](0,0)(2,0)}
& \psline[ArrowInside=->, ArrowInsidePos=0.25](0,0.1)(2,0.1) \\
\Lkeyword{ArrowInsidePos} & \texttt{\textbackslash psline[ArrowInside=->,\%}
  \hspace*{20pt}\texttt{ArrowInsidePos=10](0,0)(2,0)}
& \psline[ArrowInside=->, ArrowInsidePos=10](0,0.1)(2,0.1) \\
\Lkeyword{ArrowInsideNo} & \texttt{\textbackslash psline[ArrowInside=->,\%}
  \hspace*{20pt}\texttt{ArrowInsideNo=2](0,0)(2,0)}
& \psline[ArrowInside=->, ArrowInsideNo=2](0,0.1)(2,0.1) \\
\Lkeyword{ArrowInsideOffset} & \texttt{\textbackslash psline[ArrowInside=->,\%}
  \hspace*{20pt}\texttt{ArrowInsideNo=2,\%}\newline
  \hspace*{20pt}\texttt{ArrowInsideOffset=0.1](0,0)(2,0)}
& \psline[ArrowInside=->, ArrowInsideNo=2,ArrowInsideOffset=0.1](0,0.1)(2,0.1) \\
%
\Lkeyword{ArrowInside} & \texttt{\textbackslash psline[ArrowInside=->]\{->\}(0,0)(2,0)} &
  \psline[ArrowInside=->]{->}(0,0)(2,0)\\
\Lkeyword{ArrowInsidePos} & \texttt{\textbackslash psline[ArrowInside=->,\%}
  \hspace*{20pt}\texttt{ArrowInsidePos=0.25]\{->\}(0,0)(2,0)}
  & \psline[ArrowInside=->, ArrowInsidePos=0.25]{->}(0,0)(2,0) \\
\Lkeyword{ArrowInsidePos} & \texttt{\textbackslash psline[ArrowInside=->,\%}
  \hspace*{20pt}\texttt{ArrowInsidePos=10]\{->\}(0,0)(2,0)}
  & \psline[ArrowInside=->, ArrowInsidePos=10]{->}(0,0)(2,0) \\
\Lkeyword{ArrowInsideNo} & \texttt{\textbackslash psline[ArrowInside=->,\%}
  \hspace*{20pt}\texttt{ArrowInsideNo=2]\{->\}(0,0)(2,0)}
  & \psline[ArrowInside=->, ArrowInsideNo=2]{->}(0,0)(2,0) \\
\Lkeyword{ArrowInsideOffset} & \texttt{\textbackslash psline[ArrowInside=->,\%}
  \hspace*{20pt}\texttt{ArrowInsideNo=2,\%}\newline
  \hspace*{20pt}\texttt{ArrowInsideOffset=0.1]\{->\}(0,0)(2,0)}
  & \psline[ArrowInside=->, ArrowInsideNo=2,ArrowInsideOffset=0.1]{->}(0,0)(2,0) \\
%
\Lkeyword{ArrowFill} & \texttt{\textbackslash psline[ArrowFill=false,\%}
  \hspace*{20pt}\texttt{arrowinset=0]\{->\}(0,0)(2,0)} &
  \psline[ArrowFill=false,arrowinset=0]{->}(0,0)(2,0)\\
\Lkeyword{ArrowFill} & \texttt{\textbackslash psline[ArrowFill=false,\%}
  \hspace*{20pt}\texttt{arrowinset=0]\{<<->>\}(0,0)(2,0)} &
  \psline[ArrowFill=false,arrowinset=0]{<<->>}(0,0)(2,0)\\
\Lkeyword{ArrowFill} & \texttt{\textbackslash psline[ArrowInside=->,\%}\newline
  \hspace*{20pt}\texttt{arrowinset=0,\%}\newline
  \hspace*{20pt}\texttt{ArrowFill=false,\%}\newline
  \hspace*{20pt}\texttt{ArrowInsideNo=2,\%}\newline
  \hspace*{20pt}\texttt{ArrowInsideOffset=0.1]\{->\}(0,0)(2,0)}
  & \psline[ArrowInside=->, ArrowFill=false,ArrowInsideNo=2,ArrowInsideOffset=0.1]{->}(0,0)(2,0) \\
\end{longtable}

\medskip
Without the default arrow definition there is only the one inside
the line, defined by the type and the position. The position is
relative to the length of the whole line. $0.25$ means at $25\%$
of the line length. The peak of the arrow gets the coordinates
which are calculated by the macro. If you want arrows with an
absolute position difference, then choose a value greater than
\verb|1|, e.\,g. \verb|10| which places an arrow every 10~pt. The
default unit \verb|pt| cannot be changed.

\medskip
\noindent
\begin{tabularx}{\linewidth}{@{\color{red}\vrule width 2pt}lX@{}}
& The \Lkeyword{ArrowInside} takes only arrow definitions like \Lnotation{->} into account.
Arrows from right to left (\Lnotation{<-}) are not possible and ignored. If you need
such arrows, change the order of the pairs of coordinates for the line or curve macro.
\end{tabularx}

%--------------------------------------------------------------------------------------
\subsection{\nxLkeyword{ArrowFill} Option}
%--------------------------------------------------------------------------------------

By default all arrows are filled polygons. With the option
\Lkeyset{ArrowFill=false} there are ''white`` arrows. Only for the
beginning/end arrows are they empty, the inside arrows are
overpainted by the line.


\psset{arrowscale=1}
\begin{LTXexample}[width=3.5cm]
\psset{arrowscale=2.5}
\psline[linecolor=red,arrowinset=0]{<->}(-1,0)(2,0)
\end{LTXexample}

\begin{LTXexample}[width=3.5cm]
\psset{arrowscale=2.5}
\psline[linecolor=red,arrowinset=0,ArrowFill=false]{<->}(-1,0)(2,0)
\end{LTXexample}

\begin{LTXexample}[width=3.5cm]
\psset{arrowscale=2.5}
\psline[linecolor=red,arrowinset=0,arrowsize=0.2,
  ArrowFill=false]{<->}(-1,0)(2,0)
\end{LTXexample}

\begin{LTXexample}[width=3.5cm]
\psline[linecolor=blue,arrowscale=4,
  ArrowFill]{>>->>}(-1,0)(2,0)
\end{LTXexample}

\begin{LTXexample}[width=3.5cm]
\psline[linecolor=blue,arrowscale=4,
  ArrowFill=false]{>>->>}(-1,0)(2,0)
\rule{3cm}{0pt}\\[30pt]
\end{LTXexample}

\begin{LTXexample}[width=3.5cm]
\psline[linecolor=blue,arrowscale=4,
  ArrowFill]{>|->|}(-1,0)(2,0)
\end{LTXexample}

\begin{LTXexample}[width=3.5cm]
\psline[linecolor=blue,arrowscale=4,
  ArrowFill=false]{>|->|}(-1,0)(2,0)%
\end{LTXexample}


%--------------------------------------------------------------------------------------
\subsection{Examples}
%--------------------------------------------------------------------------------------

All examples are printed with \verb|\psset{arrowscale=2,linecolor=red}|.
\subsubsection{\nxLcs{psline}}

\bigskip
\begin{LTXexample}[width=2.5cm]
\begin{pspicture}(2,2)
\psset{arrowscale=2,ArrowFill=true}
\psline[ArrowInside=->]{|<->|}(2,1)
\end{pspicture}
\end{LTXexample}

\begin{LTXexample}[width=2.5cm]
\begin{pspicture}(2,2)
\psset{arrowscale=2,ArrowFill=true}
\psline[ArrowInside=-|]{|-|}(2,1)
\end{pspicture}
\end{LTXexample}

\begin{LTXexample}[width=2.5cm]
\begin{pspicture}(2,2)
\psset{arrowscale=2,ArrowFill=true}
\psline[ArrowInside=->,ArrowInsideNo=2]{->}(2,1)
\end{pspicture}
\end{LTXexample}

\begin{LTXexample}[width=2.5cm]
\begin{pspicture}(2,2)
\psset{arrowscale=2,ArrowFill=true}
\psline[ArrowInside=->,ArrowInsideNo=2,ArrowInsideOffset=0.1]{->}(2,1)
\end{pspicture}
\end{LTXexample}

\begin{LTXexample}[width=6.5cm]
\begin{pspicture}(6,2)
\psset{arrowscale=2,ArrowFill=true}
\psline[ArrowInside=-*]{->}(0,0)(2,1)(3,0)(4,0)(6,2)
\end{pspicture}
\end{LTXexample}

\begin{LTXexample}[width=6.5cm]
\begin{pspicture}(6,2)
\psset{arrowscale=2,ArrowFill=true}
\psline[ArrowInside=-*,ArrowInsidePos=0.25]{->}(0,0)(2,1)(3,0)(4,0)(6,2)
\end{pspicture}
\end{LTXexample}

\begin{LTXexample}[width=6.5cm]
\begin{pspicture}(6,2)
\psset{arrowscale=2,ArrowFill=true}
\psline[ArrowInside=-*,ArrowInsidePos=0.25,ArrowInsideNo=2]{->}%
   (0,0)(2,1)(3,0)(4,0)(6,2)
\end{pspicture}
\end{LTXexample}

\begin{LTXexample}[width=6.5cm]
\begin{pspicture}(6,2)
\psset{arrowscale=2,ArrowFill=true}
\psline[ArrowInside=->, ArrowInsidePos=0.25]{->}%
        (0,0)(2,1)(3,0)(4,0)(6,2)
\end{pspicture}
\end{LTXexample}

\begin{LTXexample}[width=6.5cm]
\begin{pspicture}(6,2)
\psset{arrowscale=2,ArrowFill=true}
\psline[linestyle=none,ArrowInside=->,ArrowInsidePos=0.25]{->}%
        (0,0)(2,1)(3,0)(4,0)(6,2)
\end{pspicture}
\end{LTXexample}

\begin{LTXexample}[width=6.5cm]
\begin{pspicture}(6,2)
\psset{arrowscale=2,ArrowFill=true}
\psline[ArrowInside=-<, ArrowInsidePos=0.75]{->}%
     (0,0)(2,1)(3,0)(4,0)(6,2)
\end{pspicture}
\end{LTXexample}

\begin{LTXexample}[width=6.5cm]
\begin{pspicture}(6,2)
\psset{arrowscale=2,ArrowFill=true,ArrowInside=-*}
\psline(0,0)(2,1)(3,0)(4,0)(6,2)
\psset{linestyle=none}
\psline[ArrowInsidePos=0](0,0)(2,1)(3,0)(4,0)(6,2)
\psline[ArrowInsidePos=1](0,0)(2,1)(3,0)(4,0)(6,2)
\end{pspicture}
\end{LTXexample}

\begin{LTXexample}[width=6.5cm]
\begin{pspicture}(6,5)
\psset{arrowscale=2,ArrowFill=true}
\psline[ArrowInside=->,ArrowInsidePos=20](0,0)(3,0)%
       (3,3)(1,3)(1,5)(5,5)(5,0)(7,0)(6,3)
\end{pspicture}
\end{LTXexample}

\begin{LTXexample}[width=6.5cm]
\begin{pspicture}(6,2)
\psset{arrowscale=2,ArrowFill=true}
\psline[ArrowInside=-|]{<->}(0,2)(2,0)(3,2)(4,0)(6,2)
\end{pspicture}
\end{LTXexample}

%--------------------------------------------------------------------------------------
\subsubsection{\nxLcs{pspolygon}}
%--------------------------------------------------------------------------------------
% Polygons (\pspolygon macro)

\begin{LTXexample}[width=6.5cm]
\begin{pspicture}(6,3)
\psset{arrowscale=2}
\pspolygon[ArrowInside=-|](0,0)(3,3)(6,3)(6,1)
\end{pspicture}
\end{LTXexample}

\begin{LTXexample}[width=6.5cm]
\begin{pspicture}(6,3)
\psset{arrowscale=2}
\pspolygon[ArrowInside=->,ArrowInsidePos=0.25]%
     (0,0)(3,3)(6,3)(6,1)
\end{pspicture}
\end{LTXexample}

\begin{LTXexample}[width=6.5cm]
\begin{pspicture}(6,3)
\psset{arrowscale=2}
\pspolygon[ArrowInside=->,ArrowInsideNo=4]%
       (0,0)(3,3)(6,3)(6,1)
\end{pspicture}
\end{LTXexample}

\begin{LTXexample}[width=6.5cm]
\begin{pspicture}(6,3)
\psset{arrowscale=2}
\pspolygon[ArrowInside=->,ArrowInsideNo=4,%
   ArrowInsideOffset=0.1](0,0)(3,3)(6,3)(6,1)
\end{pspicture}
\end{LTXexample}

\begin{LTXexample}[width=6.5cm]
\begin{pspicture}(6,3)
\psset{arrowscale=2}
 \pspolygon[ArrowInside=-|](0,0)(3,3)(6,3)(6,1)
 \psset{linestyle=none,ArrowInside=-*}
 \pspolygon[ArrowInsidePos=0](0,0)(3,3)(6,3)(6,1)
 \pspolygon[ArrowInsidePos=1](0,0)(3,3)(6,3)(6,1)
 \psset{ArrowInside=-o}
 \pspolygon[ArrowInsidePos=0.25](0,0)(3,3)(6,3)(6,1)
 \pspolygon[ArrowInsidePos=0.75](0,0)(3,3)(6,3)(6,1)
\end{pspicture}
\end{LTXexample}

\psset{linestyle=solid}

\begin{LTXexample}[width=6.5cm]
\begin{pspicture}(6,5)
\psset{arrowscale=2}
  \pspolygon[ArrowInside=->,ArrowInsidePos=20]%
    (0,0)(3,0)(3,3)(1,3)(1,5)(5,5)(5,0)(7,0)(6,3)
\end{pspicture}
\end{LTXexample}


%--------------------------------------------------------------------------------------
\subsubsection{\nxLcs{psbezier}}
%--------------------------------------------------------------------------------------
% Bezier curves (\psbezier macro)


\begin{LTXexample}[width=3.5cm]
\begin{pspicture}(3,3)
\psset{arrowscale=2}
  \psbezier[ArrowInside=-|](0,1)(1,0)(2,1)(3,3)
  \psset{linestyle=none,ArrowInside=-o}
  \psbezier[ArrowInsidePos=0.25](0,1)(1,0)(2,1)(3,3)
  \psbezier[ArrowInsidePos=0.75](0,1)(1,0)(2,1)(3,3)
  \psset{linestyle=none,ArrowInside=-*}
  \psbezier[ArrowInsidePos=0](0,1)(1,0)(2,1)(3,3)
  \psbezier[ArrowInsidePos=1](0,1)(1,0)(2,1)(3,3)
\end{pspicture}
\end{LTXexample}



\resetOptions
\begin{LTXexample}[width=4.5cm]
\begin{pspicture}(4,3)
\psset{arrowscale=2}
\psbezier[ArrowInside=->,showpoints]%
  {*-*}(0,0)(2,3)(3,0)(4,2)
\end{pspicture}
\end{LTXexample}




\begin{LTXexample}[width=4.5cm]
\begin{pspicture}(4,3)
\psset{arrowscale=2}
  \psbezier[ArrowInside=->,showpoints=true,
      ArrowInsideNo=2](0,0)(2,3)(3,0)(4,2)
\end{pspicture}
\end{LTXexample}


\begin{LTXexample}[width=4.5cm]
\begin{pspicture}(4,3)
\psset{arrowscale=2}
  \psbezier[ArrowInside=->,showpoints=true,
     ArrowInsideNo=2,ArrowInsideOffset=-0.2]%
      {->}(0,0)(2,3)(3,0)(4,2)
\end{pspicture}
\end{LTXexample}


\begin{LTXexample}[width=5.5cm]
\begin{pspicture}(5,3)
\psset{arrowscale=2}
  \psbezier[ArrowInsideNo=9,ArrowInside=-|,%
    showpoints=true]{*-*}(0,0)(1,3)(3,0)(5,3)
\end{pspicture}
\end{LTXexample}

\begin{LTXexample}[width=4.5cm]
\begin{pspicture}(4,3)
\psset{arrowscale=2}
  \psset{ArrowInside=-|}
  \psbezier[ArrowInsidePos=0.25,showpoints=true]{*-*}(2,3)(3,0)(4,2)
  \psset{linestyle=none}
  \psbezier[ArrowInsidePos=0.75](0,0)(2,3)(3,0)(4,2)
\end{pspicture}
\end{LTXexample}

\begin{LTXexample}[width=5.5cm]
\begin{pspicture}(5,6)
\psset{arrowscale=2}
  \pnode(3,4){A}\pnode(5,6){B}\pnode(5,0){C}
  \psbezier[ArrowInside=->,%
     showpoints=true](A)(B)(C)
  \psset{linestyle=none,ArrowInside=-<}
  \psbezier[ArrowInsideNo=4](0,0)(A)(B)(C)
  \psset{ArrowInside=-o}
  \psbezier[ArrowInsidePos=0.1](0,0)(A)(B)(C)
  \psbezier[ArrowInsidePos=0.9](0,0)(A)(B)(C)
  \psset{ArrowInside=-*}
  \psbezier[ArrowInsidePos=0.3](0,0)(A)(B)(C)
  \psbezier[ArrowInsidePos=0.7](0,0)(A)(B)(C)
\end{pspicture}
\end{LTXexample}

\psset{linestyle=solid}

\begin{LTXexample}[pos=t]
\begin{pspicture}(-3,-5)(15,5)
  \psbezier[ArrowInsideNo=19,%
      ArrowInside=->,ArrowFill=false,%
      showpoints=true]{->}(-3,0)(5,-5)(8,5)(15,-5)
\end{pspicture}
\end{LTXexample}



%--------------------------------------------------------------------------------------
\subsubsection{\nxLcs{pcline}}
%--------------------------------------------------------------------------------------
These examples need the package \verb|pst-node|.

% Lines (\pcline macro)
\begin{LTXexample}[width=2.5cm]
\begin{pspicture}(2,1)
\psset{arrowscale=2}
\pcline[ArrowInside=->](0,0)(2,1)
\end{pspicture}
\end{LTXexample}


\begin{LTXexample}[width=2.5cm]
\begin{pspicture}(2,1)
\psset{arrowscale=2}
\pcline[ArrowInside=->]{<->}(0,0)(2,1)
\end{pspicture}
\end{LTXexample}


\begin{LTXexample}[width=2.5cm]
\begin{pspicture}(2,1)
\psset{arrowscale=2}
\pcline[ArrowInside=-|,ArrowInsidePos=0.75]{|-|}(0,0)(2,1)
\end{pspicture}
\end{LTXexample}


\begin{LTXexample}[width=2.5cm]
\psset{arrowscale=2}
\pcline[ArrowInside=->,ArrowInsidePos=0.65]{*-*}(0,0)(2,0)
\naput[labelsep=0.3]{\large$g$}
\end{LTXexample}


\begin{LTXexample}[width=2.5cm]
\psset{arrowscale=2}
\pcline[ArrowInside=->,ArrowInsidePos=10]{|-|}(0,0)(2,0)
\naput[labelsep=0.3]{\large$l$}
\end{LTXexample}



%--------------------------------------------------------------------------------------
\subsubsection{\nxLcs{pccurve}}
%--------------------------------------------------------------------------------------
These examples also need the package \verb|pst-node|.

\begin{LTXexample}[width=2.5cm]
\begin{pspicture}(2,2)
\psset{arrowscale=2}
\pccurve[ArrowInside=->,ArrowInsidePos=0.65,showpoints=true]{*-*}(0,0)(2,2)
\naput[labelsep=0.3]{\large$h$}
\end{pspicture}
\end{LTXexample}


\begin{LTXexample}[width=2.5cm]
\begin{pspicture}(2,2)
\psset{arrowscale=2}
\pccurve[ArrowInside=->,ArrowInsideNo=3,showpoints=true]{|->}(0,0)(2,2)
\naput[labelsep=0.3]{\large$i$}
\end{pspicture}
\end{LTXexample}


\begin{LTXexample}[width=4.5cm]
\begin{pspicture}(4,4)
\psset{arrowscale=2}
\pccurve[ArrowInside=->,ArrowInsidePos=20]{|-|}(0,0)(4,4)
\naput[labelsep=0.3]{\large$k$}
\end{pspicture}
\end{LTXexample}

\clearpage

\subsection{Special arrows \texttt{v--V},\texttt{t--T}, and \texttt{f--F}}

Possible optional arguments are

\psset{linecolor=black}

\begin{center}
\begin{tabular}{@{}l|l@{}}\toprule
\emph{name} & \emph{meaning}\\\hline
\Lkeyword{veearrowlength} & default is 3mm\\
\Lkeyword{veearrowangle} & default is 30\\
\Lkeyword{veearrowlinewidth} & default is 0.35mm\\
\Lkeyword{filledveearrowlength} & default is 3mm\\
\Lkeyword{filledveearrowangle} & default is 15\\
\Lkeyword{filledveearrowlinewidth} & default is 0.35mm\\
\Lkeyword{tickarrowlength} & default is 1.5mm\\
\Lkeyword{tickarrowlinewidth} & default is 0.35mm\\
\Lkeyword{arrowlinestyle}     & default is solid\\\bottomrule
\end{tabular}
\end{center}


\begin{LTXexample}[width=4cm]
\psset{unit=5mm}
\begin{pspicture}(4,6)
  \psset{dimen=middle,arrows=c-c,
    arrowscale=2,linewidth=.25mm}
  \psline[linecolor=red,linewidth=.05mm](0,0)(0,6)
  \psline[linecolor=red,linewidth=.05mm](4,0)(4,6)
  \psline{v-v}(0,6)(4,6)
  \psline{v-V}(0,4)(4,4)
  \psline{V-v}(0,2)(4,2)
  \psline{V-V}(0,0)(4,0)
\end{pspicture}
\end{LTXexample}


\begin{LTXexample}[width=4cm]
\psset{unit=5mm}
\begin{pspicture}(4,6)
  \psset{dimen=middle,arrows=c-c,
    arrowscale=2,linewidth=.25mm}
  \psline[linecolor=red,linewidth=.05mm](0,0)(0,6)
  \psline[linecolor=red,linewidth=.05mm](4,0)(4,6)
  \psline{f-f}(0,6)(4,6)
  \psline{f-F}(0,4)(4,4)
  \psline{F-f}(0,2)(4,2)
  \psline{F-F}(0,0)(4,0)
\end{pspicture}
\end{LTXexample}


\begin{LTXexample}[width=4cm]
\psset{unit=5mm}
\begin{pspicture}(4,6)
  \psset{dimen=middle,arrows=c-c,linewidth=.25mm}
  \psline[linecolor=red,linewidth=.05mm](0,0)(0,6)
  \psline[linecolor=red,linewidth=.05mm](4,0)(4,6)
  \psline{t-t}(0,6)(4,6)
  \psline{t-T}(0,4)(4,4)
  \psline{T-t}(0,2)(4,2)
  \psline{T-T}(0,0)(4,0)
\end{pspicture}
\end{LTXexample}

\begin{LTXexample}[pos=t,vsep=5mm]
\psset{unit=5mm}
 \begin{pspicture}(10,6)
 \psset{dimen=middle,arrows=c-c,arrowscale=2,linewidth=.25mm,
        arrowlinestyle=dashed,dash=1.5pt 1pt}
 \psline[linecolor=red,linewidth=.05mm](0,0)(0,6)
 \psline[linecolor=red,linewidth=.05mm](4,0)(4,6)
 \psline{v-v}(0,6)(4,6) \psline{v-V}(0,4)(4,4)
 \psline{V-v}(0,2)(4,2) \psline{V-V}(0,0)(4,0)
 \psline[linecolor=red,linewidth=.05mm](6,0)(6,6)
 \psline[linecolor=red,linewidth=.05mm](10,0)(10,6)
 \psset{arrowlinestyle=dotted,dotsep=0.8pt}
 \psline{v-v}(6,6)(10,6) \psline{v-V}(6,4)(10,4)
 \psline{V-v}(6,2)(10,2) \psline{V-V}(6,0)(10,0)
\end{pspicture}
\end{LTXexample}

\begin{LTXexample}[pos=t,vsep=5mm]
\psset{unit=5mm}
 \begin{pspicture}(10,7)
 \psset{dimen=middle,arrows=c-c,arrowscale=2,linewidth=.25mm,
        arrowlinestyle=dashed,dash=1.5pt 1pt}
 \psline[linecolor=red,linewidth=.05mm](0,0)(0,6)
 \psline[linecolor=red,linewidth=.05mm](4,0)(4,6)
 \psline{t-t}(0,6)(4,6) \psline{t-T}(0,4)(4,4)
 \psline{T-t}(0,2)(4,2) \psline{T-T}(0,0)(4,0)
 \psline[linecolor=red,linewidth=.05mm](6,0)(6,6)
 \psline[linecolor=red,linewidth=.05mm](10,0)(10,6)
 \psset{arrowlinestyle=dotted,dotsep=0.8pt}
 \psline{t-t}(6,6)(10,6) \psline{t-T}(6,4)(10,4)
 \psline{T-t}(6,2)(10,2) \psline{T-T}(6,0)(10,0)
\end{pspicture}
\end{LTXexample}




\subsection{Special arrow option \texttt{arrowLW}}

Only for the arrowtype \Lnotation{o} and \Lnotation{*} it is possible to
set the arrowlinewidth with the optional keyword \Lkeyword{arrowLW}.
When scaling an arrow by the keyword \Lkeyword{arrowscale} the width
of the borderline is also scaled. With the optional argument
\Lkeyword{arrowLW} the line width can be set separately and is not
taken into account by the scaling value.

\begin{LTXexample}[width=4cm]
\begin{pspicture}(4,6)
\psline[arrowscale=3,arrows=*-o](0,5)(4,5)
\psline[arrowscale=3,arrows=*-o,
  arrowLW=0.5pt](0,3)(4,3)
\psline[arrowscale=3,arrows=*-o,
  arrowLW=0.3333\pslinewidth](0,1)(4,1)
\end{pspicture}
\end{LTXexample}



\section{Ticks and other marks along a curve}
\subsection{Quick overview}

The macros described below allow you to place tick and other marks along an arbitrary 
parametric curve with placement rules similar to those used by \Lcs{psaxes} in 
the \LPack{pst-plot} package. You have to define a metric function along the curve to 
govern tick placement. That function can be a specified function of {\tt x,y} which 
should increase along the curve, or it can be an function whose increment is a specified 
positive function of {\tt x, y, dx, dy, ds} where the last term is the arc-length element 
that you could specify alternately as {\tt dx dup mul dy dup mul add sqrt}.
% start new material


In addition, a new command \Lcs{Put} is proposed, expanding as appropriate to \Lcs{rput} or \Lcs{uput}. Its syntax is

\begin{BDef}
\LcsStar{Put}\OptArgs\OptArg*{\Largb{<ref>}}\Largr{<position>}\Largb{<stuff>}
\end{BDef}

where the optional {\tt *} blanks the background, the optional \OptArgs\ may be used to specify a rotation 
using any form acceptable to \Lcs{SpecialCoor} (eg, \nxLkeyword{rot=45} or \Lkeyword{rot}\verb|={(1,1)}| 
or \Lkeyword{rot}\verb|=(P)|, and \Larg{ref} takes one of 
two forms: \verb=(a)= a refpt such as {\tt Bl}, in which case \Lcs{rput} is called; (b) a polar form of offset 
(eg, \verb=7pt;30=, or \verb=;(P)= --- in the latter case, \Ldim{pslabelsep} is substituted for the missing 
radius), in which case a modified form of \Lcs{uput} is called. The idea of \Lcs{Put} is to allow  {\tt position}, 
{\tt ref} and {\tt rot} to be specified in any of the forms acceptable to \Lcs{SpecialCoor} and to do so with 
the same output no matter what form is used. The cost of this consistency is that \Lcs{Put} can lead to results 
that differ from \Lcs{uput} in some special cases. 


\subsection{Details}
Suppose you have drawn a parametric curve using \Lcs{psparametricplot}, and you wish to 
indicate some points on the curve using tick-marks like those  on the axes. This is a 
two-step process, the first of which serves to define at the PostScript level a 
number of data arrays containing information about the curve. Those arrays are used 
in the second step to compute tick positions and draw the ticks. The first step is 
to run the macro \Lcs{pscurvepoints}. For example,

\begin{verbatim}
\pscurvepoints[plotpoints=20]{0}{6}{t t t mul 12 div}{Pt}%
\end{verbatim}
makes a virtual (ie, data only---nothing is rendered) polyline with 20 vertices approximating 
the curve $x(t)=t, y(t)=t^2/12$, $0\le t\le 6$. The last argument {\tt Pt} is the root name 
given to the data arrays.  PostScript arrays will be created with the following names: {\tt Pt.X, Pt.Y} 
for the coordinates of the vertices, {\tt PtDelta.X, PtDelta.Y} for the increments between the 
vertices (using, eg, {\tt PtDelta.X[2]=Pt.X[2]-Pt.X[1]}) and {\tt PtNormal.X, PtNormal.Y} for 
a vector normal to {\tt PtDelta.X, PtDelta.Y} in the visual, not mathematical, sense. 
(Both senses are the same if the scales on the axes are identical.) The {\tt Normal} is 
always constructed so as to point ``upward'' (ie, to your left) as you traverse the curve 
in the positive direction. The PostScript variable {\tt unitratio} provides the ratio of 
the unit on the y axis to that on x axis, and {\tt unitratiosq} is its square. All of 
these PostScript objects are stored in the main {\tt pstricks} dictionary \Lps{tx@Dict} 
which should be automatically made available when using many {\tt pstricks} macros. 
If {\tt gs} returns you an error message like
\begin{verbatim}
Error: /undefined in Pt.X
\end{verbatim}
then you may need to enclose the offending PostScript code within a block of the form
\begin{verbatim}
tx@Dict begin ... end
\end{verbatim}
so that the dictionary is made available.

With this preparation, the main tick-making macro may be run. For example,
\begin{verbatim}
\pspolylineticks{Pt}{ dx dy add 3 div }{1}{2}%
\end{verbatim}
looks for data arrays made using \Lcs{pscurvepoints} with the root name {\tt Pt}. The next argument, 
{\tt dx dy add 3 div}, specifies the (PostScript) function of increments that should be used to 
construct the metric. If the keyword \verb|metricInitValue| is defined, eg, with 
\Lcs{psset}\Largb{\Lkeyword{metricInitValue}=2.5}, it is used as the initial value of the metric, 
otherwise it is defined to be 0. In the previous example, the increment function is always 
positive, and care should be taken to guarantee this is so or the results will not be meaningful. 
(If we wanted to use arc-length, the function would have been {\tt ds}, assuming equal scales on 
the axes.) The last two arguments determine the index of the first tick and the number of ticks. 
Tick numbering begins with index 0, so the example says to drop the first tick and draw the 
next 2 ticks. In this example, where all keywords take their default values, ticks are 
potentially located at values on the curve where the metric takes a positive integer value. 
In the arc-length example, the tick with index 0 is at the beginning of the curve, and subsequent 
ticks are at unit distance, measured along the curve. At each index where a tick is drawn, a 
\Lcs{pnode} is created: In this example, you create nodes {\tt PtTick1, PtTick2} on the curve 
where the ticks are located. This is handy for placing labels using, eg, \Lcs{Put}. In 
addition, PostScript data arrays (in this example, {\tt PtTickN.X, PtTickN.Y} of the normals 
at these nodes are stored in the dictionary {\tt TDict}. More importantly, the tangent and 
normal vectors at {\tt PtTick0} etc are constructed as nodes with names {\tt PtTangent0, PtNormal0} 
etc. See the last example below for typical usage.

The shape of the ticks is governed by the keywords \Lkeyword{ticksize} (default value {\tt -4pt 4pt})
 and \Lkeyword{tickwidth} (default value \verb|.5\linewidth|.) With the default settings, ticks 
 are drawn perpendicular the the curve extending {\tt 4pt} to each side. The line
\begin{verbatim}
\pspolylineticks[ticksize=-6pt 6pt]{Pt}{ dx dy add 3 div }{1}{2}%
\end{verbatim}
would draw longer ticks than the default.

Placement of the ticks is governed by the keywords \Lkeyword{Ds} and \Lkeyword{Os}, whose meaning for the 
curve is similar to (but not the same as) the meanings of \Lkeyword{Dx} and \Lkeyword{Ox} with respect to the x axis. 
That is, if {\tt Ds=2} and {\tt Os=0}, ticks will be drawn where the metric takes 
values 0, 2, 4 and so on. More generally, ticks are placed where the metric takes 
value {\tt Os, Os+Ds, Os+2*Ds,...}, as long as those positions are on the curve. If \Lkeyword{Os} 
has an empty value as a result, say, of \verb|\psset{Os=}|, then \Lkeyword{Os} is set internally 
to the initial metric value. If \Lkeyword{Ds} has an empty value, it is set internally to the 
final metric value less the initial metric value, divided by 10. 

To draw major and minor ticks requires two passes---one to draw the minor ticks and then one to draw the major ticks.

Note that a ticks may be placed at arbitrary metric values on the curve by running the macro once for each point, like:
\begin{verbatim}
\pspolylineticks[ticksize=-6pt 6pt,Os=1.3]{Pt}{ dx dy add 3 div }{0}{1}%
\pspolylineticks[ticksize=-6pt 6pt,Os=2.4]{Pt}{ dx dy add 3 div }{0}{1}%
\end{verbatim}

You may also dispense entirely with the tick and use the macro to generate a node sequence 
that can be used to place other graphic objects. For example:
\begin{verbatim}
\pspolylineticks[ticksize=0pt 0pt]{Pt}{ dx dy add 3 div }{0}{3}%
%This defines nodes PtTick0..PtTick2
\multido{\iA=0+1}{3}{\psdot(PtTick\iA)}
\end{verbatim}


There is another way to define a metric function without using increments. If the keywork \Lkeyword{metricFunction} is set to \true, 
then the function you present as an argument to \Lcs{pspolylineticks} must be a function of 
$x$ and $y$ only, and must be designed to increase along the curve. It is useful only in 
those cases where, in essence, the increment function can be explicitly integrated. 
For example, in the elliptical motion of planets and comets around the sun, it is not hard 
to integrate the area function explicitly, and this provides a convenient metric, being proportional to time elapsed.

There is some useful information left in the log by these macros. 
They report the starting and ending values of the metric function, 
the the range of indices for the Tick related arrays.

\subsection{Examples}
The examples in this section make use of very recent (as of May, 2010) versions 
of \LPack{pstricks} and related packages. 
%If the {\tt pst-grapha} package is not available on CTAN,  download it from
%\begin{verbatim}
%http://math.ucsd.edu/~msharpe/pst-grapha.dmg
%\end{verbatim}

The first couple of examples are constructed entirely by hand, and have no interest 
other than to illustrate what is going on under the surface in the simplest case.

\begin{LTXexample}[pos=t]
\begin{pspicture}(-1,-1)(10,4)
\psline[showpoints=true](1,2)(4,0)(9,3)%
\uput[180](1,2){$s=0$}%
\uput[-90](4,0){$s=1$}%
\uput[0](9,3){$s=2$}%
\makeatletter% need to use macro names containing @
\pstVerb{tx@Dict begin %the pstricks dictionary
% declare arrays of length 3 (indices 0,1,2) to hold points, 
% differences and normals
/unitratiosq 1 def % yunit=xunit
/P.X [ 1 4 9 ] def %array of x coords
/P.Y [ 2 0 3 ] def %array of y coords
/PDelta.X [ 0 3 5 ] def % 3=4-1, 5=9-4, 0 never used
/PDelta.Y [ 0 -2 3 ] def % -2=0-2, 3=3-0, 0 never used
% normal to (3,-2) is (2,3), normal to (5,3) is (-3,5)
/PNormal.X [ 2 2 -3 ] def % index 0 =index 1
/PNormal.Y [ 3 3 5 ] def % index 0 = index 1
end }
\def\Ppointcount{2}
\makeatother
% make ticks using metric function with values 0,1,2
\pspolylineticks[Os=.5,Ds=1]{P}{1}{0}{2}
% ticks at s=0.5,1.5 (increment function =1)
\uput[-135](PTick0){$s=0.5$}%
\uput[-45](PTick1){$s=1.5$}%
\end{pspicture}
\end{LTXexample}

\clearpage
Now the same data, but with arc-length as metric. We change the last few lines:

\begin{LTXexample}[pos=t]
\begin{pspicture}(-1,-1)(10,4)
\psline[showpoints=true](1,2)(4,0)(9,3)%
%\uput[180](1,2){$s=0$}%
%\uput[-90](4,0){$s=1$}%
%\uput[0](9,3){$s=2$}%
\makeatletter% need to use macro names containing @
\pstVerb{tx@Dict begin %the pstricks dictionary
% declare arrays of length 3 (indices 0,1,2) to hold points, 
% differences and normals
/unitratiosq 1 def % yunit=xunit
/P.X [ 1 4 9 ] def %array of x coords
/P.Y [ 2 0 3 ] def %array of y coords
/PDelta.X [ 0 3 5 ] def % 3=4-1, 5=9-4, 0 never used
/PDelta.Y [ 0 -2 3 ] def % -2=0-2, 3=3-0, 0 never used
% normal to (3,-2) is (2,3), normal to (5,3) is (-3,5)
/PNormal.X [ 2 2 -3 ] def % index 0 =index 1
/PNormal.Y [ 3 3 5 ] def % index 0 = index 1
end }
\def\Ppointcount{2}
\makeatother
% make ticks using metric function arc-length
\pspolylineticks[Os=1,Ds=1]{P}{ ds }{0}{9}
% ticks at s=1,2... (increment function = distance)
\uput[-135](PTick0){$s=1$}%
\uput[-135](PTick1){$s=2$}%
\end{pspicture}
\end{LTXexample}

\clearpage
Once again the same data, but with metric equal to the x coordinate. Change the last few lines to:

\begin{LTXexample}[pos=t]
\begin{pspicture}(-1,-1)(10,4)
\psline[showpoints=true](1,2)(4,0)(9,3)%
%\uput[180](1,2){$s=0$}%
%\uput[-90](4,0){$s=1$}%
%\uput[0](9,3){$s=2$}%
\makeatletter% need to use macro names containing @
\pstVerb{tx@Dict begin %the pstricks dictionary
% declare arrays of length 3 (indices 0,1,2) to hold points, 
% differences and normals
/unitratiosq 1 def % yunit=xunit
/P.X [ 1 4 9 ] def %array of x coords
/P.Y [ 2 0 3 ] def %array of y coords
/PDelta.X [ 0 3 5 ] def % 3=4-1, 5=9-4, 0 never used
/PDelta.Y [ 0 -2 3 ] def % -2=0-2, 3=3-0, 0 never used
% normal to (3,-2) is (2,3), normal to (5,3) is (-3,5)
/PNormal.X [ 2 2 -3 ] def % index 0 =index 1
/PNormal.Y [ 3 3 5 ] def % index 0 = index 1
end }
\def\Ppointcount{2}
\makeatother
% make ticks using metric function arc-length
\pspolylineticks[metricFunction,Os=1,Ds=2]{P}{ x }{0}{5}
% ticks at x=1,3,... , start at tick index 0, draw 5 ticks
% the tick at s=1 has index 0
% ticks at s=1,2... (increment function = distance)
\uput[-135](PTick0){$s=1$}%
\uput[-135](PTick1){$s=3$}%
\end{pspicture}
\end{LTXexample}

\clearpage
The next example is a smooth path where subticks are drawn first, followed by major ticks. 
The metric is arc-length with initial value $s=1$.
\begin{LTXexample}[pos=t]
\begin{pspicture}(-1,-1)(10,4)
%\parametricplot[algebraic]{0}{9}{(t^2)/9 | 4*Ex(-t)*(1+t+(t^{2})/2+(t^{3})/6)}
\psparametricplot[algebraic]{0}{9}{t^2/9 | sin(t)+1}%
\pscurvepoints{0}{9}{(t^2)/9 | sin(t)+1}{P}%
% make ticks using  arc-length metric
\pspolylineticks[metricInitValue=1,ticksize=-2pt 2pt,Os=1,Ds=.2]{P}{ ds }{1}{56}%
\pspolylineticks[metricInitValue=1,Os=1,Ds=2]{P}{ ds }{0}{6}%
\multido{\iA=1+1,\iB=3+2}{5}{\Put{6pt;(PNormal\iA)}(PTick\iA){\tiny \iB}}
%\nodexn{(PTick\iA)+(10pt;{(PNormal\iA)})}{Q}\rput(Q){\tiny \iB}}%
%\multido{\iA=1+1,\iB=3+2}{5}{\uput{6pt}[{(PNormal\iA)}](PTick\iA){\iB}}%
% ticks at x=1,3,... , start at tick index 0, draw 5 ticks
% the tick at s=1 has index 0
% ticks at s=1,2... (increment function = distance)
\end{pspicture}
\end{LTXexample}

\clearpage
Suppose for the next example that we have an ellipse $x^2/a^2+y^2/b^2=1$ ($a>b$) with 
eccentricity $\epsilon=(1-b^2/a^2)^{1/2}$. With planetary motion in mind, a natural metric 
for the ellipse is the area swept out by the radial line from the focus $(\epsilon a,0)$ 
starting from $(a,0)$ around to an arbitrary location $(x,y)$, where $y>0$, as this quantity 
is proportional to the time elapsed since perihelion. A routine calculation gives the following formula:
\[A=\frac{ab}{2}\arccos\bigg(\frac{x}{a}\bigg)-\frac{\epsilon a y}{2}.\]
Remembering that PostScript's {\tt acos} gives  its result in degrees, not radians, we have the 
following, drawn for the case $a=4$, $b=3$.

\begin{LTXexample}[pos=t]
\begin{pspicture}(-4.5,-.5)(4.5,3.5)
\pstVerb{ /smajor 4 def /sminor 3 def % define semimajor, semiminor 
/ecc 1 sminor smajor div dup mul sub sqrt def % compute eccentricity
/ab smajor sminor mul 2 div def %first coeff
/ea smajor ecc mul 2 div def }% second coeff
\psparametricplot[algebraic]{0}{3.142}{smajor*cos(t) | sminor*sin(t)}%
\pscurvepoints{0}{3.142}{smajor*cos(t) | sminor*sin(t)}{P}%
\pspolylineticks[metricFunction,Ds=2,ticksize=-1.5pt 0]{P}{ ab x smajor div acos %
180 div PI mul mul  ea y mul sub }{1}{9}%
\pnode(! ecc smajor mul 0){S}% focus
\psline[linecolor=lightgray](S)(!smajor 0)%
\multido{\i=1+1}{9}{\psline[linecolor=lightgray](S)(PTick\i)}
\psdot(S)
\end{pspicture}
\end{LTXexample}

\clearpage
The next examples works without visible ticks, using the macros to construct nodes at which other objects will be placed.

\begin{LTXexample}[pos=t]
\begin{pspicture}(-1,-1)(10,4)
\psparametricplot[algebraic]{0}{9}{t| 3*Ex(-t)*(1+t+t^2/2+t^3/6)}
\pscurvepoints{0}{9}{t| 3*Ex(-t)*(1+t+t^2/2+t^3/6)}{P}%
\pspolylineticks[Os=1,Ds=1,ticksize=0 0]{P}{ ds }{0}{9}%
\multido{\i=0+1}{9}{\psdot[dotscale=1.5,dotstyle=o](PTick\i)}%
% ticks at s=1,2,... , start at tick index 0, set 9 ticks
% the tick at s=1 has index 0
% ticks at s=1,2... (increment function = distance)
\multido{\i=0+3}{3}{\Put[rot=(PTangent\i)]{7pt;(PNormal\i)}(PTick\i){PTick\i}}%
\uput[-135](PTick1){$s=2$}%
\end{pspicture}
\end{LTXexample}

This variant also has no visible ticks, but makes a color gradient along the curve based on arc-length from the start.

\begin{LTXexample}[pos=t]
\begin{pspicture}(-1,-1)(10,4)
\psparametricplot[plotpoints=200,linecolor=white]{0}{360}{ t cos 1 add 4 mul t 1 add 20 div ln 2 div 1 add }
\pscurvepoints[plotpoints=200]{0}{360}{ t cos 1 add 4 mul t 1 add 20 div ln 2 div 1 add }{P}%
\pspolylineticks[Os=0,Ds=.2,ticksize=0 0]{P}{ ds }{0}{90}%
\definecolorseries{ctest}{hsb}{last}{green}{violet}
\resetcolorseries[88]{ctest}%
\multido{\iA=0+1,\iB=1+1}{87}{\psline[linewidth=2pt,linecolor=ctest!![\iB](PTick\iA)(PTick\iB)}%
\end{pspicture}
\end{LTXexample}

\clearpage
Here is a another variant of this technique which allows arrows to be placed at locations 
on the curve where the metric takes particular values.

\begin{LTXexample}[pos=t]
\begin{pspicture}(-1,-1)(10,4.5)
\psparametricplot[plotpoints=100]{0}{360}{t cos 1 add 5 mul t sin 1 add 2 mul}
\pscurvepoints[plotpoints=100]{0}{360}{t cos 1 add 5 mul t sin 1 add 2 mul}{P}%
\pspolylineticks[Os=0,Ds=2.3,ticksize=0 0]{P}%
{ ds }{0}{10}% distance
\multido{\i=0+1}{10}{\psrline[arrows=->,arrowscale=1.5](PTick\i)(2pt;{(PTangent\i)})}%
\end{pspicture}
\end{LTXexample}

\section{Troubleshooting}
If you get PostScript errors when you process your file, the  most likely culprit is the 
function you specified to define the metric. There are some  things to look out for:
\begin{itemize}
\item If \Lkeyword{metricFunction}, the function you specify in PostScript code must 
involve only {\tt x} and {\tt y}, and must leave exactly one real value on the stack as a result of 
substituting specific values for {\tt x} and {\tt y}. The function must be strictly increasing on the curve.
\item If \Lkeyword{metricFunction}=\false (the default), the function you specify in PostScript 
code must involve only the variables {\tt x}, {\tt y}, {\tt dx}, {\tt dy}, {\tt ds} (where {\tt ds} 
is defined to be the arc-length element {\tt dx dup mul dy dup mul add sqrt}, and must leave exactly 
one strictly positive real value on the stack when specific values are substituted for those variables. 
The constant function {\tt 1} gives equal weight to each segment in the curve, so in effect it gives 
you the original parametrization, up to a constant factor.
\item If the function you specify in \Lcs{parametricplot} and \Lcs{pscurvepoints} is \Lkeyword{algebraic}, 
make sure you follow precisely the syntax it understands. In complex cases, PostScript may be the safer solution.
\item It is unwise to use a different resolution for \Lcs{psparametricplot} and \Lcs{pscurvepoints}. 
The default value of \Lkeyword{plotpoints}=50 is marginal except for modest curve segments, and 200 should 
suffice for most smooth curves.
\end{itemize}


%--------------------------------------------------------------------------------------
\section{Transparent colors}
%--------------------------------------------------------------------------------------

Transparency is now part of the main \LPack{pstricks} package.
But pay attention, the names and syntax have changed and you need
to run \Lprog{ps2pdf} with the option
\Loption{-dCompatibilityLevel}=1.4.


%--------------------------------------------------------------------------------------
\section{,,Manipulating transparent colors''}
%--------------------------------------------------------------------------------------

\LPack{pstricks-add} supports real transparency and a simulated one with hatch lines:
\begin{lstlisting}
\def\defineTColor{\@ifnextchar[{\defineTColor@i}{\defineTColor@i[]}}
\def\defineTColor@i[#1]#2#3{%     transparency "Colors"
  \newpsstyle{#2}{%
     fillstyle=vlines,hatchwidth=0.1\pslinewidth,
     hatchsep=1\pslinewidth,hatchcolor=#3,#1%
  }%
}
\defineTColor{TRed}{red}
\defineTColor{TGreen}{green}
\defineTColor{TBlue}{blue}
\end{lstlisting}

There are three predefined "'transparent"` colors \verb+TRed+,
\verb+TGreen+, \verb+TBlue+. They are used as \PST{} styles and
not as colors:

\bgroup
\begin{LTXexample}[pos=t,preset=\centering]
\begin{pspicture}(-3,-5)(5,5)
\psframe(-1,-3)(5,5) % objet de base
\psrotate(2,-2){15}{%
  \psframe[style=TRed](-1,-3)(5,5)}
\psrotate(2,-2){30}{%
  \psframe[style=TGreen](-1,-3)(5,5)}
\psrotate(2,-2){45}{%
  \psframe[style=TBlue](-1,-3)(5,5)}
\psframe[linewidth=3pt](-1,-3)(5,5)
\psdots[dotstyle=+,dotangle=45,dotscale=3](2,-2) % centre de la rotation
\end{pspicture}
\end{LTXexample}
\egroup

%--------------------------------------------------------------------------------------
\section{Calculated colors}
%--------------------------------------------------------------------------------------
The \verb+xcolor+ package (version 2.6) has a new feature for defining colors:
\begin{lstlisting}[style=syntax]
  \definecolor[ps]{<name>}{<model>}{< PS code >}
\end{lstlisting}

\verb+model+ can be one of the color models, which \PS will
understand, e.g. \verb+rgb+. With this definition the color is
calculated on the \PS side.
\begin{LTXexample}[pos=t,preset=\centering]
\definecolor[ps]{bl}{rgb}{tx@addDict begin  Red Green Blue end}%
\psset{unit=1bp}
\begin{pspicture}(0,-30)(400,100)
\multido{\iLAMBDA=0+1}{400}{%
  \pstVerb{
    \iLAMBDA\space 379 add dup /lambda exch def
    tx@addDict begin  wavelengthToRGB end
  }%
  \psline[linecolor=bl](\iLAMBDA,0)(\iLAMBDA,100)%
}
\psaxes[yAxis=false,Ox=350,dx=50bp,Dx=50]{->}(-29,-10)(420,100)
\uput[-90](420,-10){$\lambda$[\textsf{nm}]}
\end{pspicture}
\end{LTXexample}


\begin{center}
\newcommand{\Touch}{%
\psframe[linestyle=none,fillstyle=solid,fillcolor=bl,dimen=middle](0.1,0.75)}
\definecolor[ps]{bl}{rgb}{tx@addDict begin Red Green Blue end}%
% Echelle 1cm <-> 40 nm
%         1 nm <-> 0.025 cm
\psframebox[fillstyle=solid,fillcolor=black]{%
\begin{pspicture}(-1,-0.5)(12,1.5)
\multido{\iLAMBDA=380+2}{200}{%
  \pstVerb{
    /lambda \iLAMBDA\space def
    lambda
    tx@addDict begin  wavelengthToRGB end
  }%
 \rput(! lambda 0.025 mul 9.5 sub 0){\Touch}
}
\multido{\n=0+1,\iDiv=380+40}{11}{%
    \psline[linecolor=white](\n,0.1)(\n,-0.1)
    \uput[270](\n,0){\textbf{\white\iDiv}}}
    \psline[linecolor=white]{->}(11,0)
    \uput[270](11,0){\textbf{\white$\lambda$(nm)}}
\end{pspicture}}

\psframebox[fillstyle=solid,fillcolor=black]{%
\begin{pspicture}(-1,-0.5)(12,1)
  \pstVerb{
    /lambda 656 def
    lambda
    tx@addDict begin  wavelengthToRGB end
  }%
 \rput(! 656 0.025 mul 9.5 sub 0){\Touch}
  \pstVerb{
    /lambda 486 def
    lambda
    tx@addDict begin  wavelengthToRGB end
  }%
 \rput(! 486 0.025 mul 9.5 sub 0){\Touch}
   \pstVerb{
    /lambda 434 def
    lambda
    tx@addDict begin  wavelengthToRGB end
  }%
 \rput(! 434 0.025 mul 9.5 sub 0){\Touch}
  \pstVerb{
    /lambda 410 def
    lambda
    tx@addDict begin  wavelengthToRGB end
  }%
 \rput(! 410 0.025 mul 9.5 sub 0){\Touch}
\multido{\n=0+1,\iDiv=380+40}{11}{%
    \psline[linecolor=white](\n,0.1)(\n,-0.1)
    \uput[270](\n,0){\textbf{\white\iDiv}}}
    \psline[linecolor=white]{->}(11,0)
    \uput[270](11,0){\textbf{\white$\lambda$(nm)}}
\end{pspicture}}

\Index{Spectrum} of \Index{hydrogen} emission (Manuel Luque)
\end{center}

\begin{lstlisting}
\newcommand\Touch{%
\psframe[linestyle=none,fillstyle=solid,fillcolor=bl,dimen=middle](0.1,0.75)}
\definecolor[ps]{bl}{rgb}{tx@addDict begin Red Green Blue end}%
% Echelle 1cm <-> 40 nm
%         1 nm <-> 0.025 cm
\psframebox[fillstyle=solid,fillcolor=black]{%
\begin{pspicture}(-1,-0.5)(12,1.5)
\multido{\iLAMBDA=380+2}{200}{%
  \pstVerb{
    /lambda \iLAMBDA\space def
    lambda
    tx@addDict begin  wavelengthToRGB end
  }%
 \rput(! lambda 0.025 mul 9.5 sub 0){\Touch}
}
\multido{\n=0+1,\iDiv=380+40}{11}{%
    \psline[linecolor=white](\n,0.1)(\n,-0.1)
    \uput[270](\n,0){\textbf{\white\iDiv}}}
    \psline[linecolor=white]{->}(11,0)
    \uput[270](11,0){\textbf{\white$\lambda$(nm)}}
\end{pspicture}}

\psframebox[fillstyle=solid,fillcolor=black]{%
\begin{pspicture}(-1,-0.5)(12,1)
  \pstVerb{
    /lambda 656 def
    lambda
    tx@addDict begin  wavelengthToRGB end
  }%
 \rput(! 656 0.025 mul 9.5 sub 0){\Touch}
  \pstVerb{
    /lambda 486 def
    lambda
    tx@addDict begin  wavelengthToRGB end
  }%
 \rput(! 486 0.025 mul 9.5 sub 0){\Touch}
   \pstVerb{
    /lambda 434 def
    lambda
    tx@addDict begin  wavelengthToRGB end
  }%
 \rput(! 434 0.025 mul 9.5 sub 0){\Touch}
  \pstVerb{
    /lambda 410 def
    lambda
    tx@addDict begin  wavelengthToRGB end
  }%
 \rput(! 410 0.025 mul 9.5 sub 0){\Touch}
\multido{\n=0+1,\iDiv=380+40}{11}{%
    \psline[linecolor=white](\n,0.1)(\n,-0.1)
    \uput[270](\n,0){\textbf{\white\iDiv}}}
    \psline[linecolor=white]{->}(11,0)
    \uput[270](11,0){\textbf{\white$\lambda$(nm)}}
\end{pspicture}}

Spectrum of hydrogen emission (Manuel Luque)
\end{lstlisting}



%--------------------------------------------------------------------------------------
\section{Gouraud shading}
%--------------------------------------------------------------------------------------
\begin{quotation}
\Index{Gouraud} shading is a method used in computer graphics to simulate the differing effects of
light and colour across the surface of an object. In practice, Gouraud shading is used to
achieve smooth lighting on low-polygon surfaces without the heavy computational requirements
of calculating lighting for each pixel. The technique was first presented by Henri Gouraud in 1971.\\
~\hfill{\small \url{http://www.wikipedia.org}}
\end{quotation}

PostScript level 3 supports this kind of shading and it can only
be seen with Acroread 7 or later. The syntax is easy:

\begin{lstlisting}[style=syntax]
  \psGTriangle(x1,y1)(x2,y2)(x3,y3){color1}{color2}{color3}
\end{lstlisting}

\psset{unit=0.75cm}

\begin{LTXexample}[pos=t,preset=\centering]
\begin{pspicture}(0,-.25)(10,10)
  \psGTriangle(0,0)(5,10)(10,0){red}{green}{blue}
\end{pspicture}
\end{LTXexample}

\begin{LTXexample}[pos=t,preset=\centering]
\begin{pspicture}(0,-.25)(10,10)
  \psGTriangle*(0,0)(9,10)(10,3){black}{white!50}{red!50!green!95}
\end{pspicture}
\end{LTXexample}

\begin{LTXexample}[pos=t,preset=\centering]
\begin{pspicture}(0,-.25)(10,10)
  \psGTriangle*(0,0)(5,10)(10,0){-red!100!green!84!blue!86}
                               {-red!80!green!100!blue!40}
                               {-red!60!green!30!blue!100}
\end{pspicture}
\end{LTXexample}

\begin{LTXexample}[pos=t,preset=\centering]
\definecolor{rose}{rgb}{1.00, 0.84, 0.88}
\definecolor{vertpommepasmure}{rgb}{0.80, 1.0, 0.40}
\definecolor{fushia}{rgb}{0.60, 0.30, 1.0}
\begin{pspicture}(0,-.25)(10,10)
  \psGTriangle(0,0)(5,10)(10,0){rose}{vertpommepasmure}{fushia}
\end{pspicture}
\end{LTXexample}

\section{Internal color macros}
The internal macros \Lcs{pswavelengthToRGB} and \Lcs{pswavelengthToRGB} can be used for own purposed.
They are defines as follows:

\begin{lstlisting}
\def\pswavelengthToGRAY{ tx@addDict begin wavelengthToGRAY end }
\def\pswavelengthToRGB{ tx@addDict begin wavelengthToRGB Red Green Blue end }
\end{lstlisting}

both macros leave the value(s) on the stack which then can be used for further
manipulating or setting the color with \Lps{setgray} or \Lps{setrgbcolor}. 
For an example see Section~\ref{sec:psMatrix}.

\appendix


%--------------------------------------------------------------------------------------
\clearpage
\section{\nxLcs{resetOptions}}
%--------------------------------------------------------------------------------------

Sometimes it is difficult to know what options, which are changed
inside a long document, are different to the default ones. With
this macro all options belonging to \LPack{pst-plot} can be reset.
This refers to all options of the packages \LPack{pstricks},
\LPack{pst-plot} and \LPack{pst-node}.



%--------------------------------------------------------------------------------------
\section{PostScript}
%--------------------------------------------------------------------------------------

\Index{PostScript} uses the stack system and the LIFO system, "'Last In, First Out"`.

\newlength{\Li}\settowidth{\Li}{Function}
\begin{table}[htbp]
\caption{Some primitive PostScript macros}\label{tab:primpost}
\centering
\ttfamily
    \begin{tabular}{@{} l | r@{ $\rightarrow$ } l @{}}\hline
    \multirow{2}{\Li}{\normalfont\emph{Function}} & \multicolumn{2}{ c }{\normalfont\emph{Meaning}}\\
    &\normalfont\emph{on stack before} & \normalfont\emph{after}\\\hline
    \Lps{add} & $x\quad y$&$x+y$\\ 
    \Lps{sub} & $x\quad y$&$x-y$\\ 
    \Lps{mul} & $x\quad y$&$x\times y$\\ 
    \Lps{div} & $x\quad y$&$x\div y$\\ 
    \Lps{sqrt} & $x$&$\sqrt{x}$\\ 
    \Lps{abs} & $x$&$|x|$\\ 
    \Lps{neg} & $x$&$-x$\\ 
    \Lps{cos} & $x$&$\cos(x)$ ($x$ in degrees)\\ 
    \Lps{sin} & $x$&$\sin(x)$ ($x$  in degrees)\\ 
    \Lps{tan} & $x$&$\tan(x)$ ($x$  in degrees)\\ 
    \Lps{atan} & $y\quad x$&$\angle{(\vec{Ox};\vec{OM})}$ (in degrees of $M(x,y)$)\\ 
    \Lps{ln} & $x$&$\ln(x)$\\ 
    \Lps{log} & $x$&$\log(x)$\\ 
    \Lps{array} & $n$&\normalfont$v$ (of dimension $n$)\\ 
    \Lps{aload} & $v$&$x_1\quad x_2\quad \cdots\quad x_n\quad v$\\ 
    \Lps{astore} & $x_1\quad x_2\quad \cdots\quad x_n\quad v$ & $[v]$\\ 
    \Lps{pop} & $x$ & --\\ 
    \Lps{dup} & $x$ & $x\quad x$ \\\hline
%    \Lps{roll} & $x_1\quad x_2\quad \cdots\quad x_n\quad n p$ &\\\hline
  \end{tabular}
\end{table}


\clearpage
\section{List of all optional arguments for \texttt{pstricks-add}}

\xkvview{family=pstricks-add,columns={key,type,default}}





\nocite{*}
\bgroup
\RaggedRight
\bibliographystyle{plain}
\bibliography{pstricks-add-doc}
\egroup

\printindex




\end{document}
