%% $Id: pst-func-doc.tex 599 2011-11-03 19:38:28Z herbert $
\documentclass[11pt,english,BCOR10mm,DIV12,bibliography=totoc,parskip=false,
   smallheadings, headexclude,footexclude,oneside]{pst-doc}
\usepackage[utf8]{inputenc}
\usepackage{pst-tools}
\let\pstToolsFV\fileversion
\renewcommand\bgImage{}

\lstset{language=PSTricks,
    morekeywords={psPrintValue},basicstyle=\footnotesize\ttfamily}
%
\begin{document}

\title{\texttt{pst-tools}}
\subtitle{Helper functions; v.\pstToolsFV}
\author{Herbert Vo\ss}
\docauthor{}
\date{\today}
\maketitle

\tableofcontents
\psset{unit=1cm}


\section{\Lcs{psPrintValue}}\label{sec:printValue}
This macro allows to \Index{print} single values of a math function. It has the syntax

\begin{BDef}
\Lcs{psPrintValue}\OptArgs\Largb{PostScript code}\\
\Lcs{psPrintValue}\OptArg{algebraic,\ldots}\Largb{x value, algebraic code}
\end{BDef}

Important is the fact, that \Lcs{psPrintValue} works on \PS\ side. For \TeX\ it is only a box of
zero dimension. This is the reason why you have to put it into a box, which reserves horizontal
space.

There are the following valid options for \Lcs{psPrintValue}:

\noindent\medskip
\begin{tabularx}{\linewidth}{@{}l|>{\ttfamily}l>{\ttfamily}lX@{}}
\textrm{name} & \textrm{value}  & \textrm{default}\\\hline
\Lkeyword{PSfont}        & PS font name & Times & only valid \PS font names are possible, e.g. 
    \Lps{Times-Roman}, \Lps{Helvetica}, \Lps{Courier}, \Lps{AvantGard}, \Lps{Bookman}\\
\Lkeyword{fontscale} & <number>     & 10     & the font scale in pt\\
\Lkeyword{valuewidth} & <number>     & 10     & the width of the string for the converted 
    real number; if it is too small, no value is printed\\
\Lkeyword{decimals}  & <number>     & -1     & the number of printed decimals, a negative value
    prints all possible digits.\\ 
\Lkeyword{xShift}  & <number>     & 0     & the x shift in pt for the output, relative to the current point.\\ 
\Lkeyword{algebraic}  & <boolean>     & false     & function in algebraic notation.\\ 
\end{tabularx}

\begin{center}
\psset{fontscale=12}
\makebox[2em]{x(deg)} \makebox[5em]{$\sin x$} \makebox[4em]{$\cos x$}\hspace{1em}
\makebox[5em]{$\sqrt x$}\makebox[7em]{$\sin x+\cos x$}\makebox[6em]{$\sin^2 x+\cos^2 x$}\\[3pt]
\multido{\iA=0+10}{18}{
  \makebox[1em]{\iA}
  \makebox[5em]{\psPrintValue[PSfont=Helvetica,xShift=-10]{\iA\space sin}}
  \makebox[4em][r]{\psPrintValue[PSfont=Courier,fontscale=10,decimals=3,xShift=-20]{\iA\space cos}}\hspace{1em}
  \makebox[5em]{\psPrintValue[dot,valuewidth=15,linecolor=blue,PSfont=AvantGarde]{\iA\space sqrt}}
  \makebox[7em]{\psPrintValue[PSfont=Times-Italic]{\iA\space dup sin exch cos add}}
  \makebox[6em]{\psPrintValue[PSfont=Palatino-Roman]{\iA\space dup sin dup mul exch cos dup mul add}}\\}
\end{center}

\bigskip

\begin{lstlisting}
\psset{fontscale=12}
\makebox[2em]{x(deg)} \makebox[5em]{$\sin x$} \makebox[4em]{$\cos x$}\hspace{1em}
\makebox[5em]{$\sqrt x$}\makebox[7em]{$\sin x+\cos x$}\makebox[6em]{$\sin^2 x+\cos^2 x$}\\[3pt]
\multido{\iA=0+10}{18}{
  \makebox[1em]{\iA}
  \makebox[5em]{\psPrintValue[PSfont=Helvetica,xShift=-10]{\iA\space sin}}
  \makebox[4em][r]{\psPrintValue[PSfont=Courier,fontscale=10,decimals=3,xShift=-20]{\iA\space cos}}\hspace{1em}
  \makebox[5em]{\psPrintValue[dot,valuewidth=15,linecolor=blue,PSfont=AvantGarde]{\iA\space sqrt}}
  \makebox[7em]{\psPrintValue[PSfont=Times-Italic]{\iA\space dup sin exch cos add}}
  \makebox[6em]{\psPrintValue[PSfont=Palatino-Roman]{\iA\space dup sin dup mul exch cos dup mul add}}\\}
\end{lstlisting}

With enabled \Lkeyword{algebraic} option there must be two arguments, separated by a comma.
The first one is the x value as a number, which can also be PostScript code, which leaves a
number on the stack. The second part is the function described in algebraic notation.
Pay attention, in algebraic notation angles must be in radian and not degrees.

\begin{center}
\psset{algebraic, fontscale=12}% All functions now in algebraic notation
\makebox[2em]{x(deg)} \makebox[5em]{$\sin x$} \makebox[4em]{$\cos x$}\hspace{1em}
\makebox[5em]{$\sqrt x$}\makebox[7em]{$\sin x+\cos x$}\makebox[6em]{$\sin^2 x+\cos^2 x$}\\[3pt]
\multido{\rA=0+0.1}{18}{\makebox[1em]{\rA}
  \makebox[5em]{\psPrintValue[PSfont=Helvetica,xShift=-10]{\rA, sin(x)}}
  \makebox[4em][r]{\psPrintValue[PSfont=Courier,fontscale=10,decimals=3,xShift=-20]{\rA,cos(x)}}\hspace{1em}
  \makebox[5em]{\psPrintValue[dot,valuewidth=15,linecolor=blue,PSfont=AvantGarde]{\rA,sqrt(x)}}
  \makebox[7em]{\psPrintValue[PSfont=Times-Italic]{\rA,sin(x)+cos(x)}}
  \makebox[6em]{\psPrintValue[PSfont=Palatino-Roman]{\rA,sin(x)^2+cos(x)^2}}\\}
\end{center}

\bigskip

\begin{lstlisting}
\psset{algebraic, fontscale=12}% All functions now in algebraic notation
\makebox[2em]{x(deg)} \makebox[5em]{$\sin x$} \makebox[4em]{$\cos x$}\hspace{1em}
\makebox[5em]{$\sqrt x$}\makebox[7em]{$\sin x+\cos x$}\makebox[6em]{$\sin^2 x+\cos^2 x$}\\[3pt]
\multido{\rA=0+0.1}{18}{\makebox[1em]{\rA}
  \makebox[5em]{\psPrintValue[PSfont=Helvetica,xShift=-10]{\rA, sin(x)}}
  \makebox[4em][r]{\psPrintValue[PSfont=Courier,fontscale=10,decimals=3,xShift=-20]{\rA,cos(x)}}\hspace{1em}
  \makebox[5em]{\psPrintValue[dot,valuewidth=15,linecolor=blue,PSfont=AvantGarde]{\rA,sqrt(x)}}
  \makebox[7em]{\psPrintValue[PSfont=Times-Italic]{\rA,sin(x)+cos(x)}}
  \makebox[6em]{\psPrintValue[PSfont=Palatino-Roman]{\rA,sin(x)^2+cos(x)^2}}\\}
\end{lstlisting}

\clearpage
\section{List of all optional arguments for \texttt{pst-tools}}

\xkvview{family=pst-tools,columns={key,type,default}}




\bgroup
\raggedright
\nocite{*}
\bibliographystyle{plain}
\bibliography{pst-tools-doc}
\egroup

\printindex



\end{document}


