\documentclass[dvips,11pt]{article}
\usepackage{amsmath}
\usepackage{amsthm} 
\usepackage{graphicx}
\usepackage{pstricks}
\usepackage{multido,pst-node,pst-bspline,pstricks-add}
\usepackage{amssymb}
\usepackage[parfill]{parskip}
\usepackage{hyperref}
\hyphenation{Post-Script}
%\date{}                                           % Activate to display a given date or no date
\begin{document}
\begin{center}{\Large Cubic B-splines Using PSTricks\\[12pt]
\large Michael Sharpe\\[10pt]
msharpe@ucsd.edu}
\end{center}
A cubic uniform B-spline curve with control points $B_0 \ldots B_n$ is a curve parametrized by the interval $[0,n]$, which is $C^2$-continuous (that is, has continuous curvature) and is on each interval $[k-1,k]$ given by a cubic B\'ezier curve whose control points are derived from the $(B_k)$. These curves are discussed in any reasonably modern text on Numerical Analysis. One easily accessible source is the UCLA lecture notes of Kirby Baker:

\noindent\url{http://www.math.ucla.edu/~baker/149.1.02w/handouts/dd_splines.pdf}

I'll focus on two special cases: (i) relaxed, uniform B-splines; (ii) periodic, uniform B-splines. Uniform refers to the condition mentioned in the first paragraph: each B\'ezier sub-curve is parametrized by an interval of  length~1. Relaxed means that the curvature at the endpoints $t=0, t=n$ is zero. Periodic means in effect that the $B_i$ repeat periodically, and the curve generated is a closed curve. 

\section{Relaxed, Open B-spline} The algorithm has the following steps.

\begin{itemize}
\item The curve starts at $B_0$ and ends at $B_n$.
\item Divide each line $B_{k-1}B_k$ into equal thirds, with subdivision points labeled $R_{k-1}$, $L_k$ respectively, so that $B_k$ has $L_k$ as its immediate neighbor to the left, and $R_k$ as its immediate neighbor to the right.
\item For $0<k<n$, divide the line segment $L_kR_k$ in half, letting $S_k$ denote the midpoint. In effect, for $0< k<n$, $S_k=(B_{k-1}+4B_k+B_{k+1})/6$.
\item Let $S_0=B_0$ and $S_n=B_n$.
\item For $0<k\le n$, construct the cubic B\'ezier curve with control points $S_{k-1}$, $R_{k-1}$, $L_k$, $S_k$, parametrized by $k-1\le t\le k$.
\end{itemize}

The {\tt pst-Bspline} package implements this algorithm as \verb|\psBspline|, whose simplest form is, for example

\noindent
\verb|\psBspline(.5,.5)(2,0)(5,2)(6,4)(4,5)(2,4)|

\noindent The coordinates are the B-spline control points. Aside from the usual keywords, like {\tt linestyle}, {\tt linecolor} and {\tt arrows}, there is a Boolean keyword {\tt showframe}. The effect of {\tt showframe=true} is to show the intermediate points and lines in the algorithm described above.

There is another optional argument that can be applied if you wish to be able to refer to any of the points constructed in the algorithm. By example,  

\noindent
\verb|\psBspline{B}(.5,.5)(2,0)(5,2)(6,4)(4,5)(2,4)|

\noindent sets the root of the naming scheme to {\tt B}, the effect of which is that the B-spline control points will be nodes of type \verb|\pnode| with names {\tt B0}, {\tt B1} and so on, the other points being similarly named {\tt BL0}, {\tt BL1}, ... , {\tt BR0}, {\tt BR1}, ... , {\tt BS0}, {\tt BS1}, ... . For example, to draw a line between {\tt BL1} and {\tt BS4}, just use \verb|\ncline(BL1)(BS4)|. 

The  algorithm is implemented entirely in PSTricks code with PostScript specials, but no PostScript header file, depending for the most part on the flexibility of nodes, and above all the \verb|\multido| macro, which allows one to construct with relative ease items that look and feel like arrays. Use of \verb|\SpecialCoor| is essential.

There is a closely related macro \verb|\psBsplineE| which removes the first and last B\'ezier segments, much as \verb|\psecurve| acts in relation to \verb|\pscurve|, allowing one one to draw B-splines with non-zero curvature at the endpoints.

\begin{verbatim}
\documentclass{article}
\usepackage{pstricks}
\usepackage{multido,pst-node,pst-bspline}
\pagestyle{empty}
\begin{document}
\SpecialCoor % essential for pst-bspline package
\psset{unit=.6in}
\begin{pspicture}[showgrid=true](-.5,-.5)(6,5)
\psBspline[showframe=true]{B}(.5,.5)(2,0)(5,2)(6,4)(4,5)(2,4)
\multido{\i=0+1}{5}{\uput[20](B\i){B\i}}
\uput[90](B5){B5}
\uput[90](BS1){S1}
\uput[90](BS2){S2}
\uput[180](BS3){S3}
\uput[270](BS4){S4}
\uput[-45](BR1){R1}
\uput[-45](BL2){L2}
\end{pspicture}
\end{document}
\end{verbatim} 

\vspace{1pc}
\begin{center}
\psset{unit=.6in}
\begin{pspicture}[showgrid=true](-.5,-.5)(6,5)
\psBspline[showframe=true]{B}(.5,.5)(2,0)(5,2)(6,4)(4,5)(2,4)
\multido{\i=0+1}{5}{\uput[20](B\i){B\i}}
\uput[90](B5){B5}
\uput[90](BS1){S1}
\uput[90](BS2){S2}
\uput[180](BS3){S3}
\uput[270](BS4){S4}
\uput[-45](BR1){R1}
\uput[-45](BL2){L2}
\end{pspicture}
\end{center}

\begin{verbatim}
\documentclass{article}
\usepackage{pstricks}
\usepackage{multido,pst-node,pst-bspline}
\pagestyle{empty}
\begin{document}
\SpecialCoor % essential for pst-bspline package
\psset{unit=.6in}
\begin{pspicture}[showgrid=true](-.5,-.5)(6,5)
\psBsplineE[showframe=true]{B}(.5,.5)(2,0)(5,2)(6,4)(4,5)(2,4)
\multido{\i=0+1}{5}{\uput[20](B\i){B\i}}
\uput[90](B5){B5}
\uput[90](BS1){S1}
\uput[90](BS2){S2}
\uput[180](BS3){S3}
\uput[270](BS4){S4}
\uput[-45](BR1){R1}
\uput[-45](BL2){L2}
\end{pspicture}
\end{document}
\end{verbatim} 

\vspace{1pc}
\begin{center}
\psset{unit=.6in}
\begin{pspicture}[showgrid=true](-.5,-.5)(6,5)
\psBsplineE[showframe=true]{B}(.5,.5)(2,0)(5,2)(6,4)(4,5)(2,4)
\multido{\i=0+1}{5}{\uput[20](B\i){B\i}}
\uput[90](B5){B5}
\uput[90](BS1){S1}
\uput[90](BS2){S2}
\uput[180](BS3){S3}
\uput[270](BS4){S4}
\uput[-45](BR1){R1}
\uput[-45](BL2){L2}
\end{pspicture}
\end{center}

\section{Periodic B-spline} The result here is a closed curve. The algorithm is essentially the same as in the preceding case, except:

\begin{itemize}
\item Extend $B_i$ periodically with period $n+1$, so that $B_{n+1}=B_0$ and $B_{n+2}=B_1$.
\item Construct  $R_i$, $L_i$ for $0<i<n+2$, as above.
\item Construct $S_k$ as above (midpoint of $L_kR_k$), for $0<k<n+2$.
\item Set $S_0=S_{n+1}$.
\item For $0<k\le n+1$, construct the cubic B\'ezier curve with control points $S_{k-1}$, $R_{k-1}$, $L_k$, $S_k$, parametrized by $k-1\le t\le k$.
\end{itemize}

The macro in this case is \verb|\psBsplineC|, where the {\tt C} stands for Closed. The code, being implemented as a \verb|\pscustom| object, does not accept the {\tt doubleline} keyword, but does accept, for example,
 
\noindent\verb|fillstyle=solid,fillcolor=gray|

\begin{verbatim}
\documentclass{article}
\usepackage{pstricks}
\usepackage{multido,pst-node,pst-bspline}
\pagestyle{empty}
\begin{document}
\SpecialCoor % essential for pst-bspline package
\psset{unit=.6in}
\begin{pspicture}[showgrid=true](-.5,-.5)(6,5)
\psBsplineC[showframe=true]{B}(.5,.5)(2,0)(5,2)(6,4)(4,5)(2,4)
\multido{\i=0+1}{5}{\uput[20](B\i){B\i}}
\uput[90](B5){B5}\uput[90](BS1){S1}
\uput[90](BS2){S2}\uput[180](BS3){S3}
\uput[270](BS4){S4}\uput[-45](BR1){R1}
\uput[-45](BL2){L2}
\end{pspicture}
\end{document}
\end{verbatim} 

\vspace{1pc}
\begin{center}
\psset{unit=.6in}
\begin{pspicture}[showgrid=true](-.5,-.5)(6,5)
\psBsplineC[showframe=true]{B}(.5,.5)(2,0)(5,2)(6,4)(4,5)(2,4)
\multido{\i=0+1}{5}{\uput[20](B\i){B\i}}
\uput[90](B5){B5}
\uput[90](BS1){S1}
\uput[90](BS2){S2}
\uput[180](BS3){S3}
\uput[270](BS4){S4}
\uput[-45](BR1){R1}
\uput[-45](BL2){L2}
\end{pspicture}
\end{center}
\section{Related constructions}
There are in addition three additional macros that draw similar curves, but organized in a slightly different way. They are particularly useful when there is a sequence of points already defined as \verb|\pnode|s. Here is a simple way to define such a sequence.
\subsection{The {\tt pnodes} macro}
The line
\begin{verbatim}
\pnodes{P}(2,1.5)(3,4)(5,1)
\end{verbatim}
defines a sequence of \verb|\pnode|s with the node root {\tt P}: {\tt P0}=(2,1.5), {\tt P1}=(3,4) and {\tt P0}=(5,1). The sequence may be any (reasonable) length. The macro leaves an entry in the console saying that it has defined nodes {\tt P0 .. P2}. The three new macros are:
\begin{verbatim}
\psBsplineNodes{<node root>}{<top index>}
\psBsplineNodesC{<node root>}{<top index>}
\psBsplineNodesE{<node root>}{<top index>}
\end{verbatim}
corresponding to the macros \verb|\psBspline|, \verb|\psBsplineC| and \verb|\psBsplineE|. The difference is that the macros with {\tt Nodes} in the name have as arguments  the root node name and the  last index, rather than the list of points. For example, with the above definition of {\tt P} in force, \verb|\psBsplineNodes{P}{2}| has exactly the same effect as  \verb|\psBspline(2,1.5)(3,4)(5,1).|

\section{B-spline Interpolation}
This is the inverse problem. Being given points $(S_k)_{0\le k\le n}$, the goal is to produce the B-spline control points $B_k$ leading to the points $S_k$, so that the associated B-spline curve interpolates the $S_k$. 
\subsection{Open curve}
We discuss first the case of an open, uniform B-spline curve with  relaxed endpoints. According to the discussion above, we have to solve the equations
\begin{align*}
B_0&=S_0\\
B_0+4B_1+B_2&=6S_1\\
B_1+4B_2+B_3&=6S_2\\
\cdots&\\
B_{n-2}+4B_{n-1}+B_n&=6S_{n-1}\\
B_n&=S_n
\end{align*}
for the $B_k$. In matrix form, this becomes the tridiagonal system
\[\begin{pmatrix}4&1\\
1&4&1\\
&1&4&1\\
&&\cdots&&1\\
&&&1&4\end{pmatrix}
\begin{pmatrix}B_1\\B_2\\B_3\\ \cdots\\ B_{n-1}\end{pmatrix}=ß
\begin{pmatrix}6S_1-S_0\\6S_2\\6S_3\\ \cdots\\6S_{n-1}-S_{n}\end{pmatrix}
\]
The LU decomposition of the tridiagonal matrix may be seen to take the form
\[
\begin{pmatrix}1\\
m_1&1\\
&m_2&1\\
&&\cdots\\
&&&m_{n-2}&1\end{pmatrix}
\begin{pmatrix}m_1^{-1}&1\\
&m_2^{-1}&1\\
&&m_3^{-1}&1\\
&&&\cdots&1\\
&&&&m_{n-1}^{-1}\end{pmatrix}
\]
where $m_1=1/4$, $m_{k+1}=1/(4-m_k)$ for $k=1,\cdots,n-2$. The solution of the original system is therefore accomplished in two steps, introducing intermediate points $(R_k)$, by (in pseudo-code)
\begin{verbatim}
R_1=6*S_1-S_0
for i=2 to n-2
  R_i=6*S_i-m_{i-1}* R_{i-1}
R_{n-1}=(6*S_{n-1}-S_n)-m_{n-2}*R_{n-2}
B_{n-1}=m_{n-1}*R_{n-1}
for i=n-2 downto 1
  B_i=m_i*(R_i-B_{i+1})
\end{verbatim}
The code for the \verb|\psBsplineInterp| command uses this algorithm to solve for the $B_k$ as nodes, except that in order to save node memory, the $B$ nodes are substituted in place for the $R$ nodes, so that, for example, the first step becomes \verb|B_1=6*S_1-S_0|. 

Assuming you have previously defined nodes {\tt S0} $\cdots ${\tt S4}, 
\begin{verbatim}
\psBsplineInterp{S}{4}
\end{verbatim}
will construct a sequence {\tt SB0} $\cdots ${\tt SB4} of nodes at the B-spline control points for the relaxed, uniform cubic B-spline interpolating the {\tt Sk}, and this curve may then be rendered with the command
\begin{verbatim}
\psBsplineNodes{SB}{4}
\end{verbatim}
If you don't care about keeping track of the internal operations and names for nodes, you may generate the curve directly with, for example,
\begin{verbatim}
\psbspline(0,0)(.5,.1)(1.5,.6)(2.5,1.4)(3.5,1.8)(4.5,1.7)%
(5.8,1.0)(7.5,.25)(10,0)
\end{verbatim}

\subsection{Closed (periodic) case}
We turn now to the periodic uniform B-spline curve interpolating $n$ points $S_0$,...,$S_{n-1}$. Extend the sequence periodically with period $n$, so that $S_n=S_0$, $S_{n+1}=S_1$, $S_{-1}=S_{n-1}$, and so on. In order to find the periodic control points $B_k$, we have to solve the $n$ equations
\begin{align*}
B_n+4B_1+B_2&=6S_1\\
B_1+4B_2+B_3&=6S_2\\
\cdots&\\
B_{n-2}+4B_{n-1}+B_n&=6S_{n-1}\\
B_{n-1}+4B_n+B_1&=6S_n
\end{align*}
for the $B_k$, $1\le k\le n$. In matrix form, this becomes the system
\[\begin{pmatrix}4&1&&&1\\
1&4&1\\
&1&4&1\\
&&\cdots&&1\\
1&&&1&4\end{pmatrix}
\begin{pmatrix}B_1\\B_2\\B_3\\ \cdots\\ B_{n}\end{pmatrix}=ß
\begin{pmatrix}6S_1\\6S_2\\6S_3\\ \cdots\\6S_{n}\end{pmatrix}
\]
Let $(x_k,y_k)=6S_k$. We perform Gaussian elimination on the matrix
\[\begin{pmatrix}4&1&&&1&x_1&y_1\\
1&4&1&&&x_2&y_2\\
&1&4&1&&x_3&y_3\\
&&\cdots&&1\\
1&&&1&4&x_n&y_n\end{pmatrix}
\]
As in the previous case, let $m_1=0.25$, $m_k=1/(4-m_{k-1})$ for $k\ge 2$. The factor $m_k$ will be the multiplier of row $k$ after the previous row operation, in order to normalize the row. These are the steps in the procedure.
\begin{itemize}
\item Initialize: multiply row 1 by $m_1$ so that its first entry (1,1) is 1. Replace $x_1$ by $m_1 x_1$ and $y_1$ by $m_1 y_1$. Entry $(1,n)$ is $m_1$.
\item Subtract new row 1 from row 2 and multiply the resulting row by $m_2$. The leading entry (2,1) becomes $1$. Entry $(2,n)$ becomes $-m_1m_2$, and $x_2, y_2$ are updated to $m_2(x_2- x_1)$, $m_2(y_2-y_1)$. The superdiagonal entry (2,3) is the only other non-zero entry, and its new value is $m_2$.
\item Subtract new row 1 from row $n$, so that its leading entry $(n,2)$ is $-m_1$.
\item Subtract new row 2 from row 3 and multiply the result by $m_3$. The leading entry (3,3) becomes $1$ and the entry $(3,n)$ becomes $m_1m_2$, with $x_3, y_3$ updating to $m_3(x_3-x_2)$, $m_3(y_3-y_2)$. The superdiagonal entry (3,4) is now $m_3$.
\item Subtract new row 2 times $-m_1$ from row $n$, whose leading entry $(n,3)$ is now $m_1m_2$.
\item Continue in this way until row $n-2$ has been subtracted as above from row $n-1$, multiplying the result by $m_{n-1}$, and a suitable multiple has been subtracted from row $n$. The leading entry of row $n-1$ (column $n-1$) is $1$ and its $n^{\text{th}}$ entry is $1-(-1)^{n}m_1\cdots m_{n-2}$. Row $n$ has leading entry in column $n-1$, equal to $1$.
\item Finally, subtract an appropriate multiple of row $n-1$ from row $n$ so that row $n$ has leading entry in column $n$. The resulting matrix is upper triangular, and we may now substitute back starting from the last row to give a complete reduction.
\end{itemize}
Here are the steps in pseudocode. We keep track of row $n$ with the array $b_k$, column $n$ with the array $c_k$. The indices for both run from 1 to $n$.
\begin{verbatim}
m(1)=0.25
for k=2 to n-1
   m(k)=1/(4-m(k-1))
b(1)=1
b(n-1)=1
b(n)=4
c(n-1)=1% don't need c(n), =b(n)
%multiply first row by m1
c(1)=m(1)
x(1)=m(1)*x(1)
y(1)=m(1)*y(1)
for k=2 to n-1
  %subtract normalized row k-1 from row k, renormalize row k
  c(k)=m(k)*(c(k)-c(k-1))%note that initially, c(k)=0 for 1<k<n-1
  x(k)=m(k)*(x(k)-x(k-1))
  y(k)=m(k)*(y(k)-y(k-1))
  %subtract normalized row k-1 times b(k-1) from row n
  b(k)=b(k)-b(k-1)*m(k-1)
  b(n)=b(n)-c(k-1)*b(k-1)
  x(n)=x(n)-x(k-1)*b(k-1)
  y(n)=y(n)-y(k-1)*b(k-1)
% subtract row n-1 times b(n-1) from row n, renormalize by 1/b(n)
b(n)=b(n)-b(n-1)*c(n-1)
x(n)=(x(n)-x(n-1)*b(n-1))/b(n)
y(n)=(y(n)-y(n-1)*b(n-1))/b(n)
%work back
x(n-1)=x(n-1)-c(n-1)*x(n)
y(n-1)=y(n-1)-c(n-1)*y(n)
for k=n-2 downto 1
   x(k)=x(k)-m(k)* x(k+1)-c(k)*x(n)
   y(k)=y(k)-m(k)* y(k+1)-c(k)*y(n)
\end{verbatim}
This algorithm is implemented in \TeX/PostScript code in {\tt pst-Bspline.tex} and may be invoked using the macro 
\begin{verbatim}
\psBsplineInterpC{<node root>}{<index>}
\end{verbatim}
 You must previously  have  defined a sequence, say {\tt S0} $\cdots$ {\tt S100} of \verb|\pnode|s that you plan to interpolate with a closed curve. Then 
\begin{verbatim}
\psBsplineInterpC{S}{100}
\end{verbatim}
constructs the sequence {\tt SB0} $\cdots$ {\tt SB100} of B-spline control points (appending~{\tt B} to the root name) for a closed curve interpolating  {\tt S0} $\cdots$ {\tt S100}, which may then be rendered with the command 
\begin{verbatim}
\psBsplineNodesC{SB}{100}
\end{verbatim}
with any keywords options you wish.

The following example illustrates that there is a difference between \verb|\psccurve| and  B-spline interpolation, the former having a rounder appearance. Generally speaking, B-spline interpolation comes closer to minimizing the average curvature.
\begin{verbatim}
\documentclass{article}
\usepackage{pstricks}
\usepackage{pst-bspline,pstricks-add}
\begin{document}
\begin{pspicture}[showgrid=true](-.5,-.5)(6,5)
\pnodes{P}(0,1)(2,0)(5,2)(6,4)(4,5)(2,4)%
\psBsplineInterpC{P}{5}%
\psBsplineNodesC*[linecolor=gray!40]{PB}{5}%
\psccurve[linecolor=red,showpoints=true](0,1)(2,0)(5,2)(6,4)(4,5)(2,4)
\end{pspicture}
\end{document}
\end{verbatim}

\vspace{1pc}
\begin{center}
Slight difference between psccurve and B-spline interpolation\\
\vspace*{2pc}
\begin{pspicture}[showgrid=true](-.5,-.5)(6,5)
\pnodes{P}(0,1)(2,0)(5,2)(6,4)(4,5)(2,4)%
\psBsplineInterpC{P}{5}%
\psBsplineNodesC*[linecolor=gray!40]{PB}{5}%
\psccurve[linecolor=red,showpoints=true](0,1)(2,0)(5,2)(6,4)(4,5)(2,4)
\end{pspicture}
\end{center}
A B-spline curve can in many cases provide a good function interpolation mechanism, but the result is not guaranteed to be the graph of a function.
\begin{verbatim}
\begin{center}
\begin{pspicture}[showgrid=true](-.5,-.5)(6,4)
\psdots(0,3.5)(1,.5)(3,2.5)(4,0)(5,2)(6,.5)
\psaxes(0,0)(-.5,-.5)(6,4)
\psbspline(0,3.5)(1,.5)(3,2.5)(4,0)(5,2)(6,.5)
\end{pspicture}
\end{center}
\end{verbatim}

\begin{center}
\begin{pspicture}(-.5,-.5)(6,4)
\psdots(0,3.5)(1,.5)(3,2.5)(4,0)(5,2)(6,.5)
\psbspline(0,3.5)(1,.5)(3,2.5)(4,0)(5,2)(6,.5)
\psaxes(0,0)(-.5,-.5)(6,4)
\end{pspicture}
\end{center}
\vspace{12pt}

\begin{verbatim}
\documentclass{article}
\usepackage{graphicx}
\usepackage{pstricks}
\usepackage{pst-bspline,pstricks-add}
\begin{document}
\psset{unit=.25in}
\begin{pspicture}[showgrid=true](-.5,-.5)(6,5)
\pnodes{P}(0,1)(2,0)(5,2)(6,4)(4,5)(2,4)
\pnode(3,3){C}
\multido{\ra=0+.05,\rb=1+.05,\i=30+1}{40}{%
\psBsplineC*[linecolor=blue!\i!brown]{B}%
([nodesep=\ra]{C}P0)([nodesep=\ra]{C}P1)%
([nodesep=\ra]{C}P2)([nodesep=\ra]{C}P3)%
([nodesep=\ra]{C}P4)([nodesep=\ra]{C}P5)}
\end{pspicture}
\end{document}
\end{verbatim}

\vspace{1pc}
\begin{center}
\psset{unit=.25in}
\begin{pspicture}[showgrid=true](-.5,-.5)(6,5)
\pnodes{P}(0,1)(2,0)(5,2)(6,4)(4,5)(2,4)
\pnode(3,3){C}
\multido{\ra=0+.05,\rb=1+.05,\i=30+1}{40}{%
\psBsplineC*[linecolor=blue!\i!brown]{B}%
([nodesep=\ra]{C}P0)([nodesep=\ra]{C}P1)%
([nodesep=\ra]{C}P2)([nodesep=\ra]{C}P3)%
([nodesep=\ra]{C}P4)([nodesep=\ra]{C}P5)}
\end{pspicture}
\end{center}
\section{Thick B-spline curves}
Inspired by the package {\tt pst-thick}, we provide a similar option for curves generated as B-spline interpolations. The new macro that accomplishes this is 
\begin{verbatim}
\thickBspline#1#2#3#4
%#1=root | #2=nsegments | #3=thickness | #4=items to clip
\end{verbatim}
 which expects the following data.
 \begin{itemize}
\item
A node sequence. This can be constructed with a command like
\begin{verbatim}
\pnodes{S}(0,0)(5,1)(4,4)(1,3)%
\end{verbatim}
which declares nodes S0..S3, and sets the macro \verb|\Snodecount| to 3.
\item An interpolation command, such as
\begin{verbatim}
\psBsplineInterp{S}{\Snodecount}%
\end{verbatim}
which creates a framework of B-spline control points {\tt SB0..SB3}.
\item Create the interpolating curve and the B\'ezier control points for its components, with names like {\tt SBR0..SBR2,SBL1..SBL3} etc, using
\begin{verbatim}
\psBsplineNodes[linestyle=none]{SB}{\Snodecount}%
\end{verbatim}
(The {\tt [linestyle=none]} may be omitted if you want the curve to show.)
\item Create a clipping path of specified thickness around the interpolating curve and place graphics to be clipped:
\begin{verbatim}
\thickBspline[plotpoints=50,linestyle=none]{S}{3}{20pt}{\psline[linecolor=red,linestyle=solid](0,0)(6,6)}%
\end{verbatim}
(The {\tt [linestyle=none]} controls whether the clipping path is rendered, and {\tt plotpoints} controls the number of subdivisions of each B\'ezier component. Its default value is 50.)
\end{itemize}
The clipping path is drawn by default positively oriented so that objects are clipped to its interior. By specifying the keyword {\tt reverseclip}, the clipping path will be reversed so that objects are clipped to the exterior.

\begin{verbatim}
\documentclass{article}
\usepackage{pstricks}
\usepackage{pst-bspline,pstricks-add}
\begin{document}
\begin{pspicture}[showgrid=true](-.5,-.5)(6,5)
\pnodes{S}(1,0)(5,1)(4,4)(1,3)%
\psdots(1,0)(5,1)(4,4)(1,3)%
\psBsplineInterp{S}{\Snodecount}%
\psBsplineNodes[linestyle=none]{SB}{\Snodecount}%
\thickBspline[plotpoints=50,linestyle=none]{S}{3}{20pt}%
{\psframe[fillstyle=crosshatch](-1,-1)(6,6)}%
\end{pspicture}
\end{document}
\end{verbatim}

\begin{center}
\begin{pspicture}[showgrid=true](-.5,-.5)(6,5)
\pnodes{S}(1,0)(5,1)(4,4)(1,3)%
\psdots(1,0)(5,1)(4,4)(1,3)%
\psBsplineInterp{S}{\Snodecount}%
\psBsplineNodes[linestyle=none]{SB}{\Snodecount}%
\thickBspline[plotpoints=50,linestyle=none]{S}{\Snodecount}{20pt}{\psframe[fillstyle=crosshatch](-1,-1)(6,6)}%
\end{pspicture}
\end{center}

\end{document}  