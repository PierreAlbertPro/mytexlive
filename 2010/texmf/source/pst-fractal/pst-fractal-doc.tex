%% $Id: pst-func-doc.tex 273 2010-01-26 18:28:55Z herbert $
\documentclass[11pt,english,BCOR10mm,DIV12,bibliography=totoc,parskip=false,
   smallheadings, headexclude,footexclude,oneside]{pst-doc}
\usepackage[utf8]{inputenc}
\usepackage{pst-fractal,pst-exa}
\let\pstFV\fileversion
\renewcommand\bgImage{\includegraphics[scale=1.5]{images/pst-fractal-doc-tmp-1.pdf}}
\def\PSLenv{\Lenv{pspicture}}

\lstset{language=PSTricks,basicstyle=\footnotesize\ttfamily}
%
\begin{document}

\title{\texttt{pst-fractal}}
\subtitle{Plotting fractals; v.\pstFV}
\author{Herbert Vo\ss}
\docauthor{}
\date{\today}
\maketitle

\tableofcontents

\clearpage

\begin{abstract}
\noindent
The well known \LPack{pstricks} package offers excellent macros to insert more or less complex 
graphics into a document. \LPack{pstricks} itself is the base for several other additional packages, 
which are mostly named \verb+pst-xxxx+, like \LPack{pst-fractal}.

This version uses the extended keyval package \LPack{xkeyval}, so be sure that you have installed
this package together with the spcecial one \LPack{pst-xkey} for PSTricks. The \LPack{xkeyval}
package is available at \url{CTAN:/macros/latex/contrib/xkeyval/}.
It is also important that after \LPack{pst-fractal} no package is loaded, which uses the old keyval interface.

The fractals are really big, which is the reason why this document is about 15 MByte
when you run it without using the external png-images.
\end{abstract}%

All images in this documentation were converted to the \Lext{jpg} format to get
a small pdf file size. When using the pdf format for the images the file size will be
more than 20 MBytes. However, having a small file size will lead into a bad image
resolution. Run the examples as single documents to see how it will be in
high quality.


\section{Sierpinski triangle}

The triangle must be given by three mandatory arguments. Depending to the kind of
arguments it is one of the two possible versions:

\begin{BDef}
\Lcs{psSier}\OptArgs\coord0\coord1\coord2\\
\Lcs{psSier}\OptArgs\coord0\Largb{Base}\Largb{Recursion}
\end{BDef}

In difference to \Lcs{psfractal} it doesn't reserve any space, this is the
reason why it should be part of a \PSLenv{} environment.

\begin{PSTexample}[pos=l]
\begin{pspicture}(5,5)
  \psSier(0,0)(2,5)(5,0)
\end{pspicture}
\end{PSTexample}


\begin{PSTexample}[pos=l]
\begin{pspicture}(5,5)
\psSier[linecolor=blue!70,
   fillcolor=red!40](0,0){5cm}{4}
\end{pspicture}
\end{PSTexample}


\section{Julia and Mandelbrot sets}

The syntax of the \Lcs{psfractal} macro is simple

\begin{BDef}
\Lcs{psfractal}\OptArgs\coord0\coord1
\end{BDef}
All Arguments are optional, \Lcs{psfractal} is the same as \Lcs{psfractal}\verb+(-1,-1)(1,1)+.
The Julia and Mandelbrot sets are a graphical representation of the following sequence
$x$ is the real and $y$ the imaginary part of the complex number $z$. $C(x,y)$ is a complex constant
and preset by $(0,0)$.
\begin{align}
z_{n+1}(x,y) &= (z_n(x,y))^2 +C(x,y)
\end{align}

\subsection{Julia sets}

A Julia set is given with

\begin{align}
z_{n+1}(x,y) &= (z_n(x,y))^2 +C(x,y)\\
z_0 	     &= (x_0;y_0)
\end{align}
$(x_0;y_0)$ is the starting value.

\begin{PSTexample}[pos=l]
\pspicture(-1,-1)(1,1)\psfractal\endpspicture
\end{PSTexample}

\begin{PSTexample}[pos=l]
\pspicture(-2,-2)(2,2)
\psfractal[xWidth=4cm,yWidth=4cm, baseColor=white, dIter=20](-2,-2)(2,2)
\endpspicture
\end{PSTexample}


\subsection{Mandelbrot sets}

A Mandelbrot set is given with

\begin{align}
z_{n+1}(x,y) &= (z_n(x,y))^2 +C(x,y)\\
z_0 	     &= (0;0)\\
C(x,y) 	     &= (x_0;y_0)
\end{align}

$(x_0;y_0)$ is the starting value.

\begin{PSTexample}[pos=l]
\pspicture(-1,-1)(1,1)
\psfractal[type=Mandel]
\endpspicture
\end{PSTexample}


\begin{PSTexample}[pos=l]
\pspicture(-2,-2)(2,2)
\psfractal[type=Mandel, xWidth=6cm, 
  yWidth=4.8cm, baseColor=white, 
  dIter=10](-2,-1.2)(1,1.2)
\endpspicture
\end{PSTexample}

\subsection{The options}


\subsection{\texttt{type}}
\Lkeyword{txpe} can be of \Lkeyval{Julia} (default) or \Lkeyval{Mandel}.


\begin{PSTexample}[pos=l]
\pspicture(-1,-1)(3,1)
\psfractal
\psfractal[type=Mandel]
\endpspicture
\end{PSTexample}

\subsection{\texttt{baseColor}}
The color for the convergent part is set by \Lkeyword{baseColor}.

\begin{PSTexample}
\begin{postscript}
\psfractal[xWidth=4cm,yWidth=4cm,dIter=30](-2,-2)(2,2)
\psfractal[xWidth=4cm,yWidth=4cm,baseColor=yellow,dIter=30](-2,-2)(2,2)
\end{postscript}
\end{PSTexample}


\subsection{\texttt{xWidth} and \texttt{yWidth}}
\Lkeyword{xWidth} and \Lkeyword{yWidth} 
 define the physical width of the fractal.

\begin{PSTexample}
\begin{postscript}
\psfractal[type=Mandel,xWidth=12.8cm,yWidth=10.8cm,dIter=5](-2.5,-1.3)(0.7,1.3)
\end{postscript}
\end{PSTexample}

\subsection{\texttt{cx} and \texttt{cy}}\xLkeyword{cx}\xLkeyword{cy}
Define the starting value for the complex constant number $C$.

\begin{PSTexample}
\begin{postscript}
\psset{xWidth=5cm,yWidth=5cm}
\psfractal[dIter=2](-2,-2)(2,2)
\psfractal[dIter=2,cx=-1.3,cy=0](-2,-2)(2,2)
\end{postscript}
\end{PSTexample}


\subsection{\texttt{dIter}}
The color is set by \Index{wavelength} to RGB conversion of the iteration number, where
\Lkeyword{dIter} is the step, predefined by 1. The wavelength is given by
the value of \Lps{iter} added by 400.

\begin{PSTexample}
\begin{postscript}
\psset{xWidth=5cm,yWidth=5cm}
\psfractal[dIter=30](-2,-2)(2,2)
\psfractal[dIter=10,cx=-1.3,cy=0](-2,-2)(2,2)
\end{postscript}
\end{PSTexample}

\subsection{\texttt{maxIter}}
\Lkeyword{maxIter} is the number of the maximum iteration until it leaves the loop.
It is predefined by 255, but internally multiplied by \Lkeyword{dIter}.

\begin{PSTexample}
\begin{postscript}
\psset{xWidth=5cm,yWidth=5cm}
\psfractal[maxIter=50,dIter=3](-2,-2)(2,2)
\psfractal[maxIter=30,cx=-1.3,cy=0](-2,-2)(2,2)
\end{postscript}
\end{PSTexample}

\subsection{\texttt{maxRadius}}
If the square of distance of $z_n$ to the origin of the complex coordinate system
is greater as \Lkeyword{maxRadius} then the algorithm  leaves the loop
and sets the point. \Lkeyword{maxRadius} should always be the square of the "`real"'
value, it is preset by 100.  

\begin{PSTexample}
\begin{postscript}
\psset{xWidth=5cm,yWidth=5cm}
\psfractal[maxRadius=30,dIter=10](-2,-2)(2,2)
\psfractal[maxRadius=30,dIter=30,cx=-1.3,cy=0](-2,-2)(2,2)
\end{postscript}
\end{PSTexample}

\subsection{\texttt{plotpoints}}\xLkeyword{plotpoints}
This option is only valid for the Sierpinski triangle and preset by 2000.

\begin{PSTexample}
\begin{pspicture}(5,5)
  \psSier(0,0)(2.5,5)(5,0)
\end{pspicture}
\begin{pspicture}(5,5)
  \psSier[plotpoints=10000](0,0)(2.5,5)(5,0)
\end{pspicture}
\end{PSTexample}



\section{Phyllotaxis}
The beautiful arrangement of leaves in some plants, called phyllotaxis, 
obeys a number of subtle mathematical relationships. For instance, the florets 
in the head of a sunflower form two oppositely directed spirals: 55 of them clockwise 
and 34 counterclockwise. Surprisingly, these numbers are consecutive Fibonacci numbers. 
The Phyllotaxis is like a Lindenmayer system.

\begin{BDef}
\Lcs{psPhyllotaxis}\OptArgs\Largr{\CAny}
\end{BDef}

The coordinates of the center are optional, if they are missing, then $(0,0)$
is assumed.


\begin{PSTexample}[pos=l]
\begin{postscript}
\psframebox{\begin{pspicture}(-3,-3)(3,3)
  \psPhyllotaxis
\end{pspicture}}
\end{postscript}
\end{PSTexample}



\begin{PSTexample}[pos=l]
\begin{postscript}
\psframebox{\begin{pspicture}(-3,-3)(4,4)
  \psPhyllotaxis(1,1)
\end{pspicture}}
\end{postscript}
\end{PSTexample}

\subsection{\texttt{angle}}\xLkeyword{angle}

\begin{PSTexample}[pos=l]
\begin{postscript}
\psframebox{\begin{pspicture}(-2.5,-2.5)(2.5,2.5)
  \psPhyllotaxis[angle=99]
\end{pspicture}}
\end{postscript}
\end{PSTexample}

\subsection{\texttt{c}}\xLkeyword{c}
This is the length of one element in the unit pt.

\begin{PSTexample}
\begin{postscript}
\psframebox{\begin{pspicture}(8,8)
  \psPhyllotaxis[c=7](4,4)
\end{pspicture}}
\end{postscript}
\end{PSTexample}

\begin{PSTexample}
\begin{postscript}
\psframebox{\begin{pspicture}(-3,-3)(3,3)
  \psPhyllotaxis[c=4,angle=111]
\end{pspicture}}
\end{postscript}
\end{PSTexample}

\subsection{\texttt{maxIter}}\xLkeyword{maxIter}
This is the number for the iterations.

\begin{PSTexample}
\begin{postscript}
\psframebox{\begin{pspicture}(-3,-3)(3,3)
  \psPhyllotaxis[c=6,angle=111,maxIter=100]
\end{pspicture}}
\end{postscript}
\end{PSTexample}



\section{Fern}

\begin{BDef}
\Lcs{psFern}\OptArgs\Largr{\CAny}
\end{BDef}

The coordinates of the starting point are optional, if they are missing, then $(0,0)$
is assumed. The default \Lkeyword{scale} is set to 10.

\begin{PSTexample}
\begin{postscript}
\psframebox{\begin{pspicture}(-1,0)(1,4)
  \psFern
\end{pspicture}}
\end{postscript}
\end{PSTexample}

\begin{PSTexample}
\begin{postscript}
\psframebox{\begin{pspicture}(-1,0)(2,5)
  \psFern(1,1)
\end{pspicture}}
\end{postscript}
\end{PSTexample}

\begin{PSTexample}
\begin{postscript}
\psframebox{\begin{pspicture}(-3,0)(3,11)
  \psFern[scale=30,maxIter=100000,linecolor=green]
\end{pspicture}}
\end{postscript}
\end{PSTexample}


\section{Koch flake}

\begin{BDef}
\Lcs{psKochflake}\OptArgs\Largr{\CAny}
\end{BDef}

The coordinates of the starting point are optional, if they are missing, then $(0,0)$
is assumed. The origin is the lower left point of the flake, marked as red 
or black point
in the following example:

\begin{PSTexample}
\begin{pspicture}[showgrid=true](-2.4,-0.4)(5,5)
  \psKochflake[scale=10]
  \psdot[linecolor=red,dotstyle=*](0,0)
\end{pspicture}
\end{PSTexample}

\begin{PSTexample}
\begin{pspicture}(-0.4,-0.4)(12,4)
  \psset{fillcolor=lime,fillstyle=solid}
  \multido{\iA=0+1,\iB=0+2}{6}{%
    \psKochflake[angle=-30,scale=3,maxIter=\iA](\iB,2.5)\psdot*(\iB,2.5)
    \psKochflake[scale=3,maxIter=\iA](\iB,0)\psdot*(\iB,0)}
\end{pspicture}
\end{PSTexample}

Optional arguments are \Lkeyword{scale}, \Lkeyword{maxIter} (iteration depth) and \Lkeyword{angle}
for the first rotation angle.


\section{Apollonius circles}

\begin{BDef}
\Lcs{psAppolonius}\OptArgs\Largr{\CAny}
\end{BDef}

The coordinates of the starting point are optional, if they are missing, then $(0,0)$
is assumed. The origin is the center of the circle:

\begin{PSTexample}
\begin{pspicture}[showgrid=true](-4,-4)(4,4)
  \psAppolonius[Radius=4cm]
\end{pspicture}
\end{PSTexample}


\begin{PSTexample}
\begin{pspicture}(-5,-5)(5,5)
  \psAppolonius[Radius=5cm,Color]
\end{pspicture}
\end{PSTexample}


\section{Trees}

\begin{BDef}
\Lcs{psPTree}\OptArgs\Largr{\CAny}
\Lcs{psFArrow}\OptArgs\Largr{\CAny}\Largb{fraction}
\end{BDef}

The coordinates of the starting point are optional, if they are missing, then $(0,0)$
is assumed. The origin is the center of the lower line, shown in the following examples
by the dot. Special parameters are the width of the lower basic line for the tree and the
height and angle for the arrow and for both the color option. The color step is given by \Lkeyword{dIter}
and the depth by \Lkeyword{maxIter}. Valid optional arguments are

\medskip
\begin{center}
\begin{tabular}{@{}>{\ttfamily}lll@{}}
\emph{Name} & \emph{Meaning} & \emph{default}\\\hline
\Lkeyword{xWidth}   & first base width & 1cm\\
\Lkeyword{minWidth} & last base width  & 1pt\\
\Lkeyword{c}	    & factor for unbalanced trees (0<c<1) & 0.5\\
\Lkeyword{Color}    & colored tree     & fasle
\end{tabular}
\end{center}

\bigskip
\begin{PSTexample}
\begin{pspicture}[showgrid=true](-3,0)(3,4)
  \psPTree   
  \psdot*(0,0)
\end{pspicture}
\end{PSTexample}

\begin{PSTexample}
\begin{pspicture}[showgrid=true](-6,0)(6,7)
  \psPTree[xWidth=1.75cm,Color=true]
  \psdot*[linecolor=white](0,0)
\end{pspicture}
\end{PSTexample}

\begin{PSTexample}
\begin{pspicture}(-7,-1)(6,8)
  \psPTree[xWidth=1.75cm,c=0.35]
\end{pspicture}
\end{PSTexample}

\begin{PSTexample}
\begin{pspicture}(-5,-1)(7,8)
  \psPTree[xWidth=1.75cm,Color=true,c=0.65]
\end{pspicture}
\end{PSTexample}

\begin{PSTexample}
\begin{pspicture}[showgrid=true](-1,0)(1,3)
  \psFArrow{0.5}
\end{pspicture}
\quad
\begin{pspicture}[showgrid=true](-2,0)(2,3)
  \psFArrow{0.6}
\end{pspicture}
\quad
\begin{pspicture*}[showgrid=true](-3,0)(3,3.5)
  \psFArrow[linewidth=3pt]{0.65}
\end{pspicture*}
\end{PSTexample}


\begin{PSTexample}
\begin{pspicture}(-1,0)(1,3)
  \psFArrow[Color]{0.5}
\end{pspicture}
\quad
\begin{pspicture}(-2,0)(2,3)
  \psFArrow[Color]{0.6}
\end{pspicture}
\quad
\begin{pspicture*}(-3,0)(3,3.5)
  \psFArrow[Color]{0.65}
\end{pspicture*}
\end{PSTexample}


\begin{PSTexample}
\begin{pspicture}(-3,-3)(2,3)
  \psFArrow[Color]{0.6}
  \psFArrow[angle=90,Color]{0.6}
\end{pspicture}
\quad
\begin{pspicture*}(-4,-3)(3,3)
  \psFArrow[Color]{0.7}
  \psFArrow[angle=90,Color]{0.7}
\end{pspicture*}
\end{PSTexample}

\section{List of all optional arguments for \texttt{pst-fractal}}

\xkvview{family=pst-fractal,columns={key,type,default}}

\bgroup
\raggedright
\nocite{*}
\bibliographystyle{plain}
\bibliography{pst-fractal-doc}
\egroup

\printindex

\end{document}