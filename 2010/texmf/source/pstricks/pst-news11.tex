%% $Id: pst-news09.tex 62 2008-11-25 19:52:07Z herbert $
\documentclass[11pt,english,BCOR10mm,DIV12,bibliography=totoc,parskip=false,smallheadings
    headexclude,footexclude,oneside]{pst-doc}
\listfiles
\let\Lfile\LFile
\usepackage[utf8]{inputenc}
\usepackage{pst-node}
\usepackage{xkvview}
\let\pstnodeFV\fileversion
\let\pstnodeFD\filedate
\renewcommand\bgImage{\psscalebox{15}{\color{blue!20}2011}}
\def\textat{\char064}
\lstset{explpreset={pos=l,width=-99pt,overhang=0pt,hsep=\columnsep,vsep=\bigskipamount,rframe={}},
    escapechar=?}
\begin{document}

%\psset{PstDebug=1}
\title{\texttt{News -- 2011}\\ \Large new macros and bugfixes for the
basic package \nxLFile{pstricks}}
\author{Herbert Voß}
\date{\today}

\maketitle

\clearpage
\tableofcontents

\clearpage
\part{\texttt{pstricks} -- package}

\section{General}
There exists a new document class \LClass{pst-doc} for writing PSTricks documentations,
like this news document. It depends on the KOMA-Script document class \LClass{scrartcl}.
\LClass{pst-doc} defines a lot of special macros to create a good index. Take one of
the already existing package documentation and look into the source file. Then it will be
easy to understand, how all these macros have to be used.

When running \Lprog{pdflatex} the title page is created with boxes and inserted 
with the macro \Lcs{AddToShipoutPicture} from the package \LPack{eso-pic}. It
inserts the background title page image \Lfile{pst-doc-pdf} to use directly
\Lprog{pdflatex}.
When running \Lprog{latex} the title page
 is created with \PST\ macros.This allows to use the Perl script \Lprog{pst2pdf} or
the package \LPack{pst-pdf} or \LPack{auto-pst-pdf} or any other program/package which
supports \PS\ code in the document.


%--------------------------------------------------------------------------------------
\section{\texttt{pstricks.sty}}
%--------------------------------------------------------------------------------------
\subsection{New optional argument}


%--------------------------------------------------------------------------------------
\section{\texttt{pstricks.tex} (\pstricksFV -- \pstricksFD)}
%--------------------------------------------------------------------------------------

\subsection{Macro \nxLcs{psLoop}}
\PST\marginpar[2.17]{2.17} already knows \Lcs{psforeach} and \Lcs{psForeach} for loops. The new
macro \Lcs{psLoop} allows a loop without defining a variable:

\begin{BDef}
\Lcs{psLoop}\Largb{n}\Largb{argument}
\end{BDef}

However, the internal \TeX\ counter \Lctr{psLoopIndex} can be used for own purposes.

\begin{LTXexample}[width=7cm]
  \psLoop{4}{PSTricks }
\end{LTXexample}

\begin{LTXexample}[width=7cm]
\tabular{|c|c|c|c|}
  \psLoop{3}{PSTricks &}\\\hline
  A & B & C & D\\\hline
\endtabular
\end{LTXexample}

\begin{LTXexample}[width=6cm]
\begin{pspicture}[showgrid](3,3)
\psLoop{4}{%
  \psdots(\the\psLoopIndex,\the\psLoopIndex)}
\end{pspicture}
\end{LTXexample}

\subsection{New fillstyle \texttt{dots}}
Instead of using the package \LPack{pst-fill} one can now use the fill style \Lkeyval{dots}.
The valid parameters are \Lkeyword{hatchsep} for the distance of two dots, \Lkeyword{hatchwidth}
for the radius of the filled dot, and \Lkeyword{hatchcolor} for the dot color.

\begin{LTXexample}[pos=t]
\begin{pspicture}(-2,-2)(10,2)
\pscircle[hatchcolor=gray,hatchsep=10pt,hatchwidth=3pt,fillstyle=dots]{2}
\pscircle[hatchcolor=gray,hatchsep=10pt,hatchwidth=3pt,fillstyle=dots*](4,0){2}
\psframe[hatchcolor=blue!30,linecolor=red,hatchsep=12pt,hatchwidth=4pt,
         fillstyle=dots*](6,-2)(10,2)
\end{pspicture}
\end{LTXexample}


%--------------------------------------------------------------------------------------
\section{The PostScript header files}
\subsection{\nxLFile{pstricks.pro}}
%--------------------------------------------------------------------------------------
There is a new subroutine \Lps{DotFill}:
\begin{lstlisting}

/DotFill {%	 on stack: dot radius
  /dotRadius ED
  abs CLW add /a ED 
  a 0 dtransform round exch round exch
  2 copy idtransform 
  exch Atan rotate 
  idtransform pop /a ED 
  .25 .25 itransform 
  pathbbox % llx lly urx ury of smallest bounding box
  /y2 ED /x2 ED /y1 ED /x1 ED 
  y2 y1 sub a div 2 add cvi /Ny ED
  x2 x1 sub a div 2 add cvi /Nx ED
  clip 
  newpath 
  /yA y1 dotRadius add CLW add def
  /xA0 x1 dotRadius add CLW add def
  Ny {
     /xA xA0 def
     Nx { 
       newpath 
       xA yA dotRadius 0 360 arc 
       SolidDot { gsave fill grestore } if 
       stroke
       xA a add /xA ED
     } repeat
     yA a add /yA ED
  } repeat
  grestore
} def
\end{lstlisting}


%--------------------------------------------------------------------------------------
%\subsection{\nxLFile{pst-algparser.pro}}
%--------------------------------------------------------------------------------------

%\section{\nxLcs{psforeach} and \nxLcs{psForeach}}
%--------------------------------------------------------------------------------------

\section{List of all optional arguments for \texttt{pstricks}}

\makeatletter
\xkvview{family=pstricks,columns={key,type,default}}
\makeatother


\clearpage
\part{Other packages}
\section{\texttt{pst-node} -- version \pstnodeFV\ | \pstnodeFD}
The command \Lcs{pnode} now knows an optional argument for an offset, which
expects two values (x,y) separated by a comma:

\begin{BDef}
\Lcs{pnode}\OptArg*{\Largs{\Larga{offset}}}\Largr{\CAny}\Largb{\Larga{node name}}
\end{BDef}


\begin{LTXexample}[width=6cm]
\begin{pspicture}[showgrid](0,-2)(6,2)
\pnode{A}\psdot(A)\uput[90](A){A} 
\pnode[0,-2]{B}\psdot(B)\uput[90](B){B} 
\pnode(2,0){C}\psdot(C)\uput[90](C){C} 
\pnode[1,-2](2,0){D}\psdot(D)\uput[90](D){D} 
\pnode[2,2](3.5,0){E}\psdot(E)\uput[90](E){E} 
\end{pspicture}
\end{LTXexample}


\clearpage
\nocite{*}
\bibliographystyle{plain}
\bibliography{PSTricks}

\printindex


\end{document}
