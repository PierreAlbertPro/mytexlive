\section {\' Evider un solide}

Certains des solides pr�d�finis ont un solide  ``{\sl creux}'' qui lui
est naturellement associ� (le c�ne, le tronc de c�ne, le cylindre,
le prisme et la calotte sph�rique). Pour ceux l�, une option
\texttt{[hallow=}\textsl{boolean}\texttt{]} est pr�vue. Positionn� �
\textsl{false}, on a le solide habituel; positionn� � \textsl{true} on
a la version creuse.

\subsubsection {Exemple 1 : cylindre et cylindre creux}

\begin{multicols}{2}
\psset{unit=0.5}
\psset{lightsrc=10 20 30,SphericalCoor,viewpoint=50 60 25,Decran=50}
\setlength{\columnseprule}{1pt}
\centerline{
\begin{pspicture}(-2,-3)(6,8)
\psframe(-2,-3)(6,8)
\psSolid[object=cylindre,h=6,r=2,
   fillcolor=yellow,incolor=red,
   hollow,
      ](0,4,0)
\end{pspicture}}
\begin{verbatim}
   \psSolid[object=cylindre,
      h=6,r=2,
      fillcolor=yellow,
      incolor=red,
      hollow,
      ](0,4,0)
\end{verbatim}
\columnbreak
\centerline{
\begin{pspicture}(-2,-3)(6,8)
\psframe(-2,-3)(6,8)
\psSolid[object=cylindre,h=6,r=2,
   fillcolor=yellow,
      ](0,4,0)
\end{pspicture}}
\begin{verbatim}
   \psSolid[object=cylindre,
      h=6,r=2,
      fillcolor=yellow,
      ](0,4,0)
\end{verbatim}
\end{multicols}

\subsubsection {Exemple 2 : prisme et prisme creux}

\begin{minipage}{6cm}
\psset{unit=0.5}
\psset{lightsrc=10 20 30,SphericalCoor,viewpoint=50 60 25,Decran=50}
\begin{pspicture}(-6,-4)(6,12)
\psframe(-9,-3)(8,11)
\defFunction{F}(t){t cos 3 mul}{t sin 3 mul}{}
\defFunction{G}(t){t cos}{t sin}{}
\psSolid[object=grille,base=-6 6 -4 4,action=draw]%
\psSolid[object=prisme,h=8,fillcolor=yellow,RotX=90,%decal=0,
      resolution=19,
      base=0 180 {F} CourbeR2+ 
           180 0 {G} CourbeR2+ 
      ](0,4,0)
\axesIIID(3,4,3)(8,6,10)
\end{pspicture}
\end{minipage}
\hfill
\begin{minipage}{8cm}
\small
\begin{verbatim}
\defFunction{F}(t){t cos 3 mul}{t sin 3 mul}{}
\defFunction{G}(t){t cos}{t sin}{}
\psSolid[object=grille,base=-6 6 -4 4,action=draw]%
\psSolid[object=prisme,h=8,fillcolor=yellow,RotX=90,
      resolution=19,
      base=0 180 {F} CourbeR2+ 
           180 0 {G} CourbeR2+ 
      ](0,4,0)
\axesIIID(3,4,3)(8,6,10)
\end{verbatim}
\end{minipage}

\begin{minipage}{6cm}
\psset{unit=0.5}
\psset{lightsrc=10 20 30,SphericalCoor,viewpoint=50 60 25,Decran=50}
\begin{pspicture}(-6,-4)(6,12)
\psframe(-9,-3)(8,11)
\defFunction{F}(t){t cos 3 mul}{t sin 3 mul}{}
\defFunction{G}(t){t cos}{t sin}{}
\psSolid[object=grille,base=-6 6 -4 4,action=draw]%
\psSolid[object=prisme,h=8,fillcolor=yellow,RotX=90,
      hollow,ngrid=4,incolor=red,
      resolution=19,
      base=0 180 {F} CourbeR2+ 
           180 0 {G} CourbeR2+ 
      ](0,4,0)
\axesIIID(3,4,3)(8,6,10)
\end{pspicture}
\end{minipage}
\hfill
\begin{minipage}{8cm}
\small
\begin{verbatim}
\defFunction{F}(t){t cos 3 mul}{t sin 3 mul}{}
\defFunction{G}(t){t cos}{t sin}{}
\psSolid[object=grille,base=-6 6 -4 4,action=draw]%
\psSolid[object=prisme,h=8,fillcolor=yellow,RotX=90,
      hollow,ngrid=4,incolor=red,
      resolution=19,
      base=0 180 {F} CourbeR2+ 
           180 0 {G} CourbeR2+ 
      ](0,4,0)
\axesIIID(3,4,3)(8,6,10)
\end{verbatim}
\end{minipage}

\subsubsection {Exemple 3 : calotte sph�rique et calotte sph�rique creuse}

\begin{multicols}{2}
\psset{unit=0.5}
\psset{lightsrc=42 24 13,SphericalCoor,viewpoint=50 30 15,Decran=50}
\setlength{\columnseprule}{1pt}
\centerline{
\begin{pspicture}(-5,-5)(5,5)
\psframe(-5,-5)(5,5)
\psSolid[object=calottesphere,r=3,ngrid=16 18,
   fillcolor=cyan!50,incolor=yellow,theta=45,phi=-30,hollow,RotY=-80]%
\axesIIID(0,3,3)(6,5,4)
\end{pspicture}}
\begin{verbatim}
\psSolid[object=calottesphere,r=3,
         ngrid=16 18,
       fillcolor=cyan!50,incolor=yellow,
       theta=45,phi=-30,hollow,RotY=-60]%
\end{verbatim}
\columnbreak
\centerline{
\begin{pspicture}(-5,-3)(5,7)
\psframe(-5,-3)(5,7)
\psSolid[object=grille,base=-5 5 -5 5,action=draw]%
\psSolid[object=calottesphere,r=3,ngrid=16 18,
   fillcolor=cyan!50,incolor=yellow,theta=45,phi=-30](0,0,1.5)%
\axesIIID(3,3,3.6)(6,6,5)
\end{pspicture}}
\begin{verbatim}
\psSolid[object=calottesphere,r=3,
      ngrid=16 18,fillcolor=cyan!50,
      incolor=yellow,theta=45,phi=-30](0,0,1.5)
\end{verbatim}
\end{multicols}
