\section {Pointage et num�rotation des sommets}

Une option permet de pointer les sommets et/ou de les num�roter soit
globalement, soit individuellement.
\begin{itemize}
  \item \texttt{[show=all]} pointe tous les sommets ;
  \item \texttt{[num=all]} num�rote tous les sommets ;
  \item \texttt{[show=0 1 2 3]} pointe les sommets \texttt{[0,1,2 et 3]} ;
  \item \texttt{[num=0 1 2 3]} num�rote les sommets \texttt{[0,1,2 et 3]}.
\end{itemize}
%
\begin{multicols}{2}
\psset{unit=0.5}
\setlength{\columnseprule}{1pt}
\centerline{
\begin{pspicture}(-5,-5)(5,5)
\psframe(-5,-5)(5,5)
\psset{Decran=20}
\psSolid[
   action=draw,
   object=cube,
   RotZ=30,
   show=all,num=all,
]%
\end{pspicture}}
\begin{verbatim}
\psSolid[action=draw,
   object=cube,RotZ=30,
   show=all,num=all,
       ]%
\end{verbatim}
\columnbreak
\centerline{
\begin{pspicture}(-5,-5)(5,5)
\psframe(-5,-5)(5,5)
\psset{Decran=20}
\psSolid[action=draw,
   object=cube,
   RotZ=30,
   show=0 1 2 3,
   num=0 1 2 3,
]%
\end{pspicture}}
\begin{verbatim}
\psSolid[object=cube,
   RotZ=30,action=draw,
   show=0 1 2 3,
   num=0 1 2 3,
   ]%
\end{verbatim}
\end{multicols}
%\newpage
