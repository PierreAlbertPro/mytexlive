\section {Utiliser une face de solide pour d�finir un plan de projection }

Plut�t que de d�finir un plan de projection par sa normale, on peut
indiquer la face d'un solide d�j� construit. Pour cela, il faut
activer la m�morisation des solides (par \verb+\psset{solidmemory}+),
donner un nom au solide vis�, puis indiquer � la macro
\verb+\psProjection+ le nom de ce solide et l'indice de la face
souhait�e. 

On utilise pour cela les param�tres \verb+solidname+ et \verb+no+ de
la macro \verb+\psProjection+.

\Cadre{$\backslash$psProjection[\ldots,solidname=A1,no=0]} utilisera le
plan de projection d�fini par la face d'indice $0$ du solide $A1$, et
l'image de l'origine sera le centre de la face d'indice $0$.

\Cadre{$\backslash$psProjection[\ldots,solidname=A1,no=0](1,2,3)}
utilisera cette fois le plan de projection parall�le � la face
d'indice $0$ du solide $A1$, et l'image de l'origine sera le point de
coordonn�es $(1,2,3)$.

\begin{multicols}{2}

\bgroup
\psset{unit=0.75}
\psset{lightsrc=10 0 10,SphericalCoor=true,viewpoint=50 -20 20,Decran=50}
\begin{pspicture}(-4,-5)(4,5)
\psframe(-4,-5)(5,5)
\psset{solidmemory}
\defFunction[algebraic]{f}(x){cos(x)}{sin(x)}{}
\psSolid[object=grille,
      RotY=90,
      ngrid=2.,
      base=-4 4 -4 4,
      fontsize=20,
      numfaces=all,
      name=A1,
      ](0,0,0)
\psProjection[object=courbeR2,
   range=-1.57 pi,
   linecolor=yellow,linewidth=0.1,
   solidname=A1,no=0,
   function=f]
\psProjection[object=courbeR2,
   range=-1.57 pi,
   linecolor=red,linewidth=0.1,
   solidname=A1,no=0,
   function=f](0,0,0)
\composeSolid
\end{pspicture}
\egroup

\columnbreak

\begin{verbatim}
\psset{solidmemory}
\defFunction[algebraic]{f}(x){cos(x)}{sin(x)}{}
\psSolid[object=grille,
      RotY=90,ngrid=2.,
      base=-4 4 -4 4,
      fontsize=20,
      numfaces=all,
      name=A1](0,0,0)
\psProjection[object=courbeR2,
   range=-1.57 pi,
   linecolor=yellow,linewidth=0.1,
   solidname=A1,no=0,
   function=f]
\psProjection[object=courbeR2,
   range=-1.57 pi,
   linecolor=red,linewidth=0.1,
   solidname=A1,no=0,
   function=f](0,0,0)
\composeSolid
\end{verbatim}

\end {multicols}

