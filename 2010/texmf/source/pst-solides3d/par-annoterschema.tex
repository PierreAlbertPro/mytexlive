\section{Annoter un sch�ma}

Il est �videmment int�ressant de pouvoir annoter un sch�ma, prenons
l'exemple de la mol�cule de m�thane dont nous voulons indiquer les
distances et les angles. 

Une premi�re �tape consiste � repr�senter la mol�cule avec uniquement
les liaisons et les grandeurs caract�ristiques, puis la mol�cule dans 
une repr�sentation plus esth�tique.
\input \datapath liaisons-methane.tex
La construction de la mol�cule est d�taill�e dans le document
\texttt{molecules.tex}. Pour annoter le sch�ma il suffit de rep�rer 
les sommets du t�tra�dre :
\begin{verbatim}
\psPoint(0,10.93,0){H1}
\psPoint(10.3,-3.64,0){H2}
\psPoint(-5.15,-3.64,8.924){H3}
\psPoint(-5.15,-3.64,-8.924){H4}
\end{verbatim}
et d'utiliser toute la puissance du package \texttt{pst-node}. D'abord pour les distances :
\begin{verbatim}
\pcline[offset=0.25]{<->}(H2)(H3)
\aput{:U}{17,8 pm}
\pcline[offset=0.15]{<->}(H2)(O)
\aput{:U}{10,93 pm}
\psPoint(-5.15,-3.64,-8.924){H4}
\end{verbatim}
Puis, pour les angles, en s'aidant du package \texttt{pst-eucl}
\begin{verbatim}
\pstMarkAngle[arrows=<->]{H1}{O}{H3}{\small 109,5$^{\mathrm{o}}$}
\end{verbatim}
