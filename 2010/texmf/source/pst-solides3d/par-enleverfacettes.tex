\section{Enlever des facettes}

L'argument \texttt{[rm=1 2 8]} permet de supprimer les facettes
visibles $1$, $2$ et $8$, afin de voir l'int�rieur des solides creux.

\begin{multicols}{2}
\setlength{\columnseprule}{1pt}
\psset{unit=0.5}
\psset{Decran=15,grid=true,viewpoint=10 10 10}
\centerline{
\begin{pspicture}(-5,-5)(5,5)
\psframe(-5,-5)(5,5)
\psSolid[rm=1 3 6,object=cylindrecreux,ngrid=2 6,
h=6,r=2,fillcolor=green!50,incolor=yellow!50,RotZ=-60](0,0,-3)
\end{pspicture}}
\columnbreak
\begin{verbatim}
\psSolid[rm=1 3 6,
         object=cylindrecreux,
         ngrid=2 6,
         h=6,r=2,fillcolor=green!50,
         incolor=yellow!50,RotZ=-60](0,0,-3)
\end{verbatim}
\end{multicols}

\begin{multicols}{2}
\setlength{\columnseprule}{1pt}
\psset{unit=0.5}
\psset{Decran=15,grid=true,viewpoint=10 10 10}
\centerline{
\begin{pspicture}(-5,-5)(5,5)
\psframe(-5,-5)(5,5)
\psSolid[object=troncconecreux,
         rm=1 12 13 14,
         r0=3,r1=1,h=6,
         fillcolor=green!50,incolor=yellow,
         mode=3](0,0,-3)
\end{pspicture}}
\columnbreak
\begin{verbatim}
\psSolid[object=troncconecreux,
         rm=1 12 13 14,
         r0=3,r1=1,h=6,
         fillcolor=green!50,incolor=yellow,
         mode=3](0,0,-3)
\end{verbatim}
\end{multicols}
