\section {Construire � partir du scratch}

L'objet \verb+new+ permet de construire son propre solide. Deux
param�tres sont utilis�s~: \verb+sommets+ qui indique la liste des
coordonn�es des diff�rents sommets, et \verb+faces+ qui donne la liste
de toutes les faces du solide, une face de solide �tant caract�ris�e
par la liste des indices des sommets la constituant, ceux-ci �tant
\textbf {rang�s dans le sens trigonom�trique lorsque l'on ragarde la
face du cot� ext�rieur}.

\subsection {Exemple 1 : une maison}

%% exemple 1

\begin{multicols}{2}

\bgroup
\psset{unit=0.5}
\psset{lightsrc=10 -20 50,SphericalCoor,viewpoint=50 -20 30,Decran=50}
\begin{pspicture*}(-7,-4)(7,9)
\psframe(-7,-4)(7,9)
\psSolid[object=new,
    sommets=
       2  4  3    -2  4  3
      -2 -4  3     2 -4  3
       2  4  0    -2  4  0
      -2 -4  0     2 -4  0
       0  4  5     0 -4  5,
    faces={
      [0 1 2 3]    [7 6 5 4]
      [0 3 7 4]    [3 9 2]
      [1 8 0]      [8 9 3 0]
      [9 8 1 2]    [6 7 3 2]
      [2 1 5 6]},
    num=all,
    show=all,
    action=draw
]%
\end{pspicture*}
\egroup

\columnbreak

\begin{verbatim}
\psSolid[object=new,
    sommets=
       2  4  3    -2  4  3
      -2 -4  3     2 -4  3
       2  4  0    -2  4  0
      -2 -4  0     2 -4  0
       0  4  5     0 -4  5,
    faces={
      [0 1 2 3]    [7 6 5 4]
      [0 3 7 4]    [3 9 2]
      [1 8 0]      [8 9 3 0]
      [9 8 1 2]    [6 7 3 2]
      [2 1 5 6]},
    num=all,show=all,action=draw]
\end{verbatim}
\end{multicols}

Il est � remarquer que le solide \verb+new+ accepte les m�mes options
que les autres solides. Par exemple, on a repr�sent� ci-dessous le
solide pr�c�dent en utilisant les param�tres \verb+hollow+,
\verb+incolor+, \verb+fillcolor+ et \verb+rm+.

%% exemple 2

\begin{multicols}{2}

\bgroup
\psset{unit=0.5}
\psset{lightsrc=10 -20 50,SphericalCoor,viewpoint=50 -20 30,Decran=50}
\begin{pspicture*}(-7,-4)(7,9)
\psframe(-7,-4)(7,9)
\psSolid[object=new,fillcolor=red!50,incolor=yellow,
    action=draw**,
    hollow,
    rm=2,
    sommets=
       2  4  3
      -2  4  3
      -2 -4  3
       2 -4  3
       2  4  0
      -2  4  0
      -2 -4  0
       2 -4  0
       0  4  5
       0 -4  5,
    faces={
      [0 1 2 3]
      [7 6 5 4]
      [0 3 7 4]
      [3 9 2]
      [1 8 0]
      [8 9 3 0]
      [9 8 1 2]
      [6 7 3 2]
      [2 1 5 6]},
    num=all,
    show=all
      ]%
\end{pspicture*}
\egroup

\columnbreak

\begin{verbatim}
\psSolid[object=new,fillcolor=red!50,
    incolor=yellow,
    action=draw**,
    hollow,
    rm=2,
    ...
\end{verbatim}
\end{multicols}

\subsection {Exemple 2 : Hyperbolo�de de rayon fixe}

%\psset{lightsrc=10 20 30,SphericalCoor=true,viewpoint=50 20 30}
%\psset{SphericalCoor=true,viewpoint=50 20 30}

\begin{multicols}{2}

Comme � chaque fois, les options de la macro \verb+\psSolid+ peuvent
embarquer du code postscript, voire du code jps.

Ci-contre un exemple en pur postscript, o� on utilise les variables
$a$, $b$ et $h$ qui sont transmises par les options de PSTricks. On
obtient ainsi un solide variable construit � partir du scratch.

Remarque~: le code utilis� provient d'un source jps pratiquement
utilis� tel que~: \url{http://melusine.eu.org/lab/bjps/solide/tour.jps}

\columnbreak

\bgroup
\psset{unit=0.75}
\psset{lightsrc=10 -20 20,SphericalCoor,viewpoint=50 -20 30,Decran=50}
\begin{pspicture*}(-3,-5)(3,6)
\psframe(-3,-5)(3,6)
\psSolid[object=new,fillcolor=red!50,incolor=yellow,
    hollow,
    a=10, %% nb d'etages
    b=20, %% diviseur de 360, nb de meridiens
    h=8,  %% hauteur
    action=draw**,
    sommets=
      /z0 h neg 2 div def
      a -1 0 {
         /k exch def
         0 1 b 1 sub {
             /i exch def
             /r z0 h a div k mul add dup mul 4 div 1 add sqrt def
             360 b idiv i mul cos r mul
             360 b idiv i mul sin r mul
             z0 h a div k mul add
         } for
      } for,
    faces={
      0 1 a 1 sub {
      /k exch def
         k b mul 1 add 1 k 1 add b mul 1 sub {
             /i exch def
             [i i 1 sub b i add 1 sub b i add]
         } for
         [k b mul k 1 add b mul 1 sub k 2 add b mul 1 sub k 1 add b mul]
      } for
    },
]
\end{pspicture*}
\egroup

\end{multicols}

Le code utilis� est le suivant~:
\begin{verbatim}
\psSolid[object=new,fillcolor=red!50,incolor=yellow,
    hollow,
    a=10, %% nb d'etages
    b=20, %% diviseur de 360, nb de meridiens
    h=8,  %% hauteur
    action=draw**,
    sommets=
      /z0 h neg 2 div def
      a -1 0 {
         /k exch def
         0 1 b 1 sub {
             /i exch def
             /r z0 h a div k mul add dup mul 4 div 1 add sqrt def
             360 b idiv i mul cos r mul
             360 b idiv i mul sin r mul
             z0 h a div k mul add
         } for
      } for,
    faces={
      0 1 a 1 sub {
      /k exch def
         k b mul 1 add 1 k 1 add b mul 1 sub {
             /i exch def
             [i i 1 sub b i add 1 sub b i add]
         } for
         [k b mul k 1 add b mul 1 sub k 2 add b mul 1 sub k 1 add b mul]
      } for
    }]
\end{verbatim}

\subsection {Exemple 3 : Import de fichiers externes}

Il est possible de faire lire les donn�es dans des fichiers
externes. Dans l'exemple ci-dessous, les fichiers
\verb+sommets_nefer.dat+ et \verb+faces_nefer.dat+ ont �t� plc�s
dans le r�pertoire de compilation.

\bgroup
\psset{unit=0.5}
\definecolor{AntiqueWhite}{rgb}{0.98,0.92,0.84}
\begin{pspicture}(-7,-7)(7,9)
\psset{lightsrc=30 -40 10}
\psset{SphericalCoor,viewpoint=50 -50 20,Decran=50}
\psframe(-7,-7)(7,9)
\psset{RotX=90,sommets= (sommets_nefer.dat) run}
\psSolid[object=new,fillcolor=AntiqueWhite,linewidth=0.5\pslinewidth,
    faces={(faces_nefer.dat) run}]%
\psSolid[object=new,fillcolor=red,linewidth=0.5\pslinewidth,
    faces={(faces_nefer_levres.dat) run}]%
\psSolid[object=new,fillcolor=black,
    faces={(faces_nefer_sourcils.dat) run}]%
\end{pspicture}
\hfill
\begin{pspicture}(-7,-7)(7,9)
\psset{lightsrc=-10 -40 -5,lightintensity=.5}
\psset{SphericalCoor,viewpoint=50 -80 10,Decran=50}
\psframe(-7,-7)(7,9)
\psset{RotX=90,RotZ=30,sommets= (sommets_nefer.dat) run}
\psSolid[object=new,fillcolor=AntiqueWhite,linewidth=0.5\pslinewidth,
    grid,
    faces={(faces_nefer.dat) run}]%
\psSolid[object=new,fillcolor=red,linewidth=0.5\pslinewidth,grid,
    faces={(faces_nefer_levres.dat) run}]%
\psSolid[object=new,fillcolor=black,
    faces={(faces_nefer_sourcils.dat) run}]%
\end{pspicture}
\egroup

\begin{verbatim}
\definecolor{AntiqueWhite}{rgb}{0.98,0.92,0.84}
\psset{lightsrc=30 -40 10}
\psset{SphericalCoor,viewpoint=50 -50 20,Decran=50}
\psframe(-7,-7)(7,9)
\psSolid[object=new,RotX=90,fillcolor=AntiqueWhite,linewidth=0.5\pslinewidth,
    sommets= (sommets_nefer.dat) run,
    faces={(faces_nefer.dat) run}]%
\end{verbatim}
