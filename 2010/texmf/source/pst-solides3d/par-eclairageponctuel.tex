\section {\' Eclairage par une source lumineuse ponctuelle}

Deux param�tres, l'un positionne la source, l'autre fixe l'intensit�
lumineuse :
\begin{itemize}
  \item \texttt{[lightsrc=20 30 50]} en coordonn�es cart�siennes.
  \item \texttt{[lightintensity=2]} (valeur par d�faut).
\end{itemize}

\begin{multicols}{2}
\setlength{\columnseprule}{1pt}
\psset{unit=0.5}
\centerline{
\begin{pspicture}(-6,-5)(4,5)
\psframe(-6,-5)(4,5)
\psset[pst-solides3d]{Decran=1e3,viewpoint=0.5e3 0 1e3,lightsrc=1e3 0 1e3,mode=5}
\psSolid[object=cube,RotZ=30](0,2,0)
\psSolid[object=cylindrecreux,RotX=30,RotZ=-30,fillcolor=cyan,incolor=red](4,-3,0)
\end{pspicture}}
\begin{verbatim}
\psset[pst-solides3d]{Decran=1e3,
       viewpoint=1e3 0 1e3,
       lightsrc=1e3 0 1e3,
       mode=5}
\psSolid[object=cube,RotZ=30](0,2,0)
\psSolid[object=cylindrecreux,
         RotX=30,RotZ=-30,
         fillcolor=cyan,incolor=red](4,-3,0)
\end{verbatim}
\columnbreak
\centerline{
\begin{pspicture}(-6,-5)(4,5)
\psframe(-6,-5)(4,5)
\psset[pst-solides3d]{Decran=30,viewpoint=30 10 20,lightsrc=30 10 20,mode=3}
\psSolid[object=cube,lightintensity=3,RotX=90,fillcolor=yellow](0,3,0)
\psSolid[object=cube,lightintensity=1,RotX=90,fillcolor=yellow](3,-3,0)
\end{pspicture}}
\begin{verbatim}
\psset[pst-solides3d]{Decran=30,
       viewpoint=30 30 30,
       lightsrc=0 1e3 1e3,
       mode=3}
\psSolid[object=cube,
         lightintensity=3,
         RotX=90](0,3,0)
\psSolid[object=cube,
         lightintensity=1,
         RotX=90](3,-3,0)
\end{verbatim}
\end{multicols}
\textdbend{} Si l'option \texttt{[lightsrc=}\textsl{value1}~\textsl{value2}~\textsl{value3}\texttt{]}
n'est pas sp�cifi�e, l'objet est uniform�ment �clair�.

\begin{center}
 \begin{pspicture}(-6,-4)(6,4)
\psframe(-6,-4)(6,4)
 \psset{Decran=15}
 \psSolid[r1=3.5,r0=1,object=tore,ngrid=18 36,fillcolor={[rgb]{.372 .62 .628}},RotY=20]%
 \axesIIID(0,4.5,0)(6,6,4)
 \end{pspicture}
\end{center}
\begin{verbatim}
 \psSolid[r1=3.5,r0=1,object=tore,ngrid=18 36,fillcolor={[rgb]{.372 .62 .628}}]%
\end{verbatim}
