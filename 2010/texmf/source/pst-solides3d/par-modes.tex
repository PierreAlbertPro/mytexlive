\section{Les modes}

Pour un certain nombre de solides, on a pr�d�fini certains
maillages. Le positionnement du param�tre \texttt{[mode=0, 1, 2, 3 ou 4]} permet de passer
du maillage pr�d�fini le plus grossier \texttt{[mode=0]} au maillage
pr�d�fini le plus fin \texttt{[mode=4]}.

Ceci permet notamment de mettre au point une image avec tous les
solides en \texttt{[mode=0]} afin d'acc�l�rer les calculs, avant de passer au
\texttt{[mode=4]} pour une image d�finitive.

%% avec mode = 0
\begin{center}
\psset{lightsrc=10 5 0,SphericalCoor,viewpoint=50 20 -40,Decran=35,unit=0.5,%
       incolor=white,fillcolor=green!50,r0=5,r1=2,h=5,object=troncconecreux,r0=5,r1=2,h=5}
\begin{pspicture}(-5,-5)(5,5)
\psframe(-5,-5)(5,5)
\psSolid[mode=0]%
\rput(0,-4.5){\psframebox[fillstyle=solid,fillcolor=black]{\small \textcolor{white}{\texttt{[mode=0]}}}}
\end{pspicture}
%
\begin{pspicture}(-5,-5)(5,5)
\psframe(-5,-5)(5,5)
\psSolid[mode=1]%
\rput(0,-4.5){\psframebox[fillstyle=solid,fillcolor=black]{\small\textcolor{white}{\texttt{[mode=1]}}}}
\end{pspicture}
%
\begin{pspicture}(-5,-5)(5,5)
\psframe(-5,-5)(5,5)
\psSolid[mode=2]%
\rput(0,-4.5){\psframebox[fillstyle=solid,fillcolor=black]{\textcolor{white}{\texttt{[mode=2]}}}}
\end{pspicture}
%
\begin{pspicture}(-5,-5)(5,5)
\psframe(-5,-5)(5,5)
\psSolid[mode=3]%
\rput(0,-4.5){\psframebox[fillstyle=solid,fillcolor=black]{\textcolor{white}{\texttt{[mode=3]}}}}
\end{pspicture}
%
\begin{pspicture}(-5,-5)(5,5)
\psframe(-5,-5)(5,5)
\psSolid[mode=4]%
\rput(0,-4.5){\psframebox[fillstyle=solid,fillcolor=black]{\textcolor{white}{\texttt{[mode=4]}}}}
\end{pspicture}
%
\begin{pspicture}(-5,-5)(5,5)
\psframe(-5,-5)(5,5)
\psSolid[mode=5]%
\rput(0,-4.5){\psframebox[fillstyle=solid,fillcolor=black]{\small\textcolor{white}{\texttt{[mode=5] => [mode=4] forc�}}}}
\end{pspicture}
\end{center}
\newpage
