\section {Positionner un solide}
\subsection{Translation}
La commande suivante
\verb+\psSolid[object=cube,+\textsl{options}\verb+](x,y,z)+ d�place le
centre du cube au point de coordonn�es $\mathtt{(x,y,z)}$.

L'exemple suivant va recopier le cube d'ar�te 1 \begin{pspicture}(-0.5,-0.5)(.5,.5)
\psset{SphericalCoor,Decran=40,viewpoint=50 35 35,a=1,lightsrc=50 30 20}
\psset{fillcolor=yellow,mode=3,
  fcol= 0 (Apricot)
        1 (Lavender)
        2 (SkyBlue)
        3 (LimeGreen)
        4 (OliveGreen)
        5 (Yellow)
        6 (GreenYellow)
        7 (Cerulean)
        8 (CarnationPink)}
\psSolid[object=cube](0.5,0.5,0.5)% c1
\end{pspicture}
 aux points de coordonn�es $\mathtt{(0.5,0.5,0.5)}$, $\mathtt{(4.5,0.5,0.5)}$ etc. afin que ces copies occupent les coins d'un cube d'ar�te 5.
\begin{center}
\begin{pspicture}(-4,-5)(5,5)
\psframe(-4,-5)(5,5)
%\psset{SphericalCoor,Decran=3,viewpoint=10 35 35,a=1,lightsrc=50 20 10}
\psset{SphericalCoor,Decran=40,viewpoint=50 35 35,a=1,lightsrc=50 30 20,
  fcol= 0 (Apricot)
        1 (Lavender)
        2 (SkyBlue)
        3 (LimeGreen)
        4 (OliveGreen)
        5 (Yellow)
        6 (GreenYellow)
        7 (Cerulean)
        8 (CarnationPink)}
\psSolid[object=grille,base=0 6 0 6,fillcolor=blue!50]%%
\psSolid[object=grille,base=0 6 0 6,RotY=90,fillcolor=blue!40](0,0,6)%
\psSolid[object=grille,base=0 6 0 6,RotX=-90,fillcolor=blue!30](0,0,6)%
\psPoint(1,0.5,0.5){c11}
\psPoint(0.5,0.5,1){c12}
\psPoint(0.5,1,0.5){c13}
\psPoint(4.5,4.5,1){c21}
\psPoint(4,4.5,0.5){c22}
\psPoint(4.5,4,0.5){c23}
\psPoint(4,0.5,0.5){c41}
\psPoint(4.5,0.5,1){c42}
\psPoint(4.5,1,0.5){c43}
\psPoint(0.5,4,0.5){c51}
\psPoint(0.5,4.5,1){c52}
\psPoint(1,4.5,0.5){c53}
\psPoint(0.5,0.5,4){c61}
\psPoint(0.5,1,4.5){c62}
\psPoint(1,0.5,4.5){c63}
\psPoint(4,0.5,4.5){c71}
\psPoint(4.5,1,4.5){c72}
\psPoint(4.5,0.5,4){c73}
\axesIIID(1,1,1)(6,6,6)
{\psset{fillcolor=yellow,mode=3}
\psSolid[object=cube](0.5,0.5,0.5)% c1
\psline[linestyle=dashed,linecolor=red](c11)(c41)
\psline[linestyle=dashed,linecolor=red](c12)(c61)
\psline[linestyle=dashed,linecolor=red](c13)(c51)
\psSolid[object=cube](4.5,0.5,0.5)
\psSolid[object=cube](0.5,4.5,0.5)
\psSolid[object=cube](0.5,0.5,4.5)
\psSolid[object=cube](4.5,4.5,4.5)
\psSolid[object=cube](4.5,0.5,4.5)
\psSolid[object=cube](4.5,4.5,0.5)
\psSolid[object=cube](0.5,4.5,4.5)}
\psSolid[object=grille,base=0 5 0 5,action=draw,linecolor=gray!50](0,0,5)%
\psSolid[object=grille,base=0 5 0 5,action=draw,linecolor=gray!50,RotY=90](5,0,5)%
\psSolid[object=grille,base=0 5 0 5,action=draw,RotX=-90,linecolor=gray!50](0,5,5)%
\end{pspicture}
\end{center}
\begin{verbatim}
\psset{fillcolor=yellow,mode=3}
\psSolid[object=cube](0.5,0.5,0.5)
\psSolid[object=cube](4.5,0.5,0.5)
\psSolid[object=cube](0.5,4.5,0.5)
\psSolid[object=cube](0.5,0.5,4.5)
\psSolid[object=cube](4.5,4.5,4.5)
\psSolid[object=cube](4.5,0.5,4.5)
\psSolid[object=cube](4.5,4.5,0.5)
\psSolid[object=cube](0.5,4.5,4.5)
\end{verbatim}
\subsection{Rotation}
La rotation s'effectue dans l'ordre autour des axes $Ox$, $Oy$ et $Oz$. Prenons l'exemple d'un parall�l�pip�de rectangle,
\begin{pspicture}(-1,-0.2)(1,.5)
\psset{SphericalCoor,Decran=40,viewpoint=50 35 35,a=2,b=3,c=1,lightsrc=50 30 30}
\psset{fillcolor=yellow,unit=0.5,
  fcol= 0 (Apricot)
        1 (Lavender)
        2 (SkyBlue)
        3 (LimeGreen)
        4 (OliveGreen)
        5 (Yellow)
        6 (Bittersweet)}
\psSolid[object=parallelepiped](0.5,0.5,0.5)%
\end{pspicture}
que l'on va faire tourner
successivement autour des axes $Ox$, $Oy$ et $Oz$.

\begin{multicols}{4}
\psset{SphericalCoor,Decran=40,viewpoint=50 35 35,a=2,b=3,c=1}
\psset{unit=0.5,
  fcol= 0 (Apricot)
        1 (Lavender)
        2 (SkyBlue)
        3 (LimeGreen)
        4 (OliveGreen)
        5 (Yellow)
        6 (Bittersweet),
  object=parallelepiped}
\setlength{\columnseprule}{1pt}
\centerline{
\begin{pspicture}(-2.5,-2.5)(2.5,2.5)
\psframe(-2.5,-2.5)(2.5,2.5)
\psSolid%%
\axesIIID(1,1.5,1)(3,3,2)
\end{pspicture}}
\columnbreak
\centerline{
\begin{pspicture}(-2.5,-2.5)(2.5,2.5)
\psframe(-2.5,-2.5)(2.5,2.5)
\psSolid[RotZ=60]%%
\psSolid[action=draw,linewidth=0.5\pslinewidth]%%
\axesIIID(1,1.5,1)(2,3,2)
\end{pspicture}}

\centerline{\texttt{[RotZ=60]}}

\columnbreak
\centerline{
\begin{pspicture}(-2.5,-2.5)(2.5,2.5))
\psframe(-2.5,-2.5)(2.5,2.5)
\psSolid[RotX=30]%%
\psSolid[action=draw,linewidth=0.5\pslinewidth]%%
\axesIIID(1,1.5,1)(2,3,2)
\end{pspicture}}

\centerline{\texttt{[RotX=30]}}

\columnbreak
\centerline{
\begin{pspicture}(-2.5,-2.5)(2.5,2.5)
\psframe(-2.5,-2.5)(2.5,2.5)
\psSolid[RotY=45]%%
\psSolid[action=draw,linewidth=0.5\pslinewidth]%%
\axesIIID(1,1.5,1)(2,3,2)
\end{pspicture}}

\centerline{\texttt{[RotY=-45]}}
\end{multicols}
