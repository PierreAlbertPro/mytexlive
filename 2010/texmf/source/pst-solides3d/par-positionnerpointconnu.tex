\section{Positionner un point connu}

\begin{verbatim}
\psPoint(x,y,z){name}
\end{verbatim}
C'est une commande analogue � \verb+\pnode(! x y){name}+. Elle place
dans le n\oe{}ud \texttt{(name)} le point de coordonn�es $(x,y,z)$, 
vu avec le point de vue choisi \texttt{viewpoint=vx vy vz}. On peut
donc ensuite s'en servir pour marquer des points, tracer des lignes,
des polygones etc.

Pla�ons les positions des centres des atomes de la mol�cule d'�thanal $\mathrm{CH_3COH}$.
\begin{center}
\begin{pspicture}(-4,-4)(4,5)
\psset{viewpoint=100 50 20,Decran=20,SphericalCoor}
\psframe(-4,-4)(4,5)
\axesIIID(3,3,3)(20,20,20)
\psPoint(-4.79,2.06,0){C1}
\psPoint(-4.79,15.76,0){Ox}
\psPoint(8.43,5.57,0){C2}
\psPoint(-14.14,3.34,0){H3}
\psPoint(14.14,-2.94,8.90){H6}
\psPoint(14.14,-2.94,-8.90){H7}
\psPoint(6.43,-16.29,0){H8}
\psline(C1)(H3)
\psline(C2)(H7)
\psline(C2)(H8)
\psline(C1)(C2)
\psline[doubleline=true](C1)(Ox)
\psline(C2)(H6)
\uput[r](H3){$\mathrm{H_1}$}
\uput[l](H6){$\mathrm{H_2}$}
\uput[l](H7){$\mathrm{H_3}$}
\uput[l](H8){$\mathrm{H_4}$}
\uput{0.25}[u](C1){$\mathrm{C_1}$}
\uput{0.25}[d](C2){$\mathrm{C_2}$}
\uput{0.25}[r](Ox){$\red\mathrm{O}$}
\psdots[dotstyle=o,dotsize=0.3](H3)(H6)(H7)(H8)
\psdots[dotsize=0.4](C1)(C2)
\psdot[linecolor=red,dotsize=0.4](Ox)
\end{pspicture}
\end{center}
\begin{verbatim}
\psPoint(-4.79,2.06,0){C1}
\psPoint(-4.79,15.76,0){Ox}
\psPoint(8.43,5.57,0){C2}
\psPoint(-14.14,3.34,0){H3}
\psPoint(14.14,-2.94,8.90){H6}
\psPoint(14.14,-2.94,-8.90){H7}
\psPoint(6.43,-16.29,0){H8}
\psline(C1)(H3)
\psline(C2)(H7)
\psline(C2)(H8)
\psline(C1)(C2)
\psline[doubleline=true](C1)(Ox)
\psline(C2)(H6)
\uput[r](H3){$\mathrm{H_1}$}
\uput[l](H6){$\mathrm{H_2}$}
\uput[l](H7){$\mathrm{H_3}$}
\uput[l](H8){$\mathrm{H_4}$}
\uput{0.25}[u](C1){$\mathrm{C_1}$}
\uput{0.25}[d](C2){$\mathrm{C_2}$}
\uput{0.25}[r](Ox){$\red\mathrm{O}$}
\psdots[dotstyle=o,dotsize=0.3](H3)(H6)(H7)(H8)
\psdots[dotsize=0.4](C1)(C2)
\psdot[linecolor=red,dotsize=0.4](Ox)
\end{verbatim}
