\section {Fusionner des solides}

Il est possible d'assembler plusieurs solides en une seule structure~:
c'est l'op�ration de \textsl {fusion\/} des solides. Cette technique
permet d'appliquer l'algorithme du peintre � toute une sc�ne.

Pour ce faire, il faut activer la possibilit� de mettre en m�moire par
\verb+\psset{solidmemory}+, puis construire ses diff�rents solides
avec \verb+\psSolid+, sans omettre de leur donner un nom chacun.

On utilise ensuite l'objet \verb+fusion+ de \verb+\psSolid+, en
indiquent dans le param�tre \verb+base+ la liste des noms des solides
� fusionner.

Pour visualiser la sc�ne, ne pas oublier de conclure par
\verb+\composeSolid+. 

\begin{multicols}{2}
\bgroup
\psset{lightsrc=50 -50 50,SphericalCoor=true,viewpoint=40 16 32,Decran=40}
\psset{unit=.6}
\begin{pspicture}(-6,-5)(6,7)
\psframe*[linecolor=gray!40](-6,-5)(6,7)
\psset{solidmemory}
\psSolid[object=cylindre,h=6,fillcolor=blue,
      r=1.5,
      ngrid=4 16,
      action=none,
      name=A1,
      ](0,0,-4)
\psSolid[object=anneau,h=6,fillcolor=red!50,
      R=4,r=3,h=1,
      action=none,
      name=B1,
      ](0,0,-1)
\psSolid[object=fusion,
      action=draw**,
      base=A1 B1,
      ](0,0,0)
\composeSolid
\end{pspicture}
\egroup

\columnbreak

\begin{verbatim}
\psset{solidmemory}
\psSolid[object=cylindre,h=6,fillcolor=blue,
      r=1.5,
      ngrid=4 16,
      action=none,
      name=A1,
      ](0,0,-4)
\psSolid[object=anneau,h=6,fillcolor=red!50,
      R=4,r=3,h=1,
      action=none,
      name=B1,
      ](0,0,-1)
\psSolid[object=fusion,
      action=draw**,
      base=A1 B1,
      ](0,0,0)
\composeSolid
\end{verbatim}
\end{multicols}


\begin{multicols}{2}
\bgroup
\psset{unit=0.5}
\begin{pspicture}(-6,-5)(6,5)
\psset{solidmemory}
\psset{lightsrc=50 -50 50,viewpoint=100 -30 40,
   Decran=100,linewidth=0.5\pslinewidth,
   ngrid=18 18,fillcolor=white,
   h=12,r=2,RotX=90}
\psframe*[linecolor=black](-6,-5)(6,5)
\psSolid[object=cylindrecreux,
   action=none,
   name=cylindre1](0, 6, 0)
\psSolid[object=cylindrecreux,
   RotZ=90,
   action=none,
   name=cylindre2](-6, 0, 0)
\psSolid[object=fusion,
   base=cylindre1 cylindre2,RotX=0]
\composeSolid
\end{pspicture}
\egroup

\columnbreak

\begin{verbatim}
\psset{solidmemory}
\psset{lightsrc=50 -50 50,viewpoint=100 -30 40,
   Decran=100,linewidth=0.5\pslinewidth,
   ngrid=18 18,fillcolor=white,
   h=12,r=2,RotX=90}
\psframe*[linecolor=black](-6,-5)(6,5)
\psSolid[object=cylindrecreux,
   action=none,
   name=cylindre1](0, 6, 0)
\psSolid[object=cylindrecreux,
   RotZ=90,
   action=none,
   name=cylindre2](-6, 0, 0)
\psSolid[object=fusion,
   base=cylindre1 cylindre2,RotX=0]
\composeSolid
\end{verbatim}

\end{multicols}

