\section {Nommer un solide}

Pour certaines utilisations, on a besoin de stocker un solide en
m�moire afin de pouvoir y faire r�f�rence par la suite. Pour ce faire
on dispose du bool�en \verb+solidmemory+, qui permet la transmission
d'une variable tout au long de la sc�ne.

En revanche, l'activation de ce bool�en d�sactive le dessin imm�diat
des macros \verb+\psSolid+, \verb+\psSurface+ et
\verb+\psProjection+. Pour obtenir ce dessin, on utilise la macro
\verb+\composeSolid+ � la fin de la sc�ne.

Lorsque l'activation \verb+\psset{solidmemory}+ est faite, on peut
alors utiliser l'option \verb+name+ de la macro \verb+\psSolid+.

Dans l'exemple ci-dessous, on construit un solide color�, que l'on
sauvagarde sous le nom $A1$. On le dessine ensuite, apr�s coup, avec
le code jps \verb+A1 draw**+. 

\`A noter que l'instruction \verb+linecolor=blue+ utilis�e lors de la
construction de notre cube n'a pas d'impact sur le dessin~: seule la
structure du solide a �t� sauvegard� (sommets, faces, couleurs des
faces), pas l'�paisseur de la ligne de trac� ou sa couleur ou la
position de la source lumineuse. C'est au moment du dessin du solide
consid�r� qu'il faut r�gler ces param�tres.

\begin{multicols}{2}
\bgroup
\psset{unit=0.75}
\psset{lightsrc=10 0 10,SphericalCoor=true,viewpoint=50 -20 10,Decran=50}
\begin{pspicture*}(-4,-4)(5,4)
\psframe(-4,-4)(5,4)
\psset{solidmemory}
\psSolid[object=cube,
      linecolor=blue,
      a=4,fillcolor=red!50,
      ngrid=3,
      action=none,
      name=A1,
      ](0,0,0)
\codejps{
   A1 draw**
}
\composeSolid
\end{pspicture*}
\egroup

\columnbreak

\begin{verbatim}
\psset{solidmemory}
\psSolid[object=cube,
      linecolor=blue,
      a=4,fillcolor=red!50,
      ngrid=3,
      action=none,
      name=A1,
      ](0,0,0)
\codejps{
   A1 draw**
}
\composeSolid
\end{verbatim}
\end{multicols}


\llap {\dbend } Il nous reste encore du travail � faire sur ces
macros, et elles ne permettent pour le moment pas de choisir des noms
de variables quelconques, car ils risquent d'entrer en conflit avec
des noms d�j� existant. Merci d'utiliser des noms analogues � ceux
utilis�s dans la documentation.

