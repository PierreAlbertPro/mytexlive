\section {Projection d'une courbe de fonction}

Les exemple pr�c�dents l'ont d�j� montr�, il s'agit de repr�senter une
fonction dans le plan d�fini par les m�thodes pr�cis�es dans les
paragraphes pr�c�dents. 

La courbe peut-�tre d�finie de trois fa�ons : soit par une �quation
simple, soit par deux �quations param�triques, soit enfin par un
chemin (liste de coordonn�es de points). 

\subsection {Courbe de fonction num�rique}

La courbe sera d�finie par une �quation, soit en notation alg�brique
avec l'option  \Cadre{[algebraic]}, soit en notation polonaise
invers�e (langage postscript), avec une variable quelconque
$(x,u,t\ldots)$, par une expression de la forme suivant le cas : 

%% \begin{boxedverbatim}
\begin{verbatim} 
\defFunction[algebraic]{nom_fonction}(x){x*sin(x)}{}{}
\end{verbatim}  
%% \end{boxedverbatim}

%% \begin{boxedverbatim}
\begin{verbatim} 
\defFunction{nom_fonction}(x){x dup sin mul}{}{}
\end{verbatim}  
%% \end{boxedverbatim}

Cette expression dans doit �tre incluse dans l'environnement \Cadre{pspicture}.

Les limites de la variable sont d�finies dans l'option \Cadre{range=-4 4}

Le trac� de la fonction ainsi d�finie fait appel � l'objet
\Cadre{courbe} et � l'option \Cadre{function}. 

%% \begin{boxedverbatim}
\begin{verbatim} 
\psProjection[object=courbe,
            range=-4 4,
            linecolor={[cmyk]{1,0,1,0.5}},
            normal=-1 1 0 1 1 2 ,
            function=nom_fonction]
\end{verbatim}  
%% \end{boxedverbatim}

Exemple :

\begin{minipage}{0.4\linewidth}
\psset{unit=0.5}
\begin{pspicture}(-6,-6)(6,7)%
\psframe*[linecolor=blue!50](-6,-6)(6,6)
\psset{lightsrc=50 20 20,viewpoint=50 30 20,Decran=70}
{\psset{linewidth=0.5\pslinewidth,linecolor=gray}
\psSolid[object=grille,base=-4 0 -4 4]
\psSolid[object=grille,base=-4 4 -4 4,RotX=90,RotZ=90]}
\defFunction[algebraic]{1_sin}(x){3*sin(1/x)}{}{}
\psProjection[object=chemin,
      linewidth=.1,
      linecolor=green,
      normal=0 1 0 1 0 0,
      path=
         newpath
            0 0 smoveto
            1 0 slineto]
\psProjection[object=chemin,
      linewidth=.1,
      linecolor=blue,
      normal=0 1 0 1 0 0,
      path=
         newpath
            0 0 smoveto
            0 1 slineto]
\psPoint(0,0,0){O}\psPoint(0.4,0.4,0.8){K}
\psline[linewidth=.1,linecolor=red](O)(K)
\psProjection[object=courbe,
   range=-4 4,resolution=720,
   linecolor=red,
   normal=0 1 0 1 0 0,
   function=1_sin]
\psProjection[object=chemin,
      linewidth=.02,
      linecolor=red,
      normal=0 1 0 1 0 0,
      path=newpath
          -4 1 4
          {-4 exch smoveto
           8 0 srlineto} for
           -4 1 4
          {-4 smoveto
           0 8 srlineto} for
            ]
\axesIIID(0,0,0)(4,4,4)
\end{pspicture}
\end{minipage}
\hfill
\begin{minipage}{0.6\linewidth}
%% \begin{boxedverbatim}
\begin{verbatim} 
\defFunction[algebraic]{1_sin}(x){3*sin(1/x)}{}{}
\psProjection[object=courbe,
            range=-4 4,resolution=720,
            linecolor=red,
            normal=0 1 0 1 0 0,
            function=1_sin]
\end{verbatim}  
%% \end{boxedverbatim}
\end{minipage}

\subsection {Courbes param�tr�es}

La technique est analogue, � la diff�rence pr�s que la fonction
�voqu�e est � valeurs dans $R^2$, et que l'objet pass� en param�tre �
\verb+\psProjection+ est \Cadre{courbeR2}.

Pour dessiner un cercle de rayon $3$, on  �crira :

%% \begin{boxedverbatim}
\begin{verbatim} 
\defFunction[algebraic]{cercle}(t){3*cos(t)}{3*sin(t)}{}
\end{verbatim}  
%% \end{boxedverbatim}

Autre exemple : les courbes de Lissajous.

\begin{minipage}{0.4\linewidth}
\psset{unit=0.5}
\begin{pspicture}(-6,-5)(6,8)%
\psframe*[linecolor=blue!50](-6,-5)(6,7)
\psset{lightsrc=50 20 20,viewpoint=50 30 20,Decran=60}
\defFunction[algebraic]{lissajous}(t){3*sin(0.57735*t)}{4*sin(0.707*t)}{}
\psProjection[object=chemin,fillstyle=solid,fillcolor=white,
            linewidth=.005,linecolor=red,
            normal=1 1 2,
            path=newpath
                -4 -4 smoveto
                -4 4 slineto
                4 4 slineto
                4 -4 slineto
                closepath
            ](1,1,2)
\psProjection[object=chemin,
      linewidth=.1,
      linecolor=green,
      normal=1 1 2,
      path=
         newpath
            0 0 smoveto
            1 0 slineto](1,1,2)
\psProjection[object=chemin,
      linewidth=.1,
      linecolor=blue,
      normal=1 1 2,
      path=
         newpath
            0 0 smoveto
            0 1 slineto](1,1,2)
\psPoint(0,0,0){O}
\psPoint(1,1,2){O1}\psPoint(1.4,1.4,2.8){K}
\psline[linewidth=.1,linecolor=red](O1)(K)
\psline[linestyle=dashed](O)(O1)
\psProjection[object=courbeR2,
   range=-25.12 25.12,resolution=720,
   normal=1 1 2,linecolor=red,
   function=lissajous](1,1,2)
\psProjection[object=chemin,
      linewidth=.02,
      normal=1 1 2,
      path=newpath
          -4 1 4
          {-4 exch smoveto
           8 0 srlineto} for
           -4 1 4
          {-4 smoveto
           0 8 srlineto} for
            ](1,1,2)
\axesIIID(4,4,2)(5,5,6)
\end{pspicture}
\end{minipage}
\hfill
\begin{minipage}{0.6\linewidth}
%% \begin{boxedverbatim}
\begin{verbatim} 
\defFunction[algebraic]%
{lissajous}(t){3*sin(0.57735*t)}{4*sin(0.707*t)}{}
\psProjection[object=courbeR2,resolution=720,
            range=-25.12 25.12,
            normal=1 1 2,
            linecolor=red,
            function=lissajous](1,1,2)
\end{verbatim}  
%% \end{boxedverbatim}
\end{minipage}
