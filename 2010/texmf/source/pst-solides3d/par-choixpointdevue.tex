\section {Choix du point de vue}
\begin{center}
\psset{lightsrc=10 20 30,SphericalCoor,viewpoint=50 30 20}
\begin{pspicture}(-5,-4)(10,7)
\definecolor{bleuciel}{rgb}{0.78,0.84,0.99}
\psSolid[object=cube,fillcolor=bleuciel,a=2,action=draw*]%%
%\psSolid[object=cubemaillage,fillcolor=bleuciel,a=2]%%
\psSolid[object=grille,base=0 8 0 10,action=draw]%%
\psSolid[object=grille,base=0 7 0 10,action=draw,RotY=90](0,0,7)%
\psSolid[object=grille,base=0 8 0 7,action=draw,RotX=-90](0,0,7)%
\psSolid[object=cube,fillcolor=bleuciel,a=1,action=draw*](0.5,0.5,0.5)%
\psSolid[object=grille,base=-1 1 -1 1,action=draw,linecolor=blue](0,0,1)%
\psSolid[object=grille,base=-1 1 -1 1,action=draw,RotY=90,linecolor=blue](1,0,0)%
\psSolid[object=grille,base=-1 1 -1 1,action=draw,RotX=-90,linecolor=blue](0,1,0)%
\axesIIID(1,1,1)(8,10,7)
\pstVerb{/dV 12 def % distance V
         /dE 6 def % distance �cran
         /Kc dV dE sub dV div def
         /Theta 60 def
         /Phi 30 def
         /xV dV Phi cos mul Theta  cos mul def
         /yV dV Phi cos mul Theta  sin mul def
         /zV dV Phi sin mul def
         /xE Kc xV mul def
         /yE Kc yV mul def
         /zE Kc zV mul def
         }%
\psPoint(xV,yV,zV){V}
\psPoint(xE,yE,zE){E}
\psPoint(xV,yV,0){Vp}
% 5 distance �cran
%\psPoint(dE Theta  cos mul Phi cos div dE Theta  sin mul Phi cos div 0){Vq}
\psPoint(xV,0,0){Vx}
\psPoint(0,yV,0){Vy}
\psPoint(0,0,zV){Vz}
\psdot(V)
{\psset{linestyle=dashed,linecolor=red}
\psline(V)(Vp)\psline(Vx)(Vp)\psline(Vy)(Vp)\psline(V)(Vz)\psline(V)(O)\psline(Vp)(O)}
\psSolid[object=grille,base=-5 5 -3 3,action=draw,RotX=-60,linecolor=red](xE,yE,zE)%
\psTransformPoint[RotX=-60](-5 -3 0)(xE,yE,zE){A}
\psTransformPoint[RotX=-60](-5 3 0)(xE,yE,zE){B}
\psTransformPoint[RotX=-60](5 3 0)(xE,yE,zE){C}
\psTransformPoint[RotX=-60](5 -3 0)(xE,yE,zE){D}
\pspolygon[fillstyle=vlines,hatchcolor=yellow!50,hatchwidth=0.02,hatchsep=0.04](A)(B)(C)(D)
\PointEcran(1,1,1){S1}
\psPoint(1,1,1){s1}
\psline(s1)(S1)(V)
%
\PointEcran(1,1,-1){S2}
\psPoint(1,1,-1){s2}
\psline(s2)(S2)(V)
%
\PointEcran(-1,1,-1){S3}
\psPoint(-1,1,-1){s3}
\psline(s3)(S3)(V)
%
\PointEcran(-1,1,1){S4}
\psPoint(-1,1,1){s4}
\psline(s4)(S4)(V)
%
\PointEcran(1,-1,-1){S5}
\psPoint(1,-1,-1){s5}
\psline(s5)(S5)(V)
%
\PointEcran(1,-1,1){S6}
\psPoint(1,-1,1){s6}
\psline(s6)(S6)(V)
%
\PointEcran(-1,-1,1){S7}
\psPoint(-1,-1,1){s7}
\psline(s7)(S7)(V)
%
\psset{linecolor=red,fillstyle=vlines,hatchsep=0.04,hatchwidth=0.02}
\pspolygon[hatchcolor=red!60](S1)(S2)(S3)(S4)
\pspolygon[,hatchcolor=red!60](S1)(S2)(S5)(S6)
\pspolygon[hatchcolor=red!10](S1)(S4)(S7)(S6)
\psdots(s1)(s2)(s3)(s4)(s5)(s6)(s7)(S1)(S2)(S3)(S4)(S5)(S6)(S7)
%
\uput[45](V){$V$}
\end{pspicture}
\end{center}
Les coordonn�es de l'objet, ici le cube bleut�, sont donn�es dans le
rep�re $Oxyz$.  Les coordonn�es du point de vue ($V$), sont donn�es
dans ce m�me rep�re, soit en coordonn�es sph�riques, avec l'option
\texttt{[SphericalCoor]}, soit en coordonn�es cart�siennes qui
est l'option par d�faut.

Exemple : \texttt{[SphericalCoor,viewpoint=50 30 20]}

L'�cran est plac� perpendiculairement � la direction $OV$, � une
distance de $V$ : \texttt{[Decran=50]} (valeur par d�faut), cette
valeur peut �tre positive ou n�gative.
