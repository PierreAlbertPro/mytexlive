%% $Id: pst-knot-doc.tex 137 2009-10-08 18:15:14Z herbert $
\documentclass[11pt,english,BCOR10mm,DIV12,bibliography=totoc,parskip=false,
   smallheadings, headexclude,footexclude,oneside]{pst-doc}
\usepackage[utf8]{inputenc}
\usepackage{pst-knot}
\let\pstKnotFV\fileversion
\renewcommand\bgImage{%
\begin{pspicture}(-2,-2)(2,2) 
  \psKnot[knotscale=2,linewidth=5pt,linecolor=blue](0,0){5-1}
\end{pspicture}
}
\lstset{pos=t,wide=false,language=PSTricks,
    morekeywords={psKnot},basicstyle=\footnotesize\ttfamily}
%
\begin{document}

\title{\texttt{pst-knot}}
\subtitle{Plotting special knots; v.\pstKnotFV}
\author{Herbert Vo\ss}
\docauthor{}
\date{\today}
\maketitle

\tableofcontents

\section{introduction}
This is the very first try of drawing knots. The package uses the \PS\ 
subroutines from the file \LFile{psMath.pro} from Matthias Buch-Kromann.)
Currentlx there is only one macro with two mandatory arguments, the origin
of the image and the knot type. The following list shows all available knot types.

\begin{BDef}
\Lcs{psKnot}\OptArgs\Largr{\CAny}\Largb{knot type}
\end{BDef}


\begin{LTXexample}
\begin{pspicture}[showgrid=true](-2,-2)(2,2) 
  \psKnot[linewidth=3pt,linecolor=red](0,0){0-1}
\end{pspicture}
\end{LTXexample}

\begin{LTXexample}[pos=t]
\begin{pspicture}[showgrid=true](-2,-2)(8,2) 
  \psKnot[linewidth=3pt,linecolor=red](0,0){3-1}
  \psKnot[linewidth=3pt,linecolor=blue](4,0){4-1}
\end{pspicture}
\end{LTXexample}

\begin{LTXexample}[pos=t]
\begin{pspicture}(-2,-2)(8,2) 
  \psKnot[linewidth=3pt,linecolor=blue](0,0){5-1}
  \psKnot[linewidth=3pt,linecolor=blue](4,0){5-2}
\end{pspicture}
\end{LTXexample}

\begin{LTXexample}[pos=t]
\begin{pspicture}(-2,-2)(10,2) 
  \psKnot[linewidth=3pt,linecolor=blue](0,0){6-1}
  \psKnot[linewidth=3pt,linecolor=blue](4,0){6-2}
  \psKnot[linewidth=3pt,linecolor=blue](8,0){6-3}
\end{pspicture}
\end{LTXexample}

\begin{LTXexample}
\begin{pspicture}(-2,-2)(10,2) 
  \psKnot[linewidth=3pt,linecolor=red](0,0){7-1}
  \psKnot[linewidth=3pt,linecolor=blue](4,0){7-2}
  \psKnot[linewidth=3pt,linecolor=green](8,0){7-3}
\end{pspicture}
\end{LTXexample}

\begin{LTXexample}
\begin{pspicture}(-2,-2)(10,2) 
  \psKnot[linewidth=3pt,linecolor=red](0,0){7-4}
  \psKnot[linewidth=3pt,linecolor=green](4,0){7-5}
  \psKnot[linewidth=3pt,linecolor=blue](8,0){7-6}
\end{pspicture}
\end{LTXexample}

\begin{LTXexample}
\begin{pspicture}[showgrid=true](-2,-2)(2,2) 
  \psKnot[linewidth=3pt,linecolor=blue](0,0){7-7}
\end{pspicture}
\end{LTXexample}

\section{Special settings}
There exists three special optional arguments for the macro \Lcs{psKnot}.

\subsection{Scaling}
The image can be scaled with \Lkeyword{scale}, which can take one or two
values for x and y scaling. For only one value it is scaled for x and y
with the same value. The default is 1 1.

\begin{LTXexample}[pos=t]
\begin{pspicture}(-4,-4)(4,4) 
  \psKnot[linewidth=5pt,linecolor=blue,knotscale=2](0,0){6-1}
\end{pspicture}
\end{LTXexample}

\subsection{Border color}
The background color of the border can be controlled by \Lkeyword{knotbgcolor}
with only a numerical value of $[0..1]$ for a grayscale color. It makes
only sense for a colored background to get the same color for the crossing.

\begin{LTXexample}[pos=t]
\begin{pspicture}(-2,-2)(6,2) 
  \psframe[fillcolor=black!20,fillstyle=solid](-2,-2)(6,2)
  \psKnot[linewidth=5pt,linecolor=red!50](0,0){7-4}
  \psKnot[linewidth=5pt,linecolor=red!50,
    knotbgcolor=0.8](4,0){7-4}
\end{pspicture}
\end{LTXexample}

Pay attention that black!20 is the same as 0,8 of gray.

\subsection{Border width}
The width of the border is controlled by the keyword \Lkeyword{knotborder} and
it is preset to 2 as a factor to the current linewidth. 
It must at least be the currentlinewidth (1.0).

\begin{LTXexample}[pos=t]
\begin{pspicture}(-2,-2)(6,2) 
  \psKnot[linewidth=3pt,linecolor=cyan!60](0,0){6-3}
  \psKnot[linewidth=3pt,linecolor=red!50,
    knotborder=5](4,0){6-3}
\end{pspicture}
\end{LTXexample}


\clearpage
\section{List of all optional arguments for \nxLPack{pst-knot}}

\xkvview{family=pst-knot,columns={key,type,default}}


\bgroup
\raggedright
\nocite{*}
\bibliographystyle{plain}
\bibliography{\jobname}
\egroup

\printindex

\end{document}


