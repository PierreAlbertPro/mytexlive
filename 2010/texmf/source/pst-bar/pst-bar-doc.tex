%% $Id: pstricks-add-doc.tex 62 2008-11-25 19:52:07Z herbert $
\documentclass[11pt,english,BCOR10mm,DIV12,bibliography=totoc,parskip=false,smallheadings
    headexclude,footexclude,oneside]{pst-doc}
\listfiles
\usepackage{pst-bar}%
\let\pstBarVersion\fileversion
\usepackage{booktabs}%
\usepackage{filecontents}%
\usepackage{verbatim}%
\makeatletter
\def\kernhack#1#2{%
    \begingroup
	\setbox\z@\hbox{#1#2}%
	\dimen@\wd\z@
	\setbox\z@\hbox{#1{}#2}\advance\dimen@-\wd\z@
	\ifdim\dimen@=\z@\else\kern\dimen@\fi
    \endgroup
}
\def\switch#1#2{%
    \begingroup
	\def\@switch##1#1##2\@nil#1##3\@stop{%
	    \def\reserved@a{##3}%
	    \ifx\reserved@a\@empty#2\else
		\setbox\z@\hbox{##1#1}%
		\setbox\tw@\hbox{\kernhack{#1}{##2}\switch{#1}{##2}}%
		\dimen@\wd\tw@\advance\dimen@\wd\z@%
		\wd\z@\dimen@%
		\rlap{\hb@xt@\dimen@{\hfil\unhbox\tw@}}%
		\box\z@
	    \fi
	}%
	\@switch#2\@nil#1\@nil#1\@stop
    \endgroup
}
% Given an e-mail address of the form abc@def.ghi.jkl, reverses it to be jkl
% ghi. def. abc@
\def\switchemail#1{\expandafter\@switchemail#1\@stop}
% Helper function for \switchemail. Note the \copy, because \kernhack will
% destroy the box otherwise. Need to do some funky stuff because of catcodes.
\begingroup
\makeatother
\def\tmp#1#2#3{%
    \gdef#1##1@##2#2{%
	\begingroup
	\setbox#3\hbox{\switch{.}{##2}}%
	\switch{@}{##1@\copy#3}%
	\endgroup
    }%
}
\makeatletter
\tmp\@switchemail\@stop\@tempboxa
\endgroup
\makeatother

\usepackage[T1]{fontenc}%

\begin{filecontents*}{example1.csv}
    Set 1, Set 2, Set 3
    1, 2, 3
    1, 2, 3
\end{filecontents*}

\begin{filecontents*}{example2.csv}
    Set 1, Set 2, Set 3
    1, 2, 3
    0.5, 1.5, 1
\end{filecontents*}

\begin{filecontents*}{example3.csv}
    Set 1, Set 2, Set 3
    1, 2, 3
    0.5, 1.5, 1
    0.75, 1.6, 0.9
    0.9, 2.2, 2.5
\end{filecontents*}

\begin{filecontents*}{example4.csv}
    Set 1, Set 2, Set 3
    1, 0, 3
\end{filecontents*}

\begin{filecontents*}{example5.csv}
    0, 2, 0
\end{filecontents*}

\begin{filecontents*}{example6.csv}
    Set 1, Set 2, Set 3
    1.3, 2.5, 0.9
\end{filecontents*}

\begin{filecontents*}{example7.csv}
    0.7, 1.9, 0.4
    1.2, 2.7, 1.4
\end{filecontents*}

\setlength{\oddsidemargin}{0.25in}%
\setlength{\evensidemargin}{0.25in}%
\setlength{\textwidth}{6in}%
\setlength{\topmargin}{0in}%
\setlength{\headheight}{0in}%
\setlength{\headsep}{0in}%
\setlength{\textheight}{9in}%


% TODO:
% 1. Add notice that pst-bar may only work with the latest pstricks.
% 2. Make clear that \psxunit must be set using \psset{xunit}.

\begin{document}
\title{\texttt{pst-bar}\\ Bar charts for \texttt{pstricks}\\
    {\small v.\pstBarVersion}}
\author{Alan Ristow\\
    \texttt{\switchemail{ristow@ee.gatech.edu}}}
\date{October 20, 2008}
\maketitle

  \begin{abstract}
    With \LPack{pst-bar}, one may use \LPack{pstricks} to produce bar charts
    directly from a data file. This package is still in the beta stage --- the
    usual caveats pertaining to beta software apply. Additional features and
    improved (in both content and layout) documentation will be provided as the
    code stabilizes.
  \end{abstract}

\section{Introduction}
    \verb|pst-bar| uses the power and flexibility of
    \verb|pstricks|~\cite{vanzandt:93} to draw bar charts from data stored in a
    comma-delimited file. Several solutions exist for drawing bar charts for
    \TeX\ and \LaTeX\ documents. The obvious solution is to use an external
    program, such as gnuplot or Matlab, that can save bar charts in a format
    that may be read directly into \TeX\ or \LaTeX. This approach has the
    disadvantage that extra effort is required to produce charts with a uniform
    appearance that blend well with the other graphics in a document. In many
    cases the text in the chart can be typeset by \LaTeX\ with the aid of the
    \LPack{psfrag} or \verb|psfragx| packages (for diagrams created with Matlab,
    this process can be accelerated by LaPrint, a Matlab program written
    specifically for this purpose~\cite{linneman:04}). However, some programs
    write the text elements in Encapsulated Postscript files in such a way that
    the \verb|psfrag|-based options cannot work.

    Bar charts may also be drawn using the built-in drawing tools provided by
    \TeX\ and \LaTeX. This is the approach used by the \verb|histogr|
    package~\cite{schopf:97}, which is intended to draw histograms, but can be
    used to draw very simple bar charts. It also appears to be the approach
    used by the \verb|bar| package~\cite{bleser:94} for \LaTeX\ 2.09, which
    defines an environment for drawing bar charts. However, the built-in
    drawing tools are quite limited and work on the \verb|bar| package appears
    to have ceased after 1994.

    More sophisticated drawing tools have been developed for use with \TeX\ and
    \LaTeX, notably \verb|eepic|, \verb|texdraw| and \verb|latexdraw|,
    \verb|LaTeXPiX|, MetaPost, \verb|mfpic|, and \verb|pstricks|. Any of these
    tools might be used to create bar charts; however, none of them has
    built-in support for doing so, and bar charts must be drawn using the
    ``brute force'' method --- manually drawing boxes to represent the bars in
    the chart. Depending on the tool, the axes of the chart might also need to
    be drawn manually. The \verb|pst-bar| package is intended to automate the
    process of drawing bar charts using \verb|pstricks|.

    There are currently two existing solutions for producing bar charts using
    \verb|pstricks|. One is the aforementioned brute force method in which a
    series of \Lcs{psframe}s are drawn with suitable heights, widths, and
    colors to produce the desired chart~\cite{voss:barcharts}. While this
    method works admirably and provides maximum flexibility, it is tedious and
    labor-intensive. The other option is the \verb|bardiag| package, which
    essentially automates this approach~\cite{stepanyan:03}. While effective,
    \verb|bardiag| requires a number of external packages to function, uses
    \LaTeX\ to perform its mathematical operations, and is not compatible with
    plain \TeX.

    These issues are all avoided by \verb|pst-bar|. It calls only on packages
    that are part of the \verb|pstricks| family, provides extensive
    customization features, relies on Postscript for nearly all mathematical
    operations, and should be compatible with plain \TeX\ (untested).
    \verb|pst-bar| provides commands for reading data from a comma-delimited
    file to produce and label a bar chart. Multiple series of data may be
    plotted in either clustered or stacked form, and bars may be plotted either
    horizontally or vertically. An additional option allows block charts that
    display the differences between subsequent rows of data in the data file
    (see Section~\ref{sec:block}).

    At this stage, \verb|pst-bar| should be regarded as beta software.
    Revisions to the code between this release and the first stable release are
    not guaranteed to be backward compatible.

  \section{Drawing a Bar Chart}
    Drawing a bar chart with \verb|pst-bar| is a three-step process: (i) read
    the data file, (ii) draw the bars, and (iii) draw the axes. The first two
    step are accomplished directly with \verb|pst-bar|; the third requires the
    \verb|pst-plot| package, a standard part of \verb|pstricks|.

    The following sections address the data file format, how to load data from
    the file, and how to use the data to produce a bar chart.

    \subsection{Data File Format}
      The data file must be comma-delimited with each row containing the same
      number of entries. A header row is permitted, but no header entry may
      contain commas or \verb|\par| commands (including \verb|\\|). One of the
      data files used for the examples in this manual looks like
        \verbatiminput{example1.csv}%
      In this case, \verb|Set 1|, \verb|Set 2|, and \verb|Set 3| are the
      headers. The subsequent rows, hereafter referred to as \emph{data rows},
      contain the data associated with these headers. The file is arranged in a
      pseudo-columnar fashion, so that the data associated with \verb|Set 1|
      are \verb|1| and \verb|1|, those with \verb|Set 2| are \verb|2| and
      \verb|2|, and so on. After these data are plotted, the bars will be
      labeled with the text from the header row.

      \emph{Warning!} There are two very important conditions on the data
      within the file:
      \begin{itemize}
        \item All data outside of the header must be numeric. If there is no
          header row, all data in the file must be numeric.
        \item Each row must contain the same number of entries.
      \end{itemize}
      Failure to meet these conditions may result in Postscript errors or (less
      frequently) incorrectly labeled or drawn bar charts.

      How these data are plotted depends on the type of plot desired. See
      Sections~\ref{sec:cluster}--\ref{sec:block} and the examples in
      Section~\ref{sec:examples} for more on this.

    \subsection{Reading the Data File}
      Use the command \verb|\readpsbardata| to read from a data file:
      \begin{verbatim}
    \readpsbardata[<options>]{<data>}{<filename>}\end{verbatim}
      For the data, simply specify a macro name not currently in use.

      The available options are:
      \begin{center}
        \begin{tabular}{ll}
          \toprule
            Parameter     & Definition\\
          \midrule
            \verb|header| & \verb|true| if first line of data file is header
                            row\\
                          & \verb|false| otherwise\\
                          & Default: \verb|true|\\
          \bottomrule
        \end{tabular}
      \end{center}
      If \verb|header| is set to \verb|false|, \verb|pst-bar| will assume that
      no header is present and treat the first line of the file as data.

    \subsection{Drawing the Bars}
      \label{sec:drawing_bars}%
      The command \verb|\psbarchart| draws the bars for the bar chart.
      \begin{verbatim}
    \psbarchart[<options>]{<data>}\end{verbatim}
      For \verb|<data>|, use the macro name supplied to \verb|\readpsbardata|.
      Bars are grouped in columns that are one \verb|\psxunit| wide, and the
      number of bars appearing in a single column is equal to the number of
      data rows in the data file. It labels the bottom of each column with the
      headers from the data file if \verb|header| was set to to \verb|true|
      during \verb|\readpsbardata|. It only creates the bars and the column
      labels --- axes and frames must be provided separately.

      The available options are:
      \begin{center}
        \begin{tabular}{ll}
          \toprule
            Parameter           & Definition\\
          \midrule
            \verb|chartstyle|   & \verb|cluster| produces a cluster of bars\\
                                & \verb|stack| stacks the cluster into a single
                                  column\\
                                & \verb|block| prints a ``floating'' bar with a
                                  nonzero minimum\\
                                & Default: \verb|cluster|\\
            \verb|barstyle|     & Name(s) of bar style(s) to use for each bar\\
                                & Default: \verb|{black,darkgray,gray,lightgray,white,red,green,blue}|\\
            \verb|barcolsep|    & Factor determining whitespace between
                                  columns\\
                                & Default: \verb|0.4|\\
            \verb|barsep|       & Factor determining whitespace between bars\\
                                & (\verb|cluster| and \verb|block| charts)\\
                                & Default: \verb|0.0|\\
            \verb|barlabelrot|  & Angle of rotation of the column labels\\
                                & Default: \verb|0|\\
            \verb|orientation|  & \verb|vertical| for vertically drawn bars\\
                                & \verb|horizontal| for horizontally drawn
                                  bars\\
                                & Default: \verb|vertical|\\
          \bottomrule%
        \end{tabular}
      \end{center}

      The number of \verb|barstyle|s specified must equal or exceed the number
      of bar segments in each column; otherwise, a Postscript error will
      result. For the \verb|cluster| and \verb|stack| \verb|chartstyle|s, this
      is equal to the number of data rows in the data file; for the
      \verb|block| \verb|chartstyle|, it is equal to half the number of data
      rows. Furthermore, when specifying the list of \verb|barstyle|s, the
      styles must be listed between curly braces, e.g.,
      \begin{verbatim}
    \psbarchart[barstyles={black,gray,white},...]{...}\end{verbatim}

      The \verb|barstyle|s available by default are \verb|red|, \verb|green|,
      \verb|blue|, \verb|black|, \verb|white|, \verb|gray|, \verb|lightgray|,
      and \verb|darkgray|. Each of these plots a bar of the named color with
      square corners and a black outline. New \verb|barstyle|s may be defined
      using the \verb|\newpsbarstyle| command described in
      Section~\ref{sec:customizing}.

      Figure~\ref{fig:bar_layout} shows how the \verb|\psbarchart| lays out the
      bars for the default \verb|chartstyle|, \verb|cluster|. For the
      \verb|stack| \verb|chartstyle|, the \verb|barsep| option is ignored.

      \begin{figure}[t]
        \centering%
        \begin{pspicture}(0,0)(4in,2in)%
          \psframe[linestyle=dashed](0,0)(4in,2in)%
          \psframe[linestyle=none,fillstyle=hlines,hatchangle=45,%
            hatchcolor=lightgray,hatchwidth=0.4pt,hatchsep=8pt](0,0)(1in,2in)%
          \psframe[fillcolor=darkgray,fillstyle=solid](0.2in,0in)(0.45in,0.8in)%
          \psframe[fillcolor=lightgray,fillstyle=solid](0.55in,0in)(0.8in,1.1in)%
          \psline[linestyle=dotted](1in,0in)(1in,2in)%
          \psframe[fillcolor=darkgray,fillstyle=solid](1.2in,0in)(1.45in,0.6in)%
          \psframe[fillcolor=lightgray,fillstyle=solid](1.55in,0in)(1.8in,0.4in)%
          \psline[linestyle=dotted](2in,0in)(2in,2in)%
          \psframe[fillcolor=darkgray,fillstyle=solid](2.2in,0in)(2.45in,0.4in)%
          \psframe[fillcolor=lightgray,fillstyle=solid](2.55in,0in)(2.8in,0.3in)%
          \psline[linestyle=dotted](3in,0in)(3in,2in)%
          \psframe[fillcolor=darkgray,fillstyle=solid](3.2in,0in)(3.45in,1.1in)%
          \psframe[fillcolor=lightgray,fillstyle=solid](3.55in,0in)(3.8in,0.9in)%
          \rput[b](0.5in,1.8in){\small$\mathtt{\backslash psxunit}$}%
          \psline{->}(0.2in,1.87in)(0in,1.87in)%
          \psline{->}(0.8in,1.87in)(1in,1.87in)%
          \psline(0.45in,0.85in)(0.45in,1.3in)%
          \psline(0.55in,1.15in)(0.55in,1.3in)%
          \rput[l](0.8in,1.25in){\small$\mathtt{barsep} \times
            \mathtt{\backslash psxunit}$}%
          \psline{->}(0.75in,1.25in)(0.55in,1.25in)%
          \psline{->}(0.25in,1.25in)(0.45in,1.25in)%
          \psline{->}(0.6in,0.9in)(0.8in,0.9in)%
          \psline{->}(1.2in,0.9in)(1in,0.9in)%
          \rput[l](1.25in,0.9in){\small$0.5 \times \mathtt{barcolsep} \times
            \mathtt{\backslash psxunit}$}%
        \end{pspicture}
        \caption{\label{fig:bar_layout}%
          Schematic diagram for the layout of the bar chart. The hatched area
          marks the region in which data from the first column of the data file
          will be plotted. Bar width is equal to $[(1 - \mathtt{barsep} - 0.5
          \times \mathtt{barcolsep}) \times \mathtt{\backslash psxunit}] / N$,
          where $N$ is the number of bars in the column.
        }%
      \end{figure}

    \subsection{Customizing the Chart}
      \label{sec:customizing}%
      To add a custom bar style, use the command \verb|\newpsbarstyle|,
      \begin{verbatim}
    \newpsbarstyle{<name>}{<definition>}\end{verbatim}
      The name is a text string not currently used by any other
      \verb|barstyle|, and the definition may consist of any \verb|pstricks|
      key or group of keys applicable to \verb|\psframe|. For example, the
      \verb|red| \verb|barstyle| is defined as
      \begin{verbatim}
    \newpsbarstyle{red}{fillcolor=red,fillstyle=solid,framearc=0}\end{verbatim}

      To customize the type used to set the column labels, redefine
      \verb|\psbarlabel|. For example, to have the label set in small italics,
      use
      \begin{verbatim}
    \renewcommand*{\psbarlabel}{\small\itshape}\end{verbatim}
      By default, \verb|\psbarlabel| sets the column label in the current text
      font.

      To adjust the spacing between the bars and the column labels, redefine
      \verb|\psbarlabelsep|. Note that it is defined as a command, not a
      length, and should be redefined using \verb|\renewcommand*|. By default
      it is \verb|0pt|.

      Finally, the data from the file may be scaled and manipulated using the
      command \verb|\psbarscale|,
      \begin{verbatim}
    \psbarscale(<scale>){<Postscript code>}\end{verbatim}
      inspired by Herbert Vo\ss' \verb|\pstScale| from
      \verb|pstricks-add|~\cite{voss:04}. The data are scaled by the value in
      parentheses and may be further manipulated with Postscript code. For
      example, to plot the logarithm of the input data one would use
      \verb|\psbarscale(1){log}|. The Postscript code is applied to the data
      prior to scaling.

  \section{The Chart Styles}
    As described in Section~\ref{sec:drawing_bars}, there are three different
    \verb|chartstyle| options. The default, \verb|cluster| is the common bar
    chart with bars clustered into groups of related data. The second option,
    \verb|stack|, draws a series of bars stacked atop one another instead of
    sitting side-by-side in a cluster. The third, \verb|block|, draws clustered
    bars between two data points, thus displaying a range of values instead of
    a single value.

    The following sections describe each \verb|chartstyle| in detail, including
    how it uses the input data. Examples of use are provided in each case. Full
    examples, including axes, appear in Section~\ref{sec:examples}.

    \subsection{\texttt{cluster}}
      \label{sec:cluster}%
      For a \verb|cluster| chart, each comma-delimited variable within a given
      row represents a bar in a different column. Each row represents a new set
      of bars. Thus the file
      \verbatiminput{example1.csv}%
      produces the bars

      \hfill%
      \begin{minipage}[b]{1.5in}%
        \psset{unit=0.5in}%
        \begin{pspicture}(0,-0.5)(3,3)%
          \readpsbardata{\data}{example1.csv}%
          \psbarchart[barstyle={red,blue}]{\data}%
        \end{pspicture}
      \end{minipage}%
      \hfill%
      \begin{minipage}[b][1.75in][c]{3.5in}%
        \begin{verbatim}
\psset{unit=0.5in}%
\begin{pspicture}(0,-0.5)(3,3)%
  \readpsbardata{\data}{example1.csv}%
  \psbarchart[barstyle={red,blue}]{\data}%
\end{pspicture}\end{verbatim}
      \end{minipage}%
      \hfill%

    \subsection{\texttt{stack}}
      For a \verb|stack| chart, each column of the chart has only one bar. This
      bar consists of as many segments as the data file has data rows, with the
      data from each row stacked onto the previous row. Thus, the file
      \verbatiminput{example2.csv}%
      produces the bars

      \hfill%
      \begin{minipage}[b]{1.5in}%
        \psset{unit=0.5in}%
        \begin{pspicture}(0,-0.5)(3,4)%
          \readpsbardata{\data}{example2.csv}%
          \psbarchart[barstyle={red,blue},chartstyle=stack]{\data}%
        \end{pspicture}
      \end{minipage}
      \hfill%
      \begin{minipage}[b][2.25in][c]{3.5in}%
        \begin{verbatim}
\psset{unit=0.5in}%
\begin{pspicture}(0,-0.5)(3,4)%
  \readpsbardata{\data}{example2.csv}%
  \psbarchart[barstyle={red,blue},%
    chartstyle=stack]{\data}%
\end{pspicture}\end{verbatim}
      \end{minipage}
      \hfill%

      Notice that these bars are wider than those produced by
      \verb|[chartstyle=cluster]| because the bars are stacked vertically
      instead of being clustered along the bottom axis. A bar width similar to
      that of the \verb|cluster| chart in the previous section could be
      obtained by setting \verb|xunit=0.25in|.

    \subsection{\texttt{block}}
      \label{sec:block}%
      For a \verb|block| chart, each bar represents a range of values. As such,
      each bar requires two data lines from the data file, one denoting the
      upper limit and the other the lower limit. If there are multiple pairs of
      data lines, they are plotted in a clustered fashion. If there are an odd
      number of data lines, the last line of the data file is ignored. Thus,
      the file
      \verbatiminput{example3.csv}%
      produces the bars

      \hfill%
      \begin{minipage}[b]{1.5in}%
        \psset{unit=0.5in}%
        \begin{pspicture}(0,-0.5)(3,3.5)%
          \readpsbardata{\data}{example3.csv}%
          \psbarchart[barstyle={red,blue},chartstyle=block]{\data}%
        \end{pspicture}
      \end{minipage}%
      \hfill%
      \begin{minipage}[b][1.5in][c]{3.5in}%
        \begin{verbatim}
\psset{unit=0.5in}%
\begin{pspicture}(0,-0.5)(3,3.5)%
  \readpsbardata{\data}{example3.csv}%
  \psbarchart[barstyle={red,blue},%
    chartstyle=block]{\data}%
\end{pspicture}\end{verbatim}
      \end{minipage}

  \section{Examples}
    \label{sec:examples}%
    Basic examples of each type including axes and gridlines:

    \begin{verbatim}
\begin{filecontents*}{example2.csv}
  Set 1, Set 2, Set 3
  1, 2, 3
  0.5, 1.5, 1
\end{filecontents*}\end{verbatim}

    \hfill%
    \begin{minipage}[b]{1.5in}%
      \psset{unit=0.5in}%
      \begin{pspicture}(0,-0.5)(3,4.5)%
        \psgrid[xunit=1.5in,gridlabels=0,subgriddiv=0,griddots=30](0,0)(1,4)%
        \psaxes[axesstyle=frame,Ox=0,Dx=1,labels=y,ticks=y](0,0)(3,4)%
        \readpsbardata{\data}{example2.csv}%
        \psbarchart[barstyle={red,blue}]{\data}%
      \end{pspicture}
    \end{minipage}%
    \hfill%
    \begin{minipage}[b][2.25in][c]{3.5in}%
      \begin{verbatim}
\psset{unit=0.5in}%
\begin{pspicture}(0,-0.5)(3,4.5)%
  \psgrid[xunit=1.5in,gridlabels=0,%
    subgriddiv=0,griddots=30](0,0)(1,4)%
  \psaxes[axesstyle=frame,Ox=0,Dx=1,labels=y,%
    ticks=y](0,0)(3,4)%
  \readpsbardata{\data}{example2.csv}%
  \psbarchart[barstyle={red,blue}]{\data}%
\end{pspicture}\end{verbatim}
    \end{minipage}

    \hfill%
    \begin{minipage}[b]{1.5in}%
      \psset{unit=0.5in}%
      \begin{pspicture}(0,-0.5)(3,4.5)%
        \psgrid[xunit=1.5in,gridlabels=0,subgriddiv=0,griddots=30](0,0)(1,4)%
        \psaxes[axesstyle=frame,Ox=0,Dx=1,labels=y,ticks=y](0,0)(3,4)%
        \readpsbardata{\data}{example2.csv}%
        \psbarchart[barstyle={red,blue},chartstyle=stack]{\data}%
      \end{pspicture}
    \end{minipage}%
    \hfill%
    \begin{minipage}[b][2.25in][c]{3.5in}%
      \begin{verbatim}
\psset{unit=0.5in}%
\begin{pspicture}(0,-0.5)(3,4.5)%
  \psgrid[xunit=1.5in,gridlabels=0,%
    subgriddiv=0,griddots=30](0,0)(1,4)%
  \psaxes[axesstyle=frame,Ox=0,Dx=1,labels=y,%
    ticks=y](0,0)(3,4)%
  \readpsbardata{\data}{example2.csv}%
  \psbarchart[barstyle={red,blue},%
    chartstyle=stack]{\data}%
\end{pspicture}\end{verbatim}
    \end{minipage}

    \hfill%
    \begin{minipage}[b]{1.5in}%
      \psset{unit=0.5in}%
      \begin{pspicture}(0,-0.5)(3,4.5)%
        \psgrid[xunit=1.5in,gridlabels=0,subgriddiv=0,griddots=30](0,0)(1,4)%
        \psaxes[axesstyle=frame,Ox=0,Dx=1,labels=y,ticks=y](0,0)(3,4)%
        \readpsbardata{\data}{example2.csv}%
        \psbarchart[barstyle={red,blue},chartstyle=block]{\data}%
      \end{pspicture}
    \end{minipage}%
    \hfill%
    \begin{minipage}[b][2.25in][c]{3.5in}%
      \begin{verbatim}
\psset{unit=0.5in}%
\begin{pspicture}(0,-0.5)(3,4.5)%
  \psgrid[xunit=1.5in,gridlabels=0,%
    subgriddiv=0,griddots=30](0,0)(1,4)%
  \psaxes[axesstyle=frame,Ox=0,Dx=1,labels=y,%
    ticks=y](0,0)(3,4)%
  \readpsbardata{\data}{example2.csv}%
  \psbarchart[barstyle={red,blue},%
    chartstyle=block]{\data}%
\end{pspicture}\end{verbatim}
    \end{minipage}

    Using \verb|\newpsbarstyle|:

    \hfill%
    \begin{minipage}[b]{1.5in}%
      \psset{unit=0.5in}%
      \newpsbarstyle{yellowhatch}{framearc=0.5,fillstyle=hlines*,rot=45,fillcolor=yellow}%
      \newpsbarstyle{redoutline}{framearc=0.5,fillcolor=black,linecolor=red,linewidth=1.5pt}%
      \begin{pspicture}(0,-0.5)(3,4.5)%
        \psgrid[xunit=1.5in,gridlabels=0,subgriddiv=0,griddots=30](0,0)(1,4)%
        \psaxes[axesstyle=frame,Ox=0,Dx=1,labels=y,ticks=y](0,0)(3,4)%
        \readpsbardata{\data}{example2.csv}%
        \psbarchart[barstyle={yellowhatch,redoutline}]{\data}%
      \end{pspicture}
    \end{minipage}%
    \hfill%
    \begin{minipage}[b][2.75in][c]{3.5in}%
      \begin{verbatim}
\psset{unit=0.5in}%
\newpsbarstyle{yellowhatch}{framearc=0.5,%
  fillstyle=hlines*,rot=45,fillcolor=yellow}%
\newpsbarstyle{redoutline}{framearc=0.5,%
  fillcolor=black,linecolor=red,%
  linewidth=1.5pt}%
\begin{pspicture}(0,-0.5)(3,4.5)%
  \psgrid[xunit=1.5in,gridlabels=0,%
    subgriddiv=0,griddots=30](0,0)(1,4)%
  \psaxes[axesstyle=frame,Ox=0,Dx=1,labels=y,%
    ticks=y](0,0)(3,4)%
  \readpsbardata{\data}{example2.csv}%
  \psbarchart[barstyle={yellowhatch,%
    redoutline}]{\data}%
\end{pspicture}\end{verbatim}
    \end{minipage}

    Using \verb|[orientation=horizontal]|:

    \hfill%
    \begin{minipage}[b]{2in}%
      \psset{unit=0.5in}%
      \newpsbarstyle{yellowhatch}{framearc=0.5,fillstyle=hlines*,rot=45,fillcolor=yellow}%
      \newpsbarstyle{redoutline}{framearc=0.5,fillcolor=black,linecolor=red,linewidth=1.5pt}%
      \begin{pspicture}(0,-0.5)(4,3.5)%
        \psgrid[yunit=1.5in,gridlabels=0,subgriddiv=0,griddots=30](0,0)(3,1)%
        \psaxes[axesstyle=frame,Ox=0,Dx=1,labels=x,ticks=x](0,0)(3,3)%
        \readpsbardata{\data}{example2.csv}%
        \psbarchart[barstyle={yellowhatch,redoutline},orientation=horizontal]{\data}%
      \end{pspicture}
    \end{minipage}%
    \hfill%
    \begin{minipage}[b][2.75in][c]{3.5in}%
      \begin{verbatim}
\psset{unit=0.5in}%
\newpsbarstyle{yellowhatch}{framearc=0.5,%
  fillstyle=hlines*,rot=45,fillcolor=yellow}%
\newpsbarstyle{redoutline}{framearc=0.5,%
  fillcolor=black,linecolor=red,%
  linewidth=1.5pt}%
\begin{pspicture}(0,-0.5)(3,4.5)%
  \psgrid[xunit=1.5in,gridlabels=0,%
    subgriddiv=0,griddots=30](0,0)(1,4)%
  \psaxes[axesstyle=frame,Ox=0,Dx=1,labels=y,%
    ticks=y](0,0)(3,4)%
  \readpsbardata{\data}{example2.csv}%
  \psbarchart[barstyle={yellowhatch,%
    redoutline}]{\data}%
\end{pspicture}\end{verbatim}
    \end{minipage}

    Sophisticated effects are possible using multiple calls to
    \verb|\psbarchart| in a single chart. For example, a particular important
    bar may be highlighted in green by splitting the data into two files, one
    containing a zero for the data value to be highlighted, and the other
    containing zeros for all of the data \emph{except} the value to be
    highlighted.

    \vspace*{2ex}%

    File 1 (\verb|example4.csv|):
    \verbatiminput{example4.csv}%

    File 2 (\verb|example5.csv|):
    \verbatiminput{example5.csv}%
    Notice that the second file has no header row in order to prevent the bar
    labels from being printed twice.

    \hfill%
    \begin{minipage}[b]{1.5in}%
      \psset{unit=0.5in}%
      \begin{pspicture}(0,-0.5)(3,4.5)%
        \psgrid[xunit=1.5in,gridlabels=0,subgriddiv=0,griddots=30](0,0)(1,4)%
        \psaxes[axesstyle=frame,Ox=0,Dx=1,labels=y,ticks=y](0,0)(3,4)%
        \readpsbardata{\data}{example4.csv}%
        \psbarchart[barstyle={blue}]{\data}%
        \readpsbardata[header=false]{\data}{example5.csv}%
        \psbarchart[barstyle={green}]{\data}%
      \end{pspicture}
    \end{minipage}%
    \hfill%
    \begin{minipage}[b][2.25in][c]{3.5in}%
      \begin{verbatim}
\psset{unit=0.5in}%
\begin{pspicture}(0,-0.5)(3,4.5)%
  \psgrid[xunit=1.5in,gridlabels=0,%
    subgriddiv=0,griddots=30](0,0)(1,4)%
  \psaxes[axesstyle=frame,Ox=0,Dx=1,labels=y,%
    ticks=y](0,0)(3,4)%
  \readpsbardata{\data}{example4.csv}%
  \psbarchart[barstyle={blue}]{\data}%
  \readpsbardata[header=false]{\data}%
    {example5.csv}%
  \psbarchart[barstyle={green}]{\data}%
\end{pspicture}\end{verbatim}
    \end{minipage}

    Different chart styles may also be combined in a single bar chart.

    \vspace*{2ex}%

    File 1 (\verb|example6.csv|): \verbatiminput{example6.csv}%

    File 2 (\verb|example7.csv|): \verbatiminput{example7.csv}%

    \hfill%
    \begin{minipage}[b]{1.5in}%
      \psset{unit=0.5in}%
      \begin{pspicture}(0,-0.5)(3,4.5)%
        \psgrid[xunit=1.5in,gridlabels=0,subgriddiv=0,griddots=30](0,0)(1,4)%
        \psaxes[axesstyle=frame,Ox=0,Dx=1,labels=y,ticks=y](0,0)(3,4)%
        \readpsbardata{\data}{example6.csv}%
        \psbarchart[barstyle={black}]{\data}%
        \readpsbardata[header=false]{\data}{example7.csv}%
        \psbarchart[barstyle={red},chartstyle=block,barcolsep=0.8]{\data}%
      \end{pspicture}
    \end{minipage}%
    \hfill%
    \begin{minipage}[b][2.25in][c]{3.5in}%
      \begin{verbatim}
\psset{unit=0.5in}%
\begin{pspicture}(0,-0.5)(3,4.5)%
  \psgrid[xunit=1.5in,gridlabels=0,%
    subgriddiv=0,griddots=30](0,0)(1,4)%
  \psaxes[axesstyle=frame,Ox=0,Dx=1,labels=y,%
    ticks=y](0,0)(3,4)%
  \readpsbardata{\data}{example6.csv}%
  \psbarchart[barstyle={black}]{\data}%
  \readpsbardata[header=false]{\data}%
    {example7.csv}%
  \psbarchart[barstyle={red},chartstyle=block,%
    barcolsep=0.8]{\data}%
\end{pspicture}
\end{pspicture}\end{verbatim}
    \end{minipage}

  \section{To Do}
    \begin{itemize}
      \item Provide commands to facilitate legend creation and placement.
      \item Allow the automatic labeling of bars with values from the data
        file.
      \item Allow the automatic labeling of bars with labels of the user's
        choice.
      \item Improve header parsing to allow commas within header entries.
      \item Add error-checking to ensure that each row of the data file
        contains the same number of entries, throwing an error in \TeX\ or
        \LaTeX\ if not.
      \item Improve documentation.
    \end{itemize}

  \bibliographystyle{plain}%
  \bibliography{\jobname}%

\printindex


\end{document}
