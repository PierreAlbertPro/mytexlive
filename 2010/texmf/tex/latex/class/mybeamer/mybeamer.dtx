% \iffalse meta-comment
%
% Copyright (C) 2011 by Jean SIMARD
%
% This file may be distributed and/or modified under the conditions of the 
%LaTeX Project Public License, either version 1.2 of this license or (at your 
%option) any later version.  The latest version of this license is in:
%
% http://www.latex-project.org/lppl.txt
%
% and version 1.2 or later is part of all distributions of LaTeX version 
%1999/12/01 or later.
%
% \fi

% \iffalse
%<*driver>
\ProvidesFile{mybeamer.dtx}
%</driver>
%<class>\NeedsTeXFormat{LaTeX2e}[1999/12/01]
%<class>\ProvidesClass{mybeamer}
%<*class>
	[2011/12/14 v1.0 Beamer presentation]
%</class>
%<*driver>
\documentclass[french]{ltxdoc}
\usepackage{holtxdoc}[2008/08/11]
\usepackage{my}
\usepackage{mymacro}
\usepackage{hyperref}
\usepackage{mycolor}
\usepackage[smartref]{myref}
\usepackage{myfloat}
\usepackage{tikz}
\EnableCrossrefs
\CodelineIndex 
\RecordChanges
\begin{document}
\DocInput{mybeamer.dtx}
\end{document}
%</driver>
% \fi
%
% \CheckSum{0}
%
%% \CharacterTable
%%  {Upper-case  \A\B\C\D\E\F\G\H\I\J\K\L\M\N\O\P\Q\R\S\T\U\V\W\X\Y\Z
%%   Lower-case  \a\b\c\d\e\f\g\h\i\j\k\l\m\n\o\p\q\r\s\t\u\v\w\x\y\z
%%   Digits    \0\1\2\3\4\5\6\7\8\9
%%   Exclamation   \!   Double quote  \"   Hash (number) \#
%%   Dollar    \$   Percent     \%   Ampersand   \&
%%   Acute accent  \'   Left paren  \(   Right paren   \)
%%   Asterisk    \*   Plus      \+   Comma     \,
%%   Minus     \-   Point     \.   Solidus     \/
%%   Colon     \:   Semicolon   \;   Less than   \<
%%   Equals    \=   Greater than  \>   Question mark \?
%%   Commercial at \@   Left bracket  \[   Backslash   \\
%%   Right bracket \]   Circumflex  \^   Underscore  \_
%%   Grave accent  \`   Left brace  \{   Vertical bar  \|
%%   Right brace   \}   Tilde     \~}
%
% \changes{v1.0}{2011/12/14}{Initial version}
%
% \GetFileInfo{mybeamer.dtx}
%
% \DoNotIndex{\begin,\end,\\}
%
% \title{The \xclass{mybeamer} class\\a nice presentation with 
%\xclass{beamer}\thanks{This document corresponds to the 
%\xclass{mybeamer}~\fileversion, dated~\filedate.}}
% \author{Jean 
%\myname{Simard}\\\href{mailto:juste.lapin@gmail.com}{\xemail{juste.lapin@gmail.com}}}
%
% \maketitle
%
% \begin{abstract}
%   In order to provide default configuration of \xclass{beamer} class, the 
%\xclass{mybeamer} class is defining a default theme and some useful commands 
%and environments to create a complete slide presentation.
% \end{abstract}
% \tableofcontents
%
% \section{Introduction}
% The \xclass{mybeamer} class is a \xclass{beamer} based class to create a 
%default configuration.
% It adds few theme configuration and a progress bar on the bottom of each 
%slide in the presentation.
%
% \section{Usage}
% The \xclass{mybeamer} class does not really define new commands and new 
%environments for the user.
% It mainly declare a new ``look'' for the slide presentation, based on the 
%\xclass{beamer} class.
% The theme is coloured with a palette on white and red.
%
% \DescribeMacro{\myinsertprogressbar}
% \cs{myinsertprogressbar} is mainly defined for use in the predefined theme of 
%the \xclass{mybeamer} class.
% It allows to draw a progress bar which take the whole width of the slide and 
%looks like the \myref{fig-LookOfTheProgressBar}.
% \begin{figure}
% \makeatletter
% \newdimen\progressbar@currentbarlength
% \newdimen\progressbar@framenumberrectangle
% \newdimen\progressbar@titlerectangle
% \newdimen\progressbar@leftbar
% \newcount\progressbar@tmpresult
% \newcount\progressbar@numer
% \newcount\progressbar@denom
% \newcount\progressbar@barlength
% \progressbar@framenumberrectangle=\textwidth
% \progressbar@titlerectangle=\textwidth
% \advance\progressbar@framenumberrectangle by -0.9cm
% \advance\progressbar@titlerectangle by -1.1cm
% \progressbar@barlength=110
% \progressbar@leftbar=\progressbar@titlerectangle
% \advance\progressbar@leftbar by -\progressbar@barlength mm
% \def\inserttotalframenumber{36}
% \def\insertframenumber{12}
% \def\inserttitle{Title of the presentation}
% \ifnum\inserttotalframenumber=1\else
% \progressbar@numer=\insertframenumber
% \advance\progressbar@numer by -1
% \progressbar@denom=\inserttotalframenumber
% \advance\progressbar@denom by -1
% \progressbar@tmpresult=\progressbar@barlength
% \multiply\progressbar@tmpresult by \progressbar@numer
% \divide\progressbar@tmpresult by \progressbar@denom
% \progressbar@currentbarlength=\progressbar@tmpresult mm
% \begin{tikzpicture}
% \shade[top color=white, bottom color=white!80!myred] (0, 0) rectangle  
(\textwidth, 0.6cm);
% \shade[left color=white,right color=white!70!myred] (.5\textwidth, 0.2cm) 
%rectangle (\textwidth, 0.22cm);
% \draw (\progressbar@framenumberrectangle, 0.25cm) node [anchor=mid west, 
%draw=white!70!myred, fill=white] 
%{\sf\tiny\color{myred!70!white}\insertframenumber~/~\inserttotalframenumber};
% \draw (\progressbar@titlerectangle, 0.32cm) node [anchor=south east] 
%{\sf\scriptsize\color{white!70!myred}\inserttitle};
% \fill (\progressbar@leftbar, 0.12cm) [fill=white, rounded corners=0.1cm] 
%rectangle (\progressbar@titlerectangle, 0.3cm);
% \ifnum\insertframenumber=1\else
% \shade[left color=myred!25!white, right color=myred!50!white, rounded 
%corners=0.1cm] (\progressbar@leftbar, 0.12cm) rectangle 
%++(\progressbar@currentbarlength, 0.2cm);
% \fi
% \draw (\progressbar@leftbar, 0.11cm) [draw=white!70!myred, rounded 
%corners=0.1cm] rectangle ++(\progressbar@barlength mm, 0.2cm);
% \end{tikzpicture}
% \mycaption[fig-LookOfTheProgressBar]{Look of the progress bar}
% \makeatother
% \end{figure}
%
% \section{Implementation}
% \subsection{Initialization}
% Load \xpackage{my} package to load default configuration.
%    \begin{macrocode}
\RequirePackageWithOptions{my}[2011/12/08]
%    \end{macrocode}
%
% Moreover, we will load the \xclass{beamer} class.
% Some default options are loaded.
% \begin{itemize}
%   \item \xoption{fleqn} place all equations one the left of page (other 
%possibility is the center)
%   \item \xoption{xcolor=pst} to make \xpackage{xcolor} aware of the fact that 
%we probably need the PSTricks package
% \end{itemize}
%    \begin{macrocode}
\ExecuteOptions{fleqn,xcolor=pst}
\LoadClassWithOptions{beamer}[2010/06/21]
%    \end{macrocode}
%
% No option is declared for this class.
% A warning is output for each non-declared option called.
%    \begin{macrocode}
\DeclareOptionX*{%
  \ClassWarning{mybeamer}{Unknown option `\CurrentOption'}%
}
\ProcessOptions\relax
%    \end{macrocode}
%
% Some useful packages are loaded here.
% \begin{description}
%   \item[\xpackage{tikz}] is used to draw the progress bar
%   \item[\xpackage{mycolor}] for a bundle of personal colors
% \end{description}
%    \begin{macrocode}
\RequirePackage{tikz}
\RequirePackage{mycolor}
%    \end{macrocode}
%
% \subsection{The progress bar}
% Some dimensions variables\dots{}
%    \begin{macrocode}
  \newdimen\progressbar@currentbarlength
  \newdimen\progressbar@framenumberrectangle
  \newdimen\progressbar@titlerectangle
  \newdimen\progressbar@leftbar
%    \end{macrocode}
% \dots{} and counter variables are defined to draw the progress bar.
%    \begin{macrocode}
  \newcount\progressbar@tmpresult
  \newcount\progressbar@numer
  \newcount\progressbar@denom
  \newcount\progressbar@barlength
%    \end{macrocode}
%
% The dimensions (defined in the previous lines) are now initialized.
% The progress bar will take the whole width of the page with little borders.
%    \begin{macrocode}
  \progressbar@framenumberrectangle=\paperwidth
  \progressbar@titlerectangle=\paperwidth

  \advance\progressbar@framenumberrectangle by -0.9cm
  \advance\progressbar@titlerectangle by -1.1cm

  \progressbar@barlength=115 % (in millimeters)
  \progressbar@leftbar=\progressbar@titlerectangle
  \advance\progressbar@leftbar by -\progressbar@barlength mm
%    \end{macrocode}
%
% \begin{macro}{\myinsertprogressbar}
% Here, the progress bar command is defined with the help of \xpackage{tikz} 
%package.
% The progress bar is drawn depending on the number of the current page and the 
%total number of pages.
%    \begin{macrocode}
  \def\myinsertprogressbar{
%    \end{macrocode}
% To begin, we define all the dimensions which are changing on each slide 
%(dimensions which depends on the number of the current page).
%    \begin{macrocode}
    \ifnum\inserttotalframenumber=1\else
    \progressbar@numer=\insertframenumber
    \advance\progressbar@numer by -1
    \progressbar@denom=\inserttotalframenumber
    \advance\progressbar@denom by -1
    \progressbar@tmpresult=\progressbar@barlength
    \multiply\progressbar@tmpresult by \progressbar@numer
    \divide\progressbar@tmpresult by \progressbar@denom
    \progressbar@currentbarlength=\progressbar@tmpresult mm
%    \end{macrocode}
%
% The \texttt{beamercolorbox} allow user to define new box in the 
%\xclass{beamer} class.
% The progress bar is insert in a \texttt{beamercolorbox}.
% Moreover, the progress bar is defined with the help of the \xpackage{tikz} 
%package (in a \texttt{tikzpicture} environment).
%    \begin{macrocode}
    \begin{beamercolorbox}{progressbar primary}
      \begin{tikzpicture}
        \shade[top color=bg, bottom color=bg!80!fg]%
          (0, 0) rectangle  (\paperwidth, 0.6cm);
        \shade[left color=bg,right color=bg!70!fg]%
          (.5\paperwidth, 0.2cm) rectangle (\paperwidth, 0.22cm);
        \draw (\progressbar@framenumberrectangle, 0.21cm) node%
          [anchor=mid west, draw=bg!70!fg, fill=bg]
          {%
            \color{structure.fg!70!bg}%
            \insertframenumber~/~\inserttotalframenumber%
          };
        \draw (\progressbar@titlerectangle, 0.32cm) node%
          [anchor=south east] {\color{bg!70!fg}\inserttitle};
        \fill (\progressbar@leftbar, 0.12cm)%
          [fill=bg, rounded corners=0.1cm]%
          rectangle (\progressbar@titlerectangle, 0.3cm);
        \ifnum\insertframenumber=1\else
        \shade[left color=progressbar primary.fg!25!bg,%
          right color=progressbar primary.fg!50!bg,%
          rounded corners=0.1cm] (\progressbar@leftbar, 0.12cm)
          rectangle ++(\progressbar@currentbarlength, 0.2cm);
        \fi
        \draw (\progressbar@leftbar, 0.11cm)%
          [draw=bg!70!fg, rounded corners=0.1cm]%
          rectangle ++(\progressbar@barlength mm, 0.2cm);
      \end{tikzpicture}
    \end{beamercolorbox}
    \fi
  }
%    \end{macrocode}
% \end{macro}
%
% \subsection{The theme}
% The theme of the \xclass{mybeamer} class use some of the default theme of the 
%\xclass{beamer} class and enhance them.
% The used theme is \xoption{CambridgeUS} with the inner theme 
%\xoption{rectangle}.
%    \begin{macrocode}
  \usetheme{CambridgeUS}
  \useinnertheme{rectangles}
%    \end{macrocode}
%
% The title page theme is totally redefined.
% Thin lines are separating the title (on the top) from the author and 
%institute (in the middle) and from the date (on the bottom).
% Progress bar is insert on the bottom border of the title page.
%    \begin{macrocode}
  \defbeamertemplate*{title page}{progressbar theme}{%
    \pgfdeclarehorizontalshading{separationtitlepagelineshading}%
      {0.5pt}%
      {color(0cm)=(bg);%
      color(0.5\textwidth)=(structure.fg); color(\textwidth)=(bg)}%

    \makeatletter%
    \begin{center}%
      \textbf{\textcolor{structure.fg}\large\inserttitle}%

      \pgfuseshading{separationtitlepagelineshading}%
      \vskip\baselineskip%
      \footnotesize\insertauthor\\[\baselineskip]%
      \ifx\insertinstitute\@empty%
      \else%
        \tiny\insertinstitute\\[\baselineskip]%
      \fi%
      \insertlogo%
      \vskip\baselineskip%
      \pgfuseshading{separationtitlepagelineshading}%
      \vfill%
      \footnotesize  %
      \insertdate%
    \end{center}%
    \makeatother%
  }
%    \end{macrocode}
%
% The footline theme is also redefined with the progress bar.
%    \begin{macrocode}
  \defbeamertemplate*{footline}{progressbar theme}{%
    \begin{beamercolorbox}%
        [wd=\paperwidth,ht=0.6cm,dp=0ex]%
        {progressbar in head/foot}%
      \myinsertprogressbar%
    \end{beamercolorbox}%
  }
%    \end{macrocode}
%
% The headline theme is displaying the main section and subsection of the 
%current slide.
% These informations are output on one line (not too much space).
%    \begin{macrocode}
  \defbeamertemplate*{headline}{split theme}{%
    \leavevmode%
    \begin{beamercolorbox}%
        [wd=.5\paperwidth,ht=2.5ex,dp=1.125ex]%
        {section in head/foot}%
      \insertsectionnavigationhorizontal{.5\paperwidth}%
        {\hskip0pt plus1filll}{}%
    \end{beamercolorbox}%
    \begin{beamercolorbox}%
        [wd=.5\paperwidth,ht=2.5ex,dp=1.125ex]%
        {subsection in head/foot}%
      \insertsubsectionnavigationhorizontal{.5\paperwidth}%
        {}%
        {\hskip0pt plus1filll}%
    \end{beamercolorbox}%
  }
%    \end{macrocode}
%
% All new theme are activated for the current presentation.
%    \begin{macrocode}
  \setbeamertemplate{title page}[progressbar theme]
  \setbeamertemplate{headline}[infolines theme]
  \setbeamertemplate{footline}[progressbar theme]
  \setbeamertemplate{navigation symbols}{}
%    \end{macrocode}
%
% Some colors are changed in order to have a more ``red coloured'' theme.
%    \begin{macrocode}
  \setbeamercolor{progressbar primary}%
    {fg=myred}
  \setbeamercolor{structure}%
    {fg=myred}
  \setbeamercolor{block title}%
    {fg=white,bg=myred}
  \setbeamercolor{block body}%
    {bg=light-black}
  \setbeamercolor{example text}%
    {fg=white,bg=myred}
  \setbeamercolor{section in head/foot}%
    {parent=palette primary,fg=black}
  \setbeamercolor{subsection in head/foot}%
    {parent=palette primary,fg=myred}
  \setbeamercolor{alerted text}%
    {fg=myred}
%    \end{macrocode}
%
%  \PrintChanges
%  \PrintIndex
%
% \Finale
\endinput
