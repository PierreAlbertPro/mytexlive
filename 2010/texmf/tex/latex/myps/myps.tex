%% $Id: myps.tex 345 2011-07-29 19:00:00Z simard $
%%
%%
%% This is file `myps.tex',
%%
%% IMPORTANT NOTICE:
%%
%% Package `myps.tex'
%%
%% Jean SIMARD <juste.lapin@gmail.com>
%%
%% This program can be redistributed and/or modified under the terms
%% of the LaTeX Project Public License Distributed from CTAN archives
%% in directory macros/latex/base/lppl.txt.
%%
%% DESCRIPTION:
%%   `myps' is a PSTricks package for additionals to the standard
%%         pstricks package
%%
\csname PSTricksMyPSLoaded\endcsname
\let\PSTricksMyPSLoaded\endinput
%
% Requires some packages
\ifx\PSTricksLoaded\endinput\else \RequirePackage{pstricks} \fi
\ifx\PSTricksAddLoaded\endinput\else \RequirePackage{pstricks-add} \fi
\ifx\PSTplotLoaded\endinput\else  \RequirePackage{pst-plot} \fi
\ifx\PSTnodesLoaded\endinput\else \RequirePackage{pst-node} \fi
\ifx\PSTthreeDLoaded\endinput\else\RequirePackage{pst-3d}   \fi
\ifx\PSTthreeDplotLoaded\endinput\else\RequirePackage{pst-3dplot}   \fi
\ifx\MultidoLoaded\endinput\else  \RequirePackage{multido}  \fi
\ifx\PSTXKeyLoaded\endinput\else  \RequirePackage{pst-xkey} \fi
\ifx\PSTmathLoaded\endinput\else  \RequirePackage{pst-math} \fi
\ifx\PSTTextPathLoaded\endinput\else  \RequirePackage{pst-text} \fi
\ifx\PSTBarLoaded\endinput\else  \RequirePackage{pst-bar} \fi
\ifx\PSTBlurLoaded\endinput\else  \RequirePackage{pst-blur} \fi
\ifx\PSTcoilsLoaded\endinput\else  \RequirePackage{pst-coil} \fi
\ifx\GradientLoaded\endinput\else  \RequirePackage{pst-grad} \fi
\RequirePackage{ifdraft}
\RequirePackage{mycolor}
%
\def\fileversion{0.1}
\def\filedate{2011/07/29}
\message{`myps' v\fileversion, \filedate\space (dr,hv)}
%
\edef\PstAtCode{\the\catcode`\@} \catcode`\@=11\relax
\AtBeginDocument{\shorthandoff{!}}
\SpecialCoor
\pst@addfams{myps}
%
%% prologue for postcript
%\pstheader{myps.pro}%
%
% To define the width of whiskers lines for boxplot
\newdimen\myps@whiskerwidth
\define@key[psset]{myps}{pswhiskerwidth}[0.5pt]{\pssetlength\myps@whiskerwidth{#1}}
\psset[myps]{pswhiskerwidth=0.5pt}
% To define the width of median line for boxplot
\newdimen\myps@medianwidth
\define@key[psset]{myps}{psmedianwidth}[1pt]{\pssetlength\myps@medianwidth{#1}}
\psset[myps]{psmedianwidth=1pt}
% The '.' will be the separator for decimal in floats
\fpDecimalSign{.}%
% For use in '\myaxes'
\newlength{\my@ps@xoffset}
\newlength{\my@ps@yoffset}
\newlength{\my@ps@temp@xoffset}
% Create default style for graphics (limited to 4 styles)
\newpsbarstyle{first-barstyle}{framearc=0.25,fillstyle=solid,fillcolor=myblue,linestyle=none,linewidth=0.0pt}
\newpsbarstyle{second-barstyle}{framearc=0.25,fillstyle=solid,fillcolor=myblue!70,linestyle=none,linewidth=0.0pt}
\newpsbarstyle{third-barstyle}{framearc=0.25,fillstyle=solid,fillcolor=myblue!45,linestyle=none,linewidth=0.0pt}
\newpsbarstyle{fourth-barstyle}{framearc=0.25,fillstyle=solid,fillcolor=myblue!25,linestyle=none,linewidth=0.0pt}
\newpsbarstyle{fifth-barstyle}{framearc=0.25,fillstyle=solid,fillcolor=myblue!10,linestyle=none,linewidth=0.0pt}
\newpsbarstyle{sixth-barstyle}{framearc=0.25,fillstyle=solid,fillcolor=myblue!0,linestyle=none,linewidth=0.0pt}
\psset{barstyle={first-barstyle,second-barstyle,third-barstyle,fourth-barstyle,fifth-barstyle,sixth-barstyle}}

%%%%%%%%%%%%%%%%%%
%% New commands %%
%%%%%%%%%%%%%%%%%%
% Define a master command for 'pspicture'
% This command introduce a border frame for draft compilation
\def\myps{\pst@object{myps}}
\def\myps@i(#1,#2){%
	\@ifnextchar({\myps@ii(#1,#2)}{\myps@ii(0,0)(#1,#2)}%)
}
\def\myps@ii(#1,#2)(#3,#4){%
	\pspicture(#1,#2)(#3,#4)%
	\use@par
	\def\my@ne{#1}%
	\def\mytw@{#2}%
	\def\mythr@@{#3}%
	\def\myf@ur{#4}%
}
\def\endmyps{%
	\ifdraft%
	{%
		% To delimit the border of figure
		\psframe[shadow=false,framearc=0,fillstyle=none,linestyle=solid,linewidth=1pt,linecolor=myred](\my@ne,\mytw@)(\mythr@@,\myf@ur)%
	}{%
	}%
	\endpspicture%
	\ignorespaces%
}
\@namedef{myps*}{\myps}
\@namedef{endmyps*}{\endmyps}

\def\mygraph{\pst@object{mygraph}}
\def\mygraph@i{\pst@getarrows\mygraph@ii}
\def\mygraph@ii(#1,#2){%
	\catcode`\!=12\relax%
	\@ifnextchar({\mygraph@iii(#1,#2)}{\mygraph@iii(0,0)(#1,#2)}%)
}
\def\mygraph@iii(#1,#2)(#3,#4){%
	\@ifnextchar({\mygraph@iv(#1,#2)(#3,#4)}{\mygraph@iv(#1,#2)(#1,#2)(#3,#4)}%)
}
\def\mygraph@iv(#1,#2)(#3,#4)(#5,#6)#7#8{%
	\psgraph(#1,#2)(#3,#4)(#5,#6){#7}{#8}%
	\use@par%
}
\def\endmygraph{%
	\endpsgraph%
	\ignorespaces%
}
\@namedef{mygraph*}{\mygraph}
\@namedef{endmygraph*}{\endmygraph}

\def\myboxgraph{\pst@object{myboxgraph}}
\def\myboxgraph@i(#1,#2){%
	\@ifnextchar[{\myboxgraph@ii({#1},{#2})}{\myboxgraph@ii({#1},{#2})[1]}%]
}
\def\myboxgraph@ii(#1,#2)[#3]#4(#5,#6)#7{%
	\psset{llx=-2.5em,lly=-4.5ex,urx=0em,ury=0ex}
	\use@par%
	\mygraph[%
	xAxisLabel={#4},%
	xAxisLabelPos={c,-4ex},%
	yAxisLabel={#7},%
	yAxisLabelPos={-2.5em,c},%
	Dy={#3},%
	labels=y,%
	ticks=y,%
	yticksize={0 #2},%
	ticklinestyle=dashed]{->}(0,0)(0,0)(#1,#5){#2}{#6}
}
\def\endmyboxgraph{%
	\endmygraph%
	\ignorespaces
}
\@namedef{myboxgraph*}{\myboxgraph}
\@namedef{endmyboxgraph*}{\endboxmygraph}

% Create an additionnal 'chartstyle'
% cluster, stack, block, boxplot
\def\psset@@chartstyle#1#2\@nil#3{%
	\def\my@ps@chartstyle{#1#2}%
	\ifthenelse{\equal{\my@ps@chartstyle}{cluster}}%
		{\let#3\z@}%
		{%
			\ifthenelse{\equal{\my@ps@chartstyle}{stack}}%
				{\let#3\@one}%
				{%
					\ifthenelse{\equal{\my@ps@chartstyle}{block}}%
						{\let#3\tw@}%
						{%
							\ifthenelse{\equal{\my@ps@chartstyle}{boxplot}}%
								{\let#3\thr@@}%
								{\@pstrickserr{Bad argument: `#1#2'}\@ehpa}%
						}%
				}%
		}%
}%

% Redefine the bar chart to have boxplot
\def\psbarchart@ii#1{%
	\begin@SpecialObj%

		% Save contents of \pst@code and load start-of-path code
		% into \pst@tempc
		\let\pst@tempb\pst@code%
		\def\pst@code{}%
		\solid@star%
		\let\pst@tempc\pst@code%

		% Load end-of-path code into \pst@tempd and restore original
		% contents of \pst@code
		\begin@barstyle%
		\expandafter\@setbarstyle\@barstylelist,\@nil\ignorespaces%
		\end@barstyle%
		\let\pst@code\pst@tempb%

		% Draw bar chart
		\pst@checknum\psk@barcolsep\pst@tempa%
		\pst@checknum\psk@barsep\pst@tempb%
		\ifx\psk@orientation\z@             % vertical
			\pst@dima=\psxunit%
			\pst@dimb=\psyunit%
			\def\pst@tempc{true}%
		\else\ifx\psk@orientation\p@        % horizontal
			\pst@dima=\psyunit%
			\pst@dimb=\psxunit%
			\def\pst@tempc{false}%
		\else
			\@pstrickserr{Bad orientation specification}\@ehpa
		\fi\fi
		\ifcase\psk@chartstyle% chartstyle=cluster
			\addto@pscode{%
				\pst@tempd%
				/BARDATA #1 def
				/nbars BARDATA length def
				/ncols BARDATA 0 get length def
				/colwidth \pst@number\pst@dima def
				/barcolsep \pst@tempa \pst@number\pst@dima mul def
				/barsep \pst@tempb \pst@number\pst@dima mul def
				/barwidth colwidth barcolsep sub nbars 1 sub barsep mul sub nbars div def
				/bXoffset 0.5 barcolsep mul def
				/colcount \pst@tempc\space {0}{ncols 1 sub} ifelse def
				/barcount 0 def
				/ybar1 0 def
				BARDATA {
					/DATAVECTOR exch def
					DATAVECTOR {
						/ybar2 exch \psbar@psop \psbar@mul mul \pst@number\pst@dimb mul def
						/xoffset barwidth barsep add barcount mul bXoffset add def
						/xbar1 colcount colwidth mul xoffset add def
						/xbar2 xbar1 barwidth add def
						ybar1 ybar2 ne {  % if ybar1 == ybar2, don't stroke a path
							newpath
							barstyles barcount get cvx exec
						} if
						/colcount \pst@tempc\space {colcount 1 add}{colcount 1 sub} ifelse def
					} forall
					/colcount \pst@tempc\space {0}{ncols 1 sub} ifelse def
					/barcount barcount 1 add def
				} forall
			}%
		\or%  chartstyle=stack
			\addto@pscode{%
				\pst@tempd%
				tx@BarDict begin
				/BARDATA #1 transpose def
				/ncols BARDATA length def
				/colwidth \pst@number\pst@dima def
				/barcolsep \pst@tempa \pst@number\pst@dima mul def
				/barwidth colwidth barcolsep sub def
				/bXoffset 0.5 barcolsep mul def
				/colcount \pst@tempc\space {0}{ncols 1 sub} ifelse def
				/barcount 0 def
				/ybar1 0 def
				BARDATA {
					/DATAVECTOR exch def
					/xbar1 colcount colwidth mul bXoffset add def
					/xbar2 xbar1 barwidth add def
					/count 0 def
					DATAVECTOR {
						count 0 eq {
							/ybar2 exch \psbar@psop \psbar@mul mul \pst@number\pst@dimb mul ybar1 add def
						}{
							/ybar2 exch \psbar@mul mul \pst@number\pst@dimb mul ybar1 add def
						} ifelse
						ybar1 ybar2 ne {  % if ybar1 == ybar2, don't stroke a path
							newpath
							barstyles barcount get cvx exec
							closepath
						} if
						/ybar1 ybar2 def
						/barcount barcount 1 add def
						/count count 1 add def
					} forall
					/barcount 0 def
					/ybar1 0 def
					/colcount \pst@tempc\space {colcount 1 add}{colcount 1 sub} ifelse def
				} forall
				end
			}%
		\or% chartstyle=block
			\addto@pscode{%
				\pst@tempd%
				tx@BarDict begin
				/BARDATA #1 transpose def
				/ncols BARDATA length def
				/nbars BARDATA 0 get length 2 idiv def
				/colwidth \pst@number\pst@dima def
				/barcolsep \pst@tempa \pst@number\pst@dima mul def
				/barsep \pst@tempb \pst@number\pst@dima mul def
				/barwidth colwidth barcolsep sub nbars 1 sub barsep mul sub nbars div def
				/bXoffset 0.5 barcolsep mul def
				/colcount \pst@tempc\space {0}{ncols 1 sub} ifelse def
				/barcount 0 def
				/ybar1 0 def
				BARDATA {
					/DATAVECTOR exch def
					0 1 nbars 1 sub {
						dup
						/ybar1 exch 2 mul DATAVECTOR exch get \psbar@psop \psbar@mul mul \pst@number\pst@dimb mul def
						/ybar2 exch 2 mul 1 add DATAVECTOR exch get \psbar@psop \psbar@mul mul \pst@number\pst@dimb mul def
						/xoffset barwidth barsep add barcount mul bXoffset add def
						/xbar1 colcount colwidth mul xoffset add def
						/xbar2 xbar1 barwidth add def
						ybar1 ybar2 ne {  % if ybar1 == ybar2, don't stroke a path
							newpath
							barstyles barcount get cvx exec
						} if
						/barcount barcount 1 add def
					} for
					/barcount 0 def
					/colcount \pst@tempc\space {colcount 1 add}{colcount 1 sub} ifelse def
				} forall
				end
			}%
		\or% chartstyle=plotbox
			\addto@pscode{%
				\pst@tempd%
				tx@BarDict begin
				/BARDATA #1 transpose def
				/ncols BARDATA length def
				/nbars BARDATA 0 get length 5 idiv def
				/colwidth \pst@number\pst@dima def
				/barcolsep \pst@tempa \pst@number\pst@dima mul def
				/barsep \pst@tempb \pst@number\pst@dima mul def
				/barwidth colwidth barcolsep sub nbars 1 sub barsep mul sub nbars div def
				/bXoffset 0.5 barcolsep mul def
				/colcount \pst@tempc\space {0}{ncols 1 sub} ifelse def
				/barcount 0 def
				/ybar1 0 def
				BARDATA {
					/DATAVECTOR exch def
					0 1 nbars 1 sub {
						dup
						dup
						dup
						dup
						/fdecile exch 5 mul DATAVECTOR exch get \psbar@psop \psbar@mul mul \pst@number\pst@dimb mul def
						/fquartile exch 5 mul 1 add DATAVECTOR exch get \psbar@psop \psbar@mul mul \pst@number\pst@dimb mul def
						/mediane exch 5 mul 2 add DATAVECTOR exch get \psbar@psop \psbar@mul mul \pst@number\pst@dimb mul def
						/lquartile exch 5 mul 3 add DATAVECTOR exch get \psbar@psop \psbar@mul mul \pst@number\pst@dimb mul def
						/ldecile exch 5 mul 4 add DATAVECTOR exch get \psbar@psop \psbar@mul mul \pst@number\pst@dimb mul def
						/xoffset barwidth barsep add barcount mul bXoffset add def
						/xoffset1 colcount colwidth mul xoffset add def
						/xoffset2 xoffset1 barwidth add def
						/xcenter xoffset1 xoffset2 add 2 div def
						/xmidoffset1 xcenter xoffset1 sub 2 div xoffset1 add def
						/xmidoffset2 xoffset2 xcenter sub 2 div xcenter add def
						/xbar1 xoffset1 def
						/xbar2 xoffset2 def
						/ybar1 fquartile def
						/ybar2 lquartile def
						ybar1 ybar2 ne {  % if ybar1 == ybar2, don't stroke a path
							newpath
							barstyles barcount get cvx exec
						} if
						\pst@number{\myps@medianwidth} setlinewidth
						newpath
						xoffset1 mediane moveto
						xoffset2 mediane lineto
						stroke
						\pst@number{\myps@whiskerwidth} setlinewidth
						newpath
						xmidoffset1 fdecile moveto
						xmidoffset2 fdecile lineto
						stroke
						newpath
						xmidoffset1 ldecile moveto
						xmidoffset2 ldecile lineto
						stroke
						fdecile fquartile ne {  % if fdecile == fquartile, don't stroke a path
							newpath
							xcenter fdecile moveto
							xcenter fquartile lineto
							stroke
						} if
						ldecile lquartile ne {  % if ldecile == lquartile, don't stroke a path
							newpath
							xcenter ldecile moveto
							xcenter lquartile lineto
							stroke
						} if
						/barcount barcount 1 add def
					} for
					/barcount 0 def
					/colcount \pst@tempc\space {colcount 1 add}{colcount 1 sub} ifelse def
				} forall
				end
			}%
		\else%
			\@pstrickserr{Unknown chart style.}\@ehpa%
		\fi%
		\ifx\psk@orientation\z@%
			\pstbar@xlabels%
		\else
			\pstbar@ylabels%
		\fi
	\end@SpecialObj%
}%

%\newcommand{\myaxes}[6][linestyle=dashed,dash=2pt 5pt,linewidth=0.5pt]%
% #1 (optional) are the options to the command '\psaxes' and '\psline' (for steps)
% #2 is the width of the histogram
% #3 is the legend for x axis
% #4 is the height of the histogram
% #5 is the legend for y axis
% #6 is the sep on y axis
\def\myaxes{%
	\@ifnextchar[{\my@ps@axes@i}{\my@ps@axes@i[]}%]
}
\def\my@ps@axes@i[#1](#2,#3)#4(#5,#6){%
	\@ifnextchar[%
		{\my@ps@axes@ii[#1](#2,#3)#4(#5,#6)}%
		{%
			\my@ps@axes@ii[#1](#2,#3)#4(#5,#6)[]%
		}%]
}
\def\my@ps@axes@ii[#1](#2,#3)#4(#5,#6)[#7]#8{%
	% For legend
	\ifstrempty{#7}%
		{%
			\ifstrempty{#1}%
				{\psaxes[labels=y,ticks=y](#2,#5)(#3,#6)}%
				{\psaxes[labels=y,ticks=y,#1](#2,#5)(#3,#6)}%
		}{%
			\ifstrempty{#1}%
				{\psaxes[Dy=#7,labels=y,ticks=y](#2,#5)(#3,#6)}%
				{\psaxes[Dy=#7,labels=y,ticks=y,#1](#2,#5)(#3,#6)}%
		}
	% Middle of the x-axis
	\fpDiv{\my@ps@xmiddle}{#3}{2}%
	% Set to the height of the x axis label
	\settoheight{\my@ps@xoffset}{#3}%
	\settodepth{\my@ps@temp@xoffset}{#3}
	\addtolength{\my@ps@xoffset}{\my@ps@temp@xoffset}
	% Label separations
	\addtolength{\my@ps@xoffset}{2\pslabelsep}%
	% Additional space
	\addtolength{\my@ps@xoffset}{0.5ex}%
	% x-label
	\uput{\my@ps@xoffset}[-90](\my@ps@xmiddle,0){\textit{#4}}%
	% Middle of the y-axis
	\fpDiv{\my@ps@ymiddle}{#6}{2}%
	% Set to the width of the wider number of y axis
	\settowidth{\my@ps@yoffset}{#6}
	% Add label separations
	\addtolength{\my@ps@yoffset}{2\pslabelsep}%
	% Additional space
	\addtolength{\my@ps@yoffset}{0.5ex}%
	% y-label
	\uput{\my@ps@yoffset}[180](0,\my@ps@ymiddle){\rotateleft{\textit{#8}}}%
	\ifstrempty{#7}%
		{}{%
		\fpDiv{\temp}{#6}{#7}%
		\fpRound{\iterations}{\temp}{0}%
		\multido{\i=1+1}{\iterations}{%
			\fpMul{\ylineoffset}{\i}{#7}
			\ifstrempty{#1}%
				{\psline[linestyle=dashed,dash=2pt 5pt,linewidth=0.5pt](0,\ylineoffset)(#3,\ylineoffset)}%
				{\psline[linestyle=dashed,dash=2pt 5pt,linewidth=0.5pt,#1](0,\ylineoffset)(#3,\ylineoffset)}%
			}%
		}%
}

% To define one element of the legend
\def\myleg#1#2{%
	\psframebox*[fillcolor=#2,framearc=0.25,framesep=1pt,shadow=false]{\textcolor{#2}{x}}~#1\hspace{1em}%
}
% Print the legend
\def\mylegend{\pst@object{mylegend}}
\def\mylegend@i{%
	\@ifnextchar<{\mylegend@ii}{\mylegend@ii<lt>}%
}
\def\mylegend@ii<#1>#2{%
	\bgroup%
	\use@par%
	\pslegend[#1]{#2}
	\egroup
}
\def\pslegend@iii[#1](#2){%
	\rput[#1](#2){%
		\parbox{\pst@dimc}{%
		\psframebox[style=legendstyle]{%
			\pslegend@text%
		}%
	}%
	}%
	\global\let\pslegend@text\relax%
}
% Create a barplot
\def\mybarplot{\pst@object{mybarplot}}
\def\mybarplot@i#1{%
	\bgroup%
	\use@par%
	\readpsbardata{\data}{files/#1}%
	\psbarchart[chartstyle=cluster,barcolsep=0.16,barsep=0.025]{\data}%
	\egroup%
}
% Create a boxplot
\def\myboxplot{\pst@object{myboxplot}}
\def\myboxplot@i#1{%
	\bgroup%
	\use@par%
	\readpsbardata[header=true]{\data}{files/#1}%
	\psbarchart[chartstyle=boxplot,barcolsep=0.16,barsep=0.025]{\data}%
	\egroup%
}
%
\resetOptions
%
\catcode`\@=\PstAtCode\relax
%
%% END: myps.tex
\endinput
